\chapter{Code management}
\label{Code management}
\index{code!management}
\index{age!source code|see{code, management}}
\index{checkin|see{code, management}}
\index{checkout|see{code, management}}
\index{directory!management|see{code, management}}
\index{file!management|see{code, management}}
\index{major version number|see{code, management}}
\index{minor version number|see{code, management}}
\index{patch version number|see{code, management}}
\index{rcs@\rcs!code management|see{code, management}}
\index{rename|see{code, management}}
\index{search!source code|see{code, management}}
\index{update!source code!selectively|see{code, management}}
\index{verification!source code|see{code, management}}
\index{version|see{code, management}}

\aipspp\ source code management utilities \footnote{Most recent change:
$ $Id$ $}.

% ----------------------------------------------------------------------------
 
\newpage
\section{\exe{asme}}
\label{asme}
\index{asme@\exe{asme}}
\index{setuid|see{\exe{asme}}}
 
Execute a command with the effective uid set to the real uid.

\subsection*{Synopsis}
 
\begin{synopsis}
   \code{\exe{asme} [command]}
\end{synopsis}
 
\subsection*{Description}
 
\exe{asme} executes a command with the effective uid set to the real uid.  The
requirement for such a utility arises because Bourne shell offers no mechanism
within a \code{setuid} script to allow a command to be executed with the real
uid.
 
\subsection*{Options}
 
None.
 
\subsection*{Notes}
 
\begin{itemize}
\item
   The command may be omitted.
\end{itemize}
 
\subsection*{Diagnostics}
 
Status return values
\\ \verb+   -1+:  fork failed
\\ \verb+    0+:  success, value returned on \file{stdout}
\\ \verb+ else+:  exit status returned by the command
 
\subsection*{Examples}
 
It uses the following
construct to ensure that the file to be checked in is readable by user and
group:

\begin{verbatim}
   asme "chmod ug+r $i"
\end{verbatim}
 
 
\subsection*{See also}
 
The unix manual page for \unixexe{chmod}(1).
 
\subsection*{Author}
 
Original: 1996/03/28 by Mark Calabretta, ATNF.

% ----------------------------------------------------------------------------

\newpage
\section{\exe{av}}
\label{av}
\index{av@\exe{av}}
\index{master host}
\index{code!management!filename validation|see{\exe{av}}}
\index{file!name validation|see{\exe{av}}}
\index{validation|see{verification}}
\index{verification!filename|see{\exe{av}}}

\aipspp\ filename validation utility.

\subsection*{Synopsis}

\begin{synopsis}
   \code{\exe{av} filename1 [filename2,...]}
\end{synopsis}

\subsection*{Description}

\exe{av} validates an \aipspp\ class implementation filename against those
already in the system to ensure uniqueness in the first 15 characters.  It
consults a cache of \aipspp\ pathnames maintained by other \aipspp\ 
code management utilities (\exeref{exhale}).  Any
matches found are printed on stdout.

\subsection*{Options}

None.

\subsection*{Notes}

\begin{itemize}
\item
   The pathname cache is stored in compressed form using \unixexe{gzip} to
   minimize the amount of data which must be fetched from the master during a
   checkin or rename.

\item
   If several filenames are to be validated it is more efficient to run
   \exe{av} once with multiple filename arguments than to run it separately
   for each filename.
\end{itemize}

\subsection*{Diagnostics}

Status return values
\\ \verb+   0+: success
\\ \verb+   1+: initialization error
\\ \verb+   2+: cache file not accessible

\subsection*{Examples}

The command

\begin{verbatim}
   av BaseMappedArrayIter
\end{verbatim}

\noindent
would print the string

\begin{verbatim}
   aips/implement/Tables/BaseMappedArrayEngine
\end{verbatim}

\noindent
indicating that \file{BaseMappedArrayIter} would not be acceptable as a new
file name because the first 15 characters conflict with
\file{BaseMappedArrayEngine}.

\subsection*{Files}

\begin{description}
\item[\file{\$MSTRETCD/av\_cache.gz}]
...pathname cache.
\end{description}

\subsection*{See also}

The GNU manual page for \unixexe{gzip}(1).\\
\aipspp\ variable names (\sref{variables}).\\
\exeref{exhale}, \aipspp\ code export utility.

\subsection*{Author}

Original: 1994/11/11 by Mark Calabretta, ATNF.

% ----------------------------------------------------------------------------

\newpage
\section{\exe{avers}}
\label{avers}
\index{avers@\exe{avers}}
\index{master host}
\index{code!management!version|see{\exe{avers}}}

\aipspp\ version reporting utility.

\subsection*{Synopsis}

\begin{synopsis}
   \code{\exe{avers} [\exe{-b} | \exe{-l}]}
\end{synopsis}

\subsection*{Description}

\exe{avers} reports version information for an \aipspp\ installation.

The \aipspp\ version consists of three fixed-length fields of the form
\code{MM.NNN.PP}.  The first field of two digits is the major version number,
the second field of three digits is the minor version number, and the third
field of two digits is the patch level.

The major version number is incremented whenever a new base release is
produced.  A base release is the publically distributed production code.
The minor version number is incremented on each cycle of the code distribution
system for \aipspp\ development sites (see \exeref{exhale} and
\exeref{inhale}), and the patch version distinguishes different patch levels
applied to the base release for the public distribution.

The long report contains the following version and timing information:

\noindent
\verb+   Master: +Version last fetched by \exeref{inhale}.\\
\verb+    Slave: +Version and time of completion of last slave update.\\
\verb+     Code: +Version and time of completion of last code update.\\
\verb+   System: +Version and time of completion of last system rebuild.\\
\verb+   Latest: +Version currently available from the master.\\
\verb+     Time: +The current time, in \aipspp\ format (see \exeref{adate}).

\noindent
Each of the installation steps is tagged as being {\em base},
{\em incremental}, or {\em cumulative}.  The short report only includes the
version last fetched by \exeref{inhale}, and without the ``\code{Master:}''
prefix.

\subsection*{Options}

\begin{description}
\item[\exe{-b}]
   Report the version last fetched by \aipsexe{inhale} (default).

\item[\exe{-l}]
   Report the version last fetched by \aipsexe{inhale}, the time at the
   completion of each stage of its installation, the latest version available
   from the master, and the current time.
\end{description}

\subsection*{Diagnostics}

Status return values
\\ \verb+   0+: success
\\ \verb+   1+: invalid option
\\ \verb+   2+: \code{\$CASAPATH} not defined

\subsection*{Examples}

Sample output from the command

\begin{verbatim}
   avers -l
\end{verbatim}

\noindent
appears as follows

\begin{verbatim}
   Master: 01.021.00 Wed 1992/09/16 10:30:32 GMT
    Slave: 01.021.00 Wed 1992/09/16 12:01:04 GMT  (incremental)
     Code: 01.021.00 Wed 1992/09/16 12:03:08 GMT  (incremental)
   System: 01.021.00 Wed 1992/09/16 12:07:55 GMT  (incremental)
   Latest: 01.022.00 Wed 1992/09/16 22:30:38 GMT
     Time:           Thu 1992/09/17 04:10:42 GMT
\end{verbatim}

\subsection*{See also}

\exeref{adate}, \aipspp\ time reporting utility.\\
\exeref{exhale}, \aipspp\ code export utility.\\
\exeref{inhale}, \aipspp\ code import utility.

\subsection*{Author}

Original: 1992/09/17 by Mark Calabretta, ATNF.

% ----------------------------------------------------------------------------

% ----------------------------------------------------------------------------

\newpage
\section{\exe{buildchangelog}}
\label{buildchangelog}
\index{buildchangelog@\exe{buildchangelog}}
\index{master host}
\index{code!management!changelog|see{\exe{buildchangelog}}}
\index{changelog|see{\exe{buildchangelog}}}

Process \aipspp\ changelog files and transform them to
html files containing the selected changes.

\subsection*{Synopsis}

\begin{synopsis}
   \code{\exe{buildchangelog} [\exe{-dir[ectory]} directory]
     [\exe{-split[file]} area|package|module] [\textrm{select-options}
     <directory>}
\end{synopsis}

\subsection*{Description}

All \aipspp\ changelog files found in the given directory and its
subdirectories are processed and the selected changes
are written to html files. 
Also a few html files containing an overview of the changes are
generated with links to the detailed html files.
\\The given directory has to be a code directory; i.e. it has to
end in \file{/code}.
\\The changes can be selected in various ways using the command
options.
By default the changes in the currently developed release are
selected.
\\The output directory can be specified in the options, which defaults
to the working directory.

\exe{buildchangelog} is run by the \exeref{sneeze} script if the
appropriate \exeref{makedefs} variable is set.
In that case it builds html files for the changes in the currently
developed release.
The \aipspp\ documentation pages have links to the overview files
generated in this way, so at all times the achievements in the
currently developed release can be seen on the \aipspp\ web pages.

One can also use \exe{buildchangelog} to find, for example, all
changes made to glish in the last year.

\subsection*{Options}

\begin{description}
\item[\exe{-dir[ectory]} directory]
   The directory of the generated html files.
   Default is the working directory.

\item[\exe{-split[file]} value]
   This case insensitive option determines how many html files are written.
   \\If not given, all changes are combined in one html file.
   \\\exe{-split area} generates an html file per area
   with name \file{changelog\_<area>.html}.
   \\\exe{- split package} generates an html file per package
   with name
   \\\file{changelog\_<area>\_<package>.html}.
   \\\exe{-split module} generates an html file per module
   with name
   \\\file{changelog\_<area>\_<package>\_<module>.html}.
   \\If the split option is given (and not blank), a summary
   file per area ia generated with name
   \file{changelog\_<area>\_summary.html}.
   \\In any case an overview file is generated with name
   \file{changelog\_summary.html}.

\item[\exe{-date} ranges]
   Select changes matching the given date range(s). Multiple ranges can be
   given separated by commas. Each range is a start date optionally
   followed by a colon and an end date.
   Each date can be given as yyyy-mm-dd or as dd-mm-yyyy, but an 
   end date has to be given in the same order as its start date.
   The month can be given as a number or as a 3 letter case
   insensitive name.
   It is not necessary to give all parts of a date, so it is possible
   to select a given year or range of years (or days or months for
   that matter). Again, an end date and start date have to match
   in that respect.
   \\For example,
\begin{verbatim}
   -date 2000
   -date 1-Jan-2000:31-Dec-2000
   -date 1995,1996,1997,1998,1999,2000
   -date 1995:2000
\end{verbatim}
   The first 2 examples are equivalent; select all of the year 2000.
   \\The latter 2 examples are also equivalent; select all of years
   1995 to 2000 (inclusive).

\item[\exe{-nodate} ranges]
   Select all changes NOT matching the given date range(s).

\item[\exe{-id} ranges]
   Select all changes matching the given id range(s).
   Multiple ranges can be given separated by commas.
   Each range is a start id optionally followed by a colon and an end id.
   \\For example,
\begin{verbatim}
   -id 1:10,20,24:30
\end{verbatim}

\item[\exe{-noid} ranges]
   Select all changes NOT matching the given id range(s).

\item[\exe{-area} names]
   Select all changes matching the given area name(s).
   Multiple areas can be given separated by commas.
   Note that the names are case sensitive.
   \\For example,
\begin{verbatim}
   -area System,Glish,Tool,Library,LearnMore
\end{verbatim}
   This eaxmple shows all available areas.

\item[\exe{-noarea} names]
   Select all changes NOT matching the given area name(s).

\item[\exe{-package} names]
   Select all changes matching the given package name(s).
   Multiple packages can be given separated by commas.
   Note that the names are case sensitive.
   \\For example,
\begin{verbatim}
   -package aips
\end{verbatim}
   Note that only areas Library and Tool have multiple packages.

\item[\exe{-nopackage} packages]
   Select all changes NOT matching the given package(s).

\item[\exe{-module} names]
   Select all changes matching the given module (or tool) name(s).
   Multiple modules can be given separated by commas.
   Note that the names are case sensitive.
   \\For example,
\begin{verbatim}
   -module Tables,Arrays
\end{verbatim}
   Note that the word module is also used for the tool name, so
   a specific tool is selected in the same way.

\item[\exe{-nomodule} names]
   Select all changes NOT matching the given module name(s).

\item[\exe{-avers} ranges]
   Select all changes matching the given \aipspp\ version range(s).
   Multiple ranges can be given separated by commas.
   Each range is a start version optionally followed by a colon and an
   end version. 
   Note that the version is treated as a float number (like 1.4).
   \\By default only the currently developed \aipspp\ release is
   selected. Therefore the special (case insensitive) value
   \exe{all} can be given to indicate that all versions have to be selected.
   \\For example,
\begin{verbatim}
   -avers all
   -avers 1.4:1.9
\end{verbatim}

\item[\exe{-noavers} ranges]
   Select all changes NOT matching the given \aipspp\ version range(s).

\item[\exe{-type} types]
   Select all changes matching the given type(s).
   Multiple types can be given separated by commas.
   Note that the type names are case insensitive.
   \\For example,
\begin{verbatim}
   -type code,documentation,test,review
\end{verbatim}
   This example shows all available types.

\item[\exe{-notype} types]
   Select all changes NOT matching the given type(s).

\item[\exe{-category} categories]
   Select all changes matching the given categories.
   Multiple categories can be given separated by commas.
   Note that the category names are case insensitive.
   \\For example,
\begin{verbatim}
   -category new,change,bugfix,removed,other
\end{verbatim}
   This example shows all available categories.

\item[\exe{-nocategory} categories]
   Select all changes NOT matching the given categories.
\end{description}

\subsection*{Notes}

\begin{itemize}
\item
   The change entries have an area, package, and module component.
   There are a few areas:
   \begin{description}
     \item[Library]
       for all changes in the library code.
     \item[Tool]
       for all changes in the \aipspp\ tools (applications).
     \item[Glish]
       for all changes in glish.
     \item[System]
       for all changes in the system area (directory install).
     \item[LearnMore]
       for all changes in notes, papers, etc. (directory doc).
   \end{description}
\end{itemize}

\subsection*{Diagnostics}

Status return values
\\ \verb+   0+: success
\\ \verb+   1+: invalid option given

\subsection*{Examples}

The command

\begin{verbatim}
   buildchangelog -split module
     -dir $(AIPSROOT)/docs/project/releasenotes/changelogs
     $(AIPSROOT)/code
\end{verbatim}
\noindent
is used in the code makefile to generate the html files
when building the system.

\begin{verbatim}
   buildchangelog -avers all -area Glish /home/casa/code
\end{verbatim}
\noindent
can be used to find all changes ever made to glish.
It will generate the files \file{./changelog.html} and
\file{./changelog\_summary.html}.

\subsection*{Bugs}

\begin{itemize}
\item
   \exe{buildchangelog} does not check if a change entry is filled in
   correctly. 
\end{itemize}

\subsection*{See also}

\subsection*{Author}

Original: 2000/06/05 by Ger van Diepen, NFRA

% ----------------------------------------------------------------------------

\newpage
\section{\exe{doover}}
\label{doover}
\index{doover@\exe{doover}}
\index{code!management!version.o@\file{version.o}|see{\exe{doover}}}
\index{version.o@\file{version.o}|see{\exe{doover}}}
 
Generate a function which returns \aipspp\ version information.

\begin{synopsis}
   \code{\exe{doover}}
\end{synopsis}
 
\subsection*{Description}
 
\exe{doover} generates \cplusplus\ code which records version information
found in \file{\$AIPSCODE/VERSION}.  It is intended to be used only by the
\aipspp\ \filref{makefiles}.

The \cplusplus\ code defines the following external variables

\begin{itemize}
\item
   \code{const int aips\_major\_version;} The \aipspp\ major version number
   (see \exeref{avers}).

\item
   \code{const int aips\_minor\_version;} The \aipspp\ minor version number.

\item
   \code{const int aips\_patch\_version;} The \aipspp\ patch version number.

\item
   \code{const char* aips\_version\_date;} The date on which this version was
   produced in \aipspp\ date format (see \exeref{adate}).

\item
   \code{const char* aips\_version\_info;} Any identifying version
   information.
\end{itemize}
 
\noindent
and it also defines the following global function which reports the version
information in a standard format:

\begin{verbatim}
   void report_aips_version(ostream &os)
\end{verbatim}

\noindent
Executables can access these external variables by virtue of declarations
made in \file{aips.h}.

\subsection*{Options}
 
None.
 
\subsection*{Notes}
 
\begin{itemize}
\item
   \filref{makefile.imp} uses \exe{doover} to check and if necessary generate
   \file{\$LIBDBGD/version.o} and \file{\$LIBOPTD/version.o} whenever it
   builds a system object library.  The \file{version.o} object modules are
   dependent on the \file{\$(AIPSCODE)/VERSION} file.  They are not inserted
   into the object library since that would cause all executables to be
   rebuilt.
\end{itemize}
 
\subsection*{Diagnostics}
 
Status return values
\\ \verb+   0+: success
\\ \verb+   1+: initialization error
\\ \verb+   2+: \file{VERSION} file not found
 
\subsection*{Examples}
 
Typing

\begin{verbatim}
   void report_aips_version(ostream &os)
\end{verbatim}

\noindent
shows the code containing the version information derived from the current
\file{\$(AIPSCODE)/VERSION}.
 
\subsection*{See also}
 
\aipspp\ variable names (\sref{variables}).\\
\exeref{adate}, \aipspp\ time reporting utility.\\
\exeref{avers}, \aipspp\ version report utility.\\
\filref{makefiles}, \textsc{gnu} makefiles used to rebuild \aipspp.
 
\subsection*{Author}
 
Original: 1996/02/29 by Mark Calabretta, ATNF.

% ----------------------------------------------------------------------------

\newpage
\section{\exe{mktree.svn}}
\label{mktree.svn}
\index{mktree.svn@\exe{mktree.svn}}
\index{directory!workspace|see{\exe{mktree.svn}}}
\index{workspace!creation|see{\exe{mktree.svn}}}

Create an \aipspp\ directory hierarchy with \rcs\ symlinks.

\subsection*{Synopsis}

\begin{synopsis}
   \code{\exe{mktree.svn} [\exe{-d}] [\exe{-l}] [\exe{-r}] [\exe{-s}]}
\end{synopsis}

\subsection*{Description}

\exe{mktree.svn} is an interactive utility used by \aipspp\ programmers to
create a local workspace.  It works incrementally in that it won't recreate
directories or symbolic links which already exist, but may create new ones
which have appeared since it was first run.

\exe{mktree.svn} creates a directory tree which shadows the \file{\$AIPSCODE}
subdirectory tree (see \sref{Directories}).  In each directory it creates an
\file{SVN} symbolic link to the corresponding \rcs\ directory under
\file{\$AIPSRCS}, thereby creating a window into the local \aipspp\ code
repository.

The main function of the \file{SVN} symlinks is to provide an access route for
\exeref{gmake} to the \aipspp\ \filref{makefiles}.  Although code checkin and
checkout must be done using \exeref{svn commit} and \exeref{svn co} which do not use the
\file{SVN} symlinks, the symlinks do allow other \rcs\ utilities such as
\unixexe{svn log} and \unixexe{svn diff} to be used as though the \aipspp\ \rcs\ 
repositories existed in the user's own workspace.  \aipsexe{svn co} allows
everything in a directory to be checked out.

\exe{mktree.svn} also creates the \file{\$CODEINCD} directory and symbolic links
therein.  These allow include files to be specified in the form
\code{\#include~<casa/Class.h>} for example (see \sref{Code directories}).

\exe{mktree.svn} can be used to create or update a limited portion of the
directory hierarchy.  If the current working directory is a subdirectory of
\file{*/code}, then only that portion of the tree will be created (see the
example below).

\exe{mktree.svn} checks for defunct directories and will delete them and the files
therein after seeking confirmation (unless the \exe{-s} option was specified).

\subsection*{Options}

\begin{description}
\item[\exe{-d}]
   Delete any pre-existing symbolic links.  This can be used to force symlinks
   to be recreated in the uncommon situation where a pre-existing programmer
   tree must be updated for a changed \file{AIPSROOT}, and where the old
   \file{AIPSROOT} still remains.  (Note that if the old \file{AIPSROOT} did
   not still remain then the symlinks would be recreated without needing to
   use this option.)

\item[\exe{-l}]
   \file{SVN} symbolic links are normally created only if \file{\$AIPSRCS} is
   being shadowed (see the \exe{-r} option below), or if the
   \file{\$AIPSCODE/rcs} symlink exists.  This option forces them to be
   created regardless.

\item[\exe{-r}]
   Shadow the \file{\$AIPSRCS} tree rather than \file{\$AIPSCODE}.  This is
   used by \exeref{inhale} to construct or update the \file{\$AIPSCODE} tree
   at consortium sites.

\item[\exe{-s}]
   Don't ask for confirmation of the parent directory, {\em or when deleting
   defunct directories and the files therein}.  Use with caution!
\end{description}

A \exe{-master} option which creates \file{SVN} symlinks directly into
\file{\$AIPSMSTR} is also recognized but is only intended for use by
\exeref{exhale} when building a new public release.

\subsection*{Notes}
 
\begin{itemize}
\item
   \exe{mktree.svn} ignores \file{tmplinst} directories, both in creating the tree
   and when deleting defunct directories.
\end{itemize}

\subsection*{Diagnostics}

Status return values
\\ \verb+   0+: success
\\ \verb+   1+: initialization error

\subsection*{Examples}

To construct an \aipspp\ workspace from scratch or update an existing one

\begin{verbatim}
   mkdir $HOME/casa
   cd $HOME/casa
   mktree.svn
\end{verbatim}

\noindent
This is equivalent to

\begin{verbatim}
   mkdir $HOME/casa/code
   cd $HOME/casa/code
   mktree.svn
\end{verbatim}

\noindent
To just create or update the \file{casa} package (see \sref{Code directories})

\begin{verbatim}
   cd $HOME/casa/code/casa
   mktree.svn
\end{verbatim}

\subsection*{See also}

The manual page for \unixexe{svn}(1).\\
\aipspp\ variable names (\sref{variables}).\\

\subsection*{Author}

Original: 1992/03/07 by Peter Teuben, BIMA.
 
% ----------------------------------------------------------------------------
 

\newpage
\section{\exe{squiz}}
\label{squiz}
\index{squiz@\exe{squiz}}
\index{code!management!source search|see{\exe{squiz}}}

Search for and examine \aipspp\ sources.

\subsection*{Synopsis}

\begin{synopsis}
   \code{\exe{squiz} [\exe{-c} category] [\exe{-e}|\exe{-E} expression]
      [\exe{-l}] [\exe{-F}] [\exe{-p} package] [file]}
\end{synopsis}

\subsection*{Description}

\exe{squiz} searches for \aipspp\ source files and optionally searches for
strings inside the files it finds.  Wildcarding is allowed in the file
specification and the search is case insensitive by default.

If no file is specified the default is to search for (i.e. list) all files.

\subsection*{Options}

\begin{description}
\item[\exe{-c} category]
   Restrict the search to a certain category:
   \begin{itemize}
      \item
      \code{application}, \code{apps}, \code{app}: Application source files.

      \item
      \code{fortran}, \code{for}, \code{ftn}: \textsc{fortran} sources.

      \item
      \code{implement}, \code{imp}: Class implementation files.

      \item
      \code{test}, \code{tst}: Test programs.
   \end{itemize}
   The default is to search everything.

\item[\exe{-e} expression]
   Perform a case insensitive search for the specified expression in all
   selected files.  \unixexe{grep}-type regular expressions are allowed.
   See also the \exe{-l} option.

\item[\exe{-E} expression]
   Case sensitive version of the \exe{-e} option.

\item[\exe{-F}]
   Do a case sensitive search for the files named.

\item[\exe{-l}]
   Only list the names of files that contain a match to the expression 
   as specified in \exe{-e}.

\item[\exe{-p} package]
   Restrict the search to a particular package (see \sref{Code directories}).
   The default is to search all packages.
\end{description}

\noindent
Whitespace is allowed between short-form options and their arguments.

\subsection*{Diagnostics}

Status return values
\\ \verb+   0+:  success
\\ \verb+   1+:  initialization error

\subsection*{Examples}

The following command will search for all files in and below the
\file{implement} directory of the \file{aips} package (that is
\file{code/aips/implement/...}) whose name contains the string ``aips'' (case
insensitive) and will search inside each file that it finds looking for the
string ``debug'' (case insensitive).

\begin{verbatim}
   squiz -c imp -e debug -p aips "*aips*"
\end{verbatim}

\noindent
The following simply counts the number of files in the \file{doc} package:

\begin{verbatim}
   squiz -p doc | wc -l
\end{verbatim}

\noindent
Finally, this example will list out all files that contain the expression
``plonk'':

\begin{verbatim}
   squiz -l -e plonk
\end{verbatim}

\subsection*{See also}

The unix manual page for \unixexe{grep}(1).\\
\sref{Code directories}, \aipspp\ directory structure.

\subsection*{Author}

Original: 1994/07/28 by Mark Calabretta, ATNF\\
Modified: 1997/05/30 by Pat Murphy, NRAO

% ----------------------------------------------------------------------------

\newpage
\section{\exe{tract}}
\label{tract}
\index{tract@\exe{tract}}
\index{code!management!file age|see{\exe{tract}}}
\index{file!age|see{\exe{tract}}}
\index{directory!age|see{\exe{tract}}}

Report the age of a file or directory.

\subsection*{Synopsis}

\begin{synopsis}
   \code{\exe{tract} [\exe{-a}|\exe{-c}|\exe{-m}] [\exe{-q}\#] [\exe{-s}]
      <file|directory>}
\end{synopsis}

\subsection*{Description}

\exe{tract} reports the age of a file or directory; by default it reports the
time in seconds since the file or directory was last modified.  In query mode
it can be used to determine if a file or directory is older than a specified
number of seconds.

\subsection*{Options}

\begin{description}
\item[\exe{-a}]
   Report time since last access.

\item[\exe{-c}]
   Report time since last status change.

\item[\exe{-m}]
   Report time since last modification (default).

\item[\code{\exe{-q}\#}]
   Return a successful exit status if the file is older than the specified
   number of seconds.  Nothing is reported on \file{stdout}.

\item[\exe{-s}]
   Report the time in sexagesimal (h:m:s) format (default is seconds).
\end{description}

\subsection*{Diagnostics}

Status return values
\\ \verb+  -1+:  In query mode, the file is younger than the specified timespan.
\\ \verb+   0+:  Success.
\\ \verb+   1+:  Usage error.
\\ \verb+   2+:  File access error.

\subsection*{Examples}

The following reports the elapsed time in h:m:s format since tract was
installed:

\begin{verbatim}
   tract -s `which tract`
\end{verbatim}

\subsection*{See also}

The unix manual page for \unixexe{find}(1).

\subsection*{Author}

Original: 1995/02/21 by Mark Calabretta, ATNF
