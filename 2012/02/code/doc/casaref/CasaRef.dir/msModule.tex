%% Copyright (C) 1999,2000,2001,2002,2003
%% Associated Universities, Inc. Washington DC, USA.
%%
%% This library is free software; you can redistribute it and/or modify it
%% under the terms of the GNU Library General Public License as published by
%% the Free Software Foundation; either version 2 of the License, or (at your
%% option) any later version.
%%
%% This library is distributed in the hope that it will be useful, but WITHOUT
%% ANY WARRANTY; without even the implied warranty of MERCHANTABILITY or
%% FITNESS FOR A PARTICULAR PURPOSE.  See the GNU Library General Public
%% License for more details.
%%
%% You should have received a copy of the GNU Library General Public License
%% along with this library; if not, write to the Free Software Foundation,
%% Inc., 675 Massachusetts Ave, Cambridge, MA 02139, USA.
%%
%% Correspondence concerning AIPS++ should be addressed as follows:
%%        Internet email: aips2-request@nrao.edu.
%%        Postal address: AIPS++ Project Office
%%                        National Radio Astronomy Observatory
%%                        520 Edgemont Road
%%                        Charlottesville, VA 22903-2475 USA
%%
%% $Id$
\providecommand{\dataSelectionURL}{http://almasw.hq.eso.org/almasw/bin/view/OFFLINE/DataSelection}
\begin{ahmodule}{ms}{Module for measurement set operations}

\begin{ahdescription}
  A \casa\ measurement set is a \casa\ table which obeys
  specific conventions. These conventions are defined in
  \htmladdnormallink{note 229}{../../notes/229/229.html}. Like all
  \casa\ tables the measurement set will always appear as a
  directory which contains a number of files and directories.

  Measurement set tables come in two slightly different versions,
  single dish and interferometric. Single dish measurement sets
  store the observed data as real numbers in the \verb|FLOAT_DATA|
  column of the measurement set, whereas interferometric ones use
  complex numbers in the \verb|DATA| column.

  A measurement set table can contain data from a variety of
  different observations with different spectral or polarimetric
  configurations, different pointings and different instruments. To
  do this it needs to handle data with differing shapes. The data
  shape referred to here is two-dimensional with the length of the
  axes being the number of correlations and the number of channels
  in the data. A typical shape might be \verb|[4, 1]|, which could
  correspond to a continuum observation where the
  \verb|[RR, LL, RL, LR]| polarizations where correlated. In the
  same measurement set there may be data with a shape of
  \verb|[1, 32]|, which corresponds to a spectral line observation,
  with 32~channels, where only the \verb|[XX]| polarizations are
  correlated.

\end{ahdescription}

\ahobjs{}
\ahfuncs{}

\input{ms.htex}
\input{msplot.htex}
\end{ahmodule}
