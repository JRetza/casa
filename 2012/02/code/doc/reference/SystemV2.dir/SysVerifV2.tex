\chapter{System verification}
\label{System verification}
\index{system!verification}

Utilities for verifying the integrity of \aipspp\ \footnote{Last change:
$ $Id$ $}.

% ----------------------------------------------------------------------------

\section{\exe{assay}}
\label{assay}
\index{assay@\exe{assay}}
\index{system!verification!tester|see{@\exe{assay}}}
 
Invoke an \aipspp\ test program or glish script, verify its output,
and delete temporary files.

\subsection*{Synopsis}
 
\begin{synopsis}
   \code{\exe{assay} [test program and arguments]}
\end{synopsis}
 
\subsection*{Description}
 
\exe{assay} invokes an \aipspp\ test program or glish script.
If the program name has the suffix \file{.g} then it is a test glish
script which is executed using the \exe{glish} command.
Otherwise it is a normal test program.
\\If the test program or glish script has an
associated \file{.exec} file then \exe{assay} invokes it using
the \unixexe{exec} command.  Otherwise if it has an associated
\file{.run} file then \exe{assay} invokes it by simply starting
it in another process.
\\
If there is a corresponding \file{.out} file, \exe{assay} compares
the output of the program or script with that.

In comparing the output \exe{assay} ignores text demarked by two lines
starting with \code{>>>} and \code{<<<}.  This may be useful as in the
following example, where the test program prints a quantity such as the
execution time which is not expected to be reproducible.

\begin{verbatim}
   The answer to the question of life the universe and everything is: 42
   >>>
   Execution time: 5.4 aeons
   <<<
\end{verbatim}

If there is a \file{.in} file associated with the test program then
\exe{assay} redirects \file{stdin} from it.
\\If there are \file{.in\_*} files then \exe{assay} takes care they
exist (temporarily) in the working directory, so the test program
can find them there.

At the end \exe{assay} takes care that all temporary files and
directories are deleted. When the test program or script creates
a temporary file or directory then it should be called
\code{name\_tmp*} where \code{name} is the name of the test program
or glish script (e.g. \code{tTable\_tmp1.data}).

It is permissible for a \file{.exec} to invoke \exe{assay} (presumably with
arguments for the test executable) -- the potential infinite recursion is
trapped.
\\Note that a \file{.run} has the advantage that one can massage the
output before \exe{assay} gets it.
 
\subsection*{Options}
 
None.
 
\subsection*{Notes}
 
\begin{itemize}
\item
   It is assumed that the test executable exists in the current \code{PATH}.
   If an environment variable called \code{ASSAYBIN} exists then
   \code{.:\$ASSAYBIN} is added to the start of \code{PATH}.  The \aipspp\
   system \file{bintest} directory is always appended to \code{PATH}.
\end{itemize}

\subsection*{Diagnostics}
 
Status return values
\\ \verb+    0+: success
\\ \verb+    1+: test execution failed
\\ \verb+    2+: test output disagreement
\\ \verb+    3+: untested (as returned from a \file{.exec})
\\ \verb+  130+: interrupt

\noindent
In addition, \exe{assay} prints one of four \textsc{posix 1003.3} compliant
status messages referring to the result of the test execution:

\noindent
\verb+   PASS (execution succeeded): +$<$test executable and arguments$>$ \\
\verb+   FAIL (execution failure): +$<$test executable and arguments$>$ \\
\verb+   PASS (output verified): +$<$test executable and arguments$>$ \\
\verb+   FAIL (output not verified): +$<$test executable and arguments$>$
 
\subsection*{Examples}

\exe{assay} is used by the test makefile (\file{makefile.tst}) which
uses \exe{assay} to invoke the test executable without arguments.  Where test
programs do need to be invoked with particular arguments a
\file{.exec} or \file{.run}  must
be provided.  The following example, \file{tIncrementalStMan.exec}, uses
\exe{assay} to invoke \exe{tIncrementalStMan} multiple times with different
arguments:

\begin{verbatim}
   #!/bin/sh
   #--------------------------------------------------------------------------
   # Script to invoke tIncrementalStMan and verify its output.
   #==========================================================================
     exec assay tIncrementalStMan "1000 && \
        tIncrementalStMan 2000 && \
        tIncrementalStMan 3000 && \
        tIncrementalStMan 4000 && \
        tIncrementalStMan 5000 && \
        tIncrementalStMan 0"
\end{verbatim}

\noindent
In this example note that \exe{assay} is \unixexe{exec}'d to save generating
another process, and also that \file{tIncrementalStMan.out} is expected to
contain the output from \exe{tIncrementalStMan} for all six runs.
 
\subsection*{See also}
 
Section \ref{Diagnostic makefile rules}, diagnostic makefile rules.
 
\subsection*{Author}
 
Original: 1995/11/01 by Mark Calabretta, ATNF.

% ----------------------------------------------------------------------------

\newpage
\section{Diagnostic makefile rules}
\label{Diagnostic makefile rules}
\index{makefile!rules!diagnostics}
\index{makefile!diagnostics}
\index{makefile!application}
\index{makefile!applications}
\index{makefile!checkout}
\index{makefile!class implementation}
\index{makefile!documentation}
\index{makefile!fortran@\textsc{fortran}}
\index{makefile!install}
\index{makefile!package}
\index{makefile!scripts}
\index{makefile!test}
\index{makefile!top-level}
\index{system!verification!runtests@\code{runtests} (rule)}
\index{compilation!diagnostics|see{makefile, diagnostics}}
\index{diagnostics|see{makefile, diagnostics}}
\index{system!verification!makefiles|see{makefile, diagnostics}}

Use of the \aipspp\ makefiles for various diagnostic purposes.

\subsection*{Synopsis}

\begin{synopsis}
   \file{makefile}\\
   \file{makefile.\{app,aps,chk,doc,ftn,imp,pkg,scr,tst\}}
\end{synopsis}

\subsection*{Description}

The \aipspp\ diagnostic targets are listed below by category.  These lists
are not exhaustive, but do aim to cover everything of practical use.  In
particular, they omit targets which are intended for the internal use of the
makefiles.

A target is labelled as ``recursive'' if it causes \exeref{gmake} to be
invoked in all subdirectories.  It is ``general'' if it applies to all
makefiles; such targets are defined in \filref{makedefs}.  A target is
``specific'' if defined in a specific makefile.

Some targets such as \file{help} have a general meaning, the specific
behaviour of which differs for specific makefiles.  These are referred to as
``general/specific'' and where appropriate the details of a target's behaviour
are described for each of the generic makefiles, for the top-level makefile
(\file{top}), and the installation makefile (\file{ins}).

\noindent
\textbf{System diagnostic targets:}

\noindent
There are two classes of diagnostic target.  The first class is
self-referential in that it deals with the \aipspp\ system and contains rules
to report and/or verify \aipsfil{makedefs} variable definitions, report
variables defined by the specific makefiles, print help information, and
especially for debugging the makefiles.  The rules to report and/or verify
\aipsfil{makedefs} variables are actually defined in \filref{testdefs}.

\begin{itemize}
\item
   \code{command} : (general)
   \\ This could be described as a ``do-it-yourself'' rule (without
   dependencies).  It invokes a command or sequence of commands specified by
   the \code{COMMAND} variable defined on the \aipsexe{gmake} command line.
   This makes most sense if the command sequence uses \aipsfil{makedefs}
   variables.

\item
   \code{printenv} : (general)
   \\ Print out the environment as seen by the commands within makefile rules.
   This is especially useful for diagnosing problems related to makefile
   variables not being exported, unresolved recursively defined variables, and
   variables which cannot be exported to the environment because they contain
   non-alphanumeric characters.

\item
   \code{eval\_vars} : (general)
   \\ Print variables specified by \code{VARS} on the \aipsexe{gmake} command
   line in a form suitable for \unixexe{eval}'ing into the environment in
   Bourne shell.  This is used in particular by \exereff{ax\_master}{ax},
   \exeref{depend} and \exeref{updatelib} for getting \aipsfil{makedefs}
   variable definitions when invoked in stand-alone mode (when invoked via a
   makefile rule the required variables are explicitly \code{export}ed to
   them).  An example of its use would be

\begin{verbatim}
   eval `gmake -f $AIPSARCH/makedefs VARS="AR ARFLAGS" eval_vars`
\end{verbatim}

   \noindent
   This would create environment variables \code{AR} and \code{ARFLAGS} and
   give them the values of the \aipsfil{makedefs} variables of the same name.

\item
   \code{diagnostics} : (general)
   \\ This target simply invokes the \code{versions} and \code{test\_global}
   targets.  It is used by the code distribution system on cumulative updates
   to produce a status report for consortium installations (see
   \exeref{sneeze}).

\item
   \code{versions} : (general)
   \\ Print the installed version of certain utilities required by \aipspp,
   the main one being \aipsexe{gmake} itself.

\item
   \code{show\_all} : (general)
   \\ This target invokes the \code{show\_global} and \code{show\_local}
   targets to show the value of all \aipsfil{makedefs} variables and all
   variables defined in the specific makefile.

\item
   \code{test\_all} : (general)
   \\ Similar to \code{show\_all} except that it invokes \code{test\_global}
   instead of \code{show\_global}.  See \filref{testdefs}.

\item
   \code{show\_global}  : (general)
   \\ This target effectively invokes the \code{show\_sys}, \code{show\_prg}
   and \code{show\_aux} targets to show all \aipsfil{makedefs} variables.

\item
   \code{test\_global}  : (general)
   \\ This target effectively invokes the \code{test\_sys}, \code{test\_prg}
   and \code{test\_aux} targets to show and verify all \aipsfil{makedefs}
   variables.  See \filref{testdefs}.

\item
   \code{show\_sys}  : (general)
   \\ Report the value of all system variables defined in \aipsfil{makedefs}.

\item
   \code{test\_sys}  : (general)
   \\ Report and verify the value of all system variables defined in
   \aipsfil{makedefs}.  See \filref{testdefs}.

\item
   \code{show\_prg} : (general)
   \\ Report the value of all variables defined in \file{makedefs} which are
   of immediate interest to \aipspp\ programmers.  This is particularly useful
   for reporting the compiler options set by the site-specific
   \aipsfil{makedefs} in response to use of the \code{OPT} variable or
   alternate programmer compilation flags.

\item
   \code{test\_prg} : (general)
   \\ Report and verify the value of all variables defined in \file{makedefs}
   of immediate interest to \aipspp\ programmers.  See \filref{testdefs}.

\item
   \code{show\_aux}  : (general)
   \\ Report the value of all auxilliary variables defined in \file{makedefs}.
   (Certain of these may have been redefined in the specific makefiles.)

\item
   \code{test\_aux}  : (general)
   \\ Report and verify the value of all auxilliary variables defined in
   \aipsfil{makedefs}.  See \filref{testdefs}.

\item
   \code{show\_local} : (specific)
   \\ Report the value of all variables defined within the specific makefile.
   Typically these contain target and dependency lists.

\item
   \code{show\_vars} : (general)
   \\ Report the value of the \aipsfil{makedefs} variables specified by
   \code{VARS} on the \aipsexe{gmake} command line.  For example
 
\begin{verbatim}
   cd $HOME/aips++/code/install
   gmake VARS="AIPSSRCS" show_vars
\end{verbatim}
 
   \noindent
   This would show all \aipspp\ system sources in the current directory.

\item
   \code{help} : (general/specific)
   \\ Print an itemized summary of all general and specific targets categorized
   as ``programmer'', ``system'', or ``diagnostic''.  The general targets are
   reported by \file{makedefs}, and the specific targets are appended by the
   specific makefile.

\noindent
\textbf{Source code testsuite targets:}

\noindent
The second class of diagnostic targets contains rules to compile and execute
the test programs and verify their output.

\item
   \code{cleansys} : (general/specific, recursive)
   \\ The \code{cleansys} rule for the \file{tst} makefile cleans up files
   produced by the \code{runtests} rule.  Refer to the description in section
   \sref{System generation makefile rules}.

\item
   \code{cleanfail} : (specific)
   \\ Delete reports for failed tests, see \code{runtests} below.
   \begin{itemize}
   \item
      \file{tst}: Delete reports for failed tests in \file{\$(BINTESTD)}.  The
      deletion of these files will cause the \code{runtests} rule to
      selectively redo those tests.  The list of failed tests is determined
      from \file{\$(BINTESTD)/runtests.report}.
   \end{itemize}

\item
   \code{\%.report} : (specific, pattern rule)
   \\ Compile and exercise a test program.
   \begin{itemize}
   \item
      \file{tst}: Compile a test program and invoke \exeref{assay} to exercise
      it.  If necessary for the compilation, template instantiation files will
      be created from a \file{templates} file and compiled into an object
      library (which is retained).  The executable and \file{.report} file,
      which contains the output from \aipsexe{assay}, are deposited in
      \file{\$(BINTESTD)}.  The \file{.report} file for each test program is
      used as a dependency file so the test will not be done if the
      \file{.report} file exists and is newer than the source files and
      libraries used to generate the test executable.  The \code{cleanfail}
      rule can be used to selectively delete reports for failed tests.  A
      one-line summary of the test result is reported by the rule which also
      collates and maintains these in \file{\$(BINTESTD)/runtests.report}.
      The test executable will be deleted when gmake terminates (unless an
      up-to-date executable already existed in \file{\$(BINTESTD)} when the
      rule was invoked).
   \end{itemize}

\item
   \code{runtests} : (general/specific)
   \\ The general rule does nothing.  Specific makefiles behave as follows:
   \begin{itemize}
   \item
      \file{app}: (none)
   \item
      \file{aps}: (none)
   \item
      \file{chk}: (none)
   \item
      \file{doc}: (none)
   \item
      \file{ftn}: (none)
   \item
      \file{imp}: Recurse into every package subdirectory.
   \item
      \file{ins}: (none)
   \item
      \file{pkg}: Recurse into the \file{implement} subdirectory.
   \item
      \file{scr}: (none)
   \item
      \file{top}: Recurse into every package subdirectory then print a summary
      of results.
   \item
      \file{tst}: Invoke the \code{\%.report} pattern rule for every test
      program in the directory.  The intermediate object library containing
      template instantiations (if any) is deleted when all tests have been
      done.
   \end{itemize}
\end{itemize}

\subsection*{Notes}

\begin{itemize}
\item
   The rules to report and/or verify \aipsfil{makedefs} variables are actually
   defined in \filref{testdefs}.
\end{itemize}

\subsection*{Examples}

After modifying any \aipsfil{makedefs} variable definitions, 

\begin{verbatim}
   gmake test_all
\end{verbatim}

\noindent
reports and verifies their values.

\subsection*{See also}

The \textsc{gnu} \code{Make} manual.\\
The \textsc{gnu} manual page for \unixexe{gmake}.\\
\aipspp\ variable names (\sref{variables}).
\exeref{gmake}, \textsc{gnu} make.\\
\filref{makedefs}, \aipspp\ makefile definitions.\\
\filref{testdefs}, rules for reporting and validating \aipspp\ makefile
   definitions.

% ----------------------------------------------------------------------------
 
\newpage
\section{\exe{runtests}}
\label{runtests}
\index{runtests@\exe{runtests}}
\index{system!verification!testing|see{\exe{runtests}}}
\index{tests!running|see{\exe{runtests}}}
 
Run the \aipspp\ test programs and mail the results to the \acct{aips2-inhale}
email exploder with a summary to \acct{aips2-workers}.
 
\subsection*{Synopsis}
 
\begin{synopsis}
   \code{\exe{runtests}}
\end{synopsis}
 
\subsection*{Description}
 
\exe{runtests} runs the \aipspp\ test programs and mails the results to the
\acct{aips2-inhale} exploder and also sends a summary to \acct{aips2-workers}.
It invokes the \code{runtests} target described in section \sref{Diagnostic
makefile rules} which stores its results in the \file{\$BINTESTD}
directory (\sref{variables}).
 
\subsection*{Options}
 
None.
 
\subsection*{Notes}
 
\begin{itemize}
\item
   The reported number of test program failures refers to the \file{aips}
   package only.

\item
   Do not confuse the \exe{runtests} utility with the \code{runtests}
   diagnostic makefile target (see \sref{Diagnostic makefile rules}).
\end{itemize}
 
\subsection*{Diagnostics}
 
Status return values
\\ \verb+   0+: success
\\ \verb+   1+: initialization error\\
 
\subsection*{Examples}
 
The command
 
\begin{verbatim}
   runtests
\end{verbatim}
 
\noindent
would invoke the test programs and mail the results to the various exploders.
 
\subsection*{See also}
 
The manual page for \unixexe{ci}(1), the \rcs\ checkin command.\\
\aipspp\ variable names (\sref{variables}).\\
Diagnostic makefile rules \sref{Diagnostic makefile rules}.
 
\subsection*{Author}
 
Original: 1997/01/29 by Mark Calabretta, ATNF.

% ----------------------------------------------------------------------------
 
\newpage
\section{\file{testdefs}}
\label{testdefs}
\index{testdefs@\file{testdefs}}
\index{system!verification!testdefs@\file{testdefs}}
\index{makedefs@\file{makedefs}!variables|see{\file{testdefs}}}
\index{makedefs@\file{makedefs}!verification|see{\file{testdefs}}}
\index{makefile!variables|see{\file{testdefs}}}
\index{makefile!verification|see{\file{testdefs}}}
\index{variables!makefile!report and test|see{\file{testdefs}}}
\index{verification!makefile|see{@\file{testdefs}}}
 
Report and test \file{makedefs} variable definitions.
 
\subsection*{Synopsis}
 
\begin{synopsis}
   \file{testdefs}
\end{synopsis}
 
\subsection*{Description}
 
\file{testdefs} is a \gnu\ makefile which defines rules for reporting the
values of \aipsfil{makedefs} variables and optionally testing their validity.
\file{testdefs} is an adjunct to \filref{makedefs}; it is not meant to be used
independently.

Test categories are:

\begin{itemize}
\item
   Check for the existence of a file.

\item
   Check for the existence of a directory.

\item
   Check for the existence of all directories in a path.

\item
   Check for the existence of (object) libraries in a directory.  Recognizes
   the syntax of the \exe{-l} option to \unixexe{ld}.

\item
   Check for the existence of an executable.  Reports the pathname if the
   executable was specified without pathname and also recognizes shell
   builtins.

\item
   Check the syntax of a library control variable.

\item
   Check the syntax of an \aipspp\ link list variable.

\item
   Check that a number lies within a specified range.

\item
   Syntax checks for some control variables; \code{AUXILIARY}, \code{BINTEST},
   \code{CONSORTIUM}, \code{DOCEXTR}, \code{DOCSYS}, \code{TESTOPT},
   \code{TIMER}.
\end{itemize}

\noindent
Note that about 40\% of \aipsfil{makedefs} variables can't be sensibly
tested.  Chief among these are the variables which define compiler options.

\subsection*{Notes}
 
\begin{itemize}
\item
   \file{testdefs} is primarily intended for the private use of
   \aipsfil{makedefs} which invokes its rules in either print or
   print-and-test mode via the \code{DO\_TEST} variable.

\item
   \aipsfil{testdefs} needs to be kept in step with \aipsfil{makedefs} if
   variables are added to or removed from the latter.
\end{itemize}

\subsection*{Files}
 
\begin{description}
\item[\file{\$AIPSARCH/testdefs}]
...rules for reporting and testing variable definitions.
\end{description}

\subsection*{Diagnostics}
 
\file{testdefs} associates one of three error levels with invalid variable
definitions:

\begin{description}
\item
   \code{ADVISORY}: A correction may be needed depending on the availability
   of resources such as libraries or modes of compiler operation.

   This error level is typically associated with missing third-party object
   libraries whose absence may cause restricted compilation or link failures.
   Roughly a quarter of all errors fall into this category.

\item
   \code{WARNING}: An incorrect definition was found which should be fixed
   otherwise some rules will fail.

   About two thirds of all errors fall into this category.

\item
   \code{SERIOUS}: A fundamental problem was found which will cause important
   rules to fail.

   This error level is given for an invalid \code{AIPSROOT} or \code{AIPSARCH},
   or problems which will cause catastrophic compile or link failures such as
   missing compilers.
\end{description}

\noindent
 
\subsection*{See also}
 
The \textit{GNU make} manual.\\
The \gnu\ manual page for \unixexe{gmake}.\\
\aipspp\ variable names (\sref{variables}).\\
\exeref{gmake}, \gnu\ make.\\
\filref{makedefs adjuncts}, Per-sourcefile makedefs definitions.\\
\filref{makefiles}, \gnu\ makefiles used to rebuild \aipspp.
 
\subsection*{Author}
 
Original: 1996/03/15 by Mark Calabretta, ATNF

% ----------------------------------------------------------------------------
 
\newpage
\section{\exe{testsuite}}
\label{testsuite}
\index{testsuite@\exe{testsuite}}
\index{master host}
\index{dinkum|see{@\exe{testsuite}}}
\index{verification!tests|see{@\exe{testsuite}}}
\index{verification!exhale@\exe{exhale}|see{@\exe{testsuite}}}
\index{verification!inhale@\exe{inhale}|see{@\exe{testsuite}}}
\index{exhale@\exe{exhale}!verification|see{@\exe{testsuite}}}
\index{inhale@\exe{inhale}!verification|see{@\exe{testsuite}}}

Maintain the most recent cumulative update to have passed the suite of
\aipspp\ test programs.

\subsection*{Synopsis}
 
\begin{synopsis}
   \code{\exe{testsuite} [\exe{-f}\#]}
\end{synopsis}
 
\subsection*{Description}
 
\exe{testsuite} maintains a copy of the most recent cumulative update
to have passed the suite of \aipspp\ test programs.  This is the ``dinkum''
update fetched by specifying the \exe{-D} option to \exeref{inhale}.

\exe{testsuite} is a special-purpose utility used only by \acct{aips2mgr} on
the master host.  It operates as follows:

\begin{itemize}
\item
   Since it generally takes longer to compile and verify an update than the
   interval between successive invokations of \exeref{exhale}, \exe{testsuite}
   begins by preserving a copy of the \file{VERSION} file and the current
   cumulative update file in \code{\$MSTRFTPD}.  It does this by creating
   hardlinks to each of these files within \code{\$AIPSMSTR/testsuite/}, a
   special directory created for its exclusive use.  (\aipsexe{exhale} does
   not delete a cumulative update file with multiple links to it.)

\item
   \exe{testsuite} then does a cumulative \aipsexe{inhale}, mailing the log to
   the \acct{aips2-inhale} exploder.

\item
   When the \aipsexe{inhale} has finished \exe{testsuite} invokes
   \exe{gmake~runtests}, preserving the log in \code{\$BINTESTD/runtests.log}.

\item
   It then mails the summary, \code{\$BINTESTD/runtests.report}, to the
   \acct{aips2-workers} exploder.

\item
   If the test suite passed, \exe{testsuite} installs the cumulative update as
   the new dinkum update by
   \begin{itemize}
   \item
      Preserving the hardlink to the cumulative update file as
      \code{\$AIPSMSTR/testsuite/dinkum}.  This hardlink is never accessed by
      the code distribution system, its only purpose is to stop
      \aipsexe{exhale} from deleting the linked-to file in \code{\$MSTRFTPD}.
   \item
      Preserving the \file{VERSION} file as \code{\$MSTRFTPD/DINKUM}.  This
      file substitutes for the \file{VERSION} file when \aipsexe{inhale} is
      invoked with the \exe{-D} option. 
   \end{itemize}

\item
   Finally \exe{testsuite} prints \code{\$BINTESTD/runtests.report} 
   and \code{\$BINTESTD/runtests.log} on \file{stdout}.
\end{itemize}

\noindent
\acct{aips2mgr} in Socorro invokes \exe{testsuite} once a week via a
\unixexe{cron} job in place of \aipsexe{inhale -c}.  It redirects the output
to \code{\$AIPSROOT/testsuite.log} and also mails it to the
\acct{aips2-inhale} exploder.

\subsection*{Options}
 
\begin{description}
\item[\exe{-f}\#]
   Allow the specified number of failures when deciding whether the test suite
   passed or not.
\end{description}
 
\subsection*{Notes}
 
\begin{itemize}
\item
   \exe{testsuite} is logically a function of \acct{aips2adm}, the \aipspp\
   master administrator.  However, \acct{aips2adm} cannot exercise the test
   programs since it does not maintain a functioning \aipspp\ system.  Thus
   the responsibility devolves to \acct{aips2mgr} on the master host.

   This puts \acct{aips2mgr} on the master host in a special position.
   \exe{testsuite} is unique in the \aipspp\ system as being the only utility
   designed for use by a particular \acct{aips2mgr}.

\item
   The \exe{testsuite} executable resides in the \file{bin} subdirectory of
   \acct{aips2mgr}'s home directory in Socorro.

\item
   ``Dinkum'' as in ``fair dinkum'' is an Australian colloquialism meaning
   true, honest or genuine. 
\end{itemize}

\subsection*{Diagnostics}
 
Status return values
\\ \verb+   0+: success
\\ \verb+   1+: initialization error
 
\subsection*{See also}
 
\aipspp\ variable names (\sref{variables}).\\
Section \sref{Accounts and groups}, \aipspp\ accounts and groups.\\
Section \sref{Diagnostic makefile rules}, diagnostic makefile rules.\\
\exeref{assay}, invoke an \aipspp\ test program and verify its output.\\
\exeref{exhale}, \aipspp\ code export utility.\\
\exeref{inhale}, \aipspp\ code import utility.\\
\exeref{runtests}, run the \aipspp\ test programs.\\
\exeref{sneeze}, \aipspp\ system rebuild utility.
 
\subsection*{Author}
 
Original: 1996/09/18 by Mark Calabretta, ATNF.
