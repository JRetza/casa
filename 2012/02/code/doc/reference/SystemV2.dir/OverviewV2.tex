\chapter{Overview}
\pagenumbering{arabic}
\label{Overview}
\index{overview}


This chapter \footnote{Last change:
$ $Id$ $}
provides an overview of the \casa\ system.

% ----------------------------------------------------------------------------
\section{Introduction}
\casa V2 is the result of some very hard work by a dedicated group of programmers
and scientists who would not let all the good work that went into \casa fade into
oblivion. While the name is changed, \casa is at the heart \casa V2, especially the
build system.  Libraries are now structured for functionality and hopefully layered
such that the developer can pick what they need.  Different packages may be easily
added to the code tree.


% ----------------------------------------------------------------------------

\section{The CASA directory hierarchy}
\label{Directories}
\index{directory!hierarchy}

In the following discussion of the \casa\ directory hierarchy we will
assume that \casa\ has been installed in directory \file{/home/casa}.  This
is the preferred location, but in practice the \casa\ ``root'' directory
can reside anywhere.  The root directory is generally referred to as
\file{\$AIPSROOT}.  Many other \casa\ directories have standard variable
names as listed in \sref{variables}.

The major subdirectories of \file{/home/casa} are \file{code} (\file{\$AIPSCODE}),
\file{docs} (\file{\$AIPSDOCS}), and one or more architecture-specific
subdirectories with names such as \file{darwin} and \file{linux\_gnu}
which contain the \casa\ system - that is, everything needed to run \casa,
including executables and sharable objects.  These architecture-specific
subdirectories are referred to collectively as \file{\$AIPSARCH}.

% ----------------------------------------------------------------------------

\subsection{Code directories}
\label{Code directories}
\index{directory!code}

The directory hierarchy beneath \file{/home/casa/code}, or \file{\$AIPSCODE},
consists of a collection of ``packages'' which are contained in separate
subdirectories.

There are a core set of \casa libraries, casa, scimath, components, coordinates, fits, flagging,
graphics, images, lattices, measures, ms, msfits, msvis, synthesis, and tables. In addition there
is display, casajni and java, which contain stand-alone applications for viewing images and for
browsing tables. The java directory contains jar files that are needed by the table browser.

There are two main application directories appspython and xmlcasa. Appspython is a lightweight limited
set of python bindins for tables, measures, quanta and functionals. Xmlcasa is a tool/tasked based system
similar to the old glish based \casa.

Anyone is entitled to download and build the \casa libraries. Write access to the core set of
libraries on the trunk in the repository is strictly controlled.

Each \casa\ consortium member is also entitled to maintain a package for
data processing applications specific to its telescope(s).  The sources for
these classes and applications reside in the consortium-specific packages:
\file{atnf}, \file{bima}, \file{drao}, \file{nfra}, \file{nral}, \file{nrao},
and \file{tifr}.  These may or may not use the standard \file{dish},
\file{synthesis}, and \file{vlbi} packages, and their installation is also
optional.

The \file{implement} and \file{fortran} subdirectories may contain module
subdirectories which serve to collect software ``modules'' in one place.  For
example, the \file{Tables} module contains all class header and implementation
files pertaining to the \file{Table} class, and classes derived from, and
related to it.  The \file{implement} directory and all module subdirectories
may also contain a \file{test} subdirectory which contains one or more
self-contained test programs specifically for the module.

All files associated with an \casa\ application reside in a subdirectory of
the \file{apps} directory of the same name as the application.  Each
application must reside in its own subdirectory.

The substructure of the consortium-specific packages is left entirely to
\casa\ consortium members to determine.

There are a number of other subdirectories of \file{/home/casa/code} which are
unrelated to packages.  The \file{install} subdirectory contains all of the
utilities required to install and maintain \casa\ as discussed in this
document.  \file{doc} contains \casa\ documentation sources, including the
\casa\ ``specs'', ``memos'', and ``notes'' series, and reference and design
documentation in the corresponding subdirectories.

Also below \file{/home/casa/code} is an \file{include} subdirectory which
contains symbolic links to the \file{implement} subdirectories for each
package.  The purpose of these symlinks is to allow \casa\ includes to be
specified as ``\code{\#include <package/Header.h>}'' by adding
\mbox{\file{-I/home/casa/code/include}} to the include path.

On \casa\ consortium installations an additional \file{admin} subdirectory
of \file{/home/casa/code} contains files relating to the administration of the
\casa\ project.

The \file{/home/casa/code} directory hierarchy appears as follows:

\begin{verbatim}
            |                            
            |                                         +-- Module1 --+--- test ----
            |                           +- implement -+-- Module2 --+--- test ----
            |                           |             +--- .... ----
            |                           |             +--- test ----
            |                           |
            |             +--- casa ----+-- mirlib  --+...
            |             |             +-- wcslib  --+...
            |             |             
            |             |
            |             |               +- fortran  --+...
            |             +- scimath -----+- implement -+...
            |             +- tables ------+...
            |             +- measures ----+...
            |             +- fits --------+...
            |             +- lattices ----+...
            |             +- coordinates -+...
            |             +- components --+...
            |             +- images ------+...
            |             +- ms ----------+...
            |             +- msfits ------+...
            |             +- msvis -------+...
            |             +- calibration -+...
            |             +- ionosphere --+...
            |             +- flagging ----+...
            |             +- synthesis ---+...
            |             +--- dish ------+...
            |             +- atmosphere --+...
            |             |
            |             +- graphics ----+...
            |             +- casajni -----+...
            |             +- display -----+...
            |             |
            |             +- simulators --+...
            |             +- singledish --+...
            |             +- vlbi --------+...
            |             +- vo ----------+...
            |             |
            |             |                +- implement
            |             +--- appspython -+- apps
            |             |                
            |             |                +- install
            |             |                +- apps
            |             |                +- implement
            |             +--- xmlcasa ----+-install
            |             |                +- scripts
            |             |                +- xml
            |             |                +- idl
            |             |
            |             +--- atnf ----+...
            |             +--- alma ----+...
            |             +--- drao ----+...
            |             +--- nfra ----+...
            |             +--- nral ----+...
            |             +--- nrao ----+...
            |             +--- tifr ----+...
/home/casa -+--- code ----+
            |             |             +- codedevl --
            |             |             +- codemgmt --
            |             +-- install --+- docutils --
            |             |             +- printer ---
            |             |             +-- <arch> ---
            |             |
            |             |             +-- design ---+...
            |             |             +--- html ----
            |             |             +--- memos ---
            |             |             +--- notes ---
            |             +---- doc ----+-- papers ---
            |             |             +-- project --
            |             |             +- reference -
            |             |             +--- specs ---
            |             |             +--- .... ----
            |             |
            |             +-- include --
            |             |
            |             |             +- personnel -
            :             +--- admin ---+- projects --
            :                           +-- system ---
\end{verbatim}

% ----------------------------------------------------------------------------

\subsection{Documentation directories}
\label{Documentation directories}
\index{directory!documentation}

The \file{/home/casa/docs} subdirectory, or \file{\$AIPSDOCS} (\sref{variables})
contains \textsc{ascii}, \textsc{html} and \textsc{PostScript} documents
compiled from the sources in \file{/home/casa/code/doc} whose directory structure
it shadows:

\begin{verbatim}
            :             +--- casa ----+...
            :             +--- scimath -+...
            |             +--- .... ----+...
            |             |
            |             +-- design ---+...
            |             +--- html ----
            |             +--- memos ---
/home/casa -+--- docs ----+--- notes ---
            |             +-- papers ---
            |             +-- project --
            |             +- reference -
            :             +--- specs ---
            :             +--- .... ----
\end{verbatim}

\noindent
\file{html} files compiled from inline comments in the \cplusplus\ source code
are deposited in the various package-specific subdirectories.

% ----------------------------------------------------------------------------

\subsection{System directories}
\label{System directories}
\index{directory!system}

The \casa\ system directory hierarchy is created in the first instance when
\casa\ is installed (see \sref{Installation}), and maintained thereafter by
the \code{sysdirs} target which is invoked by \code{allsys} in the top level
\filreff{makefile}{makefiles}.

Except for the \filref{casainit} files in \file{\$AIPSROOT}, the \casa\ 
system is completely self-contained within the architecture-specific
subdirectory, referred to as \file{\$AIPSARCH} (\sref{variables}).  In this
context ``architecture'' should be interpreted to include variants in the
operating system version and compiler.

In practical terms, the fact that the \casa\ system does not rely on
anything in the \file{\$AIPSCODE} directories allows the source code to be
deleted after the \casa\ installation is complete in a production-line
system.

At an \casa\ development site with machines of several architectures where
the source code must be retained, the strict separation of code from system
provides for the \file{\$AIPSARCH} tree to reside on a machine of the
corresponding architecture without duplication of the \file{code} directories.
With thoughtful unix filesystem management it also allows that the \casa\ 
system for one architecture may remain available even if the server for any
other architecture has crashed.

The system directories have the following structure:

\begin{verbatim}
            :             +---- lib ----+...
            :             +---- bin ----
            |             |
            |             +-- libdbg ---+...
            |             +-- bindbg ---
            |             |
            |             +-- bintest --
            |             |
            |             +-- libexec --
            |             |
            |             +---- aux ----
            |             +---- tmp ----+...
            |             |
            |             |             +--- info ----
/home/casa -+-- (arch1) --+---- doc ----+--- man1 ----
            |             |             +--- cat1 ----
            |             |             +---  :
            |             |
            |             |             +-- (host1) --
            |             +-- (site1) --+-- (host2) --
            |             |             +--   :
            |             |
            :             +-- (site2) --+...
            :             +---   :
\end{verbatim}

\noindent
The \file{lib} directory contains optimized static object libraries and
possibly sharable objects.  It sometimes also contains a subdirectory which
serves as a \cplusplus\ template repository.  The \file{bin} directory
contains \casa\ system scripts and optimized applications.  It is added to
the \code{PATH} environment variable by the \exeref{casainit} scripts.

The \file{libdbg} and \file{bindbg} directories contain debug versions of the
libraries and executables.  The \file{bindbg} directory is not usually
populated but serves as the temporary residence for executables which are in
the process of being debugged.

The \file{bintest} directory is used temporarily to store test executables and
test results, and \file{libexec} contains scripts of various kinds which are
not meant to be executed directly but are instead included by other scripts.

Files which are produced as intermediaries of system generation are cached in
the \file{aux} directory.  In particular, it includes dependency lists
generated by the makefiles.  Temporary storage is provided during a rebuild
beneath the \file{tmp} directory.  The structure and usage of the \file{tmp}
directory hierarchy is soley the concern of the \casa\ makefiles.  It
contains subdirectories specific to each \casa\ package (see
\sref{Code directories}).

Online documentation is contained in the \file{doc} subdirectory.  This
includes unix manual pages and help files.  The \exeref{casainit} scripts add
this directory to the \code{MANPATH} environment variable if it is defined at
the time that \aipsexe{casainit} is invoked.

Finally, the \file{\$AIPSARCH} directory contains site subdirectories which
contain site-specific \file{aipsrc} and \file{makedefs} files (see
\sref{Configuration files}), and possibly host-specific subdirectories which
in turn contain host-specific \file{aipsrc} files.  Multiple site-, and
host-specific directories were provided to make it easier for a central site
to administer \casa\ for a collection of remote sites.  If properly
configured, it should allow a verbatim copy of the \casa\ system at the
central site to be downloaded at the remote site with only a minimum of
reconfiguration required.

% ----------------------------------------------------------------------------

\subsection{SVN directories}
\label{svn directories}
\index{directory!.svn@\rcs}
\index{svn@\rcs!directories|see{directory, \rcs}}
\index{repository!svn@\rcs|see{directory, \rcs}}

Subversion directories (.svn) are found in the code tree. These directories
contain information needed by subversion for code management. For more details
on subversion visit the subversion site at
\htmladdnormallink{https://subversion.tigris.com}{https://subversion.tigris.com}

% ----------------------------------------------------------------------------

\section{CASA variable names}
\label{variables}
\index{variables}
\index{variables!environment}
\index{variables!makefile}
\index{environment variables}
\index{makefile!variables}
\index{CASAPATH@\code{CASAPATH}|see{variables}}
\index{AIPSROOT@\code{AIPSROOT}|see{variables}}
\index{AIPSARCH@\code{AIPSARCH}|see{variables}}
\index{AIPSSITE@\code{AIPSSITE}|see{variables}}
\index{AIPSHOST@\code{AIPSHOST}|see{variables}}
\index{AIPSMSTR@\code{AIPSMSTR}|see{variables}}
\index{AIPSLAVE@\code{AIPSLAVE}|see{variables}}
\index{AIPSRCS@\code{AIPSRCS}|see{variables}}
\index{AIPSCODE@\code{AIPSCODE}|see{variables}}
\index{AIPSDOCS@\code{AIPSDOCS}|see{variables}}
\index{MSTRETCD@\code{MSTRETCD}|see{variables}}
\index{CODEINSD@\code{CODEINSD}|see{variables}}
\index{CODEINCD@\code{CODEINCD}|see{variables}}
\index{LIBDBGD@\code{LIBDBGD}|see{variables}}
\index{LIBOPTD@\code{LIBOPTD}|see{variables}}
\index{BINDBGD@\code{BINDBGD}|see{variables}}
\index{BINOPTD@\code{BINOPTD}|see{variables}}
\index{BINTESTD@\code{BINTESTD}|see{variables}}
\index{ARCHTMPD@\code{ARCHTMPD}|see{variables}}
\index{ARCHAUXD@\code{ARCHAUXD}|see{variables}}
\index{ARCHBIND@\code{ARCHBIND}|see{variables}}
\index{ARCHDOCD@\code{ARCHDOCD}|see{variables}}
\index{ARCHMAN1@\code{ARCHMAN1}|see{variables}}
 
A standard set of variable names is used in \casa\ scripts, makefiles and
elsewhere to refer to \casa\ directories (\sref{Directories}).  Except for
\code{\$CASAPATH}, these are not stored in the environment but instead are
rederived from \code{\$CASAPATH} whenever needed.  

These variable names are freely used in this manual with the following
typographic conventions:

\begin{description}
   \item{\code{CASAPATH}}       ...the name of the variable.
   \item{\code{\$CASAPATH}}     ...the value of the \code{CASAPATH} variable,
      whether used as an environment variable or makefile variable.
   \item{\code{\$(CASAPATH)}}   ...the value of the \code{CASAPATH} variable
      used as a makefile variable.
\end{description}
 
Additional variables used by the \casa\ makefiles are described in the
entries for \filref{makedefs} and \filref{makefiles}.
 
\noindent
\verb+   ROOT = +the \casa\ home directory, also referred to as
                 \verb+~+\file{home/casa}\\
\verb+   ARCH = +the machine architecture, e.g. \code{sun4}, \code{convex},
                 etc.\\
\verb+   SITE = +the local site name\\
\verb+   HOST = +the host name
 
\begin{verbatim}
   CASAPATH = $ROOT $ARCH $SITE $HOST
 
   AIPSROOT = $ROOT
   AIPSARCH = $ROOT/$ARCH
   AIPSSITE = $ROOT/$ARCH/$SITE
   AIPSHOST = $ROOT/$ARCH/$SITE/$HOST
 
   AIPSMSTR = $AIPSROOT/master
   AIPSLAVE = $AIPSROOT/slave
   AIPSRCS  = $AIPSROOT/rcs
   AIPSCODE = $AIPSROOT/code
   AIPSDOCS = $AIPSROOT/docs
 
   MSTRETCD = $AIPSMSTR/etc
 
   CODEINSD = $AIPSCODE/install
   CODEINCD = $AIPSCODE/include

   INSTARCH = $CODEINSD/$ARCH
 
   LIBDBGD  = $AIPSARCH/libdbg
   LIBOPTD  = $AIPSARCH/lib
   BINDBGD  = $AIPSARCH/bindbg
   BINOPTD  = $AIPSARCH/bin
   BINTESTD = $AIPSARCH/bintest
 
   ARCHTMPD = $AIPSARCH/tmp
   ARCHAUXD = $AIPSARCH/aux
   ARCHBIND = $AIPSARCH/bin
   ARCHDOCD = $AIPSARCH/doc
   ARCHMAN1 = $ARCHDOCD/man1
\end{verbatim}

% ----------------------------------------------------------------------------

\section{Configuration files}
\label{Configuration files}
\index{system!configuration}
\index{configuration|see{system, configuration}}

Two sets of configuration files are used in \casa, the \file{aipsrc} and
\file{makedefs} files.

The \file{aipsrc} files store \code{keyword:value} entries used by \casa\ 
scripts and programs.  The mechanism is superficially similar to that of
\file{.Xdefaults} on which it is modelled.  The \file{aipsrc} files are
hierarchical - a user's \file{.aipsrc} is usually the one first consulted, if
no keyword definition is found therein the search continues to host-specific,
site-specific, and default \filref{aipsrc} files.

The \file{makedefs} files are \gnu\ makefiles which define
installation-specific variables.  All \casa\ \filref{makefiles} begin by
including \file{\$AIPSARCH/}\filref{makedefs} (\sref{variables}) which
contains default definitions and also some generally applicable rules and
targets.  This ``default'' \file{makedefs} in turn includes
\file{\$AIPSSITE/makedefs} which allows the default definitions to be
overridden.

% ----------------------------------------------------------------------------

\section{CASA accounts and groups}
\label{Accounts and groups}
\index{aips2adm@\acct{aips2adm}}
\index{aips2mgr@\acct{aips2mgr}}
\index{aips2prg@\acct{aips2prg}}
\index{aips2usr@\acct{aips2usr}}
\index{code!management!accounts}
\index{code!management!uid}
\index{code!management!groups}
\index{code!management!gid}
\index{master host}
\index{accounts|see{code, management}}
\index{groups|see{code, management}}
\index{uid|see{code, management}}
\index{gid|see{code, management}}

Historically there were two main adminstration accounts aips2adm and aips2mgr.
With the switch to CVS and SVN, the aips2adm account is greatly dimished
in importance. Neither account is necessary outside of the NRAO environment.
The aips2mgr account is handy in shared development environment for providing
a common \casa environment for multiple developers.

A set of \casa\ accounts and groups have been defined to perform particular
functions.  The following descriptions include the standard names, user ids,
and group ids, but different names may be defined via the \filref{aipsrc}
mechanism:

\begin{itemize}
\item
    \acct{aips2adm} (uid=31414, gid=31414): An obsolete account in Socorro,
    mostly used now for managing the data repository.

\item
    \acct{aips2mgr} (uid=31415, gid=31415): Owns the \casa\ installation,
    including directories, source files and binaries.  
    \acct{aips2mgr} runs the daily
    \exeref{inhale} \unixexe{cron} job.

\end{itemize}

