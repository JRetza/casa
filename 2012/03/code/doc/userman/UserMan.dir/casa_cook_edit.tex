%%%%%%%%%%%%%%%%%%%%%%%%%%%%%%%%%%%%%%%%%%%%%%%%%%%%%%%%%%%%%%%
%%%%%%%%%%%%%%%%%%%%%%%%%%%%%%%%%%%%%%%%%%%%%%%%%%%%%%%%%%%%%%%%%
%%%%%%%%%%%%%%%%%%%%%%%%%%%%%%%%%%%%%%%%%%%%%%%%%%%%%%%%%%%%%%%%%

% STM 2007-04-13  split from previous version
% STM 2007-06-25  bring up to Alpha Patch 1 level
% MPR 2007-07-05  minor tweaks, esp. to match frozen alpha-1 release of 4jul07
% STM 2007-09-20  pre-beta version
% STM 2007-10-09  Gustaaf's corrections
% STM 2007-10-09  beta version (spell-checked)
% STM 2007-11-09  beta 0.5, qcasabrowser
% STM 2008-03-24  patch 1.0
% STM 2008-05-14  start patch 2.0
% STM 2008-09-30  Patch 3 editing start, extendflag
% STM 2009-05-19  Patch 4 editing sta}rt, plotants and plotms
% STM 2009-11-24  Release 3.0.0 editing start
% JO 2010-02-20 Release edits for 3.0.1 start
% LC 2010-02-28 plotms chapter added
% JO 2010-04-20 release 3.2.0 edits start
% STM 2011-05-05  Release 3.2 flagcmd
% JO 2011-10-03 Release 3.3 edits
% GM 2011-10-26 Minor edits for release 3.3


%\documentclass[11pt]{article}
%\usepackage{fancyhdr}
%\usepackage{pslatex}
%\usepackage{epsfig}
%\usepackage{natbib}
%\usepackage{amssymb}
%\usepackage{amsmath}
%\usepackage{times}
%\usepackage{wrapfig}
%\usepackage{listings}
%\usepackage{multicol}
%\usepackage{longtable}          % package to allow for multiple page tables
%\usepackage{enumerate}          % package to allow for more control over enumerate environment
%\usepackage{deluxetable}
%\usepackage{url}
%\usepackage{boxedminipage}


%\renewcommand{\topfraction}{1.0}
%\renewcommand{\bottomfraction}{1.0}
%\renewcommand{\textfraction}{0.0}

%\setlength {\textwidth}{6.5in}
%\hoffset=-1.0in
%\oddsidemargin=1.00in
%\marginparsep=0.0in
%\marginparwidth=0.0in
% \setlength {\textheight}{9.0in}
%\voffset=-1.0in
%\topmargin=1.0in
%\headheight=0.00in
%\headsep=0.00in
%\headsep=0.00in
%\footskip=0.30in

%\begin{document}



\chapter{Data Examination and Editing}
\label{chapter:edit}

%%%%%%%%%%%%%%%%%%%%%%%%%%%%%%%%%%%%%%%%%%%%%%%%%%%%%%%%%%%%%%%%%
%%%%%%%%%%%%%%%%%%%%%%%%%%%%%%%%%%%%%%%%%%%%%%%%%%%%%%%%%%%%%%%%%
\section{Plotting and Flagging Visibility Data in CASA}
\label{section:edit.intro}

The tasks available for plotting and flagging of data are:
\begin{itemize}
   \item {\tt flagmanager} --- manage versions of data flags
      (\S~\ref{section:edit.flagmanager})
   \item {\tt flagautocorr} --- non-interactive flagging of auto-correlations
      (\S~\ref{section:edit.flagautocorr})
   \item {\tt plotms} --- create X-Y plots of data in MS, flag data
      (\S~\ref{section:edit.plot.plotms})
   \item {\tt plotxy} --- older X-Y plotter with some functionalities not yet implemented in \tt{plotms}
      (\S~\ref{section:edit.plot.plotxy})
   \item {\tt flagdata2} --- non-interactive flagging of data
      (\S~\ref{section:edit.flagdata})
   \item {\tt flagcmd} --- manipulate and apply flags using {\tt FLAG\_CMD} table
      (\S~\ref{section:edit.flagcmd})
   \item {\tt browsetable} --- browse data in any CASA table (including a MS)
      (\S~\ref{section:edit.browse})
   \item {\tt plotants} --- create simple plots of antenna positions
      (\S~\ref{section:edit.plot.plotants})
\end{itemize}

The following sections describe the use of these tasks.
%[SHOULD WE STRESS VIEWER MORE HERE? as an AIPS user, i found it hard to figure out what the parallels to tvflg and spflg are from this cookbook, because viewer hardly gets mentioned here. the viewer chapter is placed way at the end of the cookbook, almost like an afterthought.]

Information on other related operations can be found in:
\begin{itemize}
   \item {\tt listobs} --- list summary of a MS (\S~\ref{section:io.list})
   \item {\tt listvis} --- list data in a MS (\S~\ref{section:io.vis.listvis})
   \item {\tt selectdata} --- general data selection syntax
      (\S~\ref{section:io.selection})
   \item {\tt viewer} --- use the {\tt casaviewer} to display the MS as a 
      raster image, and flag it (\S~\ref{chapter:display})

\end{itemize}

%%%%%%%%%%%%%%%%%%%%%%%%%%%%%%%%%%%%%%%%%%%%%%%%%%%%%%%%%%%%%%%%%
%%%%%%%%%%%%%%%%%%%%%%%%%%%%%%%%%%%%%%%%%%%%%%%%%%%%%%%%%%%%%%%%%
\section{Managing flag versions with {\tt flagmanager}}
\label{section:edit.flagmanager}

The {\tt flagmanager} task will allow you to manage different
versions of flags in your data.  These are stored inside a CASA
flagversions table, under the name of the MS {\tt
  <msname>.flagversions}. 
For example, for the MS {\tt jupiter6cm.usecase.ms}, there will need
to be {\tt jupiter6cm.usecase.ms.flagversions} on disk.  This is
created on import (by {\tt importvla} or {\tt importuvfits}) or
when flagging is first done on an MS without a {\tt .flagversions}
(e.g. with {\tt plotxy}).  

By default, when the {\tt .flagversions} is created, this
directory will contain a {\tt flags.Original} in it containing
a copy of the original flags in the {\tt MAIN} table of the MS
so that you have a backup.  It will also contain a file called
{\tt FLAG\_VERSION\_LIST} that has the information on the various
flag versions there. The {\tt flagversions} are cumulative, ie. a
specific version number contains all the flags from the lower version
numbers, too. 

The inputs for {\tt flagmanager} are:
\small
\begin{verbatim}
vis                 =         ''        #   Name of input visibility file (MS)
mode                =     'list'        #   Flag management operation (list,save,restore,delete)
\end{verbatim}
\normalsize

The {\tt mode='list'} option will list the available flagversions from
the {\tt <msname>.flagversions} file.  For example:
\small
\begin{verbatim}
CASA <102>: default('flagmanager')
CASA <103>: vis = 'jupiter6cm.usecase.ms'
CASA <104>: mode = 'list'
CASA <105>: flagmanager()
MS : /home/imager-b/smyers/Oct07/jupiter6cm.usecase.ms

main : working copy in main table
Original : Original flags at import into CASA
flagautocorr : flagged autocorr
xyflags : Plotxy flags
\end{verbatim}
\normalsize

The {\tt mode} parameter expands the options.  For example, if you wish to 
save the current flagging state of {\tt vis=<msname>}, 
\small
\begin{verbatim}
mode                =     'save'        #   Flag management operation (list,save,restore,delete)
     versionname    =         ''        #   Name of flag version (no spaces)
     comment        =         ''        #   Short description of flag version
     merge          =  'replace'        #   Merge option (replace, and, or)
\end{verbatim}
\normalsize
with the output version name specified by {\tt versionname}.  For example, the
above {\tt xyflags} version was written using:
\small
\begin{verbatim}
   default('flagmanager')
   vis = 'jupiter6cm.usecase.ms'
   mode = 'save'
   versionname = 'xyflags'
   comment = 'Plotxy flags'
   flagmanager()
\end{verbatim}
\normalsize
and you can see that there is now a sub-table in the flagversions directory
\small
\begin{verbatim}
CASA <106>: ls jupiter6cm.usecase.ms.flagversions/
  IPython system call: ls -F jupiter6cm.usecase.ms.flagversions/
  flags.flagautocorr  flags.Original  flags.xyflags  FLAG_VERSION_LIST
\end{verbatim}

\normalsize
{\bf It is recommended that you use this facility regularly to save versions
during flagging.}

You can restore a previously saved set of flags using the 
{\tt mode='restore'} option:
\small
\begin{verbatim}
mode                =  'restore'        #   Flag management operation (list,save,restore,delete)
     versionname    =         ''        #   Name of flag version (no spaces)
     merge          =  'replace'        #   Merge option (replace, and, or)
\end{verbatim}
\normalsize
The {\tt merge} sub-parameter will control how the flags are restored.  
For {\tt merge='replace'}, the flags in {\tt versionname} will replace those
in the MAIN table of the MS.  For {\tt merge='and'}, only data that is
flagged in BOTH the current MAIN table and in {\tt versionname} will 
be flagged.  For {\tt merge='or'}, data flagged in EITHER the MAIN or
in {\tt versionname} will be flagged.

The {\tt mode='delete'} option can be used to remove {\tt versionname} from the
flagversions:
\small
\begin{verbatim}
mode                =   'delete'        #   Flag management operation (list,save,restore,delete)
     versionname     =         ''       #   Name of flag version (no spaces)
\end{verbatim}
\normalsize

 
%%%%%%%%%%%%%%%%%%%%%%%%%%%%%%%%%%%%%%%%%%%%%%%%%%%%%%%%%%%%%%%%%
%%%%%%%%%%%%%%%%%%%%%%%%%%%%%%%%%%%%%%%%%%%%%%%%%%%%%%%%%%%%%%%%%
\section{Flagging auto-correlations with {\tt flagautocorr}}
\label{section:edit.flagautocorr}

The {\tt flagautocorr} task can be used if all you want to do
is to flag the auto-correlations out of the MS.  Nominally,
this can be done upon filling from the VLA for example, but
you may be working from a dataset that still has them.

This task has a single input, the MS file name:
\small
\begin{verbatim}
vis                 =         ''        #   Name of input visibility file (MS)
\end{verbatim}
\normalsize
To use it, just set and go:
\small
\begin{verbatim}
CASA <90>: vis = 'jupiter6cm.usecase.ms'                                         
CASA <91>: flagautocorr()
\end{verbatim}
\normalsize

Note that the auto-correlations can also be flagged using 
{\tt flagdata2} (\S~\ref{section:edit.flagdata}) but the 
{\tt flagautocorr} task is a handy shortcut for this common
operation.

%%%%%%%%%%%%%%%%%%%%%%%%%%%%%%%%%%%%%%%%%%%%%%%%%%%%%%%%%%%%%%%%%
%%%%%%%%%%%%%%%%%%%%%%%%%%%%%%%%%%%%%%%%%%%%%%%%%%%%%%%%%%%%%%%%%
\section{X-Y Plotting and Editing of the Data}
\label{section:edit.plot}

There are three main X-Y plotting tasks in CASA:
\begin{itemize}
   \item {\tt plotms} --- create X-Y plots of data in MS, flag data
      (\S~\ref{section:edit.plot.plotms})
   \item {\tt plotxy} --- older X-Y plotter with some functionalities not yet implemented in {\tt plotms}
      (\S~\ref{section:edit.plot.plotxy})
   \item {\tt plotants} --- create simple plots of antenna positions
      (\S~\ref{section:edit.plot.plotants})
\end{itemize}

%%%%%%%%%%%%%%%%%%%%%%%%%%%%%%%%%%%%%%%%%%%%%%%%%%%%%%%%%%%%%%%%%
\subsection{MS Plotting and Editing using {\tt plotms}}
\label{section:edit.plot.plotms}

The principal way to get X-Y plots of visibility data is the {\tt plotms} task. This task also provides editing capability. {\tt Plotms} is a GUI-style plotter, based on Qt (\url{http://www.trolltech.com/qt}). It can either be started as a task within CASA ({\tt plotms}) or from outside CASA (type {\tt casaplotms} on the command line).

The current inputs to the {\tt plotms} task are:
\small
\begin{verbatim}
#  plotms :: A plotter/interactive flagger for visibility data.
vis                 =         ''        #  input visibility dataset (blank for none)
xaxis               =         ''        #  plot x-axis (blank for default/current)
yaxis               =         ''        #  plot y-axis (blank for default/current)
selectdata          =       True        #  data selection parameters
     field          =         ''        #  field names or field index numbers (blank for all)
     spw            =         ''        #  spectral windows:channels (blank for all)
     timerange      =         ''        #  time range (blank for all)
     uvrange        =         ''        #  uv range (blank for all)
     antenna        =         ''        #  antenna/baselines (blank for all)
     scan           =         ''        #  scan numbers (blank for all)
     correlation    =         ''        #  correlations (blank for all)
     array          =         ''        #  (sub)array numbers (blank for all)
     msselect       =         ''        #  MS selection (blank for all)

averagedata         =       True        #  data averaging parameters
     avgchannel     =         ''        #  average over channel?  (blank = False, otherwise
                                        #   value in channels)
     avgtime        =         ''        #  average over time? (blank = False, other value in
                                        #   seconds)
     avgscan        =      False        #  only valid if time averaging is turned on.  average
                                        #   over scans?
     avgfield       =      False        #  only valid if time averaging is turned on.  average
                                        #   over fields?
     avgbaseline    =      False        #  average over all baselines?  (mutually exclusive
                                        #   with avgantenna)
     avgantenna     =      False        #  average by per-antenna?  (mutually exclusive with
                                        #   avgbaseline)
     avgspw         =      False        #  average over all spectral windows?
     scalar         =      False        #  Do scalar averaging?

transform           =      False        #  transform data in various ways?
extendflag          =      False        #  have flagging extend to other data points?
iteraxis            =         ''        #  the axis over which to iterate
coloraxis           =         ''        #  selects which data to use for colorizing
plotrange           =         []        #  plot axes ranges: [xmin,xmax,ymin,ymax]
title               =         ''        #  Title written along top of plot
xlabel              =         ''        #  Text for horizontal axis. Blank for default.
ylabel              =         ''        #  Text for vertical axis. Blank for default.
showmajorgrid       =      False        #  Show major grid lines (horiz and vert.)
showminorgrid       =      False        #  Show minor grid lines (horiz and vert.)
plotfile            =         ''        #  Name of plot file to save automatically.
async               =      False        #  If true the taskname must be started using
                                        #   plotms(...)

\end{verbatim}
\normalsize All of these parameters can also be set or modified from
inside the {\tt plotms} window. Note that, if the {\tt vis} parameter
is set to the name of a measurement set here, when you start up {\tt
  plotms}, the {\it entire} measurement set will be plotted, which can
be time consuming. It is probably best to leave all parameters blank
for now, setting them as needed inside the {\tt plotms} GUI.

\begin{figure}[h!]
\begin{center}
\pngname{plotms_empty}{6}
\caption{\label{fig:plotms_empty} A freshly-started {\tt plotms} GUI
  window. Note that the {\bf Plots $>$ Data} tab is selected, which is
  discussed in \S~\ref{section:edit.plot.plotms.select},
  \ref{section:edit.plot.plotms.average}, and
  \ref{section:edit.plot.plotms.summary}.}
\hrulefill
\end{center}
\end{figure}

%%%%%%%%%%%%%%%%%%%%%%%%%%%%%%%%%%%%%%%%%%%%%%%%%%
\subsubsection{Loading and Selecting Data}
\label{section:edit.plot.plotms.select}

When {\tt plotms} is first started, a window will appear as in Figure
\ref{fig:plotms_empty}. It will, by default, display the {\bf Plots}
tab (as chosen from the tabs at the top of the {\tt plotms}
window---e.g., {\bf Plots, Flagging, Tools}...) and the {\bf Plots $>$
  Data} tab (as chosen from the tabs on the far left side of the {\tt
  plotms} window---e.g., {\bf Data, Axes, Trans, Iter}...). First, a
measurement set should be loaded by clicking on {\bf Browse} near the
top of the {\bf Plots $>$ Data} tab, and selecting a {\tt .ms}
directory (just select the directory itself; do not descend into the
{\tt .ms} directory). A plot can now be made of the measurement set by
clicking on {\bf Plot}---but beware, this would plot the entire
measurement set, and could take quite some time! It is probably better
to select a subset of the measurement set using the {\bf Selection}
windows in the {\bf Plots $>$ Data} tab before clicking {\bf
  Plot}. 

The options for data selection are:
\begin{itemize}
   \item {\bf field}
   \item {\bf spw}
   \item {\bf timerange}
   \item {\bf uvrange}
   \item {\bf antenna}
   \item {\bf scan}
   \item {\bf corr}
   \item {\bf array}
   \item {\bf msselect}
\end{itemize}
These are described in \S~\ref{section:io.selection}. Note that,
unlike when setting data selection parameters from the CASA command
line, no quotation marks are needed around strings.

When a plotting parameter has been changed, it will turn red (for
example, when a new measurement set is loaded, {\bf File Location}
turns red). This alerts the user that, if the {\bf Plot} button is
clicked, a change will be made to the displayed plot.

Once you have selected the desired subset of data, if you click {\bf
  Plot}, {\tt plotms} will by default plot amplitude versus time. See
the next section for information about other possible axes.

For a given data selection, {\tt plotms} will only load the data
once. This speeds up plotting considerably when changing plot
parameters such as different axes, colors etc. Sometimes, however,
the data changes on disk, e.g., when other data processing tasks were
applied. To force {\tt plotms} to reload the data, checkmark the
little {\bf force reload} box left to the {\bf Plot'} button or press
the {\tt SHIFT} key while clicking the {\bf Plot} button.

%%%%%%%%%%%%%%%%%%%%%%%%%%%%%%%%%%%%%%%%%%%%%%%%%%%%%%%%%%%%%%%%%
\subsubsection{A Brief Note Regarding {\tt plotms} Memory Usage}

In order to provide a wide range of flexible interactive plotting
options while minimizing the I/O burden, {\tt plotms} caches the data
values for the plot (along with a subset of relevant meta-info) in as
efficient a manner as possible.  For plots of large numbers of points,
the total memory requirement can be quite large. {\tt plotms} attempts
to predict the memory it will require (typically 5 or 6 bytes per
plotted point when only one axis is a data axis, depending upon the
data shapes involved), and will complain if it believes there is
insufficient memory to support the requested plot.  For most practical
interactive purposes (plots that load and draw in less than a few or a
few 10s of minutes), there is usually not a problem on typical modern
workstations (attempts to plot large datasets on small laptops might
be more likely to encounter problems here).

The absolute upper limit on the number of simultaneously plotted points
is currently set by the ability to index the points in the
cache.  For modern 64 bit machines, this is about 4.29 billion
points (requiring around 25GB of memory).   (Such plots are
not especially useful interactively, since the I/O and draw
become prohibitive.)

In general, it is usually most efficient to plot data in modest
chunks of not more than a few hundred million points or less,
either using selection or averaging.  Note that all iterations
are (currently) cached simultaneously for iterated plots, so 
iteration is not a way to manage memory use.  A few hundred
million points tends to be the practical limit of interactive
{\tt plotms} use w.r.t. information content and utility in the 
resulting plots, especially when you consider the number of available
pixels on your screen.  

Since datasets are growing very large, options for plotting
arbitrarily large numbers of points---probably in a 
non-interactive mode---are under consideration for a future release.

%%%%%%%%%%%%%%%%%%%%%%%%%%%%%%%%%%%%%%%%%%%%%%%%%%%%%%%%%%%%%%%%%

\begin{figure}[h!]
\begin{center}
\pngname{plotms_axes}{6}
\caption{\label{fig:plotms_axes} The {\bf Plots $>$ Axes} tab in the {\tt plotms} GUI window, used to make a plot of {\bf Amp} versus {\bf Channel}.} 
\hrulefill
\end{center}
\end{figure}

\subsubsection{Plot Axes}
\label{section:edit.plot.plotms.axes}

The X and Y axes of a plot are selected by clicking on the {\bf Plots $>$ Axes} tab on the left side of the {\tt plotms} window, and choosing an entry from the drop-down menus below {\bf X Axis} and {\bf Y Axis} (see Figure \ref{fig:plotms_axes}). Possible axes are:
\begin{itemize}
\item {\bf Scan} --- The scan number, as listed by {\tt listobs} (\S~\ref{section:io.list}) or the data summary in {\tt plotms} (\S~\ref{section:edit.plot.plotms.summary}).

\item {\bf Field} --- The field number, as listed by {\tt listobs} (\S~\ref{section:io.list}) or the {\tt plotms} data summary (\S~\ref{section:edit.plot.plotms.summary}).

\item {\bf Time} --- The time at which the visibility was observed, given in terms of calendar year (yyyy/mm/dd/hh:mm:ss.ss).

\item {\bf Time\_interval} --- The integration time in seconds.

\item {\bf Spw} --- The spectral window number. The characteristics of each spectral window are listed in {\tt listobs} (\S~\ref{section:io.list}) or the {\tt plotms} data summary (\S~\ref{section:edit.plot.plotms.summary}).

\item {\bf Channel} --- The spectral channel number.

\item {\bf Frequency} --- Frequency in units of GHz. The frame for the frequency (e.g., topocentric, barycentric, LSRK) can be set in the {\bf Plots $>$ Trans} tab (\S~\ref{section:edit.plot.plotms.trans}).

\item {\bf Velocity} --- Velocity in units of km s$^-1$, as defined by the {\bf Frame}, {\bf Velocity Defn}, and {\bf Rest Freq} parameters in the {\bf Plots $>$ Trans} tab (\S~\ref{section:edit.plot.plotms.trans}).

\item {\bf Corr} --- Correlations which have been assigned integer IDs:  5 = RR; 6 = RL; 7 = LR; and 8 = LL.

\item {\bf Antenna1} --- The first antenna in a baseline pair; for example, for baseline 2-4, Antenna1 = 2. Antennae are numbered according to the antenna IDs listed in {\tt listobs} (\S~\ref{section:io.list}) or the {\tt plotms} data summary (\S~\ref{section:edit.plot.plotms.summary}).

\item {\bf Antenna2} --- The second antenna in a baseline pair; for baseline 2-4, Antenna2 = 4. Antennae are numbered according to the antenna IDs listed in {\tt listobs} (\S~\ref{section:io.list}) or the {\tt plotms} data summary (\S~\ref{section:edit.plot.plotms.summary}).

\item {\bf Antenna} --- Antenna ID for plotting antenna-based quantities. Antennae are numbered according to the antenna IDs listed in {\tt listobs} (\S~\ref{section:io.list}) or the {\tt plotms} data summary (\S~\ref{section:edit.plot.plotms.summary}).

\item {\bf  Baseline} --- The baseline number.

\item {\bf UVDist} --- Projected baseline separations in units of meters. Note that {\bf UVDist} is {\it not} a function of frequency.

\item {\bf UVDist\_L} --- Projected baseline separations in units of the observing wavelength (lambda, not kilolambda). {\bf UVDist\_L} {\it is} a function of frequency, and therefore, there will be a different data point for each frequency channel.

\item {\bf U}, {\bf V}, and {\bf W} --- {\it u}, {\it v}, and {\it w} in units of meters.
%Beta Alert: Some day soon there will be U_L, V_L, and W_L, which will be in units of observing wavelength. Like UVDist_L, these quantities are functions of frequency and will change across your bandwidth.

\item {\bf Amp} --- Data amplitudes in units which are proportional to Jansky (for data which are fully calibrated, the units should be in Jy).

\item {\bf Phase} --- Data phases in units of degrees.

\item {\bf Real} and {\bf Imag} --- The real and imaginary parts of the visibility in units which are proportional to Jansky (for data which are fully calibrated, the units should be Jy).

\item {\bf Flag} --- Data which are flagged have Flag = 1, whereas unflagged data are set to Flag = 0. Note that, to display flagged data, you will have to click on the {\bf Plots $>$ Display} tab and choose a {\bf Flagged Points Symbol} (\S~\ref{section:edit.plot.plotms.symbol}).

\item {\bf Azimuth} and {\bf Ant-Azimuth} --- Azimuth in units of degrees. {\bf Azimuth} plots a fiducial value for the entire array, while {\bf Ant-Azimuth} plots the azimuth for each individual antenna (their azimuths will differ by small amounts, because each antenna is located at a slightly different longitude, latitude, and elevation).

\item {\bf Elevation} and {\bf Ant-Elevation} --- Elevation in units
  of degrees. {\bf Elevation} is a representative value for the entire
  array, while {\bf Ant-Elevation} is the elevation for each
  individual antenna (their elevations will differ by small amounts,
  because each antenna is located at a slightly different longitude,
  latitude, and elevation).

\item {\bf HourAngle} --- Hour angle in units of hours. This is a fiducial value for the entire array.
%Beta Alert: Some day soon, Ant-HourAngle will be added.

\item {\bf ParAngle} and {\bf Ant-ParAng} --- Parallactic angle in
  units of degrees. {\bf ParAngle} is the fiducial parallactic angle
  for all antennae in the array, while {\bf Ant-ParAng} plots the
  parallactic ange for each individual antenna (their parallactic
  angles will differ by small amounts, because each antenna is located
  at a slightly different longitude, latitude, and elevation).

\item {\bf Row} --- Data row number. A row number corresponds to a
  unique time, baseline, and spectral window in the measurement set.

\item {\bf FlagRow} --- In some tasks, if a whole data row is flagged,
  then FlagRow will be set to 1 for that row. Unflagged rows have
  FlagRow = 0. However, note that some tasks (like {\tt plotms}) may
  flag a row, but {\it not} set FlagRow = 1. It is probably better to
  plot Flag than FlagRow for most applications.
\end{itemize}

If the data axis selected from the drop-down menu is already stored in
the cache (therefore implying that plotting will proceed relatively
quickly), an ``X'' will appear in the checkbox next to {\bf In
  Cache?}.

For relevant data axes like {\bf Amp} and {\bf Phase}, the user will
be presented with the option to plot raw data or calibrated data. This
can be selected via a drop-down menu called {\bf Data Column}, located
directly under the drop-down menu for X or Y Axis selection (see the Y
axis in Figure \ref{fig:plotms_axes}). To plot raw data, select
``data''; to plot calibrated data, select ``corrected''. Note that
this choice will only have an impact on a plot if a calibration table
has been applied to the measurement set (see {\tt applycal},
Sect.\,\ref{section:cal.correct.apply}).

If a data model has been applied to the measurement set (e.g., with
{\tt setjy}, Sect.\,\ref{section:cal.prior.models}) it can be plotted
by selecting ``model'' from the {\bf Data Column} menu. Finally, to
plot the differences between the calibrated data and the model, select
``residual'' from {\bf Data Column}.

%%%%%%%%%%%%%%%%%%%%%%%%%%%%%%%%%%%%%%%%%%%%%%%%%%%%%%%%%%%%%%%%%

\subsubsection{Tools}
\label{section:edit.plot.plotms.tools}

Various tools---selectable as icon buttons at the bottom of the {\tt plotms} window---can be used to zoom, edit, and locate data. The icon buttons can be seen at the bottom of Figures \ref{fig:plotms_empty} and \ref{fig:plotms_axes}, and are, from left to right:
\begin{itemize}

\item {\bf Zoom} --- The ``magnifying glass'' button (1st on left) lets you draw a box around a region of the plot (left-click on one corner of the box, and drag the mouse to the opposite corner of the desired box), and then zooms in on this box.  

\item {\bf Pan} --- The ``four-arrow'' button (2nd from left) lets you pan around a zoomed plot.

\item {\bf Annotate} --- The 3rd button from the left is chosen from a drop-down menu to either {\bf Annotate Text} (``T with a green diamond'' button) or {\bf Annotate Rectangle} (``pencil'' button). In the {\bf Annotate Text} environment, click on a location in the plot where text is desired; a window will pop up, allowing you to type text in it. When you click the {\bf OK} button, this text will appear on the plot. {\bf Annotate Rectangle} simply lets you draw a box on the plot by left-clicking and dragging the mouse. By clicking on the {\bf Annotator} tab near the top of the {\tt plotms} window, different fonts, colors, line styles, etc. can be selected for annotations. 

\item {\bf Home} --- The ``house'' button (5th from left) returns to the original zoom level.

\item {\bf Stack Back} and {\bf Stack Forward} --- The left and right arrow buttons (4th and 6th from left) step through the zoom settings you've visited.

\item {\bf Mark Regions} --- The ``box with a green diamond'' button (7th from left) lets you mark a region for flagging, unflagging, or locating. Left-click on one corner of the desired region, and then drag the mouse to set the opposite corner of the region. You can mark multiple boxes before performing an operation on them.  

\item {\bf Clear Regions} --- Clicking on the ``box with a red circle'' button (8th from left) will clear {\it all} regions which have been marked using {\bf Mark Regions}.

\item {\bf Flag} --- Click on the ``flag'' button (9th from left) to flag all points in the marked regions.

\item {\bf Unflag} --- Click on the ``crossed-out flag'' button (10th from left) to unflag any flagged points that would be in the marked regions (even if invisible).

\item {\bf Locate} --- The ``magnifying glass on a sheet of paper'' button (11th from left) will print out information to the command line about points in the marked regions.  

\item {\bf Hold Drawing} --- If the ``hold drawing'' button (rightmost, or 12th from left) is depressed, and if new plot axes are selected from the {\bf Plots $>$ Axes} tab, these new data will be cached but not plotted. When the button is clicked on again and un-depressed, it will automatically plot the data that was last requested. This can be particularly useful when changing the size of the {\bf plotms} window.
\end{itemize}

There are two relevant options under the {\bf Options} tab near the top of the {\tt plotms} window. The {\bf when changing plot axes, clear any existing regions or annotations} checkbox determines when regions and annotation are deleted from the plot. The {\bf Tool Button Style} drop-drop down menu determines if icons and/or text represent the buttons at the bottom of the {\tt plotms} window. 

It is possible to hide these icons by going to the {\bf View $>$ Toolbars} menu at the top of the {\tt plotms} window and un-depressing the {\bf Tools} option (except for {\bf Hide Drawing}, which is hidden by clicking on {\bf View $>$ Toolbars $>$ Display}). In addition, the above tools can also be accessed by clicking on the {\bf Tools} tab near the top of the {\tt plotms} window (just below the {\bf View} menu). 

The {\bf Tools} tab also enables one additional tool, the {\bf
  Tracker}. To use {\bf Tracker}, click on the {\bf Hover} and/or {\bf

  Display} checkbox, and place your mouse over the plot. {\bf Tracker}
will output the X and Y position of your mouse, either as text
superimposed on the plot near your mouse (if {\bf Hover} is selected)
or in the blank window in the {\bf Tools} tab (if {\bf Display} is
selected). Pressing the {\tt SPACE} bar will copy the lines into the
larger white box below to the right. This can be repeated many times
and a log of positions and values will be created. The content in the
box can then be easily copied and pasted into any other application
that is used for data analysis. The {\bf Clear} button wipes out the
content of the box for a fresh start into new scientific adventures.



%%%%%%%%%%%%%%%%%%%%%%%%%%%%%%%%%%%%%%%%%%%%%%%%%%%%%%%%%%%%%%%%%

\subsubsection{Interactive Flagging in {\tt plotms}}
\label{section:edit.plot.plotms.flag}

Interactive flagging, on the principle of ``see it --- flag it'', is possible on the X-Y display of the data plotted by {\tt plotms}.  The user can
use the cursor to mark one or more regions, and then flag, unflag, or list the data that falls in these zones of the display.

Using the row of icons buttons at the bottom of the {\tt plotms} window (\S~\ref{section:edit.plot.plotms.tools}), click on the {\bf Mark Regions} button (which will appear to depress), then mark a region by left-clicking and dragging the mouse (each click and drag will mark an additional region).  You can get rid of all your regions by clicking on the {\bf Clear Regions}. Once regions are marked, you can then click on one of the other buttons to take action:
\begin{enumerate}
\item {\bf Flag} --- flag the points in the region(s),
\item {\bf Unflag} --- unflag flagged points in the region(s),
\item {\bf Locate} --- spew out a list of the points in the region(s) to the command line (Warning: this could be a long list!).
\end{enumerate}
Figure \ref{fig:markflags} shows an example of marking regions and then clicking the {\bf Flag} button. Whenever you click on a button, that action occurs without requiring an explicit disk-write.  If you quit {\tt plotms} and re-enter, you will see your previous edits.

\begin{figure}[h!]
\begin{center}
\pngname{plotms_reg_noflag}{3}
\pngname{plotms_reg_flag}{3}
\caption{\label{fig:markflags} Plot of amplitude versus time, before (left) and after (right) flagging two marked regions. To unflag these regions, mark the two same regions and click the {\bf Unflag} button.}
\hrulefill
\end{center}
\end{figure}

A table with the name {\tt <msname>.flagversions} (where {\tt vis=<msname>}) will be created in the same directory if it does not exist already. It is recommended that you save important flagging stages using the {\tt flagmanager} task (\S~\ref{section:edit.flagmanager}).

Flags can also be extended with options in the {\bf Flagging} tab, found near the top of the {\tt plotms} window. Flag extension enables the user to plot a subset of the data and extend the flagging to a wider set. In this release, the only functional extensions are over channel and correlation.

By checking the boxes next to {\bf Extend Flags} and {\bf Channel}, flagging will be extended to other channels in the same {\tt spw} as the displayed point.  For example, if {\tt spw='0:0'} and channel 0 is displayed, then flagging will extend to all channels in spw 0.

By checking the boxes next to {\bf Extend Flags} and {\bf Correlation}, flags will be extended beyond the correlations displayed. Currently the only option is to extend to {\bf All} correlations, implying that all correlations will be flagged, e.g.\ with RR displayed, the correlations RR, RL, LR, and LL will all be flagged. 

%Setting {\tt extendspw='all'} will extend the flagging to all other spw for the selection.  Using the same example as above, with {\tt spw='0:0'} displayed, then channel 0 in ALL spw will be flagged. Note that use of {\tt extendspw} could result in unintended behavior if the spw have different numbers of channels, or if it is used in conjunction with {\tt extendchan}.

{\bf WARNING:} use of flag extensions may lead to deletion of much more data than desired.  Be careful!

%Setting {\tt extendant='all'} will extend the flagging to all baselines that have antennas in common with those displayed and marked.  For example, if {\tt antenna='1\&2'} is shown, then ALL baselines to BOTH antennas 1 and 2 will be flagged.  Currently, there is no option to extend the flag to ONLY baselines to the first (or  second) antenna in a displayed pair, so it is better to use {\tt flagdata} to remove specific antennas.

%Setting {\tt extendtime='all'} will extend the flagging to all times matching the other selection or extension for the data in the marked region.  

%%%%%%%%%%%%%%%%%%%%%%%%%%%%%%%%%%%%%%%%%%%%%%%%%%%%%%%%%%%%%%%%%

\subsubsection{Averaging Data}
\label{section:edit.plot.plotms.average}

The {\bf Plots $>$ Data} tab enables averaging of the data in order to increase signal-to-noise of the plotted points or to increase plotting speed.The options for {\bf Averaging} ar e: 
\begin{itemize}
   \item {\bf channel}
   \item {\bf time}
   \item {\bf all baselines} or {\bf per antenna}
   \item {\bf all spectral windows}
   \item {\bf scalar}
\end{itemize}
The box next to a given {\bf Averaging} mode needs to be checked for that averaging to take effect.

For example, to average {\it n} channels together, the user would click on the box next to {\bf Channels} so that an ``X'' appears in it, and then type the number {\it n} in the empty box. When the user next clicks on {\bf Plot}, every {\it n} channels will then be averaged together and the total number of channels plotted will be decreased by a factor of {\it n}.

Time averaging is a little trickier, as it is controlled by three fields. If the checkbox next to {\bf Time} under {\bf Averaging} is clicked on, a blank box with units of seconds will become active, along with two additional checkboxes: {\bf Scan} and {\bf Field}. If averaging is desired over a relatively short interval (say, 30 seconds, shorter than the scan length), a number can simply be entered into the blank box and, when the data are replotted, the data will be time averaged. Clicking on the {\bf Scan} or {\bf Field} checkbox in this case will have no impact on the time averaging.

These checkboxes become relevant if averaging over a relatively long time---say the entire observation, which consists of multiple scans---is desired. Regardless of how large a number is typed into the {\bf Time} averaging blank box, only data within individual scans will be averaged together. In order to average data across scan boundaries, the {\bf Scan} checkbox must be clicked on and the data replotted. Finally, clicking on the {\bf Field} checkbox enables the averaging of multiple fields together in time.

Clicking on the {\bf All Baselines} checkbox will average all baselines in the array together. Alternatively, the {\bf Per Antenna} box  may be checked, which will average all baselines for a given antenna together. In this case, all baselines are represented twice; baseline 3-24 will contribute to the averages for both antenna 3 and antenna 24. This can produce some rather strange-looking plots if the user also selects on antenna---say, if the user requests to plot only antenna 0 and then averages {\bf Per Antenna}, In this case, an average of all baselines including antenna 0 will be plotted, but each individual baseline including antenna 0 will also be plotted (because the presence of baselines 0-1, 0-2, 0-3, etc.~trigger {\bf Per Antenna} averaging to try and compute averages for antennae 1, 2, 3, etc. Therefore, baseline 0-1 will contribute to the average for antenna 0, but it will also singlehandedly {\it be} the average for antenna 1.) 

Spectral windows can be averaged together by checking the box next to {\bf All Spectral Windows}. This will result in, for a given channel {\it n}, all channels {\it n} from the individual spectral windows being averaged together. 

Finally, the default mode is vector averaging, where the complex average is formed by averaging the real and imaginary parts of the relevant visibilities.  If {\bf Scalar} is chosen, then the amplitude of the average is formed by a scalar average of the individual visibility amplitudes.

When averaging, {\tt plotms} will prefer unflagged data.  I.e., if an averaging bin contains any unflagged
data at all, only the average of the unflagged will be shown.  For averaging bins that contain {\em only} unflagged data, the average of that unflagged data will be shown.  When flagging on a plot of averaged
data, the flags will be applied to the unaveraged data in the MS.

%%%%%%%%%%%%%%%%%%%%%%%%%%%%%%%%%%%%%%%%%%%%%%%%%%%%%%%%%%%%%%%%%

\subsubsection{Plot Symbols}
\label{section:edit.plot.plotms.symbol}

Plot symbols are selected in the {\bf Plots $>$ Display} tab. Most fundamentally, the user can choose to plot unflagged data and/or flagged data. By default, unflagged data is plotted (the circle next to {\bf Default} is checked under {\bf Unflagged Points Symbol}), and flagged data is not plotted (the circle next to {\bf None} is checked under {\bf Flagged Points Symbol}. We note here that plotting flagged data on an averaged plot is undertaken at the user's own risk, as the distinction between flagged points and unflagged points becomes blurred if data are averaged over a dimension that is partially flagged. Take, for example, a plot of amplitude versus. time where all channels are averaged together, but some channels have been flagged due to RFI spikes. In creating the average, {\tt plotms} will skip over the flagged channels and only use the unflagged ones. The averaged points will be considered unflagged, and the flagged data will not appear on the plot at all.

A selection of {\bf None} produces no data points, {\bf Default} results in data points which are small circles (blue for unflagged data and red for flagged data), and {\bf Custom} allows the user to define a plot symbol. If {\bf Custom} plot symbols are chosen, the user can determine the symbol size by typing a number in the blank box next to {\bf px} or by clicking on the adjacent up or down arrows. Symbol shape can be chosen from the drop-down menu to be either ``circle'', ``square'', ``diamond'', or ``pixel'' (note than ``pixel'' only has one possible size). Symbol color can be chosen by typing a hex color code in the blank box next to {\bf Fill:} (e.g., ``ff00ff''), or by clicking on the {\bf ...} button and selecting a color from the pop-up GUI. The adjacent drop-down menu provides options for how heavily the plot symbol is shaded with this color, from heaviest to lightest: ``fill'', ``mesh1'', ``mesh2'', ``mesh3'', and ``no fill''. Finally, the plot symbol can be outlined in black (if {\bf Outline: Default} is checked) or not (if {\bf Outline: None} is checked). Note that if ``no fill'' and {\bf Outline: None} are selected, the plot symbols will be invisible.

Finally, unflagged data points can be given informative symbol colors using the {\bf Colorize} parameter. By checking the box next to {\bf Colorize} and selecting a data dimension from the drop-down menu, the data will be plotted with colors that vary along that dimension. For example, if ``corr'' is chosen from the {\bf Colorize} menu, ``RR'', ``LL'', ``RL'', and ``LR'' data will each be plotted with a different color. Note that, currently, {\bf colorize} and plotting flagged data appear to be incompatible; a plot can only include one of these special features at a time.

%%%%%%%%%%%%%%%%%%%%%%%%%%%%%%%%%%%%%%%%%%%%%%%%%%%%%%%%%

\subsubsection{Summarizing Data}
\label{section:edit.plot.plotms.summary}

Information about the measurement set can be obtained from within {\tt plotms} by clicking on the {\bf Summary} button, found at the bottom of the {\bf Plots $>$ Data} tab window. If ``All'' is chosen from the pull-down menu next to {\bf Summary}, {\tt listobs}-style output about scans, correlator configurations, and antennae will be written to the command line from which {\tt plotms} was started. For more detail, click on the {\bf Verbose} checkbox. For a specific subset of the data, choose a selection from the pull-down menu like ``Antenna'' or ``Field''.
%[THIS SUBSECTION COULD USE MORE DETAIL ABOUT WHAT THE INDIVIDUAL SELECTIONS MEAN.)

%%%%%%%%%%%%%%%%%%%%%%%%%%%%%%%%%%%%%%%%%%%%%%%%%%%%%%%%%%%%%%%%%

\subsubsection{Defining Frequency and Velocity}
\label{section:edit.plot.plotms.trans}

If the user plans to plot {\bf Frequency}, the reference frame must be defined. By default, the plotted frequency is simply that observed at the telescope. However, transformations can be made by choosing a {\bf Frame} from the drop-down menu in the {\bf Plots $>$ Trans} tab. Frequency reference frames can be chosen to be:
\begin{itemize}
\item {\bf LSRK} --- local standard of rest (kinematic)
\item {\bf LSRD} --- local standard of rest (dynamic)
\item {\bf BARY} --- barycentric
\item {\bf GEO} --- geocentric
\item {\bf TOPO} --- topocentric
\item {\bf GALACTO} --- galactocentric
\item {\bf LGROUP} --- Local Group
\item {\bf CMB} --- cosmic microwave background dipole
\end{itemize}

{\bf Velocity} is affected by the user's choice of {\bf Frame}, but it is also impacted by the choice of velocity definition and spectral line rest frequency. The velocity definition is chosen from the {\bf Velocity Defn} drop-down menu in the {\bf Plots $>$ Trans} tab, offering selections of {\bf Radio}, {\bf True}, or {\bf Optical}. 

For more information on frequency frames and spectral coordinate systems, see the paper by Greisen et al. (A\&A, 446, 747, 2006) \footnote{Also at \url{http://www.aoc.nrao.edu/~egreisen/scs.ps}}.

Finally, the spectral line's rest frequency in units of MHz should be typed into the blank box next to {\bf Rest Freq} in the {\bf Plots $>$ Trans} tab. You can use the {\tt me.spectralline} tool method to turn transition names into frequencies 
\small
\begin{verbatim}
CASA <16>: me.spectralline('HI')
  Out[17]: 
{'m0': {'unit': 'Hz', 'value': 1420405751.786},
 'refer': 'REST',
 'type': 'frequency'}
\end{verbatim}
\normalsize

For a list of known lines in the CASA {\tt measures} system, use the toolkit command {\tt me.linelist()}.  For example:
\small
\begin{verbatim}
CASA <21>: me.linelist()
  Out[21]: 'HI H186A H185A H184A H183A H182A H181A H180A H179A H178A H177A H176A H175A 
H174A H173A H172A H171A H170A H169A H168A H167A H166A H165A H164A H163A H162A H161A H160A... 
He182A He181A He180A He179A He178A He177A He176A He175A He174A He173A He172A He171A He170A 
He169A He168A He167A He166A He165A He164A He163A He162A He161A He160A He159A He158A He157A...
C186A C185A C184A C183A C182A C181A C180A C179A C178A C177A C176A C175A C174A C173A C172A 
C171A C170A C169A C168A C167A C166A C165A C164A C163A C162A C161A C160A C159A C158A C157A... 
NH3_11 NH3_22 NH3_33 NH3_44 NH3_55 NH3_66 NH3_77 NH3_88 NH3_99 NH3_1010 NH3_1111 NH3_1212 
OH1612 OH1665 OH1667 OH1720 OH4660 OH4750 OH4765 OH5523 OH6016 OH6030 OH6035 OH6049 OH13433 
OH13434 OH13441 OH13442 OH23817 OH23826 CH3OH6.7 CH3OH44 H2O22 H2CO4.8 CO_1_0 CO_2_1 CO_3_2 
CO_4_3 CO_5_4 CO_6_5 CO_7_6 CO_8_7 13CO_1_0 13CO_2_1 13CO_3_2 13CO_4_3 13CO_5_4 13CO_6_5 
13CO_7_6 13CO_8_7 13CO_9_8 C18O_1_0 C18O_2_1 C18O_3_2 C18O_4_3 C18O_5_4 C18O_6_5 C18O_7_6 
C18O_8_7 C18O_9_8 CS_1_0 CS_2_1 CS_3_2 CS_4_3 CS_5_4 CS_6_5 CS_7_6 CS_8_7 CS_9_8 CS_10_9 
CS_11_10 CS_12_11 CS_13_12 CS_14_13 CS_15_14 CS_16_15 CS_17_16 CS_18_17 CS_19_18 CS_12_19 
SiO_1_0 SiO_2_1 SiO_3_2 SiO_4_3 SiO_5_4 SiO_6_5 SiO_7_6 SiO_8_7 SiO_9_8 SiO_10_9 SiO_11_10 
SiO_12_11 SiO_13_12 SiO_14_13 SiO_15_14 SiO_16_15 SiO_17_16 SiO_18_17 SiO_19_18 SiO_20_19 
SiO_21_20 SiO_22_21 SiO_23_22'
\end{verbatim}
\normalsize

%%%%%%%%%%%%%%%%%%%%%%%%%%%%%%%%%%%%%%%%%%%%%%%%%%%%%%%%%%%%%%%%%

\subsubsection{Shifting the Phase Center}
\label{section:edit.plot.plotms.phcen}

The plot's phase center can be shifted in the {\bf Plots $>$ Trans} tab. Enter the X and Y shifts in units of arcseconds in the blank boxes under {\bf Phase center shift}.

%%%%%%%%%%%%%%%%%%%%%%%%%%%%%%%%%%%%%%%%%%%%%%%%%%%%%%%%%%%%%%%%%

\subsubsection{Plot Ranges}
\label{section:edit.plot.plotms.range}

The X and Y ranges of the plot can be set in the {\bf Plots $>$ Axes} tab. By default, the circle next to {\bf Automatic} will be checked, and the ranges will be auto-scaled. To define the range, click on the circle below {\bf Automatic} and enter a minimum and maximum value in the blank boxes (as for the X Axis in Figure \ref{fig:plotms_axes}. Note that if identical values are placed in the blank boxes ({\tt xmin=xmax} and/or {\tt ymin=ymax}), then the values will be ignored and a best guess will be made to auto-range that axis.

%%%%%%%%%%%%%%%%%%%%%%%%%%%%%%%%%%%%%%%%%%%%%%%%%%%%%%%%%%%%%%%%%

\subsubsection{Plot Labels}
\label{section:edit.plot.plotms.labels}

The plot and axes labels which are displayed in the plot window are
set in the {\bf Plots $>$ Canvas} tab. To change the plot title, under
{\bf Canvas Title}, click on the circle next to the blank box and
enter the desired text. 
%(note that this is independent of the plot's title in the {\bf Plot
%Selection} menu; see \S~\ref{section:edit.plot.plotms.window}). 
To change the X- and Y-axis labels, similarly click on the circles next to the blank boxes under {\bf Show X Axis} and {\bf Show Y Axis} and type the desired text in the blank box. To display these new labels, simply click the {\bf Plot} button.

The user can determine the locations of axis labels in the {\bf Plots $>$ Axes} tab. The X-axis label switches from the bottom to the top of the plot depending on what is selected for {\bf Attach to:}. Similarly for the Y-Axis, the user can choose to attach axis labels and tick marks to the {\bf Top} or {\bf Bottom} (note that the axis labels have been attached to the {\bf Bottom} and {\bf Right} in Figure \ref{fig:plotms_axes}.

Finally, axis labels can be removed all together by unchecking the boxes next to {\bf Show X Axis} and {\bf Show Y Axis} om the {\bf Plots $>$ Canvas} tab.

%%%%%%%%%%%%%%%%%%%%%%%%%%%%%%%%%%%%%%%%%%%%%%%%%%%%%%%%%%%%%%%%%

\subsubsection{Grid Lines}
\label{section:edit.plot.plotms.grid}

A grid of lines can be superimposed on the plot using {\bf Grid Lines} in the {\bf Plots $>$ Canvas} tab. ``Major'' grid lines are drawn at the locations of major tick marks, while ``minor'' grid lines are drawn at minor tick marks.

Grid line colors, thicknesses, and styles are selected independently for the ``major'' and ``minor'' grid lines. Desired line thickness should be typed into the blank boxes just to the right of the {\bf Major} and {\bf Minor} labels. Colors are set  by clicking on the {\bf ...} buttons. The blank boxes to the left of the {\bf ...} buttons will then contain the hex codes for the selected colors (e.g., ``808080''). Line styles can also be selected from the drop-down menus to the right of {\bf ...} buttons.  

%%%%%%%%%%%%%%%%%%%%%%%%%%%%%%%%%%%%%%%%%%%%%%%%%%%%%%%%%%%%%%%%%

\subsubsection{Legend}
\label{section:edit.plot.plotms.legend}

A plot symbol legend can be added to the plot by clicking on the checkbox next to {\bf Legend} in the {\bf Plots $>$ Canvas} tab. However, given the current functionalities of {\tt plotms}, a symbol legend is of very limited use. This option will become more relevant when overplotting capabilities are included in {\tt plotms}.

%%%%%%%%%%%%%%%%%%%%%%%%%%%%%%%%%%%%%%%%%%%%%%%%%%%%%%%%%%%%%%%%%%
%
%\begin{figure}[h!]
%\begin{center}
%\pngname{plotms_select_red}{6}
%\caption{\label{fig:plotms_select} Multiple ``single'' plots displayed in a {\tt plotms} window. The {\bf Plot Selection} menu is circled in red (See \S~\ref{section:edit.plot.plotms.window} for more detail). } 
%\hrulefill
%\end{center}
%\end{figure}
%
%\subsubsection{New Plot Windows and Plot Selection}
%\label{section:edit.plot.plotms.window}
%
%A new plot window can be opened using the drop-down menu which is located directly below the upper row of tabs (hereafter called the {\bf Plot Selection} menu; it is labelled as ``Amp vs. Time'' and circled in red in Figure \ref{fig:plotms_select}). To open a new window, select ``New Single Plot'' from this menu, and then click on the {\bf Go} button which will appear after this selection. There will now be two plot windows, one on top of the other, as in Figure \ref{fig:plotms_select}. The new window will be initially blanks, and to plot in it, the measurement set must be re-loaded and re-selected upon(\S~\ref{section:edit.plot.plotms.select}; then click the {\bf Plot} button. 
%
%Both plot windows should now be listed in the {\bf Plot Selection} drop-down menu. To return to editing the original plot, select it from this menu and click the {\bf Edit} button which will appear on selection. The name of each plot in this drop-down menu can be changed using {\bf Plot Title} in the {\bf Plots $>$ Axes} tab. Click on the circle next to the blank box under {\bf Plot Title} and enter the desired text for the title. Note that this is, by default, the same as the plot title displayed in the plot window, but it is not required to be. The plot title in the plot window is determined by {\bf Canvas Title} in the {\bf Plots $>$ Canvas} tab (see \S~\ref{section:edit.plot.plotms.labels}).
%
%The plot windows can be completely cleared by choosing ``Clear Plotter'' from the {\bf Plot Selection} menu. Clicking the {\bf Go} button will not only clear the plots themselves, but will also clear the plot parameters, loaded data, and the cache. 
%
%The {\bf Plot Selection} menu also presents the user with a ``New Multi Plot'' option, but note that multi-plots are not functional in this release of CASA. Selecting ``New Multi Plot'' and clicking the {\bf Go} button will currently crash {\tt plotms}.
%
%%%%%%%%%%%%%%%%%%%%%%%%%%%%%%%%%%%%%%%%%%%%%%%%%%%%%%%%%%%%%%%%%%

% NB (gmoellen, 11Oct26): The Cache tab is now suppressed in the gui
%\subsubsection{The Cache}
%\label{section:edit.plot.plotms.cache}
%
%Data dimensions which are loaded into the cache can be plotted relatively quickly. To see which data dimensions %are loaded in the {\tt plotms} cache, navigate to the {\bf Plots $>$ Cache} tab. The {\bf Meta-Information Axes} %are always loaded in the cache, and include data like {\bf Channels} and {\bf Time}. 
%
%The {\bf Loaded Axes} are cached data dimensions, and have probably been recently plotted. To remove one of %these axes from the cache, click on its name so that it is highlighted and then click the {\bf Release} button.
%
%Finally, the {\bf Available Axes} are those data dimensions which are currently not stored in the cache. To load %one of them to the cache, click on its name to highlight it and then click the {\bf Load} button.

%%%%%%%%%%%%%%%%%%%%%%%%%%%%%%%%%%%%%%%%%%%%%%%%%%%%%%%%%%%%%%%%%

\subsubsection{The Options Tab}
\label{section:edit.plot.plotms.options}

A few miscellaneous options are available in the {\bf Options} tab,
the last tab at the top of the {\tt plotms} window. 
 The {\bf Tool Button Style} drop-drop down menu determines if icons and/or text represent the buttons in the toolbar near the bottom of the {\tt plotms} window. 

The {\bf Log Events} drop down menu determines how verbose {\tt plotms} is in documenting its actions on the command line.

There is a checkbox that determines the persistence of regions and annotations on new plots, labelled {\bf When changing plot axes, clear any existing regions and annotations}.

A useful option is the {\bf fixed size for cached image}
checkbox. It determines how large the dots in the panel are with
respect to the screen resolution. The values influence how the data
is redrawn on the panel. When the {\bf Screen resolution} is selected,
the {\tt plotms} window can be resized without redrawing on
the canvas -- a considerable speedup for large data sets. The penalty is
that the dots of the data points are the size of a pixel on the
screen, which may be very small for high resolution monitors. 

Finally, the {\bf File chooser history limit} determines the number of
remembered directories in the file loading pop-up of the {\bf Browse}
selection of the {\bf Data} tab.

%%%%%%%%%%%%%%%%%%%%%%%%%%%%%%%%%%%%%%%%%%%%%%%%%%%%%%%%%%%%%%%%%

\subsubsection{Iteration}
\label{section:edit.plot.plotms.iter}

In many cases, it is desirable to iterate through the data that were
selected in the {\it Data} tab. A typical example is to display a single baseline in a time vs. amplitude plot and
then proceed to the next baselines step by step. This can be done via the {\it Iter} tab on the left hand
side of {\tt plotms}. A drop-down menu allows you to select the
parameter to be iterated on, such as baseline or spw (press {\it plot}
after changing your selection). The plot titles in the main panel in
{\tt plotms} show which data slice is currently displayed. To proceed
to the next plot use the {\it green buttons} below the main panel. The
different button symbols let you to proceed panel by panel or to jump to the first or last
panel directly.
 

There are two scaling options for the axes: {\it Global} and {\it
  Self}. {\it Global} will use a common axis range based on data
loaded with the selection criteria specified in the {\it Data}
tab. {\it Self} readjusts the axes scaling to the data for each
individual panel of the iteration. 


Note that exporting iterated data only refers to the
shown panel, the exported files do not (yet) collate all the iteration
panels in say a
multi-page pdf. So please export every panel separately.

The {\it Rows} and {\it Columns} selection boxes will, in the future, control the number of
panels for multi-panel plots. This feature is likely available in CASA
3.3 and higher.



%%%%%%%%%%%%%%%%%%%%%%%%%%%%%%%%%%%%%%%%%%%%%%%%%%%%%%%%%%%%%%%%%
\subsubsection{Saving your plot}
\label{section:edit.plot.plotms.save}

You can save a copy of a plot to file in the {\bf Plots $>$ Export} tab. Click the {\bf Browse} button for a GUI-based selection of the directory and file name to which the plot will be saved. The file format can also be determined in this GUI by the suffix given to the filename: {\tt .png} (PNG), {\tt .jpg} (JPG), {\tt .ps} (PS), or {\tt .pdf} (PDF). Alternatively, the file format can be selected from the {\bf Format} drop-down menu located just below the {\bf Browse} button. In this case, {\tt plotms} will add a suffix to the file name depending on the format chosen.

{\bf ALERT:} The plot files produced by the PS and PDF options can be
large and time-consuming to export.  The JPG is the smallest.\\ 
{\bf On
  MacOS there are known problems when exporting to anything but a file
  in png
  format. Until this is fixed, please export to png and use third
  party software, such as the MacOS
  application {\tt Preview} or the ImageMagick Tool {\tt convert} to
  convert the png to pdf, jpg, eps, etc. }

The exported plot resolution can be manipulated using the {\bf High Resolution}, {\bf DPI}, and {\bf Size} options.

Click on {\bf Export} to create the file. Note that, if there is more
than one plot displayed in the {\tt plotms} window,
% (\S~\ref{section:edit.plot.plotms.window}), 
it will only export the currently selected plot.

%%%%%%%%%%%%%%%%%%%%%%%%%%%%%%%%%%%%%%%%%%%%%%%%%%%%%%%%%%%%%%%%%
\subsubsection{Exiting {\tt plotms}}
\label{section:edit.plot.plotms.exit}

To exit the {\tt plotms} GUI, select {\bf Quit} from the {\bf File} menu at the top of the {\tt plotms} window. You can also dismiss the window by killing it with the ``X'' on the frame.

Alternatively, you can just leave it alone, and {\bf plotms} will keep
running in the background. If the data file changes in the background,
you can force reloading the data via the 'force reload' checkbox
next to the 'Plot' button. Alternatively, press SHIFT while clicking
on 'Plot' for the same purpose.

%%%%%%%%%%%%%%%%%%%%%%%%%%%%%%%%%%%%%%%%%%%%%%%%%%%%%%%%%%%%%%%%%%%%







%%%%%%%%%%%%%%%%%%%%%%%%%%%%%%%%%%%%%%%%%%%%%%%%%%%%%%%%%%%%%%%%%
\subsection{Plotting and Editing using {\tt plotxy}}
\label{section:edit.plot.plotxy}


\begin{wrapfigure}{r}{2.5in}
  \begin{boxedminipage}{2.5in}
     \centerline{\bf Inside the Toolkit:}
     Access to {\tt matplotlib} is also provided through 
     the {\tt pl} tool. 
     See below for a description of the {\tt pl} tool functions. 
  \end{boxedminipage}
\end{wrapfigure}

\vspace{5cm}

{\bf ALERT:} The {\tt plotxy} code is fragile and slow, and is being replaced by the {\tt plotms} (\S~\ref{section:edit.plot.plotms}). We retain {\tt plotxy} in this release as not all functionality is available yet in {\tt plotms}.

%\begin{wrapfigure}{r}{2.5in}
%  \begin{boxedminipage}{2.5in}
%     \centerline{\bf Inside the Toolkit:}
%     Access to {\tt matplotlib} is also provided through 
%     the {\tt pl} tool. 
%     See below for a description of the {\tt pl} tool functions. 
%  \end{boxedminipage}
%\end{wrapfigure}

{\tt Plotxy} is a tool for visualizing and editing visibility
data. Unlike {\tt plotms}, it is useful in scripting, as it can
non-interactively produce a hardcopy plot (see
\S~\ref{section:edit.plot.plotxy.print}). It also has multi-plot
(\S~\ref{section:edit.plot.plotxy.subplot}), iteration
(\S~\ref{section:edit.plot.plotxy.iter}), and overplotting
(\S~\ref{section:edit.plot.plotxy.overplot}) functionality---unlike
{\tt plotms} in the current release. {\tt Plotxy} uses the {\tt
  matplotlib} plotting library to display its plots. You can find
information on {\tt matplotlib} at
\url{http://matplotlib.sourceforge.net/}.







\begin{figure}[h!]
\begin{center}
\pngname{plotxy_jupiter}{5.5}
\caption{\label{fig:matplotlib}The {\tt plotxy} plotter, showing the
  Jupiter data versus uv-distance.  You can see bad data in this plot.
  The {\bf bottom set of buttons} on the
  lower left are: 1,2,3) {\bf Home, Back, and Forward}. Click to
  navigate between previously defined views (akin to web navigation).
  4) {\bf Pan}. Click and drag to pan to a new position. 5) {\bf
  Zoom}. Click to define a rectangular region for zooming. 6) {\bf
  Subplot Configuration}. Click to configure the parameters of the
  subplot and spaces for the figures. 7) {\bf Save}. Click to launch a
  file save dialog box.  The {\bf upper set of buttons in the lower left} are:
  1) {\bf Mark Region}. Press this to begin marking regions (rather than
  zooming or panning).  2,3,4) {\bf Flag, Unflag, Locate}.  Click on these
  to flag, unflag, or list the data within the marked regions.  5) {\bf Next}.
  Click to move to the next in a series of iterated plots.
  Finally, the {\bf cursor readout} is on the bottom right.}
\hrulefill
\end{center}
\end{figure}

To bring up this plotter use the {\tt plotxy} task.  The inputs are: 
\small
\begin{verbatim}
#  plotxy :: X-Y plotter/interactive flagger for visibility data

vis              =         ''   #  Name of input visibility
xaxis            =     'time'   #  X-axis: def = 'time': see help for options
yaxis            =      'amp'   #  Y-axis: def = 'amp': see help for options
     datacolumn  =     'data'   #  data (raw), corrected, model, residual (corrected - model)

selectdata       =      False   #  Other data selection parameters
spw              =         ''   #  spectral window:channels: ''==>all, spw='1:5~57'
field            =         ''   #  field names or index of calibrators: ''==>all
averagemode      =         ''   #  Select averaging type: 'vector', 'scalar'
restfreq         =         ''   #  a frequency quanta or transition name. see help for options
extendflag       =      False   #  Have flagging extend to other data points?
subplot          =        111   #  Panel number on display screen (yxn)
plotsymbol       =        '.'   #  Options include . : , o ^ v > < s + x D d 2 3 4 h H | _
plotcolor        =  'darkcyn'   #  Plot color
plotrange        = [-1, -1, -1, -1]  #  The range of data to be plotted (see help)
multicolor       =     'corr'   #  Plot in different colors: Options: none, both, chan, corr
selectplot       =      False   #  Select additional plotting options (e.g, fontsize, title,etc)
overplot         =      False   #  Overplot on current plot (if possible)
showflags        =      False   #  Show flagged data?
interactive      =       True   #  Show plot on gui?
figfile          =         ''   #  ''= no plot hardcopy, otherwise supply name
async            =      False   #  If true the taskname must be started using plotxy(...)
\end{verbatim}
\normalsize

{\bf ALERT:} The {\tt plotxy} task expects all of the scratch columns to
be present in the MS, even if it is not asked to plot the contents.
If you get an error to the effect {\tt "Invalid Table operation:
Table: cannot add a column"} then use {\tt clearcal()} to force these
columns to be made in the MS.  Note that this will clear anything in
all scratch columns (in case some were actually there and being used).

Setting {\tt selectdata=True} opens up the selection sub-parameters:
\small
\begin{verbatim}
selectdata       =       True   #   Other data selection parameters
     antenna     =         ''   #   antenna/baselines: ''==>all, antenna = '3,VA04' 
     timerange   =         ''   #   time range: ''==>all 
     correlation =         ''   #   correlations: default = '' 
     scan        =         ''   #   scan numbers: Not yet implemented
     feed        =         ''   #   multi-feed numbers: Not yet implemented
     array       =         ''   #   array numbers: Not yet implemented
     uvrange     =         ''   #   uv range''==>all; uvrange = '0~100kl' (default unit=meters)
\end{verbatim}
\normalsize
These are described in \S~\ref{section:io.selection}.

Averaging is controlled with the set of parameters
\small
\begin{verbatim}
averagemode      =   'vector'   #  Select averaging type: vector, scalar
     timebin     =        '0'   #  Length of time-interval in seconds to average
     crossscans  =      False   #  Have time averaging cross scan boundaries?
     crossbls    =      False   #  have averaging cross over baselines?
     crossarrays =      False   #  have averaging cross over arrays?
     stackspw    =      False   #  stack multiple spw on top of each other?
     width       =        '1'   #  Number of channels to average
\end{verbatim}
\normalsize
See \S~\ref{section:edit.plot.plotxy.average} below for more on averaging.

You can extend the flagging beyond the data cell plotted:
\small
\begin{verbatim}
extendflag          =   True    #  Have flagging extend to other data points?
     extendcorr     =     ''    #  flagging correlation extension type
     extendchan     =     ''    #  flagging channel extension type
     extendspw      =     ''    #  flagging spectral window extension type
     extendant      =     ''    #  flagging antenna extension type
     extendtime     =     ''    #  flagging time extension type
\end{verbatim}
\normalsize
See \S~\ref{section:edit.plot.plotxy.extend} below for more on flag
extension.

The {\tt restfreq} parameter can be set to a transition or frequency:
\small
\begin{verbatim}
restfreq            =    'HI'   #  a frequency quanta or transition name. see help for options
     frame          =  'LSRK'   #  frequency frame for spectral axis. see help for options
     doppler        = 'RADIO'   #  doppler mode. see help for options
\end{verbatim}
\normalsize
See \S~\ref{section:edit.plot.plotxy.restfreq} below for more on setting rest
frequencies and frames.

Setting {\tt selectplot=True} will open up a set of plotting control
sub-parameters.  
These are described in \S~\ref{section:edit.plot.plotxy.select} below.

The {\tt interactive} and {\tt figfile} parameters allow
non-interactive production of hardcopy plots.  See
\S~\ref{section:edit.plot.plotxy.print} for more details on saving plots to disk.

The {\tt iteration}, {\tt overplot}, {\tt plotrange}, 
{\tt plotsymbol}, {\tt showflags} and {\tt subplot}
parameters deserve extra explanation, and are described below.

For example:
\small
\begin{verbatim}
plotxy(vis='jupiter6cm.ms',                # jupiter 6cm dataset
       xaxis='uvdist',                     # plot uv-distance on x-axis
       yaxis='amp',                        # plot amplitude on y-axis
       field='JUPITER',                    # plot only JUPITER
       selectdata=True,                    # open data selection
       correlation='RR,LL',                #   plot RR and LL correlations
       selectplot=True,                    # open plot controls
       title = 'Jupiter 6cm uncalibrated') #   give it a title 
\end{verbatim}
\normalsize
The plotter resulting from these settings is shown in figure \ref{fig:matplotlib}.  

{\bf ALERT:} The {\tt plotxy} task still has a number of issues.
The averaging has been greatly speeded up in this release, but there
are cases where the plots will be made incorrectly.  In particular,
there are problems plotting multiple {\tt spw} at the same time.
There are sometimes also cases where data that you have flagged in 
{\tt plotxy} from averaged data is done so incorrectly.  This task is
under active development for the next cycle to fix these remaining 
problems, so users should be aware of this.

{\bf ALERT:} Another know problem with ({\tt plotxy}) is that it
fails if the path to your working directory contains spaces in
its name, e.g. {\tt /users/smyers/MyTest/} is fine, but 
{\tt /users/smyers/My\ Test/} is not!

% There are a number of things to keep in mind and to be aware of
% in order to make your plotting and editing go smoother:

% The {\tt field} selection is a minimum match on a
% space-separated list of names. You can use the {\tt
% selectfield(vis=filename, minstring='string')} to test what field
% names and indices you are matching. Similarly, {\tt
% selectantenna(vis=filename, minstring='antname')} will also give you
% this information for antenna names.

%%%%%%%%%%%%%%%%%%%%%%%%%%%%%%%%%%%%%%%%%%%%%%%%%%%%%%%%%%%%%%%%%
\subsubsection{GUI Plot Control}
\label{section:edit.plot.plotxy.control}

You can use the various buttons on the {\tt plotxy} GUI to control
its operation -- in particular, to determine flagging and unflagging
behaviors.

There is a standard row of buttons at the bottom.  These include
(left to right):
\begin{itemize}
\item {\bf Home} --- The ``house'' button (1st on left) returns to
  the original zoom level.
\item {\bf Step} --- The left and right arrow buttons (2nd and 3rd
  from left) step through the zoom settings you've visited.
\item {\bf Pan} --- The ``four-arrow button'' (4th from left) lets you pan
  in zoomed plot.
\item {\bf Zoom} --- The most useful is the ``magnifying glass'' (5th
  from the left) which lets you draw a box and zoom in on the plot.  
\item {\bf Panels} --- The ``window-thingy'' button (second from
  right) brings up a menu to adjust the panel placement in the plot.
\item {\bf Save} -- The ``disk'' button (last on right) saves a
  {\tt .png} copy of the plot to a generically named file on disk.
\end{itemize}

In a row above these, there are a set of other buttons (left to right):
\begin{itemize}
\item {\bf Mark Region} --- If depressed lets you draw rectangles to
  mark points in the panels.  This is done by left-clicking and
  dragging the mouse.  You can Mark multiple boxes before doing
  something.  Clicking the button again will un-depress it and forget
  the regions.  ESC will remove the last region marked.
\item {\bf Flag} --- Click this to Flag the points in a marked region.
\item {\bf Unflag} --- Click this to Unflag any flagged point that
  would be in that region (even if invisible).
\item {\bf Locate} --- Print out some information to the logger on
  points in the marked regions.  
\item {\bf Next} --- Step to the next plot in an iteration.
\item {\bf Quit} --- Exit {\tt plotcal}, clear the window and detach from the MS.
\end{itemize}

These buttons are shared with the {\tt plotcal} tool.

%%%%%%%%%%%%%%%%%%%%%%%%%%%%%%%%%%%%%%%%%%%%%%%%%%%%%%%%%%%%%%%%%
\subsubsection{The {\tt selectplot} Parameters}
\label{section:edit.plot.plotxy.select}

These parameters work in concert with the native matplotlib
functionality to enable flexible representations of data displays. 

Setting {\tt selectplot=True} will open up a set of plotting control
sub-parameters:
\small
\begin{verbatim}
selectplot       =       True   #  Select additional plotting options (e.g, fontsize, title,etc)
     markersize  =        5.0   #  Size of plotted marks
     linewidth   =        1.0   #  Width of plotted lines
     skipnrows   =          1   #  Plot every nth point
     newplot     =      False   #  Replace the last plot or not when overplotting
     clearpanel  =     'Auto'   #  Specify if old plots are cleared or not
     title       =         ''   #  Plot title (above plot)
     xlabels     =         ''   #  Label for x-axis
     ylabels     =         ''   #  Label for y-axis
     fontsize    =       10.0   #  Font size for labels
     windowsize  =        5.0   #  Window size: not yet implemented
\end{verbatim}
\normalsize

\begin{wrapfigure}{r}{2.5in}
  \begin{boxedminipage}{2.5in}
     \centerline{\bf Inside the Toolkit:}
        For even more functionality, you can access the 
        {\tt pl} tool directly using Pylab functions that 
        allow one to annotate, alter, or add
        to any plot displayed in the {\tt matplotlib} plotter 
        (e.g. {\tt plotxy}).
  \end{boxedminipage}
\end{wrapfigure}

The {\tt markersize} parameter will change the size of the plot
symbols.  Increasing it will help legibility when doing screen shots.
Decreasing it can help in congested plots.  The {\tt linewidth}
parameter will do similar things to the lines.

The {\tt skipnrows} parameter, if set to an integer {\tt n} greater than 1,
will allow only every {\tt n}th point to be plotted.  It does this,
as the name suggests, by skipping over whole rows of the MS, so beware
(channels are all within the same row for a given {\tt spw}).  Be
careful flagging on data where you have skipped points!  Note
that you can also reduce the number of points plotted via averaging
(\S~\ref{section:edit.plot.plotxy.average}) or channel striding in the 
{\tt spw} specification (\S~\ref{section:io.selection.spw}).

The {\tt newplot} toggle lets you choose whether or not the
last layer plotted is replaced when {\tt overplot=True}, or whether
a new layer is added.

The {\tt clearpanel} parameter turns on/off the clearing of plot panels
that lie under the current panel layer being plotted. The options are:
{\tt 'none'} (clear nothing), {\tt 'auto'} (automatically clear the
plotting area), {\tt 'current'} (clear the current plot area only), 
and {\tt 'all'} (clear the whole plot panel).

The {\tt title}, {\tt xlabels}, and {\tt ylabels} parameters can
be used to change the plot title and axes labels.

The {\tt fontsize} parameter is useful in order to enlarge the label
fonts so as to be visible when making plots for screen capture, or
just to improve legibility.  Shrinking can help if you have lots of
panels on the plot also.

The {\tt windowsize} parameter is supposed to allow adjustments on
the window size. {\bf ALERT:} This currently does nothing,
unless you set it below 1.0, in which case it will produce an 
error.

%%%%%%%
\subsubsection{ The {\tt iteration} parameter}
\label{section:edit.plot.plotxy.iter}

There are currently four iteration options available:
{\tt 'field'}, {\tt 'antenna'}, and {\tt 'baseline'}.  
If one of these options
is chosen, the data will be split into separate plot displays for each
value of the iteration axis (e.g., for the VLA, the 'antenna' option
will get you 27 displays, one for each antenna).  

An example use of iteration:
\small
\begin{verbatim}
  # choose channel averaging, every 5 channels
  plotxy('n5921.ms','channel',subplot=221,iteration='antenna',width='5')
\end{verbatim}
\normalsize
The results of this are shown in Figure~\ref{fig:plotiter}.  Note
that this example combines the use of {\tt width}, {\tt iteration}
and {\tt subplot}.

\begin{figure}[h!]
\begin{center}
\pngname{plotxy_iter}{6}
\caption{\label{fig:plotiter} The {\tt plotxy} iteration plot.
  The first set of plots from the example in
  \S~\ref{section:edit.plot.plotxy.iter} with {\tt iteration='antenna'}.
  Each time you press the {\bf Next} button, you
  get the next series of plots.} 
\hrulefill
\end{center}
\end{figure}

{\bf NOTE:} If you use {\tt iteration='antenna'} or {\tt 'baseline'},
be aware if you have set {\tt antenna} selection.  You can also
control whether you see auto-correlations or not using the appropriate
syntax, e.g. {\tt antenna='*\&\&*'} or {\tt antenna='*\&\&\&'}
(\S~\ref{section:io.selection.selectdata.antenna}).

%%%%%%%
\subsubsection{ The {\tt overplot} parameter}
\label{section:edit.plot.plotxy.overplot}

The {\tt overplot} parameter toggles whether the current plot will
be overlaid on the previous plot or subpanel (via the {\tt subplot}
setting, \S~{section:edit.plot.plotxy.subplot}) or will overwrite it.
The default is {\tt False} and the new plot will replace the old.

The {\tt overplot} parameter interacts with the {\tt newplot}
sub-parameter (see \S~\ref{section:edit.plot.plotxy.select}).

See \S~\ref{section:edit.plot.plotxy.showflags} for an example using 
{\tt overplot}. 

%%%%%%%
\subsubsection{ The {\tt plotrange} parameter}
\label{section:edit.plot.plotxy.plotrange}

The {\tt plotrange} parameter can be used to specify the size of the
plot.  The format is {\tt [xmin, xmax, ymin, ymax]}.  The units are
those on the plot.  For example,
\small
\begin{verbatim}
   plotrange = [-20,100,15,30]
\end{verbatim}
\normalsize
Note that if {\tt xmin=xmax} and/or {\tt ymin=ymax}, then the values
will be ignored and a best guess will be made to auto-range that axis.
{\tt ALERT:} Unfortunately, the units for the time axis must be
in Julian Days, which are the plotted values.

%%%%%%%
\subsubsection{ The {\tt plotsymbol} parameter}
\label{section:edit.plot.plotxy.symb}

The {\tt plotsymbol} parameter defines both the line or
symbol for the data being drawn as well as the color; from the
matplotlib online documentation (e.g., type {\tt pl.plot?} for help):

\small
\begin{verbatim}
    The following line styles are supported:
        -     : solid line
        --    : dashed line
        -.    : dash-dot line
        :     : dotted line
        .     : points
        ,     : pixels
        o     : circle symbols
        ^     : triangle up symbols
        v     : triangle down symbols
        <     : triangle left symbols
        >     : triangle right symbols
        s     : square symbols
        +     : plus symbols
        x     : cross symbols
        D     : diamond symbols
        d     : thin diamond symbols
        1     : tripod down symbols
        2     : tripod up symbols
        3     : tripod left symbols
        4     : tripod right symbols
        h     : hexagon symbols
        H     : rotated hexagon symbols
        p     : pentagon symbols
        |     : vertical line symbols
        _     : horizontal line symbols
        steps : use gnuplot style 'steps' # kwarg only
    The following color abbreviations are supported
        b  : blue
        g  : green
        r  : red
        c  : cyan
        m  : magenta
        y  : yellow
        k  : black
        w  : white
    In addition, you can specify colors in many weird and
    wonderful ways, including full names 'green', hex strings
    '#008000', RGB or RGBA tuples (0,1,0,1) or grayscale
    intensities as a string '0.8'.
    Line styles and colors are combined in a single format string, as in
    'bo' for blue circles.
\end{verbatim}
\normalsize

%%%%%%
\subsubsection{ The {\tt showflags} parameter}
\label{section:edit.plot.plotxy.showflags}

The {\tt showflags} parameter determines whether {\em only} unflagged
data ({\tt showflags=False}) or flagged ({\tt showflags=True}) data is
plotted by this execution.  The default is {\tt False} and will show
only unflagged ``good'' data.

Note that if you want to plot both unflagged and flagged data, in
different colors, then you need to run {\tt plotxy} twice using
{\tt overplot} (see \S~\ref{section:edit.plot.plotxy.overplot}) 
the second time, e.g.
\small
\begin{verbatim}
> plotxy(vis="myfile", xaxis='uvdist', yaxis='amp' )
> plotxy(vis="myfile", xaxis='uvdist', yaxis='amp', overplot=True, showflags=True )
\end{verbatim}
\normalsize

%%%%%%
\subsubsection{ The {\tt subplot} parameter }
\label{section:edit.plot.plotxy.subplot}

The {\tt subplot} parameter takes three
numbers. The first is the number of y panels (stacking vertically),
the second is the number of xpanels (stacking horizontally) and the
third is the number of the panel you want to draw into. For example,
{\tt subplot=212} would draw into the lower of two
panels stacked vertically in the figure.

An example use of subplot capability is shown in
Fig~\ref{fig:multiplot}.  These were drawn with the commands (for the
top, bottom left, and bottom right panels respectively):
\small
\begin{verbatim}
plotxy('n5921.ms','channel',     # plot channels for the n5921.ms data set
       field='0',                  # plot only first field
       datacolumn='corrected',     # plot corrected data
       plotcolor='',               # over-ride default plot color
       plotsymbol='go',            # use green circles
       subplot=211)                # plot to the top of two panels

plotxy('n5921.ms','x',           # plot antennas for n5921.ms data set
       field='0',                  # plot only first field
       datacolumn='corrected',     # plot corrected data
       subplot=223,                # plot to 3rd panel (lower left) in 2x2 grid
       plotcolor='',               # over-ride default plot color
       plotsymbol='r.')            # red dots

plotxy('n5921.ms','u','v',       # plot uv-coverage for n5921.ms data set
       field='0',                  # plot only first field
       datacolumn='corrected',     # plot corrected data
       subplot=224,                # plot to the lower right in a 2x2 grid
       plotcolor='',               # over-ride default plot color
       plotsymbol='b,')            # blue, somewhat larger dots
                                   # NOTE: You can change the gridding
                                   # and panel size by manipulating
                                   # the ny x nx grid.

\end{verbatim}
\normalsize

\begin{figure}[h!]
\begin{center}
\pngname{ngc5921_multiplot}{5.5}
\caption{\label{fig:multiplot} Multi-panel display of visibility
  versus channel ({\bf top}), antenna array configuration ({\bf bottom left})
  and the resulting uv coverage ({\bf bottom right}). The commands to
  make these three panels respectively are: 
  1) {\tt plotxy('ngc5921.ms', xaxis='channel',
    datacolumn='data', field='0', subplot=211, plotcolor='',
    plotsymbol='go')}
  2) {\tt plotxy('ngc5921.ms', xaxis='x', field='0', subplot=223, plotsymbol='r.')}, 
  3) {\tt plotxy('ngc5921.ms', xaxis='u', yaxis='v', field='0',
    subplot=224, plotsymbol='b,',figfile='ngc5921\_multiplot.png')}.
  }
\hrulefill
\end{center}
\end{figure}

See also \S~\ref{section:edit.plot.plotxy.iter} above, and
Figure~\ref{fig:plotiter} for an example of channel 
averaging using {\tt iteration} and {\tt subplot}.
 
%%%%%%%%%%%%%%%%%%%%%%%%%%%%%%%%%%%%%%%%%%%%%%%%%%%%%%%%%%%%%%%%%
\subsubsection{Averaging in {\tt plotxy}}
\label{section:edit.plot.plotxy.average}

The averaging parameters and sub-parameters are:
\small
\begin{verbatim}
averagemode      = 'vector' #  Select averaging type: vector, scalar
     timebin     =      '0' #  length of time in seconds to average, default='0', or: 'all'
     crossscans  =    False #  have time averaging cross over scans?
     crossbls    =    False #  have averaging cross over baselines?
     crossarrays =    False #  have averaging cross over arrays?
     stackspw    =    False #  stack multiple spw on top of each other?
     width       =      '1' #  number of channels to average, default: '1', or: 'all', 'allspw'
\end{verbatim}
\normalsize

The choice of {\tt averagemode} controls how the amplitudes are calculated
in the average.  The default mode is {\tt 'vector'}, where the complex
average is formed by averaging the real and imaginary parts of the
relevant visibilities.  If {\tt 'scalar'} is chosen, then the
amplitude of the average is formed by a scalar average of the
individual visibility amplitudes.

Time averaging is effected by setting the {\tt timebin} parameter to a
value larger than the integration time.  Currently, {\tt timebin}
takes a string containing the averaging time in seconds, e.g.
\small
\begin{verbatim}
   timebin = '60.0'
\end{verbatim}
\normalsize
to plot one-minute averages.

Channel averaging is invoked by setting {\tt width} to a value greater
than 1.  Currently, the averaging {\tt width} is given as a number of
channels.

By default, the averaging will not cross {\tt scan} boundaries (as set
in the import process).  However, if {\tt crossscans=True}, then
averaging will cross scans.  

Note that data taken in different sub-arrays are never averaged
together. Likewise, there is no way to plot data averaged over {\tt field}.

%%%%%%%%%%%%%%%%%%%%%%%%%%%%%%%%%%%%%%%%%%%%%%%%%%%%%%%%%%%%%%%%%
\subsubsection{Interactive Flagging in {\tt plotxy}}
\label{section:edit.plot.plotxy.flag}

\begin{wrapfigure}{r}{2.5in}
  \begin{boxedminipage}{2.5in}
     \centerline{\bf Hint!}
     In the plotting environments such as {\tt plotxy}, the
     {\tt ESC} key can be used to remove the last region box
     drawn. 
  \end{boxedminipage}
\end{wrapfigure}

Interactive flagging, on the principle of ``see it --- flag it'', is
possible on the X-Y display of the data plotted by {\tt plotxy}.  The user can
use the cursor to mark one or more regions, and then flag, unflag, or list
the data that falls in these zones of the display.

There is a row of buttons below the plot in the window.  You can punch the
{\bf Mark Region} button (which will appear to depress), then mark a region
by left-clicking and dragging the mouse (each click and drag will mark an
additional region).  You can get rid of all your regions by clicking again
on the {\bf Mark Region} button (which will appear to un-depress), or you
can use the {\tt ESC} key to remove the last box you drew.  Once regions are
marked, you can then click on one of the other buttons to take action:
\begin{enumerate}
\item {\bf Flag} --- flag the points in the region(s),
\item {\bf Unflag} --- unflag flagged points in the region(s),
\item {\bf Locate} --- spew out a list of the points in the region(s) to 
   the logger (Warning: this could be a long list!).
\end{enumerate}
Whenever you click on a button, that action occurs without 
forcing a disk-write (unlike previous versions).  If you quit {\tt plotxy}
and re-enter, you will see your previous edits.

\begin{figure}[h!]
\begin{center}
\pngname{ngc5921_markflag2}{2.75}
\pngname{ngc5921_markflag3}{2.75}
\caption{\label{fig:markflags.plotxy} Plot of amplitude versus
uv distance, before (left) and after (right) flagging two marked regions.
The call was:
{\tt plotxy(vis='ngc5921.ms',xaxis='uvdist', field='1445*')}.
}
\hrulefill
\end{center}
\end{figure}

A table with the name {\tt <msname>.flagversions} (where
{\tt vis=<msname>}) will be created in the same directory if
it does not exist already.

It is recommended that you save important flagging stages using
the {\tt flagmanager} task (\S~\ref{section:edit.flagmanager}).

%%%%%%%%%%%%%%%%%%%%%%%%%%%%%%%%%%%%%%%%%%%%%%%%%%%%%%%%%%%%%%%%%
\subsubsection{Flag extension in {\tt plotxy}}
\label{section:edit.plot.plotxy.extend}

Flag extension is controlled using {\tt extendflag=T} and its sub-parameters:
\small
\begin{verbatim}
extendflag          =   True    #  Have flagging extend to other data points?
     extendcorr     =     ''    #  flagging correlation extension type
     extendchan     =     ''    #  flagging channel extension type
     extendspw      =     ''    #  flagging spectral window extension type
     extendant      =     ''    #  flagging antenna extension type
     extendtime     =     ''    #  flagging time extension type
\end{verbatim}
\normalsize
The use of {\tt extendflag} enables the user to plot a subset of the
data and extend the flagging to a wider set.

{\bf ALERT:} Using the {\tt extendflag} options will greatly
slow down the flagging in {\tt plotxy}.  You will see a long delay
after hitting the {\bf Flag} button, with lots of logger messages
as it goes through each flag.  Fixing this requires a refactoring
of {\tt plotxy} which is underway starting in Patch 4 development.

Setting {\tt extendchan='all'} will extend the flagging to other
channels in the same {\tt spw} as the displayed point.  For example,
if {\tt spw='0:0'} and channel 0 is displayed, then flagging will
extend to all channels in spw 0.

The {\tt extendcorr} sub-parameter will extend the flagging beyond the
correlations displayed.  If {\tt extendcorr='all'}, then all
correlations will be flagged, e.g.\ with RR displayed RR,RL,LR,LL will 
be flagged.  If {\tt extendcorr='half'}, then the extension will be
to those correlations in common with that show, e.g.\ with RR
displayed then RR,RL,LR will be flagged.

Setting {\tt extendspw='all'} will extend the flagging to all other
spw for the selection.  Using the same example as above, with
{\tt spw='0:0'} displayed, then channel 0 in ALL spw will be flagged.
Note that use of {\tt extendspw} could result in unintended behavior
if the spw have different numbers of channels, or if it is used in
conjunction with {\tt extendchan}.

{\bf WARNING:} use of the following options, particularly in
conjunction with other flag extensions, may lead to deletion of much
more data than desired.  Be careful!

Setting {\tt extendant='all'} will extend the flagging to all
baselines that have antennas in common with those displayed and
marked.  For example, if {\tt antenna='1\&2'} is shown, then ALL
baselines to BOTH antennas 1 and 2 will be flagged.  Currently, there
is no option to extend the flag to ONLY baselines to the first (or 
second) antenna in a displayed pair, so it is better to use
{\tt flagdata2} to remove specific antennas.

Setting {\tt extendtime='all'} will extend the flagging to all times 
matching the other selection or extension for the data in the marked
region.  

%%%%%%%%%%%%%%%%%%%%%%%%%%%%%%%%%%%%%%%%%%%%%%%%%%%%%%%%%%%%%%%%%
\subsubsection{Setting rest frequencies in {\tt plotxy}}
\label{section:edit.plot.plotxy.restfreq}

The {\tt restfreq} parameter can be set to a transition or frequency
and expands to allow setting of frame information.  For example,
\small
\begin{verbatim}
restfreq            =    'HI'   #  a frequency quanta or transition name. see help for options
     frame          =  'LSRK'   #  frequency frame for spectral axis. see help for options
     doppler        = 'RADIO'   #  doppler mode. see help for options
\end{verbatim}
\normalsize
Examples of transitions include:
\small
\begin{verbatim}
   restfreq='1420405751.786Hz'  #  21cm HI frequency
   restfreq='HI'                #  21cm HI transition name
   restfreq='115.2712GHz'       #  CO 1-0 line frequency
\end{verbatim}
\normalsize
For a list of known lines in the CASA {\tt measures} system, use the
toolkit command {\tt me.linelist()}.  For example:
\small
\begin{verbatim}
CASA <14>: me.linelist()
  Out[14]: 'C109A CI CII166A DI H107A H110A H138B H166A H240A H272A H2CO HE110A HE138B HI OH1612 OH1665 OH1667 OH1720'
\end{verbatim}
\normalsize
{\bf ALERT:} The list of known lines in CASA is currently very
restricted, and will be increased in upcoming releases (to include lines
in ALMA bands for example).

You can use the {\tt me.spectralline} tool method to turn transition names into
frequencies 
\small
\begin{verbatim}
CASA <16>: me.spectralline('HI')
  Out[17]: 
{'m0': {'unit': 'Hz', 'value': 1420405751.786},
 'refer': 'REST',
 'type': 'frequency'}
\end{verbatim}
\normalsize
(not necessary for this task, but possibly useful).

The {\tt frame} sub-parameter sets the frequency frame.  The allowed
options can be listed using the {\tt me.listcodes} method on the
{\tt me.frequency()} method, e.g.
\small
\begin{verbatim}
CASA <17>: me.listcodes(me.frequency())
  Out[17]: 
{'extra': array([], 
      dtype='|S1'),
 'normal': array(['REST', 'LSRK', 'LSRD', 'BARY', 'GEO', 'TOPO', 'GALACTO', 'LGROUP',
       'CMB'], 
      dtype='|S8')}
\end{verbatim}
\normalsize

The {\tt doppler} sub-parameter likewise sets the Doppler system.  The
allowed codes can be listed using the {\tt me.listcodes} method on the
{\tt me.doppler()} method,
\small
\begin{verbatim}
CASA <18>: me.listcodes(me.doppler())
  Out[18]: 
{'extra': array([], 
      dtype='|S1'),
 'normal': array(['RADIO', 'Z', 'RATIO', 'BETA', 'GAMMA', 'OPTICAL', 'TRUE',
       'RELATIVISTIC'], 
      dtype='|S13')}
\end{verbatim}
\normalsize
For most cases the {\tt 'RADIO''} Doppler system is appropriate, but
be aware of differences.

For more information on frequency frames and spectral coordinate
systems, see the paper by Greisen et al. (A\&A, 446, 747, 2006)
\footnote{Also at \url{http://www.aoc.nrao.edu/~egreisen/scs.ps}}.

%%%%%%%%%%%%%%%%%%%%%%%%%%%%%%%%%%%%%%%%%%%%%%%%%%%%%%%%%%%%%%%%%
\subsubsection{Printing from {\tt plotxy}}
\label{section:edit.plot.plotxy.print}

There are two ways to get hardcopy plots in {\tt plotxy}.  

The first is to use the ``disk save'' icon from the interactive plot GUI to 
print the current plot.  This will bring up a sub-menu GUI that will
allow you to choose the filename and format.  The allowed formats are {\tt .png} (PNG), 
{\tt .eps} (EPS), and {\tt svg} (SVG).  If you give the filename with
a suffix ({\tt .png}, {\tt .eps}, or {\tt svg}) it will make a plot of
that type.  Otherwise it will put a suffix on depending on the format
chosen from the menu.

{\bf ALERT:} The plot files produced by the EPS option can be
large, and the SVG files can be very large.  The PNG is the smallest.

The second is to specify a {\tt figfile}.  You probably want to disable the
GUI using {\tt interactive=False} in this case.  The type of plot file
that is made will depend upon the filename suffix.  The allowed
choices are {\tt .png} (PNG), {\tt .eps} (EPS), and {\tt svg} (SVG).

This latter option is most useful from scripts.  For example,
\small
\begin{verbatim}
   default('plotxy')
   vis = 'ngc5921.ms'
   field = '2'
   spw = ''
   xaxis = 'uvdist'
   yaxis = 'amp'
   interactive=False
   figfile = 'ngc5921.uvplot.amp.png'
   plotxy()
\end{verbatim}
\normalsize
will plot amplitude versus uv-distance in PNG format.  No {\tt plotxy}
GUI will appear.

{\bf ALERT:} if
you use this option to print to {\tt figfile} with an {\tt iteration}
set, you will only get the first plot.

%%%%%%%%%%%%%%%%%%%%%%%%%%%%%%%%%%%%%%%%%%%%%%%%%%%%%%%%%%%%%%%%%
\subsubsection{Exiting {\tt plotxy}}
\label{section:edit.plot.plotxy.exit}

You can use the {\bf Quit} button to clear the plot from the
window and detach from the MS.  You can also dismiss the 
window by killing it with the X on the frame, which will also
detach the MS.

You can also just leave it alone.  The plotter pretty much keeps running
in the background even when it looks like it's done!  You can
keep doing stuff in the plotter window, which is where the
{\tt overplot} parameter comes in.  Note that the {\tt plotcal}
task (\S~\ref{section:cal.tables.plotcal}) will use the same window, and
can also overplot on the same panel.

If you leave {\tt plotxy} running, beware of (for instance)
deleting or writing over the MS without stopping.  It may work
from a memory version of the MS or crash.

%%%%%%%%%%%%%%%%%%%%%%%%%%%%%%%%%%%%%%%%%%%%%%%%%%%%%%%%%%%%%%%%%
\subsubsection{Example session using {\tt plotxy}}
\label{section:edit.plot.plotxy.example}

The following is an example of interactive plotting and flagging
using {\tt plotxy} on the Jupiter 6cm continuum VLA dataset.
This is extracted from the script {\tt jupiter6cm\_usecase.py}
available in the script area.

This assumes that the MS {\tt jupiter6cm.usecase.ms} is
on disk with {\tt flagautocorr} already run.

\small
\begin{verbatim}
default('plotxy')

vis = 'jupiter6cm.usecase.ms'

# The fields we are interested in: 1331+305,JUPITER,0137+331
selectdata = True

# First we do the primary calibrator
field = '1331+305'

# Plot only the RR and LL for now
correlation = 'RR LL'

# Plot amplitude vs. uvdist
xaxis = 'uvdist'
yaxis = 'amp'
multicolor = 'both'

# The easiest thing is to iterate over antennas
iteration = 'antenna'

plotxy()

# You'll see lots of low points as you step through RR LL RL LR
# A basic clip at 0.75 for RR LL and 0.055 for RL LR will work
# If you want to do this interactively, set
iteration = ''

plotxy()

# You can also use flagdata2 to do this non-interactively
# (see below)

# Now look at the cross-polar products
correlation = 'RL LR'

plotxy()

#---------------------------------------------------------------------
# Now do calibrater 0137+331
field = '0137+331'
correlation = 'RR LL'
xaxis = 'uvdist'
spw = ''
iteration = ''
antenna = ''

plotxy()

# You'll see a bunch of bad data along the bottom near zero amp
# Draw a box around some of it and use Locate
# Looks like much of it is Antenna 9 (ID=8) in spw=1

xaxis = 'time'
spw = '1'
correlation = ''

# Note that the strings like antenna='9' first try to match the 
# NAME which we see in listobs was the number '9' for ID=8.
# So be careful here (why naming antennas as numbers is bad).
antenna = '9'

plotxy()

# YES! the last 4 scans are bad.  Box 'em and flag.

# Go back and clean up
xaxis = 'uvdist'
spw = ''
antenna = ''
correlation = 'RR LL'

plotxy()

# Box up the bad low points (basically a clip below 0.52) and flag

# Note that RL,LR are too weak to clip on.

#---------------------------------------------------------------------
# Finally, do JUPITER
field = 'JUPITER'
correlation = ''
iteration = ''
xaxis = 'time'

plotxy()

# Here you will see that the final scan at 22:00:00 UT is bad
# Draw a box around it and flag it!

# Now look at whats left
correlation = 'RR LL'
xaxis = 'uvdist'
spw = '1'
antenna = ''
iteration = 'antenna'

plotxy()

# As you step through, you will see that Antenna 9 (ID=8) is often 
# bad in this spw. If you box and do Locate (or remember from
# 0137+331) its probably a bad time.

# The easiset way to kill it:

antenna = '9'
iteration = ''
xaxis = 'time'
correlation = ''

plotxy()

# Draw a box around all points in the last bad scans and flag 'em!

# Now clean up the rest
xaxis = 'uvdist'
correlation = 'RR LL'
antenna = ''
spw = ''

# You will be drawing many tiny boxes, so remember you can
# use the ESC key to get rid of the most recent box if you
# make a mistake.

plotxy()

# Note that the end result is we've flagged lots of points
# in RR and LL.  We will rely upon imager to ignore the
# RL LR for points with RR LL flagged!

\end{verbatim}
\normalsize

%%%%%%%%%%%%%%%%%%%%%%%%%%%%%%%%%%%%%%%%%%%%%%%%%%%%%%%%%%%%%%%%%
\subsection{Plotting antenna positions using {\tt plotants}}
\label{section:edit.plot.plotants}

This task is a simple plotting interface (to the {\tt plotxy}
functionality) to produce plots of the antenna positions (taken from
the {\tt ANTENNA} sub-table of the MS).

The inputs to {\tt plotants} are:
\small
\begin{verbatim}
#  plotants :: Plot the antenna distribution in the local reference frame:
vis       =         ''   #  Name of input visibility file (MS)
figfile   =         ''   #  Save the plotted figure to this file
async     =      False   #  
\end{verbatim}
\normalsize

%%%%%%%%%%%%%%%%%%%%%%%%%%%%%%%%%%%%%%%%%%%%%%%%%%%%%%%%%%%%%%%%%
%%%%%%%%%%%%%%%%%%%%%%%%%%%%%%%%%%%%%%%%%%%%%%%%%%%%%%%%%%%%%%%%%
\section{Non-Interactive Flagging using {\tt flagdata2}}
\label{section:edit.flagdata}

{\bf Alert:} We have a task {\tt flagdata2} that contains the same
functionality of {\tt flagdata} but has a better interface and it is
considerably faster in some situations. Consider using {\tt flagdata2}
instead of {\tt flagdata}.  If you have a number of flagging commands
you want to carry out, there is also now a {\tt flagcmd} task
(\S~\ref{section:edit.flagcmd}) that takes commands from files or
tables as input.

Task {\tt flagdata2} will flag the visibility data set based on the
specified data selections, most of the information coming from a run
of the {\tt listobs} task (with/without {\tt verbose=True}). Currently you can
select based on any combination of: 

\begin{itemize}
   \item antennas ({\tt antenna})
   \item baselines ({\tt antenna})
   \item spectral windows and channels ({\tt spw})
   \item correlation types ({\tt correlation})
   \item field ids or names ({\tt field})
   \item uv-ranges ({\tt uvrange})
   \item times ({\tt timerange}) or scan numbers ({\tt scan})
   \item antenna arrays ({\tt array})
   \item scan intents ({\tt intent})
   \item observation id ({\tt observation})
\end{itemize}

and choose to flag, unflag, clip, and
remove the first part of each scan and/or the 
autocorrelations.

The inputs to {\tt flagdata2} are:
\small
\begin{verbatim}
#  flagdata2 ::  All purpose flagging task based on selections. It allows the combination of several modes.
vis                 =         ''        #  Name of file to flag
flagbackup          =       True        #  Automatically back up the state of flags before the
                                        #   run?
selectdata          =      False        #  Data selection parameters, which will affect all modes
                                        #   (antenna, timerange etc)
manualflag          =      False        #  Flag by antenna, time, field, etc.
clip                =      False        #  Clip data according to value
shadow              =      False        #  Flag shadowed antennas
elevation           =      False        #  Flag data from antennas below and above the given
                                        #   elevation range
quack               =      False        #  Clip beginning or ending of scans
unflag              =      False        #  Unflag the data specified
summary             =      False        #  List rows and data points flagged
async               =      False        #  If true the taskname must be started using
                                        #   flagdata2(...)
\end{verbatim}
\normalsize

There is no default flagging {\tt mode} in flagdata2.  The desired mode needs to be set to True to be used. A set of sub-parameters will
become visible and can be set as desired.

The mode {\tt summary=True} will print out a summary of the current
state of flagging into the {\tt logger}.

By setting {\tt quack=True} it will allow dropping of integrations from the
beginning or ending of scans.  See \S~\ref{section:edit.flagdata.quack} for
details.

By setting {\tt shadow=True} it will allow shadowed data to be flagged,
e.g.\ if it has not already during filling 
(\S~\ref{section:edit.flagdata.shadow}).

Note that with {\tt 'flagdata2'}, it is possible to run multiple modes in one single call to the task. When {\tt 'selectdata=True'}, the
selection will apply to all modes that are set to True. {\tt 'manualflag'} contains a set of sub-parameters for further selection.
\small
\begin{verbatim}
   flagdata2(vis='my.ms', selectdata=True, field='0~3', manualflag=True, mf_field='1', quack=True)
   flagdata2(vis='my.ms', manualflag=True, mf_field='3', mf_timerange = '6:0:0~6:23:00', shadow=True)
\end{verbatim}
\normalsize


Note that often you want to apply many different flagging operations
in a single pass through the data.  For this, we have implemented
a ``list'' mode for the {\tt 'manualflag'} selections where
several flagdata2 task invocations can be combined into a single
{\tt flagdata2} run by giving lists of parameters. 
This is possible for the {\tt 'manualflag'} mode, only.
    
For example, the following three flagdata2 runs:
\small
\begin{verbatim}
   flagdata2(vis='my.ms', manualflag=True, mf_field='3')
   flagdata2(vis='my.ms', manualflag=True, mf_field='3', mf_timerange = '6:0:0~6:23:00')
   flagdata2(vis='my.ms', manualflag=True, mf_field='3', mf_scan='0', mf_spw='0:60;62;64')
\end{verbatim}
\normalsize
can be combined into a single run by:
\small
\begin{verbatim}
   vis = 'my.ms'
   mode = 'manualflag'
   selectdata = True
   
   mf_field     = '3'
   mf_spw       = [ ''   , ''            , '0:60;62:64' ]
   mf_timerange = [ ''   , '6:0:0~6:23:0', ''           ]
   mf_scan      = [ ''   , ''            , '0'          ]
   
   flagdata2()
\end{verbatim}
\normalsize
Note that {\tt mf\_field='3'} is equivalent to {\tt mf\_field=['3','3','3']}.

%%%%%%%%%%%%%%%%%%%%%%%%%%%%%%%%%%%%%%%%%%%%%%%%%%%%%%%%%%%%%%%%%
-------------------------
\subsection{Flag Antenna/Channels}
\label{section:edit.flagdata.ant}

The following commands give the results shown in 
Figure\,\ref{fig:flagdata_antchan}:
\small
\begin{verbatim}
  default{'plotxy')
  plotxy('ngc5921.ms','channel',iteration='antenna',subplot=311)
  default('flagdata2')
  flagdata2(vis='ngc5921.ms',selectdata=True, manualflag=True, antenna='0',spw='0:10~15')
  default plotxy
  plotxy('ngc5921.ms','channel',iteration='antenna',subplot=311)
\end{verbatim}
\normalsize

\begin{figure}[h!]
\begin{center}
\pngname{msplot_channelants}{3}
\pngname{msplot_flagantchan}{3}
\caption{\label{fig:flagdata_antchan} {\tt flagdata2}: Example showing before
  and after displays using a selection of one antenna and a range of
  channels. Note that each invocation of the flagdata2 task represents
  a cumulative selection, i.e., running antenna='0' will flag all
  data with antenna 0, while antenna='0', spw='0:10~15'
  will flag only those channels on antenna 0. }
\hrulefill
\end{center}
\end{figure}


%%%%%%%%%%%%%%%%%%%%%%%%%%%%%%%%%%%%%%%%%%%%%%%%%%%%%%%%%%%%%%%%%
\subsubsection{Manual flagging and clipping in {\tt flagdata}}
\label{section:edit.flagdata.clip}

Manualflag and clipping have been separated in {\tt flagdata2}. They are controlled by separated
parameters, {\tt manualflag=True} and {\tt clip=True} and contain there own sub-parameters:

\small
\begin{verbatim}
manualflag          =       True        #  Flag by antenna, time, field, etc.
     mf_field       =         ''        #  Field names or field index numbers: ''==>all,
                                        #   field='0~2,3C286'
     mf_spw         =         ''        #  spectral-window/frequency/channel
     mf_antenna     =         ''        #  antenna/baselines: ''==>all, antenna = '3,VA04'
     mf_timerange   =         ''        #  time range: ''==>all, timerange='09:14:0~09:54:0'
     mf_scan        =         ''        #  scan numbers: ''==>all
     mf_intent      =         ''        #  observation intent: ''==>all
     mf_feed        =         ''        #  multi-feed numbers: Not yet implemented
     mf_array       =         ''        #  (sub)array numbers: ''==>all
     mf_uvrange     =         ''        #  uv range: ''==>all; uvrange = '0~100klambda', default
                                        #   units=meters
     mf_observation =         ''        #  Select data based on observation ID: ''==>all

clip                =       True        #  Clip data according to value
     clipexpr       =   'ABS RR'        #  Expression to clip on
     clipminmax     =         []        #  Range to use for clipping
     clipcolumn     =     'DATA'        #  Data column to use for clipping
     clipoutside    =       True        #  Clip outside the range, or within it
     channelavg     =      False        #  Average over channels (scalar average)

\end{verbatim}
\normalsize

The following commands give the results shown in 
Figure\,\ref{fig:flagdata}:
\small
\begin{verbatim}
  plotxy('ngc5921.ms','uvdist')
  flagdata2(vis='ngc5921.ms',clip=True, clipexpr='LL',clipminmax=[0.0,1.6],clipoutside=True)
  plotxy('ngc5921.ms','uvdist')
\end{verbatim}
\normalsize

\begin{figure}[h!]
\begin{center}
\pngname{msplot_clipbefore}{3}
\pngname{msplot_clipafter}{3}
\caption{\label{fig:flagdata} {\tt flagdata2}: Flagging example using the clip mode. }
\hrulefill
\end{center}
\end{figure}

The {\tt channelavg} toggle (new in Version 3.0.0) is now available
to (vector) average the data over all channels before doing the
clipping test.  This is most useful when flagging on phase stable or
corrected data (e.g.\ after {\tt applycal} and split to a new
dataset).

%%%%%%%%%%%%%%%%%%%%%%%%%%%%%%%%%%%%%%%%%%%%%%%%%%%%%%%%%%%%%%%%%
\subsubsection{Flagging the beginning of scans}
\label{section:edit.flagdata.quack}

You can use {\tt quack=True} to drop integrations from
the beginning of scans (as in the AIPS task {\tt QUACK}):
\small
\begin{verbatim}
quack               =       True        #  Clip beginning or ending of scans
     quackinterval  =        0.0        #  Quack n seconds from scan beginning/end
     quackmode      =      'beg'        #  Quack mode. 'beg' ==> beginning of scan. 'endb' ==>
                                        #   end of scan. 'end' ==> all but end of scan. 'tail'
                                        #   ==> all but beginning of scan
     quackincrement =      False        #  Flag incrementally in time?
\end{verbatim}
\normalsize
Note that the time is measured from the first integration in the MS
for a given scan, and this is often already flagged by the online
system.

For example, assuming the integration time is 3.3 seconds (e.g. for
VLA), then
\small
\begin{verbatim}
   quack = True
   quackinterval = 14.0 
\end{verbatim}
\normalsize
will flag the first 4 integrations in every scan.

%%%%%%%%%%%%%%%%%%%%%%%%%%%%%%%%%%%%%%%%%%%%%%%%%%%%%%%%%%%%%%%%%
\subsubsection{Flagging shadowed data with mode {\tt 'shadow'} }
\label{section:edit.flagdata.shadow}

Visibilities where one antenna is obscured by another antenna
(typically at low elevations) can be flagged with {\tt shadow=True}:
\small
\begin{verbatim}
shadow        =   True  #  Flag shadowed antennas.
   diameter   =   -1.0  #  Effective diameter (m) to use. -1 ==>antenna diameter
\end{verbatim}
\normalsize
Note that this can only flag data shadowed by antennas known in the MS
(in the same subarray for example), not by antennas not in the dataset.

%%%%%%%%%%%%%%%%%%%%%%%%%%%%%%%%%%%%%%%%%%%%%%%%%%%%%%%%%%%%%%%%%
\subsubsection{Autoflagging.}  

{\bf Alert:} Please use the {\tt testautoflag} task for autoflagging
and RFI detections. Not that this task is temporary and will be
replaced by its final version in the near future. See
\url{http://www.aoc.nrao.edu/~rurvashi/TFCrop/} for details.

%%%%%%%%%%%%%%%%%%%%%%%%%%%%%%%%%%%%%%%%%%%%%%%%%%%%%%%%%%%%%%%%%
%%%%%%%%%%%%%%%%%%%%%%%%%%%%%%%%%%%%%%%%%%%%%%%%%%%%%%%%%%%%%%%%%
\section{Command-based flagging using {\tt flagcmd}}
\label{section:edit.flagcmd}

{\bf Alert:} The {\tt flagcmd} task is under development and
active testing.  

Task {\tt flagcmd} will flag the visibility data set based on a
a specified set of flagging commands using a special flagging 
syntax (\S~\ref{section:edit.flagcmd.syntax}).  These commands
can be input from the {\tt FLAG\_CMD} MS table, from a 
{\tt Flag.xml} SDM table, from an ascii
file, or from input python strings.  Facilities for manipulation,
listing, or plotting of these flags are also provided.

The inputs to {\tt flagcmd} are:
\small
\begin{verbatim}
#  flagcmd :: Flagging task based on flagging commands
vis                 =         ''        #  Name of file to flag
flagmode            =    'table'        #  Flag input mode (table/file/xml/cmd)
     flagfile       =         ''        #  Name of flag command file to input
     flagrows       =         []        #  Rows of flagfile to read
     useapplied     =      False        #  Read in flags marked as applied also?
     reason         =      'Any'        #  Allowed flag REASON types to select

optype              =    'apply'        #  Flagging mode (apply/unapply/save/list/plot/clear/set)
     outfile        =         ''        #  Name of output flag or list file
     flagsort       =  'antenna'        #  combine/sort flags by (antenna/id/reason)
     flagbackup     =       True        #  Automatically backup the FLAG column before execution?
     reset          =      False        #  Reset all flags before application?

async               =      False        #  If true the taskname must be started using flagcmd(...)
\end{verbatim}
\normalsize

The default input mode is {\tt flagmode='table'} which directs the
task to input flag commands from the {\tt FLAG\_CMD} MS internal
table. See \S~\ref{section:edit.flagcmd.flagmode} for more options.

The default operation mode is {\tt optype='apply'} directing the
task to apply relevant flagging commands to the vis data main table.
See \S~\ref{section:edit.flagcmd.optype} for more options.

See \S~\ref{section:edit.flagcmd.syntax} for a description of the
flagging command syntax.

%%%%%%%%%%%%%%%%%%%%%%%%%%%%%%%%%%%%%%%%%%%%%%%%%%%%%%%%%%%%%%%%%
\subsection{Input modes {\tt flagmode}}
\label{section:edit.flagcmd.flagmode}

The {\tt flagmode} parameter selects options for the input mode for
the flagging commands.

Available {\tt flagmode} options are:
\begin{itemize}
   \item {\tt 'table'} --- input from MS table (\S~\ref{section:edit.flagcmd.flagmode.table})
   \item {\tt 'xml'} --- input from XML table (\S~\ref{section:edit.flagcmd.flagmode.xml})
   \item {\tt 'file'} --- input from ASCII file (\S~\ref{section:edit.flagcmd.flagmode.file})
   \item {\tt 'cmd'} --- input from Python strings (\S~\ref{section:edit.flagcmd.flagmode.cmd})
\end{itemize}

\subsubsection{Input flag mode {\tt 'table'}}
\label{section:edit.flagcmd.flagmode.table}

The default input mode is {\tt flagmode='table'} which directs the
task to input flag commands from a {\tt FLAG\_CMD} MS table.  
This has the sub-parameters:
\small
\begin{verbatim}
flagmode            =    'table'        #  Flag input mode (table/file/xml/cmd)
     flagfile       =         ''        #  Name of flag command file to input
     flagrows       =         []        #  Rows of flagfile to read
     useapplied     =      False        #  Read in flags marked as applied also?
     reason         =      'Any'        #  Allowed flag REASON types to select
\end{verbatim}
\normalsize
If {\tt flagfile = ''} then it will look for the {\tt FLAG\_CMD} 
table in the MS given by {\tt vis}.  You can use this sub-parameter to
direct the task to look directly at another table.

The {\tt flagrows} sub-parameter is a simple Python list of the row
numbers of the table to consider in processing flags.  The default is
all rows.

The {\tt useapplied} sub-parameter toggles whether only flag commands
marked as not having been applied are considered (the default), or
to allow (re)processing using all commands.

The {\tt useapplied} sub-parameter selects the {\tt REASON} type to
process.  The default {\tt 'Any''} means all commands, note that
{\tt reason=''} would only select flags who have a blank {\tt REASON}
column entry.

\subsubsection{Input flag mode {\tt 'xml'}}
\label{section:edit.flagcmd.flagmode.xml}

The input mode {\tt flagmode='xml'} directs the
task to input flag commands from a XML SDM online flagging 
{\tt Flag.xml} file.  
When set this opens the sub-parameters:
\small
\begin{verbatim}
flagmode            =      'xml'        #  Flag input mode (table/file/xml/cmd)
     tbuff          =        1.0        #  Time buffer (sec) to pad flags
     antenna        =         ''        #  Allowed flag antenna names to select by
     reason         =      'Any'        #  Allowed flag REASON types to select
\end{verbatim}
\normalsize
Set {\tt flagfile} to the path to the file.  The default 
{\tt flagfile=''} in this mode will look for a file called 
{\tt Flag.xml} inside the MS directory.  Note that if the
data was filled from the SDM using {\tt importevla}
(\S~\ref{section:io.import.evla}) then the relevant XML file
will have been copied to the MS already.

The {\tt tbuff} sub-parameter sets a padding buffer (in seconds)
to the begin and end times of the online flags in the XML file.
As in {\tt importevla}, the online flag time buffer {\tt tbuff} is specified in
seconds, but in fact should be keyed to the intrinsic online 
integration time to allow for events (like slewing) that occur
within an integration period.  This is particularly true for EVLA data,
where a {\tt tbuff} value of $0.5\times$ to $1.5\times$ the
integration time is needed.  For example, if data were taken with
1-second integrations, then at least {\tt tbuff=0.5} should be used,
likewise {\tt tbuff=5} for 10-second integrations.
{\bf Note:} For EVLA data you should use $1.5\times$ (e.g.\ 
{\tt tbuff=15} for 10-second integrations) for data taken in 
early 2011 or before due to a timing error.  We do not yet know what
ALMA data will need for padding (if any).

The {\tt antenna} sub-parameter selects the antennas from which
online flags will be selected (default is all antennas).  For example,
{\tt antenna='ea01'} is a valid choice for EVLA data.

The {\tt antenna} sub-parameter selects by the {\tt reason} field in
the {\tt Flag.xml} file.  The default {\tt 'Any''} means all commands.
Note that {\tt reason=''} would only select flags who have a blank {\tt reason}
field entry.

\subsubsection{Input flag mode {\tt 'file'}}
\label{section:edit.flagcmd.flagmode.file}

The input mode {\tt flagmode='file'} directs the
task to input flag commands from an ASCII file.  
When set this opens the sub-parameters:
\small
\begin{verbatim}
flagmode            =     'file'        #  Flag input mode (table/file/xml/cmd)
     flagfile       =         ''        #  Name of flag command file to input
\end{verbatim}
\normalsize
Set {\tt flagfile} to the path to the file.  The default 
{\tt flagfile=''} in this mode will look for a file called 
{\tt FlagCMD.txt} inside the MS directory.

\subsubsection{Input flag mode {\tt 'cmd'}}
\label{section:edit.flagcmd.flagmode.cmd}

The input mode {\tt flagmode='cmd'} directs the
task to input flag commands from a Python list of strings. 
When set this opens the sub-parameters:
\small
\begin{verbatim}
flagmode            =      'cmd'        #  Flag input mode (table/file/xml/cmd)
     command        =       ['']        #  flag command(s) as list of strings
\end{verbatim}
\normalsize
The {\tt command} sub-parameter is the command string list, with one
command per list element.  For example:
\small
\begin{verbatim}
     command=["mode='shadow'",
              "mode='clip' cliprange='0~1E-10' clipexpr='ABS_RR'",
              "mode='clip' cliprange='0~1E-10' clipexpr='ABS_LL'",
              "mode='quack' quackmode='end' quackinterval=1.0",
              "antenna='ea01' timerange='00:00:00~01:00:00'",
              "antenna='ea11' timerange='00:00:00~03:00:00' spw='0~4'"]
\end{verbatim}
\normalsize
is a valid set of flagging commands.  These are equivalent to the
lines in an ASCII flagging command file for {\tt flagmode='file'}.

%%%%%%%%%%%%%%%%%%%%%%%%%%%%%%%%%%%%%%%%%%%%%%%%%%%%%%%%%%%%%%%%%
\subsection{Operation types {\tt optype}}
\label{section:edit.flagcmd.optype}

The {\tt optype} selects options for operating on the selected
flags and possibly the data.

Available {\tt optype} options are:
\begin{itemize}
   \item {\tt 'apply'} --- apply flag commands to data (\S~\ref{section:edit.flagcmd.optype.apply})
   \item {\tt 'unapply'} --- unapply flags in data (\S~\ref{section:edit.flagcmd.optype.unapply})
   \item {\tt 'save'} --- save commands to {\tt FLAG\_CMD} table (\S~\ref{section:edit.flagcmd.optype.save})
   \item {\tt 'list'} --- list flag commands (\S~\ref{section:edit.flagcmd.optype.list})
   \item {\tt 'plot'} --- plot flag commands (\S~\ref{section:edit.flagcmd.optype.plot})
   \item {\tt 'clear'} --- clear rows from {\tt FLAG\_CMD} table (\S~\ref{section:edit.flagcmd.optype.clear})
   \item {\tt 'set'} --- (re)set row/column values in {\tt FLAG\_CMD} table (\S~\ref{section:edit.flagcmd.optype.set})
\end{itemize}

\subsubsection{Apply flags --- {\tt optype} option {\tt 'apply'}}
\label{section:edit.flagcmd.optype.apply}

The default operation mode is {\tt optype='apply'} directing the
task to apply relevant flagging commands to the vis data main table.
This choice opens the sub-parameters:
\small
\begin{verbatim}
optype              =    'apply'        #  Flagging mode (apply/unapply/save/list/plot/clear/set)
     outfile        =         ''        #  Name of output flag or list file
     flagsort       =  'antenna'        #  combine/sort flags by (antenna/id/reason)
     flagbackup     =       True        #  Automatically backup the FLAG column before execution?
     reset          =      False        #  Reset all flags before application?
\end{verbatim}
\normalsize

The {\tt outfile} sub-parameter tells where to record the applied
flags after application to the data.  If {\tt flagmode='table'} and 
{\tt flagmode=''} (the flag commands come from the {\tt FLAG\_CMD}
table) then this is the the default option {\tt outfile=''} will set
the {\tt APPLIED} column value to {\tt True} in the relevant row of
the table.  If the flag commands came from elsewhere, then they will
be added to the {\tt FLAG\_CMD} table and marked as applied. If
{\tt flagmode} is non-blank, then the commands will be written to the
indicated ASCII file, but not recorded or marked as applied in the
{\tt FLAG\_CMD} table.

The {\tt flagsort} sub-parameter, when non-blank, directs the sorting
of flags by the options {\tt 'antenna'}, {\tt 'id'}, or {\tt
  'reason'}.  The default option {\tt 'antenna'} is most useful, in
particular for cases where the only selection is by antenna-time
(e.g.\ from current EVLA online flags).  In the case the number of
flagging agents activated at the tool level is minimized, the flagging
will be most efficient.  You can also sort by {\tt id} (same as {\tt
  ''}, keeping each flag separate) and by {\tt 'reason'} (not
particularly useful yet).

The {\tt flagbackup} toggle sets whether a new copy of the MS main
table {\tt FLAG} column is written to the {\tt .flagversions} backup
directory for that MS before the requested flagging operation.

The {\tt reset} toggle determines whether {\em all} flags in the MS
are reset (to ``unflagged'') before applying the current set of
flagging commands.  This is equivalent to unflagging all the data
and starting fresh with the new flags.  
{\bf ALERT:} Use this option with care!

\subsubsection{Unapply flags --- {\tt optype} option {\tt 'unapply'}}
\label{section:edit.flagcmd.optype.unapply}

The {\tt unapply} option allows unflagging of data based on the selected flag commands.
This choice opens the sub-parameters:
\small
\begin{verbatim}
optype              =  'unapply'        #  Flagging mode (apply/unapply/save/list/plot/clear/set)
     flagsort       =  'antenna'        #  combine/sort flags by (antenna/id/reason)
     flagbackup     =       True        #  Automatically backup the FLAG column before execution?
\end{verbatim}
\normalsize

The {\tt flagsort} and {\tt flagbackup} sub-parameters behave in this
option as they do in {\tt optype='apply'} 
(\S~\ref{section:edit.flagcmd.optype.apply}).

{\tt ALERT:} Flags based on the {\tt shadow} flagging option cannot be
undone by this method.  You will need to reset all flags and then do
all other flagging operations (or revert to an appropriate backup of
the flags using {\tt flagmanager}).

\subsubsection{Save flags --- {\tt optype} option {\tt 'save'}}
\label{section:edit.flagcmd.optype.save}

The {\tt save} option allows saving the selected flag commands to
the {\tt FLAG\_CMD} table.  These flag commands are not applied to the
data at this time.
This choice opens the sub-parameters:
\small
\begin{verbatim}
optype              =     'save'        #  Flagging mode (apply/unapply/save/list/plot/clear/set)
     outfile        =         ''        #  Name of output flag or list file
     flagsort       =  'antenna'        #  combine/sort flags by (antenna/id/reason)
\end{verbatim}
\normalsize

The {\tt outfile} sub-parameter directs the task to where to save the
flag commands.  The default option {\tt ''} indicates to save to 
the MS {\tt FLAG\_CMD} table, unless the flags come from the MS
{\tt FLAG\_CMD} table ({\tt flagmode='table'} with {\tt flagfile=''})
in which case it will be just listed to the terminal (there is no
point in writing it back to the table it came from).
A non-blank {\tt outfile} value will be interpreted
as a path to an ASCII file that will contain the commands (one string
per line).

The {\tt flagsort} sub-parameter operates in this option as it does in
{\tt optype='apply'} (\S~\ref{section:edit.flagcmd.optype.apply}).

\subsubsection{List flags --- {\tt optype} option {\tt 'list'}}
\label{section:edit.flagcmd.optype.list}

The {\tt 'list'} option will give a listing of the flagging commands.
This choice opens the sub-parameters:
\small
\begin{verbatim}
optype              =     'list'        #  Flagging mode (apply/unapply/save/list/plot/clear/set)
     outfile        =         ''        #  Name of output flag or list file
\end{verbatim}
\normalsize

The {\tt outfile} sub-parameter gives the name of the output ASCII
file to which the listing will be directed.  The default action for
{\tt outfile=''} will be to list the commands to the terminal.

The format of the listing output depends on the source of the flagging
commands. A set of flagging commands specified through 
{\tt flagmode='file'} or {\tt 'cmd'} will be listed directly. The
flagging commands extracted through {\tt flagmode='table'} will
reflect the columns in the table:
\begin{verbatim}
        'Row', 'Timerange', 'Reason', 'Type', 'Applied', 'Lev', 'Sev', 'Command'
\end{verbatim}
while commands from {\tt flagmode='xml'} will be shown with the SDM
XML table fields:
\begin{verbatim}
        'Key', 'FlagID', 'Antenna', 'Reason', 'Timerange'
\end{verbatim}

\subsubsection{Plot flags --- {\tt optype} option {\tt 'plot'}}
\label{section:edit.flagcmd.optype.plot}

The {\tt 'plot'} option will produce a graphical plot of flags of time versus antenna.
This choice opens the sub-parameters:
\small
\begin{verbatim}
optype              =     'plot'        #  Flagging mode (apply/unapply/save/list/plot/clear/set)
     outfile        =         ''        #  Name of output flag or list file
\end{verbatim}
\normalsize
This is only useful for online flags or general flag commands that are
specified by antenna plus timerange using the standard {\tt REASON}
codes that are known SDM {\tt Flag.xml} enumerations.

If the {\tt outfile} sub-parameter is non-blank, then a plotfile will
be made with that name instead of appearing in a matplotlib plotter window
on the users workstation.

{\tt ALERT:} The plotted enumerations are currently only those known
to be allowed EVLA online flags as of 15 April 2011, and include:
\begin{verbatim}
        'FOCUS', 'SUBREFLECTOR', 'OFF SOURCE', 'NOT IN SUBARRAY'
\end{verbatim}
with all others being plotted as {\tt 'Other'}.

\subsubsection{Clear flags --- {\tt optype} option {\tt 'clear'}}
\label{section:edit.flagcmd.optype.clear}

The {\tt 'clear'} option will delete selected rows from the 
{\tt FLAG\_CMD} MS table.
This choice opens the sub-parameters:
\small
\begin{verbatim}
optype              =    'clear'        #  Flagging mode (apply/unapply/save/list/plot/clear/set)
     clearall       =      False        #  Delete all rows from FLAG_CMD?
     rowlist        =         []        #  FLAG_CMD rows to operate on
\end{verbatim}
\normalsize

The {\tt rowlist} sub-parameter is a simple Python list of the row
numbers of the table to consider in processing flags.  The default is
a blank list which indicates the desire to clear all rows.

In either case, if {\tt clearall=False} then nothing will
happen by default as a safeguard.  If {\tt clearall=True}, then a 
blank list will direct the deletion of the selected rows from the table.

{\bf ALERT:} Use this option with care.  You can easily mess up the
{\tt FLAG\_CMD} table.

\subsubsection{Set flags --- {\tt optype} option {\tt 'set'}}
\label{section:edit.flagcmd.optype.set}

The {\tt 'set'} option allows the user to replace the current values
for row/column entries in the {\tt FLAG\_CMD} table with new ones.
This choice opens the sub-parameters:
\small
\begin{verbatim}
optype              =      'set'        #  Flagging mode (apply/unapply/save/list/plot/clear/set)
     rowlist        =         []        #  FLAG_CMD rows to operate on
     setcol         =         ''        #  Name of FLAG_CMD column to set
     setval         =         ''        #  value to set column to
\end{verbatim}
\normalsize

The {\tt rowlist} sub-parameter is a simple Python list of the row
numbers of the table to consider in setting values.  The default is
a blank list which indicates the desire to set values in all rows.

The {\tt setcol} sub-parameter is a Python string indicating the name
of the {\tt FLAG\_CMD} table column to operate on for the selected 
rows.  No default value is given (you will have to set this).  Column
choices are:
\begin{verbatim}
     'APPLIED', 'COMMAND', 'INTERVAL', 'LEVEL', 'REASON', 'SEVERITY', 'TIME', 'TYPE'
\end{verbatim}
You will need to use a value of the appropriate type for that column.
See the MSv2 documentation for the types of these columns (or use 
{\tt browsetable} to look at them, and edit them directly also).

{\bf ALERT:} Use this option with care.  You can easily mess up the
{\tt FLAG\_CMD} table.

%%%%%%%%%%%%%%%%%%%%%%%%%%%%%%%%%%%%%%%%%%%%%%%%%%%%%%%%%%%%%%%%%
\subsection{Flagging command syntax}
\label{section:edit.flagcmd.syntax}

A flagging command syntax has been devised to populate the {\tt
COMMAND} column of the {\tt FLAG\_CMD} table and to direct the 
operation of the {\tt flagcmd} task.

You can use {\tt help flagcmd} inside casapy for this syntax guide also.

This flag command syntax was based on the {\tt flagdata} task
parameters and so should be familiar to users of that task.

Commands are a string (which may contain internal "strings") consisting of
{\tt KEY=VALUE} pairs separated by whitespace (see examples below). 

NOTE: There should be no whitespace between {\tt KEY=VALUE} or within
each {\tt KEY} or {\tt VALUE}, since the simple parser first breaks
command lines on whitespace, then on {\tt "="}.

Each key should only appear once on a given command line/string.

There is an implicit "mode" for each command, with the default
being {\tt 'manualflag'} if not given.

Comment lines start with '\#' and will be ignored. 
Parts of a command line after a ' \# ' token will also be ignored

1. Selection options (used by all flag modes):
\begin{verbatim}
  timerange=''
  antenna=''
  spw=''
  correlation=''
  field=''
  scan=''
  feed=''
  array=''
  uvrange=''
\end{verbatim}
{\bf Note:} a command consisting only of selection key-value pairs
without one of the special {\tt mode} values given below is a 
basic ``manualflag'' operation, 
ie.\ flag all data meeting the selection.

2. Basic elaboration options for online and interface use:
\begin{verbatim}
  id=''              # flag ID tag (not necessary)
  reason=''          # reason string for flag
  flagtime=''        # a timestamp for when this flag was generated (for 
                       user history use)

                       NOTE: there is no flagtime column in FLAG_CMD at
                       this time, but we will propose to add this as an
                       optional column
\end{verbatim}
{\bf NOTE:} These elaboration are currently ignored and not used, but
can be included safely in commands.

3. Extended elaboration options for online and interface use:
\begin{verbatim}
  level=N            # flagging "level" for flags with same reason
  severity=N         # Severity code for the flag, on a scale of 0-10 in order 
                       of increasing severity; user specified
\end{verbatim}
{\bf Note:} these are FLAG\_CMD columns, but their use is not clearly
defined. They are included here for compatibility and future expansion.

4. Extended manual flagging options:
\begin{verbatim}
  unflag=T/F         # this operation is to unflag instead of flag

  mode='clip'
     cliprange='A~B'    # enable clipping based on value
     clipcolumn=''
     clipextend=''
     clipchanavg=T/F
     clipexpr=''        # Note: these expression use "_" instead of whitespace
                          Example: clipexpr='ABS_RR'         

  mode='quack'
     quackinterval=''   # enable scan-based "quacking" 
     quackmode=''
     quackincrement=T/F
\end{verbatim}
{\bf Note:} These flagging operations are currently processed in a separate internal run
to any selection-based basic ``manualflag'' commands.

5. Extended "shadow" option:
\begin{verbatim}
  mode='shadow'      # this is a shadowing command
     diameter=-1
\end{verbatim}
{\bf Note:} Again processed in a separate internal run to any
selection-based basic ``manualflag'' commands.  This currently keys action
of the ``shadow'' flagging tool, but doesn't result in individual
flagging commands for the antennas flagged by this.  Stay tuned for
a future enhancement.


%%%%%%%%%%%%%%%%%%%%%%%%%%%%%%%%%%%%%%%%%%%%%%%%%%%%%%%%%%%%%%%%%
%%%%%%%%%%%%%%%%%%%%%%%%%%%%%%%%%%%%%%%%%%%%%%%%%%%%%%%%%%%%%%%%%
\section{Browse the Data}
\label{section:edit.browse}

The {\tt browsetable} task is available for viewing data directly
(and handles all CASA tables, including Measurement Sets, calibration tables,
and images). This task brings up the CASA Qt
{\tt casabrowser}, which is a separate program.  You can launch this
from outside {\tt casapy}.  

The default inputs are:
\small
\begin{verbatim}
#  browsetable :: Browse a table (MS, calibration table, image)

tablename         =         ''  #   Name of input table
async             =      False  #  If true the taskname must be started using browsetable(...)

\end{verbatim}
\normalsize

Currently, its single input is the {\tt tablename}, so an example would
be:
\small
\begin{verbatim}
   browsetable('ngc5921.ms')
\end{verbatim}
\normalsize
For an MS such as this, it will come up with a browser of the 
{\tt MAIN} table (see Fig~\ref{fig:qcasabrowser1}).  
If you want to look at sub-tables, use the tab 
{\bf table keywords} along the left side to bring up a panel with the sub-tables
listed (Fig~\ref{fig:qcasabrowser2}), then choose (left-click) a table and
{\bf View:Details} to bring it up (Fig~\ref{fig:qcasabrowser3}).  
You can left-click on a cell in a table to view the
contents.

\begin{figure}[h!]
\begin{center}
\pngname{qcasabrowser1}{6}
\caption{\label{fig:qcasabrowser1} {\tt browsetable}: The browser displays
  the main table within a frame. You can scroll
  through the data (x=columns of the {\tt MAIN} table, and y=the rows) or
  select a specific page or row as desired.  By default, 1000 rows of
  the table are loaded at a time, but you can step through the MS in batches.} 
\hrulefill
\end{center}
\end{figure}

\begin{figure}[h!]
\begin{center}
\pngname{qcasabrowser2}{6}
\caption{\label{fig:qcasabrowser2} {\tt browsetable}: You can use the
  tab for {\tt Table Keywords} to look at other tables within an MS.
  You can then double-click on a table to view its contents.} 
\hrulefill
\end{center}
\end{figure}
 
\begin{figure}[h!]
\begin{center}
\pngname{qcasabrowser3}{6}
\caption{\label{fig:qcasabrowser3} {\tt browsetable}: Viewing the 
{\tt SOURCE} table of the MS.}
\hrulefill
\end{center}
\end{figure}

Note that one useful feature is that you can Edit the table and its
contents.  Use the {\tt Edit table} choice from the {\bf Edit} menu,
or click on the {\bf Edit} button.  Be careful with this, and make
a backup copy of the table before editing!

Use the {\tt Close Tables and Exit} option from the {\bf Files} menu
to quit the {\tt casabrowser}.

There are a lot of features in the {\tt casabrowser}
that are not fully documented here.  Feel free to explore the
capabilities such as plotting and sorting!

{\bf ALERT:} You are likely to find that the {\tt casabrowser}
needs to get a table lock before proceeding.  Use the {\tt clearstat}
command to clear the lock status in this case.

%%%%%%%%%%%%%%%%%%%%%%%%%%%%%%%%%%%%%%%%%%%%%%%%%%%%%%%%%%%%%%%%%
%%%%%%%%%%%%%%%%%%%%%%%%%%%%%%%%%%%%%%%%%%%%%%%%%%%%%%%%%%%%%%%%%
%\section{Examples of Data Display and Flagging}
%\label{section:edit.examples}

%See the scripts provied in Appendix~\ref{chapter:scripts} for examples of
%data examination and flagging.  In particular, we refer
%the interested user to the demonstrations for:
%\begin{itemize}
%\item NGC5921 (VLA HI) --- a quick demo of basic CASA capabilities
%      (\ref{section:scripts.ngc5921})
%\item Jupiter (VLA 6cm continuum polarimetry) --- more extensive
%      editing
%      (\ref{section:scripts.jupiter})
%\end{itemize}

%%%%%%%%%%%%%%%%%%%%%%%%%%%%%%%%%%%%%%%%%%%%%%%%%%%%%%%%%%%%%%%%%
%%%%%%%%%%%%%%%%%%%%%%%%%%%%%%%%%%%%%%%%%%%%%%%%%%%%%%%%%%%%%%%%%
%\end{document}

