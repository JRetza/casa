%%%%%%%%%%%%%%%%%%%%%%%%%%%%%%%%%%%%%%%%%%%%%%%%%%%%%%%%%%%%%%%%%
%%%%%%%%%%%%%%%%%%%%%%%%%%%%%%%%%%%%%%%%%%%%%%%%%%%%%%%%%%%%%%%%%
%%%%%%%%%%%%%%%%%%%%%%%%%%%%%%%%%%%%%%%%%%%%%%%%%%%%%%%%%%%%%%%%%

% STM 2007-04-13  split from previous version
% STM 2007-04-15  tool guide version

\chapter{Introduction}
\label{chapter:intro}

%\vspace{5mm}

This document describes how to calibrate and image interferometric and
single-dish radio astronomical data using the CASA (Common Astronomy
Software Application) package.  CASA is a suite of
astronomical data reduction tools and tasks that can be run via the
IPython interface to Python.  CASA is being developed in order to 
fulfill the data post-processing requirements of the ALMA and EVLA
projects, but also provides basic and advanced capabilities useful for
the analysis of data from other radio, millimeter, and submillimeter
telescopes.

The CASA home page can be found at:
\begin{itemize}
  \item \url{http://casa.nrao.edu}
\end{itemize}
From there you can find documentation and assistance for the use
of the package.
Currently, CASA is in an {\bf alpha release}
and this should be taken into account as users begin to learn the
package.

Tools in CASA provide the full capability of the package, and are the
atomic functions that form the basis of data reduction.  Tasks
represent the more streamlined operations that a typical user would
carry out --- in many cases these are Python interface scripts to the
tools, but with specific, limited access to them and a standardized
interface for parameter setting.  The idea for having tasks is that
they are simpler to use, provide a more familiar interface, and are
easier to learn for most astronomers who are familiar with radio
interferometric data reduction (and hopefully for novice users as well).

For the moment, the audience is assumed to have some basic grasp of
the fundamentals of synthesis imaging, so details of how a radio
interferometer or telescope works and why the data needs to undergo
calibration in order to make synthesis images are left to other
documentation --- a good place to start might be Synthesis Imaging in
Radio Astronomy II (1999, ASP Conference Series Vol. 180, eds. Taylor,
Carilli \& Perley).

The CASA Reference Library consists of:
\begin{itemize}
   \item CASA {\bf Synthesis \& Single Dish Reduction Cookbook} --- 
     task-based data analysis walkthrough and instructions;
   \item CASA {\bf in-line help} --- accessed using {\tt help} in the 
              {\bf casapy} interface;
   \item The {\bf CASA Toolkit Guide} --- this document; useful when the 
         tasks do not have everything you want and you need more power
         and functionality, also contains more detailed descriptions
         of the philosophy of data analysis;
   \item The {\bf CASA User Reference Manual} --- 
         to find out what a specific task or tool does and how to use it.   
\end{itemize}



%%%%%%%%%%%%%%%%%%%%%%%%%%%%%%%%%%%%%%%%%%%%%%%%%%%%%%%%%%%%%%%%%
%%%%%%%%%%%%%%%%%%%%%%%%%%%%%%%%%%%%%%%%%%%%%%%%%%%%%%%%%%%%%%%%%
