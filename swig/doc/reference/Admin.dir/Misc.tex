\chapter{Miscellaneous Duties}
\section{ClearDDTS}
We use ClearDDTS as our defect tracking system.  There is fairly extensive
written and web base documentation for the administrator.  ClearDDTs defects
reside in /users/ddts and is licensed to run on tarzan. A cron job runs four
times a day looking for lost defects.

From time-to-time, and email address will need to be corrected. The procedure follows:
\begin{enumerate}
\item logon on to tarzan as ddts
\item run patchbug .i.e.
\begin{verbatim}
patchbug -a Submitter-mail right@address AOCsoXXXXX
\end{verbatim}
\item run adminbug; and inside adminbug run dbms.
\end{enumerate}
Note: this can be done for any field in the ClearDDTs bug system.  I suggest  
consulting the ClearDDTs Administrator's Guide.
\section{Linkscan}
We use linkscan software to scan the AIPS++ web pages for broken and missing
links.  As aips2mgr on tarzan; cd /home/tarzan/httpd/linkscan; ./linkscan.pl.
Once the script has run it produces a series of reports that can be accessed
at http://aips2.nrao.edu/linkscan.

Configuration scripts for linkscan are found in
/home/tarzan/httpd/linkscan/default and userdoc.  Default is setup to scan the
entire aips++ web site, where userdoc will scan only the release doc tree.
\section{Htdig}
Htdig provides the search engine for the AIPS++ site.  Several cron jobs 
(run as wyoung) 
on tarzan that scan the AIPS++ code and docs tree.  Configuration files are found
in /aips++/local/htdig-3.1/conf.  Details on the care and feeding of htdig maybe
found at it's web site http://www.htdig.org.

\section{Email}
Several email aliases are maintained on tarzan in /etc/mail/tarzan.aliases-tail.
The aliases are update once a day around 0300 and require root privileges to 
make any changes.

We use mailman 
(maintained in Charlottesville) to handle the mailing lists although messages are
sent to local aips++ logs in /export/aips++/Mail on tarzan.

\section{Compiler updates}
One of they joys is porting to new compilers or updating the compiler.  Use the following
change proposal template to initiate the process.

\begin{verbatim}

Title:                  Adopt gcc X.X as project compiler
 Person responsible:     The Boss (bossdude@nrao.edu)
 Originator of proposal: The Worker (workerdude@nrao.edu)
 Exploders targeted:     aips2-lib
 Time table:
 Date of issue:          200X Mth XX
 Comments due:           200X Mth XX
 Revised proposal:       200X Mth XX
 Final comments due:     200X Mth XX
 Decision date:          200X Mth XX
 
 Statement of goals:
 
 Adopt gcc X.X as project compiler.
 
 Background:

 Gcc-X.X is the latest version of the gcc compiler series.  It has been adopted as the default
 compiler by most of the major Linux releases.  

 [--- additional back ground text goes here ---]

 For those looking for further details please visit http://www.gnu.org/software/gcc/gcc.html and
 http://www.gnu.org/software/gcc/gcc-3.1/changes.html
 
 Summary:
 --------

 All that should be required is a switch to the new compiler in the local makedefs.


 Expected Impact:
 ----------------

 [ Address the following points for both solaris and linux ]

 Changes to exsisting code:
     [ Detail any changes to existing code here, in particular what needed changing and possibly why
       include filename and line numbers ]

 Unit test conformance:
     [ Run runtests and report differences, and how to resolve unexpected failures (with modified code)]
    

 Assay results:
     [ Run the assay, should give the same results as the old compiler ]

 Benchmark tests
     [ Run a set of benchmarks, current test is imagertester().runtests('mftest'), possibly get a more
       complete set from Sanjay. ] Summarize results.

 Support Tools
    [ Try a few of our standard tools and see if there are any problems and what work arounds may be
      necessary ]

    Rational
    Debugger
 
 ---------------------------------------
\end{verbatim}
