% STM 2007-04-13  split from previous version
% STM 2007-10-11  add into beta
% STM 2007-10-22  put in appendix for Beta Release
% RI  2009-12-17  Release 0 (3.0.0) draft
% STM 2009-12-21  Release 0 (3.0.0) final
% RI 2010-04-10 Release 3.0.1

%\chapter{Simulation}
\chapter[Simulation]{Simulation}
\label{chapter:sim}

{\bfseries New in 3.2:} Updated receiver temperatures for ALMA Cycle 0. (Note
that integration times in submitted proposals should always use the
online or Observation Preparation Tool sensitivity calculator, and
simdata only used to make arguments about imaging quality.)

The capability for simulating observations and datasets from the EVLA
and ALMA are an important use-case for CASA.  This not only allows one
to get an idea of the capabilities of these instruments for doing
science, but also provides benchmarks for the performance and utility
of the software for processing ``realistic'' datasets (with
atmospheric and instrumental effects).  CASA can calculate
visibilities (create a measurement set) for any interferometric array,
and calculate and apply calibration tables representing some of the
most important corrupting effects. {\tt simdata} can also simulate total power observations, instead of or in
combination with interferometric data.


\begin{wrapfigure}{r}{2.5in}
 \begin{boxedminipage}{2.5in}
    \centerline{\bf Inside the Toolkit:}
    The simulator methods are in the {\tt sm} tool.
    Many of the other tools are also helpful when
    constructing and analyzing simulations.
 \end{boxedminipage}
\end{wrapfigure}

CASA's simulation capabilities continue to be improved with each CASA release.
For the most current information, please refer to
\url{http://www.casaguides.nrao.edu}, and click on ``Simulating
Observations in CASA''.
%
Following general CASA practice, the greatest flexibility and richest
functionality is at the Toolkit level.  The most commonly used
procedures for interferometric simulation are encapsulated in the {\tt
simdata} task.  Total power simulation is 
incorporated directly into {\tt
simdata} (one can create and jointly deconvolve synthetic single dish
and interferometric datasets).
%which follows somewhat parallel inputs
%and whose capabilities will being developed more fully in the next
%release.

%{\bf BETA ALERT:} The simulation capabilities are currently under
%development.  What we do have is mostly at the Toolkit level.
%We have only a single task {\tt almasimmos} at the present time.
%Stay tuned.  For the Beta Release, we include this chapter
%in the Appendix for the use of telescope commissioners and software
%developers.

%%%%%%%%%%%%%%%%%%%%%%%%%%%%%%%%%%%%%%%%%%%%%%%%%%%%%%%%%%%%%%%%%
%%%%%%%%%%%%%%%%%%%%%%%%%%%%%%%%%%%%%%%%%%%%%%%%%%%%%%%%%%%%%%%%%
\section{Simulating ALMA with {\tt simdata}}
\label{section:sim.almasimmos}

The inputs are:
\small
\begin{verbatim}
#  simdata :: mosaic simulation task:
project             =      'sim'        #  root for output file names
modifymodel         =      False        #  modify model image
     skymodel       = '$project.skymodel' #  model image to observe or modify

setpointings        =       True        
     integration    =      '10s'        #  integration (sampling) time
     direction      = ['J2000 19h00m00 -40d00m00'] #  "J2000 19h00m00 -40d00m00" or "" to center on model
     mapsize        = ['1arcmin', '1arcmin'] #  angular size of map or "" to cover model
     maptype        = 'hexagonal'       #  hexagonal, square, etc
     pointingspacing =  '1arcmin'       #  spacing in between pointings or "" for 0.5 PB

predict             =       True        #  calculate visibilites using ptgfile
     complist       =         ''        #  optional componentlist to observe with skymodel
     antennalist    = 'alma.out10.cfg'  #  antenna position file or "" for no interferometric MS
     refdate        = '2012/05/21/22:05:00' #  time/date of observation *see help
     totaltime      =    '7200s'        #  total time of observation
     caldirection   =         ''        #  pt source calibrator [experimental]
     calflux        =      '1Jy'        
     sdantlist      =         ''        #  single dish antenna position file or "" for no total power MS
     sdant          =          0        #  single dish antenna index in file

thermalnoise        = 'tsys-atm'        #  add thermal noise: [tsys-atm|tsys-manual|""]
     user_pwv       =        1.0        #  Precipitable Water Vapor in mm
     t_ground       =      269.0        #  ambient temperature

leakage             =        0.0        #  cross polarization
image               =       True        #  (re)image $project.ms to $project.image
     vis            = '$project.ms'     #  Measurement Set(s) to image
     modelimage     =         ''        #  prior image to use in clean e.g. existing single dish image
     imsize         = [128, 128]        #  output image size in pixels (x,y) or 0 to match model
     cell           = '0.1arcsec'       #  cell size with units or "" to equal model
     niter          =        500        #  maximum number of iterations (0 for dirty image)
     threshold      =  '0.01mJy'        #  flux level (+units) to stop cleaning
     weighting      =  'natural'        #  weighting to apply to visibilities
     outertaper     =         []        #  uv-taper on outer baselines in uv-plane
     stokes         =        'I'        #  Stokes params to image

analyze             =       True        #  (only first 6 selected outputs will be displayed)
     showarray      =      False        #  like plotants
     showuv         =       True        #  display uv coverage
     showpsf        =       True        #  display synthesized (dirty) beam
     showmodel      =       True        #  display sky model at original resolution
     showconvolved  =      False        #  display sky model convolved with output beam
     showclean      =       True        #  display the synthesized image
     showresidual   =      False        #  display the clean residual image
     showdifference =       True        #  display difference image
     showfidelity   =       True        #  display fidelity

graphics            =   'screen'        #  display graphics at each stage to [screen|file|both|none]
verbose             =      False        
overwrite           =      False        #  overwrite files starting with $project
async               =      False        #  If true the taskname must be started using simdata(...)
\end{verbatim}
\normalsize

This task takes an input model image or list of components, plus a
list of antennas (locations and sizes), and simulates a particular
observation (specifies by mosaic setup and observing cycles and
times).  The output is a MS suitable for further analysis in CASA, a
synthesized image created from those visibilities, a difference image
between the synthesized image and your sky model convolved with the
output synthesized beam, and a fidelity image. (see ALMA memo 398 for
description of fidelity, which is approximately the output image
divided by the difference between input and output)

{\tt simdata} is modular: one can
modify one's sky model, predict visibilities, corrupt the Measurement
Set, re-image, and analyze the result all separately, provided in a
few cases the filenames are set correctly.  
%%The description of the sky
%%model is not coupled directly to the size and cell size of the
%%output image as it was in {\tt oldsimdata}
%
%Random thermal noise (from the atmosphere and receiver) can optionally
%be added. A realistic model of the troposphere is created from known
%site characteristics (altitude, etc) and used to calculate the
%frequency dependence of the noise.  Similarly, specifications for ALMA
%and EVLA receiver temperature and antenna efficiencies are coded into
%the task to determine a realistic system temperature.  Other
%corrupting effects, such as phase noise, gain drift, and
%cross-polarization, can be added the the MS with the toolkit.
%
%Much more detailed information, and use-case examples, can be found at 
%
%\url{http://casaguides.nrao.edu/index.php?title=Simulating_Observations_in_CASA}
%
% %%%%%%%%%%%%%%%%%%%%%%%%%%%%%%%%%%%%%%%%%%%%%%%%%%%%%%%%%%%%%%%%%
%
%\section{Simulating ALMA with {\tt oldsimdata}}
%\label{section:sim.almasimmos_old}
%
%{\bfseries Warning:} Please switch to the new version of the task {\tt simdata}
%(what was called {\tt simdata2} in CASA 3.0.2).
%
%The inputs are:
%\small
%\begin{verbatim}
%#  oldsimdata :: mosaic simulation task:
%project             =      'sim'        #  root for output files
%complist            =         ''        #  [optional] componentlist table to observe
%modelimage          =         ''        #  model sky image name
%inbright            = 'unchanged'       #  set peak surface brightness in Jy/pixel or "unchanged"
%ignorecoord         =      False        #  change model coordinates
%startfreq           =    '89GHz'        #  [only if ignorecoord=T] frequency of first channel
%chanwidth           =    '10MHz'        #  [only if ignorecoord=T] channel width
%refdate             = '2012/05/21/22:05:00' #  center time/date of observation *see help
%totaltime           =    '7200s'        #  total time of observation
%integration         =      '10s'        #  integration (sampling) time
%scanlength          =          5        #  number of integrations per pointing in the mosaic
%direction           = ['J2000 19h00m00 -40d00m00'] #  mosaic center, or list of pointings
%pointingspacing     =  '1arcmin'        #  spacing in between beams in mosaic
%mosaicsize          = ['1.0arcmin', '1.0arcmin'] #  angular size of desired area to map [*NEW*]
%caldirection        =         ''        #  pt source calibrator [experimental]
%calflux             =      '1Jy'        #
%checkinputs         =       'no'        #  graphically verify parameters [yes|no|only]
%antennalist         = 'alma.out10.cfg'  #  antenna position file
%noise_thermal       =       True        #  add thermal noise
%     noise_mode     = 'tsys-atm'        #  tsys-atm: set PWV and use ATM library; tsys-manual:
%                                        #   set t_sky and tau
%     user_pwv       =        1.0        #  Precipitable Water Vapor in mm [tsys-atm only]
%     t_ground       =      269.0        #  ambient temperature
%     t_sky          =      263.0        #  atmospheric temperature [tsys-manual only]
%     tau0           =        0.1        #  zenith opacity [tsys-manual only]
%cell                = '0.1arcsec'       #  output cell/pixel size
%imsize              = [128, 128]        #  output image size in pixels (x,y)
%threshold           =  '0.01mJy'        #  flux level (+units) to stop cleaning
%niter               =        500        #  maximum number of iterations
%psfmode             =    'clark'        #  minor cycle PSF calculation method
%weighting           =  'natural'        #  weighting to apply to visibilities
%uvtaper             =      False        #  apply additional uv tapering of  visibilities.
%stokes              =        'I'        #  Stokes params to image
%fidelity            =       True        #  Calculate fidelity images
%display             =       True        #  Plot simulation result images,figures
%verbose             =      False        
%\end{verbatim}
%\normalsize
%
%Much more detailed information, and use-case examples, can be found at 
%
%\url{http://casaguides.nrao.edu/index.php?title=Simulating_Observations_in_CASA}

%%%%%%%%%%%%%%%%%%%%%%%%%%%%%%%%%%%%%%%%%%%%%%%%%%%%%%%%%%%%%%%%%
