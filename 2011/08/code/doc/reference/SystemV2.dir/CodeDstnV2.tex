\chapter{Code distribution}
\label{Code distribution}
\index{code!distribution}

\aipspp\ source code distribution utilities \footnote{Last change:
$ $Id$ $}.

% ----------------------------------------------------------------------------

\section{\exe{exhale}}
\label{exhale}
\index{exhale@\exe{exhale}}
\index{code!distribution!export|see{\exe{exhale}}}
\index{base release|see{\exe{exhale}}}
\index{cumulative update|see{\exe{exhale}}}
\index{incremental update|see{\exe{exhale}}}
\index{distribution|see{code, distribution}}
\index{ftp@\unixexe{ftp}!distribution|see{code, distribution}}
\index{update!source code!en masse|see{code, distribution}}

Create \aipspp\ source code updates. No longer used, but maybe 
redone at a later time (WKY, Jan 2007).

\subsection*{Synopsis}

\begin{synopsis}
   \code{\exe{exhale} [\exe{-e}] [\exe{-l}]}
\end{synopsis}

\subsection*{Description}

\exe{exhale} generates updates for the \aipspp\ code distribution system.  It
allows participating \aipspp\ code development sites to maintain a slave
\rcs\ repository which tracks changes to the master (see \exeref{inhale.svn}).

\exe{exhale} runs on the master \aipspp\ host, \host{aips2.nrao.edu}, and is
responsible for producing the following update files which is deposits in
\verb+~+\file{ftp/pub/master} (see \sref{ftp directories}):

\begin{itemize}
\item
   The base release of the \aipspp\ \rcs\ version files (see \exeref{avers}).
   These update files have names of the form
   \file{master-\textit{mm}.000.tar.gz}, where \textit{mm} is the two-digit
   major, or base-release version number.  The base release files are produced
   infrequently, only by special request, and are preserved indefinitely.

\item
   Incremental updates, each of which contains copies of all \rcs\ version
   files which changed since the last time \exe{exhale} ran.  These have names
   of the form \file{master-\textit{mm}.\textit{nnn}.tar.gz}, where
   \textit{nnn} is the three-digit minor, or incremental version number.  They
   are kept for thirty days.

\item
   The \file{VERSION} file is used as a timestamp file for the incremental
   updates.  It also contains the major and minor version numbers of the most
   recent update and the time it was produced.  The update is skipped if no
   files have been modified since the last incremental update.

\item
   A cumulative update file which contains a copy of every \rcs\ version file
   which has changed since the last base release.  This has a name of the form
   \file{master-\textit{mm}.\textit{nnn}.ALL.tar.gz}.  Cumulative update 
   files are kept for seven days.  In addition, cumulative update files
   are kept for all versions listed in the \file{BLESSED} file as having 
   passed the \exeref{testsuite} or having met whatever other criteria
   is deemed suitable.

   The base release file for the current major version is used as the
   timestamp for cumulative updates.
\end{itemize}

\noindent
The update files are \unixexe{tar} files which have been compressed by
\unixexe{gzip} for transmission to remote \aipspp\ sites via \unixexe{ftp}.

\exe{exhale} also maintains a \file{MANIFEST} of all files distributed in each
incremental update, a \file{LOGFILE} of every invokation, and a \file{CHKSUM}
file used by \exeref{inhale.svn} for verification of received files.

\exe{exhale} has many bookkeeping tasks to perform on the infrequent occasions
when it is called upon to produce a new base release, \code{\textit{mm}.000}
(see \exeref{avers}):

\begin{itemize}
\item
   Make temporary copies of the master and archive on which to operate.

\item
   Apply \aipsexee{ax\_master}{ax} in ``archive'' mode (see \exeref{ax}) to
   rescind or rename archive files and directories.

\item
   Create any new directories in the archive that appeared in the master
   during \aipspp\ major version \textit{ll}, where
   \textit{ll}~=~\textit{mm}~-~1.

\item
   Delete rescinded files corresponding to \aipspp\ major version \textit{kk}
   from the master, where \textit{kk}~=~\textit{mm}~-~2.  These will have
   been archived when the base release for version \textit{ll} was produced.

\item
   Perform the following operations on every \rcs\ version file in the master
   repository:

   \begin{itemize}
   \item
      ``Outdate'' \rcs\ revisions corresponding to \aipspp\ major version
      \textit{kk} using the \exe{rcs -o} command, and also delete any symbolic
      names corresponding to the outdated revisions (these older revisions
      will have been archived when the base release for version \textit{ll}
      was produced).  The old revisions are deleted in order to reduce the
      size of the \rcs\ version file for efficient checkin (\exeref{ai}) and
      checkout (\exeref{ao}).  This operation is only performed when
      \exe{exhale} is invoked with the \exe{-e} option.

   \item
      ``Uprev'' the \rcs\ version file to \rcs\ revision number
      \code{\textit{m}.0}, subsequent revisions are numbered from that point.
      In this process, any exant locks are carried forward from \rcs\ revision
      number \code{\textit{l}.\textit{?}} to \code{\textit{m}.0}.  Revision
      \code{\textit{m}.0}, which contains no differences from the previous
      revision, is forcibly created using the \unixexe{ci -f} command.

   \item
      Concatenate \rcs\ revisions numbered \code{\textit{l}.0} to
      \code{\textit{m}.0} in the master into the archive \rcs\ version file
      using a specially written utility called \exeref{rcscat}.  This includes
      rescinded files.  Files created in the master since the previous base
      release are simply copied to the archive.
   \end{itemize}

\item
   Verify the master and archive using \exeref{xrcs}.

\item
   Create the base release tar file for consortium installations in
   \file{/pub/master}.

\item
   Create the base release tar files for end-user installations in
   \file{/pub/code}.
\end{itemize}

\subsection*{Options}

\begin{description}
\item[\exe{-e}]
   Expire mode; ``outdates'' old RCS revisions.  Only useful when
   \exe{exhale} is creating a new base release; ignored otherwise.

\item[\exe{-l}]
   Latchkey mode (see notes); the update produced will be invisible to
   \exeref{inhale.svn} unless it is also invoked with the \exe{-l} option.
\end{description}

\subsection*{Resources}

\exe{exhale} always ignores \file{\$HOME/.aipsrc} (see \exeref{getrc}).

\begin{itemize}
\item
   \code{account.manager}: Account name (and group) of the owner of the
   \aipspp\ source code.  \exe{exhale} only allows itself to be run from this
   account.

\item
   \code{account.programmer}: (Account name and) name of the \aipspp
   programmer group.
\end{itemize}

\subsection*{Notes}

\begin{itemize}
\item
   Latchkey updates are used to restrict the \aipspp\ distribution
   temporarily, for example, when the \aipspp\ sources are undergoing
   significant structural changes which may take several days to complete.
   Instead of switching \exe{exhale} off during this period, it may continue
   to be run in latchkey mode.  The latchkey mechanism hides updates from
   general view, but slave installations which do want to track the changes
   may do so by specifying the \exe{-l} latchkey option to \exeref{inhale.svn}.
   Latchkey mode is terminated as soon as \exe{exhale} is invoked without the
   \exe{-l} option, even if there are no files to be updated.

   The latchkey mechanism works by recording a version and timestamp on the
   second line of the \file{VERSION} file.  Normally \exeref{inhale.svn} obtains
   the current version number from the first line, but if invoked with the
   \exe{-l} option it reads the last line instead.

\item
   \exe{exhale} should only to be run from the machine which holds the master
   copy of \aipspp\ source code in Socorro.

\item
   \exe{exhale} disables checkins (\exeref{ai}), checkouts (\exeref{ao}),
   header changes (\exeref{am}), deletions (\exeref{ax}), and renames
   (\exeref{amv}) for the duration of its execution.

\item
   A new base release is signalled by manually incrementing the major version
   number in \verb+~+\file{ftp/pub/master/VERSION} and setting the minor
   version number to ``-''.

\item
   \exe{exhale} regenerates the \exeref{av} filename cache, consolidating all
   addenda into a single cache file.
\end{itemize}

\subsection*{Diagnostics}

Status return values correspond to the various phases of the operation
\\ \verb+   0+: success
\\ \verb+   1+: initialization error
\\ \verb+   2+: error preparing a new base release
\\ \verb+   3+: error creating incremental or base update
\\ \verb+   4+: error creating cumulative update
\\ \verb+   5+: error creating public release
\\ \verb+   6+: installation error

\subsection*{Examples}

The following \acct{aips2mgr} \unixexe{cron} entry runs \exe{exhale} at 0630
and 1830 each night in Socorro.

\begin{verbatim}
   30 06,18 * * * (. $HOME/.profile ; exhale)
\end{verbatim}

\subsection*{Bugs}

\begin{itemize}
\item
The \file{\$AIPSMSTR} and \file{/pub} directory hierarchies must reside in the
same unix filesystem.

\item
Newly created directories are not propagated if they don't contain newly
created files.

\item
\exe{exhale} does most of its work in temporary directories beneath
\file{/pub/master} and \file{/pub/code}, and cleans up on receiving a
catchable interrupt.  However, sending it a \code{SIGKILL} signal (e.g. via
\unixexe{kill -9}) would at the least cause it not to clean up its work
directories, but if given during the brief though crucial installation phase,
could cause serious damage to the integrity of the code distribution system.
\end{itemize}

\subsection*{See also}

The unix manual page for \unixexe{cron}(1).\\
The unix manual page for \unixexe{ftp}(1).\\
The unix manual page for \unixexe{tar}(1).\\
The GNU manual page for \unixexe{gzip}(1).\\
Section \sref{Accounts and groups}, \aipspp\ accounts and groups.\\
\aipspp\ variable names (\sref{variables}).\\
\exeref{avers}, \aipspp\ version report utility.\\
\exeref{ax}, \aipspp\ code deletion utility.\\
\exeref{getrc}, query \aipspp\ resource database.\\
\exeref{inhale.svn}, \aipspp\ code import utility.\\
\exeref{av}, \aipspp\ filename validation utility.\\
\exeref{rcscat}, concatenate two \rcs\ version files.\\
\exeref{testsuite}, maintain the most recent good cumulative update.\\
\exeref{xrcs}, verify the internal consistency of \rcs\ version files.

\subsection*{Author}

Original: 1992/04/06 by Mark Calabretta, ATNF.

% ----------------------------------------------------------------------------

\newpage
\section{\exe{inhale.svn}}
\label{inhale.svn}
\index{inhale.svn@\exe{inhale.svn}}
\index{code!distribution!import|see{\exe{inhale.svn}}}

Retrieve \aipspp\ source code updates.

\subsection*{Synopsis}

\begin{synopsis}
   \code{\exe{inhale.svn} [\exe{-b}] [\exe{-c|-m}] [\exe{-D}] [\exe{-d}]
	[\exe{-l}] [\exe{-n}] [\exe{-P}] [\exe{-R} NAME] [\exe{-r}\#]
        [\exe{-w}\#] [\exe{-cvsup} data|docs|all] [\exe{binary}]}
\end{synopsis}

\subsection*{Description}

\exe{inhale.svn} is a part of the \aipspp\ code distribution system.  It allows
participating \aipspp\ code development sites access to the code repository.

\exe{inhale.svn} fetches updates from the repository using subversion (SVN).
After installing the updates, \exe{inhale.svn} invokes \exeref{sneeze} to rebuild
the system and documentation.  The rebuild may be done serially or in parallel
for any number of \aipspp architectures on various hosts.

\exe{inhale.svn} maintains the \file{LOGFILE} files in 
\file{\$AIPSCODE}.  These record the version, time of completion, and mode of
the last code updates (see \exeref{avers}).

\subsection*{Options}

\begin{description}
\item[\exe{-b}]
   Use the most recently ``blessed'' release (see notes).  Useful for
   sites that wish to track development but that also require stable
   releases.

\item[\exe{-c}]
   Force a cumulative update.  This should be done regularly (usually once a
   week) to guarantee that the slave copy does not deviate from the master.

\item[\exe{-cvsup} data|docs|all]
   Update the data, docs or both using cvsup.  Cvsup must be in the PATH for
   this option to work.  If a supfile exists it will be used otherwise the 
   supfile will be created.

\item[\exe{-D}]
   Fetch the latest update which has passed the \exeref{testsuite}.

\item[\exe{-d}]
   This option is operative only if the \exe{-n} option is also specified, it
   causes \code{docsys} to be omitted from the list of targets (see
   \exeref{makefiles}.  Note that even if documentation compilation is
   defeated via the \code{DOCSYS} variable in \filref{makedefs} (see
   \exeref{makedefs}) this option may still be useful in preventing
   \code{docsys} from being recorded in the \file{LOGFILE} entry for the
   rebuild.

\item[\exe{-l}]
   Latchkey mode; fetch latchkey updates (if any) together with updates
   intended for general distribution (see \exeref{exhale}).

\item[\exe{-m}]
   Perform a ``mixed'' update: retrieve the incremental update files,
   then invoke \exeref{sneeze} in cumulative mode.  This is useful for
   minimizing file transfer times when performing cumulative style
   updates over slow network links.

\item[\exe{-n}]
   Don't remake the system (\code{allsys}) after installing the update.  It
   does however remake the \code{install} target, and also \code{docsys}
   provided that the \exe{-d} option is not also specified.

\item[\exe{-P}]
  Use passive mode when doing FTP. Useful for firewalls etc. The
  FTP client being used must support passive mode.

\item[\exe{-R} NAME]
  instead of downloading the default distribution, get the one called "NAME".

\item[\exe{-r} count]
   Set the retry count for the ftp transfer, default is 12.  Set this to zero
   to circumvent the \unixexe{ftp}, \exe{inhale.svn} will then assume that all
   necessary files have been fetched from \file{/pub/master} into directory
   \file{\$AIPSLAVE/tmp}, including \file{/pub/master/VERSION}.

\item[\exe{-w} interval]
   Set the retry interval for the ftp transfer in seconds, default is 900.

\item[\exe{binary}]
   This option will download the latest stable linux binary executables,
   libraries and scripts. /home/casa/stable must exist (typically as a symbolic
   link) for the binaries to work.

\end{description}

\subsection*{Resources}

\exe{inhale.svn} always ignores \file{\$HOME/.aipsrc} (see \exeref{getrc}).

\noindent
General resources.

\begin{itemize}
\item
   \code{account.manager}: Account name (and group) of the owner of the
   \aipspp\ source code.  \exe{inhale.svn} only allows itself to be run from this
   account.

\item
   \code{account.programmer}: (Account name and) name of the \aipspp
   programmer group.
\end{itemize}

\noindent
The following resources used by \exe{inhale.svn} should be defined in
\file{\$AIPSROOT/.aipsrc}.

\begin{itemize}
\item
   \code{inhale.svn.base.slave.preserve}: If not false (see \exeref{affirm}) then
   preserve the slave repository when a new base release is installed.

\item
   \code{inhale.svn.base.code.preserve}: If not false then preserve the code areas
   when a new base release is installed.

\item
   \code{inhale.svn.sneeze.hosts}: List of hosts on which to invoke
   \exeref{sneeze} to rebuild \aipspp\ for architectures which differ from
   that of the host on which \exe{inhale.svn} is executing.  An architecture
   extension (see (\exeref{casainit}) may be specified by appending it to the
   hostname with an intervening colon, i.e. \code{<hostname>:<aips\_ext>}.
   The \code{aips\_ext} is explicitly permitted to be null or set to
   "\code{\_}" to signal the host's default architecture.  The \exe{inhale.svn}
   host's name will be ignored if included in the list (unless an extension is
   specified).

\item
   \code{inhale.svn.sneeze.\$HOST.rcmd}: The command \exe{inhale.svn} should use
   to invoke \exeref{sneeze} on remote systems.  Defaults to
   \unixexe{rsh}.  \unixexe{ssh}, the ``secure shell,'' is one possible
   alternative.

\item
   \code{sneeze.sleep}: Use this option to adjust the delay time (in seconds)
   for asyncronous builds.  The default is 600s which should be adequate to 
   avoid the glish concurrency configure problem.

\end{itemize}

\subsection*{Notes}

\begin{itemize}
\item
   \exe{inhale.svn} fetches a \file{CHKSUM} file which it uses to verify the
   integrity of the files it has received.  The checksums are computed with
   the BSD version of \unixexe{sum} which algorithm differs from that of SysV.
   \exe{inhale.svn} may complain about ``files not received in good order'' if 
   \acct{aips2mgr} uses the SysV version of \unixexe{sum}, leading to repeated
   attempts to fetch the update files.  See \sref{Consortium installation}
   for advice on sidestepping this problem.

\item
   \exe{inhale.svn} fetches the file \file{/pub/master/BLESSED} which it
   uses, when invoked with the \exe{-b} option, in conjunction with the
   \file{VERSION} files to determine which files need to be fetched.
   \exe{inhale.svn} will abort if it is invoked with the \exe{-b} option and
   it cannot parse the \file{BLESSED} file to determine the most
   recently-blessed version for the local system's architecture.
   \exe{inhale.svn} will default to normal behavior if the \file{BLESSED}
   file is not present.  Versions are listed, individually by
   architecture, in \file{BLESSED} when they have either passed the
   \exeref{testsuite} or have met whatever other criteria is deemed
   suitable.

\item
   If ``blessed'' mode is used on an \exe{inhale.svn} host that invokes
   \exeref{sneeze} on multiple client machines of different
   architectures then those clients will be updated to the most
   recently-blessed version for the \exe{inhale.svn} host's architecture,
   not their own.

\item
   The incremental (and cumulative) update files contain the \file{VERSION}
   file for the {\em preceeding} version.  This is used in multi-increment
   updates.

\item
   \exe{inhale.svn} checks the directory ownerships and permissions on all
   \aipspp\ slave and code areas, reports incorrect ownerships, and
   silently fixes the group ownerships and permissions where possible.

\item
   Since \exe{inhale.svn} may invoke \exeref{sneeze} to rebuild the \aipspp\ 
   system which includes the \exe{inhale.svn} executable it must take precautions
   to avoid overwriting itself.  It does so by copying itself to \exe{inhale.svn-}
   and then \unixexe{exec}'ing this.

\item
   \exe{inhale.svn} is a critical part of the \aipspp\ code distribution system.
   If a broken version was distributed to consortium sites, then it probably
   would not be able to fetch and install a repaired version subsequently
   distributed.  The code distribution system would be broken.
\end{itemize}

\subsection*{Diagnostics}

Status return values correspond to the various phases of the operation
\\ \verb+   0+: success
\\ \verb+   1+: initialization error
\\ \verb+   2+: initial ftp error
\\ \verb+   3+: main ftp error
\\ \verb+   4+: error extracting or installing the new \rcs\ files
\\ \verb+   5+: error checking out the new code
\\ \verb+   6+: error rebuilding the system

\subsection*{Examples}

The following \acct{aips2mgr} \unixexe{cron} job entry runs \exe{inhale.svn} at
2300 each night performing a cumulative update on Saturday evening, and
incremental updates on all other evenings:

\begin{verbatim}
   # Update the AIPS++ slave directories, incremental Sun-Fri, cumulative Sat.
   00 23 * * 0-5 (. $HOME/.profile ; inhale.svn) 2>&1 | \
      mail -s "Epping inhale.svn" aips2-inhale.svn@nrao.edu
   00 23 * * 6   (. $HOME/.profile ; inhale.svn -c) 2>&1 | \
      mail -s "Epping inhale.svn -c" aips2-inhale.svn@nrao.edu
\end{verbatim}

\noindent
(Note that all \unixexe{cron} entries must be one-liners but they are broken
here for clarity.)  The \exe{inhale.svn} log is emailed to the \code{aips2-inhale.svn}
exploder which allows the code distribution system to be monitored.  An
alternative method is to forward all of \acct{aips2mgr}'s mail via the
\file{\$AIPSROOT/.forward} file, or as a mail alias.

\subsection*{See also}

The unix manual page for \unixexe{tar}(1).\\
The unix manual page for \unixexe{cron}(1).\\
The unix manual page for \unixexe{rsh}(1).\\
The unix manual page for \unixexe{ssh}(1).\\
\aipspp\ variable names (\sref{variables}).\\
Section \sref{Accounts and groups}, \aipspp\ accounts and groups.\\
\exeref{affirm}, get the Boolean value of a set of arguments.\\
\exeref{avers}, \aipspp\ version report utility.\\
\exeref{ax}, \aipspp\ code deletion utility.\\
\exeref{exhale}, \aipspp\ code export utility.\\
\exeref{getrc}, query \aipspp\ resource database.\\
\filref{makefiles}, GNU makefiles used to rebuild \aipspp .\\
\exeref{sneeze}, \aipspp\ system rebuild utility.\\
\exeref{testsuite}, maintain the most recent good cumulative update.

\subsection*{Author}

Original: 1992/04/08 by Mark Calabretta, ATNF.

% ----------------------------------------------------------------------------
 
\newpage
\section{\exe{inhold}}
\label{inhold}
\index{inhold@\exe{inhold}}
\index{code!development!synchronization|see{\exe{inhold}}}
\index{code!distribution!synchronization|see{\exe{inhold}}}
\index{inhale.svn@\exe{inhale.svn}!synchronization|see{\exe{inhold}}}
\index{synchronization|see{code, distribution}}
 
Wait for \exeref{inhale.svn} to finish.

\subsection*{Synopsis}
 
\begin{synopsis}
   \code{\exe{inhold} [\exe{-t}\#]}
\end{synopsis}
 
\subsection*{Description}

\exe{inhold} waits for \exeref{inhale.svn}) to finish rebuilding the system.  It
may be used to synchronize a programmer rebuild, especially as initiated via a
\unixexe{cron} job, so that it starts shortly after \exeref{inhale.svn} finishes.

\exe{inhold} uses \exeref{tract} to determine when \file{\$AIPSARCH/LOGFILE}
(\sref{variables}) is updated.
 
\subsection*{Options}
 
\begin{description}
\item[\exe{-t} interval]
   Time resolution, in seconds, to test for the completion of \exeref{inhale.svn}
   (default 300).

\end{description}
 
\subsection*{Resources}
 
none

\subsection*{Notes}
 
\begin{itemize}
\item
   The time resolution specified by the \exe{-t} option determines the sleep
   interval.  An extra 60 seconds is added to this to derive the maximum age
   of the \file{LOGFILE}.  This accounts for slow \textsc{nfs} updates of
   \file{LOGFILE}'s timestamp.
\end{itemize}

\subsection*{Diagnostics}
 
Status return values:
\\ \verb+   0+: success
\\ \verb+   1+: initialization error
 
\subsection*{Examples}
 
An example programmer \unixexe{cron} entry might be as follows:

\begin{verbatim}
   00 23 * * 6 . $HOME/.profile ; inhold && rebuild
\end{verbatim}

\noindent
where \exe{rebuild} is some script written by the programmer to rebuild the
\aipspp\ workspace.  The \unixexe{cron} start time should coincide with the
start time of \exeref{inhale.svn}.

 
\subsection*{See also}
 
The unix manual page for \unixexe{cron}(1).\\
\aipspp\ variable names (\sref{variables}).\\
\exeref{inhale.svn}, \aipspp\ code import utility.\\
\exeref{tract}, Report the age of a file or directory.

% ----------------------------------------------------------------------------

\newpage
\section{\exe{sneeze}}
\label{sneeze}
\index{sneeze@\exe{sneeze}}
\index{code!distribution!rebuild|see{\exe{sneeze}}}
\index{system!generation!rebuild|see{\exe{sneeze}}}

\aipspp\ system rebuild utility.

\subsection*{Synopsis}

\begin{synopsis}
   \code{\exe{sneeze} [\exe{-e} extlist] [-l] [\exe{-m} mode] [\exe{-s} seconds] targets}
\end{synopsis}

\subsection*{Description}

\exe{sneeze} is a part of the \aipspp\ code distribution system.  It is
invoked by \exeref{inhale.svn}) to rebuild the \aipspp\ system areas
including object libraries, executables, and documentation.  It can rebuild
serially the systems for several \code{aips\_ext}s (see \exeref{casainit}).

Before rebuilding the system, \exe{sneeze} invokes \aipsexee{ax\_master}{ax}
with the \exe{-system} option to delete unwanted files and directories from
the system areas.  On cumulative updates it invokes \aipsexee{ax\_master}{ax}
in cumulative mode and this deletes all static object libraries and class
template repositories in addition to applying all deletions recorded in
\aipsexee{ax\_master}{ax}'s action list.  \exe{sneeze} also invokes
'\code{gmake -C \$AIPSCODE cleansys}' (\sref{variables}) on cumulative updates
to further clean up the system areas.

At the end of a cumulative update \exe{sneeze} invokes
'\code{gmake -f \$AIPSARCH/makedefs diagnostics}' to report the installed
versions of third-party utilities essential for \aipspp\ such as
\exeref{gmake}, and also to print the current values of all \filref{makedefs}
variables.

For a base installation (\exe{-m} \code{base}), after the rebuild is complete
and if not contra-indicated by the \code{sneeze.base.\$ARCH.preserve}
resource, \exe{sneeze} preserves a copy of the system in
\file{\$AIPSROOT/base-MM/...}, where \file{MM} is the major version number of
the new base release, and then invokes
'\code{gmake -C \$AIPSROOT/base-\$NEWMAJOR/code/install aipsroot}' to remake
the root level directory.

\exe{sneeze} maintains the \file{LOGFILE} file in \file{\$AIPSARCH} which
records the version, time of completion, and \exeref{gmake} targets of the
last rebuild (see \exeref{avers}).

\subsection*{Options}

\begin{description}
\item[\exe{-e} extlist]
   Serially rebuild the systems corresponding to each \code{aips\_ext} in the
   blank-, or colon-separated list.  If no \exe{-e} option is given
   \exe{sneeze} just rebuilds the host's default architecture as specified in
   \filref{aipshosts}.  An ``\code{\_}'' in the extlist signals no extension.

   If there are any more \code{aips\_ext}s to be built after \exe{sneeze} has
   finished the current one then it reinvokes itself with the \exe{-e} option
   specifying an extlist from which the current \code{aips\_ext} has been
   removed.

\item[\exe{-l}]
   Log mode - if specified then the output of sneeze will be logged to a file
   specified by the \code{sneeze.\$HOST.\$AIPSEXT.logfile} resource and mailed
   to the addresses specified by the \code{sneeze.\$HOST.\$AIPSEXT.logmail}
   resource.  If these are not defined the log is sent to
   \file{\$AIPSARCH/sneeze.log} and the mail to \acct{aips2-inhale.svn@nrao.edu}.

\item[\exe{-m} mode]
   Mode of the rebuild, \code{incremental} (default), \code{cumulative}, or
   \code{base}.

\item[\exe{-s} seconds]
   Interval in seconds to sleep before commencing build.  This option is used
   to prevent concurrent builds from configuring glish at the same time.

\end{description}

If no targets are specified (\filref{makefiles}) \exe{sneeze} remakes
\code{allsys}.

\subsection*{Resources}

\exe{sneeze} always ignores \file{\$HOME/.aipsrc} (see \exeref{getrc}).

\noindent
General resources.

\begin{itemize}
\item
   \code{account.manager}: Account name (and group) of the owner of the
   \aipspp\ source code.  \exe{sneeze} only allows itself to be run from this
   account.

\item
   \code{account.programmer}: (Account name and) name of the \aipspp
   programmer group.
\end{itemize}

\noindent
The following resources used by \exe{sneeze} should be defined in
\file{\$AIPSROOT/.aipsrc}.

\begin{itemize}
\item
   \code{sneeze.\$HOST.\$AIPSEXT.logfile}: File in which the log produced by
   \exe{sneeze} is preserved.

\item
   \code{sneeze.\$HOST.\$AIPSEXT.logmail}: Addresses to which the log produced
   by \exe{sneeze} is mailed.

\item
   \code{sneeze.\$HOST.\$AIPSEXT.mailer}: Mail delivery program to use
for log mailings.  Defaults to \unixexe{mail}.

\item
   \code{sneeze.base.\$ARCH.preserve}: If not false (see \sref{affirm}) then
   preserve the system for \file{\$AIPSARCH} in \file{\$AIPSROOT/base-MM/...}
   when a new base release is installed, where \file{MM} is the major version
   number of the new base release.
\end{itemize}

See also the \code{inhale.svn.sneeze.hosts} resource used by \exeref{inhale.svn} to
control how \exe{sneeze} is invoked.

\subsection*{Notes}

\begin{itemize}
\item
   \exe{sneeze} is mainly intended for the use of \exeref{inhale.svn} rather than
   for general usage.

\item
   It is assumed that the code areas have already been brought up-to-date with
   respect to the slave repositories (by \exeref{inhale.svn}).

\item
   \exe{sneeze} checks the directory ownerships and permissions on all
   \aipspp\ system areas, reports incorrect ownerships, and silently fixes the
   group ownerships and permissions where possible.

\item
   Since \exe{sneeze} rebuilds the \aipspp\ system which includes the
   \exe{sneeze} executable it must take precautions to avoid overwriting
   itself.  It does so by copying itself to \exe{sneeze-} which it then
   \unixexe{exec}s.  This also provides an opportunity for it to redirect its
   own output.
\end{itemize}

\subsection*{Diagnostics}

Status return values
\\ \verb+   0+: success
\\ \verb+   1+: initialization error

\subsection*{Examples}

\exe{sneeze} is used by \exeref{inhale.svn} and should not normally be invoked
manually.  \exeref{inhale.svn} invokes it asynchronously on remote hosts with the
\exe{-l} option using \exeref{.rshexec}), then executes it synchronously on
the local host without the \exe{-l} option.

\subsection*{See also}

The unix manual page for \unixexe{mail}(1).\\
\aipspp\ variable names (\sref{variables}).\\
\exeref{affirm}, get the Boolean value of a set of arguments.\\
\filref{aipshosts}, \aipspp\ hosts database.\\
\exeref{avers}, \aipspp\ version report utility.\\
\exeref{ax}, \aipspp\ code deletion utility.\\
\exeref{getrc}, query \aipspp\ resource database.\\
\exeref{inhale.svn}, \aipspp\ code import utility.\\
\filref{makefiles}, GNU makefiles used to rebuild \aipspp.

\subsection*{Author}

Original: 1993/09/13 by Mark Calabretta, ATNF.
