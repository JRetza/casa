\chapter{Introduction}\label{Programming:Intro}

This document is intended for use by people who are either implementing
code for {\tt aips++} or who are using {\tt aips++} libraries for their
own work but are not actively extending the system.  

\label{Intro:Rumbaugh Methodology}
\label{Intro:OMT --- Object Modeling Technique}

While this document is meant for people who are actively writing code,
it must be emphasized that work in the {\tt aips++} system should be
proceeded by analysis and design phases unless the functionality is very
straightforward. The methodology that the {\tt aips++} project has chosen
is the {\em Object Modeling Technique} (OMT) \footnote{Object-Oriented
Modeling and Design; Rumbaugh, Blaha, Premerlani, Eddy, and Lorensen
(Prentice-Hall:1991).} or {\em Rumbaugh} methodology.

\label{Intro:aips2documentation}
The Programmers Reference Manual is part of the {\tt aips++} {\em Documentation System} ---
other titles which may be of interested include:

\begin{itemize}
\item \htmladdnormallink{System}{../System/System.html}
System \index{System} installation and management.
\item \htmladdnormallink{Getting Started in AIPS++}{../../user/gettingstarted/gettingstarted.html}
A User's guide to getting started in AIPS++.
\item \htmladdnormallink{AIPS++ How to .*}{../HowTos/HowTos.html}
Useful tidbits from the AIPS++ group.
\end{itemize}

This manual assumes that you are reasonably fluent in C++. If that is
not the case, we suggest that the following would make a good bookshelf
of C++ books:

\label{Intro:C++ book reccomendations}\index{C++ programming books}
\begin {itemize}
\item
{\em The C++ Programming Language (Second Edition)}; Stroustrup (Addison
Wesley:1991). This is a good introductory book (the first edition,
however, was terrible).
\item
{\em The C++ Primer???? (Second Edition)}; Lippman (Addison
Wesley?:1991??). This is another good introductory book.
\item
{\em The Annotated C++ Reference Manual}; Ellis and Stroustrup
(Addison Wesley:1990). This is the book the (unfinished) C++
standardization effort is based on. Until the language is standardized
this is the "bible."
\item
{\em Advanced C++: Programming Styles and Idioms}; Coplien (Addison
Wesley: 1992). A book that shows you how to do things when the more
elementary techniques fail.
\item
{\em Effective C++}; Meyer (Addison Wesley:1992). A very useful
compendium of things
to know about programming in C++. You should probably be able to
understand everything in this book before writing any {\tt aips++}
classes.
\item
{\em C++ Programming Guidelines}; Plum and Saks (Plum Hall:1991). This
book is dull reading by it has a lot of useful information.
\end{itemize}

Please let us (Brian Glendenning, bglenden@nrao.edu, or Wes Young,
wyoung@nrao.edu) know of any enhancements or
changes that you feel should be made to this document.
If you are an aips2-worker you can
of course make any additions or changes yourself (please, however, check
with us first).

A word about {\em Standards} and {\em guidelines}. There are two
competing views about these:

\begin {enumerate}
\item
That they are invaluable. A consistent "look and feel" allows you to much
more quickly understand what the code is trying to do. Moreover, some
standards will outlaw questionable practices and squash bugs before they
can be written.
\item
They are not very important. The semantics of the code take all the time
it takes to understand it. We're professional enough to understand what
we're doing and rigid rules won't always work. Besides, I just gotta be
me.
\end{enumerate}

In {\tt aips++} we adopt the first view-point. It might only alleviate
only the most superficial problems, but better something than nothing.
The current list is not that large or onerous in any event. (In fact,
for standards and guidelines to be followed, there must be few enough of
them that it is reasonable to expect people to know them).
