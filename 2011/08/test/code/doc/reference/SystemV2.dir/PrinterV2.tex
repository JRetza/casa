\chapter{Printer}
\label{Printer}
\index{print}

Printer utilities provided with \aipspp\ \footnote{Last change:
$ $Id$ $}.

% ----------------------------------------------------------------------------

\section{\exe{lwf\_a}}
\label{lwfa}
\index{lwf\_a@\exe{lwf\_a}}
\index{PostScript@\textsc{PostScript}!filter|see{\exe{lwf\_a}}}
\index{print!PostScript filter@\textsc{PostScript} filter|see{\exe{lwf\_a}}}

\textsc{ascii} to \textsc{PostScript} filter.

\subsection*{Synopsis}

\begin{synopsis}
   \code{\exe{lwf\_a} [\exe{-1}] [\exe{-b}] [\exe{-c}\#] [\exe{-d}\#]
      [\exe{-f}font] [\exe{-g}\#] [\exe{-i}\#] [\exe{-l}] [\exe{-n}file]
      [\exe{-o}list] [\exe{-p}\#] [\exe{-r}] [\exe{-s}\#] [\exe{-t}\#]
      [\exe{-u}] [\exe{-x}] [file1 [file2 ...]]}
\end{synopsis}

\subsection*{Description}

\exe{lwf\_a} is a \textsc{PostScript} filter.  It can process multiple input
files into a single multi-document \textsc{PostScript} file.  If no files are
specified input is read from \file{stdin}.

\exe{lwf\_a} correctly processes form feed characters and tabs and also
understands backspacing.  If a fixed-width font is used, such as one of those
in the Courier family, then backspaces followed by underscores may be used for
underlining.

\exe{lwf\_a} can also display 7-bit and 8-bit \textsc{ascii} control codes as
described by the \exe{-d} option below.

\subsection*{Options}

\begin{description}
\item[\exe{-1}]
   Print in single-sided mode; the default is to print in two-sided (duplex)
   mode if the printer supports it.

\item[\exe{-b}]
   Normally a header is written at the top of each page.  The \exe{-b} flag
   prevents this and instead produces a separate banner page.

\item[\exe{-c}\code{\#}]
   Specifies multicolumn output.  The page is divided into the specified
   number of equally spaced columns without regard to whether the text fits
   within them.  The default is 1.

\item[\exe{-d}\code{\#}]
   Display \textsc{ascii} control codes (\textsc{ascii} \code{00-1f},
   \code{7f-ff}):

   \begin{itemize}
   \item
      0: Interpret \code{BS}, \code{HT}, \code{NL}, \code{NP}, and \code{CR}
      and ignore all other control codes.

   \item
      1: Interpret \code{NUL}, \code{BS}, \code{HT}, \code{NL}, \code{NP}, and
      \code{CR} and display all other control codes in two-character mnemonic
      form (default).

   \item
      2: Display all control codes in two-character mnemonic form and also
      interpret \code{HT}, \code{NL}, \code{NP}, and \code{CR}.

   \item
      3: Same as 2 except that lines are broken at 80 characters.  Note that
      since horizontal tabs are still interpreted the printed line may exceed
      80 columns.

   \item
      4: Display all control codes in two-character mnemonic form and don't
      interpret any of them.  Lines will be broken at 80 characters.

   \item
      5: Interpret the input as a non-\textsc{ascii} byte stream, displaying
      each byte in hexadecimal notation.  Lines will be broken at 80
      characters.
   \end{itemize}

   Modes 1, 2, and 3 are suitable for printing text files which may contain a
   few control codes, for example a \file{.cshrc} file which sets variables
   containing terminal escape sequences.  Modes 2 and 3 might be useful for
   debugging makefiles since horizontal tabs in these have a special
   significance.  Modes 4 and 5 are suitable for dumping binary files.   Mode
   4 might be used for examining the \textsc{ascii} character strings contined
   in an object module, while mode 5 might be used for dumping a data file.

   The two-character mnemonics are printed at half size.

   \textsc{ascii} control codes are listed below.

\item[\exe{-f}\code{font}]
   Set the font type, the default is Courier.  Courier, Helvetica, and Times
   are \textsc{PostScript} standards; the following fonts are supported by
   most \textsc{PostScript} printers:

\begin{verbatim}
   AGB    AvantGarde-Book
   AGBO   AvantGarde-BookOblique
   AGD    AvantGarde-Demi
   AGDO   AvantGarde-DemiOblique
   BD     Bookman-Demi
   BDI    Bookman-DemiItalic
   BL     Bookman-Light
   BLI    Bookman-LightItalic
   C      Courier
   CO     Courier-Oblique
   CB     Courier-Bold
   CBO    Courier-BoldOblique
   H      Helvetica
   HB     Helvetica-Bold
   HO     Helvetica-Oblique
   HBO    Helvetica-BoldOblique
   HN     Helvetica-Narrow
   HNB    Helvetica-Narrow-Bold
   HNO    Helvetica-Narrow-Oblique
   HNBO   Helvetica-Narrow-BoldOblique
   NCSR   NewCenturySchlbk-Roman
   NCSB   NewCenturySchlbk-Bold
   NCSI   NewCenturySchlbk-Italic
   NCSBI  NewCenturySchlbk-BoldItalic
   PR     Palatino-Roman
   PB     Palatino-Bold
   PI     Palatino-Italic
   PBI    Palatino-BoldItalic
   TR     Times-Roman
   TB     Times-Bold
   TI     Times-Italic
   TBI    Times-BoldItalic
   ZCMI   ZapfChancery-MediumItalic
\end{verbatim}

\noindent
   The full font name (exactly as specified above) or the abbreviation given
   on the left may be used.

\item[\exe{-g}\code{\#}]
   Group lines of text in multiples, as specified, by printing grey bands in
   the background.  The default is not to produce banding.

\item[\exe{-i}\code{\#}]
   The distance in centimetres to indent all text from the left-hand edge of
   the page instead of the default 1\,cm.

\item[\exe{-l}]
   Use landscape format instead of the default portrait format.  The default
   font type, font size, and indentation allow 80 characters in portrait, and
   132 in landscape format.

\item[\exe{-n}\code{file}]
   Use \file{file} as the file name in the page header.

\item[\exe{-o}\code{M:N}]
   Only pages whose page numbers appear in the comma separated list of numbers
   and ranges will be printed.  A range \code{M:N} means pages \code{M} through
   \code{N} inclusive.  An initial \code{:N} means from the beginning to page
   \code{N}, and a final \code{M:} means from \code{M} to the end.  The
   default, ``\code{:}'', is to print all pages.

   Formfeeds are inserted to maintain the correct parity when printing in
   duplex mode (see the \exe{-x} option).

\item[\exe{-p}\code{\#}]
   Set the paper type.  The default is \code{A4} and recognized values are

\begin{verbatim}
   -----------------------------------------------------
     name          point        inch           cm
   -----------------------------------------------------
    A3          (842 x 1190)              29.7  x 42.0
    A4          (595 x  842)              21.0  x 29.7
    A5          (420 x  595)              14.82 x 21.0
    B4          (729 x 1032)              25.72 x 36.41
    B5          (516 x  729)              18.20 x 25.72
    statement   (396 x  612)  5.5 x 8.5  (13.97 x 21.59)
    executive   (540 x  720)  7.5 x 10   (19.05 x 25.40)
    quarto      (610 x  780)             (21.52 x 27.52)
    letter      (612 x  792)  8.5 x 11   (21.59 x 27.94)
    folio       (612 x  936)  8.5 x 13   (21.59 x 33.02)
    legal       (612 x 1008)  8.5 x 14   (21.59 x 35.56)
    10x14       (720 x 1008)  10  x 14   (25.40 x 35.56)
    tabloid     (792 x 1224)  11  x 17   (27.94 x 43.18)
    ledger     (1224 x  792)  17  x 11   (43.18 x 27.94)
\end{verbatim}

   \noindent
   In addition, \code{A4/letter} is recognized as being the width of \code{A4}
   and height of \code{letter} size paper.

\item[\exe{-r}]
   Reverse the page order and file sequence.  This may be useful for printers
   like the Apple LaserWriter which delivers pages face up, but is not
   appropriate for those which deliver them face down.

\item[\exe{-s}\code{\#}]
   Set the font size, in points.  The default is 10 and legal values lie in
   the range from 5 to 1000.

\item[\exe{-t}\code{\#}]
   Set the line spacing.  The default is 1.0 (single space) and legal values
   lie in the range from 1.0 to 3.0 (fractional values are allowed).

\item[\exe{-u}]
   Invert text on the reverse side of the page when printing in duplex mode.
   This may be used when the pages are to be bound on the upper edge of the
   paper (as defined by the orientation of the text on the front side).

\item[\exe{-x}]
   Exchange parity when printing in duplex mode so that the odd numbered pages
   are printed on the reverse side of the paper.

\end{description}

\subsection*{Notes}

\begin{itemize}
\item
   \exe{lwf\_a} produces \textsc{PostScript} which conforms to level 3.0
   document structuring conventions.

\item
   The default font and font size (Courier 10pt) allows 80 characters per line
   in portrait mode, and 132 characters in landscape on \code{A4} paper.
   However, the left margin is then too narrow to be punched for a ring
   binder.  If this is required, Courier 9pt with a 2\,cm margin may be used,
   specify \exe{-i}\code{2} \exe{-s}\code{9}.

\item
   The following table lists the number of rows and columns per page for a
   variety of input options.  It applies only for Courier with single spacing
   and the default indentation.

\begin{verbatim}
                A3                 A4               letter
   Font  Portrait Landsc.   Portrait Landsc.   Portrait Landsc.
   size  row/col  row/col   row/col  row/col   row/col  row/col

     5   220/266  150/389   150/184  101/266   140/190  104/250
     6   183/222  125/329   125/153   84/222   117/158   87/209
     7   157/190  107/274   107/131   72/190   100/135   74/179
     8   137/166   94/239    94/115   63/166    87/118   65/156
     9   122/148   83/213    83/102   56/148    78/105   58/139
    10   110/133   75/192    75/92    50/133    70/95    52/125
    11   100/121   68/174    68/83    46/121    63/86    47/114
    12    91/111   62/160    62/76    42/111    58/79    43/104
    15    73/89    50/128    50/61    33/89     46/63    34/83
    18    61/74    41/106    41/51    28/74     39/53    29/69
    24    45/55    31/80     31/38    21/55     29/39    21/51
\end{verbatim}

\item
   The \textsc{ascii} codes in hexadecimal, with two-character mnemonics for
   control codes, are as follows:

\begin{verbatim}
   00  NU  (NUL - null character)
   01  SH  (SOH - start of heading)
   02  SX  (STX - start of text)
   03  EX  (ETX - end of text)
   04  ET  (EOT - end of transmission)
   05  EQ  (ENQ - enquiry)
   06  AK  (ACK - acknowledge)
   07  BL  (BEL - bell)
   08  BS  (BS  - backspace)
   09  HT  (HT  - horizontal tab)
   0a  NL  (NL  - new line, or LF - line feed)
   0b  VT  (VT  - vertical tab)
   0c  NP  (NP  - new page, or FF - form feed)
   0d  CR  (CR  - carriage return)
   0e  SO  (SO  - shift out)
   0f  SI  (SI  - shift in)

   10  DL  (DLE - data link escape)
   11  D1  (DC1 - device control 1)
   12  D2  (DC1 - device control 2)
   13  D3  (DC1 - device control 3)
   14  D4  (DC1 - device control 4)
   15  NK  (NAK - negative acknowledge)
   16  SY  (SYN - synchonous idle)
   17  EB  (ETB - end of transmission block)
   18  CN  (CAN - cancel)
   19  EM  (EM  - end of medium)
   1a  SB  (SUB - substitute)
   1b  ES  (ESC - escape)
   1c  FS  (FS  - file separator)
   1d  GS  (GS  - group separator)
   1e  RS  (RS  - record separator)
   1f  US  (US  - unit separator)

   20 < >    30  0     40  @     50  P     60  `     70  p
   21  !     31  1     41  A     51  Q     61  a     71  q
   22  "     32  2     42  B     52  R     62  b     72  r
   23  #     33  3     43  C     53  S     63  c     73  s
   24  $     34  4     44  D     54  T     64  d     74  t
   25  %     35  5     45  E     55  U     65  e     75  u
   26  &     36  6     46  F     56  V     66  f     76  v
   27  '     37  7     47  G     57  W     67  g     77  w
   28  (     38  8     48  H     58  X     68  h     78  x
   29  )     39  9     49  I     59  Y     69  i     79  y
   2a  *     3a  :     4a  J     5a  Z     6a  j     7a  z
   2b  +     3b  ;     4b  K     5b  [     6b  k     7b  {
   2c  ,     3c  <     4c  L     5c  \     6c  l     7e  ~
   2d  -     3d  =     4d  M     5d  ]     7c  |     6d  m
   2e  .     3e  >     4e  N     5e  ^     7d  }     6e  n
   2f  /     3f  ?     4f  O     5f  _     6f  o

   7f DL (DEL - delete)
\end{verbatim}

   Eight-bit codes (from \code{80} to \code{ff}) are displayed via their two
   hexadecimal digits using ``\code{abcdef}'' (in lower case) to distinguish
   them from certain seven-bit control codes all of which are printed in upper
   case - in particular, the eight-bit codes (\code{d1}, \code{d2}, \code{d3},
   \code{d4}, and \code{eb}) and the seven-bit codes (\code{D1}, \code{D2},
   \code{D3}, \code{D4}, and \code{EB}).

   Do not confuse the hexadecimal code \code{ff} with ``formfeed'' which is
   displayed as \code{NP} ``newpage''.
\end{itemize}

\subsection*{Examples}

The command

\begin{verbatim}
   lwf_a -s12 -i2.5 file1
\end{verbatim}

\noindent
would convert \file{file1} to \textsc{PostScript}, indented 2.5\,cm, and in
12pt Courier font.

The command

\begin{verbatim}
   lwf_a -b -c2 -fPR -s6 foo.f
\end{verbatim}

\noindent
would convert \file{foo.f} to \textsc{PostScript} in two column form using a
6pt Palatino font.  A separate banner page would be printed.

The command

\begin{verbatim}
   man lwf_a | lwf_a
\end{verbatim}

\noindent
would convert the manual page for \exe{lwf\_a} to \textsc{PostScript}.

\subsection*{Bugs}

If you find any bugs please report them to \acct{mcalabre@atnf.csiro.au}.

\subsection*{See also}

The unix manual page for \unixexe{lpr}(1).\\
\exeref{pri}, print \textsc{ascii} or \textsc{PostScript} files.

\subsection*{Author}

Barry Brachman wrote the original version called \unixexe{lwf}.\\
Extensively modified by Mark Calabretta, 1989/May - 1994.

% ----------------------------------------------------------------------------

\newpage
\section{\exe{pra}}
\label{pra}
\index{pra@\exe{pra}}
\index{c++@\cplusplus!class printer|see{\exe{pra}}}
\index{print!c++ classes@\cplusplus\ classes|see{\exe{pra}}}

Print \aipspp\ class implementation files.

\subsection*{Synopsis}

\begin{synopsis}
   \code{\exe{pra} [\exe{-l}] [\exe{-P} printer] [class1 [class2 ...]]}
\end{synopsis}

\subsection*{Description}

\exe{pra} prints checked-in \aipspp\ class files in a compact form with the
header file followed by the class implementation file.

Unless the \exe{-l} option is specified, \exe{pra} only prints checked-in code
in the \file{\$AIPSROOT} directories, any code in the user's own workspace is
ignored.

If no \aipspp\ classes are specified, all classes in the current directory
will be printed.

If no printer is specified via the \exe{-P} option, the printer is that
returned by the \exeref{prd} command.

\subsection*{Options}

\begin{description}
\item[\exe{-l}]
   Print programmer class files from the current directory if they exist.

\item[\exe{-P}\code{printer}]
   Printer to use instead of the default (see \exeref{prd}).
\end{description}

\subsection*{Examples}

The command

\begin{verbatim}
   cd ~/casa/code/aips/implement/Tables
   pra Table
\end{verbatim}

\noindent
would print \file{Table.h}, and \file{Table.cc} in two-column landscape
format.

\subsection*{See also}

\aipspp\ variable names (\sref{variables}).\\
\exeref{lwfa}, \textsc{PostScript} filter.\\
\exeref{prd}, report the default printer.\\
\exeref{pri}, print \textsc{ascii} or \textsc{PostScript} files.\\
\exeref{prm}, delete entries from the printer queue.\\
\exeref{prq}, list contents of the default printer queue.

\subsection*{Author}

Original: 1992/03/04 by Mark Calabretta, ATNF.

% ----------------------------------------------------------------------------

\newpage
\section{\exe{prd}}
\label{prd}
\index{prd@\exe{prd}}
\index{print!queue!default|see{\exe{prd}}}

Report the default printer.

\subsection*{Synopsis}

\begin{synopsis}
   \exe{prd}
\end{synopsis}

\subsection*{Description}

\exe{prd} returns the name of a host's (or user's) default \aipspp\ printer
on \file{stdout}.  If the \code{PRINTER} environment variable is defined
\exe{prd} returns its value, otherwise it queries the \aipspp\ resource
database.

\subsection*{Options}

None.

\subsection*{Resources}

\begin{itemize}
\item
   \code{printer*default}: Default printer (no default).
\end{itemize}

\subsection*{Examples}

The command

\begin{verbatim}
   prd
\end{verbatim}

\noindent
reports the default \aipspp\ printer.

\subsection*{See also}

\exeref{getrc}, query \aipspp\ resource database.\\
\exeref{lwfa}, \textsc{PostScript} filter.\\
\exeref{pra}, print \aipspp\ class implementation files.\\
\exeref{pri}, print \textsc{ascii} or \textsc{PostScript} files.\\
\exeref{prm}, delete entries from the printer queue.\\
\exeref{prq}, list contents of the default printer queue.

\subsection*{Author}

Original: 1992/03/04 by Mark Calabretta, ATNF

% ----------------------------------------------------------------------------

\newpage
\section{\exe{pri}}
\label{pri}
\index{pri@\exe{pri}}
\index{print!ASCII file@\textsc{ascii} file|see{\exe{pri}}}
\index{print!PostScript file@\textsc{PostScript} file|see{\exe{pri}}}
\index{ASCII@\textsc{ascii}!print|see{print, \textsc{ascii}}}
\index{PostScript@\textsc{PostScript}!print|see{print, \textsc{PostScript}}}

Print \textsc{ascii} or \textsc{PostScript} files.

\subsection*{Synopsis}

\begin{synopsis}
   \code{\exe{pri} [\exe{-m} mode] [\exe{-p} paper] [\exe{-P} printer]
      [\exe{-t} type] [\exe{-w}] [file1 [file2 ...]]}
\end{synopsis}

\subsection*{Description}

\exe{pri} prints \textsc{ascii} or \textsc{PostScript} files on a
\textsc{PostScript} printer with a variety of formatting options.  If no files
are specified the input is taken from \file{stdin}.

If the file is in standard \textsc{PostScript} form in which the first line
begins with \code{\%!} it will be sent to the \textsc{PostScript} printer
without conversion.  Otherwise, it will be converted to \textsc{PostScript}
using the \exereff{lwf\_a}{lwfa} utility.  This behaviour can be explicitly
overridden with the \exe{-t} option.

If no printer is specified via the \exe{-P} option, the printer is that
returned by the \exeref{prd} command.

\subsection*{Options}

A space is allowed between the option letter and its value (if any).

\begin{description}
\item[\exe{-m}\code{mode}]
   Print mode for \textsc{ascii} files (doesn't apply to \textsc{PostScript}):

   \verb+p     +Portrait mode.\\
   \verb+l     +Landscape mode.\\
   \verb+72    +Two-column portrait mode for files with lines not exceeding 72
                characters in length.\\
   \verb+80    +Two-column landscape mode for files with lines not exceeding 80
                characters in length.

\item[\exe{-p}\code{paper}]
   Paper type (doesn't apply to \textsc{PostScript}):

   \verb+3 or A3       +Metric A3\\
   \verb+4 or A3       +Metric A4\\
   \verb+l or letter   +Letter

\item[\exe{-P}\code{printer}]
   Printer to use instead of the default (see \exeref{prd}).
   \exe{pri} recognizes a \code{ghostview} pseudo-printer which may be used to
   ``print'' the file to a \unixexe{ghostview} \textsc{PostScript} previewer
   on a workstation.

\item[\exe{-t}\code{type}]
   Input file type.

   \verb+txt     +\textsc{ascii} text file.\\
   \verb+ps      +\textsc{PostScript} file.

\item[\exe{-w}]
   Wait for the print queue to empty before submitting each job.  \exe{pri}
   checks the state of the queue once a minute, and will wait for up to one
   hour before submitting the job.
\end{description}

\noindent
All other options are passed to \exereff{lwf\_a}{lwfa}.

\subsection*{Resources}

\begin{itemize}
\item
   \code{printer*paper}: Paper type -

   \verb+A3      +metric \code{A3}\\
   \verb+A4      +metric \code{A4} (default)\\
   \verb+letter  +American letter size
\end{itemize}

\subsection*{Examples}

The command:

\begin{verbatim}
   pri -m 80 aips.h
\end{verbatim}

\noindent
would print \file{aips.h} in two-column landscape format.

\subsection*{See also}

\exeref{getrc}, query \aipspp\ resource database.\\
\exereff{lwf\_a}{lwfa}, \textsc{PostScript} filter.\\
\exeref{pra}, print \aipspp\ class implementation files.\\
\exeref{prd}, report the default printer.\\
\exeref{prm}, delete entries from the printer queue.\\
\exeref{prq}, list contents of the default printer queue.

\subsection*{Author}

Original: 1992/03/04 by Mark Calabretta, ATNF.

% ----------------------------------------------------------------------------

\newpage
\section{\exe{prm}}
\label{prm}
\index{prm@\exe{prm}}
\index{print!queue!delete entry|see{\exe{prd}}}

Delete entries from the printer queue.

\subsection*{Synopsis}

\begin{synopsis}
   \code{\exe{prm} [\exe{-P} printer] job1 [job2 ...]}
\end{synopsis}

\subsection*{Description}

\exe{prm} deletes entries from a printer queue.  If no printer is specified
via the \exe{-P} option, the printer is that returned by the \exeref{prd}
command.

\subsection*{Options}

\begin{description}
\item[\exe{-P}\code{printer}]
   Printer to use instead of the default.
\end{description}

\subsection*{Examples}

The commands

\begin{verbatim}
   prq
   prm 578
\end{verbatim}

\noindent
would list the contents of the default queue and then delete job number 578.

\subsection*{See also}

\exereff{lwf\_a}{lwfa}, \textsc{PostScript} filter.\\
\exeref{pra}, print \aipspp\ class implementation files.\\
\exeref{prd}, report the default printer.\\
\exeref{pri}, print \textsc{ascii} or \textsc{PostScript} files.\\
\exeref{prq}, list contents of the default printer queue.

\subsection*{Author}

Original: 1992/03/04 by Mark Calabretta, ATNF.

% ----------------------------------------------------------------------------

\newpage
\section{\exe{prq}}
\label{prq}
\index{prq@\exe{prq}}
\index{print!queue!list|see{\exe{prq}}}

List contents of the default printer queue.

\subsection*{Synopsis}

\begin{synopsis}
   \code{\exe{prq} [\exe{-P} printer]}
\end{synopsis}

\subsection*{Description}

\exe{prq} lists the contents of a printer queue.  If no printer is specified
via the \exe{-P} option, the printer is that returned by the \exeref{prd}
command.

\subsection*{Options}

\begin{description}
\item[\exe{-P}\code{printer}]
   Printer to use instead of the default.
\end{description}

\subsection*{Examples}

The command

\begin{verbatim}
   prq
\end{verbatim}

\noindent
lists the contents of the default printer queue.

\subsection*{See also}

\exereff{lwf\_a}{lwfa}, \textsc{PostScript} filter.\\
\exeref{pra}, print \aipspp\ class implementation files.\\
\exeref{prd}, report the default printer.\\
\exeref{pri}, print \textsc{ascii} or \textsc{PostScript} files.\\
\exeref{prm}, delete entries from the printer queue.

\subsection*{Author}

Original: 1992/03/04 by Mark Calabretta, ATNF.
