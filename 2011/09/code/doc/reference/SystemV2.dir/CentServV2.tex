\chapter{Central services}
\label{Central services}
\index{central services}
\index{services!central}
\index{administration}
\index{master host}

This chapter \footnote{Last change:
$ $Id$ $}
describes centrally provided \aipspp\ services.  These are associated with
the \acct{aips2adm} account on the master host, \host{aips2.nrao.edu} in
Socorro.  The relevant utilities are installed in \code{\$(MSTRETCD)} on the
master.

% ----------------------------------------------------------------------------
 
\section{CASA web services}
\label{web services}
\index{web!master services}
\index{master host}

This section describes web services provided by the \aipspp\ master host.

\subsection*{Master services}

In times past, there was a master host, \host{aips2.nrao.edu} that served
as a centralized server for all web/ftp services for AIPS++. This is no 
longer the case.
The \aipspp\ master host, \host{casa.nrao.edu}, is set up as the web server
for \aipspp\ at NRAO Socorro.  The master web server directly provides a
web pages for casa, \url{http://casa.nrao.edu}.

\noindent
The \aipspp\ master home page is at
\htmladdnormallink{http://casa.nrao.edu}{http://casa.nrao.edu}.

\subsection*{Server configuration}
\index{web!master server configuration}
\index{web!robot exclusion}

\host{casa.nrao.edu} runs Apache \unixexe{httpd}.  The essential features of
the server configuration are as follows:

\begin{itemize}
\item
   The content of server can be found by any AOC computer at \file{/home/casa.nrao.edu/content/...}.
The material is typically rsync'd from the latest stable build of the docs tree.

\item
   The System's group at NRAO-Socorro is responsible for all WEB managment. The CASA group is
responsible for content.
\end{itemize}

\noindent
The \aipspp\ links are enabled via unix symlinks in \code{DocumentRoot}:

\begin{verbatim}
   aips++/archive  -> /export/aips++/archive/
   aips++/ftp      -> /export/aips++/pub/
   aips++/mail     -> /export/aips++/Mail/
   aips++/master   -> /export/aips++/master/
   aips++/RELEASED -> ftp/RELEASED/
\end{verbatim}

\noindent
plus there are also links to the local \aipspp\ installations:

\begin{verbatim}
   aips++/daily  -> /aips++/daily/
   aips++/weekly -> /aips++/weekly/

   aips++/code   -> daily/code/
   aips++/docs   -> daily/docs/
\end{verbatim}

\noindent
The \aipspp\ links are presented in a \textsc{html} document
\file{/home/tarzan/index.html}.

\subsection*{See also}
 
Section \sref{email exploders}, \aipspp\ email exploders.\\
\href{http://hoohoo.ncsa.uiuc.edu}{, }{NCSA HTTPd}, etc.\\
\href{http://info.webcrawler.com/mak/projects/robots/robots.html}{, }
   {Web robots home page}.

% ----------------------------------------------------------------------------

\newpage
\section{CASA email exploders}
\label{email exploders}
\index{email!exploders}
\index{master host}
\index{electronic mail|see{email}}
\index{exploder|see{email}}
\index{reflector|see{exploder}}

CASA email exploders are managed via mailman in Charlottesville.
Management of the lists is done via the mailman admindb interface


Current lists are: 
\begin{itemize}
\item aips2-reports,
\item  aips2-glish,
\item  aips2,
\item  aips2-weekly-reports,
\item  aips2-naug,
\item  aips2-sitemgr,
\item aips2-aoc,
\item  aips2-inhale, and
\item  casa-framework.
\end{itemize}

To manage a particular list, tack on the list name to the following URL
\url{http://listmgr.cv.nrao.edu/mailmain/admindb/}
% ----------------------------------------------------------------------------

\newpage
\section{\exe{parseform}}
\label{parseform}
\index{parseform@\exe{parseform}}
\index{html@\textsc{html}!forms output decoder|see{\exe{parseform}}}

Decode \textsc{html} forms output.

\subsection*{Synopsis}
 
\begin{synopsis}
   \code{\exe{eval `parseform`}}
\end{synopsis}
 
\subsection*{Description}
 
\exe{parseform} decodes the output from an \textsc{html} form reporting it on
\file{stdout} in a format suitable for \unixexe{eval}'ing into the environment
in a Bourne shell \textsc{cgi} script.  Both the \code{GET} and \code{POST}
methods of \textsc{html} form output are supported.
 
\subsection*{Options}
 
None.
 
\subsection*{Notes}
 
\begin{itemize}
\item
   \exe{parseform} only works when called from a \textsc{cgi} script.  Note
   that the form output generated by the \code{POST} method appears on
   \file{stdin} and \exe{parseform} consumes this.  Form output generated by
   the \code{GET} method is obtained from environment variable
   \code{QUERY\_STRING}.

\item
   \exe{parseform} is implemented as a \unixexe{perl} script.
\end{itemize}
 
\subsection*{Diagnostics}
 
Status return values
\\ \verb+   0+: success

\subsection*{Examples}

\exe{scanpick} uses the following to get and parse the form output generated
by \exe{pickhtml}:

\begin{verbatim}
   # Get the form output.
     eval `$MSTRETCD/parseform`
\end{verbatim}
 
\subsection*{See also}
 
Section \sref{email exploders}, \aipspp\ email exploders.\\
\exeref{pickhtml}, \aipspp\ \textsc{html} form generator for \exe{scanpick}.\\
\exeref{scanpick}, \aipspp\ \textsc{html} interface to \unixexe{pick}.\\
\href{http://hoohoo.ncsa.uiuc.edu/cgi/forms.html}{, }
   {Decoding FORMs with CGI}.
 
\subsection*{Author}
 
Original: 1996/05/21 by Mark Calabretta, ATNF

% ----------------------------------------------------------------------------

\newpage
\section{\exe{pickhtml}}
\label{pickhtml}
\index{pickhtml@\exe{pickhtml}}
\index{email!web interface!search form|see{\exe{pickhtml}}}

\textsc{html} forms based interface to search an \aipspp\ email archive.

\subsection*{Synopsis}

\begin{synopsis}
   \code{\exe{pickhtml} [folder]}
\end{synopsis}

\subsection*{Description}

\exe{pickhtml} is a \textsc{cgi} script which generates an \textsc{html} form
to obtain options for the \textsc{mh} \unixexe{pick} command to search an
\aipspp\ email archive folder.  These options are passed on to
\exeref{scanpick} via the \code{POST} method.

\exeref{scanhtml} creates a link to \exe{pickhtml} in the \textsc{html} index
to each \aipspp\ email archive folder.

If not supplied the folder name defaults to \code{general}.

\subsection*{Options}

None.

\subsection*{Notes}

\begin{itemize}
\item
   \exe{pickhtml} is web-enabled via a symlink in
   \file{/home/tarzan/httpd/cgi-bin}, the \unixexe{httpd} \code{ScriptAlias}
   directory:

\begin{verbatim}
 /home/tarzan/httpd/cgi-bin/pickhtml -> /export/aips++/master/etc/pickhtml
\end{verbatim}
\end{itemize}

\subsection*{Diagnostics}

Status return values
\\ \verb+   0+: success

\subsection*{See also}

The manual page for \unixexe{mh}(1), the message handler system.\\
The manual page for \unixexe{pick}(1), the \textsc{mh} search command.\\
Section \sref{email exploders}, \aipspp\ email exploders.\\
\exeref{scanpick}, \aipspp\ \textsc{html} interface to \unixexe{pick}.

\subsection*{Author}

Original: 1996/05/21 by Mark Calabretta, ATNF

% ----------------------------------------------------------------------------

\newpage
\section{\exe{reap}}
\label{reap}
\index{reap@\exe{reap}}
\index{reports, programmer!collate and disseminate|see{\exe{reap}}}

Collate and disseminate \aipspp\ weekly reports.

\subsection*{Synopsis}
 
\begin{synopsis}
   \code{\exe{reap}}
\end{synopsis}
 
\subsection*{Description}
 
\exe{reap} collates and disseminates \aipspp\ reports.  It maintains a
timestamp file, \file{.reap.time}, within the \mbox{\acct{aips2-reports}} mail
folder, collects all reports newer than the timestamp, strips out the mail
headers, concatenates them and posts the result to the
\mbox{\acct{aips2-weekly-reports}} exploder.  It is invoked regularly by an
\acct{aips2adm} \unixexe{cron} job running on \host{aips2.nrao.edu}.

\subsection*{Options}
 
None.

\subsection*{Diagnostics}
 
Status return values
\\ \verb+   0+: success
\\ \verb+   1+: initialization error

\subsection*{See also}

The unix manual page for \unixexe{cron}(1).\\
Section \sref{Accounts and groups}, \aipspp\ accounts and groups.\\
Section \sref{email exploders}, \aipspp\ email exploders.\\
\file{report\_form}, \aipspp\ weekly report form.

\subsection*{Author}
 
Original: 1996/05/05 by Mark Calabretta, ATNF

% ----------------------------------------------------------------------------

\newpage
\section{\file{report\_form}}
\label{report_form}
\index{report\_form@\file{report\_form}}
\index{reports, programmer!form|see{\file{report\_form}}}

\aipspp\ weekly report form.

\subsection*{Synopsis}
 
\begin{synopsis}
   \code{\exe{/usr/lib/sendmail} -t < /export/aips++/master/etc/report\_form}
\end{synopsis}
 
\subsection*{Description}

\file{report\_form} is sent to the \mbox{\acct{aips2-weekly-reports}} email
exploder each week by an \acct{aips2adm} \unixexe{cron} job running on
\host{aips2.nrao.edu}.  It solicits a report which is to be sent to the
\mbox{\acct{aips2-reports}} exploder where they are collected by \exeref{reap}
which concatenates and mails them back to \mbox{\acct{aips2-weekly-reports}}.

\subsection*{Notes}
 
\begin{itemize}
\item
   The \file{report\_form} is fed directly to \unixexe{sendmail}.
\end{itemize}

\subsection*{See also}
 
The unix manual page for \unixexe{cron}(1).\\
Section \sref{Accounts and groups}, \aipspp\ accounts and groups.\\
Section \sref{email exploders}, \aipspp\ email exploders.\\
\exeref{reap}, \aipspp\ report collator.
 
\subsection*{Author}
 
Original: Jim Horstkotte, NRAO
 
% ----------------------------------------------------------------------------

\newpage
\section{\exe{scanhtml}}
\label{scanhtml}
\index{scanhtml@\exe{scanhtml}}
\index{email!web interface!index|see{\exe{scanhtml}}}

Produce an \textsc{html} index of an \aipspp\ mail folder.

\subsection*{Synopsis}

\begin{synopsis}
   \code{\exe{scanhtml} [folder]}
\end{synopsis}

\subsection*{Description}

\exe{scanhtml} produces an \textsc{mh} \unixexe{scan} listing of an \aipspp\
exploder mail archive folder and converts it to an \textsc{html} index with
links to the individual messages.  The index, \file{index.html}, is deposited
in the folder directory.  Any pre-existing index is overwritten.

As described in \sref{email exploders} the index generated by \exe{scanhtml}
also contains links to the currently installed copies of the exploder lists,
and a link to a form which allows searching of the exploder archive.

If not supplied the folder name defaults to \code{general}.

\subsection*{Options}

None.

\subsection*{Notes}

\begin{itemize}
\item
   \exe{scanhtml} is invoked by \exeref{aipsmail} immediately after new
   exploder mail is archived.  It is also used by the \acct{aips2adm}
   \unixexe{cron} job which deletes old mail from the
   \mbox{\acct{aips2-inhale}} archive.  It can be invoked manually (by
   \acct{aips2adm}) at any time if required.
\end{itemize}

\subsection*{Diagnostics}

Status return values
\\ \verb+   0+: success

\subsection*{See also}

The unix manual page for \unixexe{cron}(1).\\
The manual page for \unixexe{mh}(1), the message handler system.\\
The manual page for \unixexe{scan}(1), the \textsc{mh} index command.\\
Section \sref{Accounts and groups}, \aipspp\ accounts and groups.\\
Section \sref{email exploders}, \aipspp\ email exploders.

\subsection*{Author}

Original: 1995/07/18 by Mark Calabretta, ATNF

% ----------------------------------------------------------------------------

\newpage
\section{\exe{scanpick}}
\label{scanpick}
\index{scanpick@\exe{scanpick}}
\index{email!web interface!search results|see{\exe{scanpick}}}

Produce an \textsc{html} index of selected mail in an \aipspp\ folder.

\subsection*{Synopsis}

\begin{synopsis}
   \code{\exe{scanpick} [folder]}
\end{synopsis}

\subsection*{Description}

\exe{scanpick} is a \textsc{cgi} script which produces a \unixexe{scan}
listing of messages in an \aipspp\ \textsc{mh} email archive folder selected
according to \unixexe{pick} arguments.

The \unixexe{pick} arguments are acquired by another \textsc{cgi} script,
\exeref{pickhtml}, which generates an \textsc{html} form and passes the result
to \exe{scanpick} via the \code{POST} method.

\exe{scanpick} uses an auxiliary \unixexe{perl} script called
\exeref{parseform} to decode the form output.

\subsection*{Options}

None.

\subsection*{Notes}

\begin{itemize}
\item
   \exe{scanpick} is web-enabled via a symlink in
   \file{/home/tarzan/httpd/cgi-bin}, the \unixexe{httpd} \code{ScriptAlias}
   directory:

\begin{verbatim}
 /home/tarzan/httpd/cgi-bin/scanpick -> /export/aips++/master/etc/scanpick
\end{verbatim}
\end{itemize}

\subsection*{Diagnostics}

Status return values
\\ \verb+   0+: success

\subsection*{See also}

The manual page for \unixexe{mh}(1), the message handler system.\\
The manual page for \unixexe{pick}(1), the \textsc{mh} search command.\\
The manual page for \unixexe{scan}(1), the \textsc{mh} index command.\\
Section \sref{email exploders}, \aipspp\ email exploders.\\
\exeref{pickhtml}, \aipspp\ \textsc{html} form generator for \exe{scanpick}.\\
\exeref{parseform}, \aipspp\ decoder for \textsc{html} forms output.

\subsection*{Author}

Original: 1996/05/21 by Mark Calabretta, ATNF
