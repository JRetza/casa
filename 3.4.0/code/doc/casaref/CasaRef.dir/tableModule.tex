%% Copyright (C) 1999,2000,2001,2002,2003
%% Associated Universities, Inc. Washington DC, USA.
%%
%% This library is free software; you can redistribute it and/or modify it
%% under the terms of the GNU Library General Public License as published by
%% the Free Software Foundation; either version 2 of the License, or (at your
%% option) any later version.
%%
%% This library is distributed in the hope that it will be useful, but WITHOUT
%% ANY WARRANTY; without even the implied warranty of MERCHANTABILITY or
%% FITNESS FOR A PARTICULAR PURPOSE.  See the GNU Library General Public
%% License for more details.
%%
%% You should have received a copy of the GNU Library General Public License
%% along with this library; if not, write to the Free Software Foundation,
%% Inc., 675 Massachusetts Ave, Cambridge, MA 02139, USA.
%%
%% Correspondence concerning AIPS++ should be addressed as follows:
%%        Internet email: aips2-request@nrao.edu.
%%        Postal address: AIPS++ Project Office
%%                        National Radio Astronomy Observatory
%%                        520 Edgemont Road
%%                        Charlottesville, VA 22903-2475 USA
%%
%% $Id$
\begin{ahmodule}{table}{Casapy interface to table system}

%%\ahinclude{table.g}

\begin{ahdescription}

\casa\ stores all data inside \casa\ tables which 
are stored on disk.
A \casa\ table consists of an unlimited number of columns of data,
with optional column keywords and optional table keywords. Columns are
named and rows are numbered (starting at 0).  The columns hold data,
such as visibilities and uv coordinates, while the keywords hold
general information such as units or revision numbers or table author
or even other tables.

To make this concrete, examples of columns might be:

\begin{verbatim}
  U       V        W         TIME        ANT1   ANT2      VISIBILITY
124.011 54560.0  3477.1  43456789.0990    1      2        4.327 -0.1132
34561.0 45629.3  3900.5  43456789.0990    1      3        5.398 0.4521
....
....
....
\end{verbatim}

and examples of keywords might be:

\begin{verbatim}
REVISION=2.01
AUTHOR="Tim Cornwell"
INSTRUMENT="VLA"
\end{verbatim}

Everything in a \casa\ table (and thus all data stored in \casa) is
potentially accessible and changable from casapy. The table module
provides a convenient way of accessing and changing \casa\ tables from
inside casapy. To do so one uses {\sl tb}, the table tool, inside
casapy. The table tool provides functions for:

\begin{itemize}
\item Opening and copying of existing tables, (using
\ahlink{open}{table:table.open}, and
\ahlink{copy}{table:table.copy}),
\item Get and put of table cells, columns and keywords,
\item Selection and sorting with 
 \htmladdnormallink{TaQL}{../../notes/199/199.html} (using the
 \ahlink{query}{table:table.query} or
 \ahlink{calc}{table:table.calc} functions),
%% or
%% \ahlink{tablecommand}{table:tablecommand}),
%%\item Iteration by subtables (using  \ahlink{tableiterator}{table:tableiterator}),
\item Get and put of table information strings,
%%\item Access via table rows (using \ahlink{tablerow}{table:tablerow}),
%%\item Indexed access (using \ahlink{tableindex}{table:tableindex}),
\item Browsing of tables (using the \ahlink{browse}{table:table.browse} function),
\item Printing of a summary of a table (using \ahlink{summary}{table:table.summary}
function),
%%\item Determine if a casapy value is a valid table (using \ahlink{is\_table}{table:is_table}),
\item Reading or writing a table from or to an ASCII format (using
\ahlink{fromascii}{table:table.fromascii}, and 
\ahlink{toasciifmt}{table:table.toasciifmt})
%\item Reading a table from binary FITS format (using
%\ahlink{tablefromfits}{table:table.tablefromfits.constructor})
\end{itemize}

%%There is no \texttt{tofits} function available. It might get available 
%%in the future.

%%In addition this module contains a number of global functions
%%related to tables, such as determining if a table exists 
%%(\ahlink{tableexists}{table:tableexists}).

All operations are done inside the casapy client and are not written
to disk until an explicit flush command is performed. 

The most typical operation on a \casa\ table is to open it,
%% by creating a table tool inside casapy,
load a column from the table
into casapy, alter it using casapy capabilities, and then write it
back to the table. For this only a few commands are relevant: see the
example below.

Sorting and selecting of tables is possible. The table thus produced
is a {\em reference} table, and points back to the original table.

%%The casapy client handling all table events and operations is the
%%\texttt{tableserver}. At startup of \texttt{table.g}
%%the \texttt{defaulttableserver} is
%%started which runs until the casapy session ends. This server is by
%%default used. It is, however, possible to start another one like:
%%\begin{verbatim}
%%  mytableserver:=tableserver();
%%\end{verbatim}
%%and use that in the various constructors and functions taking a 
%%\texttt{tableserver} argument.
%%In a similar way the \texttt{defaulttableserver} can be restarted
%%in case it crashes in one way or another.
\end{ahdescription}

\begin{ahexample}
\begin{verbatim}
  tb.open('3C273XC1.MS')
  tb.summary();
  uvw=tb.getcol("UVW");
  tb.close()
  tb.open("3C273XC1.MS/SPECTRAL_WINDOW")
  freq=tb.getcell("REF_FREQUENCY", 0)
  tb.close()
  for i in range(len(uvw)):
    for j in range(len(uvw[0])):
       uvw[i][j] = uvw[i][j]*1.420E9/freq
  tb.open('3C273XC1.MS', nomodify=False)
  tb.putcol("UVW", uvw);
  tb.close()
\end{verbatim}
\end{ahexample}
\begin{ahseealso}
%%\item[\ahlink{tablerow}{table:tablerow}]
%%\item[\ahlink{tableiterator}{table:tableiterator}]
%%\item[\ahlink{tableindex}{table:tableindex}]
\item[\ahlink{tableplot}{table:tableplot}]
\end{ahseealso}

%%\begin{ahaipsrc}
%%\ahaddarg{table.relinquish.reqautolocks.interval}{nr of seconds to wait before relinquishing autolocks requested in another process}{5}{float}
%%\ahaddarg{table.relinquish.allautolocks.interval}{nr of seconds to
%%wait before relinquishing all autolocks}{60}{float}
%%\ahaddarg{table.endianformat}{endian format to be used for storing data in new tables}{big}{big,little,local}
%%\end{ahaipsrc}

\ahobjs{}
\ahfuncs{}
\input{table.htex}
%%%\input{tableindex.htex}
%%\input{tableiterator.htex}
%%\input{tablerow.htex}
\input{tableplot.htex}
\end{ahmodule}
