%% Copyright (C) 1999,2000,2001,2002,2003
%% Associated Universities, Inc. Washington DC, USA.
%%
%% This library is free software; you can redistribute it and/or modify it
%% under the terms of the GNU Library General Public License as published by
%% the Free Software Foundation; either version 2 of the License, or (at your
%% option) any later version.
%%
%% This library is distributed in the hope that it will be useful, but WITHOUT
%% ANY WARRANTY; without even the implied warranty of MERCHANTABILITY or
%% FITNESS FOR A PARTICULAR PURPOSE.  See the GNU Library General Public
%% License for more details.
%%
%% You should have received a copy of the GNU Library General Public License
%% along with this library; if not, write to the Free Software Foundation,
%% Inc., 675 Massachusetts Ave, Cambridge, MA 02139, USA.
%%
%% Correspondence concerning AIPS++ should be addressed as follows:
%%        Internet email: aips2-request@nrao.edu.
%%        Postal address: AIPS++ Project Office
%%                        National Radio Astronomy Observatory
%%                        520 Edgemont Road
%%                        Charlottesville, VA 22903-2475 USA
%%
%% $Id$
\begin{ahmodule}{calibrater}{Module for synthesis calibration}
\ahinclude{calibrater.g}

\begin{ahdescription}

The {\tt calibrater} module provides synthesis calibration
capabilities within \aipspp. The primary purpose of this module is to
solve for calibration components, and to optionally apply these
corrections to the observed data. The calibration module is designed
to be used in conjunction with the \ahlink{imager}{imager} module
which provides support for synthesis imaging.

The calibration model adopted by {\tt calibrater} is that of the
Hamaker-Bregman-Sault measurement equation for synthesis radio
telescopes (see \htmladdnormallink{\aipspp Note
189}{../../notes/189/189.html}). This represents calibration corrections as
matrices acting on 4-vectors representing the four possible
correlations measured by an interferometer in full polarization. The
calibration matrices cover a diversity of instrumental effects,
including: parallactic angle and feed configuration (P,C), atmospheric
phase (T), electronic gain (G), bandpass (B), instrumental
polarization (D), baseline-based (correlator) corrections (M, MF), 
and baseline-based fringe-fitting (K).

The calibration data are stored in \aipspp tables and can be directly
examined, manipulated or edited in the Glish command line interpreter
(CLI) via the \ahlink{table}{table} tool. The calibration tables may be
interpolated when applied to the observed uv-data.

The solver allows the calibration components to be determined over
different time intervals, thus allowing, as an example, the solution
for atmospheric phase effects (T) over a much shorter interval than
electronic gain terms (G). This also allows polarization
self-calibration for time variable instrumental polarization
corrections.

The measurement equation is designed to model calibration effects for
a generic radio telescope and the calibration and synthesis modules
are, in general, not instrument specific.

Each {\tt calibrater} tool created acts on a specified Measurement Set
(MS), containing the observed uv-data. The Measurement Set format is
described in (see \htmladdnormallink{\aipspp Note
191}{../../notes/191/191.html}). The interaction of the {\tt calibrater}
tool with specific MS data columns is important. The observed data, as
recorded by the instrument, are stored in the {\bf DATA} column of the
MS, and are referred to as the observed data. If calibration
corrections are applied by {\tt calibrater}, the resulting calibrated
data are stored in a separate {\bf CORRECTED\_DATA} column in the
MS. These columns can be selected when imaging the data using the
\ahlink{imager}{imager} tool. A further MS column is used by {\tt
calibrater}, namely the {\bf MODEL\_DATA} column. The difference
between the model data and corrected data columns is used to form
$\chi^2$, when solving for individual calibration components. It is
important to set the {\bf MODEL\_DATA} column before using {\tt
calibrater} to solve for calibration. This can be done using the {\tt
imager} functions \ahlink{setjy}{imager:imager.setjy} or
\ahlink{ft}{imager:imager.ft}.

The capabilities of the {\tt calibrater} module are made available by
including the associated Glish initialization script for the module, as:

\begin{verbatim}
- include 'calibrater.g'
T
\end{verbatim}

where a hyphen precedes user input. The Glish response is indicated
without the prompt.

A {\tt calibrater} tool is created and attached to a specified
measurement set as indicated in the following example:

\begin{verbatim}
- c:=calibrater('3C273XC1.MS');
T
\end{verbatim}

A variety of functions can be invoked for any given {\tt calibrater}
tool. These functions fall broadly into two categories: i) functions
which set parameters to be used by the calibrater; and ii) the
execution of explicit calibration procedures such as solving for, or
applying calibration corrections.

Option (i) may equivalently be viewed as setting the state of the {\tt
calibrater} tool. These functions are named with the prefix {\tt
set}, such as in {\tt setdata} and {\tt setapply}. When created, the
{\tt calibrater} tool sets default internal information for each of
the calibration components (measurement equation correction
matrices). This information is modified using the {\tt setapply}
and {\tt setsolve} functions as shown in the following example:

\begin{verbatim}
#
# Set the solution interval for the electronic gain matrix (G) to
# 300 seconds, specify input and output calibration table names,
# and enable this component for phase and amplitude solution.
# Use antenna number 3 as the reference for the solutions.
#
- c.setapply ("G", 0.0, "gcal_in", "");
T
- c.setsolve ("G", 300, F, 3, "gcal_out", F);
T
\end{verbatim}

Once the state of the {\tt calibrater} tool has been set, explicit
calibration functions, as outlined in option (ii) above, are executed as
follows:

\begin{verbatim}
#
# Solve for the selected calibration components
#
- c.solve()
T
#
# Apply the calibration components to the measurement set data
#
- c.correct()
T
\end{verbatim}
\end{ahdescription}

\begin{ahexample}

The following example illustrates the quickest way to perform simple
self-calibration, starting from an input FITS file in the local
area. The {\tt imager} module should be consulted for detailed
information on the imaging functions.

\begin{verbatim}
#
# Include the synthesis scripts
#
include 'imager.g';
include 'calibrater.g';
include 'ms.g';
#
# Construct a measurement set from the input FITS file
#
m:=fitstoms (msfile='3C273XC1.MS', fitsfile='3C273XC1.FITS');
m.close(); 
m.done();
#
# Create an imager tool
#
sk:=imager('3C273XC1.MS');
#
# Set image parameters
#
sk.setimage (nx=256, ny=256, cellx='0.7arcsec', celly='0.7arcsec');
#
# Make a dirty image and deconvolve using CLEAN
#
sk.image ('observed', image='3C273XC1.dirty');
sk.clean (niter=1000, threshold='3mJy', model='3C273XC1.clean.model');
#
# Fourier transform the model to the uv-plane
#
sk.ft (model='3C273XC1.clean.model');
#
# Close the imager tool
#
sk.close();
sk.done();
#
# Create a calibrater tool
#
c:=calibrater('3C273XC1.MS');
#
# Select solution for electronic gain (G) and atmospheric phase (T)
#
c.setsolve ("G", 300.0, F, 3, "gcal_out", F);
c.setsolve ("T", 30.0, T, 3, "tcal_out", F);
#
# Solve for the selected G and T components
#
c.solve();
#
# Close the calibrater tool
#
c.close();
\end{verbatim}
\end{ahexample}

%\ahobjs{}
%\ahfuncs{}

\input{calibrater.htex}
                                                                                              
\end{ahmodule}
