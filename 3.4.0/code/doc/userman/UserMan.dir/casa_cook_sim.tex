% STM 2007-04-13  split from previous version
% STM 2007-10-11  add into beta
% STM 2007-10-22  put in appendix for Beta Release
% RI  2009-12-17  Release 0 (3.0.0) draft
% STM 2009-12-21  Release 0 (3.0.0) final
% RI 2010-04-10 Release 3.0.1

%\chapter{Simulation}
\chapter[Simulation]{Simulation}
\label{chapter:sim}

{\bfseries New in 3.4:} 
Representative configurations for ALMA and ACA Cycle 1.

{\bfseries Important note on task names:}
Users are encouraged to use {\tt simobserve} and {\tt simanalyze}.  The combined task {\tt simdata} is still present, but will be removed in the future.

The capability for simulating observations and datasets from the JVLA
and ALMA are an important use-case for CASA.  This not only allows one
to get an idea of the capabilities of these instruments for doing
science, but also provides benchmarks for the performance and utility
of the software for processing ``realistic'' datasets (with
atmospheric and instrumental effects).  CASA can calculate
visibilities (create a measurement set) for any interferometric array,
and calculate and apply calibration tables representing some of the
most important corrupting effects. {\tt simobserve} can also simulate
total power observations, which can be combined with interferometric
data in {\tt simanalyze} (i.e. one would run {\tt simobserve} twice,
{\tt simanalyze} once.)



\begin{wrapfigure}{r}{2.5in}
 \begin{boxedminipage}{2.5in}
    \centerline{\bf Inside the Toolkit:}
    The simulator methods are in the {\tt sm} tool.
    Many of the other tools are also helpful when
    constructing and analyzing simulations.
 \end{boxedminipage}
\end{wrapfigure}

CASA's simulation capabilities continue to be improved with each CASA release.
For the most current information, please refer to
\url{http://www.casaguides.nrao.edu}, and click on ``Simulating
Observations in CASA''.
%
Following general CASA practice, the greatest flexibility and richest
functionality is at the Toolkit level.  The most commonly used
procedures for interferometric and single dish simulation are
encapsulated in the {\tt simobserve} task.

%%%%%%%%%%%%%%%%%%%%%%%%%%%%%%%%%%%%%%%%%%%%%%%%%%%%%%%%%%%%%%%%%
%%%%%%%%%%%%%%%%%%%%%%%%%%%%%%%%%%%%%%%%%%%%%%%%%%%%%%%%%%%%%%%%%
\section{Simulating ALMA with {\tt simobserve} and {\tt simanalyze}}`
\label{section:sim.almasimmos}

The {\tt simobserve} inputs are (submenus expand slightly differently for thermalnoise=manual and single dish observing):
\small
\begin{verbatim}
project             =      'sim'        #  root prefix for output file names
skymodel            =         ''        #  model image to observe
     inbright       =         ''        #  scale surface brightness of brightest pixel e.g. "1.2Jy/pixel"
     indirection    =         ''        #  set new direction e.g. "J2000 19h00m00 -40d00m00"
     incell         =         ''        #  set new cell/pixel size e.g. "0.1arcsec"
     incenter       =         ''        #  set new frequency of center channel e.g. "89GHz" (required even for 2D model)
     inwidth        =         ''        #  set new channel width e.g. "10MHz" (required even for 2D model)

complist            =         ''        #  componentlist to observe
     compwidth      =     '8GHz'        #  bandwidth of components

setpointings        =       True        
     integration    =      '10s'        #  integration (sampling) time
     direction      =         ''        #  "J2000 19h00m00 -40d00m00" or "" to center on model
     mapsize        =   ['', '']        #  angular size of map or "" to cover model
     maptype        =     'ALMA'        #  hexagonal, square, etc
     pointingspacing =         ''       #  spacing in between pointings or "0.25PB" or "" for 0.5 PB

obsmode             =      'int'        #  observation mode to simulate
                                        #   [int(interferometer)|sd(singledish)|""(none)]
     antennalist    = 'alma.out10.cfg'  #  interferometer antenna position file
     refdate        = '2012/05/21'      #  date of observation - not critical unless concatting
                                        #   simulations
     hourangle      =  'transit'        #  hour angle of observation center e.g. -3:00:00, or "transit"
     totaltime      =    '7200s'        #  total time of observation or number of repetitions
     caldirection   =         ''        #  pt source calibrator [experimental]
     calflux        =      '1Jy'        

thermalnoise        = 'tsys-atm'        #  add thermal noise: [tsys-atm|tsys-manual|""]
     user_pwv       =        1.0        #  Precipitable Water Vapor in mm
     t_ground       =      269.0        #  ambient temperature
     seed           =      11111        #  random number seed

leakage             =        0.0        #  cross polarization (interferometer only)
graphics            =     'both'        #  display graphics at each stage to [screen|file|both|none]
verbose             =      False        
overwrite           =       True        #  overwrite files starting with $project
async               =      False        #  If true the taskname must be started using simobserve(...)
\end{verbatim}
\normalsize

This task takes an input model image or list of components, plus a
list of antennas (locations and sizes), and simulates a particular
observation (specifies by mosaic setup and observing cycles and
times).  The output is a measurement set suitable for further analysis in CASA.

The {\tt simanalyze} inputs are:
\small
\begin{verbatim}
project             =      'sim'        #  root prefix for output file names
image               =       True        #  (re)image $project.*.ms to $project.image
     vis            =  'default'        #  Measurement Set(s) to image
     modelimage     =         ''        #  prior image to use in clean e.g. existing single dish image
     imsize         =          0        #  output image size in pixels (x,y) or 0 to match model
     imdirection    =         ''        #  set output image direction, (otherwise center on the model)
     cell           =         ''        #  cell size with units or "" to equal model
     niter          =        500        #  maximum number of iterations (0 for dirty image)
     threshold      =   '0.1mJy'        #  flux level (+units) to stop cleaning
     weighting      =  'natural'        #  weighting to apply to visibilities
     mask           =         []        #  Cleanbox(es), mask image(s), region(s), or a level
     outertaper     =         []        #  uv-taper on outer baselines in uv-plane
     stokes         =        'I'        #  Stokes params to image

analyze             =       True        #  (only first 6 selected outputs will be displayed)
     showuv         =       True        #  display uv coverage
     showpsf        =       True        #  display synthesized (dirty) beam (ignored in single dish simulation)
     showmodel      =       True        #  display sky model at original resolution
     showconvolved  =      False        #  display sky model convolved with output beam
     showclean      =       True        #  display the synthesized image
     showresidual   =      False        #  display the clean residual image (ignored in single dish simulation)
     showdifference =       True        #  display difference image
     showfidelity   =       True        #  display fidelity

graphics            =     'both'        #  display graphics at each stage to [screen|file|both|none]
verbose             =      False        
overwrite           =       True        #  overwrite files starting with $project
async               =      False        #  If true the taskname must be started using simanalyze(...)
\end{verbatim}
\normalsize

This task analyzses one more more measurement sets - interferometric and/or single dish.
The output is a synthesized image created from those visibilities, a difference image
between the synthesized image and your sky model convolved with the
output synthesized beam, and a fidelity image. (see ALMA memo 398 for
description of fidelity, which is approximately the output image
divided by the difference between input and output)

The combined task {\tt simdata} is modular: one can
modify one's sky model, predict visibilities, corrupt the Measurement
Set, re-image, and analyze the result all separately, provided in a
few cases the filenames are set correctly.  
%%%%%%%%%%%%%%%%%%%%%%%%%%%%%%%%%%%%%%%%%%%%%%%%%%%%%%%%%%%%%%%%%
