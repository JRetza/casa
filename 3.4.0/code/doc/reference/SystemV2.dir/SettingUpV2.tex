\chapter{Setting Up}
\label{Setting Up}

This chapter \footnote{Last change:
$ $Id$ $}
describes how to set up your \aipspp\ environment.

% ----------------------------------------------------------------------------

\section{Setting up your account to use CASA}
\label{Setup}
\index{setting up!user}

\subsection{General CASA setup}

The \aipspp\ environment is defined via a once-only modification to your
shell's startup script.

Assuming that \aipspp\ has been installed under \file{/home/casa}, users of
Bourne-like shells (\unixexe{sh}, \unixexe{ksh}, \unixexe{bash}) must add the
following to their \file{.profile} file at a point {\em after} \code{PATH}
(and \code{MANPATH}) are defined:

\begin{verbatim}
   # Get the CASA environment.
     [ -f /home/casa/casainit.sh ] && . /home/casa/casainit.sh
\end{verbatim}

\noindent
The equivalent entry in \file{.login} for C-like shells (\unixexe{csh},
\unixexe{tcsh}) is:

\begin{verbatim}
   # Get the CASA environment.
     if (-f /home/casa/casainit.csh) source /home/casa/casainit.csh
\end{verbatim}

\noindent
Users of the \unixexe{rc} shell might use

\begin{verbatim}
   # Get the CASA environment.
     if (test -f /home/casa/casainit.rc) . /home/casa/casainit.rc
\end{verbatim}

\noindent
in \file{.rcrc}, or in \file{.esrc} for the \unixexe{es} shell:

\begin{verbatim}
   # Get the CASA environment.
     if {test -f /home/casa/casainit.es} {. /home/casa/casainit.es}
\end{verbatim}

The \exeref{casainit} scripts define a single environment variable called
\code{CASAPATH} (\sref{variables}) and add the \aipspp\ \file{bin} area to the
\code{PATH} environment variable and the \aipspp\ \file{man} directory to the
\code{MANPATH} environment variable.

For more detailed information, see \exeref{casainit}.  This contains an
explanation of \code{CASAPATH} and a mechanism for controlling the point where
the \aipspp\ \file{bin} areas are added to \code{PATH}.

\subsection{CASA programmer setup}
\label{CASA programmer setup}
\index{code!development!setup}
\index{setting up!programmer|see{code, development}}

Normally, \aipspp\ programmers must belong to the \aipspp\ programmer group in
order to have permission to write to the \aipspp\ source directories.  The
conventional name for this group is \acct{aips2prg} (see \sref{Accounts and
groups}), but it may be different at your site.  Type

\begin{verbatim}
   yourhost% groups
\end{verbatim}

\noindent
to list all groups that you are a member of, and consult your local \aipspp\ 
manager if in doubt.

Apart from invoking \exeref{casainit}, \aipspp\ programmers need to create a
shadow copy of the \aipspp\ \file{code} directory tree to serve as their
\aipspp\ workspace.  The \exeref{mktree.svn} utility does this:

\begin{verbatim}
   yourhost% mkdir $HOME/casa
   yourhost% cd $HOME/casa
   yourhost% mktree.svn
\end{verbatim}

\noindent
Apart from creating a shadow copy of the \aipspp\ \file{code} directory tree,
\aipsexe{mktree.svn} creates symbolic links into the \aipspp\ \file{rcs} directory
tree thereby linking the programmer's workspace to the local copy of the
\aipspp\ \rcs\ repository (see \sref{SVN directories}).

\aipsexe{mktree.svn} works incrementally so that if any workspace directories or
\file{RCS} symbolic links are accidently deleted, or if new \aipspp\ 
directories are created, \aipsexe{mktree.svn} will recreate only what is
necessary.  For a more detailed description, see \exeref{mktree.svn}.

Once the programmer workspace has been created by \aipsexe{mktree.svn} \aipspp\ 
sources can be checked in and out, updated, renamed, or deleted by the
the standard subversion commands.
For a more detailed description see
chapter \ref{Code management}.
