%%%%%%%%%%%%%%%%%%%%%%%%%%%%%%%%%%%%%%%%%%%%%%%%%%%%%%%%%%%%%%%%%
%%%%%%%%%%%%%%%%%%%%%%%%%%%%%%%%%%%%%%%%%%%%%%%%%%%%%%%%%%%%%%%%%
%%%%%%%%%%%%%%%%%%%%%%%%%%%%%%%%%%%%%%%%%%%%%%%%%%%%%%%%%%%%%%%%%

% STM 2007-04-13  split from previous version
% STM 2007-04-15  remove tool stuff
% STM 2007-04-19  start main update
% STM 2007-05-14  kumar's corrections
% STM 2007-06-15  start to bring to Alpha Patch 1 level
% STM 2007-06-16  add NGC5921 example
% STM 2007-09-25  beta version
% STM 2007-10-09  add Jupiter example
% STM 2007-10-10  spell-checked
% STM 2008-03-18  update to patch 1.0
% STM 2008-06-10  patch 2.0, new clean task
% STM 2008-09-30  Patch 3 editing start
% STM 2008-10-05  Patch 3 add EF widefield text
% STM 2008-12-02  Add list of primary beams
% STM 2009-05-29  Patch 4 
% STM 2009-06-05  Patch 4 clarkstokes
% STM 2009-06-19  Patch 4 clean output images
% STM 2009-11-09  Release 0 places for boxit and autoclean

\chapter{Synthesis Imaging}
\label{chapter:im}

\begin{wrapfigure}{r}{2.5in}
  \begin{boxedminipage}{2.5in}
     \centerline{\bf Inside the Toolkit:}
     The {\tt im} tool handles synthesis imaging operations.
  \end{boxedminipage}
\end{wrapfigure}

This chapter describes how to make and deconvolve images starting from
calibrated interferometric data, possibly supplemented with
single-dish data or an image made from single-dish data.  This data
must be available in CASA (see \S~\ref{chapter:io} on importing data).
See \S~\ref{chapter:cal} for information on calibrating synthesis 
data.  In the following sections, the user will learn how to 
make various types of images from synthesis data, reconstruct images
of the sky using the available deconvolution techniques, include
single-dish information in the imaging process, and to prepare
to use the results of imaging for improvement of the calibration 
process (``self-calibration'').

%%%%%%%%%%%%%%%%%%%%%%%%%%%%%%%%%%%%%%%%%%%%%%%%%%%%%%%%%%%%%%%%%
%%%%%%%%%%%%%%%%%%%%%%%%%%%%%%%%%%%%%%%%%%%%%%%%%%%%%%%%%%%%%%%%%
\section{Imaging Tasks Overview}
\label{section:im.tasks}

The current imaging and deconvolution tasks are:
\begin{itemize}
   \item {\tt clean} --- calculate a deconvolved image with a selected clean
         algorithm, including mosaicing, or make a dirty image
         (\S~\ref{section:im.clean})
   \item {\tt feather} --- combine a single dish and synthesis image in the
         Fourier plane (\S~\ref{section:im.feather})
   \item {\tt deconvolve} --- image-plane only deconvolution based on
         the dirty image and beam, using one of several algorithms
         (\S~\ref{section:im.deconvolve})
   \item {\tt autoclean} --- an experimental prototype task to
         automatically clean a dataset
         (\S~\ref{section:im.autoclean})
\end{itemize}

There are also tasks that help you set up the imaging or interface
imaging with calibration:

\begin{itemize}
   \item {\tt boxit} - create ``cleanbox'' deconvolution regions automatically
         from an image (\S~\ref{section:im.mask.boxit})
   \item {\tt makemask} - create ``cleanbox'' deconvolution regions
         from a box specification or file (\S~\ref{section:im.mask.makemask})
   \item {\tt ft} - Fourier transform the specified model (or component
         list) and insert the source model into the MODEL column of a
         visibility set (\S~\ref{section:im.ft})
\end{itemize}

The full ``tool kit'' that allows expert-level imaging must still be
used if you do not find enough functionality within the tasks above. 

Information on other useful tasks and parameter setting can be found in:
\begin{itemize}
   \item {\tt listobs} --- list whats in a MS (\S~\ref{section:io.list}),
   \item {\tt split}--- Write out new MS containing calibrated data
      from a subset of the original MS (\S~{section:cal.split}),
   \item data selection --- general data selection syntax
      (\S~\ref{section:io.selection}).
   \item {\tt viewer} --- image display including region statistics and
         image cube slice and profile capabilities
         (\S~\ref{chapter:display})
\end{itemize}

%%%%%%%%%%%%%%%%%%%%%%%%%%%%%%%%%%%%%%%%%%%%%%%%%%%%%%%%%%%%%%%%%
%%%%%%%%%%%%%%%%%%%%%%%%%%%%%%%%%%%%%%%%%%%%%%%%%%%%%%%%%%%%%%%%%
\section{Common Imaging Task Parameters}
\label{section:im.pars}

\begin{wrapfigure}{r}{2.5in}
  \begin{boxedminipage}{2.5in}
     \centerline{\bf Inside the Toolkit:}
     The {\tt im.setimage} method is used to set many of the
     common image parameters.  The {\tt im.advise} method
     gives helpful advice for setting up for imaging.
  \end{boxedminipage}
\end{wrapfigure}

We now describe some parameters are are common to the imaging
tasks.  These should behave the same way in any imaging task
that they are found in.  These are in alphabetical order.

{\bf ALERT:} There are still a subset of
data selection parameters used in the imaging tasks:
{\tt field}, {\tt spw}, {\tt timerange}.
In a later patch, we will use 
the standard data selection set (\S~\ref{section:io.selection}).

%%%%%%%%%%%%%%%%%%%%%%%%%%%%%%%%%%%%%%%%%%%%%%%%%%%%%%%%%%%%%%%%%
\subsection{Parameter {\tt cell} }
\label{section:im.pars.cell}

The {\tt cell} parameter defines the pixel size in the
x and y axes for the output
image.  If given as floats or integers, this is the cell size
in arc seconds, e.g.
\small
\begin{verbatim}
  cell=[0.5,0.5]
\end{verbatim}
\normalsize
make 0.5$''$ pixels.
You can also give the cell size in {\it quantities}, e.g.
\small
\begin{verbatim}
  cell=['1arcmin', '1arcmin']
\end{verbatim}
\normalsize
If a single value is given, then square pixels of that size
are assumed.

%%%%%%%%%%%%%%%%%%%%%%%%%%%%%%%%%%%%%%%%%%%%%%%%%%%%%%%%%%%%%%%%%
\subsection{Parameter {\tt field} }
\label{section:im.pars.field}

The {\tt field} parameter selects the field indexes or names
to be used in imaging.  Unless you are making a mosaic, this
is usually a single index or name:
\small
\begin{verbatim}
  field = '0'             #   First field (index 0)
  field = '1331+305'      #   3c286
  field = '*'             #   all fields in dataset
\end{verbatim}
\normalsize

The syntax for {\tt field} selection is given in 
\S~\ref{section:io.selection.field}.

%%%%%%%%%%%%%%%%%%%%%%%%%%%%%%%%%%%%%%%%%%%%%%%%%%%%%%%%%%%%%%%%%
\subsection{Parameter {\tt imagename} }
\label{section:im.pars.imagename}

The value of the {\tt imagename} parameter is used as the root name
of the output image.  Depending on the particular task and the
options chosen, one or more images with names built from that root
will be created.  
For example, the {\tt clean} task run with
{\tt imagename='ngc5921} 
a series of output images will be created with the names
{\tt ngc5921.clean}, {\tt ngc5921.residual}, {\tt ngc5921.model}, etc.

If an image with that name already exists, it will in general
be overwritten.  Beware using names of existing images however.
If the {\tt clean} is run using an {\tt imagename} where 
{\tt <imagename>.residual} and {\tt <imagename>.model} already
exist then {\tt clean} will continue starting from these
(effectively restarting from the end of the previous {\tt clean}). 
Thus, if multiple runs of {\tt clean} are run consecutively
with the same {\tt imagename}, then the cleaning is incremental
(as in the {\tt difmap} package).

%%%%%%%%%%%%%%%%%%%%%%%%%%%%%%%%%%%%%%%%%%%%%%%%%%%%%%%%%%%%%%%%%
\subsection{Parameter {\tt imsize} }
\label{section:im.pars.imsize}

The image size in numbers of pixels on the x and y axes is
set by {\tt imsize}.  For example,
\small
\begin{verbatim}
  imsize = [288, 288]
\end{verbatim}
\normalsize
makes a square image 288 pixels on a side.  If a single value
is given, then a square image of that dimension is made.  This
need not be a power of two, but for optimal performance should
be a ``composite'' number divisible only by 2 and also 3 and/or
5.  Note that in the example above $288 = 2^5 \cdot 3^2$.

{\bf ALERT:} You will be warned if you give a non-composite
{\tt imsize} and it will suggest the nearest appropriate value.
But it will continue cleaning so you may have to abort if you want
to make use of this suggestion.  This restriction will be lifted
in the future when we migrate to a better FFTW engine.

%%%%%%%%%%%%%%%%%%%%%%%%%%%%%%%%%%%%%%%%%%%%%%%%%%%%%%%%%%%%%%%%%
\subsection{Parameter {\tt mode} }
\label{section:im.pars.mode}

The {\tt mode} parameter defines how the frequency channels in the
synthesis MS are mapped onto the image.  The allowed values are:
{\tt mfs}, {\tt channel}, {\tt velocity}, {\tt frequency}.
The {\tt mode} parameter
is expandable, with some options uncovering a number of
sub-parameters, depending upon its value.

%%%%%%
\subsubsection{Mode {\tt mfs} }
\label{section:im.pars.mode.mfs}

The default {\tt mode='mfs'} emulates multi-frequency synthesis in
that each visibility-channel datum $k$ with baseline vector
$\bvec{B}_k$ at wavelength $\lambda_k$ is gridded into the uv-plane at
$\bvec{u}_k = \bvec{B}_k/\lambda_k$.  The result is a single image
plane, regardless of how many channels are in the input dataset.
This image plane is at the frequency given by the midpoint between
the highest and lowest frequency channels in the input {\tt spw}(s).
Currently, there is no way to choose the center frequency of the 
output image plane independently.

%%%%%%
\subsubsection{Mode {\tt channel} }
\label{section:im.pars.mode.channel}

If {\tt mode='channel'} is chosen, then an image cube will be
created. This is an expandable parameter, with dependent parameters:
\small
\begin{verbatim}
mode              =  'channel'   #  Type of selection (mfs, channel, velocity, frequency)
   nchan          =          1   #  Number of channels (planes) in output image
   start          =          0   #  first input channel to use
   width          =          1   #  Number of input channels to average
   interpolation  =  'nearest'   #  Type of spectral interpolation of visibilities (nearest, linear, cubic)
\end{verbatim}
\normalsize
The channelization of the resulting image is determined by the 
channelization in the first MS of {\tt vis} of the first {\tt spw} 
specified (the ``reference {\tt spw}''). The actual channels to
be gridded and used in the clean are selected via the {\tt spw} 
parameter as usual. 
The resulting image cube will have {\tt nchan} channels spaced
evenly in frequency.  The
first output channel will be located at the frequency of channel 
{\tt start} in the (first) reference {\tt spw}.  If {\tt width}$ > 1$,
then input MS channels with centers within a frequency range given by 
$({\tt width}+1)/2$ times the reference {\tt spw} spacing
will be gridded together (as in {\tt mode = 'mfs'} above)
into the channels of the output image cube.  
The output channel 
spacing is thus given by {\tt width} channels in the reference {\tt spw}
of the MS.  
  
The {\tt interpolation} sub-parameter
(\S~\ref{section:im.pars.mode.interpolation}) sets how channels are
gridded into the image cube planes.  For {\tt 'nearest'}, the
channels in {\tt spw} beyond the first are mapped
into the nearest output image channel within half a channel (if
any).  Otherwise, the chosen interpolation scheme will be used.
Image channels that lie outside the MS frequency range or
have no data mapped to them will be blank in the output image,
but will be in the cube.  

See the example in \S~\ref{section:scripts.ngc5921} for using the {\tt 'channel'}
mode to image a spectral-line cube. In this case, we use:
\small
\begin{verbatim}
mode         = 'channel'       
     nchan   =         46   
     start   =          5   
     width   =          1   
\end{verbatim}
\normalsize
which will produce a 46-channel cube starting with channel 5 of the MS
with the same channel width as the MS.

%%%%%%
\subsubsection{Mode {\tt frequency} }
\label{section:im.pars.mode.frequency}

For {\tt mode='frequency'}, an output image cube is created
with {\tt nchan} channels spaced evenly in frequency.
\small
\begin{verbatim}
mode              = 'frequency' # Type of selection (mfs, channel, velocity, frequency)
          nchan   =          1  # Number of channels (planes) in output image
          start   =   '1.4GHz'  # Frequency of first image channel: e.q. '1.4GHz'
          width   =    '10kHz'  # Image channel width in frequency units: e.g '1.0kHz'
   interpolation  =  'nearest'  # Type of spectral interpolation of visibilities (nearest, linear, cubic)
\end{verbatim}
\normalsize
The frequency of the first output channel is given by {\tt start}
and spacing by {\tt width}.  The sign of {\tt width}
determines whether the output channels ascend or descend in
frequency.  Output channels have a width also given by {\tt width}.
Data from the input MS with centers that lie within one-half an input
channel overlap of the frequency range of $\pm{\tt width}/2$ centered
on the output channels are gridded together.  
The {\tt interpolation} sub-parameter
(\S~\ref{section:im.pars.mode.interpolation}) sets how channels are
gridded into the image cube planes.

Using the NGC5921 dataset as an example:
\small
\begin{verbatim}
mode         = 'frequency'       
     nchan   =         21        
     start   = '1412.830MHz'     
     width   =    '50kHz'        

\end{verbatim}
\normalsize
would produce a 21-channel output cube with 50~kHz wide channels
rather than the default channelization of the MS (24.4~kHz).

%{\bf ALERT:} In Patch 3 and earlier, it will current grid each
%channel of the input MS to the nearest image plane, or two if it lies 
%on or close to the border between two planes in frequency or velocity.
%Thus if you choose to make an image with the channelization
%non-integral multiples of the MS channel width, you will see effects
%in the image cube where some planes have higher or lower noise than
%others depending on how many input channels went into the plane.  We
%will fix this behavior and use true interpolation in future releases.
%In the meantime we suggest you use {\tt mode='channel'} or be careful
%with the choice of {\tt start} and {\tt width}.

%%%%%%
\subsubsection{Mode {\tt velocity} }
\label{section:im.pars.mode.velocity}

If {\tt mode='velocity'} is chosen, then an output image cube
with {\tt nchan} channels will be created, with channels spaced
evenly in velocity.  Parameters are:
\small
\begin{verbatim}
mode              = 'velocity' # Type of selection (mfs, channel, velocity, frequency)
          nchan   =          1 # Number of channels (planes) in output image
          start   =  '0.0km/s' # Velocity of first image channel: e.g '0.0km/s'
          width   =    '1km/s' # image channel width in velocity units: e.g '-1.0km/s'
   interpolation  =  'nearest' # Type of spectral interpolation of visibilities (nearest, linear, cubic)
\end{verbatim}
\normalsize
The velocity of the first output channel is given by {\tt start}
and spacing by {\tt width}.  Note that the velocity frame is given
by the rest frequency in the MS header, which can be overridden
by the {\tt restfreq} parameter.  Averaging is as in
{\tt mode='frequency'}.  The {\tt interpolation} sub-parameter
(\S~\ref{section:im.pars.mode.interpolation}) sets how channels are
gridded into the image cube planes.

Again, using the NGC5921 dataset as an example:
\small
\begin{verbatim}
mode         = 'velocity'        
     nchan   =         21        
     start   = '1383.0km/s'      
     width   =   '10km/s'        
\end{verbatim}
\normalsize
Note that in this case the velocity axis runs forward, as opposed to
the default channelization for {\tt 'channel'} or {\tt 'frequency'}.

{\bf ALERT:} Note that the velocities are expressed in the 
{\tt LSRK} frame.  This is not currently selectable, but in the
future will be through {\tt restfreq} (\S~\ref{section:im.pars.restfreq}).

%{\bf ALERT:} In Patch 3 and earlier, it will current grid each
%channel of the input MS to the nearest image plane, or two if it lies 
%on or close to the border between two planes in frequency or velocity.
%Thus if you choose to make an image with the channelization
%non-integral multiples of the MS channel width, you will see effects
%in the image cube where some planes have higher or lower noise than
%others depending on how many input channels went into the plane.  We
%will fix this behavior and use true interpolation in future releases.
%In the meantime we suggest you use {\tt mode='channel'} or be careful
%with the choice of {\tt start} and {\tt width}.

%%%%%%
\subsubsection{Sub-parameter {\tt interpolation} }
\label{section:im.pars.mode.interpolation}

The {\tt interpolation} sub-parameter controls how spectral channels
in the MS are gridded into the output image cube.  This is available 
in all {\tt mode}s except {\tt 'mfs'}.  The options are: 
{\tt 'nearest'}, {\tt 'linear'}, {\tt 'cubic'}.

For {\tt 'nearest'}, the channels in {\tt spw} beyond the first are
mapped into the nearest output image channel within half a channel (if
any).

For {\tt 'linear'}, the channels are gridded into the planes using
weights given by a linear function of the frequency of the MS channel
versus the plane.  Each input channel will be mapped to 1 or 2 output
planes.  For most users, this is the best choice.

For {\tt 'cubic'}, the channels are gridded using a cubic
interpolation function.  

%%%%%%%%%%%%%%%%%%%%%%%%%%%%%%%%%%%%%%%%%%%%%%%%%%%%%%%%%%%%%%%%%
\subsection{Parameter {\tt phasecenter} }
\label{section:im.pars.phasecenter}

The {\tt phasecenter} parameter indicates which of the field IDs 
should be used to define the phase center of the mosaic image,
or what that phase center is in RA and Dec.
The default action is to use the first one given in the 
{\tt field} list.

For example:
\small
\begin{verbatim}
   phasecenter='5'                        # field 5 in multi-src ms
   phasecenter='J2000 19h30m00 -40d00m00' # specify position
\end{verbatim}
\normalsize

Note that the format for angles prefers to use {\tt hm} for RA/time
units and {\tt dm} for Dec/Angle units as separators.  The colon
{\tt ::} separator is interpreted as RA/time even if its used for the
Dec, so be careful not to copy/paste from other sources.

%%%%%%%%%%%%%%%%%%%%%%%%%%%%%%%%%%%%%%%%%%%%%%%%%%%%%%%%%%%%%%%%%
\subsection{Parameter {\tt restfreq} }
\label{section:im.pars.restfreq}

The value of the {\tt restfreq} parameter, if set, will over-ride
the rest frequency in the header of the first input MS to define
the velocity frame of the output image.

{\bf ALERT:} The {\tt restfreq} parameter takes the options
of transitions and frequencies as in the corresponding {\tt plotxy}
parameter (\S~\ref{section:edit.plot.restfreq}), but does not
currently expand to allow frame information (is hardwired into LSRK).

For example:
\small
\begin{verbatim}
   restfreq='115.2712GHz',
\end{verbatim}
\normalsize
will set the rest frequency to that of the CO 1-0 line.

{\bf ALERT:} Setting {\tt restfreq} explicitly here in
{\tt clean} is good practice, and may be necessary if your MS has
been concatenated from different files for different spectral
windows (\S~\ref{section:io.concat}).

%%%%%%%%%%%%%%%%%%%%%%%%%%%%%%%%%%%%%%%%%%%%%%%%%%%%%%%%%%%%%%%%%
\subsection{Parameter {\tt spw} }
\label{section:im.pars.spw}

The {\tt spw} parameter selects the spectral windows that will
be used to form the image, and possibly a subset of channels
within these windows.

The {\tt spw} parameter is a string with an integer, list
of integers, or a range, e.g.  
\small
\begin{verbatim}
  spw = '1'                #  select spw 1
  spw = '0,1,2,3'          #  select spw 0,1,2,3
  spw = '0~3'              #  same thing using ranges
\end{verbatim}
\normalsize
You can select channels in the same string with a {\tt :} separator,
for example
\small
\begin{verbatim}
  spw = '1:10~30'          #  select channels 10-30 of spw 1
  spw = '0:5~55,3:5;6;7'   #  chans 5-55 of spw 0 and 5,6,7 of spw 3
\end{verbatim}
\normalsize
This uses the standard syntax for {\tt spw} selection is given in 
\S~\ref{section:io.selection.spw}.  See that section for more
options.

Note that the order in which multiple {\tt spw}s are given is
important for {\tt mode = 'channel'}, as
this defines the origin for the channelization of the resulting
image.

%%%%%%%%%%%%%%%%%%%%%%%%%%%%%%%%%%%%%%%%%%%%%%%%%%%%%%%%%%%%%%%%%
\subsection{Parameter {\tt stokes} }
\label{section:im.pars.stokes}

The {\tt stokes} parameter specifies the Stokes parameters for the
resulting images.  Note that forming Stokes {\tt Q} and {\tt U} images
requires the presence of
cross-hand polarizations (e.g. {\tt RL} and {\tt LR} for circularly
polarized systems such as the VLA) in the data.  
Stokes {\tt V} requires both parallel
hands ({\tt RR} and {\tt :LL}) for circularly polarized systems or
the cross-hands ({\tt XY} and {\tt YX}) for linearly polarized systems
such as ALMA and ATCA.

This parameter is specified as a string of up to four letters ({\tt IQUV}).
For example,
\small
\begin{verbatim}
  stokes = 'I'            # Intensity only
  stokes = 'IQU'          # Intensity and linear polarization
  stokes = 'IV'           # Intensity and circular polarization
  stokes = 'IQUV'         # All Stokes imaging
\end{verbatim}
\normalsize
are common choices. ()
The output image will have planes (along the ``polarization axis'')
corresponding to the chosen Stokes parameters.  

If as input to deconvolution tasks such as {\tt clean}, the {\tt stokes} parameter
includes polarization planes other than I, then choosing {\tt psfmode='hogbom'} 
(\S~\ref{section:im.clean.psfmode.hogbom}) or {\tt psfmode='clarkstokes'} 
(\S~\ref{section:im.clean.psfmode.clarkstokes}) will clean (search for
components) each plane sequentially, while {\tt psfmode='clark'} 
(\S~\ref{section:im.clean.psfmode.clark}) will deconvolve jointly.

%Fixed in 2.4.0
%{\bf ALERT:} The {\tt stokes = 'QU'} for linear polarization
%only is not currently an option.  There is also no option to
%make single polarization product (e.g. separate {\tt RR} 
%and {\tt LL}, or {\tt XX} and {\tt YY}) images from data with
%dual polarizations available.  You currently would have to make
%{\tt stokes='I'} images from data with a single polarization product
%(e.g. {\tt RR} or {\tt LL}) split out.

%%%%%%%%%%%%%%%%%%%%%%%%%%%%%%%%%%%%%%%%%%%%%%%%%%%%%%%%%%%%%%%%%
\subsection{Parameter {\tt uvtaper} }
\label{section:im.pars.uvtaper}

This controls the radial weighting of visibilities in the uv-plane
(see \S~\ref{section:im.pars.weighting} below) through the multiplication
of the visibilities by the Fourier transform of an elliptical Gaussian.
This is itself a Gaussian, and thus the visibilities are ``tapered'' with
weights decreasing as a function of uv-radius.

The {\tt uvtaper} parameter expands the menu upon setting {\tt uvtaper=True} 
to reveal the following sub-parameters:
\small
\begin{verbatim}
uvtaper             =       True        #  Apply additional uv tapering of  visibilities.
     outertaper     =         []        #  uv-taper on outer baselines in uv-plane
     innertaper     =         []        #  uv-taper in center of uv-plane (not implemented)\end{verbatim}
\normalsize
The sub-parameters specify the size and optionally shape and
orientation of this Gaussian in the uv-plane or optionally the
sky plane.  The {\tt outertaper} refers to a Gaussian centered on 
the origin of the uv-plane.

Some examples:
\small
\begin{verbatim}
   outertaper=[]                                  # no outer taper applied
   outertaper=['5klambda']                        # circular uv taper FWHM=5 kilo-lambda
   outertaper=['5klambda','3klambda','45.0deg']   # elliptical Gaussian
   outertaper=['10arcsec']                        # on-sky FWHM 10"
   outertaper=['300.0']                           # 300m in aperture plane
\end{verbatim}
\normalsize
Note that if no units are given on the taper, then the default units
are assumed to be meters in aperture plane.

{\bf ALERT:} The {\tt innertaper} option is not yet implemented. 

%Note that since this filter effectively {\it multiplies} the intrinsic
%visibility weights, the resulting image will not have a PSF given by the size of the
%filter, but a PSF given by its intrinsic size convolved by the filter.  Thus you should
%end up with a synthesized beam of size equal to the quadratic sum of the original beam
%and the filter.

%%%%%%%%%%%%%%%%%%%%%%%%%%%%%%%%%%%%%%%%%%%%%%%%%%%%%%%%%%%%%%%%%
\subsection{Parameter {\tt weighting} }
\label{section:im.pars.weighting}

\begin{wrapfigure}{r}{2.5in}
  \begin{boxedminipage}{2.5in}
     \centerline{\bf Inside the Toolkit:}
     The {\tt im.weight} method has more weighting options
     than available in the imaging tasks.  See the 
     {\bf User Reference Manual} for more information on
     imaging weights.
  \end{boxedminipage}
\end{wrapfigure}

In order to image your data, we must have a map from the visibilities
to the image.  Part of that map, which is effectively a convolution,
is the weights by which each visibility is multiplied before gridding.
The first factor in the weighting is the ``noise'' in that visibility,
represented by the data weights in the MS (which is calibrated along
with the visibility data).
The weighting function can also depend upon the uv locus of that visibility
(e.g. a ``taper'' to change resolution).  This is actually controlled
by the {\tt uvtaper} parameter (see \S~\ref{section:im.pars.uvtaper}).
The weighting matrix also includes the convolution kernel that
distributes that visibility onto the uv-plane during gridding before
Fourier transforming to make the image of the sky.  This depends upon
the density of visibilities in the uv-plane (e.g. ``natural'',
``uniform'', ``robust'' weighting).

The user has control over all of these.

{\bf ALERT:} You can find a weighting description in the online
User Reference Manual at:

\url{http://casa.nrao.edu/docs/casaref/imager.weight.html}


The {\tt weighting} parameter expands the menu to include various 
sub-parameters depending upon the mode chosen:

%%%%%%
\subsubsection{{\tt 'natural'} weighting }
\label{section:im.pars.weighting.natural}

For {\tt weighting='natural'}, visibilities are weighted only by the
data weights, which are calculated during filling and calibration and
should be equal to the inverse noise variance on that visibility.
Imaging weight $w_i$ of sample $i$ is given by
\begin{equation}
  w_i = \omega_i = {1 \over \sigma^2_k}
\end{equation}
where the data weight $\omega_i$ is determined from $\sigma_i$ is the
rms noise on visibility $i$.
When data is gridded into the same uv-cell for imaging, the weights
are summed, and thus a higher uv density results in higher imaging
weights.  No sub-parameters are linked to this mode choice.  It is the
default imaging weight mode, and it should produce ``optimum'' image
with with the lowest noise (highest signal-to-noise ratio).  Note that
this generally produces images with the poorest angular resolution,
since the density of visibilities falls radially in the uv-plane

%%%%%%
\subsubsection{{\tt 'uniform'} weighting }
\label{section:im.pars.weighting.uniform}

For {\tt weighting = 'uniform'}, the data weights are calculated
as in {\tt 'natural'} weighting.  The data is then gridded to
a number of cells in the uv-plane, and after all data is gridded
the uv-cells are re-weighted to have ``uniform'' imaging weights.
This pumps
up the influence on the image of data with low weights (they are
multiplied up to be the same as for the highest weighted data), which
sharpens resolution and reduces the sidelobe level in the
field-of-view, but increases the rms image noise. No sub-parameters are
linked to this mode choice.

For uniform weighting, we first grid the inverse variance $\omega_i$
for all selected data onto a grid with uv cell-size given by $2/FOV$
where $FOV$ is the specified field of view (defaults to the image
field of view).  This forms the gridded weights $W_k$. The weight of
the $i$-th sample is then: 
\begin{equation}
  w_i = {\omega_i \over W_k}.
\end{equation}

%%%%%%
\subsubsection{{\tt 'superuniform'} weighting }
\label{section:im.pars.weighting.superuniform}

The {\tt weighting = 'superuniform'} mode is similar to the
{\tt 'uniform'} weighting mode but there is now an additional
{\tt npixels} sub-parameter that specifies a change to the number of 
cells on a side (with respect to uniform weighting) to define a
uv-plane patch for the weighting renormalization.  
If {\tt npixels=0} you get uniform weighting.

%%%%%%
\subsubsection{{\tt 'radial'} weighting }
\label{section:im.pars.weighting.radial}

The {\tt weighting = 'radial'} mode is a seldom-used option that
increases the weight by the radius in the uv-plane, ie.
\begin{equation}
  w_i = \omega_i \cdot \sqrt{u^2_i + v^2_i}.
\end{equation}
Technically, I would call that an inverse uv-taper since it depends on
uv-coordinates and not on the data per-se.
Its effect is to reduce the rms sidelobes for an east-west synthesis
array.  This option has limited utility.

%%%%%%
\subsubsection{{\tt 'briggs'} weighting }
\label{section:im.pars.weighting.briggs}

The {\tt weighting = 'briggs'} mode is an implementation of the
flexible weighting scheme developed by Dan Briggs in his PhD thesis.
See:

\url{http://www.aoc.nrao.edu/dissertations/dbriggs/}

This choice brings up the sub-parameters:
\small
\begin{verbatim}
weighting      =   'briggs'   #   Weighting to apply to visibilities 
     robust    =        0.0   #   Briggs robustness parameter
     npixels   =          0   #   number of pixels to determine uv-cell size 0=> field of view
\end{verbatim}
\normalsize

The actual weighting scheme used is:
\begin{equation}
   w_i={\omega_i\over{1 + W_k f^2}}
\end{equation}
where $W_k$ is defined as in {\tt uniform} and {\tt superuniform}
weighting, and
\begin{equation}
   f^2={{(5*10^{-R})^2}\over{{\sum_k W_k^2}\over{\sum_i \omega_i}}}
\end{equation}
and $R$ is the robust parameter. 

The key parameter is the {\tt robust} parameter, which sets $R$ in the
Briggs equations.  The scaling of $R$ is such that $R = 0$ gives a
good trade-off between resolution and sensitivity. The {\tt robust}
$R$ takes value between $-2.0$ (close to uniform weighting) 
to $2.0$ (close to natural).

Superuniform weighting can be combined with Briggs weighting
using the {\tt npixels} sub-parameter.  This works as in 
{\tt 'superuniform'} weighting 
(\S~\ref{section:im.pars.weighting.superuniform}).

%%%%%%
\subsubsection{{\tt 'briggsabs'} weighting }
\label{section:im.pars.weighting.briggsabs}

For {\tt weighting='briggsabs'}, a slightly different Briggs weighting is used,
with
\begin{equation}
  w_i={\omega_i\over{W_k R^2 + 2 \sigma_R^2}}
\end{equation}
where $R$ is the robust parameter and $\sigma_R$ is the {\tt noise}
parameter. 

This choice brings up the sub-parameters:
\small
\begin{verbatim}
weighting      = 'briggsabs'  #   Weighting to apply to visibilities 
     robust    =      0.0     #   Briggs robustness parameter
     noise     =  '0.0Jy'     #   noise parameter for briggs weighting when rmode='abs'
     npixels   =        0     #   number of pixels to determine uv-cell size 0=> field of view
\end{verbatim}
\normalsize

Otherwise, this works as {\tt weighting='briggs'} above 
(\S~\ref{section:im.pars.weighting.briggs}).

%%%%%%%%%%%%%%%%%%%%%%%%%%%%%%%%%%%%%%%%%%%%%%%%%%%%%%%%%%%%%%%%%
\subsection{Parameter {\tt vis} }
\label{section:im.pars.vis}

\begin{wrapfigure}{r}{2.5in}
  \begin{boxedminipage}{2.5in}
     \centerline{\underline{\bf Beta Alert!}}
     Multi-MS handling is not percolated to the tasks yet, as
     we are still working on this.  Use single MS only.
  \end{boxedminipage}
\end{wrapfigure}

The value of the {\tt vis} parameter is either the name of a single
MS, or a list of strings containing the names of multiple MSs, that
should be processed to produce the image.  The MS referred to by the
first name in the list (if more than one) is used to determine
properties of the image such as channelization and rest frequency.

For example,
\small
\begin{verbatim}
  vis = 'ngc5921.ms'
\end{verbatim}
\normalsize
set a single input MS, while
\small
\begin{verbatim}
  vis = ['ngc5921_day1.ms', 'ngc5921_day2.ms', 'ngc5921_day3.ms']
\end{verbatim}
\normalsize
points to three separate measurement sets that will be gridded
together to form the image.  This means that you do not have
to concatenate datasets, for example from different configurations,
before imaging.

%%%%%%%%%%%%%%%%%%%%%%%%%%%%%%%%%%%%%%%%%%%%%%%%%%%%%%%%%%%%%%%%%
\subsection{Primary beams in imaging }
\label{section:im.pars.pb}
% Added STM 2008-12-02 based on KG info

The CASA imaging task and tools use primary beams based on models
for each observatory's antenna types.  In addition to different 
antenna diameters, different functions may be used.  

The voltage patterns are based on the following antenna primary beams,
based on the {\tt TELESCOPE\_NAME} keyword in the {\tt OBSERVATION}
table:
\begin{description}
\item[VLA] --- Airy disk fitted to measurement. Note that a R/L beam
squint is also included with feed dependent angle;

\item[ALMA] --- Airy disk for 12m dish with a blockage of 1m;

\item[ATA] --- Airy disk for 6m dish;

\item[ATCA] --- polynomial fitted to measurement of main lobe;

\item[BIMA, HATCREEK] --- Gaussian with halfwidth of $\lambda$/2D;

\item[CARMA] --- Airy patterns for the BIMA or OVRO dish sizes as appropriate;

\item[GBT] --- polynomial fitted to measurement of main lobe;

\item[GMRT] --- VLA Airy disk scaled to 45.0m;

\item[IRAMPDB] --- Airy disk for dish of 15m with a blockage of 1m;

\item[NRAO12M] ---  VLA beam scaled to 12m;

\item[OVRO] --- VLA Airy disk scaled to 10.4m;

\item[SMA] --- Spheroidal function fit to FWHM;

\item[WSRT] --- polynomial fitted to measurement of main lobe;

\end{description}

If the telescope name is unknown, or is CARMA or ALMA, then the
{\tt DISH\_DIAMETER} in the {\tt ANTENNA} table is used with a
scaled VLA pattern.

% Fixed in 2.4.0
%{\bf ALERT:} Currently this generic beam is only implemented for 
%{\tt imagermode='mosaic'} in {\tt clean}, but in the future will
%be enabled for all cases.

Note that for the purposes of mosaicing in {\tt clean}, the primary
beams that are Airy or spheroidal are best-behaved 
(see \S~\ref{section:im.clean.mosaic}).

%%%%%%%%%%%%%%%%%%%%%%%%%%%%%%%%%%%%%%%%%%%%%%%%%%%%%%%%%%%%%%%%%
%%%%%%%%%%%%%%%%%%%%%%%%%%%%%%%%%%%%%%%%%%%%%%%%%%%%%%%%%%%%%%%%%
\section{Deconvolution using CLEAN ({\tt clean})}
\label{section:im.clean}

To create an image and then deconvolve it with the CLEAN algorithm,
use the {\tt clean} task.  This task will work for single-field data,
or for multi-field mosaics (\S~\ref{section:im.clean.mosaic}).  
{\bf ALERT:} There is now an experimental feature to image data taken with
``heterogeneous'' arrays with non-identical dish sizes 
(\S~\ref{section:im.clean.hetero}).

The {\tt clean} task uses many
of the common imaging parameters.  These are described above in
\S~\ref{section:im.pars}.  There are also a number of parameters
specific to {\tt clean}.  These are listed and described below.

The default inputs to {\tt clean} are:
\small
\begin{verbatim}
#  clean :: Deconvolve an image with selected algorithm
vis                 =         ''   #  name of input visibility file
imagename           =       ['']   #  Pre-name of output images
outlierfile         =         ''   #  Text file with image names, sizes, centers
field               =         ''   #  Field Name
spw                 =         ''   #  Spectral windows:channels: '' is all
selectdata          =      False   #  Other data selection parameters
mode                =      'mfs'   #   Type of selection (mfs, channel, velocity,frequency)
     nterms         =          1   #  Number of taylor terms to use for modeling the sky frequency dependence
     reffreq        =         ''   #  Reference frequency for MFS (relevant only if nterms > 1)

gridmode            =         ''   #  The kind gridding kernel to be used for FFT-based transforms
niter               =        500   #  Maximum number of iterations
gain                =        0.1   #  Loop gain for cleaning
threshold           =   '0.0mJy'   #  Flux level to stop cleaning.  Must include units
psfmode             =    'clark'   #  method of PSF calculation to use during minor cycles
imagermode          =         ''   #   Use csclean or mosaic.  If '', use psfmode
multiscale          =         []   #  deconvolution scales (pixels); [] = default standard clean
interactive         =      False   #  use interactive clean (with GUI viewer)
mask                =         []   #  cleanbox(es), mask image(s), and/or region(s)  used in cleaning
imsize              = [256, 256]   #  x and y image size in pixels, symmetric for single value
cell                = ['1.0arcsec', '1.0arcsec'] #  x and y cell size. default unit arcsec
phasecenter         =         ''   #  Image phase center: position or field index
restfreq            =         ''   #  rest frequency to assign to image (see help)
stokes              =        'I'   #  Stokes params to image (eg I,IV, QU,IQUV)
weighting           =  'natural'   #  Weighting of uv (natural, uniform, briggs, ...)
uvtaper             =      False   #  Apply additional uv tapering of  visibilities.
modelimage          =         ''   #  Name of model image(s) to initialize cleaning
restoringbeam       =       ['']   #  Output Gaussian restoring beam for CLEAN image
pbcor               =      False   #  Output primary beam-corrected image
minpb               =        0.1   #  Minimum PB level to use
calready            =       True   #  Create scratch columns and store model visibilities for selfcal
async               =      False   #  If true the taskname must be started using clean(...)
\end{verbatim}
\normalsize

The {\tt clean} task will produce a number of output images based
on the root name given in {\tt imagename}.  
These include:
\small
\begin{verbatim}
   <imagename>.clean.image                # the restored image
   <imagename>.clean.flux                 # the effective response (e.g. for pbcor)
   <imagename>.clean.flux.pbcoverage      # the PB coverage (ftmachine='mosaic' only)
   <imagename>.clean.model                # the model image
   <imagename>.clean.residual             # the residual image
   <imagename>.clean.psf                  # the synthesized (dirty) beam
\end{verbatim}
\normalsize
{\bf ALERT:} The {\tt <imagename>.clean.flux.pbcoverage} image is
new in Patch 4 version 2.4.0 and reflects the unweighted primary beam
coverage (without weighting or gridding kernel factors) used for
the {\tt minpb} cutoff (\S~\ref{section:im.clean.minpb}).  This is
produced only for {\tt imagermode='mosaic'} with {\tt ftmachine='mosaic'}.

The {\tt mode}, {\tt psfmode}, {\tt imagermode}, and {\tt weighting}
parameters open up other sub-parameters.  These are detailed
in the common imaging task parameters section
(\S~\ref{section:im.pars}).

A typical setup for {\tt clean} on the NGC5921 dataset, 
after setting parameter values, might look like:
\small
\begin{verbatim}
vis             = 'ngc5921.usecase.ms.contsub' #  Name of input visibility file
imagename       = 'ngc5921.usecase.clean' #  Pre-name of output images
field           =        '0' # Field Name
spw             =         '' # Spectral windows:channels: '' is all
selectdata      =      False # Other data selection parameters
mode            =  'channel' # Type of selection (mfs, channel, velocity, frequency)
     nchan      =         46 # Number of channels (planes) in output image
     start      =          5 # first input channel to use
     width      =          1 # Number of input channels to average
 interpolation  =  'nearest' # Spectral interpolation (nearest, linear, cubic)

niter           =       6000 # Maximum number of iterations
gain            =        0.1 # Loop gain for cleaning
threshold       =        8.0 # Flux level to stop cleaning.  Must include units
psfmode         =    'clark' # method of PSF calculation to use during minor cycles
imagermode      =         '' # Use csclean or mosaic, or image-plane only if ''
multiscale      =         [] # set deconvolution scales (pixels)
interactive     =      False # use interactive clean (with GUI viewer)
mask            = [108, 108, 148, 148] #  cleanbox(es), mask image(s), and/or region(s)
imsize          = [256, 256]   #  x and y image size in pixels
cell            = [15.0, 15.0] #  x and y cell size. default unit arcsec
phasecenter     =         '' # Image phase center: position or field index
restfreq        =         '' # rest frequency to assign to image (see help)
stokes          =        'I' # Stokes params to image (eg I,IV, QU,IQUV)
weighting       =   'briggs' # Weighting to apply to visibilities
     robust     =        0.5 # Briggs robustness parameter
     npixels    =          0 # uv-cell size in pixels 0=> field of view

uvtaper         =      False # Apply additional uv tapering of  visibilities.
modelimage      =         '' # Name of model image(s) to initialize cleaning
restoringbeam   =       [''] # Output Gaussian restoring beam for CLEAN image
pbcor           =      False # Output primary beam-corrected image
minpb           =        0.1 # Minimum PB level to use
async           =      False        
\end{verbatim}
\normalsize

An example of the {\tt clean} task to
create a continuum image from many channels is given below: 
\small
\begin{verbatim}
clean(vis='ggtau.1mm.split.ms', # Use data in ggtau.1mm.split.ms
      imagename='ggtau.1mm',    # Name output images 'ggtau.1mm.*' on disk
      psfmode='clark',          # Use the Clark CLEAN algorithm
      imagermode='',            # Do not mosaic or use csclean
      mask='',                  # Do not use clean box or mask
      niter=500, gain=0.1,      # Iterate 500 times using gain of 0.1
      mode='mfs',               # multi-frequency synthesis (combine channels)
      spw='0~2:2~57',           # Combine channels from 3 spectral windows
      field='0',                # 
      stokes='I',               # Image stokes I polarization
      weighting='briggs',       # Use Briggs robust weighting 
      rmode='norm',robust=0.5,  #    with robustness parameter of 0.5
      cell=[0.1,0.1],           # Using 0.1 arcsec pixels
      imsize=[256,256])         # Set image size = 256x256 pixels

\end{verbatim}
\normalsize

This example will clean the entire inner quarter of the primary beam.
However, if you want to limit the region over which you allow the
algorithm to find clean components then you can make a deconvolution
region (or mask).  To use a deconvolution region, box, or mask, set
the {\tt mask} parameter (\S~\ref{section:im.clean.mask}).

\begin{wrapfigure}{r}{2.5in}
  \begin{boxedminipage}{2.5in}
     \centerline{\bf Inside the Toolkit:}
     The {\tt im.clean} method is used for CLEANing data.
     There are a number of methods used to set up the clean,
     including {\tt im.setoptions}.
  \end{boxedminipage}
\end{wrapfigure}

For example, you can set up a simple 'cleanbox' region.  To do this, make
a first cut at the image and clean the inner quarter.  Then use the
{\tt viewer} to look at the image and get an idea of where the
emission is located.  You can use the {\tt viewer adjustment} panel to
view the image in pixel coordinates and read out the pixel locations
of your cursor.  

Then, you can use those pixel read-outs you just go to define a clean
box region where you specify the bottom-left-corner (blc) x \& y and
top-right-corner x\& y locations.  For example, say you have a
continuum source near the center of your image between 
{\tt blcx, blcy, trcx, trcy = 80, 80, 120, 120}.  
Then to use this region: 
\small
\begin{verbatim}
   mask=[80,80,120,120]    # Set the deconvolution region as a simple box in the center. 
\end{verbatim}
\normalsize

The following are the {\tt clean} specific parameters and their
allowed values, followed by a description of carrying out 
interactive cleaning.

%%%%%%%%%%%%%%%%%%%%%%%%%%%%%%%%%%%%%%%%%%%%%%%%%%%%%%%%%%%%%%%%%
\subsection{Parameter {\tt psfmode}}
\label{section:im.clean.psfmode}

The {\tt psfmode} parameter chooses the ``algorithm'' that will be used
to calculate the synthesized beam for use during the minor cycles in
the image plane. The value types are strings.
Allowed choices are {\tt 'clark'} (default) and {\tt 'hogbom'}. 

\subsubsection{The {\tt clark} algorithm}
\label{section:im.clean.psfmode.clark}

In the {\tt 'clark'} algorithm, the cleaning is split into minor and
major cycles. In the minor cycles only the brightest points are
cleaned, using a subset of the point spread function. In the major
cycle, the points thus found are subtracted correctly by using an
FFT-based convolution.  This algorithm is reasonably fast.
Also, for polarization imaging, Clark searches for the peak in 
$I^2+Q^2+U^2+V^2$.

\subsubsection{The {\tt hogbom} algorithm}
\label{section:im.clean.psfmode.hogbom}

The {\tt hogbom} algorithm is the ``Classic'' image-plane CLEAN, where
model pixels are found iteratively by searching for the peak. Each
point is subtracted from the full residual image using the shifted and
scaled point spread function.  In general, this is not a good choice for most
imaging problems ({\tt clark} or {\tt csclean} are preferred) as it does not
calculate the residuals accurately.  But in some cases, with poor
uv-coverage and/or a PSF with bad sidelobes, the Hogbom algorithm will
do better as it uses a smaller beam patch.  For polarization cleaning,
Hogbom searches for clean peak in $I$, $Q$, $U$, and $V$ independently.

\subsubsection{The {\tt clarkstokes} algorithm}
\label{section:im.clean.psfmode.clarkstokes}

In the {\tt 'clarkstokes'} algorithm, the Clark psf 
(\S~\ref{section:im.clean.psfmode.clark}) is used, but for 
polarization imaging the Stokes planes are cleaned sequentially for
components instead of jointly as in {\tt 'clark'}.  This means
that this is the same as {\tt 'clark'} for Stokes I imaging only.
This option can also be combined with {\tt imagermode='csclean'}
(\S~\ref{section:im.clean.imagermode}).

\subsection{The {\tt multiscale} parameter}
\label{section:im.clean.multiscale}

{\bf ALERT:} The {\tt multiscale} option is currently under
development and should be used with caution and be considered
as an ``experimental'' algorithm.  The multi-scale CLEAN
method is known to need careful tuning in order to properly converge.
However, currently the only control for
{\tt multiscale} in the {\tt clean} task is the
setting of the {\tt scales}.

\begin{wrapfigure}{r}{2.5in}
  \begin{boxedminipage}{2.5in}
     \centerline{\bf Inside the Toolkit:}
     The {\tt im.setscales} method sets the multi-scale Gaussian
     widths.  In addition to choosing a list of sizes in pixels,
     you can just pick a number of scales and get a geometric series 
     of sizes.
  \end{boxedminipage}
\end{wrapfigure}

To activate multi-scale mode, specify a non-blank list of scales in the 
{\tt multiscale} parameter.  
e.g.
\small
\begin{verbatim}
  multiscale = [0,3,10,30]    # Four scales including point sources
\end{verbatim}
\normalsize
These are given in numbers of pixels, and specify FWHM of the
Gaussians used to compute the filtered images.

Setting the {\tt multiscale} parameter to a non-empty list opens up the sub-parameter:
\small
\begin{verbatim}
multiscale          = [0, 3, 10, 30]    #  set deconvolution scales (pixels)
     negcomponent   =         -1        #  Stop if largest scale finds this many neg components
\end{verbatim}
\normalsize
The {\tt negcomponent} sub-parameter is here to set
the point at which the clean terminates because of negative
components.  For {\tt negcomponent > 0}, component search will cease
when this number of negative components are found at the largest
scale.  If {\tt negcomponent = -1} then component search will continue
even if the largest component is negative.

The CASA multi-scale algorithm uses ``Multi-scale CLEAN'' to
deconvolve using delta-functions and circular Gaussians as the basis
functions for the model, instead of just delta-functions or pixels as
in the other clean algorithms.  This algorithm is still in the
experimental stage, mostly because we are working on better algorithms
for setting the scales for the Gaussians.  The sizes of the Gaussians
are set using the {\tt scales} sub-parameter.  

We are working on defining a better algorithm for scale setting.  In
the toolkit, there is an {\tt nscale} argument which sets scales
\begin{equation}
  \theta_i = \theta_{bmin}\,10^{(i-N/2)/2}
\end{equation}
where $N=${\tt nscales} and $\theta_{bmin}$ is the fitted FWHM of the minor
axis of the CLEAN beam.

%%%%%%%%%%%%%%%%%%%%%%%%%%%%%%%%%%%%%%%%%%%%%%%%%%%%%%%%%%%%%%%%%
\subsection{Parameter {\tt gain} }
\label{section:im.clean.gain}

The {\tt gain} parameter sets the fraction of the flux density in
the residual image that is removed and placed into the clean model
at each minor cycle iteration.  The default value is {\tt gain = 0.1}
and is suitable for a wide-range of imaging problems.  Setting it to
a smaller gain per cycle, such as {\tt gain = 0.05}, can sometimes 
help when cleaning images with lots of diffuse emission.  Larger values,
up to {\tt gain=1}, are probably too aggressive and are not recommended.

\begin{figure}[h!]
\begin{center}
\pngname{clean_inter_control}{6}
\caption{\label{fig:clean_inter_control} Close-up of the top of the 
interactive {\tt clean} window.  Note the boxes at the left
(where the {\tt iterations}, {\tt cycles}, and {\tt threshold} can
be changed), the buttons that control add/erase, the application
of mask to channels, and whether to stop, complete, or continue
cleaning, and the row of Mouse-button tool assignment icons.} 
\hrulefill
\end{center}
\end{figure}

%%%%%%%%%%%%%%%%%%%%%%%%%%%%%%%%%%%%%%%%%%%%%%%%%%%%%%%%%%%%%%%%%
\subsection{Parameter {\tt imagermode} }
\label{section:im.clean.imagermode}

This choose the mode of operation of {\tt clean}, either as
single-field deconvolution using image-plane major and minor
cycles only ({\tt imagermode=''}), single-field deconvolution
using Cotton-Schwab (CS) residual visibilities for major cycles
({\tt imagermode='csclean'}), or multi-field mosaics using CS
major cycles ({\tt imagermode='mosaic'}).

In the default mode ({\tt imagermode=''}), the major and minor
clean cycles work off of the gridded FFT dirty image, with residuals
updated using the PSF calculation algorithm set by the {\tt psfmode}
parameter (\S~\ref{section:im.clean.psfmode}).  This method is not
recommended for high dynamic range or high fidelity imaging 
applications, but can be significantly faster than CS clean.
Note that for this option only, if {\tt mask=''} (no mask or box
set) then it will clean the inner quarter of the image by
default.

The {\tt csclean} choice specifies the Cotton-Schwab algorithm.  This
opens up the sub-parameters
\small
\begin{verbatim}
imagermode          =  'csclean'   #  Use csclean or mosaic.  If '', use psfmode
     cyclefactor    =        1.5   #  Change depth in between of  csclean cycle
     cyclespeedup   =         -1   #  Cycle threshold doubles in this number of iterations

\end{verbatim}
\normalsize
These options are explained below.
In the CS mode, cleaning is split into minor and major cycles. For each field, a
minor cycle is performed using the PSF algorithm specified in
{\tt psfmode} (\S~\ref{section:im.clean.psfmode}).
At major-cycle breakpoints, the points
thus found are subtracted from the original visibilities. A fast
variant does a convolution using a FFT. This will be faster for large
numbers of visibilities.  If you want to be extra careful, double the image 
size from that used for the Clark clean and set a mask to clean only
the inner quarter or less (this is not done by default).
This is probably the best choice for high-fidelity deconvolution 
of images without lots of large-scale structure.

Note that when using the Cotton-Schwab algorithm with a {\tt threshold}
(\S~\ref{section:im.clean.thresh}), there may be strange
behavior when you hit the threshold with a major cycle.  In
particular, it may be above threshold again at the start of the next
major cycle.  This is particularly noticeable when cleaning a cube,
where different channels will hit the threshold at different times.

{\bf ALERT:} You will see a warning message in the logger, 
similar to this:
\small
\begin{verbatim}
Zero Pixels selected with a Flux limit of 0.000551377 and a maximum Residual of 0.00751239
\end{verbatim}
\normalsize
whenever it find 0 pixels above the threshold.  This is normal,
and not a problem, if you've specified a non-zero threshold.
On the other hand, if you get this warning with the threshold set to
the default of {\tt '0Jy'}, then you should look carefully at your
inputs or your data, since this usually means that the masking is bad.

The option {\tt imagermode='mosaic'} is for multi-field mosaics.  This
choice opens up the sub-parameters 
\small
\begin{verbatim}
imagermode          =   'mosaic'   #  Use csclean or mosaic.  If '', use psfmode
     mosweight      =      False   #  Individually weight the fields of the mosaic
     ftmachine      =   'mosaic'   #  Gridding method for the image
     scaletype      =    'SAULT'   #  Controls scaling of pixels in the image plane.
     cyclefactor    =        1.5   #  change depth in between of  csclean cycle
     cyclespeedup   =         -1   #  Cycle threshold doubles in this number of iteration
\end{verbatim}
\normalsize
These options are explained below.

%%%%%%
\subsubsection{Sub-parameter {\tt cyclefactor} }
\label{section:im.clean.imagermode.cyclefactor}

\begin{wrapfigure}{r}{2.5in}
  \begin{boxedminipage}{2.5in}
     \centerline{\bf Inside the Toolkit:}
     The {\tt im.setmfcontrol} method sets the parameters that control
     the cycles and primary beam used in mosaicing.
  \end{boxedminipage}
\end{wrapfigure}

This sub-parameter is activated for {\tt imagermode='csclean'} and
{\tt 'mosaic'}.

The {\tt cyclefactor} parameter allows the user to change the
threshold at which the deconvolution cycle will stop and then degrid and
subtract the model from the visibilities to form the residual. This is
with respect to the breaks between minor and major cycles that the
{\tt clean} part would normally force.  Larger values force a major
cycle more often.  

This parameter in effect controls the threshold used by CLEAN to test whether
a major cycle break and reconciliation occurs:
\small
\begin{verbatim}
     cycle threshold = cyclefactor * max sidelobe * max residual
\end{verbatim}
\normalsize

If your uv-coverage results in a poor PSF, then you should reconcile often
(a {\tt cyclefactor} of 4 or 5); For reasonable PSFs, use {\tt cyclefactor} in the range
$1.5$ to $2.0$.  For good PSFs, or for faster cleaning at the expense
of some fidelity, we recommend trying a lower value, e.g. 
{\tt cyclefactor = 0.25}, which at least in some of our mosaicing
tests led to a speedup of
around a factor of two with little change in residuals.

%%%%%%
\subsubsection{Sub-parameter {\tt cyclespeedup} }
\label{section:im.clean.imagermode.cyclespeedup}

This sub-parameter is activated for {\tt imagermode='csclean'} and
{\tt 'mosaic'}.

The {\tt cyclespeedup} parameter allows the user to let {\tt clean}
to raise the threshold at which a major cycle is forced if it is not
converging to that threshold.  To do this, set {\tt cyclespeedup} to
an integer number of iterations at which if the threshold is not reached,
the threshold will be doubled.  See {\tt cyclefactor} above for more
details.  By default this is turned off ({\tt cyclespeedup = -1}).
In our tests, a value like {\tt cyclespeedup = 50} has been used successfully.

%%%%%%
\subsubsection{Sub-parameter {\tt ftmachine} }
\label{section:im.imagermode.mosaic.ftmachine}

This sub-parameter is activated for {\tt imagermode='mosaic'}.

The {\tt ftmachine} parameter controls the gridding method and kernel to be
used to make the image.  A string value type is expected.
Choices are: {\tt 'ft'}, {\tt 'sd'}, {\tt 'both'}, or {\tt 'mosaic'} (the default).

The {\tt 'ft'} option uses the standard gridding kernel (as used in 
{\tt clean}).

The {\tt 'sd'} option forces gridding as in single-dish data.

For combining single-dish and interferometer MS in the imaging, the
{\tt 'both'} option will allow {\tt clean} to choose the {\tt `ft'}
or {\tt 'sd'} machines as appropriate for the data.

\begin{wrapfigure}{r}{2.5in}
  \begin{boxedminipage}{2.5in}
     \centerline{\bf Inside the Toolkit:}
     The {\tt im.setoptions} method sets the parameters relevant to
     mosaic imaging, such as the {\tt ftmachine}.
  \end{boxedminipage}
\end{wrapfigure}

The {\tt 'mosaic'} option (the default) uses the Fourier transform of the
primary beam (the aperture cross-correlation function in the uv-plane) as
the gridding kernel.  This allows the data from the multiple fields to be
gridded down to a single uv-plane, with a significant speed-up in 
performance in most (non-memory limited) cases.  The effect of this extra
convolution is an additional multiplication (apodization) by the primary
beam in the image plane.  This can be corrected for, but does result in
an image with optimal signal to noise ratio across it.

The primary beams used in CASA are described in \S~\ref{section:im.pars.pb}.

{\tt ALERT:} Note that making a non-square image (e.g. using
unequal values in {\tt imsize}) for {\tt ftmachine='mosaic'} will grid 
the data into a uv-plane with correspondingly non-square cells.  This
has not been extensively tested, and may results in undesired image
artifacts.  We recommend that the user make square mosaic images when
possible, but in principle non-square images should work.

%%%%%%%%%%%%%%%%%%%%%%%%%%%%%%%%%%%%%%%%%%%%%%%%%%%%%%%%%%%%%%%%%
\subsubsection{Sub-parameter {\tt mosweight} }
\label{section:im.imagermode.mosaic.mosweight}

If {\tt mosweight=True} then individual mosaic fields will receive independent 
weights, which will give optimum signal to noise ratio. 

If {\tt mosweight=False} then the data will be
weighted so that the signal-to-noise ratio is as uniform as possible
across the mosaic image.

%%%%%%%%%%%%%%%%%%%%%%%%%%%%%%%%%%%%%%%%%%%%%%%%%%%%%%%%%%%%%%%%%
\subsubsection{Sub-parameter {\tt scaletype} }
\label{section:im.clean.imagermode.scaletype}

\begin{wrapfigure}{r}{2.5in}
  \begin{boxedminipage}{2.5in}
     \centerline{\bf Inside the Toolkit:}
     The {\tt im.setmfcontrol} method gives more options for
     controlling the primary beam and noise across the image.
  \end{boxedminipage}
\end{wrapfigure}

The {\tt scaletype} parameter controls weighting of pixels in the 
image plane.  This sub-parameter is activated for {\tt imagermode='mosaic'}.

The default {\tt scaletype='PBCOR'} scales the
image to have the correct flux scale across it (out to the beam
level cutoff {\tt minpb}).  This means that the noise level will
vary across the image, being increased by the inverse of the 
weighted primary beam responses that are used to rescale the
fluxes.  This option should
be used with care, particularly if your data has very different exposure
times (and hence intrinsic noise levels) between the mosaic fields.

If {\tt scaletype='SAULT'} then the image will be scaled so as to have
constant noise across it.  This means that the point source response
function varies across the image attenuated by the weighted primary
beam(s).  However, this response is output in the {\tt .flux} image
and can be later used to correct this.

Note that this scaling as a function of position in the image occurs
after the weighting of mosaic fields specified by {\tt mosweight}
and implied by the gridding weights ({\tt ftmachine} and {weighting}).

%%%%%%%%%%%%%%%%%%%%%%%%%%%%%%%%%%%%%%%%%%%%%%%%%%%%%%%%%%%%%%%%%
\subsubsection{The {\tt threshold} revisited}
\label{section:im.clean.imagermode.threshold}

For mosaics, the specification of the threshold is not
straightforward, as it is in the single field case.  This is because
the different fields can be observed to different depths, and get
different weights in the mosaic.  For efficiency, {\tt clean} does
its component search on a weighted and scales version of the sky.

For {\tt ftmachine='ft'}, the minor cycles of the deconvolution are
performed on an image that has been weighted to have constant noise,
as in {\tt 'SAULT'} weighting (see
\S~\ref{section:im.clean.imagermode.scaletype}).  This is equivalent to making
a dirty mosaic by coadding dirty images made from the individual 
pointings with a sum of the mosaic contributions to a given pixel
weighted by so as to give constant noise across the image.
This means that the flux scale can vary across the mosaic depending
on the effective noise (higher weighted regions have lower noise, and
thus will have higher ``fluxes'' in the {\tt 'SAULT'} map).  Effectively,
the flux scale that threshold applies to is that at the center of the
highest-weighted mosaic field, with higher-noise regions down-scaled
accordingly.  Compared to the true sky, this image has a factor of
the PB, plus a scaling map (returned in the {\tt .flux} image).
You will preferentially find components in the low-noise
regions near mosaic centers.

When {\tt ftmachine='mosaic'}, the underlying deconvolution is
performed on a constant signal-to-noise image.  This is equivalent
to a dirty mosaic that is formed by coadding dirty images made from
the individual fields after apodizing each by the PB function.  Thus
compared to the true sky, this has a factor of the PB$^2$ in it.  You
will thus preferentially find components in the centers of the mosaic
fields (even more so than in the {\tt 'ft'} mosaics).

Both these cases should have the same flux scale in the centers of the
lowest-noise pointings in the mosaic.  This is where the 
{\tt threshold} units match those in the image being used in the
minor cycle.

{\bf ALERT:} This is fairly complicated, and we are working on
explaining this better and possibly making this more straightforward
to specify.

%%%%%%%%%%%%%%%%%%%%%%%%%%%%%%%%%%%%%%%%%%%%%%%%%%%%%%%%%%%%%%%%%
\subsection{Parameter {\tt interactive} }
\label{section:im.clean.interactive}

If {\tt interactive=True} is set, then an interactive window
will appear at various ``cycle'' stages while you clean, so you
can set and change mask regions.  These breakpoints are controlled
by the {\tt npercycle} sub-parameter which sets the number of
iterations of clean before stopping.
\small
\begin{verbatim}
interactive      =       True   #  use interactive clean (with GUI viewer)
     npercycle   =        100   #  Number of iterations before interactive prompt

\end{verbatim}
\normalsize
{\bf ALERT}: {\tt npercycle} is currently the only way to
control the breakpoints in interactive clean.

See the example of interactive cleaning in 
\S~\ref{section:im.clean.exampleinteractive}.

%%%%%%%%%%%%%%%%%%%%%%%%%%%%%%%%%%%%%%%%%%%%%%%%%%%%%%%%%%%%%%%%%
\subsection{Parameter {\tt mask} }
\label{section:im.clean.mask}

The {\tt mask} parameter takes a list of elements, each of which can
be a list of coordinates specifying a box,
or a string pointing to the name of a cleanbox file, mask image, or
region file.  These are used by CLEAN to define a region to search for components.  

Note that for {\tt imagermode=''} (\S~\ref{section:im.clean.imagermode}) 
the default with {\tt mask=''} is to restrict {\tt clean} to the inner
quarter of the image.

%%%%%%
\subsubsection{Setting clean boxes }
\label{section:im.clean.mask.box}

If {\tt mask} is given a list, these are taken to be pixel
coordinates for the {\tt blc} and {\tt trc} (bottom-left and top-right
corners) of one or more rectangular boxes.  Each box is a four element
list.  For example,
\small
\begin{verbatim}
  mask = [ [110,110,150,145], [180,70,190,80] ]
\end{verbatim}
\normalsize
defines two boxes.

%%%%%%
\subsubsection{Using clean box files}
\label{section:im.clean.mask.boxfiles}

You can provide {\tt mask} a string with the name of  
an ASCII file containing the BLC, TRC of the boxes with one
box per line.  Each line should contain five numbers
\small
\begin{verbatim}
  <fieldindex> <blc-x> <blc-y> <trc-x> <trc-y>
\end{verbatim}
\normalsize
with whitespace separators.  Currently the {\tt <fieldindex>} is 
ignored.

Here is an example cleanbox file:
\small
\begin{verbatim}
CASA <21>: !cat mycleanbox.txt
IPython system call: cat mycleanbox.txt
1 108 108 148 148
2 160 160 180 180
\end{verbatim}
\normalsize

NOTE: In future patches we will include options for the specification
of circular and polygonal regions in the {\tt cleanbox} file, as
well as the use of world coordinates (not just pixel) and control
of plane ranges for the boxes.  For now, use the {\tt mask} mechanism
for more complicated CLEAN regions.

%%%%%%
\subsubsection{Using clean mask images}
\label{section:im.clean.mask.maskimage}

You can give the {\tt mask} parameter a string containing the
name of a mask image to be used for CLEAN to search for components.  
You can use the {\tt makemask} task
(\S~\ref{section:im.mask.makemask}) to construct this mask, or use 
one made using {\tt interactive=True}
(\S~\ref{section:im.clean.interactive}).

%%%%%%
\subsubsection{Using region files}
\label{section:im.clean.mask.regions}

You can give the {\tt mask} parameter a string pointing to a file
that describes a {\it region}.  This region file can be generated
in the {\tt viewer} (\S~\ref{chapter:display}) or in the {\tt boxit}
task (\S~\ref{section:im.mask.boxit}) from an image.

%%%%%%%%%%%%%%%%%%%%%%%%%%%%%%%%%%%%%%%%%%%%%%%%%%%%%%%%%%%%%%%%%
\subsection{Parameter {\tt minpb} }
\label{section:im.clean.minpb}

The {\tt minpb} parameter sets the level down to which the primary
beam (or more correctly the voltage patterns in the array) can go and
have a given pixel included in the image.  This is important as it
defines where the edge of the visible image or mosaic is. 
The default is $0.1$ or
equivalent to the 10\% response level.  If there is alot of emission
near the edge, then set this lower if you want to be
able to clean it out.

NOTE: The {\tt minpb} parameter is the level in the ``primary beam''
(PB) at which the cut is made.  If you are using {\tt ftmachine='mosaic'}
(\S~\ref{section:im.imagermode.mosaic.ftmachine}), this will show
up in the {\tt .flux.pbcoverage} image (new in version 2.4.0).
See the discussion of {\tt threshold} 
(\S~\ref{section:im.clean.imagermode.threshold}) for related issues.

%%%%%%%%%%%%%%%%%%%%%%%%%%%%%%%%%%%%%%%%%%%%%%%%%%%%%%%%%%%%%%%%%
\subsection{Parameter {\tt modelimage} }
\label{section:im.clean.modelimage}

The {\tt modelimage} parameter specifies the name(s) of one or more
input starting image(s) to use to calculate the first residual before
cleaning.  These are used in addition to any image with a name
defaulting from the {\tt imagename} root (e.g. on a restart).
The output model will contain this model plus
clean components found during deconvolution.  

If the units of the image are {\tt Jy/pixel}, then this is treated
as a model image.

If the units of the image are {\tt Jy/beam} or Jy per solid angle,
then this is treated as a ``single-dish'' image and rescaled by
the resolution (in the {\tt 'beam'} image header keyword). Inclusion
of the SD image here is superior to feathering it in later.
See \S~\ref{section:im.feather} for more information on feathering.

%%%%%%%%%%%%%%%%%%%%%%%%%%%%%%%%%%%%%%%%%%%%%%%%%%%%%%%%%%%%%%%%%
\subsection{Parameter {\tt niter} }
\label{section:im.clean.niter}

The {\tt niter} parameter sets the maximum total number of minor-cycle CLEAN
iterations to be performed during this run of {\tt clean}.  If restarting
from a previous state, it will carry on from where it was.  Note that
the {\tt threshold} parameter can cause the CLEAN to be terminated before
the requested number of iterations is reached.

%%%%%%%%%%%%%%%%%%%%%%%%%%%%%%%%%%%%%%%%%%%%%%%%%%%%%%%%%%%%%%%%%
\subsection{Parameter {\tt pbcor} }
\label{section:im.clean.pbcor}

The {\tt pbcor} parameter controls whether the final {\tt .image} 
is scaled to correct for the Primary Beam of the array or not.

If {\tt pbcor=False} (the default), then no such scaling is done
and the image is in whatever ``raw'' scaling used by the 
{\tt imagermode} algorithm underneath.  For single-field cleaning
with {\tt imagermode=''} or {\tt 'csclean'}, this is the standard
constant-noise image.  If {\tt imagermode='mosaic'}, then this is
the {\tt 'SAULT'} scaled image (regardless of what {\tt scaletype}
is set to).

If {\tt pbcor=True}, the at the end of deconvolution and imaging the
``raw'' image is rescaled by dividing by the noise and PB correction
image.  This is what is output by {\tt clean} as the {\tt .flux}
image.  

Note that regardless of what you set {\tt pbcor} to, you can recover
the other option using {\tt immath} (\S~\ref{section:analysis.immath})
to either multiply or divide by the {\tt .flux} image.

%%%%%%%%%%%%%%%%%%%%%%%%%%%%%%%%%%%%%%%%%%%%%%%%%%%%%%%%%%%%%%%%%
\subsection{Parameter {\tt restoringbeam} }
\label{section:im.clean.restoringbeam}

The {\tt restoringbeam} parameter allows the user to set a specific
Gaussian restoring beam to make the final restored {\tt .image} from
the final {\tt .model} and residuals.

If {\tt restoringbeam=''} (the default), then the restoring beam
is calculated by fitting to the PSF (e.g. the {\tt .psf} image).
For a mosaic, this is at the center of the field closest to the
{\tt phasecenter}.

To specify a restoring beam, provide {\tt restoringbeam} a list of 
{\tt [bmaj, bmin, bpa]} which are the parameters of an elliptical
Gaussian.  The default units are in arc-seconds for {\tt bmaj, bmin}
components and degrees for the {\tt bpa} component.

For example,
\small
\begin{verbatim}
   restoringbeam=['10arcsec']                  # circular Gaussian FWHM 10"
   restoringbeam=['10.0','5.0','45.0deg']      # 10"x5" at PA=45 degrees
\end{verbatim}
\normalsize

%%%%%%%%%%%%%%%%%%%%%%%%%%%%%%%%%%%%%%%%%%%%%%%%%%%%%%%%%%%%%%%%%
\subsection{Parameter {\tt threshold} }
\label{section:im.clean.thresh}

The {\tt threshold} parameter instructs {\tt clean} to terminate when
the maximum (absolute?) residual reaches this level or below.  Note
that it may not reach this residual level due to the value of the
{\tt niter} parameter which may cause it to terminate early.

If {\tt threshold} is given a floating-point number, then this is the
threshold in milli-Jansky.

You can also supply a flux density {\em quanta} to threshold, e.g.
\small
\begin{verbatim}
   threshold = '8.0mJy'
   threshold = '0.008Jy'
\end{verbatim}
\normalsize
(these do the same thing).

%%%%%%%%%%%%%%%%%%%%%%%%%%%%%%%%%%%%%%%%%%%%%%%%%%%%%%%%%%%%%%%%%
\subsection{Interactive Cleaning --- Example}
\label{section:im.clean.exampleinteractive}

If {\tt interactive=True} is set, then an interactive window
will appear at various ``cycle'' stages while you clean, so you
can set and change mask regions.  These breakpoints are controlled
by the {\tt npercycle} sub-parameter which sets the number of
iterations of clean before stopping.

\begin{figure}[h!]
\begin{center}
\pngname{clean_jupiter_inter_0}{3}
\pngname{clean_jupiter_inter_1}{3}
\caption{\label{fig:clean_inter1} Screen-shots of the interactive 
{\tt clean} window during deconvolution of the VLA 6m Jupiter
dataset.  We start from the calibrated data, but before any 
self-calibration. In the initial stage (left), the window pops up 
and you can see it dominated by a bright source in the center.
Next (right), we zoom in and draw a box around this emission.
We have also at this stage dismissed the tape deck and Position
Tracking parts of the display 
(\S~\ref{section:display.viewerGUI.displaypanel}) 
as they are not used here.  We have also changed the 
{\tt iterations} to 30 for this boxed clean.
We will now hit the Next Action {\bf Continue Cleaning} button 
(the green clockwise arrow) to start cleaning. } 
\hrulefill
\end{center}
\end{figure}

The window controls are fairly self-explanatory.  It is basically a
form of the {\tt viewer}.  A close-up of the controls are shown in
Figure~\ref{fig:clean_inter_control}, and an example can be found in 
Figures~\ref{fig:clean_inter1}--\ref{fig:clean_inter3}.  
You assign one of the drawing functions
(rectangle or polygon, default is rectangle) to the right-mouse button
(usually), then use it to mark out regions on the image.  Zoom in if
necessary (standard with the left-mouse button assignment).
Double-click inside the marked region to add it to the mask.  If you
want to reduce the mask, click the {\bf Erase} radio button (rather
than {\bf Add}), then mark and select as normal.  
When finished setting or changing your mask, click the green
clockwise arrow ``Continue Cleaning'' Next Action button.  
If you want to finish your clean with
no more changes to the mask, hit the blue right arrow 
``Apply mask edits and proceed with non-interactive clean''
button.  If you want to terminate the clean, click the red X
``Stop deconvolving now'' button.

While stopped in an interactive step, you can change a number of
control parameters in the boxes provided at the left of the menu
bar.  The main use of this is
to control how many {\tt iterations} before the next breakpoint
(initially set to {\tt npercycle}), how many cycles before completion
(initially equal to {\tt niter}/{\tt npercycle}), and to
change the {\tt threshold} for ending cleaning.  
Typically, the user would start
with a relatively small number of iterations (50 or 100) to clean the
bright emission in tight mask regions, and then increase this as you
get deeper and the masking covers more of the emission region.  For
extended sources, you may end up needing to clean a large number
of components (10000 or more) and thus it is useful to set {\tt niter}
to a large number to begin with --- you can always terminate the 
clean interactively when you think it is done.  Note that if you
change {\tt iterations} you may also want to change {\tt cycles} or
your clean may terminate before you expect it to.

\begin{figure}[h!]
\begin{center}
\pngname{clean_jupiter_inter_2}{3}
\pngname{clean_jupiter_inter_3}{3}
\caption{\label{fig:clean_inter2} We continue in our interactive 
{\tt clean}ing of Jupiter from where Figure~\ref{fig:clean_inter1}
left off.  In the first (left) panel, we have cleaned 30 iterations
in the region previously marked, and are zoomed in again ready to
extend the mask to pick up the newly revealed emission. 
Next (right), we have used the Polygon
tool to redraw the mask around the emission, and are ready to 
{\bf Continue Cleaning} for another 100 iterations.} 
\hrulefill
\end{center}
\end{figure}

For strangely shaped emission regions, you may find using the polygon
region marking tool (the second from the right in the button
assignment toolbar) the most useful.

See the example of cleaning and self-calibrating the Jupiter 6cm continuum
data given below in Appendix~\ref{section:scripts.jupiter}.  
The sequence of cleaning starting with the ``raw'' externally
calibrated data is shown in
Figures~\ref{fig:clean_inter1} -- \ref{fig:clean_inter3}.

\begin{figure}[h!]
\begin{center}
\pngname{clean_jupiter_inter_4}{3}
\pngname{clean_jupiter_inter_5}{3}
\caption{\label{fig:clean_inter3} We continue in our interactive 
{\tt clean}ing of Jupiter from where Figure~\ref{fig:clean_inter2}
left off.  In the first (left) panel,  it has cleaned deeper, and
we come back and zoom in to see that our current mask is good and
we should clean further.  We change {\tt npercycle} to 500 (from 100)
in the box at upper right of the window.  In the final panel (right),
we see the results after this clean.  The residuals are such that we
should terminate the {\tt clean} using the red X button 
and use our model for self-calibration.} 
\hrulefill
\end{center}
\end{figure}

The final result of all this cleaning for Jupiter is shown in
Figure~\ref{fig:clean_polfinal}.  The {\tt viewer} 
(\S~\ref{chapter:display}) was used to overplot the polarized
intensity contours and linear polarization vectors calculated using
{\tt immath} (\S~\ref{section:analysis.immath}) on the total
intensity.  See the following chapters on how to make the most of
your imaging results.

\begin{figure}[h!]
\begin{center}
\pngname{clean_jupiter_viewpol}{6}
\caption{\label{fig:clean_polfinal} 
After clean and self-calibration using the intensity image, we 
arrive at the final polarization image of Jupiter.  Shown in
the {\tt viewer} superimposed on the intensity raster is the
linear polarization intensity (green contours) and linear polarization
B-vectors (vectors).  The color of the contours and the sampling and
rotation by 90 degrees of the vectors was set in the Display Options
panel.  A LEL expression was used in the Load Data panel to mask the
vectors on the polarized intensity.}
\hrulefill
\end{center}
\end{figure}

For spectral cube images you can use the tapedeck to move through the
channels.  You also use the panel with radio buttons for choosing
whether the mask you draw applies to the
{\tt Diplayed Plane} or to {\tt All Channels}.
See Figure~\ref{fig:clean_n5921_inter} for an example.  Note that
currently the {\tt Displayed Plane} option is set by default.  
This toggle is unimportant for single-channel images or {\tt mode='mfs'}.

\begin{figure}[h!]
\begin{center}
\pngname{clean_n5921_inter}{4.5}
\caption{\label{fig:clean_n5921_inter} Screen-shot of the interactive 
{\tt clean} window during deconvolution of the NGC5921 spectral
line dataset.  Note where we have selected
the mask to apply to the {\tt Displayed Plane} rather than
{\tt All Channels}.  We have just used the
Polygon tool to draw a mask region around the emission in this
channel, which will apply to this channel only. } 
\hrulefill
\end{center}
\end{figure}

{\bf Advanced Tip:} Note that while in interactive {\tt clean}, you
are using the {\tt viewer}. Thus, you have the ability to open and
register other images in order to help you set up the clean mask.
For example, if you have a previously cleaned image of a complex
source or mosaic that you wish to use to guide the placement of boxes
or polygons, just use the {\bf Open} button or menu item to bring in
that image, which will be visible and registered on top of your dirty
residual image that you are cleaning on.  You can then draw masks as
usual, which will be stored in the mask layer as before.  Note you can
blink between the new and dirty image, change the colormap and/or
contrast, and carry out other standard viewer operations.  See
\S~\ref{chapter:display} for more on the use of the {\tt viewer}.

{\bf ALERT:} Currently, interactive spectral line cleaning
is done globally over the cube, with halts for interaction after
searching all channels for the requested {\tt npercycle} total
iterations.  It is more convenient for the user to treat the
channels in order, cleaning each in turn before moving on.  This
will be implemented in an upcoming update.

%%%%%%%%%%%%%%%%%%%%%%%%%%%%%%%%%%%%%%%%%%%%%%%%%%%%%%%%%%%%%%%%%
\subsection{Mosaic imaging}
\label{section:im.clean.mosaic}

The {\tt clean} task contains the capability to image multiple
pointing centers together into a single ``mosaic'' image.  This
ability is controlled by setting {\tt imagermode='mosaic'}
(\S~\ref{section:im.clean.imagermode}).

The key parameter that controls how {\tt clean} produces the mosaic
is the {\tt ftmachine} sub-parameter 
(\S~\ref{section:im.imagermode.mosaic.ftmachine}).  For 
{\tt ftmachine='ft'}, {\tt clean} will perform a weighted combination
of the images produced by transforming each mosaic pointing
separately.  This can be slow, as the individual sub-images must be
recombined in the image plane.  {\bf NOTE:} this option is preferred
for data taken with sub-optimal mosaic sampling (e.g. fields too 
far apart, on a sparse irregular pattern, etc.).

The primary beams used in CASA are described in \S~\ref{section:im.pars.pb}.

If {\tt ftmachine='mosaic'}, then the data are gridded onto a single
uv-plane which is then transformed to produce the single output 
image.  This is accomplished by using a gridding kernel that 
approximates the transform of the primary beam pattern.  Note that
for this mode the {\tt <imagename>.flux} image includes this
convolution kernel in its effective weighted response pattern (needed
to ``primary-beam correct'' the output image).  For this mode only,
an additional image {\tt <imagename>.flux.pbcoverage} is produced that
is the primary-beam coverage only used to compute the {\tt minpb}
cutoff (\S~\ref{section:im.clean.minpb}).

{\bf ALERT:} In order to avoid aliasing artifacts for
{\tt ftmachine='mosaic'} in the mosaic image, due to the discrete
sampling of the mosaic pattern on the sky, you should make an image
in which the desired unmasked part of the image (above {\tt minpb}) 
lies within the inner quarter.  In other words, make an image twice
as big as necessary to encompass the mosaic.

It is also important to choose an appropriate {\tt phasecenter} for
your output mosaic image (\S~{section:im.pars.phasecenter}).

An example of a simple mosaic {\tt clean} call is shown below: 
\small
\begin{verbatim}
clean(vis='n4826_tboth.ms',        
       imagename='tmosaic',         
       mode='channel',
       nchan=30,start=46,           # Make the output cube 30 chan
       width=4,                     # start with 46 of spw 0, avg by 4 chans
       spw='0~2',
       field='0~6',
       cell=[1.,1.],
       imsize=[256,256],
       stokes='I',
       psfmode='clark',
       niter=500,
       imagermode='mosaic',
       scaletype='SAULT',
       cyclefactor=0.1)
\end{verbatim}
\normalsize

Another example of mosaic imaging, this time using 3mm BIMA data,
is given in Appendix~\ref{section:scripts.ngc4826}.

%%%%%%%%%%%%%%%%%%%%%%%%%%%%%%%%%%%%%%%%%%%%%%%%%%%%%%%%%%%%%%%%%
\subsection{Heterogeneous imaging}
\label{section:im.clean.hetero}

The {\tt clean} task and underlying tools can now handle cases where
there are multiple dish sizes, and thus voltage patterns and primary
beams, in the array.  This is effected by using the dish sizes stored
in the {\tt ANTENNA} sub-table of the MS.  Depending on how the data
was written and imported into CASA, the user may have to manually edit
this table to insert the correct dish sizes (e.g. using {\tt
browsetable} or the {\tt tb} table tool).  

{\bf ALERT:} This feature is new in Patch 3, and has not been
extensively tested.  Currently, this works only for an MS where the
{\tt OBSERVATORY} keyword is CARMA, ALMA, or is unknown.  You must
set {\tt imagermode='mosaic'} with {\tt ftmachine='mosaic'}, even
when imaging a single field, to use this feature.
This will be improved, and made easier to set and use, in future
releases.  

% \subsection{The {\tt entropy} algorithm}
% \label{section:im.clean.entropy}
% 
% There is an option to use the Maximum Entropy deconvolution algorithm
% instead of CLEAN.  The MEM implementation here is fairly basic, and
% does not include advanced convergence criteria or other improvements.
% 
% The sub-parameters are:
% \small
% \begin{verbatim}
% alg                 =  'entropy' # deconvolution algorithm: clark, hogbom, multiscale, entropy
%      sigma          =   '0.01Jy' # Image sigma to try to achieve
%      targetflux     =    '1.0Jy' # Target flux for final image
%      constrainflux  =      False # Constrain image to match target flux
%      prior          =       [''] # Name of MEM prior images
% \end{verbatim}
% \normalsize
% 
% The standard usage is to supply MEM with an estimate of the total flux
% in the image ({\tt targetflux}), an estimated noise level ({\tt sigma}),
% and perhaps a ``prior'' model image ({\tt prior}).  For example,
% the {\tt targetflux} can be measured using autocorrelations, or more
% commonly guessed using {\tt plotxy} with {\tt xaxis='uvdist'} and
% {\tt yaxis='amp'}.  The {\tt sigma} parameter will control the convergence.
% 
% {\bf ALERT:}  This is a basic version of MEM, and also an
% experimental implementation for mosaicing.  Beware.
% 

%%%%%%%%%%%%%%%%%%%%%%%%%%%%%%%%%%%%%%%%%%%%%%%%%%%%%%%%%%%%%%%%%
\subsection{Polarization imaging}
\label{section:im.clean.polarization}

The {\tt clean} task handles full and partial Stokes polarization
imaging through the setting of the {\tt stokes} parameter 
(\S~\ref{section:im.pars.stokes}).  The subsequent deconvolution of
the polarization planes of the image and the search for clean
components is controlled by the {\tt psfmode} parameter
(\S~\ref{section:im.clean.psfmode}).  If the {\tt stokes} parameter
includes polarization planes other than I, then choosing {\tt psfmode='hogbom'} 
(\S~\ref{section:im.clean.psfmode.hogbom}) or {\tt psfmode='clarkstokes'} 
(\S~\ref{section:im.clean.psfmode.clarkstokes}) will clean (search for
components) each plane sequentially, while {\tt psfmode='clark'} 
(\S~\ref{section:im.clean.psfmode.clark}) will deconvolve jointly.

The interactive {\tt clean} example given above 
(\S~\ref{section:im.clean.exampleinteractive}) shows
a case of polarization imaging.  You can also find the script for
this example in Appendix~\ref{section:scripts.jupiter}.

%%%%%%%%%%%%%%%%%%%%%%%%%%%%%%%%%%%%%%%%%%%%%%%%%%%%%%%%%%%%%%%%%
\subsection{Wide-field imaging and deconvolution in {\tt clean}}
\label{section:im.clean.widefield}

When imaging sufficiently large angular regions, the sky can no longer
be treated as a two-dimensional plane and the use of the standard {\tt clean}
task will produce distortions around sources that become
increasingly severe with increasing distance from the phase center.
In this case, one must use a ``wide-field'' imaging algorithm such as
w-projection or faceting.

When is wide-field imaging needed?  It depends on the expected dynamic
range the image.  In order to keep the phase error associated with the
sky/array curvature less than about $5^\circ$ (good to about 500:1
dynamic range), use the following table, suitably scaled, for
guidance:
\small
\begin{verbatim}
        Maximum Radius of Image Before widefield is Needed
          Assuming 5 deg phase error and 35-km Baseline

           Wavelength                 Radius of image
              6 cm                       1.4 arcmin
             20 cm                       2.6 arcmin
             90 cm                       5.3 arcmin

  Radius of image ~ SQRT (Wavelength * phase error / Maximum baseline)
      (arcmin)              (cm)          (deg)           (km)
\end{verbatim}
\normalsize
If a relatively small image is being made, but there are outliers
sources beyond the above limits, then widefield should also be used.

The relevant inputs for {\tt clean} for wide-field imaging are:
\small
\begin{verbatim}
gridmode            = 'widefield'       #  The kind gridding kernel to be used for FFT-based transforms
     wprojplanes    =          1        #  Number of w-projection planes for convolution
     facets         =          1        #  Number of facets along each axis (main image only)
\end{verbatim}
\normalsize
Most of the {\tt clean} parameters behave as described previously.

Wide-field imaging can be carried out using two major modes: First, the
w-projection mode as chosen with {\tt ftmachine} deals with the w-term
(the phase associated with the sky/array curvature) internally.  Secondly, the image can be
broken into many facets, each small enough so that the w-term is not
significant.  These two basic methods can be combined, as discussed
below in \S~\ref{section:im.clean.widefield.combo}.

%%%%%%%%%%%%%%
\subsubsection{Outlier fields}
\label{section:im.clean.widefield.outliers}

When using wide-field imaging, the position and image size of any 
independent images must be specified.  This can be done in two ways.  The first
method uses input from a text file {\tt outlierfile}.  For example:
\small
\begin {verbatim}
vis           = 'wfield.ms'   #  name of input visibility file
imagename     = 'wf'          #  Pre-name of output images
outlierfile   = 'setup.txt'   #  Text file with image names, sizes, centers
imsize        = [256, 256]    #  Image size in pixels (nx,ny), symmetric for single value
cell          = '1.0arcsec'   #  The image cell size in arcseconds [x,y].
phasecenter   =         ''    #  Field Identififier or direction of the image phase center
\end{verbatim}
\normalsize
The {\tt cell} is identical for all fields, and {\tt imsize}
and {\tt phasecenter} are not used.  
The output image names will be {\tt wf.main}, {\tt wf.out1}, etc.

The file {\tt outlierfile="setup.txt"} is an AIPS-style field file
used to locate outlier fields.  For example:
\small
\begin {verbatim}
C   main 2048 2048   13 27 20.98     43 26 28.0   # Main field with image size and phase center
C   out1  128  128   13 30 52.158    43 23 08.00  # First outlier field specification
                      etc
\end{verbatim}
\normalsize
will make a big main image with outliers.
The {\tt C} in column 1 must be present, although it not presently used.

The alternative input, needing no additional outlier file, versus {\tt imsize} plus {\tt
phasecenter}: is: 
\small
\begin{verbatim}
vis           = 'wfield.ms'       #  name of input visibility file
imagename     = 'wf'              #  Pre-name of output images
outlierfile   = 'image_setup.txt' #  Text file with image names, sizes, centers
imsize        = [2048, 2048, 128, 128] #  Image size in pixels (nx,ny)
cell          = '1.0arcsec'       #  The image cell size in arcseconds [x,y].
phasecenter   = ['J2000 13h27m20.98 43d26m28.0', 'J2000 13h30m52.158 43d23m08.00']
\end{verbatim}
\normalsize

The other parameters are described in {\tt clean}, and multi-scale
cleaning in available.  Mosaicing and MEM is not yet included in {\tt
widefield}.

%%%%%%%%%%%%%%
\subsubsection{Setting up w-projection}
\label{section:im.clean.widefield.wproj}

The w-projection mode is controlled using {\tt wprojplanes}
sub-parameter, e.g. 
\small
\begin{verbatim}
gridmode            = 'widefield'       #  The kind gridding kernel to be used for FFT-based transforms
     wprojplanes    =         64        #  Number of w-projection planes for convolution
     facets         =          1        #  Number of facets along each axis (main image only)
\end{verbatim}
\normalsize
will construct 64 w-projection planes.

A reasonable value for {\tt wprojplanes} is equal to $n_w = B_{max}({\rm in k}\lambda)
\times {\rm imagewidth}({\rm in arcmin}^2) / 600$, with a minimum value of $n_w=16$.  The
w-projection algorithm is much faster than using faceting, but it does
consume a lot of memory.  On most 32-bit machines with 1 or 2 Mbytes
of memory, images larger than about $4000\times 4000$ cannot be made.

%%%%%%%%%%%%%%
\subsubsection{Setting up faceting}
\label{section:im.clean.widefieldfacet}

Faceting will break the image into many small parts.  This
is invoked using {\tt facets}:
\small
\begin{verbatim}
gridmode            = 'widefield'       #  The kind gridding kernel to be used for FFT-based transforms
     wprojplanes    =          1        #  Number of w-projection planes for convolution
     facets         =          7        #  Number of facets along each axis (main image only)
\end{verbatim}
\normalsize
In this example the image is broken into 49 ($7\times7$) facets.

A reasonable value of facets is such that the image width of each facet
does not need the w-term correction.  The computation method with pure
faceting is slow, so that w-projection is recommended

%%%%%%%%%%%%%%
\subsubsection{Combination of w-projection and faceting}
\label{section:im.clean.widefield.combo}

You can also use a combination of w-projection and faceting:
\small
\begin{verbatim}
gridmode            = 'widefield'       #  The kind gridding kernel to be used for FFT-based transforms
     wprojplanes    =         32        #  Number of w-projection planes for convolution
     facets         =          3        #  Number of facets along each axis (main image only)
\end{verbatim}
\normalsize
This hybrid method allows for a smaller number of {\tt wprojplanes} in order
to try to conserve memory if the image size approached the memory limit
of the computer.  However, there is a large penalty in execution time.

%%%%%%%%%%%%%%%%%%%%%%%%%%%%%%%%%%%%%%%%%%%%%%%%%%%%%%%%%%%%%%%%%
%%%%%%%%%%%%%%%%%%%%%%%%%%%%%%%%%%%%%%%%%%%%%%%%%%%%%%%%%%%%%%%%%
\section{Combined Single Dish and Interferometric Imaging 
         ({\tt feather})}
\label{section:im.feather}

The term ``feathering'' is used in radio imaging to describe how to
combine or ``feather'' two images together by forming a weighted
sum of their Fourier transforms in the (gridded) uv-plane.
Intermediate size scales are down-weighted to give interferometer
resolution while preserving single-dish total flux density.

The feathering technique does the following:
\begin{enumerate}
\item The single-dish and interferometer images are Fourier
      transformed.
\item The beam from the single-dish image is Fourier transformed
      ($FTSDB(u,v)$).
\item The Fourier transform of the interferometer image is multiplied
      by ($1-FTSDB(u,v)$).  This basically down weights the shorter
      spacing data from the interferometer image.
\item The Fourier transform of the single-dish image is scaled by the
      volume ratio of the interferometer restoring beam to the single
      dish beam.
\item The results from 3 and 4 are added and Fourier transformed
      back to the image plane.
\end{enumerate}

\begin{wrapfigure}{r}{2.5in}
  \begin{boxedminipage}{2.5in}
     \centerline{\bf Other Packages:}
     The {\tt feather} task is analogous to the AIPS {\tt IMERG} 
     task and the MIRIAD {\tt immerge} task with option 
     {\tt 'feather'}.
  \end{boxedminipage}
\end{wrapfigure}
The term feathering derives from the tapering or down-weighting of the
data in this technique; the overlapping, shorter spacing data from the
deconvolved interferometer image is weighted down compared to the
single dish image while the overlapping, longer spacing data from the
single-dish are weighted down compared to the interferometer image.

The tapering uses the transform of the low resolution point spread
function. This can be specified as an input image or the appropriate
telescope beam for the single-dish.  The point spread function for a
single dish image may also be calculated using {\tt clean}.

Advice: Note that if you are feathering large images, be advised to
have the number of pixels along the X and Y axes to be composite
numbers and definitely not prime numbers. In general FFTs work much
faster on even and composite numbers. You may use subimage function of
the image tool to trim the number of pixels to something desirable.

The inputs for {\tt feather} are:
\small
\begin{verbatim}
imagename   =         ''   #   Name of output feathered image
highres     =         ''   #   Name of high resolution (synthesis) image
lowres      =         ''   #   Name of low resolution (single dish) image
async       =      False   #  If true the taskname must be started using feather(...)
\end{verbatim}
\normalsize
Note that the only inputs are for images.
Note that {\tt feather} does not do any deconvolution but combines presumably
deconvolved images after the fact.

Starting with a cleaned synthesis image and a low resolution image
from a single dish telescope, the following example shows how they
can be feathered:
\small
\begin{verbatim}
feather(imagename='feather.im',      # Create an image called feather.im
       highres='synth.im',           # The synthesis image is called synth.im
        owres='single_dish.im'       # The SD image is called single_dish.im
       )
\end{verbatim}
\normalsize

Note that the single dish image must have a
well-defined beam shape and the correct flux units for a model image
(Jy/beam instead of Jy/pixel) so use the tasks {\tt
imhead} and {\tt immath} first to convert if needed.

%%%%%%%%%%%%%%%%%%%%%%%%%%%%%%%%%%%%%%%%%%%%%%%%%%%%%%%%%%%%%%%%%
\section{Making Deconvolution Masks or Box Regions}
\label{section:im.mask}

For most careful imaging, you will want to restrict the region over
which you allow CLEAN components to be found.  To do this, you can
create a 'deconvolution region' or 'mask' image using the {\tt boxit}
or {\tt makemask} tasks.  Note that {\tt clean} can take simple boxes
or box files as direct input to its {\tt mask} parameter, so these
tasks are most useful when direct input to clean (or use of
interactive clean) will not suffice.

There are two ways to construct region files or mask images for use
in deconvolution.  The {\tt boxit} task will find a set of box regions
based upon an input image and control parameters.  The {\tt makemask}
task will construct a mask image from a box specfication or a box
file.  

%%%%%%%%%%%
\subsection{Making Deconvolution Regions from an Image ({\tt boxit})}
\label{section:im.mask.boxit}

%%%XXX begin Amy modifications XXX%%%
The {\tt boxit} task creates ``cleanbox'' deconvolution regions automatically
from an image.  It searches the image to find ``islands'': all contiguous
sets of pixels above the given threshold.  The extreme x- and y-pixels of the
island are used to determine the corners of a rectangular box that covers each
island.  The set of boxes are written out into a single region file with
extension {\tt .rgn}.  {\tt Boxit} works on single-plane images as well as
multi-channel images: in the latter case, the thresholding and boxing is done
separately in each plane of the image.  The output region file from {\tt boxit}
can be used as the {\tt mask} input parameter for the {\tt clean} task
(\S~\ref{section:im.clean}).
%%%XXX end Amy modifications XXX%%%

The parameter inputs for {\tt boxit} are:
\small
\begin{verbatim}
#  boxit :: Box regions in image above given threshold value.
imagename    =         ''   #  Name of image to threshold
regionfile   =         ''   #  Output region file
threshold    =   '0.0mJy'   #  Threshold value.  Must include units.
minsize      =          2   #  Minimum number of pixels for a boxable island
diag         =      False   #  Count diagonal connections?
boxstretch   =          1   #  Increase box sizes by this many pixels beyond thresholded pixels.
overwrite    =      False   #  Overwrite existing region file?
async        =      False   #  If true the taskname must be started using boxit(...)
\end{verbatim}
\normalsize 

%%%XXX begin Amy modifications XXX%%%
%XX: Some of the parameters are standard imaging parameters
%(\S~\ref{section:im.pars}).
The {\tt regionfile} parameter specifies the root name of the region file.  It
will automatically be given {\tt .rgn} as the file extension.  
The {\tt minsize} parameter specifies the smallest island that qualifies to be
boxed.  It refers to the total number of pixels in the island. 
To include pixels connected only on the diagonal as being part of the same
island, set the {\tt diag} parameter to {\tt True}.  The {\tt boxstretch}
parameter increases the size of the boxes beyond the extent of the island, and
can range from -1 to 5.  For a value of 1 (the default), the
box is stretched by one pixel in each outward direction; therefore, each side
of the box lengthens by two pixels.  Finally, the parameter {\tt overwrite}
specifies whether an existing region file can be overwritten.
%%%XXX end Amy modifications XXX%%%

{\bf ALERT:} The {\tt boxit} task is a prototype under active development and
coded in Python. Eventually we will add functionality to deal with the creation of
non-rectangular regions and with multi-plane masks, as well as efficiency
improvements.  %%%XXX Amy modification; added missing word "as" to above
%%%sentence before "efficiency" XXX%%%

%%%%%%%%%%%
\subsection{Making Deconvolution Masks from Boxes ({\tt makemask})}
\label{section:im.mask.makemask}

The parameter inputs for {\tt makemask} are:
\small
\begin{verbatim}
#  makemask :: Derive a mask image from a cleanbox and set of imaging parameters:

cleanbox      =         []   #   Clean box file or regions
vis           =         ''   #   Name of input visibility file (if no input image)
imagename     =         ''   #   Name of output mask images
mode          =      'mfs'   #   Type of selection (mfs, channel, velocity)
imsize        = [256, 256]   #   Image size in spatial pixels [x,y]
cell          =     [1, 1]   #   Cell size in arcseconds
phasecenter   =         ''   #   Field identifier or direction of the phase center
stokes        =        'I'   #   Stokes parameter to image (I,IV,IQU,IQUV)
field         =        '0'   #   Field ids list to use in mosaic
spw           =        '0'   #   Spectral window identifier (0-based)
\end{verbatim}
\normalsize 
The majority of the parameters are the standard imaging parameters
(\S~\ref{section:im.pars}).  The {\tt cleanbox} parameter uses the
syntax for cleanboxes as in the {\tt clean} parameter {\tt mask} (see
\S~\ref{section:im.clean.mask}), with the option for {\tt 'interactive'}
also.  The {\tt imagename} parameter specifies the name
for the output mask image.

You can use the {\tt viewer} to figure out the cleanbox blc-trc x-y
settings, make the mask image, and then bring it into the viewer as a
contour image over your deconvolved image to compare exactly where
your mask regions are relative to the actual emission.  In this
example, create a mask from many cleanbox regions specified in a file
on disk ({\tt cleanboxes.txt}) containing
\small
\begin{verbatim}
1 80 80 120 120
2 20 40 24 38
3 70 42 75 66
\end{verbatim}
\normalsize
where each line specifies the field index and the blc x-y and trc x-y
positions of that cleanbox.  For example, in {\tt casapy}, you can do
this easily:
\small
\begin{verbatim}
CASA <29>: !cat > cleanboxes.txt
IPython system call: cat > cleanboxes.txt
1 80 80 120 120
2 20 40 24 38
3 70 42 75 66
<CNTL-D>
CASA <30>: !cat cleanboxes.txt
IPython system call: cat cleanboxes.txt
1 80 80 120 120
2 20 40 24 38
3 70 42 75 66
\end{verbatim}
\normalsize

Then, in CASA,
\small
\begin{verbatim}
makemask(vis='source.ms',
         imagename='source.mask',
	 cleanbox='cleanboxes.txt',
         mode='mfs',                  # make a multi-frequency synthesis map (combine channels)
         imsize=[200,200])            # Set image size = 200x200 pixels
         cell=[0.1,0.1],              # Using 0.1 arcsec pixels
         spw='0,1,2',                 # Combine channels from 3 spectral windows
         field='0',                   # Use the first field in this split dataset
         stokes='I')                  # Image stokes I polarization
\end{verbatim}
\normalsize
This task will then create a mask image that has the 3 cleanboxes
specified in the {\tt cleanboxes.txt} file.  

You can also specify the {\tt cleanbox} as a list (of lists) of
blc,trc pairs (4 veritices), e.g.
\small
\begin{verbatim}
   cleanbox = [[80, 80, 120, 120], [20, 40, 24, 38], [70, 42, 75, 66]]
\end{verbatim}
\normalsize
is equivalent to the {\tt cleanboxes.txt} given above.  Likewise,
\small
\begin{verbatim}
   cleanbox = [80, 80, 120, 120]
\end{verbatim}
\normalsize
puts in a single cleanbox.

Note that you must specify a visibility dataset and create the image
properties so the mask image will have the same dimensions as the
image you want to actually clean.  

{\bf ALERT:} Eventually we will add functionality to deal with the creation of
non-rectangular regions and with multi-plane masks.  There is also no
{\tt cleanbox='interactive'} version currently available.  You have to run
{\tt clean} with {\tt cleanbox='interactive'} to generate a mask graphically.

%%%%%%%%%%%%%%%%%%%%%%%%%%%%%%%%%%%%%%%%%%%%%%%%%%%%%%%%%%%%%%%%%
\section{Transforming an Image Model ({\tt ft})}
\label{section:im.ft}

\begin{wrapfigure}{r}{2.5in}
  \begin{boxedminipage}{2.5in}
     \centerline{\bf Inside the Toolkit:}
     The {\tt im.ft} method does what the {\tt ft} task does.
     Its main use is setting the {\tt MODEL\_DATA} column in the
     MS so that the {\tt cb} tool can use it for subsequent
     calibration.
  \end{boxedminipage}
\end{wrapfigure}

The {\tt ft} task will Fourier transform an image and insert the resulting model into the
{\tt MODEL\_DATA} column of a Measurement Set.
You can also convert a CLEAN component list to a model and insert that
into the {\tt MODEL\_DATA} column.  The MS {\tt MODEL\_DATA} column is
used, for example, to hold the model for calibration purposes in 
the tasks and toolkit.  This is especially useful if you have a resolved
calibrator and you want to start with a model of the source before you
derive accurate gain solutions.  This is also helpful for self-calibration
(see \S~\ref{section:im.selfcal} below).

The inputs for {\tt ft} are:
\small
\begin{verbatim}
vis                 =         ''   #   Name of input visibility file
fieldid             =          0   #   Field index identifier
field               =         ''   #   Field name list
model               =         ''   #   Name of input model image
complist            =         ''   #   Name of component list
incremental         =      False   #   Add to the existing MODEL_DATA column?
\end{verbatim}
\normalsize

An example of how to do this: 
\small
\begin{verbatim}
ft(vis='n75.ms',                   # Start with the visibility dataset n75.ms
   field='1328',                   # Select field name '1328+307' (minimum match) 
   model='1328.model.image')       # Name of the model image you have already
\end{verbatim}
\normalsize
This task will Fourier transform the model image and insert the
resulting model in the {\tt MODEL\_DATA} column of the rows of the MS 
corresponding to the source {\tt 1328+307}.

Note that after {\tt clean}, the transform of the final model is
left in the {\tt MODEL\_DATA} column so you can go directly to
a self-calibration step without explicitly using {\tt ft}.

%%%%%%%%%%%%%%%%%%%%%%%%%%%%%%%%%%%%%%%%%%%%%%%%%%%%%%%%%%%%%%%%%
\section{Image-plane deconvolution ({\tt deconvolve})}
\label{section:im.deconvolve}

If you have only an image (obtained from some telescope) and an image
of its point spread function, then you can attempt a simple image-plane
deconvolution.  Note that for interferometer data, full uv-plane 
deconvolution using {\tt clean} or similar algorithm is superior!

The default inputs for {\tt deconvolve} are:
\small
\begin{verbatim}
#  deconvolve :: Deconvoving a point spread function from an image

imagename   =         '' #   Name of image to decolvolve
model       =         '' #   Name of output image to which deconvolved components are stored
psf         =         '' #   Name of psf or gaussian parameters if psf is assumed gaussian
alg         =    'clark' #   Deconvolution alorithm to use
niter       =         10 #   number of iteration to use in deconvolution process
gain        =        0.1 #   CLEAN gain parameter 
threshold   =    '0.0Jy' #   level below which sources will not be deconvolved
mask        =         '' #   Name of image that has mask to limit region of deconvolution
async       =      False #   if True run in the background, prompt is freed

\end{verbatim}
\normalsize

The algorithm ({\tt alg}) options are: {\tt 'clark'}, {\tt 'hogbom'}, 
{\tt 'multiscale'} or {\tt 'mem'}.  The {\tt 'multiscale'} and
{\tt 'mem'} options will open the usual set of sub-parameters for these
methods.

%%%%%%%%%%%%%%%%%%%%%%%%%%%%%%%%%%%%%%%%%%%%%%%%%%%%%%%%%%%%%%%%%
\section{Self-Calibration}
\label{section:im.selfcal}

Once you have a model image or set of model components reconstructed
from your data using one of the deconvolution techniques described
above, you can use it to refine your calibration.  This is called
{\it self-calibration} as it uses the data to determine its own
calibration (rather than observations of special calibration sources).

In principle, self-calibration is no different than the calibration
process we described earlier (\S~\ref{chapter:cal}).  In effect, you
alternate between calibration and imaging cycles, refining the calibration
and the model as you go.  The trick is you have to be careful, as defects
in early stages of the calibration can get into the model, and thus
prevent the calibration from improving.  In practice, it is best to
not clean very deeply early on, so that the CLEAN model contains
correct components only.  

One important thing to keep in mind is that the self-calibration
relies upon having the most recent Fourier transform of the
model in the {\tt MODEL\_DATA} column of the MS.  This is indeed
the case if you follow the imaging (using {\tt clean})
directly by the self-calibration.  If you have done something strange
in between and have lost or overwritten the {\tt MODEL\_DATA} column
(for example done some extra cleaning that you do not want to keep),
then use the {\tt ft} task (see \S~\ref{section:im.ft} above), which 
fills the {\tt MODEL\_DATA} column with the Fourier transform of the
specified model or model image.  

Likewise, during self-calibration (once you have a new calibration
solution) the imaging part
relies upon having the {\tt CORRECTED\_DATA} column contain
the self-calibrated data.  This is done with the {\tt applycal}
task (\S~\ref{section:cal.correct.apply}).  

The {\tt clearcal} command can be used during the self-calibration if
you need to clear the {\tt CORRECTED\_DATA} column and revert to the
original {\tt DATA}.  If you need to restore the {\tt CORRECTED\_DATA}
to any previous stage in the self-calibration, use {\tt applycal}
again with the appropriate calibration tables.

{\bf ALERT:} In later patches we
will change the tasks so that users need not worry what is contained in
the MS scratch columns and how to fill them.  CASA will handle that 
underneath for you!

For now, we refer the user back to the calibration chapter for
a reminder on how to run the calibration tasks. 

See the example of cleaning and self-calibrating the Jupiter 6cm continuum
data given below in Appendix~\ref{section:scripts.jupiter}.  

%%%%%%%%%%%%%%%%%%%%%%%%%%%%%%%%%%%%%%%%%%%%%%%%%%%%%%%%%%%%%%%%%
\section{Automatic Cleaning}
\label{section:im.autoclean}

%%%XXX begin Amy modifications XXX%%%
{\tt Autoclean} is a task to automatically clean a dataset and create an output
image by automating the ``cleanbox'' selection procedure.  The cleaning and
boxing progresses iteratively.  The majority of the parameters are standard
imaging parameters, with many added parameters that specify the boxing process
and the stop conditions.  Note that, like {\tt clean}
(\S~\ref{section:im.clean}), {\tt autoclean} has the option to run
interactively.  In this case, the clean regions are still determined
automatically, but they can be examined (and modified if necessary) in the {\tt
viewer} after each stage of boxing.
%%%XXX end Amy modifications XXX%%%

The parameter inputs for {\tt autoclean} are:
\small
\begin{verbatim}
#  autoclean :: CLEAN an image with automatically-chosen clean regions.
vis                 =         ''   #  name of input visibility file
imagename           =         ''   #  Pre-name of output images
field               =         ''   #  Field Name
spw                 =         ''   #  Spectral windows:channels: '' is all
selectdata          =      False   #  Other data selection parameters
mode                =      'mfs'   #   Type of selection (mfs, channel, velocity, frequency)
psfmode             =    'clark'   #  method of PSF calculation to use during minor cycles
imagermode          =         ''   #   Use csclean or mosaic.  If '', use psfmode
imsize              = [256, 256]   #  x and y image size in pixels, symmetric for single value
cell                = ['1.0arcsec', '1.0arcsec'] #  x and y cell size. default unit arcsec
phasecenter         =         ''   #  Image phase center: position or field index
restfreq            =         ''   #  rest frequency to assign to image (see help)
stokes              =        'I'   #  Stokes params to image (currently requires Stokes="I")
weighting           =  'natural'   #  Weighting of uv (natural, uniform, briggs, ...)
interactive         =      False   #  use interactive clean (with GUI viewer)
mask                =         []   #  Initial cleanbox(es), mask image(s), and/or region(s)  used in cleaning
modelimage          =         ''   #  Name of model image(s) to initialize cleaning
uvtaper             =      False   #  Apply additional uv tapering of visibilities.
niter               =        500   #  Max number of total clean cycles per channel
npercycle           =        100   #  Clean iterations before adding new clean boxes
npercycle_speedup   =        1.5   #  Increase npercycle by this factor if no new regions found.
gain                =        0.1   #  Loop gain for cleaning
pbcor               =      False   #  Output primary beam-corrected image
minpb               =        0.1   #  Minimum PB level to use
clean_threshold     =   '0.0mJy'   #  Flux level to stop cleaning.  Must include units.
Nrms                =          6   #  Threshold to stop cleaning: current rms * Nrms
eps_maxres          =       0.01   #  Stop cleaning if max abs(residual) changes by less than this factor
useabsresid         =      False   #  If True, use absolute value of residual to test for clean convergence.
allow_maxres_inc    =          3   #  Stop cleaning if max abs(residual) increases this many times (set to -1 to ignore)
island_rms          =          4   #  Threshold for islands when finding clean regions: current rms * island_rms
diag                =      False   #  Count diagonal connection as same island?
peak_rms            =          6   #  Threshold for island peak current rms * peak_rms
gain_threshold      =        0.1   #  Threshold for island peak current max residual * gain_threshold
Npeak               =          3   #  Max number of new clean regions to add in each iteration.
shape               =          1   #  Shape of clean regions: 0 for circles, 1 for boxes, 2 for both.
boxstretch          =          1   #  Increase clean region size by this many pixels beyond island size.
irregsize           =        100   #  For regions this large (in pixels), use irregular clean region.
async               =      False   #  If true the taskname must be started using autoclean(...)
\end{verbatim}
\normalsize 
Many of the parameters are the standard imaging parameters
(\S~\ref{section:im.pars}).  
%%%XXX begin Amy modifications XXX%%%
The new additions in general fall into three categories: clean cycle
parameters, stopping parameters, and boxing parameters.

The new or modified clean cycle parameters are {\tt niter}, {\tt npercycle},
and {\tt npercycle_speedup}.  The {\tt niter} parameter specifies the total
number of clean minor cycles per image channel; {\tt Autoclean} always cleans
on a channel by channel basis.  {\tt Npercycle} specifies the initial number of
minor clean cycles before the next stage of clean boxing.  The {\tt
npercycle_speedup} parameter is a way to increase the value of {\tt npercycle}
if the cleaning is not deep enough.  If no new clean regions are found, the
value of {\tt npercycle} increases by a factor of {\tt npercycle_speedup}.  Set
this parameter to {\tt 1} to always keep the value of {\tt npercycle} constant.

{\tt Clean_threshold}, {\tt Nrms}, {\tt eps_maxres}, {\tt useabsresid}, and
{\tt allow_maxres_inc} are parameters which determine the stopping conditions
for cleaning in each image channel.  Any single one of the following conditions
is sufficient to discontinue cleaning the current image channel.  The {\tt
clean_threshold} parameter is the flux level at which to stop cleaning.  (It is
equivalent to the {\tt threshold} parameter for the {\tt Clean} task.)  The
{\tt Nrms} parameter is used to stop cleaning when the maximum residual is less
than {\tt Nrms} times the rms of the image.  The maximum residual of the image
is checked between each automated clean boxing cycle, and cleaning stops if the
it has changed by a factor of less than {\tt eps_maxres} from one cycle to the
next.  The maximum residual is allowed to increase between clean boxing cycles,
but cleaning stops if it increases more than {\tt allow_maxres_inc} times.
And finally, the parameter {\tt useabsresid}, primarily available for testing
purposes, is set to {\tt False} to use the (positive) maximum residual
for these stopping conditions; set it to {\tt True} to use the absolute value
of the image residuals.

The parameters which determine the progression of the automated clean boxing
are {\tt island_rms}, {\tt diag}, {\tt peak_rms}, {\tt gain_threshold}, {\tt
Npeak}, {\tt shape}, {\tt boxstretch}, and {\tt irregsize}.  The {\tt diag} and
{\tt boxstretch} parameters are equivalent to those used in the {\tt boxit}
task (\S~\ref{section:im.mask.boxit}).  {\tt Island_rms} determines the
threshold level for clean regions; the threshold (equivalent to {\tt boxit}
parameter {\tt threshold}) is {\tt island_rms} times the rms in the current
residual image.  A thresholded island only qualifies for a clean region if its
largest pixel value is at least {\tt peak_rms} times the image rms; this
parameter should be larger than {\tt island_rms}.  The following two parameters
are intended to avoid placing a clean region over a sidelobe of a bright
source.  {\tt Npeak} strictly limits the number of new clean regions added in
each automated boxing cycle, while the {\tt gain_threshold} parameter ensures
that a thresholded island qualifies for a new clean region only if its maximum
pixel is greater than {\tt gain_threshold} times the current maximum residual.

The parameters {\tt shape} and {\tt irregsize} determine the shape of the clean
regions that go into the mask.  Set {\tt shape} to 0 to make only
circular regions, to 1 for only rectangular regions, and to 2 for
either.  If {\tt shape = 2}, clean regions will be circular for islands whose
extent in the x and y directions are within one pixel of each other, and will
be rectangular for all others.  Clean regions will have the exact (irregular)
shape of the island for all islands which are larger than {\tt irregsize}
pixels in extent in either the x or y direction.  To ignore this option, set
{\tt irregsize} to -1 (or to the image size).  Note that {\tt Autoclean} will
produce a {\tt .mask} image as well as a {\tt .rgn} region file.  The mask
image will contain the full set of clean regions.  The region file, owing to
current CASA functionality, will contain only the rectangular regions.
Eventually we will add functionality to deal with the creation of
non-rectangular regions.
%%%XXX end Amy modifications XXX%%%

{\bf ALERT:} The {\tt autoclean} task is a prototype under active development and
coded in Python. Eventually we will integrate this better with the
clean task to improve efficiency.  This task is provided for users who
wish to experiment with this functionality, e.g.\ for pipeline
development.  We welcome feedback on the use of this tasks, in
particular on heuristics used to set the control parameters.

%%%%%%%%%%%%%%%%%%%%%%%%%%%%%%%%%%%%%%%%%%%%%%%%%%%%%%%%%%%%%%%%%
%%%%%%%%%%%%%%%%%%%%%%%%%%%%%%%%%%%%%%%%%%%%%%%%%%%%%%%%%%%%%%%%%
\section{Examples of Imaging}
\label{section:im.examples}

See the scripts provied in Appendix~\ref{chapter:scripts} for examples of
imaging.  In particular, we refer
the interested user to the demonstrations for:
\begin{itemize}
\item NGC5921 (VLA HI) --- a quick demo of basic CASA spectral line
      cube imaging and analysis
      (\ref{section:scripts.ngc5921})
\item Jupiter (VLA 6cm continuum polarimetry) --- polarization imaging
      and analysis
      (\ref{section:scripts.jupiter})
\item NGC4826 (BIMA 3mm CO) --- mosaic imaging of spectral line data
      (\ref{section:scripts.ngc4826})
\end{itemize}

%%%%%%%%%%%%%%%%%%%%%%%%%%%%%%%%%%%%%%%%%%%%%%%%%%%%%%%%%%%%%%%%%
%%%%%%%%%%%%%%%%%%%%%%%%%%%%%%%%%%%%%%%%%%%%%%%%%%%%%%%%%%%%%%%%%
