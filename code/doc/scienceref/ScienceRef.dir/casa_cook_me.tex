%%%%%%%%%%%%%%%%%%%%%%%%%%%%%%%%%%%%%%%%%%%%%%%%%%%%%%%%%%%%%%%%%
%%%%%%%%%%%%%%%%%%%%%%%%%%%%%%%%%%%%%%%%%%%%%%%%%%%%%%%%%%%%%%%%%
%%%%%%%%%%%%%%%%%%%%%%%%%%%%%%%%%%%%%%%%%%%%%%%%%%%%%%%%%%%%%%%%%

% STM 2007-04-13  split from previous version
% STM 2007-10-02  add back in to cookbook appendix
% STM 2007-10-10  beta release (spell-checked)

\chapter[Appendix: The Measurement Equation and Calibration]
         {The Measurement Equation and Calibration}
\label{chapter:me}

The visibilities measured by an interferometer must be calibrated before
formation of an image. This is because the wavefronts received and
processed by the observational hardware have been corrupted by a
variety of effects. These include (but are not exclusive to): the
effects of transmission through 
the atmosphere, the imperfect details amplified electronic (digital)
signal and transmission through the signal processing system, and the
effects of formation of the cross-power spectra by a
correlator. Calibration is the process of reversing these effects to
arrive at corrected visibilities which resemble as closely as possible
the visibilities that would have been measured in vacuum by a perfect
system. The subject of this chapter is the
determination of these effects by using the visibility data itself. 

%%%%%%%%%%%%%%%%%%%%%%%%%%%%%%%%%%%%%%%%%%%%%%%%%%%%%%%%%%%%%%%%%
%%%%%%%%%%%%%%%%%%%%%%%%%%%%%%%%%%%%%%%%%%%%%%%%%%%%%%%%%%%%%%%%%
\section{The HBS Measurement Equation}
\label{section:me.intro}

The relationship between the observed and ideal (desired) visibilities
on the baseline between antennas i and j may be expressed by the
Hamaker-Bregman-Sault {\it Measurement Equation}\footnote{Hamaker, 
J.P., Bregman, J.D. \& Sault, R.J. (1996), Astronomy and Astrophysics
Supplement, v.117, p.137-147}:

$\vec{V}_{ij}~=~J_{ij}~\vec{V}_{ij}^{\mathrm{~IDEAL}}$

where $\vec{V}_{ij}$ represents the observed visibility,
$\vec{V}_{ij}^{\mathrm{~IDEAL}}$ represents the corresponding ideal
visibilities, and $J_{ij}$ represents the accumulation of all
corruptions affecting baseline $ij$. The visibilities are indicated as
vectors spanning the four correlation combinations which can be formed
from dual-polarization signals. These four correlations are related
directly to the Stokes parameters which fully describe the
radiation. The $J_{ij}$ term is therefore a 4$\times$4 matrix. 

Most of the effects contained in $J_{ij}$ (indeed, the most important
of them) are antenna-based, i.e., they arise from measurable physical
properties of (or above) individual antenna elements in a synthesis
array. Thus, adequate calibration of an array of $N_{ant}$ antennas
forming $N_{ant}(N_{ant}-1)/2$ baseline visibilities is usually
achieved through the determination of only $N_{ant}$ factors, such
that $J_{ij} = J_i \otimes J_j^{*}$. For the rest of this chapter, we
will usually assume that $J_{ij}$ is factorable in this way, unless
otherwise noted. 

As implied above, $J_{ij}$ may also be factored into the sequence of
specific corrupting effects, each having their own particular
(relative) importance and physical origin, which determines their
unique algebra. Including the most commonly considered effects, the
Measurement Equation can be written: 

$\vec{V}_{ij}~=~M_{ij}~B_{ij}~G_{ij}~D_{ij}~E_{ij}~P_{ij}~T_{ij}~\vec{V}_{ij}^{\mathrm{~IDEAL}}$   

where:

\begin{itemize}
   \item $T_{ij}~=~$ Polarization-independent multiplicative effects
     introduced by the troposphere, such as opacity and path-length
     variation. 
   \item  $P_{ij}~=~$ Parallactic angle, which describes the
     orientation of the polarization coordinates on the plane of the
     sky. This term varies according to the type of the antenna mount. 
   \item  $E_{ij}~=~$ Effects introduced by properties of the optical
     components of the telescopes, such as the collecting area's
     dependence on elevation. 
   \item  $D_{ij}~=~$ Instrumental polarization response. "D-terms"
     describe the polarization leakage between feeds (e.g. how much the
     R-polarized feed picked up L-polarized emission, and vice versa). 
   \item  $G_{ij}~=~$ Electronic gain response due to components in
     the signal path between the feed and the correlator. This complex
     gain term $G_{ij}$ includes the scale factor for absolute flux
     density calibration, and may include phase and amplitude
     corrections due to changes in the atmosphere (in lieu of
     $T_{ij}$). These gains are polarization-dependent. 
   \item  $B_{ij}~=~$ Bandpass (frequency-dependent) response, such as
     that introduced by spectral filters in the electronic transmission
     system 
   \item  $M_{ij}~=~$ Baseline-based correlator (non-closing)
     errors. By definition, these are not factorable into antenna-based
     parts.  
\end{itemize}

Note that the terms are listed in the order in which they affect the
incoming wavefront ($G$ and $B$ represent an arbitrary sequence of
such terms depending upon the details of the particular electronic
system). Note that $M$ differs from all of the rest in that it is not
antenna-based, and thus not factorable into terms for each antenna. 

As written above, the measurement equation is very general; not all
observations will require treatment of all effects, depending upon the
desired dynamic range. E.g., bandpass need only be considered for
continuum observations if observed in a channelized mode and very high
dynamic range is desired. Similarly, instrumental polarization
calibration can usually be omitted when observing (only) total
intensity using circular feeds. Ultimately, however, each of these
effects occurs at some level, and a complete treatment will yield the
most accurate calibration. Modern high-sensitivity instruments such as
ALMA and EVLA will likely require a more general calibration treatment
for similar observations with older arrays in order to reach the
advertised dynamic ranges on strong sources. 

In practice, it is usually far too difficult to adequately measure
most calibration effects absolutely (as if in the laboratory) for use
in calibration. The effects are usually far too changeable. Instead,
the calibration is achieved by making observations of calibrator
sources on the appropriate timescales for the relevant effects, and
solving the measurement equation for them using the fact that we have
$N_{ant}(N_{ant}-1)/2$ measurements and only $N_{ant}$ factors to
determine (except for $M$ which is only sparingly used). ({\it Note: By
partitioning the calibration factors into a series of consecutive
effects, it might appear that the number of free parameters is some
multiple of $N_{ant}$, but the relative algebra and timescales of the
different effects, as well as the the multiplicity of observed
polarizations and channels compensate, and it can be shown that the
problem remains well-determined until, perhaps, the effects are
direction-dependent within the field of view. Limited solvers for such
effects are under study; the {\tt calibrater} tool currently only handles
effects which may be assumed constant within the field of
view. Corrections for the primary beam are handled in the {\tt imager}
tool.}) Once determined, these terms are used to correct the
visibilities measured for the scientific target. This procedure is
known as cross-calibration (when only phase is considered, it is
called phase-referencing). 

The best calibrators are point sources at the phase center (constant
visibility amplitude, zero phase), with sufficient flux density to
determine the calibration factors with adequate SNR on the relevant
timescale. The primary gain calibrator must be sufficiently close to
the target on the sky so that its observations sample the same
atmospheric effects. A bandpass calibrator usually must be
sufficiently strong (or observed with sufficient duration) to provide
adequate per-channel sensitivity for a useful calibration. In
practice, several calibrators are usually observed, each with
properties suitable for one or more of the required calibrations. 

Synthesis calibration is inherently a bootstrapping process. First,
the dominant calibration term is determined, and then, using this
result, more subtle effects are solved for, until the full set of
required calibration terms is available for application to the target
field. The solutions for each successive term are relative to the
previous terms. Occasionally, when the several calibration terms are
not sufficiently orthogonal, it is useful to re-solve for earlier
types using the results for later types, in effect, reducing the
effect of the later terms on the solution for earlier ones, and thus
better isolating them. This idea is a generalization of the
traditional concept of self-calibration, where initial imaging of the
target source supplies the visibility model for a re-solve of the gain
calibration ($G$ or $T$). Iteration tends toward convergence to a
statistically optimal image. In general, the quality of each
calibration and of the source model are mutually dependent. In
principle, as long as the solution for any calibration component (or
the source model itself) is likely to improve substantially through
the use of new information (provided by other improved solutions), it
is worthwhile to continue this process. 

In practice, these concepts motivate certain patterns of calibration
for different types of observation, and the {\tt calibrater} tool in CASA is
designed to accommodate these patterns in a general and flexible
manner. For a spectral line total intensity observation, the pattern
is usually: 

\begin{enumerate}
   \item Solve for $G$ on the bandpass calibrator
   \item Solve for $B$ on the bandpass calibrator, using $G$
   \item Solve for $G$ on the primary gain (near-target) and flux
      density calibrators, using $B$ solutions just obtained 
   \item Scale $G$ solutions for the primary gain calibrator according to
      the flux density calibrator solutions 
   \item Apply $G$ and $B$ solutions to the target data
   \item Image the calibrated target data 
\end{enumerate}

If opacity and gain curve information are relevant and available,
these types are incorporated in each of the steps (in future, an
actual solve for opacity from appropriate data may be folded into this
process): 
 
\begin{enumerate}
   \item Solve for $G$ on the bandpass calibrator, using $T$ (opacity)
      and $E$ (gain curve) solutions already derived. 
   \item Solve for $B$ on the bandpass calibrator, using $G$, $T$
      (opacity), and $E$ (gain curve) solutions. 
   \item Solve for $G$ on primary gain (near-target) and flux density
      calibrators, using $B$, $T$ (opacity), and $E$ (gain curve)
      solutions. 
   \item Scale $G$ solutions for the primary gain calibrator according to
      the flux density calibrator solutions 
   \item Apply $T$ (opacity), $E$ (gain curve), $G$, and $B$ solutions
      to the target data 
   \item Image the calibrated target data 
\end{enumerate}

For continuum polarimetry, the typical pattern is:

\begin{enumerate}
   \item Solve for $G$ on the polarization calibrator, using (analytical)
      $P$ solutions. 
   \item Solve for $D$ on the polarization calibrator, using $P$ and $G$
      solutions. 
   \item Solve for $G$ on primary gain and flux density calibrators,
      using $P$ and $D$ solutions. 
   \item Scale $G$ solutions for the primary gain calibrator according to
      the flux density calibrator solutions. 
   \item Apply $P$, $D$, and $G$ solutions to target data.
   \item Image the calibrated target data. 
\end{enumerate}

For a spectro-polarimetry observation, these two examples would be
folded together. 

In all cases the calibrator model must be adequate at each solve
step. At high dynamic range and/or high resolution, many calibrators
which are nominally assumed to be point sources become slightly
resolved. If this has biased the calibration solutions, the offending
calibrator may be imaged at any point in the process and the resulting
model used to improve the calibration. Finally, if sufficiently
strong, the target may be self-calibrated as well. 

%%%%%%%%%%%%%%%%%%%%%%%%%%%%%%%%%%%%%%%%%%%%%%%%%%%%%%%%%%%%%%%%%
%%%%%%%%%%%%%%%%%%%%%%%%%%%%%%%%%%%%%%%%%%%%%%%%%%%%%%%%%%%%%%%%%
\section{General Calibrater Mechanics}
\label{section:me.mech}

The calibrater tasks/tool are designed to solve and apply solutions
for all of the solution types listed above (and more are in the
works). This leads to a single basic sequence of execution for all
solves, regardless of type: 

\begin{enumerate}
   \item Set the calibrator model visibilities
   \item Select the visibility data which will be used to solve for a
      calibration type 
   \item Arrange to apply any already-known calibration types (the first
      time through, none may yet be available) 
   \item Arrange to solve for a specific calibration type, including
      specification of the solution timescale and other specifics 
   \item Execute the solve process
   \item Repeat 1-4 for all required types, using each result, as it
      becomes available, in step 2, and perhaps repeating for some types
      to improve the solutions  
\end{enumerate}

By itself, this sequence doesn't guarantee success; the data provided
for the solve must have sufficient SNR on the appropriate timescale,
and must provide sufficient leverage for the solution (e.g., D
solutions require data taken over a sufficient range of parallactic
angle in order to separate the source polarization contribution from
the instrumental polarization). 

%%%%%%%%%%%%%%%%%%%%%%%%%%%%%%%%%%%%%%%%%%%%%%%%%%%%%%%%%%%%%%%%%
%%%%%%%%%%%%%%%%%%%%%%%%%%%%%%%%%%%%%%%%%%%%%%%%%%%%%%%%%%%%%%%%%
