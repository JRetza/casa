%%%%%%%%%%%%%%%%%%%%%%%%%%%%%%%%%%%%%%%%%%%%%%%%%%%%%%%%%%%%%%%%%
%%%%%%%%%%%%%%%%%%%%%%%%%%%%%%%%%%%%%%%%%%%%%%%%%%%%%%%%%%%%%%%%%
%%%%%%%%%%%%%%%%%%%%%%%%%%%%%%%%%%%%%%%%%%%%%%%%%%%%%%%%%%%%%%%%%

% STM 2007-04-13  split from previous version
% STM 2007-10-11  add into beta
% STM 2007-10-22  put in appendix for Beta Release

%\chapter{Simulation}
\chapter[Appendix: Simulation]{Simulation}
\label{chapter:sim}


The capability for simulating observations and datasets from the EVLA
and ALMA are an important use-case for CASA.  This not only allows one
to get an idea of the capabilities of these instruments for doing
science, but also provides benchmarks for the performance and utility
of the software for processing ``realistic'' datasets (with
atmospheric and instrumental effects).  CASA can now calculate
visibilities (create a measurement set) for any interferometric array,
and calculate and apply calibration tables representing some of the
most important corrupting effects.  Version 3.0.0 now also contains a
{\em beta} version of a task to simulate single dish or total power
data from the same sky model as interferometric data.

CASA's simulation capabilities have been under development, and will
likely continue to evolve through this and the next release of CASA.
For the most current information, please refer to
\url{http://www.casaguides.nrao.edu}, and click on ``Simulating
Observations in CASA''.
%
Following general CASA practice, the greatest flexibility and richest
functionality is at the Toolkit level.  The most commonly used
procedures for interferometric simulation are encapsulated in the {\tt
  simdata} task.  Total power simulation is possible at the task level
using the {\tt sdsimdata} task, which follows somewhat parallel inputs
and whose capabilities will being developed more fully in the next
release.

%{\bf BETA ALERT:} The simulation capabilities are currently under
%development.  What we do have is mostly at the Toolkit level.
%We have only a single task {\tt almasimmos} at the present time.
%Stay tuned.  For the Beta Release, we include this chapter
%in the Appendix for the use of telescope commissioners and software
%developers.

\begin{wrapfigure}{r}{2.5in}
  \begin{boxedminipage}{2.5in}
     \centerline{\bf Inside the Toolkit:}
     The simulator methods are in the {\tt sm} tool.
     Many of the other tools are also helpful when
     constructing and analyzing simulations.
  \end{boxedminipage}
\end{wrapfigure}

%%%%%%%%%%%%%%%%%%%%%%%%%%%%%%%%%%%%%%%%%%%%%%%%%%%%%%%%%%%%%%%%%
%%%%%%%%%%%%%%%%%%%%%%%%%%%%%%%%%%%%%%%%%%%%%%%%%%%%%%%%%%%%%%%%%
\section{Simulating ALMA with {\tt simdata}}
\label{section:sim.almasimmos}

The inputs are:
\small
\begin{verbatim}
#  simdata :: mosaic simulation task:
modelimage          =         ''        #  input image name
ignorecoord         =      False        #  scale model coordinates to output parameters
inbright            = 'unchanged'       #  set peak surface brightness in Jy/pixel or "unchanged"
complist            =         ''        #  componentlist table to observe
antennalist         = 'alma.out10.cfg'  #  antenna position file
checkinputs         =       'no'        #  graphically verify parameters [yes|no|only]
project             =      'sim'        #  root for output files
refdate             = '2012/05/21/22:05:00' #  center time/date of observation *see help
totaltime           =    '7200s'        #  total time of observation
integration         =      '10s'        #  integration (sampling) time
scanlength          =          5        #  number of integrations per pointing in the mosaic
startfreq           =    '89GHz'        #  frequency of first channel
chanwidth           =    '10MHz'        #  channel width
nchan               =          1        #  number of channels
direction           = ['J2000 19h00m00 -40d00m00'] #  mosaic center, or list of pointings
pointingspacing     =  '1arcmin'        #  spacing in between beams in mosaic
relmargin           =        1.0        #  space btw. pointings and edge, relative to pointingspacing
cell                = '0.1arcsec'       #  output cell/pixel size
imsize              = [128, 128]        #  output image size in pixels (x,y)
niter               =        500        #  maximum number of iterations
threshold           =  '0.01mJy'        #  flux level (+units) to stop cleaning
psfmode             =    'clark'        #  method of PSF calculation to use during minor cycles
weighting           =  'natural'        #  weighting to apply to visibilities
uvtaper             =      False        #  apply additional uv tapering of  visibilities.
stokes              =        'I'        #  Stokes params to image
noise_thermal       =       True        #  add thermal noise
     noise_mode     = 'tsys-atm'        #  tsys-atm: set PWV and use ATM library; tsys-manual: set t_sky and tau
     user_pwv       =        1.0        #  Precipitable Water Vapor in mm
     t_ground       =      269.0        #  ambient temperature
     t_sky          =      263.0        #  atmospheric temperature
     tau0           =        0.1        #  zenith opacity
fidelity            =       True        #  Calculate fidelity images
display             =       True        #  Plot simulation result images,figures
verbose             =      False        
async               =      False        #  If true the taskname must be started using simdata(...)

\end{verbatim}
\normalsize

This task takes an input model image or list of components, plus a
list of antennas (locations and sizes), and simulates a particular
observation (specifies by mosaic setup and observing cycles and
times).  The output is a MS suitable for further analysis in CASA, a
synthesized image create from those visibilities, a difference image
between the synthesized image and your sky model convolved with the
output synthesized beam, and a fidelity image.

Random thermal noise (from the atmosphere and receiver) can optionally
be added. A realistic model of the troposphere is created from known
site characteristics (altitude, etc) and used to calculate the
frequency dependence of the noise.  Similarly, specifications for ALMA
and EVLA receiver temperature and antenna efficiencies are coded into
the task to determine a realistic system temperature.  Other
corrupting effects, such as phase noise, gain drift, and
cross-polarization, can be added the the MS with the toolkit - see the
web documentation for more information.

Much more detailed information, and use-case examples, can be found at 

\url{http://casaguides.nrao.edu/index.php?title=Simulating_Observations_in_CASA}

%%%%%%%%%%%%%%%%%%%%%%%%%%%%%%%%%%%%%%%%%%%%%%%%%%%%%%%%%%%%%%%%%
%%%%%%%%%%%%%%%%%%%%%%%%%%%%%%%%%%%%%%%%%%%%%%%%%%%%%%%%%%%%%%%%%
