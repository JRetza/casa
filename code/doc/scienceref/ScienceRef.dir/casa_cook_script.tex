%%%%%%%%%%%%%%%%%%%%%%%%%%%%%%%%%%%%%%%%%%%%%%%%%%%%%%%%%%%%%%%%%
%%%%%%%%%%%%%%%%%%%%%%%%%%%%%%%%%%%%%%%%%%%%%%%%%%%%%%%%%%%%%%%%%
%%%%%%%%%%%%%%%%%%%%%%%%%%%%%%%%%%%%%%%%%%%%%%%%%%%%%%%%%%%%%%%%%

% STM 2007-04-13  split from previous version
% STM 2007-06-29  replace with N5921 usecase, add Jupiter
% STM 2007-09-04  update Jupiter script
% STM 2008-03-25  update scripts to Patch 1, add polcal script
% STM 2008-05-14  start update to Patch 2
% STM 2008-06-11  final Patch 2 versions
% STM 2009-05-29  start Patch 4 versions, add ngc4826 tutorial
% STM 2009-12-20  Release 0 versions (3.0.0)
% JO 2010-03-01 Release 3.0.1 edits

\chapter[Appendix: Annotated Example Scripts]{Annotated Example Scripts}
\label{chapter:scripts}

Note: These data sets are available with the full
CASA rpm distribution.  Other data sets can be made available upon
request.  The scripts are intended to illustrate the types of
commands needed for different types of reduction/astronomical
observations.

{\bf ALERT:} Some tasks are still under development and we will be
occasionally updating the syntax of the tasks, which may break older
versions of the scripts. You can find the latest versions of these
(and other) scripts at:
\begin{quote}
   \url{http://casa.nrao.edu/Doc/Scripts/}
\end{quote}

We also provide a number of annotated data reduction tutorials and
scripts on our wiki page 

\begin{quote}
\url{http://casaguides.nrao.edu}
\end{quote}

Those will cover more recent data such as EVLA OSRO data.

%%%%%%%%%%%%%%%%%%%%%%%%%%%%%%%%%%%%%%%%%%%%%%%%%%%%%%%%%%%%%%%%%%%%%%%%
%%%%%%%%%%%%%%%%%%%%%%%%%%%%%%%%%%%%%%%%%%%%%%%%%%%%%%%%%%%%%%%%%%%%%%%%
\section{NGC 5921 --- VLA red-shifted HI emission}
\label{section:scripts.ngc5921}

This script demonstates basic spectral calibration and imaging, but
does not include any self-calibration steps.

The latest version of this script can be found at:
\begin{quote}
   \url{http://casa.nrao.edu/Doc/Scripts/ngc5921_demo.py}
\end{quote}

\small
\begin{verbatim}
##########################################################################
#                                                                        #
# Use Case Script for NGC 5921                                           #
#                                                                        #
# Converted by STM 2007-05-26                                            #
# Updated      STM 2007-06-15 (Alpha Patch 1)                            #
# Updated      STM 2007-09-05 (Alpha Patch 2+)                           #
# Updated      STM 2007-09-18 (Alpha Patch 2+)                           #
# Updated      STM 2007-09-18 (Pre-Beta) add immoments                   #
# Updated      STM 2007-10-04 (Beta) update                              #
# Updated      STM 2007-10-10 (Beta) add export                          #
# Updated      STM 2007-11-08 (Beta Patch 0.5) add RRusk stuff           #
# Updated      STM 2008-03-25 (Beta Patch 1.0)                           #
# Updated      STM 2008-05-23 (Beta Patch 2.0) new tasking/clean/cal     #
# Updated      STM 2008-06-11 (Beta Patch 2.0)                           #
# Updated      STM 2008-06-13 (Beta Patch 2.0) demo version              #
# Updated      STM 2008-06-14 (Beta Patch 2.0) post-school update        #
# Updated      STM 2008-07-06 (Beta Patch 2.0) regression version        #
# Updated      STM 2009-05-26 (Beta Patch 4.0) McMaster demo             #
# Updated      STM 2009-12-02 (Release 0)                                #
#                                                                        #
# Features Tested:                                                       #
#    The script illustrates end-to-end processing with CASA              #
#    as depicted in the following flow-chart.                            #
#                                                                        #
#    Filenames will have the <prefix> = 'ngc5921.demo'                   #
#                                                                        #
#    Input Data           Process          Output Data                   #
#                                                                        #
#   NGC5921.fits --> importuvfits  -->  <prefix>.ms   +                  #
#   (1.4GHz,               |            <prefix>.ms.flagversions         #
#    63 sp chan,           v                                             #
#    D-array)           listobs    -->  casapy.log                       #
#                          |                                             #
#                          v                                             #
#                     flagautocorr                                       #
#                          |                                             #
#                          v                                             #
#                        setjy                                           #
#                          |                                             #
#                          v                                             #
#                       bandpass   -->  <prefix>.bcal                    #
#                          |                                             #
#                          v                                             #
#                       gaincal    -->  <prefix>.gcal                    #
#                          |                                             #
#                          v                                             #
#                      fluxscale   -->  <prefix>.fluxscale               #
#                          |                                             #
#                          v                                             #
#                      applycal    -->  <prefix>.ms                      #
#                          |                                             #
#                          v                                             #
#                        split     -->  <prefix>.cal.split.ms            #
#                          |                                             #
#                          v                                             #
#                        split     -->  <prefix>.src.split.ms            #
#                          |                                             #
#                          v                                             #
#                    exportuvfits  -->  <prefix>.split.uvfits            #
#                          |                                             #
#                          v                                             #
#                      uvcontsub   -->  <prefix>.ms.cont +               #
#                          |            <prefix>.ms.contsub              #
#                          v                                             #
#                        clean     -->  <prefix>.clean.image +           #
#                          |            <prefix>.clean.model +           #
#                          |            <prefix>.clean.residual          #
#                          v                                             #
#                     exportfits   -->  <prefix>.clean.fits              #
#                          |                                             #
#                          v                                             #
#                       imhead     -->  casapy.log                       #
#                          |                                             #
#                          v                                             #
#                       imstat     -->  xstat (parameter)                #
#                          |                                             #
#                          v                                             #
#                      immoments   -->  <prefix>.moments.integrated +    #
#                          |            <prefix>.moments.weighted_coord  #
#                          v                                             #
##########################################################################
print 'Demo Script for NGC5921 VLA HI observation'
print 'Version for Release 0 (3.0.0) 2-Dec-2009'
print ''

import time
import os

scriptmode = True
# 
# The prefix to use for all output files
prefix='ngc5921.demo'

# Set up some useful variables (these will also be set later on)
msfile = prefix + '.ms'
btable = prefix + '.bcal'
gtable = prefix + '.gcal'
ftable = prefix + '.fluxscale'
splitms = prefix + '.src.split.ms'
imname = prefix + '.cleanimg'

#
# Get to path to the CASA home and stip off the name
pathname=os.environ.get('CASAPATH').split()[0]

# This is where the NGC5921 UVFITS data will be
fitsdata=pathname+'/data/demo/NGC5921.fits'
#
# Or uncomment the following to use data in current directory
#fitsdata='NGC5921.fits'

# Clean up old files
# Use rmtables on ms and cal tables to clear cache
# (not working on multiple runs for 2.4.0 release)
#rmtables(msfile)
#rmtables(btable)
#rmtables(gtable)
#rmtables(ftable)
#rmtables(ftable)
#rmtables(splitms+'*')
#rmtables(imname+'.*')
#rmtables(prefix+'.moments*')

# Final clean up of auxiliary files
os.system('rm -rf '+prefix+'*')

#
#=====================================================================
#
# Import the data from FITS to MS
#
print '--Import--'

# Safest to start from task defaults
default('importuvfits')

# Set up the MS filename and save as new global variable
msfile = prefix + '.ms'

# Use task importuvfits
fitsfile = fitsdata
vis = msfile

saveinputs('importuvfits',prefix+'.importuvfits.saved')

importuvfits()

#
# Note that there will be a ngc5921.demo.ms.flagversions
# there containing the initial flags as backup for the main ms
# flags.
#
#=====================================================================
#
# List a summary of the MS
#
print '--Listobs--'

# Don't default this one and make use of the previous setting of
# vis.  Remember, the variables are GLOBAL!

# You may wish to see more detailed information, like the scans.
# In this case use the verbose = True option
verbose = True

listobs()

# You should get in your logger window and in the casapy.log file
# something like:
#
# MeasurementSet Name:  /home/sandrock2/smyers/Testing2/Sep07/ngc5921.demo.ms
# MS Version 2
# 
# Observer: TEST     Project:   
# Observation: VLA
# 
# Data records: 22653       Total integration time = 5280 seconds
#    Observed from   09:19:00   to   10:47:00
# 
#    ObservationID = 0         ArrayID = 0
#   Date        Timerange                Scan  FldId FieldName      SpwIds
#   13-Apr-1995/09:19:00.0 - 09:24:30.0     1      0 1331+30500002_0  [0]
#               09:27:30.0 - 09:29:30.0     2      1 1445+09900002_0  [0]
#               09:33:00.0 - 09:48:00.0     3      2 N5921_2        [0]
#               09:50:30.0 - 09:51:00.0     4      1 1445+09900002_0  [0]
#               10:22:00.0 - 10:23:00.0     5      1 1445+09900002_0  [0]
#               10:26:00.0 - 10:43:00.0     6      2 N5921_2        [0]
#               10:45:30.0 - 10:47:00.0     7      1 1445+09900002_0  [0]
# 
# Fields: 3
#   ID   Code Name          Right Ascension  Declination   Epoch   
#   0    C    1331+30500002_013:31:08.29      +30.30.32.96  J2000   
#   1    A    1445+09900002_014:45:16.47      +09.58.36.07  J2000   
#   2         N5921_2       15:22:00.00      +05.04.00.00  J2000   
# 
# Spectral Windows:  (1 unique spectral windows and 1 unique polarization setups)
#   SpwID  #Chans Frame Ch1(MHz)    Resoln(kHz) TotBW(kHz)  Ref(MHz)    Corrs   
#   0          63 LSRK  1412.68608  24.4140625  1550.19688  1413.44902  RR  LL  
# 
# Feeds: 28: printing first row only
#   Antenna   Spectral Window     # Receptors    Polarizations
#   1         -1                  2              [         R, L]
# 
# Antennas: 27:
#   ID   Name  Station   Diam.    Long.         Lat.         
#   0    1     VLA:N7    25.0 m   -107.37.07.2  +33.54.12.9  
#   1    2     VLA:W1    25.0 m   -107.37.05.9  +33.54.00.5  
#   2    3     VLA:W2    25.0 m   -107.37.07.4  +33.54.00.9  
#   3    4     VLA:E1    25.0 m   -107.37.05.7  +33.53.59.2  
#   4    5     VLA:E3    25.0 m   -107.37.02.8  +33.54.00.5  
#   5    6     VLA:E9    25.0 m   -107.36.45.1  +33.53.53.6  
#   6    7     VLA:E6    25.0 m   -107.36.55.6  +33.53.57.7  
#   7    8     VLA:W8    25.0 m   -107.37.21.6  +33.53.53.0  
#   8    9     VLA:N5    25.0 m   -107.37.06.7  +33.54.08.0  
#   9    10    VLA:W3    25.0 m   -107.37.08.9  +33.54.00.1  
#   10   11    VLA:N4    25.0 m   -107.37.06.5  +33.54.06.1  
#   11   12    VLA:W5    25.0 m   -107.37.13.0  +33.53.57.8  
#   12   13    VLA:N3    25.0 m   -107.37.06.3  +33.54.04.8  
#   13   14    VLA:N1    25.0 m   -107.37.06.0  +33.54.01.8  
#   14   15    VLA:N2    25.0 m   -107.37.06.2  +33.54.03.5  
#   15   16    VLA:E7    25.0 m   -107.36.52.4  +33.53.56.5  
#   16   17    VLA:E8    25.0 m   -107.36.48.9  +33.53.55.1  
#   17   18    VLA:W4    25.0 m   -107.37.10.8  +33.53.59.1  
#   18   19    VLA:E5    25.0 m   -107.36.58.4  +33.53.58.8  
#   19   20    VLA:W9    25.0 m   -107.37.25.1  +33.53.51.0  
#   20   21    VLA:W6    25.0 m   -107.37.15.6  +33.53.56.4  
#   21   22    VLA:E4    25.0 m   -107.37.00.8  +33.53.59.7  
#   23   24    VLA:E2    25.0 m   -107.37.04.4  +33.54.01.1  
#   24   25    VLA:N6    25.0 m   -107.37.06.9  +33.54.10.3  
#   25   26    VLA:N9    25.0 m   -107.37.07.8  +33.54.19.0  
#   26   27    VLA:N8    25.0 m   -107.37.07.5  +33.54.15.8  
#   27   28    VLA:W7    25.0 m   -107.37.18.4  +33.53.54.8  
# 
# Tables:
#    MAIN                   22653 rows     
#    ANTENNA                   28 rows     
#    DATA_DESCRIPTION           1 row      
#    DOPPLER             <absent>  
#    FEED                      28 rows     
#    FIELD                      3 rows     
#    FLAG_CMD             <empty>  
#    FREQ_OFFSET         <absent>  
#    HISTORY                  273 rows     
#    OBSERVATION                1 row      
#    POINTING                 168 rows     
#    POLARIZATION               1 row      
#    PROCESSOR            <empty>  
#    SOURCE                     3 rows     
#    SPECTRAL_WINDOW            1 row      
#    STATE                <empty>  
#    SYSCAL              <absent>  
#    WEATHER             <absent>  
# 
#
#=====================================================================
#
# Get rid of the autocorrelations from the MS
#
print '--Flagautocorr--'

# Don't default this one either, there is only one parameter (vis)

flagautocorr()

#
#=====================================================================
#
# Set the fluxes of the primary calibrator(s)
#
print '--Setjy--'
default('setjy')

vis = msfile

#
# 1331+305 = 3C286 is our primary calibrator
# Use the wildcard on the end of the source name
# since the field names in the MS have inherited the
# AIPS qualifiers
field = '1331+305*'

# This is 1.4GHz D-config and 1331+305 is sufficiently unresolved
# that we dont need a model image.  For higher frequencies
# (particularly in A and B config) you would want to use one.
modimage = ''

# Setjy knows about this source so we dont need anything more

saveinputs('setjy',prefix+'.setjy.saved')

# Pause script if you are running in scriptmode
if scriptmode:
    inp()
    user_check=raw_input('Return to continue script\n')

setjy()

#
# You should see something like this in the logger and casapy.log file:
#
# 1331+30500002_0  spwid=  0  [I=14.76, Q=0, U=0, V=0] Jy, (Perley-Taylor 99)
#
# So its using 14.76Jy as the flux of 1331+305 in the single Spectral Window
# in this MS.
#
#=====================================================================
#
# Bandpass calibration
#
print '--Bandpass--'
default('bandpass')

# We can first do the bandpass on the single 5min scan on 1331+305
# At 1.4GHz phase stablility should be sufficient to do this without
# a first (rough) gain calibration.  This will give us the relative
# antenna gain as a function of frequency.

vis = msfile

# set the name for the output bandpass caltable
btable = prefix + '.bcal'
caltable = btable

# No gain tables yet
gaintable = ''
gainfield = ''
interp = ''

# Use flux calibrator 1331+305 = 3C286 (FIELD_ID 0) as bandpass calibrator
field = '0'
# all channels
spw = ''
# No other selection
selectdata = False

# In this band we do not need a-priori corrections for
# antenna gain-elevation curve or atmospheric opacity
# (at 8GHz and above you would want these)
gaincurve = False
opacity = 0.0

# Choose bandpass solution type
# Pick standard time-binned B (rather than BPOLY)
bandtype = 'B'

# set solution interval arbitrarily long (get single bpass)
solint = 'inf'
combine = 'scan'

# reference antenna Name 15 (15=VLA:N2) (Id 14)
refant = '15'

saveinputs('bandpass',prefix+'.bandpass.saved')

# Pause script if you are running in scriptmode
if scriptmode:
    inp()
    user_check=raw_input('Return to continue script\n')

bandpass()

#
#=====================================================================
#
# Use plotcal to examine the bandpass solutions
#
print '--Plotcal (bandpass)--'
default('plotcal')

caltable = btable
field = '0'

# Set up 2x1 panels - upper panel amp vs. channel
subplot = 211
yaxis = 'amp'
# No output file yet (wait to plot next panel)

saveinputs('plotcal',prefix+'.plotcal.b.amp.saved')

if scriptmode:
    showgui = True
else:
    showgui = False

plotcal()
#
# Set up 2x1 panels - lower panel phase vs. channel
subplot = 212
yaxis = 'phase'

saveinputs('plotcal',prefix+'.plotcal.b.phase.saved')

#
# Note the rolloff in the start and end channels.  Looks like
# channels 6-56 (out of 0-62) are the best

# Pause script if you are running in scriptmode
if scriptmode:
    # If you want to do this interactively and iterate over antenna, set
    # iteration = 'antenna'
    showgui = True
    plotcal()
    user_check=raw_input('Return to continue script\n')
else:
    # No GUI for this script
    showgui = False
    # Now send final plot to file in PNG format (via .png suffix)
    figfile = caltable + '.plotcal.png'
    plotcal()
    
#=====================================================================
#
# Gain calibration
#
print '--Gaincal--'
default('gaincal')

# Armed with the bandpass, we now solve for the
# time-dependent antenna gains

vis = msfile

# set the name for the output gain caltable
gtable = prefix + '.gcal'
caltable = gtable

# Use our previously determined bandpass
# Note this will automatically be applied to all sources
# not just the one used to determine the bandpass
gaintable = btable
gainfield = ''

# Use nearest (there is only one bandpass entry)
interp = 'nearest'

# Gain calibrators are 1331+305 and 1445+099 (FIELD_ID 0 and 1)
field = '0,1'

# We have only a single spectral window (SPW 0)
# Choose 51 channels 6-56 out of the 63
# to avoid end effects.
# Channel selection is done inside spw
spw = '0:6~56'

# No other selection
selectdata = False

# In this band we do not need a-priori corrections for
# antenna gain-elevation curve or atmospheric opacity
# (at 8GHz and above you would want these)
gaincurve = False
opacity = 0.0

# scan-based G solutions for both amplitude and phase
gaintype = 'G'
solint = 'inf'
combine = ''
calmode = 'ap'

# minimum SNR allowed
minsnr = 1.0

# reference antenna 15 (15=VLA:N2)
refant = '15'

saveinputs('gaincal',prefix+'.gaincal.saved')

if scriptmode:
    inp()
    user_check=raw_input('Return to continue script\n')

gaincal()

#
#=====================================================================
#
# Bootstrap flux scale
#
print '--Fluxscale--'
default('fluxscale')

vis = msfile

# set the name for the output rescaled caltable
ftable = prefix + '.fluxscale'
fluxtable = ftable

# point to our first gain cal table
caltable = gtable

# we will be using 1331+305 (the source we did setjy on) as
# our flux standard reference - note its extended name as in
# the FIELD table summary above (it has a VLA seq number appended)
reference = '1331*'

# we want to transfer the flux to our other gain cal source 1445+099
transfer = '1445*'

saveinputs('fluxscale',prefix+'.fluxscale.saved')

# Pause script if you are running in scriptmode
if scriptmode:
    inp()
    user_check=raw_input('Return to continue script\n')

fluxscale()

# In the logger you should see something like:
# Flux density for 1445+09900002_0 in SpW=0 is:
#     2.48576 +/- 0.00123122 (SNR = 2018.94, nAnt= 27)

# If you run plotcal() on the tablein = 'ngc5921.demo.fluxscale'
# you will see now it has brought the amplitudes in line between
# the first scan on 1331+305 and the others on 1445+099

#
#=====================================================================
#
# Now use plotcal to examine the gain solutions
#
print '--Plotcal (fluxscaled gains)--'
default('plotcal')

caltable = ftable
field = '0,1'

# Set up 2x1 panels - upper panel amp vs. time
subplot = 211
yaxis = 'amp'
# No output file yet (wait to plot next panel)

saveinputs('plotcal',prefix+'.plotcal.gscaled.amp.saved')

if scriptmode:
    showgui = True
else:
    showgui = False

plotcal()
#
# Set up 2x1 panels - lower panel phase vs. time
subplot = 212
yaxis = 'phase'

saveinputs('plotcal',prefix+'.plotcal.gscaled.phase.saved')

#
# The amp and phase coherence looks good

# Pause script if you are running in scriptmode
if scriptmode:
    # If you want to do this interactively and iterate over antenna, set
    #iteration = 'antenna'
    showgui = True
    plotcal()
    user_check=raw_input('Return to continue script\n')
else:
    # No GUI for this script
    showgui = False
    # Now send final plot to file in PNG format (via .png suffix)
    figfile = caltable + '.plotcal.png'
    plotcal()

#=====================================================================
#
# Apply our calibration solutions to the data
# (This will put calibrated data into the CORRECTED_DATA column)
#
print '--ApplyCal--'
default('applycal')

vis = msfile

# We want to correct the calibrators using themselves
# and transfer from 1445+099 to itself and the target N5921

# Start with the fluxscale/gain and bandpass tables
gaintable = [ftable,btable]

# pick the 1445+099 out of the gain table for transfer
# use all of the bandpass table
gainfield = ['1','*']

# interpolation using linear for gain, nearest for bandpass
interp = ['linear','nearest']

# only one spw, do not need mapping
spwmap = []

# all channels
spw = ''
selectdata = False

# as before
gaincurve = False
opacity = 0.0

# select the fields for 1445+099 and N5921
field = '1,2'

applycal()

# Now for completeness apply 1331+305 to itself

field = '0'
gainfield = ['0','*']

# The CORRECTED_DATA column now contains the calibrated visibilities

saveinputs('applycal',prefix+'.applycal.saved')

# Pause script if you are running in scriptmode
if scriptmode:
    inp()
    user_check=raw_input('Return to continue script\n')

applycal()

#
#=====================================================================
#
# Now use plotxy to plot the calibrated target data (before contsub)
#
print '--Plotxy (NGC5921)--'
default('plotxy')

vis = msfile

field = '2'
# Edge channels are bad
spw = '0:4~59'

# Time average across scans
timebin = '86000.'
crossscans = True

# Set up 2x1 panels - upper panel amp vs. channel
subplot = 211
xaxis = 'channel'
yaxis = 'amp'
datacolumn = 'corrected'
# No output file yet (wait to plot next panel)

saveinputs('plotxy',prefix+'.plotxy.final.amp.saved')

print "Amp averaged across time and baseline (upper)"

figfile = ''
if scriptmode:
    interactive = True
else:
    interactive = False

plotxy()
#
# Set up 2x1 panels - lower panel phase vs. time
subplot = 212
yaxis = 'phase'
datacolumn = 'corrected'
# Time average across scans and baselines
timebin = '86000.'
crossscans = True
crossbls = True

saveinputs('plotxy',prefix+'.plotxy.final.phase.saved')

print "Phase averaged across time and baseline (lower)"

print "Final calibrated data"

# Pause script if you are running in scriptmode
if scriptmode:
    interactive = True
    figfile = ''
    plotxy()
    user_check=raw_input('Return to continue script\n')
else:
    interactive = False
    # Now send final plot to file in PNG format (via .png suffix)
    figfile = vis + '.plotxy.png'
    plotxy()
    

#=====================================================================
#
# Split the sources out, pick off the CORRECTED_DATA column
#
#
# Split NGC5921 data (before continuum subtraction)
#
print '--Split NGC5921 Data--'
default('split')

vis = msfile
splitms = prefix + '.src.split.ms'
outputvis = splitms
field = 'N5921*'
spw = ''
datacolumn = 'corrected'

saveinputs('split',prefix+'.split.n5921.saved')

split()

print "Created "+splitms

# If you want, split out the calibrater 1445+099 field, all chans
#print '--Split 1445+099 Data--'
#
#calsplitms = prefix + '.cal.split.ms'
#outputvis = calsplitms
#field = '1445*'
#
#saveinputs('split',prefix+'.split.1445.saved')
#
#split()

#=====================================================================
#
# Here is how to export the NGC5921 data as UVFITS
# Start with the split file.
# Since this is a split dataset, the calibrated data is
# in the DATA column already.
# Write as a multisource UVFITS (with SU table)
# even though it will have only one field in it
# Run asynchronously so as not to interfere with other tasks
# (BETA: also avoids crash on next importuvfits)
#
#print '--Export UVFITS--'
#default('exportuvfits')
#
#srcuvfits = prefix + '.split.uvfits'
#
#vis = splitms
#fitsfile = srcuvfits
#datacolumn = 'data'
#multisource = True
#async = True
#
#saveinputs('exportuvfits',prefix+'.exportuvfits.saved')
#
#myhandle = exportuvfits()
#
#print "The return value for this exportuvfits async task for tm is "+str(myhandle)

#=====================================================================
#
# UV-plane continuum subtraction on the target
# use the split ms
# (this will update the CORRECTED_DATA column)
#
print '--UV Continuum Subtract--'
default('uvcontsub')

vis = splitms

field = 'N5921*'
# Use channels 4-6 and 50-59 for continuum
fitspw='0:4~6;50~59'

# Output all of spw 0
spw = '0'

# Averaging time (none)
solint = 0.0

# Fit only a mean level
fitorder = 0

# Do the uv-plane subtraction
fitmode = 'subtract'

# Let it split out the data automatically for us
splitdata = True

saveinputs('uvcontsub',prefix+'.uvcontsub.saved')

# Pause script if you are running in scriptmode
if scriptmode:
    inp()
    user_check=raw_input('Return to continue script\n')

uvcontsub()

# You will see it made two new MS:
# <vis>.cont
# <vis>.contsub

srcsplitms = splitms + '.contsub'

# Note that ngc5921.demo.ms.contsub contains the uv-subtracted
# visibilities (in its DATA column), and ngc5921.demo.ms.cont
# the pseudo-continuum visibilities (as fit).

# The original ngc5921.demo.ms now contains the uv-continuum
# subtracted vis in its CORRECTED_DATA column and the continuum
# in its MODEL_DATA column as per the fitmode='subtract'

# Done with calibration
#=====================================================================
#
# Here is how to make a dirty image cube
#
#print '--Clean (dirty image)--'
#default('clean')

# Pick up our split source continuum-subtracted data
#vis = srcsplitms
#dirtyname = prefix + '.dirtyimg'
#imagename = dirtyname
#
#mode = 'channel'
#nchan = 46
#start = 5
#width = 1
#
#field = '0'
#spw = ''
#imsize = [256,256]
#cell = [15.,15.]
#weighting = 'briggs'
#robust = 0.5

# No cleaning
#niter = 0

#saveinputs('clean',prefix+'.invert.saved')

# Pause script if you are running in scriptmode
#if scriptmode:
#    inp()
#    user_check=raw_input('Return to continue script\n')
#
#clean()

#dirtyimage = dirtyname+'.image'
 
# Get the dirty image cube statistics
#dirtystats = imstat(dirtyimage)

#=====================================================================
#
# Now clean an image cube of N5921
#
print '--Clean (clean)--'
default('clean')

# Pick up our split source continuum-subtracted data
vis = srcsplitms

# Make an image root file name
imname = prefix + '.cleanimg'
imagename = imname

# Set up the output image cube
mode = 'channel'
nchan = 46
start = 5
width = 1

# This is a single-source MS with one spw
field = '0'
spw = ''

# Standard gain factor 0.1
gain = 0.1

# Set the output image size and cell size (arcsec)
imsize = [256,256]

# Do a simple Clark clean
psfmode = 'clark'
# No Cotton-Schwab iterations
csclean = False

# If desired, you can do a Cotton-Schwab clean
# but will have only marginal improvement for this data
#csclean = True
# Twice as big for Cotton-Schwab (cleans inner quarter)
#imsize = [512,512]

# Pixel size 15 arcsec for this data (1/3 of 45" beam)
# VLA D-config L-band
cell = [15.,15.]

# Fix maximum number of iterations
niter = 6000

# Also set flux residual threshold (in mJy)
threshold=8.0

# Set up the weighting
# Use Briggs weighting (a moderate value, on the uniform side)
weighting = 'briggs'
robust = 0.5

# Set a cleanbox +/-20 pixels around the center 128,128
mask = [108,108,148,148]

# But if you had a cleanbox saved in a file, e.g. "regionfile.txt"
# you could use it:
#mask='regionfile.txt'
#
# If you don't want any clean boxes or masks, then
#mask = ''

# If you want interactive clean set to True
#interactive=True
interactive=False

saveinputs('clean',prefix+'.clean.saved')

# Pause script if you are running in scriptmode
if scriptmode:
    inp()
    user_check=raw_input('Return to continue script\n')

clean()

# Should find stuff in the logger like:
#
# Fitted beam used in restoration: 51.5643 by 45.6021 (arcsec)
#     at pa 14.5411 (deg)
#
# It will have made the images:
# -----------------------------
# ngc5921.demo.cleanimg.flux
# ngc5921.demo.cleanimg.image
# ngc5921.demo.cleanimg.mask
# ngc5921.demo.cleanimg.model
# ngc5921.demo.cleanimg.psf
# ngc5921.demo.cleanimg.residual

clnimage = imname+'.image'

#=====================================================================
#
# Done with imaging
# Now view the image cube of N5921
#
if scriptmode:
    print '--View image--'
    print "Use Spectral Profile Tool to get line profile in box in center"
    viewer(clnimage,'image')
    user_check=raw_input('Return to continue script\n')

#=====================================================================
#
# Here is how to export the Final CLEAN Image as FITS
# Run asynchronously so as not to interfere with other tasks
# (BETA: also avoids crash on next importfits)
#
#print '--Final Export CLEAN FITS--'
#default('exportfits')
#
#clnfits = prefix + '.cleanimg.fits'
#
#imagename = clnimage
#fitsimage = clnfits
#async = True
#
#saveinputs('exportfits',prefix+'.exportfits.saved')
#
#myhandle2 = exportfits()
#
#print "The return value for this exportfits async task for tm is "+str(myhandle2)

#=====================================================================
#
# Print the image header
#
print '--Imhead--'
default('imhead')

imagename = clnimage

mode = 'summary'

imhead()

# A summary of the cube will be seen in the logger

#=====================================================================
#
# Get the cube statistics
#
print '--Imstat (cube)--'
default('imstat')

imagename = clnimage

# Do whole image
box = ''
# or you could stick to the cleanbox
#box = '108,108,148,148'

cubestats = imstat()

# Statistics will printed to the terminal, and the output 
# parameter will contain a dictionary of the statistics

#=====================================================================
#
# Get some image moments
#
print '--ImMoments--'
default('immoments')

imagename = clnimage

# Do first and second moments
moments = [0,1]

# Need to mask out noisy pixels, currently done
# using hard global limits
excludepix = [-100,0.009]

# Collapse along the spectral (channel) axis
axis  = 'spectral'
# Include all planes
chans = ''

# Output root name
momfile = prefix + '.moments'
outfile = momfile

saveinputs('immoments',prefix+'.immoments.saved')

# Pause script if you are running in scriptmode
if scriptmode:
    inp()
    user_check=raw_input('Return to continue script\n')

immoments()

momzeroimage = momfile + '.integrated'
momoneimage = momfile + '.weighted_coord'

# It will have made the images:
# --------------------------------------
# ngc5921.demo.moments.integrated
# ngc5921.demo.moments.weighted_coord

#
#=====================================================================
#
# Get some statistics of the moment images
#
print '--Imstat (moments)--'
default('imstat')

imagename = momzeroimage
momzerostats = imstat()

imagename = momoneimage
momonestats = imstat()

#=====================================================================
#
# Now view the moments
#
if scriptmode:
    print '--View image (Moments)--'
    viewer(momzeroimage)
    print "You can add mom-1 image "+momoneimage
    print "as a contour plot"
    user_check=raw_input('Return to continue script\n')

#=====================================================================
#
# Set up an output logfile
import datetime
datestring=datetime.datetime.isoformat(datetime.datetime.today())

outfile = 'out.'+prefix+'.'+datestring+'.log'
logfile=open(outfile,'w')
print >>logfile,'Results for '+prefix+' :'
print >>logfile,""

#=====================================================================
#
# Can do some image statistics if you wish
# Treat this like a regression script
# WARNING: currently requires toolkit
#
print ' NGC5921 results '
print ' =============== '

print >>logfile,' NGC5921 results '
print >>logfile,' =============== '

#
# Use the ms tool to get max of the MSs
# Eventually should be available from a task
#
# Pull the max cal amp value out of the MS (if you split this)
#ms.open(calsplitms)
#thistest_cal = max(ms.range(["amplitude"]).get('amplitude'))
#ms.close()
#oldtest_cal = 34.0338668823
#diff_cal = abs((oldtest_cal-thistest_cal)/oldtest_cal)
#
#print ' Calibrator data ampl max = ',thistest_cal
#print '   Previous: cal data max = ',oldtest_cal
#print '   Difference (fractional) = ',diff_cal
#print ''
#
#print >>logfile,' Calibrator data ampl max = ',thistest_cal
#print >>logfile,'   Previous: cal data max = ',oldtest_cal
#print >>logfile,'   Difference (fractional) = ',diff_cal
#print >>logfile,''

# Pull the max src amp value out of the MS
ms.open(srcsplitms)
thistest_src = max(ms.range(["amplitude"]).get('amplitude'))
ms.close()
oldtest_src =  46.2060050964 # now in all chans
diff_src = abs((oldtest_src-thistest_src)/oldtest_src)

print ' Target Src data ampl max = ',thistest_src
print '   Previous: src data max = ',oldtest_src
print '   Difference (fractional) = ',diff_src
print ''

print >>logfile,' Target Src data ampl max = ',thistest_src
print >>logfile,'   Previous: src data max = ',oldtest_src
print >>logfile,'   Difference (fractional) = ',diff_src
print >>logfile,''

#
# Now use the stats produced by imstat above
#
# DIRTY IMAGE MAX & RMS (IF YOU MADE A DIRTY IMAGE)
#
#thistest_dirtymax=dirtystats['max'][0]
#oldtest_dirtymax = 0.0515365377069
#diff_dirtymax = abs((oldtest_dirtymax-thistest_dirtymax)/oldtest_dirtymax)
#
#print ' Dirty image max = ',thistest_dirtymax
#print '   Previous: max = ',oldtest_dirtymax
#print '   Difference (fractional) = ',diff_dirtymax
#print ''
#
#print >>logfile,' Dirty Image max = ',thistest_dirtymax
#print >>logfile,'   Previous: max = ',oldtest_dirtymax
#print >>logfile,'   Difference (fractional) = ',diff_dirtymax
#print >>logfile,''
#
#thistest_dirtyrms=dirtystats['rms'][0]
#oldtest_dirtyrms = 0.00243866862729
#diff_dirtyrms = abs((oldtest_dirtyrms-thistest_dirtyrms)/oldtest_dirtyrms)
#
#print ' Dirty image rms = ',thistest_dirtyrms
#print '   Previous: rms = ',oldtest_dirtyrms
#print '   Difference (fractional) = ',diff_dirtyrms
#print ''
#
#print >>logfile,' Dirty Image rms = ',thistest_dirtyrms
#print >>logfile,'   Previous: rms = ',oldtest_dirtyrms
#print >>logfile,'   Difference (fractional) = ',diff_dirtyrms
#print >>logfile,''

# Now the clean image
#
thistest_immax=cubestats['max'][0]
oldtest_immax = 0.052414759993553162
diff_immax = abs((oldtest_immax-thistest_immax)/oldtest_immax)

print ' Clean image max = ',thistest_immax
print '   Previous: max = ',oldtest_immax
print '   Difference (fractional) = ',diff_immax
print ''

print >>logfile,' Clean Image max = ',thistest_immax
print >>logfile,'   Previous: max = ',oldtest_immax
print >>logfile,'   Difference (fractional) = ',diff_immax
print >>logfile,''

thistest_imrms=cubestats['rms'][0]
oldtest_imrms = 0.0020218724384903908
diff_imrms = abs((oldtest_imrms-thistest_imrms)/oldtest_imrms)

print ' Clean image rms = ',thistest_imrms
print '   Previous: rms = ',oldtest_imrms
print '   Difference (fractional) = ',diff_imrms
print ''

print >>logfile,' Clean image rms = ',thistest_imrms
print >>logfile,'   Previous: rms = ',oldtest_imrms
print >>logfile,'   Difference (fractional) = ',diff_imrms
print >>logfile,''

# Now the moment images
#
thistest_momzeromax=momzerostats['max'][0]
oldtest_momzeromax = 1.40223777294
diff_momzeromax = abs((oldtest_momzeromax-thistest_momzeromax)/oldtest_momzeromax)

print ' Moment 0 image max = ',thistest_momzeromax
print '   Previous: m0 max = ',oldtest_momzeromax
print '   Difference (fractional) = ',diff_momzeromax
print ''

print >>logfile,' Moment 0 image max = ',thistest_momzeromax
print >>logfile,'   Previous: m0 max = ',oldtest_momzeromax
print >>logfile,'   Difference (fractional) = ',diff_momzeromax
print >>logfile,''

thistest_momoneavg=momonestats['mean'][0]
oldtest_momoneavg = 1479.77119646
diff_momoneavg = abs((oldtest_momoneavg-thistest_momoneavg)/oldtest_momoneavg)

print ' Moment 1 image mean = ',thistest_momoneavg
print '   Previous: m1 mean = ',oldtest_momoneavg
print '   Difference (fractional) = ',diff_momoneavg
print ''
print '--- Done ---'

print >>logfile,' Moment 1 image mean = ',thistest_momoneavg
print >>logfile,'   Previous: m1 mean = ',oldtest_momoneavg
print >>logfile,'   Difference (fractional) = ',diff_momoneavg
print >>logfile,''
print >>logfile,'--- Done ---'

# Should see output like:
#
# Clean image max should be  0.0524147599936
# Found : Image Max =  0.0523551553488
# Difference (fractional) =  0.00113717290288
#
# Clean image rms should be  0.00202187243849
# Found : Image rms =  0.00202226242982
# Difference (fractional) =  0.00019288621809
#
# Moment 0 image max should be  1.40223777294
# Found : Moment 0 Max =  1.40230333805
# Difference (fractional) =  4.67574844349e-05
#
# Moment 1 image mean should be  1479.77119646
# Found : Moment 1 Mean =  1479.66974528
# Difference (fractional) =  6.85586935973e-05
#
#=====================================================================
# Done
#
logfile.close()
print "Results are in "+outfile
\end{verbatim}
\normalsize

%%%%%%%%%%%%%%%%%%%%%%%%%%%%%%%%%%%%%%%%%%%%%%%%%%%%%%%%%%%%%%%%%
%%%%%%%%%%%%%%%%%%%%%%%%%%%%%%%%%%%%%%%%%%%%%%%%%%%%%%%%%%%%%%%%%
\section{Jupiter --- VLA continuum polarization}
\label{section:scripts.jupiter}

This script demonstrates continuum polarization calibration and
imaging, including self-calibration.  There is also extensive
interactive flagging, and image analysis.

The latest version of this script can be found at:
\begin{quote}
   \url{http://casa.nrao.edu/Doc/Scripts/jupiter6cm_demo.py}
\end{quote}

\small
\begin{verbatim}
######################################################################
#                                                                    #
# Use Case Script for Jupiter 6cm VLA                                #
# Trimmed down from Use Case jupiter6cm_usecase.py                   #
#                                                                    #
# Updated STM 2008-05-15 (Beta Patch 2.0)                            #
# Updated STM 2008-06-11 (Beta Patch 2.0)                            #
# Updated STM 2008-06-12 (Beta Patch 2.0) for summer school demo     #
# Updated STM 2008-06-13 (Beta Patch 2.0) make a bit faster          #
# Updated STM 2008-12-24 (Beta Patch 3.0) extendflags                #
# Updated STM 2009-05-29 (Beta Patch 4.0) for mcmaster tutorial      #
# Updated STM 2009-12-20 (Release 0)      for Version 3.0.0          #
#                                                                    #
# This is a VLA 6cm dataset that was observed in 1999 to set the     #
# flux scale for calibration of the VLA.  Included in the program    #
# were observations of the planets, including Jupiter.               #
#                                                                    #
# This is D-configuration data, with resolution of around 14"        #
#                                                                    #
# Includes polarization imaging and analysis                         #
#                                                                    #
######################################################################

import time
import os

print "Jupiter 6cm Interactive Tutorial/Demo Script"
print "Version 2009-12-20 (Version 3.0.0 Release 0)"
print ""

# 
#=====================================================================
#
# This script has some interactive commands: scriptmode = True
# if you are running it and want it to stop during interactive parts.

scriptmode = True

#=====================================================================
#
# Set up some useful variables - these will be set during the script
# also, but if you want to restart the script in the middle here
# they are in one place:

# This will prefix all output file names
prefix='jupiter6cm.demo'

# Clean up old files
os.system('rm -rf '+prefix+'*')

# This is the output MS file name
msfile = prefix + '.ms'

#
#=====================================================================
# Calibration variables
#
# Use same prefix as rest of script
calprefix = prefix

# spectral windows to process
usespw = ''
usespwlist = ['0','1']

# prior calibration to apply
usegaincurve = True
gainopacity = 0.0

# reference antenna 11 (11=VLA:N1)
calrefant = '11'

gtable = calprefix + '.gcal'
ftable = calprefix + '.fluxscale'
atable = calprefix + '.accum'

#
#=====================================================================
# Polarization calibration setup
#
dopolcal = True

ptable = calprefix + '.pcal'
xtable = calprefix + '.polx'

# Pol leakage calibrator
poldfield = '0137+331'

# Pol angle calibrator
polxfield = '1331+305'
# At Cband the fractional polarization of this source is 0.112 and
# the R-L PhaseDiff = 66deg (EVPA = 33deg)
polxfpol = 0.112
polxrlpd_deg = 66.0
# Dictionary of IPOL in the spw
polxipol = {'0' : 7.462,
            '1' : 7.510}

# Make Stokes lists for setjy
polxiquv = {}
for spw in ['0','1']:
    ipol = polxipol[spw]
    fpol = polxfpol
    ppol = ipol*fpol
    rlpd = polxrlpd_deg*pi/180.0
    qpol = ppol*cos(rlpd)
    upol = ppol*sin(rlpd)
    polxiquv[spw] = [ipol,qpol,upol,0.0]

#
# Split output setup
#
srcname = 'JUPITER'
srcsplitms = calprefix + '.' + srcname + '.split.ms'
calname = '0137+331'
calsplitms = calprefix + '.' + calname + '.split.ms'

#
#=====================================================================
#
# Intensity imaging parameters
#
# Same prefix for this imaging demo output
#
imprefix = prefix

# This is D-config VLA 6cm (4.85GHz) obs
# Check the observational status summary
# Primary beam FWHM = 45'/f_GHz = 557"
# Synthesized beam FWHM = 14"
# RMS in 10min (600s) = 0.06 mJy (thats now, but close enough)

# Set the output image size and cell size (arcsec)
# 4" will give 3.5x oversampling
clncell = [4.,4.]

# 280 pix will cover to 2xPrimaryBeam
# clean will say to use 288 (a composite integer) for efficiency
clnalg = 'clark'
clnmode = ''
# For Cotton-Schwab use
clnmode = 'csclean'
clnimsize = [288,288]

# iterations
clniter = 10000

# Also set flux residual threshold (0.04 mJy)
# From our listobs:
# Total integration time = 85133.2 seconds
# With rms of 0.06 mJy in 600s ==> rms = 0.005 mJy
# Set to 10x thermal rms
clnthreshold=0.05

#
# Filenames
#
imname1 = imprefix + '.clean1'
clnimage1 = imname1+'.image'
clnmodel1 = imname1+'.model'
clnresid1 = imname1+'.residual'
clnmask1  = imname1+'.clean_interactive.mask'

imname2 = imprefix + '.clean2'
clnimage2 = imname2+'.image'
clnmodel2 = imname2+'.model'
clnresid2 = imname2+'.residual'
clnmask2  = imname2+'.clean_interactive.mask'

imname3 = imprefix + '.clean3'
clnimage3 = imname3+'.image'
clnmodel3 = imname3+'.model'
clnresid3 = imname3+'.residual'
clnmask3  = imname3+'.clean_interactive.mask'

#
# Selfcal parameters
#
# reference antenna 11 (11=VLA:N1)
calrefant = '11'

#
# Filenames
#
selfcaltab1 = imprefix + '.selfcal1.gtable'

selfcaltab2 = imprefix + '.selfcal2.gtable'
smoothcaltab2 = imprefix + '.smoothcal2.gtable'

#
#=====================================================================
#
# Polarization imaging parameters
#
# New prefix for polarization imaging output
#
polprefix = prefix + '.polimg'

# Set up clean slightly differently
polclnalg = 'hogbom'
polclnmode = 'csclean'

polimname = polprefix + '.clean'
polimage  = polimname+'.image'
polmodel  = polimname+'.model'
polresid  = polimname+'.residual'
polmask   = polimname+'.clean_interactive.mask'

#
# Other files
#
ipolimage = polimage+'.I'
qpolimage = polimage+'.Q'
upolimage = polimage+'.U'

poliimage = polimage+'.poli'
polaimage = polimage+'.pola'

#
#=====================================================================
#=====================================================================
# Start processing
#=====================================================================
#
# Get to path to the CASA home and stip off the name
pathname=os.environ.get('CASAPATH').split()[0]

# This is where the UVFITS data should be
#fitsdata=pathname+'/data/demo/jupiter6cm.fits'
# Or
#fitsdata=pathname+'/data/nrao/VLA/planets_6cm.fits'
#fitsdata='/home/ballista/casa/devel/data/nrao/VLA/planets_6cm.fits'
#
# Can also be found online at
#http://casa.nrao.edu/Data/VLA/Planets6cm/planets_6cm.fits

# Use version in current directory
fitsdata='planets_6cm.fits'

#
#=====================================================================
# Data Import and List
#=====================================================================
#
# Import the data from FITS to MS
#
print '--Import--'

# Safest to start from task defaults
default('importuvfits')

print "Use importuvfits to read UVFITS and make an MS"

# Set up the MS filename and save as new global variable
msfile = prefix + '.ms'

print "MS will be called "+msfile

# Use task importuvfits
fitsfile = fitsdata
vis = msfile
importuvfits()

#=====================================================================
#
# List a summary of the MS
#
print '--Listobs--'

# Don't default this one and make use of the previous setting of
# vis.  Remember, the variables are GLOBAL!

print "Use listobs to print verbose summary to logger"

# You may wish to see more detailed information, in this case
# use the verbose = True option
verbose = True

listobs()

# You should get in your logger window and in the casapy.log file
# something like:
#
#    Observer: FLUX99     Project:   
# Observation: VLA
# 
# Data records: 2021424       Total integration time = 85133.2 seconds
#    Observed from   23:15:27   to   22:54:20
# 
#    ObservationID = 0         ArrayID = 0
#   Date        Timerange                Scan  FldId FieldName      SpwIds
#   15-Apr-1999/23:15:26.7 - 23:16:10.0     1      0 0137+331       [0, 1]
#               23:38:40.0 - 23:48:00.0     2      1 0813+482       [0, 1]
#               23:53:40.0 - 23:55:20.0     3      2 0542+498       [0, 1]
#   16-Apr-1999/00:22:10.1 - 00:23:49.9     4      3 0437+296       [0, 1]
#               00:28:23.3 - 00:30:00.1     5      4 VENUS          [0, 1]
#               00:48:40.0 - 00:50:20.0     6      1 0813+482       [0, 1]
#               00:56:13.4 - 00:57:49.9     7      2 0542+498       [0, 1]
#               01:10:20.1 - 01:11:59.9     8      5 0521+166       [0, 1]
#               01:23:29.9 - 01:25:00.1     9      3 0437+296       [0, 1]
#               01:29:33.3 - 01:31:10.0    10      4 VENUS          [0, 1]
#               01:49:50.0 - 01:51:30.0    11      6 1411+522       [0, 1]
#               02:03:00.0 - 02:04:30.0    12      7 1331+305       [0, 1]
#               02:17:30.0 - 02:19:10.0    13      1 0813+482       [0, 1]
#               02:24:20.0 - 02:26:00.0    14      2 0542+498       [0, 1]
#               02:37:49.9 - 02:39:30.0    15      5 0521+166       [0, 1]
#               02:50:50.1 - 02:52:20.1    16      3 0437+296       [0, 1]
#               02:59:20.0 - 03:01:00.0    17      6 1411+522       [0, 1]
#               03:12:30.0 - 03:14:10.0    18      7 1331+305       [0, 1]
#               03:27:53.3 - 03:29:39.9    19      1 0813+482       [0, 1]
#               03:35:00.0 - 03:36:40.0    20      2 0542+498       [0, 1]
#               03:49:50.0 - 03:51:30.1    21      6 1411+522       [0, 1]
#               04:03:10.0 - 04:04:50.0    22      7 1331+305       [0, 1]
#               04:18:49.9 - 04:20:40.0    23      1 0813+482       [0, 1]
#               04:25:56.6 - 04:27:39.9    24      2 0542+498       [0, 1]
#               04:42:49.9 - 04:44:40.0    25      8 MARS           [0, 1]
#               04:56:50.0 - 04:58:30.1    26      6 1411+522       [0, 1]
#               05:24:03.3 - 05:33:39.9    27      7 1331+305       [0, 1]
#               05:48:00.0 - 05:49:49.9    28      1 0813+482       [0, 1]
#               05:58:36.6 - 06:00:30.0    29      8 MARS           [0, 1]
#               06:13:20.1 - 06:14:59.9    30      6 1411+522       [0, 1]
#               06:27:40.0 - 06:29:20.0    31      7 1331+305       [0, 1]
#               06:44:13.4 - 06:46:00.0    32      1 0813+482       [0, 1]
#               06:55:06.6 - 06:57:00.0    33      8 MARS           [0, 1]
#               07:10:40.0 - 07:12:20.0    34      6 1411+522       [0, 1]
#               07:28:20.0 - 07:30:10.1    35      7 1331+305       [0, 1]
#               07:42:49.9 - 07:44:30.0    36      8 MARS           [0, 1]
#               07:58:43.3 - 08:00:39.9    37      6 1411+522       [0, 1]
#               08:13:30.0 - 08:15:19.9    38      7 1331+305       [0, 1]
#               08:27:53.4 - 08:29:30.0    39      8 MARS           [0, 1]
#               08:42:59.9 - 08:44:50.0    40      6 1411+522       [0, 1]
#               08:57:09.9 - 08:58:50.0    41      7 1331+305       [0, 1]
#               09:13:03.3 - 09:14:50.1    42      9 NGC7027        [0, 1]
#               09:26:59.9 - 09:28:40.0    43      6 1411+522       [0, 1]
#               09:40:33.4 - 09:42:09.9    44      7 1331+305       [0, 1]
#               09:56:19.9 - 09:58:10.0    45      9 NGC7027        [0, 1]
#               10:12:59.9 - 10:14:50.0    46      8 MARS           [0, 1]
#               10:27:09.9 - 10:28:50.0    47      6 1411+522       [0, 1]
#               10:40:30.0 - 10:42:00.0    48      7 1331+305       [0, 1]
#               10:56:10.0 - 10:57:50.0    49      9 NGC7027        [0, 1]
#               11:28:30.0 - 11:35:30.0    50     10 NEPTUNE        [0, 1]
#               11:48:20.0 - 11:50:10.0    51      6 1411+522       [0, 1]
#               12:01:36.7 - 12:03:10.0    52      7 1331+305       [0, 1]
#               12:35:33.3 - 12:37:40.0    53     11 URANUS         [0, 1]
#               12:46:30.0 - 12:48:10.0    54     10 NEPTUNE        [0, 1]
#               13:00:29.9 - 13:02:10.0    55      6 1411+522       [0, 1]
#               13:15:23.3 - 13:17:10.1    56      9 NGC7027        [0, 1]
#               13:33:43.3 - 13:35:40.0    57     11 URANUS         [0, 1]
#               13:44:30.0 - 13:46:10.0    58     10 NEPTUNE        [0, 1]
#               14:00:46.7 - 14:01:39.9    59      0 0137+331       [0, 1]
#               14:10:40.0 - 14:12:09.9    60     12 JUPITER        [0, 1]
#               14:24:06.6 - 14:25:40.1    61     11 URANUS         [0, 1]
#               14:34:30.0 - 14:36:10.1    62     10 NEPTUNE        [0, 1]
#               14:59:13.4 - 15:00:00.0    63      0 0137+331       [0, 1]
#               15:09:03.3 - 15:10:40.1    64     12 JUPITER        [0, 1]
#               15:24:30.0 - 15:26:20.1    65      9 NGC7027        [0, 1]
#               15:40:10.0 - 15:45:00.0    66     11 URANUS         [0, 1]
#               15:53:50.0 - 15:55:20.0    67     10 NEPTUNE        [0, 1]
#               16:18:53.4 - 16:19:49.9    68      0 0137+331       [0, 1]
#               16:29:10.1 - 16:30:49.9    69     12 JUPITER        [0, 1]
#               16:42:53.4 - 16:44:30.0    70     11 URANUS         [0, 1]
#               16:54:53.4 - 16:56:40.0    71      9 NGC7027        [0, 1]
#               17:23:06.6 - 17:30:40.0    72      2 0542+498       [0, 1]
#               17:41:50.0 - 17:43:20.0    73      3 0437+296       [0, 1]
#               17:55:36.7 - 17:57:39.9    74      4 VENUS          [0, 1]
#               18:19:23.3 - 18:20:09.9    75      0 0137+331       [0, 1]
#               18:30:23.3 - 18:32:00.0    76     12 JUPITER        [0, 1]
#               18:44:49.9 - 18:46:30.0    77      9 NGC7027        [0, 1]
#               18:59:13.3 - 19:00:59.9    78      2 0542+498       [0, 1]
#               19:19:10.0 - 19:21:20.1    79      5 0521+166       [0, 1]
#               19:32:50.1 - 19:34:29.9    80      3 0437+296       [0, 1]
#               19:39:03.3 - 19:40:40.1    81      4 VENUS          [0, 1]
#               20:08:06.7 - 20:08:59.9    82      0 0137+331       [0, 1]
#               20:18:10.0 - 20:19:50.0    83     12 JUPITER        [0, 1]
#               20:33:53.3 - 20:35:40.1    84      1 0813+482       [0, 1]
#               20:40:59.9 - 20:42:40.0    85      2 0542+498       [0, 1]
#               21:00:16.6 - 21:02:20.1    86      5 0521+166       [0, 1]
#               21:13:53.4 - 21:15:29.9    87      3 0437+296       [0, 1]
#               21:20:43.4 - 21:22:30.0    88      4 VENUS          [0, 1]
#               21:47:26.7 - 21:48:20.1    89      0 0137+331       [0, 1]
#               21:57:30.0 - 21:59:10.0    90     12 JUPITER        [0, 1]
#               22:12:13.3 - 22:14:00.1    91      2 0542+498       [0, 1]
#               22:28:33.3 - 22:30:19.9    92      4 VENUS          [0, 1]
#               22:53:33.3 - 22:54:19.9    93      0 0137+331       [0, 1]
# 
# Fields: 13
#   ID   Name          Right Ascension  Declination   Epoch   
#   0    0137+331      01:37:41.30      +33.09.35.13  J2000   
#   1    0813+482      08:13:36.05      +48.13.02.26  J2000   
#   2    0542+498      05:42:36.14      +49.51.07.23  J2000   
#   3    0437+296      04:37:04.17      +29.40.15.14  J2000   
#   4    VENUS         04:06:54.11      +22.30.35.91  J2000   
#   5    0521+166      05:21:09.89      +16.38.22.05  J2000   
#   6    1411+522      14:11:20.65      +52.12.09.14  J2000   
#   7    1331+305      13:31:08.29      +30.30.32.96  J2000   
#   8    MARS          14:21:41.37      -12.21.49.45  J2000   
#   9    NGC7027       21:07:01.59      +42.14.10.19  J2000   
#   10   NEPTUNE       20:26:01.14      -18.54.54.21  J2000   
#   11   URANUS        21:15:42.83      -16.35.05.59  J2000   
#   12   JUPITER       00:55:34.04      +04.45.44.71  J2000   
# 
# Spectral Windows: (2 unique spectral windows and 1 unique polarization setups)
#   SpwID  #Chans Frame Ch1(MHz)    Resoln(kHz) TotBW(kHz)  Ref(MHz)    Corrs           
#   0           1 TOPO  4885.1      50000       50000       4885.1      RR  RL  LR  LL  
#   1           1 TOPO  4835.1      50000       50000       4835.1      RR  RL  LR  LL  
# 
# Feeds: 28: printing first row only
#   Antenna   Spectral Window     # Receptors    Polarizations
#   1         -1                  2              [         R, L]
# 
# Antennas: 27:
#   ID   Name  Station   Diam.    Long.         Lat.         
#   0    1     VLA:W9    25.0 m   -107.37.25.1  +33.53.51.0  
#   1    2     VLA:N9    25.0 m   -107.37.07.8  +33.54.19.0  
#   2    3     VLA:N3    25.0 m   -107.37.06.3  +33.54.04.8  
#   3    4     VLA:N5    25.0 m   -107.37.06.7  +33.54.08.0  
#   4    5     VLA:N2    25.0 m   -107.37.06.2  +33.54.03.5  
#   5    6     VLA:E1    25.0 m   -107.37.05.7  +33.53.59.2  
#   6    7     VLA:E2    25.0 m   -107.37.04.4  +33.54.01.1  
#   7    8     VLA:N8    25.0 m   -107.37.07.5  +33.54.15.8  
#   8    9     VLA:E8    25.0 m   -107.36.48.9  +33.53.55.1  
#   9    10    VLA:W3    25.0 m   -107.37.08.9  +33.54.00.1  
#   10   11    VLA:N1    25.0 m   -107.37.06.0  +33.54.01.8  
#   11   12    VLA:E6    25.0 m   -107.36.55.6  +33.53.57.7  
#   12   13    VLA:W7    25.0 m   -107.37.18.4  +33.53.54.8  
#   13   14    VLA:E4    25.0 m   -107.37.00.8  +33.53.59.7  
#   14   15    VLA:N7    25.0 m   -107.37.07.2  +33.54.12.9  
#   15   16    VLA:W4    25.0 m   -107.37.10.8  +33.53.59.1  
#   16   17    VLA:W5    25.0 m   -107.37.13.0  +33.53.57.8  
#   17   18    VLA:N6    25.0 m   -107.37.06.9  +33.54.10.3  
#   18   19    VLA:E7    25.0 m   -107.36.52.4  +33.53.56.5  
#   19   20    VLA:E9    25.0 m   -107.36.45.1  +33.53.53.6  
#   21   22    VLA:W8    25.0 m   -107.37.21.6  +33.53.53.0  
#   22   23    VLA:W6    25.0 m   -107.37.15.6  +33.53.56.4  
#   23   24    VLA:W1    25.0 m   -107.37.05.9  +33.54.00.5  
#   24   25    VLA:W2    25.0 m   -107.37.07.4  +33.54.00.9  
#   25   26    VLA:E5    25.0 m   -107.36.58.4  +33.53.58.8  
#   26   27    VLA:N4    25.0 m   -107.37.06.5  +33.54.06.1  
#   27   28    VLA:E3    25.0 m   -107.37.02.8  +33.54.00.5  
# 
# Tables:
#    MAIN                 2021424 rows     
#    ANTENNA                   28 rows     
#    DATA_DESCRIPTION           2 rows     
#    DOPPLER             <absent>  
#    FEED                      28 rows     
#    FIELD                     13 rows     
#    FLAG_CMD             <empty>  
#    FREQ_OFFSET         <absent>  
#    HISTORY                 7058 rows     
#    OBSERVATION                1 row      
#    POINTING                2604 rows     
#    POLARIZATION               1 row      
#    PROCESSOR            <empty>  
#    SOURCE               <empty> (see FIELD)
#    SPECTRAL_WINDOW            2 rows     
#    STATE                <empty>  
#    SYSCAL              <absent>  
#    WEATHER             <absent>  

# 
#=====================================================================
# Data Examination and Flagging
#=====================================================================
# 
# Use Plotxy to interactively flag the data
#
print '--Plotxy--'
default('plotxy')

print "Now we use plotxy to examine and interactively flag data"

vis = msfile

# The fields we are interested in: 1331+305,JUPITER,0137+331
selectdata = True

# First we do the primary calibrator
field = '1331+305'

# Plot only the RR and LL for now
correlation = 'RR LL'

# As of 2.3.0 (Patch 3) you can extend the flags to the cross-correlations
# But this slows things down immensely
#extendflag = T
#extendcorr = 'all'

# Plot amplitude vs. uvdist
xaxis = 'uvdist'
yaxis = 'amp'
multicolor = 'both'

# Use the field name as the title
selectplot = True
title = field+"  "

iteration = ''

plotxy()

print ""
print "-----------------------------------------------------"
print "Plotxy"
print "Showing 1331+305 RR LL for all antennas"
print "Use MarkRegion then draw boxes around points to flag"
print "You can use ESC to drop last drawn box"
print "When happy with boxes, hit Flag to flag"
print "You can repeat as necessary"
print ""
#print "NOTE: These flags will extend to the RL LR cross-hands"
#print "Because of this the flagging will be slower than otherwise"

# Pause script if you are running in scriptmode
if scriptmode:
    user_check=raw_input('Return to continue script\n')

# You can also use flagdata to do this non-interactively
# (see below)

# Now look at the cross-polar products
correlation = 'RL LR'
extendflag = F

plotxy()

print ""
print "-----------------------------------------------------"
print "Looking at RL LR"
print "Now flag any remaining bad data here"

# Pause script if you are running in scriptmode
if scriptmode:
    user_check=raw_input('Return to continue script\n')

#---------------------------------------------------------------------
# Now do calibrater 0137+331
field = '0137+331'
correlation = 'RR LL'
xaxis = 'uvdist'
spw = ''
iteration = ''
antenna = ''

# As of 2.3.0 (Patch 3) you can extend the flags to the cross-correlations
# But this slows things down immensely
#extendflag = T
#extendcorr = 'all'

title = field+"  "

plotxy()

# You'll see a bunch of bad data along the bottom near zero amp
# Draw a box around some of it and use Locate
# Looks like much of it is Antenna 9 (ID=8) in spw=1

print ""
print "-----------------------------------------------------"
print "Plotting 0137+331 RR LL all antennas"
print "You see bad data along bottom"
print "Mark a box around a bit of it and hit Locate"
print "Look in logger to see what it is"
print "You see much is Antenna 9 (ID=8) in spw 1"

# Pause script if you are running in scriptmode
if scriptmode:
    user_check=raw_input('Return to continue script\n')

xaxis = 'time'
spw = '1'
correlation = ''
extendflag = F

# Note that the strings like antenna='9' first try to match the 
# NAME which we see in listobs was the number '9' for ID=8.
# So be careful here (why naming antennas as numbers is bad).
antenna = '9'

plotxy()

# YES! the last 4 scans are bad.  Box 'em and flag.

print ""
print "-----------------------------------------------------"
print "Plotting vs. time antenna='9' and spw='1' "
print "Box up last 4 scans which are bad and Flag"

# Pause script if you are running in scriptmode
if scriptmode:
    user_check=raw_input('Return to continue script\n')

# Go back and clean up
xaxis = 'uvdist'
spw = ''
antenna = ''

correlation = 'RR LL'

# Note that RL,LR are too weak to clip on.
# As of 2.3.0 (Patch 3) you can extend the flags to the cross-correlations
# But this slows things down immensely
#extendflag = T
#extendcorr = 'all'

plotxy()

# Box up the bad low points (basically a clip below 0.52) and flag

print ""
print "-----------------------------------------------------"
print "Back to all data"
print "Clean up remaining bad points"
print ""
#print "NOTE: These flags will extend to the RL LR cross-hands"
#print "Because of this the flagging will be slower than otherwise"

# Pause script if you are running in scriptmode
if scriptmode:
    user_check=raw_input('Return to continue script\n')

#---------------------------------------------------------------------
# Finally, do JUPITER
field = 'JUPITER'
correlation = 'RR LL'
iteration = ''
xaxis = 'uvdist'

title = field+"  "

plotxy()

# Here you will see that the final scan at 22:00:00 UT is bad
# Draw a box around it and flag it!

print ""
print "-----------------------------------------------------"
print "Now plot JUPITER versus uvdist"
print "Lots of bad stuff near bottom"
print "Lets go and find it - try Locate"
print "Looks like lots of different antennas but at same time"

# Pause script if you are running in scriptmode
if scriptmode:
    user_check=raw_input('Return to continue script\n')

correlation = ''
extendflag = F
xaxis = 'time'

plotxy()

# Here you will see that the final scan at 22:00:00 UT is bad
# Draw a box around it and flag it!

print ""
print "-----------------------------------------------------"
print "Now plotting vs. time"
print "See bad scan at end - flag it!"

# Pause script if you are running in scriptmode
if scriptmode:
    user_check=raw_input('Return to continue script\n')

# Now look at whats left
correlation = 'RR LL'
# As of 2.3.0 (Patch 3) you can extend the flags to the cross-correlations
# But this slows things down immensely
#extendflag = T
#extendcorr = 'all'

xaxis = 'uvdist'
spw = '1'
antenna = ''
iteration = 'antenna'

plotxy()

# As you step through, you will see that Antenna 9 (ID=8) is often 
# bad in this spw. If you box and do Locate (or remember from
# 0137+331) its probably a bad time.

print ""
print "-----------------------------------------------------"
print "Looking now at SPW 1"
print "Now we set iteration to Antenna"
print "Step through antennas with Next"
print "See bad Antenna 9 (ID 8) as in 0137+331"
print "Do not flag yet, we will isolate this next"

# Pause script if you are running in scriptmode
if scriptmode:
    user_check=raw_input('Return to continue script\n')

# The easiset way to kill it:

antenna = '9'
iteration = ''
xaxis = 'time'
correlation = ''
extendflag = F

plotxy()

# Draw a box around all points in the last bad scans and flag 'em!

print ""
print "-----------------------------------------------------"
print "Now plotting vs. time antenna 9 spw 1"
print "Box up the bad scans and Flag"

# Pause script if you are running in scriptmode
if scriptmode:
    user_check=raw_input('Return to continue script\n')

# Now clean up the rest
xaxis = 'uvdist'
correlation = 'RR LL'
# As of 2.3.0 (Patch 3) you can extend the flags to the cross-correlations
# But this slows things down immensely
#extendflag = T
#extendcorr = 'all'

antenna = ''
spw = ''

# You will be drawing many tiny boxes, so remember you can
# use the ESC key to get rid of the most recent box if you
# make a mistake.

plotxy()

# Note that the end result is we've flagged lots of points
# in RR and LL.  We will rely upon imager to ignore the
# RL LR for points with RR LL flagged!

print ""
print "-----------------------------------------------------"
print "Final cleanup of JUPITER data"
print "Back to uvdist plot, see remaining bad data"
print "You can draw little boxes around the outliers and Flag"
print "Depends how patient you are in drawing boxes!"
print "Could also use Locate to find where they come from"
print ""
#print "NOTE: These flags will extend to the RL LR cross-hands"
#print "Because of this the flagging will be slower than otherwise"

# Pause script if you are running in scriptmode
if scriptmode:
    user_check=raw_input('Return to continue script\n')

print "Done with plotxy!"

#
#=====================================================================
#
# Use Flagmanager to save a copy of the flags so far
#
print '--Flagmanager--'
default('flagmanager')

print "Now will use flagmanager to save a copy of the flags we just made"
print "These are named xyflags"

vis = msfile
mode = 'save'
versionname = 'xyflags'
comment = 'Plotxy flags'
merge = 'replace'

flagmanager()

#=====================================================================
#
# Use Flagmanager to list all saved versions
#
print '--Flagmanager--'
default('flagmanager')

print "Now will use flagmanager to list all the versions we saved"

vis = msfile
mode = 'list'

flagmanager()

#
# Done Flagging
print '--Done with flagging--'

#
#=====================================================================
# Calibration
#=====================================================================
#
# Set the fluxes of the primary calibrator(s)
#
print '--Setjy--'
default('setjy')

print "Use setjy to set flux of 1331+305 (3C286)"

vis = msfile

#
# 1331+305 = 3C286 is our primary calibrator
field = '1331+305'     

# Setjy knows about this source so we dont need anything more

setjy()

#
# You should see something like this in the logger and casapy.log file:
#
# 1331+305  spwid=  0  [I=7.462, Q=0, U=0, V=0] Jy, (Perley-Taylor 99)
# 1331+305  spwid=  1  [I=7.51, Q=0, U=0, V=0] Jy, (Perley-Taylor 99)
# 

print "Look in logger for the fluxes (should be 7.462 and 7.510 Jy)"

#
#=====================================================================
#
# Initial gain calibration
#
print '--Gaincal--'
default('gaincal')

print "Solve for antenna gains on 1331+305 and 0137+331"
print "We have 2 single-channel continuum spw"
print "Do not want bandpass calibration"

vis = msfile

# set the name for the output gain caltable
caltable = gtable

print "Output gain cal table will be "+gtable

# Gain calibrators are 1331+305 and 0137+331 (FIELD_ID 7 and 0)
# We have 2 IFs (SPW 0,1) with one channel each

# selection is via the field and spw strings
field = '1331+305,0137+331'
spw = ''

# a-priori calibration application
gaincurve = usegaincurve
opacity = gainopacity

# scan-based G solutions for both amplitude and phase
gaintype = 'G'
calmode = 'ap'

# one solution per scan
solint = 'inf'
combine = ''

# do not apply parallactic angle correction (yet)
parang = False

# reference antenna
refant = calrefant

# minimum SNR 3
minsnr = 3

gaincal()

#
#=====================================================================
#
# Bootstrap flux scale
#
print '--Fluxscale--'
default('fluxscale')

print "Use fluxscale to rescale gain table to make new one"

vis = msfile

# set the name for the output rescaled caltable
fluxtable = ftable

print "Output scaled gain cal table is "+ftable

# point to our first gain cal table
caltable = gtable

# we will be using 1331+305 (the source we did setjy on) as
# our flux standard reference
reference = '1331+305'

# we want to transfer the flux to our other gain cal source 0137+331
# to bring its gain amplitues in line with the absolute scale
transfer = '0137+331'

fluxscale()

# You should see in the logger something like:
#Flux density for 0137+331 in SpW=0 is:
#   5.42575 +/- 0.00285011 (SNR = 1903.7, nAnt= 27)
#Flux density for 0137+331 in SpW=1 is:
#   5.46569 +/- 0.00301326 (SNR = 1813.88, nAnt= 27)

#
#---------------------------------------------------------------------
# Plot calibration
#
print '--PlotCal--'
default('plotcal')

showgui = True
    
caltable = ftable
multiplot = True
yaxis = 'amp'

showgui = True
    
plotcal()

print ""
print "-------------------------------------------------"
print "Plotcal"
print "Looking at amplitude in cal-table "+caltable

# Pause script if you are running in scriptmode
if scriptmode:
    user_check=raw_input('Return to continue script\n')

#
# Now go back and plot to file
#
showgui = False

yaxis = 'amp'

#figfile = caltable + '.plotcal.amp.png'
#print "Plotting calibration to file "+figfile
#saveinputs('plotcal',caltable.plotcal.amp.saved')
#plotcal()

yaxis = 'phase'

#figfile = caltable + '.plotcal.phase.png'
#print "Plotting calibration to file "+figfile
#saveinputs('plotcal',caltable.plotcal.phase.saved')
#plotcal()

#
#=====================================================================
# Polarization Calibration
#=====================================================================
#
if (dopolcal):
    print '--Polcal (D)--'
    default('polcal')
    
    print "Solve for polarization leakage on 0137+331"
    print "Pretend it has unknown polarization"

    vis = msfile

    # Start with the un-fluxscaled gain table
    gaintable = gtable

    # use settings from gaincal
    gaincurve = usegaincurve
    opacity = gainopacity
    
    # Output table
    caltable = ptable

    # Use a 3C48 tracked through a range of PA
    field = '0137+331'
    spw = ''

    # No need for further selection
    selectdata=False

    # Polcal mode (D+QU = unknown pol for D)
    poltype = 'D+QU'

    # One solution for entire dataset
    solint = 'inf'
    combine = 'scan'

    # reference antenna
    refant = calrefant

    # minimum SNR 3
    minsnr = 3

    #saveinputs('polcal',calprefix+'.polcal.saved')
    polcal()
    
    #=====================================================================
    #
    # List polcal solutions
    #
    print '--Listcal (PolD)--'

    listfile = caltable + '.list'

    print "Listing calibration to file "+listfile

    listcal()
    
    #=====================================================================
    #
    # Plot polcal solutions
    #
    print '--Plotcal (PolD)--'
    
    iteration = ''
    showgui = False
    
    xaxis = 'antenna'
    yaxis = 'amp'

    showgui = True
    figfile = ''

    plotcal()

    print "These are the amplitudes of D-terms versus antenna"

    # Pause script if you are running in scriptmode
    if scriptmode:
        user_check=raw_input('Return to continue script\n')

    # Now plot to files
    showgui = False

    #figfile = caltable + '.plotcal.antamp.png'
    #print "Plotting calibration to file "+figfile
    #saveinputs('plotcal',caltable+'.plotcal.antamp.saved')
    #plotcal()

    xaxis = 'antenna'
    yaxis = 'phase'

    #figfile = caltable + '.plotcal.antphase.png'
    #print "Plotting calibration to file "+figfile
    #saveinputs('plotcal',caltable+'.plotcal.antphase.saved')
    #plotcal()

    xaxis = 'antenna'
    yaxis = 'snr'

    #figfile = caltable + '.plotcal.antsnr.png'
    #print "Plotting calibration to file "+figfile
    #saveinputs('plotcal',caltable+'.plotcal.antsnr.saved')
    #plotcal()

    xaxis = 'real'
    yaxis = 'imag'

    #figfile = caltable + '.plotcal.reim.png'
    #print "Plotting calibration to file "+figfile
    #saveinputs('plotcal',caltable+'.plotcal.reim.saved')
    #plotcal()

    #=====================================================================
    # Do Chi (X) pol angle calibration
    #=====================================================================
    # First set the model
    print '--Setjy--'
    default('setjy')
        
    vis = msfile
        
    print "Use setjy to set IQU fluxes of "+polxfield
    field = polxfield
    
    for spw in usespwlist:
        fluxdensity = polxiquv[spw]
        
        #saveinputs('setjy',calprefix+'.setjy.polspw.'+spw+'.saved')
        setjy()
    
    #
    # Polarization (X-term) calibration
    #
    print '--PolCal (X)--'
    default('polcal')
    
    print "Polarization R-L Phase Calibration (linear approx)"
    
    vis = msfile
    
    # Start with the G and D tables
    gaintable = [gtable,ptable]
    
    # use settings from gaincal
    gaincurve = usegaincurve
    opacity = gainopacity
    
    # Output table
    caltable = xtable

    # previously set with setjy
    field = polxfield
    spw = ''
    
    selectdata=False
    
    # Solve for Chi
    poltype = 'X'
    solint = 'inf'
    combine = 'scan'
    
    # reference antenna
    refant = calrefant
    
    # minimum SNR 3
    minsnr = 3
    
    #saveinputs('polcal',calprefix+'.polcal.X.saved')
    polcal()
    
#=====================================================================
# Apply the Calibration
#=====================================================================
#
# Interpolate the gains onto Jupiter (and others)
#
# print '--Accum--'
# default('accum')
# 
# print "This will interpolate the gains onto Jupiter"
# 
# vis = msfile
# 
# tablein = ''
# incrtable = ftable
# calfield = '1331+305, 0137+331'
# 
# # set the name for the output interpolated caltable
# caltable = atable
# 
# print "Output cumulative gain table will be "+atable
# 
# # linear interpolation
# interp = 'linear'
# 
# # make 10s entries
# accumtime = 10.0
# 
# accum()
#
# NOTE: bypassing this during testing
atable = ftable

# #=====================================================================
#
# Correct the data
# (This will put calibrated data into the CORRECTED_DATA column)
#
print '--ApplyCal--'
default('applycal')

print "This will apply the calibration to the DATA"
print "Fills CORRECTED_DATA"

vis = msfile

# Start with the interpolated fluxscale/gain table
gaintable = [atable,ptable,xtable]

# use settings from gaincal
gaincurve = usegaincurve
opacity = gainopacity

# select the fields
field = '1331+305,0137+331,JUPITER'
spw = ''
selectdata = False

# IMPORTANT set parang=True for polarization
parang = True

# do not need to select subset since we did accum
# (note that correct only does 'nearest' interp)
gainfield = ''

applycal()

#
#=====================================================================
#
# Now split the Jupiter target data
#
print '--Split Jupiter--'
default('split')

vis = msfile

# Now we write out the corrected data to a new MS

# Select the Jupiter field
field = srcname
spw = ''

# pick off the CORRECTED_DATA column
datacolumn = 'corrected'

# Make an output vis file
outputvis = srcsplitms

print "Split "+field+" data into new ms "+srcsplitms

split()

# Also split out 0137+331 as a check
field = calname

outputvis = calsplitms

print "Split "+field+" data into new ms "+calsplitms

split()

#=====================================================================
# Force scratch column creation so plotxy will work
#
vis = srcsplitms
clearcal()

vis = calsplitms
clearcal()

#=====================================================================
# Use Plotxy to look at the split calibrated data
#
print '--Plotxy--'
default('plotxy')

selectdata = True
correlation = 'RR LL'
xaxis = 'uvdist'
datacolumn = 'data'
multicolor = 'both'
iteration = ''
selectplot = True
field = 'JUPITER'

#vis = srcsplitms
#interactive = True
#yaxis = 'amp'
#title = field+"  "
#
# Plotxy interactively if desired
#plotxy()
#
#print ""
#print "-----------------------------------------------------"
#print "Plotting JUPITER corrected visibilities"
#print "Look for outliers"

# Pause script if you are running in scriptmode
#if scriptmode:
#    user_check=raw_input('Return to continue script\n')
#
# Now go back and plot to files
interactive = False

#
# First the target
#
vis = srcsplitms
field = srcname
yaxis = 'amp'
# Use the field name as the title
title = field+"  "

#figfile = vis + '.plotxy.amp.png'
#print "Plotting to file "+figfile
#saveinputs('plotxy',vis+'.plotxy.amp.saved')
#plotxy()

yaxis = 'phase'
# Use the field name as the title

#figfile = vis + '.plotxy.phase.png'
#print "Plotting to file "+figfile
#saveinputs('plotxy',vis+'.plotxy.phase.saved')
#plotxy()

#
# Now the calibrator
#
vis = calsplitms
field = calname
yaxis = 'amp'
# Use the field name as the title
title = field+"  "

#figfile = vis + '.plotxy.amp.png'
#print "Plotting to file "+figfile
#saveinputs('plotxy',vis+'.plotxy.amp.saved')
#plotxy()

yaxis = 'phase'

#figfile = vis + '.plotxy.phase.png'
#print "Plotting to file "+figfile
#saveinputs('plotxy',vis+'.plotxy.phase.saved')
#plotxy()

print 'Calibration completed'
#
#=====================================================================
#
# Intensity Imaging/Selfcal
#
#=====================================================================
#
# Make the scratch columns in the split ms
#
print '--Clearcal--'
default('clearcal')

vis = srcsplitms

clearcal()

print "Created scratch columns for MS "+vis
print ""
#
#=====================================================================
# FIRST CLEAN / SELFCAL CYCLE
#=====================================================================
#
# Now clean an image of Jupiter
# NOTE: this uses the new combined invert/clean/mosaic task Patch 2
#
print '--Clean 1--'
default('clean')

# Pick up our split source data
vis = srcsplitms

# Make an image root file name
imagename = imname1

print "Output images will be prefixed with "+imname1

# Set up the output continuum image (single plane mfs)
mode = 'mfs'
stokes = 'I'

print "Will be a single MFS continuum image"

# NOTE: current version field='' doesnt work
field = '*'

# Combine all spw
spw = ''

# Imaging mode params
psfmode = clnalg
imagermode = clnmode

# Imsize and cell
imsize = clnimsize
cell = clncell

# NOTE: will eventually have an imadvise task to give you this
# information

# Standard gain factor 0.1
gain = 0.1

# Fix maximum number of iterations and threshold
niter = clniter
threshold = clnthreshold

# Note - we can change niter and threshold interactively
# during clean

# Set up the weighting
# Use Briggs weighting (a moderate value, on the uniform side)
weighting = 'briggs'
robust = 0.5

# No clean mask or box
mask = ''

# Use interactive clean mode
interactive = True

# We will be doing selfcal
calready = True

# Moderate number of iter per interactive cycle
npercycle = 100

saveinputs('clean',imagename+'.clean.saved')
clean()

# When the interactive clean window comes up, use the right-mouse
# to draw rectangles around obvious emission double-right-clicking
# inside them to add to the flag region.  You can also assign the
# right-mouse to polygon region drawing by right-clicking on the
# polygon drawing icon in the toolbar.  When you are happy with
# the region, click 'Done Flagging' and it will go and clean another
# 100 iterations.  When done, click 'Stop'.

print ""
print "----------------------------------------------------"
print "Clean"
print "Final clean model is "+clnmodel1
print "Final restored clean image is "+clnimage1
print "The clean residual image is "+clnresid1
print "Your final clean mask is "+clnmask1

print ""
print "This is the final restored clean image in the viewer"
print "Zoom in and set levels to see faint emission"
print "Use rectangle drawing tool to box off source"
print "Double-click inside to print statistics"
print "Move box on-source and get the max"
print "Calcualte DynRange = MAXon/RMSoff"
print "I got 1.060/0.004 = 270"
print "Still not as good as it can be - lets selfcal"
print "Close viewer panel when done"

#
#---------------------------------------------------------------------
#
# If you did not do interactive clean, bring up viewer manually
viewer(clnimage1,'image')

# Pause script if you are running in scriptmode
if scriptmode:
    user_check=raw_input('Return to continue script\n')

# You can use the right-mouse to draw a box in the lower right
# corner of the image away from emission, the double-click inside
# to bring up statistics.  Use the right-mouse to grab this box
# and move it up over Jupiter and double-click again.  You should
# see stuff like this in the terminal:
#
# jupiter6cm.demo.clean1.image     (Jy/beam)
# 
# n           Std Dev     RMS         Mean        Variance    Sum
# 4712        0.003914    0.003927    0.0003205   1.532e-05   1.510     
# 
# Flux        Med |Dev|   IntQtlRng   Median      Min         Max
# 0.09417     0.002646    0.005294    0.0001885   -0.01125    0.01503   
#
#
# On Jupiter:
#
# n           Std Dev     RMS         Mean        Variance    Sum
# 3640        0.1007      0.1027      0.02023     0.01015     73.63     
# 
# Flux        Med |Dev|   IntQtlRng   Median      Min         Max
# 4.592       0.003239    0.007120    0.0001329   -0.01396    1.060     
#
# Estimated dynamic range = 1.060 / 0.003927 = 270 (poor)
#
# Note that the exact numbers you get will depend on how deep you
# take the interactive clean and how you draw the box for the stats.

#=====================================================================
#
# Do some non-interactive image statistics
print '--Imstat--'
default('imstat')

imagename = clnimage1
on_statistics1 = imstat()

# Now do stats in the lower right corner of the image
# remember clnimsize = [288,288]
box = '216,1,287,72'
off_statistics1 = imstat()

# Pull the max and rms from the clean image
thistest_immax=on_statistics1['max'][0]
print ' Found : Max in image = ',thistest_immax
thistest_imrms=off_statistics1['rms'][0]
print ' Found : rms in image = ',thistest_imrms
print ' Clean image Dynamic Range = ',thistest_immax/thistest_imrms
print ''
#
#---------------------------------------------------------------------
#
# Self-cal using clean model
#
# Note: clean will have left FT of model in the MODEL_DATA column
# If you've done something in between, can use the ft task to
# do this manually.
#
print '--SelfCal 1--'
default('gaincal')

vis = srcsplitms

print "Will self-cal using MODEL_DATA left in MS by clean"

# New gain table
caltable = selfcaltab1

print "Will write gain table "+selfcaltab1

# Don't need a-priori cals
selectdata = False
gaincurve = False
opacity = 0.0

# This choice seemed to work
refant = calrefant

# Do amp and phase
gaintype = 'G'
calmode = 'ap'

# Do 30s solutions with SNR>1
solint = 30.0
minsnr = 1.0
print "Calibrating amplitudes and phases on 30s timescale"

# Do not need to normalize (let gains float)
solnorm = False

gaincal()

#
#---------------------------------------------------------------------
# It is useful to put this up in plotcal
#
#
print '--PlotCal--'
default('plotcal')

caltable = selfcaltab1
multiplot = True
yaxis = 'amp'

plotcal()

print ""
print "-------------------------------------------------"
print "Plotcal"
print "Looking at amplitude in self-cal table "+caltable

# Pause script if you are running in scriptmode
if scriptmode:
    user_check=raw_input('Return to continue script\n')

yaxis = 'phase'

plotcal()

print ""
print "-------------------------------------------------"
print "Plotcal"
print "Looking at phases in self-cal table "+caltable

#
# Pause script if you are running in scriptmode
if scriptmode:
    user_check=raw_input('Return to continue script\n')

#
#---------------------------------------------------------------------
#
# Correct the data (no need for interpolation this stage)
#
print '--ApplyCal--'
default('applycal')

vis = srcsplitms

print "Will apply self-cal table to over-write CORRECTED_DATA in MS"

gaintable = selfcaltab1

gaincurve = False
opacity = 0.0
field = ''
spw = ''
selectdata = False

calwt = True

applycal()

# Self-cal is now in CORRECTED_DATA column of split ms
#=====================================================================
# Use Plotxy to look at the self-calibrated data
#
#print '--Plotxy--'
#default('plotxy')
#
#vis = srcsplitms
#selectdata = True
#field = 'JUPITER'
#correlation = 'RR LL'
#xaxis = 'uvdist'
#yaxis = 'amp'
#datacolumn = 'corrected'
#multicolor = 'both'
#selectplot = True
#title = field+"  "
#
#iteration = ''
#
#plotxy()
#
#print ""
#print "-----------------------------------------------------"
#print "Plotting JUPITER self-corrected visibilities"
#print "Look for outliers, and you can flag them"

# Pause script if you are running in scriptmode
#if scriptmode:
#    user_check=raw_input('Return to continue script\n')
#

#
#=====================================================================
# SECOND CLEAN / SELFCAL CYCLE
#=====================================================================
#
print '--Clean 2--'
default('clean')

print "Now clean on self-calibrated data"

vis = srcsplitms

imagename = imname2

field = '*'
spw = ''
mode = 'mfs'
gain = 0.1

# Imaging mode params
psfmode = clnalg
imagermode = clnmode
imsize = clnimsize
cell = clncell
niter = clniter
threshold = clnthreshold

weighting = 'briggs'
robust = 0.5

mask = ''
interactive = True
npercycle = 100
calready = True

saveinputs('clean',imagename+'.clean.saved')
clean()

print ""
print "----------------------------------------------------"
print "Clean"
print "Final clean model is "+clnmodel2
print "Final restored clean image is "+clnimage2
print "The clean residual image is "+clnresid2
print "Your final clean mask is "+clnmask2

print ""
print "This is the final restored clean image in the viewer"
print "Zoom in and set levels to see faint emission"
print "Use rectangle drawing tool to box off source"
print "Double-click inside to print statistics"
print "Move box on-source and get the max"
print "Calculate DynRange = MAXon/RMSoff"
print "This time I got 1.050 / 0.001 = 1050 (better)"
print "Still not as good as it can be - you can try selfcal again"
print "We will stop here"
print "Close viewer panel when done"

#
#---------------------------------------------------------------------
#
# If you did not do interactive clean, bring up viewer manually
viewer(clnimage2,'image')

# Pause script if you are running in scriptmode
if scriptmode:
    user_check=raw_input('Return to continue script\n')

# jupiter6cm.demo.clean2.image     (Jy/beam)
# 
# n           Std Dev     RMS         Mean        Variance    Sum
# 5236        0.001389    0.001390    3.244e-05   1.930e-06   0.1699    
# 
# Flux        Med |Dev|   IntQtlRng   Median      Min         Max
# 0.01060     0.0009064   0.001823    -1.884e-05  -0.004015   0.004892  
# 
# 
# On Jupiter:
# 
# n           Std Dev     RMS         Mean        Variance    Sum
# 5304        0.08512     0.08629     0.01418     0.007245    75.21     
# 
# Flux        Med |Dev|   IntQtlRng   Median      Min         Max
# 4.695       0.0008142   0.001657    0.0001557   -0.004526   1.076     
#
# Estimated dynamic range = 1.076 / 0.001389 = 775 (better)
#
# Note that the exact numbers you get will depend on how deep you
# take the interactive clean and how you draw the box for the stats.
#
print ""
print "--------------------------------------------------"
print "After this script is done you can continue on with"
print "more self-cal, or try different cleaning options"

#
#=====================================================================
# Image Analysis
#=====================================================================
#
# Can do some image statistics if you wish
print '--Imstat (Cycle 2)--'
default('imstat')

imagename = clnimage2
on_statistics2 = imstat()

# Now do stats in the lower right corner of the image
# remember clnimsize = [288,288]
box = '216,1,287,72'
off_statistics2 = imstat()

# Pull the max and rms from the clean image
thistest_immax=on_statistics2['max'][0]
print ' Found : Max in image = ',thistest_immax
thistest_imrms=off_statistics2['rms'][0]
print ' Found : rms in image = ',thistest_imrms
print ' Clean image Dynamic Range = ',thistest_immax/thistest_imrms
print ''

#=====================================================================
#
# Print results and regression versus previous runs
#
print ""
print ' Final Jupiter results '
print ' ===================== '
print ''
# Pull the max and rms from the clean image
thistest_immax=on_statistics2['max'][0]
oldtest_immax = 1.07732224464
print '   Clean image  ON-SRC max = ',thistest_immax
print '   Previously found to be  = ',oldtest_immax
diff_immax = abs((oldtest_immax-thistest_immax)/oldtest_immax)
print '   Difference (fractional) = ',diff_immax

print ''
thistest_imrms=off_statistics2['rms'][0]
oldtest_imrms = 0.0010449
print '   Clean image OFF-SRC rms = ',thistest_imrms
print '   Previously found to be  = ',oldtest_imrms
diff_imrms = abs((oldtest_imrms-thistest_imrms)/oldtest_imrms)
print '   Difference (fractional) = ',diff_imrms

print ''
print ' Final Clean image Dynamic Range = ',thistest_immax/thistest_imrms
print ''
print '--- Done with I Imaging and Selfcal---'

#
#=====================================================================
# Polarization Imaging
#=====================================================================
#
print '--Clean (Polarization)--'
default('clean')

print "Now clean polarized data"

vis = srcsplitms

imagename = polimname

field = '*'
spw = ''
mode = 'mfs'
gain = 0.1

# Polarization
stokes = 'IQUV'

psfmode = polclnalg
imagermode = polclnmode

niter = clniter
threshold = clnthreshold

imsize = clnimsize
cell = clncell

weighting = 'briggs'
robust = 0.5

interactive = True
npercycle = 100

saveinputs('clean',imagename+'.clean.saved')
clean()

print ""
print "----------------------------------------------------"
print "Clean"
print "Final restored clean image is "+polimage
print "Final clean model is "+polmodel
print "The clean residual image is "+polresid
print "Your final clean mask is "+polmask

#
#=====================================================================
# Image Analysis
#=====================================================================
#
# Polarization statistics
print '--Final Pol Imstat--'
default('imstat')

imagename = polimage

on_statistics = {}
off_statistics = {}

# lower right corner of the image (clnimsize = [288,288])
onbox = ''
# lower right corner of the image (clnimsize = [288,288])
offbox = '216,1,287,72'

for stokes in ['I','Q','U','V']:
    box = onbox
    on_statistics[stokes] = imstat()
    box = offbox
    off_statistics[stokes] = imstat()

#
# Peel off some Q and U planes
#
print '--Immath--'
default('immath')

mode = 'evalexpr'

stokes = 'I'
outfile = ipolimage
imagename = polimage
expr = 'IM0'

immath()
print "Created I image "+outfile

stokes = 'Q'
outfile = qpolimage

immath()
print "Created Q image "+outfile

stokes = 'U'
outfile = upolimage

immath()
print "Created U image "+outfile

#
#---------------------------------------------------------------------
# Now make POLI and POLA images
#
# Use our rms above for debiasing
mysigma = 0.5*( off_statistics['Q']['rms'][0] + off_statistics['U']['rms'][0] )
#sigma = str(mysigma)+'Jy/beam'
# This does not work well yet
sigma = '0.0Jy/beam'

#stokes = ''
#imagename = [qpolimage,upolimage]
#outfile = poliimage
#mode = 'poli'
#immath()
#
#outfile = polaimage
#mode = 'pola'
#immath()

#Call as function instead:
immath(imagename=[qpolimage,upolimage],mode='poli',outfile=poliimage,sigma='0.0Jy/beam')
print "Created POLI image "+poliimage
immath(imagename=[qpolimage,upolimage],mode='pola',outfile=polaimage,sigma='0.0Jy/beam')
print "Created POLA image "+polaimage

#
#---------------------------------------------------------------------
# Save statistics of these images
default('imstat')

imagename = poliimage
stokes = ''
box = onbox
on_statistics['POLI'] = imstat()
box = offbox
off_statistics['POLI'] = imstat()

#
#
#---------------------------------------------------------------------
# Display clean I image in viewer but with polarization vectors
#
# If you did not do interactive clean, bring up viewer manually
viewer(polimage,'image')

print "Displaying pol I now.  You should overlay pola vectors"
print "Bring up the Load Data panel:"
print ""
print "Use LEL for POLA VECTOR with cut above 6*mysigma in POLI = "+str(6*mysigma)
print "For example:"
print "\'"+polaimage+"\'[\'"+poliimage+"\'>0.0048]"
print ""
print "In the Data Display Options for the vector plot:"
print "  Set the x,y increments to 2 (default is 3)"
print "  Use an extra rotation this 90deg to get B field"
print "Note the lengths are all equal. You can fiddle these."
print ""
print "You can also load the poli image as contours"

# Pause script if you are running in scriptmode
if scriptmode:
    user_check=raw_input('Return to continue script\n')

# NOTE: the LEL will be something like
# 'jupiter6cm.demo.polimg.clean.image.pola'['jupiter6cm.demo.polimg.clean.image.poli'>0.005]

#
# NOTE: The viewer can take complex images to make Vector plots, although
# the image analysis tasks (and ia tool) cannot yet handle these.  But we
# can use the imagepol tool (which is not imported by default) to make
# a complex image of the linear polarized intensity for display.
# See CASA User Reference Manual:
# http://casa.nrao.edu/docs/casaref/imagepol-Tool.html
#
# Make an imagepol tool and open the clean image 
potool = casac.homefinder.find_home_by_name('imagepolHome')
po = potool.create()
po.open(polimage)
# Use complexlinpol to make a Q+iU image
complexlinpolimage = polimname + '.cmplxlinpol'
po.complexlinpol(complexlinpolimage)
po.close()

# You can now display this in the viewer, in particular overlay this
# over the intensity raster with the poli contours.  The vector lengths
# will be proportional to the polarized intensity.  You can play with
# the Data Display Options panel for vector spacing and length.
# You will want to have this masked, like the pola image above, on
# the polarized intensity.  When you load the image, use the LEL:
# 'jupiter6cm.demo.polimg.clean.cmplxlinpol'['jupiter6cm.demo.polimg.clean.image.poli'>0.005]

#=====================================================================
#
# Print results
#
print ""
print ' Jupiter polarization results '
print ' ============================ '
print ''
for stokes in ['I','Q','U','V','POLI']:
    print ''
    print ' =============== '
    print ''
    print ' Polarization (Stokes '+stokes+'):'
    mymax = on_statistics[stokes]['max'][0]
    mymin = on_statistics[stokes]['min'][0]
    myrms = off_statistics[stokes]['rms'][0]
    absmax = max(mymax,mymin)
    mydra = absmax/myrms
    print '   Clean image  ON-SRC max = ',mymax
    print '   Clean image  ON-SRC min = ',mymin
    print '   Clean image OFF-SRC rms = ',myrms
    print '   Clean image dynamic rng = ',mydra


print '--- Done ---'

#
#=====================================================================
\end{verbatim}
\normalsize

%%%%%%%%%%%%%%%%%%%%%%%%%%%%%%%%%%%%%%%%%%%%%%%%%%%%%%%%%%%%%%%%%%%%%%%%
%%%%%%%%%%%%%%%%%%%%%%%%%%%%%%%%%%%%%%%%%%%%%%%%%%%%%%%%%%%%%%%%%%%%%%%%
\section{BIMA Mosaic Spectral Imaging}
\label{section:scripts.ngc4826}

This script analyzes a BIMA SONG mosaic of the galaxy NGC 4826
at 3mm.

The latest version of this script can be found at:
\begin{quote}
   \url{http://casa.nrao.edu/Doc/Scripts/ngc4826_tutorial.py}
\end{quote}

\small
\begin{verbatim}
##########################################################################
#                                                                        #
# Demo Script for NGC 4826 (BIMA line data)                              #
#                                                                        #
# Converted by  STM 2008-05-27 (Beta Patch 2.0) new tasking/clean/cal    #
# Updated by     CB 2008-05-30                  start from raw data      #
# Updated by    STM 2008-06-01                  scriptmode, plotting     #
# Updated by CB,STM 2008-06-02                  improvements             #
# Updated by CB,STM 2008-06-03                  bigger cube, pbcor       #
# Updated by CB,STM 2008-06-04                  pbcor stuff              #
# Updated by CB,STM 2008-06-04                  tutorial script          #
# Updated by CB     2008-06-05                  small tweaks             #
# Updated by CB,STM 2008-06-12                  final revisions (DS)     #
# Updated by STM    2008-06-14                  post-school fix          #
# Updated by DP     2009-05-05   pre-Garching   immoments chans          #
# Updated by DP     2009-05-05                  messages for flagging    #
# Updated by CB     2009-05-18   pre-Canada     spw fix not needed now   #
# Updated by CB     2009-05-18                  immoments axis=spectral  #
# Updated by CB     2009-05-19                  added restfreq to clean  #
# Updated by CB     2009-05-19                  removed clearcals        #
# Updated by STM    2009-06-26                  Patch 4, web download    #
# Updated by STM    2009-12-20                  Release 3.0 final        #
#                                                                        #
# N4826 - BIMA SONG Data                                                 #
#                                                                        #
# This data is from the BIMA Survey of Nearby Galaxies (BIMA SONG)       #
# Helfer, Thornley, Regan, et al., 2003, ApJS, 145, 259                  #
# Many thanks to Michele Thornley for providing the data and description #
#                                                                        #
# First day of observations only                                         #
#                                                                        #
# Script Notes:                                                          #
#    o This script will automatically download the data from the web     #
#      if it cannot find the data on disk.                               #
#    o The "default" commands are not necessary, but are included        #
#      in case we want to change from function calls to globals          #
#    o This script has some interactive commands, such as with plotxy    #
#      and the viewer.  This script will stop and require a              #
#      carriage-return to continue at these points.                      #
#    o Sometimes cut-and-paste of a series of lines from this script     #
#      into the casapy terminal will get garbled (usually a single       #
#      dropped character). In this case, try fewer lines, like groups    #
#      of 4-6.                                                           #
#    o If you want to re-run the script, quit casapy and start again     #
#                                                                        #
##########################################################################
import os

# needed for existence check
from os import F_OK

# Might need these if having problems with plotxy windows
#pl.ion()
#pl.clf()
#
##########################################################################
#                                                                        
# Clear out previous run results
#rmtables('ngc4826.tutorial.*')
os.system('rm -rf ngc4826.tutorial.*')

# Sets a shorthand for the ms, not necessary
prefix='ngc4826.tutorial'
msfile = prefix + '.16apr98.ms'

print 'Tutorial Script for BIMASONG NGC4826 Mosaic'
print 'Version for Release 0 (3.0.0) 20-Dec-2009'
print 'Will do: import, flagging, calibration, imaging'
print ''
#
##########################################################################
#
# 
##########################################################################
#
# N4826 - BIMA SONG Data CO(1-0) 115.2712 GHz
# 16apr98
#	source=ngc4826
#	phasecal=1310+323
#	fluxcal=3c273, Flux = 23 Jy on 16apr98
#	passcal= none - data were observed with online bandpass correction.
#
# NOTE: This data has been filled into MIRIAD, line-length correction 
#	done, and then exported as separate files for each source.
#	3c273 was not line length corrected since it was observed
#	for such a short amount of time that it did not need it.  
#
# From miriad: source Vlsr = 408; delta V is 20 km/s 
#
##########################################################################
# Locate the data
##########################################################################
#
fitsdir = 'fitsfiles'
# See if this sub-directory exists
if os.access(fitsdir,F_OK):
    # already in current directory
    print "  Found "+fitsdir+" for UVFITS files"
else:
    pathname=os.environ.get('CASAPATH').split()[0]
    datapath=pathname+'/data/demo/tutorial/BIMA/NGC4826/'
    dataname='ngc4826.fitsfiles.tgz'
    datafiles = datapath+dataname
    # Path to web archive
    webfiles = 'http://casa.nrao.edu/Data/BIMA/NGC4826/'+dataname
    if os.access(dataname,F_OK):
        print '--Found data tarball in local directory--'
        print "  Using "+dataname
    elif os.access(datafiles,F_OK):
        print '--Copy data tarball to local directory--'
        print "  Using "+datafiles
        os.system("cp -r "+datafiles+" .")
        os.system('chmod -R a+wx '+datafiles)
    else:
	# try web retrieval
        print '--Trying to retrieve data from '+webfiles
        # Use curl (for Mac compatibility)
        os.system('curl '+webfiles+' -f -o '+dataname)
        # NOTE: could also use wget
        #os.system('wget '+webfiles)
        
    print '--Unpacking tarball '
    os.system('tar xzf '+dataname)
    if os.access(fitsdir,F_OK):
        # should now be in current directory
	print "  Will use "+fitsdir+" in current directory"
    else:
        raise IOError," ERROR: "+fitsdir+" not found"

##########################################################################
# Import and concatenate sources
##########################################################################
#
# USB spectral windows written separately by miriad for 16apr98
# Assumes these are in sub-directory called "fitsfiles" of working directory
print '--Importuvfits (16apr98)--'
default('importuvfits')

print "Starting from the uvfits files exported by miriad"
print "The USB spectral windows were written separately by miriad for 16apr98"

#### We could read in each of the individual fits files as example
#### below -- this works well if you only have one or two files to
#### read, but here we have many, so instead we use some useful
#### pythonease to simplify the commands.

#importuvfits(fitsfile='fitsfiles/3c273.fits5', vis='ngc4826.tutorial.3c273.5.ms')

#### Tutorial Note: For the loop to work, the high end of range must be
#### 1+ number of actual files.

for i in range(5,9):
	importuvfits(fitsfile="fitsfiles/3c273.fits"+str(i),
		vis="ngc4826.tutorial.3c273."+str(i)+".ms")

for i in range(9,17):
	importuvfits(fitsfile="fitsfiles/1310+323.ll.fits"+str(i),
		vis="ngc4826.tutorial.1310+323.ll."+str(i)+".ms")

for i in range(5,9):
	importuvfits(fitsfile="fitsfiles/ngc4826.ll.fits"+str(i),
		vis="ngc4826.tutorial.ngc4826.ll."+str(i)+".ms")
#
##########################################################################
#
print '--Concat--'
default('concat')

concat(vis=['ngc4826.tutorial.3c273.5.ms',
	    'ngc4826.tutorial.3c273.6.ms',
	    'ngc4826.tutorial.3c273.7.ms',
	    'ngc4826.tutorial.3c273.8.ms',
	    'ngc4826.tutorial.1310+323.ll.9.ms',
	    'ngc4826.tutorial.1310+323.ll.10.ms',
	    'ngc4826.tutorial.1310+323.ll.11.ms',
	    'ngc4826.tutorial.1310+323.ll.12.ms',
	    'ngc4826.tutorial.1310+323.ll.13.ms',
	    'ngc4826.tutorial.1310+323.ll.14.ms',
	    'ngc4826.tutorial.1310+323.ll.15.ms',
	    'ngc4826.tutorial.1310+323.ll.16.ms',
	    'ngc4826.tutorial.ngc4826.ll.5.ms',
	    'ngc4826.tutorial.ngc4826.ll.6.ms',
	    'ngc4826.tutorial.ngc4826.ll.7.ms',
	    'ngc4826.tutorial.ngc4826.ll.8.ms'],
       concatvis='ngc4826.tutorial.ms',
       freqtol="",dirtol="1arcsec",async=False)

#
##########################################################################
#
# TUTORIAL NOTES:
#
# You can invoke tasks in two ways:
#
# (1) As function calls with arguments as shown above for concat and used
#     extensively in this script, e.g.
#
#        task( par1=val1, par2=val2, ... )
#
#     with parameters set as arguments in the call.  Note that in this
#     case, the global parameter values are NOT used or changed, and any
#     task parameters that are not specified as arguments to the call
#     will be defaulted to the task-specific default values (see the
#     "help task" description).
#
# (2) By setting the values of the global parameters and then using the
#     "go" command (if taskname is set) or calling the task with no
#     arguments.  For example:
#
#        default task
#        par1 = val1
#        par2 = val2
#        ...
#        inp
#        task()
#
#     In this case, the "default" command sets the parmeters to their
#     task defaults, and sets the "taskname" paramter to the task to be
#     run.  The "inp" command displays the current values for the task
#     parameters.  Then the call with no arguments runs with the globals.
#
#     Warning: "go" does not work inside scripts. See Cookbook.
#
# Using the concat call above as an example, we would do:
#
#default('concat')
#
#vis = ['ngc4826.tutorial.3c273.5.ms',
#       'ngc4826.tutorial.3c273.6.ms',
#       'ngc4826.tutorial.3c273.7.ms',
#       'ngc4826.tutorial.3c273.8.ms',
#       'ngc4826.tutorial.1310+323.ll.9.ms',
#       'ngc4826.tutorial.1310+323.ll.10.ms',
#       'ngc4826.tutorial.1310+323.ll.11.ms',
#       'ngc4826.tutorial.1310+323.ll.12.ms',
#       'ngc4826.tutorial.1310+323.ll.13.ms',
#       'ngc4826.tutorial.1310+323.ll.14.ms',
#       'ngc4826.tutorial.1310+323.ll.15.ms',
#       'ngc4826.tutorial.1310+323.ll.16.ms',	    
#       'ngc4826.tutorial.ngc4826.ll.5.ms',
#       'ngc4826.tutorial.ngc4826.ll.6.ms',
#       'ngc4826.tutorial.ngc4826.ll.7.ms',
#       'ngc4826.tutorial.ngc4826.ll.8.ms']
#
#concatvis='ngc4826.tutorial.ms'
#freqtol = ""
#dirtol = "1arcsec"
#async=False
#
#concat()

#
##########################################################################
#
# Fix up the MS
# This ensures that the rest freq will be found for all spws. 
# NOTE: STILL NECESSARY
#
# print '--Fixing up spw rest frequencies in MS--'
vis='ngc4826.tutorial.ms'
tb.open(vis+'/SOURCE',nomodify=false)
spwid=tb.getcol('SPECTRAL_WINDOW_ID')
#spwid.setfield(-1,int)
# Had to do this for 64bit systems 08-Jul-2008
spwid.setfield(-1,'int32')
tb.putcol('SPECTRAL_WINDOW_ID',spwid)
tb.close()

##########################################################################
# 16 APR Calibration
##########################################################################
#
# List contents of MS
#
print '--Listobs--'
listobs(vis='ngc4826.tutorial.ms')

# Should see the listing included at the end of this script
#

print "There are 3 fields observed in a total of 16 spectral windows"
print "   field=0    3c273    spwids 0,1,2,3               64 chans "
print "   field=1    1310+323 spwids 4,5,6,7,8,9,10,11     32 chans "
print "   field=2~8  NGC4826  spwids 12,13,14,15           64 chans "
print ""
print "See listobs summary in logger"
#
##########################################################################
# Plotting and Flagging
##########################################################################
#
# The plotxy task is the interactive x-y display and flagging GUI
#
print '--Plotxy--'
default(plotxy)

# Here we will suggest things to plot, and actually only do a few
# (near the end of this task block).  If you like you can
# uncomment these when you run this script
#
# First look at amplitude as a funciton of uv-distance using an
# average over all channels for each source.
#
# Interactive plotxy
#plotxy(vis='ngc4826.tutorial.ms',xaxis='uvdist',yaxis='amp',field='0',spw='0~3',
#       averagemode='vector',width='1000',selectplot=True,title='Field 0 SPW 0~3')
#	
# Pause script if you are running in scriptmode
#user_check=raw_input('Return to continue script\n')

# NOTE: width here needs to be larger than combination of all channels
# selected with spw and/or field. Since field and spw are unique in this
# case, both don't need to be specified, however plotting is much faster
# if you "help it" by selecting both.
#
# Now average over all times across scans but not over channel and
# plot versus channel. There are four 64-channel spws set end-to-end
# by plotxy You should see bad edge channels in each spw segment We
# will flag these (non-interactively) later
#
# Interactive plotxy
#plotxy(vis='ngc4826.tutorial.ms',xaxis='channel',yaxis='amp',field='0',spw='0~3',
#       averagemode='vector',timebin='1e7',crossscans=True,selectplot=True,
#       newplot=False,title='Field 0 SPW 0~3')
#
# Pause script if you are running in scriptmode
#user_check=raw_input('Return to continue script\n')

# You can also plot versus velocity by setting xaxis='velocity'
#
# You might do this for all the other spw/field combos
#

# You can also non-interactively plotxy to a file
# (note this works only if iteration is not set)
# Also, see how we use Python variables to make this easier
#
#field = '0'
#spw = '0~3'
#figfile = 'ngc4826.tutorial.ms' + '.plotxy.'+field+'.spectrum.raw.png'
#plotxy(vis='ngc4826.tutorial.ms',xaxis='velocity',yaxis='amp',field=field,spw=spw,
#       averagemode='vector',timebin='1e7',crossscans=True,selectplot=True,
#       newplot=False,title='Field '+field+' SPW '+spw,interactive=False,figfile=figfile)

# Now lets look at the target source, the first of the NGC4826 mosaic fields
# which are 2~8 in this MS.
#
# Since we are plotting versus velocity we can clearly see the bad edge
# channels and the overlap between spw
#
# There is nothing terribly wrong with this data and again we will flag the
# edge channels non-interactively later for consistency.
#
# Normally, if there were obviously bad data, you would flag it here
# before calibration.  To do this, hit the Mark Region button, then draw a box around
# some of the moderately high outliers, and then Flag.
#
# But this data is relatively clean, and flagging will not improve results.
#
# Interactive plotxy
plotxy(vis='ngc4826.tutorial.ms',xaxis='velocity',yaxis='amp',field='2',spw='12~15',
       averagemode='vector',timebin='1e7',crossscans=True,
       selectplot=True,newplot=False,title='Field 2 SPW 12~15')

print "You could Mark Region around outliers and Flag"
# Pause script if you are running in scriptmode
user_check=raw_input('Return to continue script\n')

#
# You could set up a Python loop to do all the N4826 fields, like this:
#
#for fld in range(2,9):
#	field = str(fld)
#	plotxy(vis='ngc4826.tutorial.ms',xaxis='velocity',yaxis='amp',field=field,spw=spw,
#              averagemode='vector',timebin='1e7',crossscans=True,selectplot=True,
#              newplot=False,title='Field '+field+' SPW '+spw)
#
#	print "Nominally, Mark Region around outliers and Flag"
#	# Pause script if you are running in scriptmode
#	user_check=raw_input('Return to continue script\n')

# Back to first field.
# You can also have it iterate over baselines, using Next to advance
# Ignore baselines 1:1, 2:2 etc. as they would correspond to autocorrelations
# if they were present (they are not in this dataset)
#
# Interactive plotxy
#plotxy(vis='ngc4826.tutorial.ms',xaxis='channel',yaxis='amp',field='0',spw='0~3',
#       averagemode='vector',timebin='1e7',crossscans=True,iteration='baseline',
#       selectplot=True,newplot=False,title='Field 0 SPW 0~3')
#	
# Pause script if you are running in scriptmode
#user_check=raw_input('Return to continue script\n')

#
# Finally, look for bad data. Here we look at field 8 w/o averaging
plotxy(vis='ngc4826.tutorial.ms',xaxis='time',yaxis='amp',field='8',spw='12~15',
       selectplot=True,newplot=False,title='Field 8 SPW 12~15')

print "You can see some bad data here"
print "Mark Region and Locate, look in logger"
print "This is a correlator glitch in baseline 3-9 at 06:19:30"
print "PLEASE DON\'T FLAG ANYTHING HERE. THE SCRIPT WILL DO IT!"
print "In a normal session you could Mark Region and Flag."
print "Here we will use flagdata instead."
# Pause script if you are running in scriptmode
user_check=raw_input('Return to continue script\n')

# If you change xaxis='channel' you see its all channels
#
##########################################################################
#
# Flag end channels
#
print '--Flagdata--'
default('flagdata')

print ""
print "Flagging edge channels in all spw"
print "  0~3:0~1;62~63 , 4~11:0~1;30~31, 12~15:0~1;62~63 "
print ""

flagdata(vis='ngc4826.tutorial.ms', mode='manualflag',
         spw='0~3:0;1;62;63,4~11:0;1;30;31,12~15:0;1;62;63')

#
# Flag correlator glitch
#
print ""
print "Flagging bad correlator field 8 antenna 3&9 spw 15 all channels"
print "  timerange 1998/04/16/06:19:00.0~1998/04/16/06:20:00.0"
print ""

flagdata(vis='ngc4826.tutorial.ms', mode='manualflag', field='8', spw='15', antenna='3&9', 
         timerange='1998/04/16/06:19:00.0~1998/04/16/06:20:00.0')


#
##########################################################################
#
# Some example clean-up editing
# Slightly high almost-edge channel in field='1', spw='4' (channel 2)
# can be flagged interactively with plotxy.
#
#plotxy(vis='ngc4826.tutorial.ms',
#       xaxis='channel',yaxis='amp',field='1',spw='4',
#       averagemode='vector',timebin='1e7',crossscans=True,
#       selectplot=True,newplot=False,title='Field 1 SPW 4')

print "Completed pre-calibration flagging"

#
##########################################################################
#
# Use Flagmanager to save a copy of the flags so far
#
print '--Flagmanager--'
default('flagmanager')

print "Now will use flagmanager to save a copy of the flags we just made"
print "These are named myflags"

flagmanager(vis='ngc4826.tutorial.ms',mode='save',versionname='myflags',
            comment='My flags',merge='replace')

# Can also use Flagmanager to list all saved versions
#
#flagmanager(vis='ngc4826.tutorial.ms',mode='list')

#
##########################################################################
#
# CALIBRATION
#
##########################################################################
#
# Bandpasses are very flat because of observing mode used (online bandpass
# correction) so bandpass calibration is unecessary for these data.
#
##########################################################################
#
# Derive gain calibration solutions.
# We will use VLA-like G (per-scan) calibration:
#
##########################################################################
#
# Set the flux density of 3C273 to 23 Jy
#
print '--Setjy (3C273)--'
default('setjy')

setjy(vis='ngc4826.tutorial.ms',field='0',fluxdensity=[23.0,0.,0.,0.],spw='0~3')
#
# Not really necessary to set spw but you get lots of warning messages if
# you don't
#
##########################################################################
#
# Gain calibration
#
print '--Gaincal--'
default('gaincal')

# This should be combining all spw for the two calibrators for single
# scan-based solutions

print 'Gain calibration for fields 0,1 and spw 0~11'
print 'Using solint=inf combining over spw'
print 'Output table ngc4826.tutorial.16apr98.gcal'

gaincal(vis='ngc4826.tutorial.ms', caltable='ngc4826.tutorial.16apr98.gcal',
	field='0,1', spw='0~11', gaintype='G', minsnr=2.0,
	refant='ANT5', gaincurve=False, opacity=0.0,
	solint='inf', combine='spw')

#
##########################################################################
#
# Transfer the flux density scale:
#
print '--Fluxscale--'
default('fluxscale')

print ''
print 'Transferring flux of 3C273 to sources: 1310+323'
print 'Output table ngc4826.tutorial.16apr98.fcal'

fluxscale(vis='ngc4826.tutorial.ms', caltable='ngc4826.tutorial.16apr98.gcal',
	  fluxtable='ngc4826.tutorial.16apr98.fcal',
	  reference='3C273', transfer=['1310+323'])

# Flux density for 1310+323 is: 1.48 +/- 0.016 (SNR = 90.6, nAnt= 8)
#
##########################################################################
#
# Plot calibration
print '--Plotcal (fluxscale)--'
default(plotcal)

# Interactive plotcal
plotcal(caltable='ngc4826.tutorial.16apr98.fcal', yaxis='amp', field='')
print ''
print 'Plotting final scaled gain calibration table'
print 'First amp vs. time for all fields '
	
# Pause script if you are running in scriptmode
user_check=raw_input('Return to continue script\n')

plotcal(caltable='ngc4826.tutorial.16apr98.fcal', yaxis='phase', field='')
print ''
print 'and phase vs. time '

# Pause script if you are running in scriptmode
user_check=raw_input('Return to continue script\n')

# And you can plot the SNR of the solution
#plotcal(caltable='ngc4826.tutorial.16apr98.fcal', yaxis='snr', field='')

# You can also plotcal to file
#figfile = 'ngc4826.tutorial.16apr98.fcal.plotcal.amp.png'
#plotcal(caltable='ngc4826.tutorial.16apr98.fcal',yaxis='amp',field='',showgui=False,figfile=figfile)
#figfile = 'ngc4826.tutorial.16apr98.fcal.plotcal.phase.png'
#plotcal(caltable='ngc4826.tutorial.16apr98.fcal',yaxis='phase',field='',showgui=False,figfile=figfile)

#
##########################################################################
#
# Correct the calibrater/target source data:
# Use new parm spwmap to apply gain solutions derived from spwid1
# to all other spwids... 
print '--Applycal--'
default('applycal')

print 'Applying calibration table ngc4826.tutorial.16apr98.fcal to data'

applycal(vis='ngc4826.tutorial.ms',
	 field='', spw='',
	 gaincurve=False, opacity=0.0, 
         gaintable='ngc4826.tutorial.16apr98.fcal',
	 spwmap=[0,0,0,0,0,0,0,0,0,0,0,0,0,0,0,0])

#
##########################################################################
#
# Check calibrated data
print '--Plotxy--'
default(plotxy)

#
# Here we plot the first of the NGC4826 fields unaveraged versus velocity
# Notice how the spw fit together

# Interactive plotxy
#print "Will plot all the NGC4826 calibrated data unaveraged - this will take a while"
#plotxy(vis='ngc4826.tutorial.ms',xaxis='velocity',yaxis='amp',field='2~8',spw='12~15',
#       averagemode='vector',datacolumn='corrected',selectplot=True,
#       newplot=False,title='Field 2~8 SPW 12~15')

#print ""
#print "Look for outliers, flag them if there are any bad ones"
#print ""
	
# Pause script if you are running in scriptmode
#user_check=raw_input('Return to continue script\n')

# You can also plot all the N4826 fields 2 through 8, for example using a loop:

#for fld in range(2,9):
#	field = str(fld)
#	plotxy(vis='ngc4826.tutorial.ms',xaxis='velocity',yaxis='amp',field=field,spw='11~15',
#              averagemode='vector',datacolumn='corrected',selectplot=True,
#              newplot=False,title='Field '+field+' SPW 11~15')
#	
#	user_check=raw_input('Return to continue script\n')

# Now here we time-average the data, plotting versus velocity

#plotxy(vis='ngc4826.tutorial.ms',xaxis='velocity',yaxis='amp',field=field,spw=spw,
#       averagemode='vector',datacolumn='corrected',timebin='1e7',crossscans=True,
#       plotcolor='blue',selectplot=True,newplot=False,title='Field '+field+' SPW '+spw)
#print ""
#print 'Final Spectrum field '+field+' spw '+spw+' TimeAverage Corrected Data'
	
# Pause script if you are running in scriptmode
#user_check=raw_input('Return to continue script\n')

# Here we overplot 3C273 the Time+Chan averaged calibrated and uncalibrated data

#
# First the corrected column in blue
#field = '0'
#spw = '0~3'
#plotxy(vis='ngc4826.tutorial.ms',xaxis='uvdist',yaxis='amp',field=field,spw=spw,
#       averagemode='vector',width='1000',datacolumn='corrected',timebin='1e7',
#       crossscans=True,plotcolor='blue',selectplot=True,newplot=False,
#       title='Field '+field+' SPW '+spw)
#print ""
#print 'Plotting field '+field+' spw '+spw+' TimeChanAverage Corrected Data in blue'
#
# Now the original data column in red
#plotxy(vis='ngc4826.tutorial.ms',xaxis='uvdist',yaxis='amp',field=field,spw=spw,
#       averagemode='vector',width='1000',datacolumn='data',timebin='1e7',
#       crossscans=True,plotcolor='red',overplot=True,selectplot=True,
#       newplot=False,title='Field '+field+' SPW '+spw)
#
#print 'OverPlotting field '+field+' spw '+spw+' TimeChanAverage Original Data in red'
	
## Pause script if you are running in scriptmode
#user_check=raw_input('Return to continue script\n')

# Can repeat for
#field = '1'
#spw = '4~11'

print "Done calibration and plotting"
#
##########################################################################
#
# SPLIT THE DATA INTO SINGLE-SOURCE MS
# AND THEN IMAGE THE CALIBRATOR
#
##########################################################################
#
#
# Split out calibrated target source and calibrater data:
#
print '--Split--'
default('split')

print 'Splitting 3C273 data to ngc4826.tutorial.16apr98.3C273.split.ms'

split(vis='ngc4826.tutorial.ms',
      outputvis='ngc4826.tutorial.16apr98.3C273.split.ms',
      field='0',spw='0~3:0~63', datacolumn='corrected')

print 'Splitting 1310+323 data to ngc4826.tutorial.16apr98.1310+323.split.ms'

split(vis='ngc4826.tutorial.ms',
      outputvis='ngc4826.tutorial.16apr98.1310+323.split.ms',
      field='1', spw='4~11:0~31', datacolumn='corrected')

print 'Splitting NGC4826 data to ngc4826.tutorial.16apr98.src.split.ms'

split(vis='ngc4826.tutorial.ms',
      outputvis='ngc4826.tutorial.16apr98.src.split.ms',
      field='2~8', spw='12~15:0~63',
      datacolumn='corrected')

#
##########################################################################
#
# If you want to use plotxy before cleaning to look at the split ms
#plotxy(vis='ngc4826.tutorial.16apr98.src.split.ms',xaxis='time',yaxis='amp')
#
##########################################################################
#
# You might image the calibrater data:
#
#print '--Clean (1310+323)--'
#default('clean')
#
#
#clean(vis='ngc4826.tutorial.16apr98.1310+323.split.ms',
#      imagename='ngc4826.tutorial.16apr98.cal.clean',
#      cell=[1.,1.],imsize=[256,256],field='0',spw='0~7',
#      threshold=10.,mode='mfs',psfmode='clark',niter=100,stokes='I')

# You can look at this in the viewer
#viewer('ngc4826.tutorial.16apr98.cal.clean.image')

#
#
##########################################################################
#
# IMAGING OF NGC4826 MOSAIC
#
##########################################################################
#
#          Mosaic field spacing looks like:
#
#          F3 (field 3)         F2 (field 2)
#
#   F4 (field 4)      F0 (field 0)        F1 (field 1)
#
#          F5 (field 5)         F6 (field 6)
#
# 4x64 channels = 256 channels 
#
# Primary Beam should be about 1.6' FWHM (7m dishes, 2.7mm wavelength)
# Resolution should be about 5-8"
##########################################################################
#
# Image the target source mosaic:
#
print '--Clean (NGC4826)--'
default('clean')

clean(vis='ngc4826.tutorial.16apr98.src.split.ms',
      imagename='ngc4826.tutorial.16apr98.src.clean',
      field='0~6',spw='0~3',
      cell=[1.,1.],imsize=[256,256],stokes='I',
      mode='channel',nchan=36,start=35,width=4,
      interpolation='nearest',
      psfmode='clark',imagermode='mosaic',scaletype='SAULT',
      cyclefactor=4,ftmachine='mosaic',
      niter=10000,threshold='45mJy',
      restfreq='115.2712GHz',interactive=F,
      minpb=0.3,pbcor=False)

### NOTE: Interpolation = 'nearest', can try 'linear' also

### NOTE: Explicitly set restfreq to CO 1-0

### NOTE: Sault weighting implies a uniform noise mosaic 

### NOTE: that niter is set to large number so that stopping point is
### controlled by threshold.

### NOTE: with pbcor=False, the final image is not "flux correct",
### instead the image has constant noise despite roll off in power as
### you move out from the phase center(s). Though this format makes it
### "look nicest", for all flux density measurements, and to get an
### accurate integrated intensity image, one needs to divide the
### srcimage.image/srcimage.flux in order to correct for the mosaic
### response pattern. One could also achieve this by setting pbcor=True
### in clean.

# Try running clean adding the parameter interactive=True.
# This parameter will periodically bring up the viewer to allow
# interactive clean boxing. For poor uv-coverage, deep negative bowls
# from missing short spacings, this can be very important to get correct
# integrated flux densities.

#
##########################################################################
#
# Do interactive viewing of clean image
print '--Viewer--'
viewer('ngc4826.tutorial.16apr98.src.clean.image')

print ""
print "This is the non-pbcorrected cube of NGC4826"
print "Use tape deck to move through channels"
print "Close the viewer when done"
print ""
#
# Pause script if you are running in scriptmode
user_check=raw_input('Return to continue script\n')

#
##########################################################################
#
# Statistics on clean image cube
#
print '--ImStat (Clean cube)--'

srcstat = imstat('ngc4826.tutorial.16apr98.src.clean.image')

print "Found image max = "+str(srcstat['max'][0])

# offbox = '106,161,153,200'

offstat = imstat('ngc4826.tutorial.16apr98.src.clean.image',
                 box='106,161,153,200')

print "Found off-source image rms = "+str(offstat['sigma'][0])

# cenbox = '108,108,148,148'
# offlinechan = '0,1,2,3,4,5,30,31,32,33,34,35'

offlinestat = imstat('ngc4826.tutorial.16apr98.src.clean.image',
                     box='108,108,148,148',
                     chans='0,1,2,3,4,5,30,31,32,33,34,35')

print "Found off-line image rms = "+str(offlinestat['sigma'][0])

#
##########################################################################
#
# Manually correct for mosaic response pattern using .image/.flux images
#
print '--ImMath (PBcor)--'

immath(outfile='ngc4826.tutorial.16apr98.src.clean.pbcor',
       imagename=['ngc4826.tutorial.16apr98.src.clean.image','ngc4826.tutorial.16apr98.src.clean.flux'],
       mode='evalexpr',expr='IM0/IM1')

#
##########################################################################
#
# Statistics on PBcor image cube
#
print '--ImStat (PBcor cube)--'

pbcorstat = imstat('ngc4826.tutorial.16apr98.src.clean.pbcor')

print "Found image max = "+str(pbcorstat['max'][0])

pbcoroffstat = imstat('ngc4826.tutorial.16apr98.src.clean.pbcor',
                      box='106,161,153,200')

print "Found off-source image rms = "+str(pbcoroffstat['sigma'][0])

pbcorofflinestat = imstat('ngc4826.tutorial.16apr98.src.clean.pbcor',
                          box='108,108,148,148',
                          chans='0,1,2,3,4,5,30,31,32,33,34,35')

print "Found off-line image rms = "+str(pbcorofflinestat['sigma'][0])

#
##########################################################################
#
# Do zeroth and first moments
#
# NGC4826 LSR velocity is 408 km/s; delta is 20 km/s
#
print '--ImMoments--'
default('immoments')

momfile = 'ngc4826.tutorial.16apr98.moments'
momzeroimage = 'ngc4826.tutorial.16apr98.moments.integrated'
momoneimage = 'ngc4826.tutorial.16apr98.moments.mom1'

print "Calculating Moments 0,1 for PBcor image"

immoments(imagename='ngc4826.tutorial.16apr98.src.clean.pbcor',
	  moments=0,axis='spectral',
	  chans='7~28',
          outfile='ngc4826.tutorial.16apr98.moments.integrated') 

# TUTORIAL NOTES: For moment 0 we use the image corrected for the
# mosaic response to get correct integrated flux densities. However,
# in *real signal* regions, the value of moment 1 does not dependent on
# the flux being correct so the non-pb corrected SAULT image can be
# used, this avoids having lots of junk show up at the edges of your
# moment 1 image due to the primary beam correction. Try it both ways
# and see for yourself.

# TUTORIAL NOTES:
#
# Moments greater than zero need to have a conservative lower
# flux cutoff to produce sensible results.

immoments(imagename='ngc4826.tutorial.16apr98.src.clean.image',
	  moments=1,axis='spectral',includepix=[0.2,1000.0],
	  chans='7~28',
          outfile='ngc4826.tutorial.16apr98.moments.mom1') 

# Now view the resulting images
viewer('ngc4826.tutorial.16apr98.moments.integrated')
#
print "Now viewing Moment-0 ngc4826.tutorial.16apr98.moments.integrated"
print "Note PBCOR effects at field edge"
print "Change the colorscale to get better image"
print "You can also Open and overlay Contours of Moment-1 ngc4826.tutorial.16apr98.moments.mom1"
print "Close the viewer when done"
#
# Pause script if you are running in scriptmode
user_check=raw_input('Return to continue script\n')

#
##########################################################################
#
# Statistics on moment images
#
print '--ImStat (Moment images)--'

momzerostat=imstat('ngc4826.tutorial.16apr98.moments.integrated')

print "Found moment 0 max = "+str(momzerostat['max'][0])

print "Found moment 0 rms = "+str(momzerostat['rms'][0])

momonestat=imstat('ngc4826.tutorial.16apr98.moments.mom1')

print "Found moment 1 median = "+str(momonestat['median'][0])

#
##########################################################################
#
# An alternative is to mask the pbcor image before calculating
# moments.  The following block shows how to do this.

#print '--Viewer--'
#viewer(ngc4826.tutorial.16apr98.moments.integrated)

# TUTORIAL NOTES: After loading change "unit contour level" in "Data
# Display Options" gui to something like 70. select "region manager
# tool" from "tool" drop down menu Then assign the sqiggly Polygon
# button to a mouse button by clicking on it with a mouse button. Then
# draw a polygon region around galaxy emission, avoiding edge regions
# and double click in the region you created. Then in "region manager
# tool" select "save to file" and give file name mom0mask.rgn.

#print '--ImMoments (masked)--'
#print 'Creating masked moment 0 image ngc4826.tutorial.16apr98.moments.integratedmasked'
#	
#immoments(imagename='ngc4826.tutorial.16apr98.src.clean.pbcor',moments=0,
#          axis='spectral',chans='7~28',region='mom0mask.rgn',
#          outfile='ngc4826.tutorial.16apr98.moments.integratedmasked') 
#
#print 'Creating masked moment 1 image ngc4826.tutorial.16apr98.moments.mom1masked'
#
#immoments(imagename='ngc4826.tutorial.16apr98.src.clean.pbcor.masked',moments=1,
#          axis='spectral',includepix=[0.2,1000.0],chans='7~28',
#          region='mom0mask.rgn',outfile='ngc4826.tutorial.16apr98.moments.mom1masked') 

# Now view the resulting images
#viewer('ngc4826.tutorial.16apr98.moments.integratedmasked')
#
#print "Now viewing masked Moment-0 ngc4826.tutorial.16apr98.moments.integratedmasked"
#print "You can Open and overlay Contours of Moment-1 ngc4826.tutorial.16apr98.moments.mom1masked"
#
# Pause script if you are running in scriptmode
#user_check=raw_input('Return to continue script\n')

# Finally, can compute and print statistics
#print '--ImStat (masked moments)--'
#
#maskedmomzerostat = imstat('ngc4826.tutorial.16apr98.moments.integratedmasked')
#print "Found masked moment 0 max = "+str(maskedmomzerostat['max'][0])
#print "Found masked moment 0 rms = "+str(maskedmomzerostat['rms'][0])
#
#maskedmomonestat=imstat('ngc4826.tutorial.16apr98.moments.mom1masked')
#print "Found masked moment 1 median = "+str(maskedmomonestat['median'][0])

#
##########################################################################
#
# Now show how to print out results
#
print '--Results (16apr98)--'
print ''
#
# Currently using non-PBcor values
#
im_srcmax16 = srcstat['max'][0]
im_offrms16 = offstat['sigma'][0]
im_offlinerms16 = offlinestat['sigma'][0]
thistest_immax = momzerostat['max'][0]
thistest_imrms = momzerostat['rms'][0]

#
# Report a few key stats
#
print '  NGC4826 Image Cube Max = '+str(im_srcmax16)
print "          At ("+str(srcstat['maxpos'][0])+","+str(srcstat['maxpos'][1])+") Channel "+str(srcstat['maxpos'][3])
print '          '+srcstat['maxposf']
print ''
print '          Off-Source Rms = '+str(im_offrms16)
print '          Signal-to-Noise ratio = '+str(im_srcmax16/im_offrms16)
print ''
print '          Off-Line   Rms = '+str(im_offlinerms16)
print '          Signal-to-Noise ratio = '+str(im_srcmax16/im_offlinerms16)

print "Done with NGC4826 Tutorial"
#
##########################################################################
\end{verbatim}
\normalsize

%%%%%%%%%%%%%%%%%%%%%%%%%%%%%%%%%%%%%%%%%%%%%%%%%%%%%%%%%%%%%%%%%%%%%%%%
%%%%%%%%%%%%%%%%%%%%%%%%%%%%%%%%%%%%%%%%%%%%%%%%%%%%%%%%%%%%%%%%%%%%%%%%
