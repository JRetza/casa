%%%%%%%%%%%%%%%%%%%%%%%%%%%%%%%%%%%%%%%%%%%%%%%%%%%%%%%%%%%%%%%%%
%%%%%%%%%%%%%%%%%%%%%%%%%%%%%%%%%%%%%%%%%%%%%%%%%%%%%%%%%%%%%%%%%
%%%%%%%%%%%%%%%%%%%%%%%%%%%%%%%%%%%%%%%%%%%%%%%%%%%%%%%%%%%%%%%%%

% STM 2007-04-13  split from previous version
% STM 2007-08-24  quick update
% DK  2007-10-10  beta release
% STM 2007-10-10  spell-checked

\chapter{Visualization With The CASA Viewer}
\label{chapter:display}

This chapter describes how to display data with the {\tt casaviewer}
either as a stand-alone or through the {\tt viewer} task. You can
display both images and Measurement Sets.

%%%%%%%%%%%%%%%%%%%%%%%%%%%%%%%%%%%%%%%%%%%%%%%%%%%%%%%%%%%%%%%%%
%%%%%%%%%%%%%%%%%%%%%%%%%%%%%%%%%%%%%%%%%%%%%%%%%%%%%%%%%%%%%%%%%
\section{Starting the {\tt viewer}}
\label{section:display.start}

\begin{figure}[h!]
\pngname{viewer_n5921_1}{3.25}
\pngname{viewer_n5921_2}{3.25}
\caption{\label{fig:viewer_start} The {\bf Viewer Display Panel} (left) and 
{\bf Data Display Options} (right) panels that appear when the 
{\tt viewer} is called with the image cube from NGC5921
({\tt viewer('ngc5921.usecase.clean.image')}).  The initial display is
of the first channel of the cube.}
\hrulefill
\end{figure}

Within the casapy environment, there is a {\tt viewer} task
which can be used to call up an image.  The inputs are:
\small
\begin{verbatim}
#  viewer :: View an image or visibility data set.

infile              =         ''   #   Name of file to visualize
filetype            =    'image'   #   Type of file (ms, image, or vector)
\end{verbatim}
\normalsize

Examples of starting the {\tt viewer}:
\small
\begin{verbatim}
  CASA <4>: viewer()

  CASA <5>: viewer('ngc5921.usecase.ms','ms')

  CASA <6>: viewer('ngc5921.usecase.clean.image')
\end{verbatim}
\normalsize
The first of these starts an empty {\tt viewer}, which will bring up
an empty {\bf Viewer Display Panel} 
(\S~\ref{section:display.viewerGUI.displaypanel}) and a {\bf Load Data} panel 
(\S~\ref{section:display.viewerGUI.load}) .  The second starts the
{\tt viewer} loaded with a Measurement Set.  The last of these examples
starts the {\tt viewer} with an image cube 
(see Figure~\ref{fig:viewer_start}).

{\bf BETA ALERT:} the {\tt viewer} task cannot currently figure out whether
a given file is an image or MS, so for now you need to specify 
{\tt filetype='ms'} explicitly if you want to view an MS in raster mode.

\begin{figure}[h!]
\pngname{viewer_n5921ms_1}{3.25}
\pngname{viewer_n5921ms_2}{3.25}
\caption{\label{fig:viewer_start_ms} The {\bf Viewer Display Panel}
(left) and {\bf Data Display Options} (right) panels that appear when the 
{\tt viewer} is called with the NGC5921 Measurement Set
({\tt viewer('ngc5921.usecase.ms','ms')}).} 
\hrulefill
\end{figure}

%%%%%%%%%%%%%%%%%%%%%%%%%%%%%%%%%%%%%%%%%%%%%%%%%%%%%%%%%%%%%%%%%
\subsection{Starting the {\tt casaviewer} outside of {\tt casapy}}
\label{section:display.start.casaviewer}

The {\tt casaviewer} is the name of the stand-alone application that is
available with a CASA installation.  From outside {\tt casapy}, you
can call this command from the command line in the following ways:

Start the {\tt casaviewer} with no default image/MS loaded; it will
pop up the {\bf Load Data} frame 
(\S~\ref{section:display.viewerGUI.load}) and a blank, 
standard {\bf Viewer Display Panel} 
(\S~\ref{section:display.viewerGUI.displaypanel}).

\small
\begin{verbatim}
  > casaviewer &
\end{verbatim}
\normalsize

Start the {\tt casaviewer} with the selected image; the image will be
displayed in the {\tt Viewer Display Panel}. If the image is a cube (more
than one plane for frequency or polarization) then it will be one the
first plane of the cube.

\small
\begin{verbatim}
  > casaviewer image_filename &
\end{verbatim}
\normalsize

Start the {\tt casaviewer} with the selected Measurement Set; note the
additional parameter indicating that it is an ms; the default is
'image'.

\small
\begin{verbatim}
  > casaviewer ms_filename ms &
\end{verbatim}
\normalsize

%%%%%%%%%%%%%%%%%%%%%%%%%%%%%%%%%%%%%%%%%%%%%%%%%%%%%%%%%%%%%%%%%
%%%%%%%%%%%%%%%%%%%%%%%%%%%%%%%%%%%%%%%%%%%%%%%%%%%%%%%%%%%%%%%%%
\section{The {\tt viewer} GUI}
\label{section:display.viewerGUI}

The CASA {\tt viewer} application consists of a number of
graphical user interfaces (GUIs) that are mouse cursor and button
controlled.  There are a number of panels to this GUI. 

We describe the {\bf Viewer Display Panel} 
(\S~\ref{section:display.viewerGUI.displaypanel}) and the
{\bf Load Data - Viewer} 
(\S~\ref{section:display.viewerGUI.load})
below, as these are common to whether
you are viewing and image or MS.  The other panels are context
specific and described in the following sections on viewing
images (\S~\ref{section:display.image}) and 
Measurement Sets (\S~\ref{section:display.ms}).

%%%%%%%%%%%%%%%%%%%%%%%%%%%%%%%%%%%%%%%%%%%%%%%%%%%%%%%%%%%%%%%%%

\subsection{The Viewer Display Panel}
\label{section:display.viewerGUI.displaypanel}

% \begin{figure}[h!]
% \gname{viewer0}{4}
% \caption{\label{fig:viewer0} Viewer Display Panel with no data
%   loaded. Each section of the GUI is explained below} 
% \hrulefill
% \end{figure}
 
The Viewer Display Panel GUI is the the panel that contains the
image or MS display.  This is shown in the left panels of 
Figures~\ref{fig:viewer_start} and \ref{fig:viewer_start_ms}.
Note that this panel is the same whether an image or MS is
being displayed.
 
At the top of the Viewer Display Panel GUI are the menus:
\begin{itemize}
\item {\bf Data}
  \begin{itemize}
      \item  {\tt Open} --- open an image from disk
      \item  {\tt Register} --- register/unregister selected image (menu
             expands to the right containing all loaded images) 
      \item  {\tt Close} --- close selected image (menu expands to the right)
      \item  {\tt Adjust} --- open the Data Display Options ('Adjust') panel 
      \item  {\tt Print} --- print the displayed image
      \item  {\tt Close Panel} --- close the Viewer Display Panel (will exit if
             this is the last display panel open)
      \item  {\tt Quit Viewer} --- close all display panels and exit
  \end{itemize}
\item {\bf Display Panel}
  \begin{itemize}
      \item {\tt New Panel} --- create a new Viewer Display Panel
      \item {\tt Panel Options} --- open the Display Panel's options window
      \item {\tt Print} --- print displayed image
      \item {\tt Close Panel} --- close the Viewer Display Panel (will exit if
            this is the last display panel open)
  \end{itemize}
\item {\bf Tools}
  \begin{itemize}
      \item {\tt Annotations} --- not yet available (greyed out) 
      \item {\tt Spectral Profile} --- plot frequency/velocity profile
                 of point or region of image
  \end{itemize}
\item {\bf View}
  \begin{itemize}
      \item {\tt Main Toolbar} --- show/hide top row of icons
      \item {\tt Mouse Toolbar} --- show/hide second row of
                 mouse-button action selection icons
      \item {\tt Animator} --- show/hide tapedeck control panel
      \item {\tt Position Tracking} --- show/hide bottom position
                 tracking report box  
  \end{itemize}
\end{itemize}

Below this is the {\bf Main Toolbar}, the top row of icons for fast
access to some of these menu items:
\begin{itemize}
   \item {\bf folder} ({\tt Data:Open} shortcut) --- pulls up Load Data panel
   \item {\bf wrench} ({\tt Data:Adjust} shortcut) --- pulls up Data Display
              Options ('Adjust') panel 
   \item {\bf panels} ({\tt Data:Register} shortcut) --- pull up 
              menu of loaded data
   \item {\bf delete} ({\tt Data:Close} shortcut) --- closes/unloads 
              selected data
   \item {\bf panel} ({\tt Display Panel:New Panel})
   \item {\bf panel wrench} ({\tt Display Panel:Panel Options}) --- pulls up
              the Display Panel's options window 
   \item {\bf print} ({\tt Display Panel:Print}) --- print data
   \item {\bf magnifier box} --- Zoom out all the way
   \item {\bf magnifier plus} --- Zoom in (by a factor of 2)
   \item {\bf magnifier minus} --- Zoom out (by a factor of 2)
\end{itemize}

Below this are the eight {\bf Mouse Tool} buttons. These allow assignment of
{\it each} of the three mouse buttons to a different operation on the display
area. Clicking a mouse tool icon will [re-]assign {\bf the mouse button that
was clicked} to that tool.  The icons show which mouse button is currently
assigned to which tool.  

The 'escape' key can be used to cancel any mouse tool operation that was
begun but not completed, and to erase any tool showing in the display area.
\begin{itemize}
   \item {\bf Zooming (magnifying glass icon):}
     To zoom into a selected area, press the Zoom tool's mouse button
     (the {\bf left} button by default) on one corner of the desired
     rectangle and drag to the desired opposite corner. Once the button is
     released, the zoom rectangle can still be moved or resized by dragging.
     To complete the zoom, double-click inside the selected rectangle
     (double-clicking {\it outside} it will zoom {\it out} instead).
   \item {\bf Panning (hand icon):} Press the tool's mouse button on a 
     point you wish to move, drag it to the position where you want it
     moved, and release. {\it Note: The arrow keys, Page Up, Page Down,
     Home and End keys can also be used to scroll through your data any time
     you are zoomed in. (Click on the main display area first, to be sure
     the keyboard is 'focused' there).}
   \item {\bf Stretch-shift colormap fiddling (crossed arrows):} This is
     usually the handiest color adjustment; it is assigned to the {\bf middle}
     mouse button by default.
   \item {\bf Brightness-contrast colormap fiddling (light/dark sun):} 
   \item {\bf Positioning (bombsight):} This tool can place a 'crosshair'
     marker on the display to select a position. It is used to flag
     Measurement Set data or to select an image position for spectral profiles.
     Click on the desired position with the tool's mouse button to place
     the crosshair; once placed you can drag it to other locations.
     Double-click is not needed for this tool.  
     See \S~\ref{section:display.viewerGUI.displaypanel.region} for more
     detail.
   \item {\bf Rectangle and Polygon region drawing:} The rectangle region tool
     is assigned to the {\bf right} mouse button by default. As with the zoom
     tool, a rectangle region is generated by dragging with the assigned mouse
     button; the selection is confirmed by double-clicking within the rectangle.
     Polygon regions are created by clicking the assigned mouse button
     at the desired vertices, clicking the final location twice to finish.
     Once created, a polygon can be moved by dragging from inside, or
     reshaped by dragging the handles at the vertices.  Double-click to
     confirm region selection.
     See \S~\ref{section:display.viewerGUI.displaypanel.region} for the uses
     of this tool.
   \item {\bf Polyline drawing:}
     A polyline can be created by selecting this tool. It is manipulated
     similarly to the polygon region tool: create segments by clicking at
     the desired positions and then double-click to finish the line.
     [Uses for this tool are still to be implemented].
\end{itemize}

The main display area lies below the toolbars.

Underneath the display area is an {\bf Animator} panel.  The most prominent
feature is the ``tape deck'' which provides movement between image planes
along a selected third dimension of an image cube. This set of buttons is
only enabled when a registered image reports that it has more than one plane
along its 'Z axis'. In the most common case, the animator selects the frequency
channel. From left to right, the tape deck controls allow the user to:
\begin{itemize}
   \item {\bf rewind} to the start of the sequence (i.e., the first plane)
   \item {\bf step backwards} by one plane
   \item {\bf play backwards}, or repetitively step backwards
   \item {\bf stop} any current play
   \item {\bf play forward}, or repetitively step forward
   \item {\bf step forward} by one plane
   \item {\bf fast forward} to the end of the sequence
\end{itemize}
To the right of the tape deck is an editable text box indicating the
current frame (channel) number and a label showing the total number of
frames. Below that is a slider for controlling the (nominal) animation
speed. To the right is a 'Full/Compact' toggle. In 'Full' mode (the
default), a slider controlling frame number and a 'Blink mode' control
are also available.

'Blink' mode is useful when more than one raster image is
registered. In that mode, the tapedeck controls which image is
displayed at the moment, rather then the image plane (set that in
'Normal' mode first). The registered images must cover the same
portion of the sky and use the same coordinate projection.

{\bf Note:} {\it In 'Normal' mode, it is advisable to have only} one {\it
raster image registered at a time, to avoid confusion. Unregister (or
close) the others).}

At the bottom of the Display Panel is the {\bf Position Tracking} panel.  As
the mouse moves over the main display, this panel shows information such as
flux density, position (e.g. RA and Dec), Stokes, and frequency (or velocity),
for the point currently under the cursor.  Each registered image/MS displays
its own tracking information.  Tracking can be 'frozen' (and unfrozen again)
with the space bar.  (Click on the main display area first, to be sure the
keyboard is 'focused' there).

The Animator or Tracking panels can be hidden or detached (and later
re-attached) by using the boxes at upper right of the panels; this is
useful for increasing the size of the display area.  (Use the 'View'
menu to show a hidden panel again).  The individual tracking areas
(one for each registered image) can be hidden using the checkbox at
upper left of each area.

%%%%%%

\subsection{Region Selection and Positioning}
\label{section:display.viewerGUI.displaypanel.region}

You can draw regions or select positions on the display with the mouse,
once you have selected the appropriate tool(s) on the {\tt Mouse Toolbar}
(see above).

The {\tt Rectangle region drawing} tool currently works (only) for the
following: 
\begin{itemize}
  \item Region statistics reporting for images,
  \item Region spectral profiles for images, via the
        {\tt Tools:Spectral Profile} menu,
  \item Flagging of Measurement Sets
\end{itemize}

The {\tt Polygon region drawing} tool works for the first two of the
above only (polygon region flagging of an MS is not supported).

The {\tt Positioning} crosshair tool works for the last two of the above.

The {\tt Spectral Profile} display
(see \S~\ref{section:display.image.specprof}), when active, updates on
each change of the rectangle, polygon, or crosshair.  
Flagging with the crosshair also responds to single click or drag.

Region statistics are printed in the terminal window (not the logger)
by double-clicking the completed region.  Flagging a rectangular region
also requires a double-click in order to take effect.

Here is an example of region statistics from the viewer:
\small
\begin{verbatim}
ngc5921.usecase.clean.image-contour     (Jy/beam)

n           Std Dev     RMS         Mean        Variance    Sum
52          0.01067     0.02412     0.02168     0.0001139   1.127     

Flux        Med |Dev|   IntQtlRng   Median      Min         Max
0.09526     0.009185    0.01875     0.02076     0.003584    0.04181   
\end{verbatim}
\normalsize

%%%%%%%%%%%%%%%%%%%%%%%%%%%%%%%%%%%%%%%%%%%%%%%%%%%%%%%%%%%%%%%%%

\subsection{The Load Data Panel}
\label{section:display.viewerGUI.load}

\begin{figure}[h!]
\pngname{viewer_load}{6}
\caption{\label{fig:viewer_load} The {\bf Load Data - Viewer} panel
that appears if you open the {\tt viewer} without any {\tt infile}
specified, or if you use the {\tt Data:Open} menu or Open icon.
You can see the images and MS available in your current directory,
and the options for loading them.} 
\hrulefill
\end{figure}

You can use the {\bf Load Data - Viewer} GUI to interactively
choose images or MS to load into the viewer.  An example of
this panel is shown in Figure~\ref{fig:viewer_load}.  This
panel is accessed through the {\tt Data:Open} menu or Open icon
of the {\bf Viewer Display Panel}.  It also appears if you open 
the {\tt viewer} without any {\tt infile} specified.

Selecting a file on disk in the {\tt Load Data} panel will
provide options for how to display the data. Images can be displayed
as: 
\begin{enumerate}
\item Raster Image, 
\item Contour Map, 
\item Vector map, or 
\item Marker Map.  
\end{enumerate}
A MS can only be displayed as a raster.

%%%%%%%%%%%%%%%%%%%%%%%%%%%%%%%%%%%%%%%%%%%%%%%%%%%%%%%%%%%%%%%%%
%%%%%%%%%%%%%%%%%%%%%%%%%%%%%%%%%%%%%%%%%%%%%%%%%%%%%%%%%%%%%%%%%
\section{Viewing Images}
\label{section:display.image}

You have several options for viewing an image.  These are seen
at the right of the {\bf Load Data - Viewer} panel 
described in \S~\ref{section:display.viewerGUI.load} and shown in 
Figure~\ref{fig:viewer_load_image} when an image is selected.  They are:
\begin{itemize}
   \item {\tt Raster Image} --- a greyscale or color image,
   \item {\tt Contour Map} --- contours of intensity as a line plot,
   \item {\tt Vector Map} --- vectors (as in polarization) as a line plot,
   \item {\tt Marker Map} --- a line plot with symbols to mark positions.
\end{itemize}

The {\tt Raster Image} is the default image display, and is what you
get if you invoke the {\tt viewer} from {\tt casapy} with an image
file name.  In this case, you will need to use the {\tt Open} menu to
bring up the {\bf Load Data} panel to choose a different display.

\begin{figure}[h!]
\pngname{viewer_load_image}{6}
\caption{\label{fig:viewer_load_image} The {\bf Load Data - Viewer} panel
as it appears if you select an image.  You can see all options
are available to load the image as a {\tt Raster Image}, 
{\tt Contour Map}, {\tt Vector Map}, or {\tt Marker Map}.
In this example, clicking on the {\tt Raster Image} button would 
bring up the displays shown in Figure~\ref{fig:viewer_start}.}
\hrulefill
\end{figure}

%%%%%%%%%%%%%%%%%%%%%%%%%%%%%%%%%%%%%%%%%%%%%%%%%%%%%%%%%%%%%%%%%

\subsection{Viewing a raster map}
\label{section:display.image.raster}

A raster map of an image shows pixel intensities in a two-dimensional
cross-section of gridded data with colors selected from a finite set
of (normally) smooth and continuous colors, i.e., a colormap.

% \begin{figure}[h!]
% \gname{viewer1}{3.5}
% \gname{viewer_loaddata}{3.5}
% \caption{\label{fig:viewer1} casaviewer: Illustration of a raster
%   image in the Viewer Display Panel(left) and the Load Data panel
%   (right).} 
% \hrulefill
% \end{figure}

Starting the {\tt casaviewer} with an image as a raster map will look
something like the example in Figure~\ref{fig:viewer_start}. 
 
You will see the GUI which consists of two main windows, entitled
"Viewer Display Panel" and "Load Data". In the "Load Data" panel, you
will see all of the viewable files in the current working directory along
with their type (Image, Measurement Set, etc).  After selecting a file, you
are presented with the available data types for these data. Clicking
on the button {\tt Raster Map} will create a display as above. 

The data display can be adjusted by the user as needed.  This
is done through the {\bf Data Display Options} panel.  This window
appears when you choose the {\tt Data:Adjust} menu or use the
wrench icon from the {\bf Main Toolbar}.  This also comes up
by default along with the {\bf Viewer Display Panel}.

The {\bf Data Display Options} window is shown in the right panel
of Figure~\ref{fig:viewer_start}.  It consists of a tab for each
image or MS loaded, under which are a cascading series of expandable
categories.  For an image, these are:
\begin{itemize}
   \item {\bf Display axes}
   \item {\bf Hidden axes}
   \item {\bf Basic Settings}
   \item {\bf Position tracking}
   \item {\bf Axis labels}
   \item {\bf Axis label properties}
   \item {\bf Beam Ellipse}
   \item {\bf Color Wedge}
\end{itemize}
The {\bf Basic Settings} category is expanded by
default.  To expand a category to show its options, click on it with
the left mouse button.

We now describe the options available for each category.

%%%%%%
\subsubsection{Adjust --- Basic Settings}
\label{section:display.image.raster.adjust.basic}

This roll-up is rolled down by default.  It has
the parameters that most dramatically alter the generation of the
raster map itself from the image or array data.  An example of
this part of the panel is shown in Figure~\ref{fig:viewer_raster_basic}.

\begin{figure}[h!]
\begin{center}
\pngname{viewer_ras_basic}{5}
\caption{\label{fig:viewer_raster_basic} The {\tt Basic Settings}
category of the {\bf Data Display Options} panel
as it appears if you load the image as a {\tt Raster Image}.
This is a zoom-in for the data displayed in Figure~\ref{fig:viewer_start}.}
\hrulefill
\end{center}
\end{figure}

The options available are:
\begin{itemize}

\item {\tt Basic Settings:Aspect ratio}

This option controls the horizontal-vertical size ratio of data pixels
on screen.  {\tt Fixed world} (the default) means that the aspect
ratio of the pixels is set according to the coordinate system of
the image (i.e., true to the projected sky). {\tt Fixed lattice}
means that data pixels will always be square on the screen.  Selecting
{\tt flexible} allows the raster map to stretch independently in each
direction to fill as much of the display area as possible.

\item {\tt Basic Settings:Pixel treatment}

This option controls the precise alignment of the edge of the current
'zoom window' with the data lattice.  {\tt edge} (the default) means
that whole data pixels are always drawn, even on the edges of the display.
For most purposes, {\tt edge} is recommended.  {\tt center} means that
data pixels on the edge of the display are drawn only from their centers
inwards. (Note that a data pixel's center is considered its 'definitive'
position, and corresponds to a whole number in 'data pixel' or 'lattice'
coordinates).

\item {\tt Basic Settings: Resampling mode}

This setting controls how the data are re-sampled to the resolution of
the screen.  {\tt nearest} (the default) means that screen pixels are
colored according to the intensity of the nearest data point, so that
each data pixel is shown in a single color. {\tt bilinear} applies a
bilinear interpolation between data pixels to produce smoother looking images
when data pixels are large on the screen.  {\tt bicubic} applies an
even higher-order (and somewhat slower) interpolation.

\item {\tt Basic Settings: Data Range}

You can use the entry box provided to set the minimum and maximum data values
mapped to the available range of colors. For very high dynamic range images,
you will probably want to enter a value much less than the maximum, to see
detail in lower brightness-level pixels. The next setting also helps quite
a lot in this regard.

\item {\tt Basic settings: Scaling power cycles}

This option allows logarithmic scaling of data values to colormap cells.  

The color of a pixel is determined as follows: the data value is clipped to lie
within the data range specified above, then mapped to an index to the
available colors as described below. The color corresponding to this index is
ultimately determined by the current colormap and its 'fiddling' (shift/slope)
and brightness/contrast settings (see {\bf Mouse Toolbar}, above).  (Adding a
{\bf Color Wedge} to your image can help clarify the effect of the various
color controls).

The {\tt Scaling power cycles} option controls the mapping of clipped data
values to colormap indices.  Set to zero (the default), a straight linear
relation is used.  For negative scaling values, a logarithmic mapping
assigns an larger fraction of the available colors to lower data values (this
is usually what you want). For positive values, an larger fraction of the
colormap is used for the high data values.
\footnote{The actual functions are computed as follows:

For negative scaling values (say $-p$), the data is scaled linearly
from the range ({\tt dataMin} -- {\tt dataMax}) to the range (1 -- $10^{p}$).
Then the program takes the $\log$ (base 10) of that value (arriving at
a number from 0 to $p$) and scales that linearly to the number of
available colors.  Thus the data is treated as if it had $p$ decades
of range, with an equal number of colors assigned to each decade.

For positive scaling values, the inverse (exponential) functions are used.
If $p$ is the (positive) value chosen,  The data value is scaled linearly to
lie between 0 and $p$, and 10 is raised to this power, yielding a value in the
range (1 -- $10^{p}$).  Finally, that value is scaled linearly to the number
of available colors.}

See Figure~\ref{fig:scalingpower} for sample curves.
\begin{figure}[h]
\begin{center}
\pngname{viewer_scalingpower}{3.6}
\caption{\label{fig:scalingpower} Example curves for scaling power cycles.}
\hrulefill
\end{center}
\end{figure}

\item {\tt Basic settings: Colormap}

You can select from a variety of colormaps here.  {\tt Hot Metal},
{\tt Rainbow} and {\tt Greyscale} colormaps are the ones most commonly used.

\end{itemize}


%%%%%%%%%%%%%%%%%%%%%%%%%%%%%%%%%%%%%%%%%%%%%%%%%%%%%%%%%%%%%%%%%
\subsection{Viewing a contour map}
\label{section:display.image.contour}

Viewing a contour image is similar the process above. A contour map
shows lines of equal pixel intensity (e.g., flux density) in a two
dimensional cross-section of gridded data (Figure~\ref{fig:viewer_con}).
Contour maps are particularly useful for overlaying on raster images so
that two different measurements of the same part of the sky can be shown
simultaneously. 

% \begin{figure}[h!]
% \gname{viewer5}{3.5}
% \gname{viewer_displaydata5}{3.5}
% \caption{\label{fig:viewer5} Example of a contour
%   image in the {\bf Viewer Display Panel} (left) and the 
%   {\bf Load Data} panel (right).} 
% \hrulefill
% \end{figure}
 
\begin{figure}[h!]
\pngname{viewer_n5921_con_1}{3.25}
\pngname{viewer_n5921_con_2}{3.25}
\caption{\label{fig:viewer_con} The {\bf Viewer Display Panel}
(left) and {\bf Data Display Options} panel (right) after choosing
{\tt Contour Map} from the {\bf Load Data} panel.  The
image shown is for channel 11 of the NGC5921 cube, selected using
the {\bf Animator} tape deck, and zoomed in using the tool bar icon.
Note the different options in the open {\tt Basic Settings} category
of the {\bf Data Display Options} panel.} 
\hrulefill
\end{figure}

%%%%%%%%%%%%%%%%%%%%%%%%%%%%%%%%%%%%%%%%%%%%%%%%%%%%%%%%%%%%%%%%%
\subsection{Overlay contours on a raster map}
\label{section:display.image.viewcontours}

Contours of either a second data set or the same data set can be used
for comparison or to enhance visualization of the data. The Data Options
Panel will have multiple tabs which allow adjusting each overlay
individually (Note tabs along the top.  {\bf Beware:} it's easy to
forget which tab is active!) 

To add a Contour overlay, open the {\bf Load Data} panel (Use the {\bf Data}
menu or click on the Folder icon), select the data set and select {\tt Contour}.
See Figure~\ref{fig:viewer_rascon} for an example using NGC5921.

% \begin{figure}[h!]
% \gname{viewer_datadisplay1}{3.5}
% \gname{viewer3}{3.5}
% \caption{\label{fig:viewer_overlay}  Display of a contour
% overlay on top of a raster image.} 
% \hrulefill
% \end{figure}
 
\begin{figure}[h!]
\pngname{viewer_n5921_rascon_1}{3.25}
\pngname{viewer_n5921_rascon_2}{3.25}
\caption{\label{fig:viewer_rascon} The {\bf Viewer Display Panel}
(left) and {\bf Data Display Options} panel (right) after overlaying
a {\tt Contour Map} on a {\tt Raster Image} from the same image cube.  The
image shown is for channel 11 of the NGC5921 cube, selected using
the {\bf Animator} tape deck, and zoomed in using the tool bar icon.
The tab for the contour plot is open in the {\bf Data Display Options} 
panel.} 
\hrulefill
\end{figure}

%%%%%%%%%%%%%%%%%%%%%%%%%%%%%%%%%%%%%%%%%%%%%%%%%%%%%%%%%%%%%%%%%
\subsection{Spectral Profile Plotting}
\label{section:display.image.specprof}

>From the {\bf Tools} menu, the {\tt Spectral Profile} plotting
tool can be selected.  This will pop up a new {\bf Image Profile}
window containing an x-y plot of the intensity versus spectral
axis (usually velocity).  You can then select a region with the
{\bf Rectangle} or {\bf Polygon Region} drawing tools, or pinpoint
a position using the {\bf Crosshair} tool.  The profile
for the region or position selected will then appear in the
{\bf Image Profile} window.  This profile will update in real time
to track changes to the region or crosshair, which can be moved
by click-dragging the mouse.  See Figure~\ref{fig:viewer_specprof}.

\begin{figure}[h!]
\pngname{viewer_specprof}{6.5}
\caption{\label{fig:viewer_specprof} The {\bf Image Profile} panel
that appears if you use the {\tt Tools:Spectral Profile} menu,
and then use the rectangle or polygon tool to select a region in the image.
You can also use the crosshair to get the profile at a single
position in the image.  The profile will change to track movements
of the region or crosshair if moved by dragging with the mouse.} 
\hrulefill
\end{figure}

%%%%%%%%%%%%%%%%%%%%%%%%%%%%%%%%%%%%%%%%%%%%%%%%%%%%%%%%%%%%%%%%%

% \begin{figure}[h!]
% \gname{viewer_datadisplay2}{3.5}
% \gname{viewer_datadisplay3}{3.5}
% \caption{\label{fig:datadisplay} casaviewer: Data display options. In
%   the left panel, the Display axes, Hidden axes, and Basic Settings
%   options are shown; in the right panel, the Position tracking and
%   Axis labels options are shown. }
% \end{figure}

% \begin{figure}[h!]
% \gname{viewer_datadisplay4}{3.5}
% \caption{\label{fig:datadisplay-p3} casaviewer: Data display options. In
%   this final, third panel , the Axis label properties are shown. }
% \hrulefill
% \end{figure}

% This older web page gives details of individual display options.
% Although it has not yet been integrated into the reference manual for
% the newer CASA, it is accurate in most cases: 
% 
% \url{http://aips2.nrao.edu/daily/docs/user/Display/node267.html}

%%%%%%%%%%%%%%%%%%%%%%%%%%%%%%%%%%%%%%%%%%%%%%%%%%%%%%%%%%%%%%%%%
\subsection{Adjusting Canvas Parameters/Multi-panel displays}
\label{section:display.viewerGUI.canvas}

The display area can also be manipulated with the following controls in
the {\bf Panel Display Options} (or 'Viewer Canvas Manager') window.
Use the wrench icon with a 'P' (or the 'Display Panel' menu) to show this
window.
\begin{itemize}
   \item Margins - specify the spacing for the left, right, top, and bottom margins
   \item Number of panels - specify the number of panels in x and y
         and the spacing between those panels.
   \item Background Color - white or black (more choices to come)
\end{itemize}

Figure~\ref{fig:viewer_canvas} illustrates a multi-panel display along
with the Viewer Canvas Manager settings which created it. 

\begin{figure}[h!]
\gname{viewer_canvas}{3.5}
\gname{viewer4}{3.5}
\caption{\label{fig:viewer_canvas} A multi-panel display
set up through the {\bf Viewer Canvas Manager}.} 
\hrulefill
\end{figure}
 

%%%%%%%%%%%%%%%%%%%%%%%%%%%%%%%%%%%%%%%%%%%%%%%%%%%%%%%%%%%%%%%%%
\section{Viewing Measurement Sets}
\label{section:display.ms}

\begin{figure}[h!]
\pngname{viewer_load_ms}{6}
\caption{\label{fig:viewer_load_ms} The {\bf Load Data - Viewer} panel
as it appears if you select an MS.  The only option available is
to load this as a {\tt Raster Image}.  In this example, clicking
on the {\tt Raster Image} button would bring up the displays shown
in Figure~\ref{fig:viewer_start_ms}.}
\hrulefill
\end{figure}

Visibility data can also be displayed and flagged directly from the
viewer. For Measurement Set files the only option for display is 'Raster'
(similar to AIPS task TVFLG).  An example of MS display is
shown in Figure~\ref{fig:viewer_start_ms}; loading of an
MS is shown in Figure~\ref{fig:viewer_load_ms}.  {\it Only one MS should
be registered at a time on a Display Panel (only one can be shown in
any case).}  You do not have to close other images/MSs, but you should at
least 'unregister' them from the Display Panel used for viewing the MS.
If you wish to see other images/MSs at the same time, create multiple
Display Panel windows.

% \begin{figure}[h]
% \gname{viewer_ms1}{3.5}
% \gname{viewer_ms2}{3.5}
% \caption{\label{fig:viewer_ms1} Display of visibility
%   data. The default axes are time vs. baseline.} 
% \hrulefill
% \end{figure}
 

%%%%%%%%%%%%%%%%%%%%%%%%%%%%%%%%%%%%%%%%%%%%%%%%%%%%%%%%%%%%%%%%%

\subsection{The Data Display Options Panel for MSs}
\label{section:display.ms.adjust}

The {\bf Data Display Options} panel provides adjustments for MSs
similar to those for images, and also includes flagging options.
As with images, this window appears when you choose the {\tt Data:Adjust}
menu or use the wrench icon from the {\bf Main Toolbar}. It is also shown
by default when an MS is loaded.

%%%%%%
\subsubsection{MS --- Data Display Options}
\label{section:display.ms.adjust.ddo}

The {\bf Data Display Options} window is shown in the right panel
of Figure~\ref{fig:viewer_start_ms}.  It consists of a tab for each
MS loaded, under which are a cascading series of expandable
categories.  For a Measurement Set, the categories are:
\begin{itemize}
   \item {\bf Advanced}
   \item {\bf MS and Visibility Selection}
   \item {\bf Display Axes}
   \item {\bf Flagging Options}
   \item {\bf Basic Settings}
   \item {\bf Axis Drawing and Labels}
   \item {\bf Color Wedge}
\end{itemize}
The {\bf Basic Settings} category is expanded by
default.  To expand a category to show its options, click on it with
the left mouse button.

{\bf Basic Settings:} This rollup contains entries similar to
those for a raster image (\S~\ref{section:display.image.raster.adjust.basic}). 
Together with the brightness/contrast and colormap adjustment icons
on the {\bf Mouse Toolbar} of the {\bf Viewer Display Panel},
they are especially important for adjusting
the color display of your MS.

The options available are:
\begin{itemize}

\item {\tt Basic settings: Data minimum/maximum}

This has exactly the same usage as for image raster.  
Lowering the data maximum will help brighten
weaker data values.

\item {\tt Basic settings: Scaling power cycles}

This has exactly the same usage as for image raster.  Again, lowering
this value often helps make weaker data visible.  If you want to view
several fields with very different amplitudes simultaneously, this is
probably the first adjustment you should make.

\item {\tt Basic settings: Colormap}

Greyscale or Hot Metal colormaps are generally good choices for MS
data.

\end{itemize}

%%%%%%%%%%%%%%%%%%%%%%%%%%%%%%%%%%%%%%%%%%%%%%%%%%%%%%%%%%%%%%%%%
%%%%%%%%%%%%%%%%%%%%%%%%%%%%%%%%%%%%%%%%%%%%%%%%%%%%%%%%%%%%%%%%%
