%%%%%%%%%%%%%%%%%%%%%%%%%%%%%%%%%%%%%%%%%%%%%%%%%%%%%%%%%%%%%%%%%
%%%%%%%%%%%%%%%%%%%%%%%%%%%%%%%%%%%%%%%%%%%%%%%%%%%%%%%%%%%%%%%%%
%%%%%%%%%%%%%%%%%%%%%%%%%%%%%%%%%%%%%%%%%%%%%%%%%%%%%%%%%%%%%%%%%

% STM 2008-11-07  Patch3 draft from Wes
% STM 2009-11-23  Release 3.0.0 version

%\chapter{Writing Tasks in CASA}
\chapter[Appendix: Writing Tasks in CASA]{Writing Tasks}
\label{chapter:write}

{\bf ALERT:} This prescription for writing and incorporating
tasks in CASA is for the power-user.  This procedure is also likely
to change in future releases.

It is possible to write your own task and have it appear in
casapy. For example, if you want to create a task named ``yourtask'',
then must create two files, {\tt yourtask.xml} and a
{\tt task\_yourtask.py}. The {\tt .xml} file is use to describe the interface to
the task and the {\tt task\_yourtask.py} does the actual work.  The argument
names must be the same in both the {\tt yourtask.xml} and {\tt task\_yourtask.py}
file. The {\tt yourtask.xml} file is used to generate all the interface
files so {\tt yourtask} will appear in the casapy system.  It is
easiest to start from one of the existing tasks when constructing
these.  You would make the name of the function in the 
{\tt  yourtask.py} be ``yourtask'' in this example.

We have provided the {\tt buildmytasks} command (outside casa) in order
to assemble your Python and XML into a loadable Python file.  Thus,
the steps you need to execute (again for an example task named ``yourtask''):
\begin{itemize}
\item Create python code for task as {\tt yourtask\_task.py}
\item Create xml for task as {\tt yourtask.xml}
\item Execute outside casapy: {\tt buildmytasks yourtask}
\item Execute inside casapy: {\tt execfile 'mytasks.py'}
\end{itemize}
After this, you should see the help and inputs inside casapy, e.g.\ 
{\tt inp yourtask} should work.  Note that for the final step you
invoke the file called {\tt mytasks.py}, regardless of what you named
the actual task.

Note that in order to run the {\tt buildmytasks} command in your OS,
you need the path to that executable.  This is set up if you've sourced
one of the {\tt casainit.sh} files, otherwise you will have to find
this on your system, 
\begin{verbatim}
    /usr/lib64/casapy/24.0.8115-001/bin/buildmytasks yourtask
\end{verbatim}
for the 2.4.0 beta release on Linux at the AOC.  Note also if the files you have
provided are not in the directory you are working from in CASA, you
would provide that path
\begin{verbatim}
    execfile '<MYPATH>/mytasks.py'
\end{verbatim}
where {\tt <MYPATH>} is that path.

{\bf ALERT:} This has changed since Beta Patch 2.4 in that before you
would do {\tt execfile 'yourtask.py'} instead of {\tt execfile 'mytasks.py'}.

%%%%%%%%%%%%%%%%%%%%%%%%%%%%%%%%%%%%%%%%%%%%%%%%%%%%%%%%%%%%%%%%%
%%%%%%%%%%%%%%%%%%%%%%%%%%%%%%%%%%%%%%%%%%%%%%%%%%%%%%%%%%%%%%%%%

\section{The XML file}
\label{section:write.xml}

The key to getting your task into casapy is constructing a task
interface description XML file.

Some XML basics, an xml element begins with $<$element$>$ and ends
with $<$/element$>$. If an XML element contains no other XML element
you may specify it via $<$element/$>$. An XML element may have zero or
more attributes which are specified by attribute="attribute
value". You must put the attribute value in quotes, i.e. $<$element
myattribute="attribute value"$>$.

All task xml files must start with this header information.
\begin{verbatim}
<?xml version="1.0" encoding="UTF-8"?>
<?xml-stylesheet type="text/xsl" ?>
<casaxml xmlns="http://casa.nrao.edu/schema/psetTypes.html"
xmlns:xsi="http://www.w3.org/2001/XMLSchema-instance"
xsi:schemaLocation="http://casa.nrao.edu/schema/casa.xsd
file:///opt/casa/code/xmlcasa/xml/casa.xsd">

\end{verbatim}
and the file must have the end tag
\begin{verbatim}
</casaxml>
\end{verbatim}
Inside a $<$task$>$ tags you will need to specify the following elements.
\begin{description}
\item \textbf{$<$task$>$}
\begin{description}
\item \textbf{Attributes}
\begin{description}
\item {type}  required, allowed value is "function"
\item {name}  required 
\end{description}
\item \textbf{Subelements}
\begin{description}
\item [shortdescription] required
\item [description]  required
\item [input]  optional
\item [output]  optional
\item [returns] optional
\item [constraints]  optional
\end{description}
\end{description}
\item [$<$shortdescription$>$] - required by $<$task$>$; A short one-line description describing your task
\begin{description}
\item \textbf{Attributes}
\begin{description}
\item None
\end{description}
\item \textbf{Subelements}
\begin{description}
\item None
\end{description}
\end{description}
\item [$<$description$>$] - required] by $<$task$>$, Also used by $<$param$>$a; A longer description describing your task with multiple lines
\begin{description}
\item \textbf{Attributes}
\begin{description}
\item None
\end{description}
\item \textbf{Subelements}
\begin{description}
\item None
\end{description}
\end{description}
\item[$<$input$>$] - optional element used by $<$task$>$;
 An input block specifies which parameters are used for input
\begin{description}
\item \textbf{Attributes}
\begin{description}
\item None
\end{description}
\item \textbf{Subelements}
\begin{description}
\item [$<$param$>$], optional
\end{description}
\end{description}
\item[$<$output$>$ - optional]
An output element that contains a list of parameters that are "returned" by the task.
\begin{description}
\item \textbf{Attributes}
\begin{description}
\item None
\end{description}
\item \textbf{Subelements}
\begin{description}
\item [$<$param$>$], optional
\end{description}
\end{description}
\item [$<$returns$>$ - optional]
Value returned by the task
\begin{description}
\item \textbf{Attributes}
\begin{description}
\item [type] optional; as specfied in $<$param$>$
\end{description}
\item \textbf{Subelements}
\begin{description}
\item [$<$description$>$], optional
\end{description}
\end{description}
\item[$<$constraints$>$ - optional]
A constraints element that lets you constrain params based on the values of other params.
\begin{description}
\item \textbf{Attributes}
\begin{description}
\item None
\end{description}
\item \textbf{Subelements}
\begin{description}
\item [$<$when$>$], required.
\end{description}
\end{description}
\item [$<$param$>$ - optional]
The input and output elements consist of param elements.
\begin{description}
\item \textbf{Attributes}
\begin{description}
\item [type], required; allowed values are record, variant, string int,
double, bool, intArray, doubleArray, boolArray, stringArray
\item [name], required;
\item [subparam], optional; allowed values True, False, Yes or No.
\item [kind], optional;
\item [mustexist], optional; allowed values True, False, Yes or No.
\end{description}
All param elements require name and type attributes.
\item \textbf{Subelements}
\begin{description}
\item [$<$description$>$], required;
\item [$<$value$>$], optional;
\item [$<$allowed$>$], optional;
\end{description}
\end{description}
\item [$<$value$>$ - optional]
Value returned by the task
\begin{description}
\item \textbf{Attributes}
\begin{description}
\item [type], required; as specified in $<$param$>$ attributes.
\end{description}
\item \textbf{Subelements}
\begin{description}
\item [$<$value$>$], optional
\end{description}
\end{description}
\item [$<$allowed$>$] - optional;
Block of allowed values
\begin{description}
\item \textbf{Attributes}
\begin{description}
\item [enum], required; maybe enum or range. If specfied as enum only specific values are allowed
If specified as range then the value tags may have min and max attributes.
\end{description}
\item \textbf{Subelements}
\begin{description}
\item [$<$value$>$], optional
\end{description}
\end{description}
\item [$<$when$>$ - optional]
When blocks allow value specific handling for parameters
\begin{description}
\item \textbf{Attributes}
\begin{description}
\item [param], required; Specifies special handling for a $<$param$>$
\end{description}
\item \textbf{Subelements}
\begin{description}
\item [$<$equals$>$], optional
\item [$<$notequals$>$], optional
\end{description}
\end{description}
\item [$<$equals$>$ - optional]
Reset parameters if equal to the specified value
\begin{description}
\item \textbf{Attributes}
\begin{description}
\item [value], required; the value of the parameter
\end{description}
\item \textbf{Subelements}
\begin{description}
\item [$<$default$>$], required
\end{description}
\end{description}
\item [$<$notequals$>$ - optional]
Reset specified parameters if not equal to the specified value
\begin{description}
\item \textbf{Attributes}
\begin{description}
\item [value], required; The value of the parameter
\end{description}
\item \textbf{Subelements}
\begin{description}
\item [$<$default$>$], optional
\end{description}
\end{description}
\item [$<$default$>$ - optional]
Resets default values for specified parameters
\begin{description}
\item \textbf{Attributes}
\begin{description}
\item [param], required; Name of the $<$param$>$ to be reset.
\end{description}
\item \textbf{Subelements}
\begin{description}
\item [$<$value$>$], required, the revised value of the $<$param$>$.
\end{description}
\end{description}
\item [$<$example$>$ - optional]
An example block, typically in python
\begin{description}
\item \textbf{Attributes}
\begin{description}
\item \textbf{lang} optional; specifies the language of the example, defaults to python.
\end{description}
\item \textbf{Subelements}
\begin{description}
\item \textbf{None}
\end{description}
\end{description}
\end{description}

%%%%%%%%%%%%%%%%%%%%%%%%%%%%%%%%%%%%%%%%%%%%%%%%%%%%%%%%%%%%%%%%%
%%%%%%%%%%%%%%%%%%%%%%%%%%%%%%%%%%%%%%%%%%%%%%%%%%%%%%%%%%%%%%%%%

\section{The {\tt task\_yourtask.py} file}
\label{section:write.py}

You must write the python code that does the actual work. The
task\_*.py file function call sequence must be the same as specified
in the XML file. We may relax the requirement that the function call
sequence exactly match the sequence in the XML file in a future
release.

%%%%%%%%%%%%%%%%%%%%%%%%%%%%%%%%%%%%%%%%%%%%%%%%%%%%%%%%%%%%%%%%%
%%%%%%%%%%%%%%%%%%%%%%%%%%%%%%%%%%%%%%%%%%%%%%%%%%%%%%%%%%%%%%%%%

\section{Example: The {\tt clean} task}
\label{section:write.clean}

%%%%%%%%%%%%%%%%%%%%%%%%%%%%%%%%%%%%%%%%%%%%%%%%%%%%%%%%%%%%%%%%%

\subsection{File {\tt clean.xml} }
\label{section:write.clean.xml}

Clean.xml gives a fairly comprehensive example of how to construct the
XML file.
\begin{verbatim}

<?xml version="1.0" encoding="UTF-8"?>
<?xml-stylesheet type="text/xsl" ?>
<casaxml xmlns="http://casa.nrao.edu/schema/psetTypes.html"
xmlns:xsi="http://www.w3.org/2001/XMLSchema-instance"
xsi:schemaLocation="http://casa.nrao.edu/schema/casa.xsd
file:///opt/casa/code/xmlcasa/xml/casa.xsd">


<!-- This is the param set for clean -->
<!-- This does the equivalent of -->
<!-- imgr:=imager('anyfile.ms'); -->
<!-- imgr.setdata(mode='channel',nchan=100,start=1,step=1,fieldid=1) -->
<!-- imgr.setimage(nx=512,ny=,cellx='1arcsec',celly='1arcsec',stokes='I',-->
<!--               mode='channel',start=35,step=1,nchan=40, -->
<!--               fieldid=[1]) -->
<!-- imgr.weight('natural'); -->
<!-- imgr.clean(algorithm='csclean',niter=500,model='field1') -->

<task type="function" name="clean">

  <shortdescription>Deconvolve an image with selected algorithm</shortdescription>

  <description>
  Form images from visibilities. Handles continuum and spectral line cubes.
  </description>

  <input>

    <param type="string" name="vis" kind="ms" mustexist="true">
    <description>name of input visibility file</description>
    <value></value>
    </param>

    <param type="string" name="imagename">
	    <description>Pre-name of output images</description>
	    <value></value>
    </param>

    <param type="string" name="field">
      <description>Field Name</description>
      <value></value>
    </param>

    <param type="any" name="spw">
	    <description>Spectral windows:channels: \'\' is all </description>
	    <any type="variant"/>
	    <value type="string"></value>
    </param>
    <param type="bool" name="selectdata">
	    <description>Other data selection parameters</description>
	    <value>False</value>
    </param>
    
    <param type="string" name="timerange" subparam="true">
	    <description>Range of time to select from data</description>
	    <value></value>
    </param>
    <param type="string" name="uvrange" subparam="true">
	    <description>Select data within uvrange </description>
	    <value></value>
    </param>
    <param type="string" name="antenna" subparam="true">
	    <description>Select data based on antenna/baseline</description>
	    <value></value>
    </param>
    <param type="string" name="scan" subparam="true">
	    <description>scan number range</description>
	    <value></value>
    </param>
    

    <param type="string" name="mode">
	    <description>
               Type of selection (mfs, channel, velocity, frequency)
            </description>
	    <value>mfs</value>
	    <allowed kind="enum">
	    <value>mfs</value>
	    <value>channel</value>
	    <value>velocity</value>
	    <value>frequency</value>
    </allowed>
    </param>

    

    <param type="int" name="niter">
    <description>Maximum number of iterations</description>
    <value>500</value>
    </param>

    <param type="double" name="gain">
	    <description>Loop gain for cleaning</description>
	    <value>0.1</value>
    </param>

    <param type="double" name="threshold" units="mJy">
	    <description>Flux level to stop cleaning.  Must include units</description>
	    <value>0.0</value>
    </param>

<!-- Getting rid of this
    <param type="bool" name="csclean">
      <description>Use Cotton-Schwab style reconciliation with UV-data</description>
      <value>False</value>
      
    </param>
-->

    <param type="string" name="psfmode">
	    <description>method of PSF calculation to use during minor cycles</description>
	    <value>clark</value>
	    <allowed kind="enum">
	      <value>clark</value>
	      <value>hogbom</value>
	    </allowed>
    </param>

    <param type="string" name="imagermode">
      <description> Use csclean or mosaic.  If \'\', use psfmode</description>
      <value></value>
      <allowed kind="enum">
	      <value></value>
	      <value>csclean</value>
	      <value>mosaic</value>
      </allowed>


    </param>
    <param type="string" name="ftmachine" subparam="true">
	    <description>Gridding method for the image</description>
	    <value>mosaic</value>
	    <allowed kind="enum">
	      <value>mosaic</value>
	      <value>ft</value>
	      <value>sd</value>
	      <value>both</value>
	    </allowed>
	    
    </param>
    <param type="bool" name="mosweight" subparam="true">
      <description>Individually weight the fields of the mosaic</description>
      <value>False</value>
    </param>
    <param type="string" name="scaletype" subparam="true">
      <description>Controls scaling of pixels in the image plane. 
          default=\'SAULT\'; 
          example: scaletype=\'PBCOR\' Options: \'PBCOR\',\'SAULT\'</description>
      <value>SAULT</value>
      <allowed kind="enum">
	<value>SAULT</value>
	<value>PBCOR</value>
      </allowed>
    </param>
    

    <param type="intArray" name="multiscale">
      <description>set deconvolution scales (pixels), 
        default: multiscale=[] (standard CLEAN)</description>
      <value type="vector">
	<value></value>
      </value>

    </param>
    <param type="int" name="negcomponent" subparam="true">
      <description>
         Stop cleaning if the largest scale finds this number of neg components
      </description>
      <value>0</value>
    </param>


    <param type="bool" name="interactive">
	    <description>use interactive clean (with GUI viewer)</description>
	    <value>False</value>
    </param>

    <param type="any" name="mask">
	    <description>cleanbox(es), mask image(s), and/or region(s)  used in cleaning
            </description>
	    <any type="variant"/>
	    <value type="stringArray"></value>
    </param>


    <param type="int" name="nchan" subparam="true">
	    <description>Number of channels (planes) in output image</description>
	    <value>1</value>
    </param>

    <param type="any" name="start" subparam="true">
	    <description>First channel in input to use</description>
	    <any type="variant"/>
	    <value type="int">0</value>
    </param>

    <param type="any" name="width" subparam="true">
	    <description>Number of input channels to average</description>
	    <any type="variant"/>
	    <value type="int">1</value>
    </param>


    <param type="intArray" name="imsize">
	    <description>x and y image size in pixels, symmetric for single value
            </description>
	    <value type="vector">
    <value>256</value><value>256</value>
	    </value>
    </param>

    <param type="doubleArray" name="cell" units="arcsec">
    <description>x and y cell size. default unit arcsec</description>
    <value type="vector"><value>1.0</value><value>1.0</value></value>
    </param>

    <param type="any" name="phasecenter">
	    <description>Image phase center: position or field index</description>
	    <any type="variant"/>
	    <value type="string"></value>
    </param>

    <param type="string" name="restfreq">
	    <description>rest frequency to assign to image (see help)</description>
	    <value></value>
    </param>

    <param type="string" name="stokes">
	    <description>Stokes params to image (eg I,IV, QU,IQUV)</description>
	    <value>I</value>
    <allowed kind="enum">
	    <value>I</value>
	    <value>IV</value>
	    <value>QU</value>
	    <value>IQUV</value>
	    <value>RR</value>
	    <value>LL</value>
	    <value>RRLL</value>
	    <value>XX</value>
	    <value>YY</value>
	    <value>XXYY</value>
    </allowed>
    </param>

    

    

    <param type="string" name="weighting">
    <description>Weighting to apply to visibilities</description>
    <value>natural</value>
    <allowed kind="enum">
	    <value>natural</value>
	    <value>uniform</value>
	    <value>briggs</value>
	    <value>briggsabs</value>
	    <value>radial</value>
	    <value>superuniform</value>
    </allowed>
    </param>


    <param type="double" name="robust" subparam='true'>
	    <description>Briggs robustness parameter</description>
	    <value>0.0</value>
	    <allowed kind="range">
	    <value range="min">-2.0</value>
	    <value range="max">2.0</value>
    </allowed>
    </param>

    <param type="bool" name="uvtaper">
	    <description>Apply additional uv tapering of  visibilities.</description>
	    <value>False</value>
    </param>

    <param type="stringArray" name="outertaper" subparam="true">
	    <description>uv-taper on outer baselines in uv-plane</description>
	    <value type="vector">
	      <value></value>
	    </value>
    </param>

    <param type="stringArray" name="innertaper" subparam="true">
	    <description>uv-taper in center of uv-plane</description>
	    <value>1.0</value>
    </param>

    <param type="string" name="modelimage">
	    <description>Name of model image(s) to initialize cleaning</description>
	    <value></value>
    </param>
    <param type="stringArray" name="restoringbeam">
      <description>Output Gaussian restoring beam for CLEAN image</description>
      <value></value>
    </param>
    <param type="bool" name="pbcor">
	    <description>Output primary beam-corrected image</description>
	    <value>False</value>
    </param>

    <param type="double" name="minpb">
	    <description>Minimum PB level to use</description>
	    <value>0.1</value>
    </param>


    <param type="any" name="noise"  subparam='true'>
	    <description>noise parameter for briggs abs mode weighting</description>
	    <any type="variant"/>
	    <value type="string">1.0Jy</value>	    
    </param>

    <param type="int" name="npixels" subparam='true'>
	    <description>number of pixels for superuniform or briggs weighting
            </description>
	    <value>0</value>
    </param>



    <param type="int" name="npercycle" subparam='true'>
	    <description>Number of iterations before interactive prompt</description>
	    <value>100</value>
    </param>
    <param type="double" name="cyclefactor" subparam='true'>
      <description>change depth in between of  csclean cycle</description>
      <value>1.5</value>
    </param>
    <param type="int" name="cyclespeedup" subparam='true'>
	    <description>Cycle threshold doubles in this number of iteration</description>
	    <value>-1</value>
    </param>
    

    

    <constraints>
            <when param="selectdata">
		  <equals type="bool" value="False"/>
		  <equals type="bool" value="True">
		    <default param="timerange"><value type="string"></value>
		    </default>
		    <default param="uvrange"><value type="string"></value>
		    </default>
		    <default param="antenna"><value type="string"></value>
		    </default>
		    <default param="scan"><value type="string"></value>
		    </default>
	        </equals>
            </when>
	    <when param="multiscale">
		  <notequals type="vector" value="[]" > 
		    <default param="negcomponent"><value>-1</value>
		    </default>
	        </notequals>
            </when>
	    <when param="mode">
		<equals value="mfs"/>
		<equals value="channel">
			<default param="nchan"><value>1</value></default>
			<default param="start"><value>0</value>
			    <description>first input channel to use</description>
			</default>
			<default param="width"><value>1</value></default>
		</equals>
	        <equals value="velocity">
			<default param="nchan"><value>1</value></default>
			<default param="start"><value type="string">0.0km/s</value>
			    <description>Velocity of first image channel: e.g \'0.0km/s\'
                            </description>
		    </default>
		    <default param="width"><value type="string">1km/s</value>
			    <description>image channel width in velocity units: 
                               e.g \'-1.0km/s\'</description>
		    </default>
	        </equals>
	        <equals value="frequency">
			<default param="nchan"><value>1</value></default>
			<default param="start"><value type="string">1.4GHz</value>
			    <description>Frequency of first image channel: e.q. \'1.4GHz\'
                            </description>
		    </default>
		    <default param="width"><value type="string">10kHz</value>
			    <description>Image channel width in frequency units:
                                e.g \'1.0kHz\'</description>
		    </default>
	        </equals>
	    </when>
	    
	    <when param="weighting">
		<equals value="natural"/>
	        <equals value="uniform"/>
	        <equals value="briggs">
			<default param="robust"><value>0.0</value></default>
			<default param="npixels"><value>0</value>
			    <description>number of pixels to determine uv-cell size
                                 0=&gt; field of view</description>
		    </default>
	        </equals>
		<equals value="briggsabs">
			<default param="robust"><value>0.0</value></default>
			<default param="noise"><value type="string">1.0Jy</value></default>
			<default param="npixels"><value>0</value>
			    <description>number of pixels to determine uv-cell size
                                 0=&gt; field of view</description>
		    </default>
	        </equals>
	        <equals value="superuniform">
			<default param="npixels"><value>0</value>
			    <description>number of pixels to determine uv-cell size
                                 0=&gt; +/-3pixels</description>
		    </default>
	        </equals>
            </when>
	    <when param="uvtaper">
		<equals type="bool" value="False"/>
		<equals type="bool" value="True">
			<default param="outertaper"><value type="vector"></value></default>
			<default param="innertaper"><value type="vector"></value></default>
	        </equals>
            </when>
	    <when param="interactive">
		<equals type="bool" value="False"/>
		<equals type="bool" value="True">
		      <default param="npercycle"><value>100</value></default>
	        </equals>
            </when>
	    <when param="imagermode">
		<equals value=""/>
		<equals value="csclean">
			<default param="cyclefactor"><value>1.5</value></default>
			<default param="cyclespeedup"><value>-1</value></default>
	        </equals>
                <equals value="mosaic">
			<default param="mosweight"><value>False</value></default>
			<default param="ftmachine"><value type="string">mosaic</value>
                                </default>
			<default param="scaletype"><value type="string">SAULT</value>
                                </default>
			<default param="cyclefactor"><value>1.5</value></default>
			<default param="cyclespeedup"><value>-1</value></default>
	        </equals>
            </when>
<!--Get rid of that soon
	    <when param="mosaicmode">
		<equals type="bool" value="False"/>
		<equals type="bool" value="True">
			<default param="mosweight"><value>False</value></default>
			<default param="ftmachine"><value type="string">mosaic</value></default>
			<default param="scaletype"><value type="string">SAULT</value></default>
	        </equals>
            </when>
-->
    </constraints>

    </input>

  <returns type="void"/>

  <example>

       The main clean deconvolution task.  It contains many functions
 
        1)  Make 'dirty' image and 'dirty' beam (psf)
        2)  Multi-frequency-continuum images or spectral channel imaging
        3)  Full Stokes imaging
        4)  Mosaicking of several pointings
        5)  Multi-scale cleaning
        6)  Interactive clean boxing
        7)  Initial starting model
 
 
       vis -- Name of input visibility file
               default: none; example: vis='ngc5921.ms'
       imagename -- Pre-name of output images:
               default: none; example: imagename='m2'
               output images are:
                 m2.image; cleaned and restored image
                        With or without primary beam correction
                 m2.psf; point-spread function (dirty beam)
                 m2.flux;  relative sky sensitivity over field
                 m2.model; image of clean components
                 m2.residual; image of residuals
                 m2.interactive.mask; image containing clean regions
       field -- Select fields in mosaic.  Use field id(s) or field name(s).
                  ['go listobs' to obtain the list id's or names]
              default: ''=all fields
              If field string is a non-negative integer, it is assumed to
                  be a field index otherwise, it is assumed to be a 
		  field name
              field='0~2'; field ids 0,1,2
              field='0,4,5~7'; field ids 0,4,5,6,7
              field='3C286,3C295'; field named 3C286 and 3C295
              field = '3,4C*'; field id 3, all names starting with 4C
       spw -- Select spectral window/channels
              NOTE: This selects the data passed as the INPUT to mode
              default: ''=all spectral windows and channels
                spw='0~2,4'; spectral windows 0,1,2,4 (all channels)
                spw='0:5~61'; spw 0, channels 5 to 61
                spw='&lt;2';   spectral windows less than 2 (i.e. 0,1)
                spw='0,10,3:3~45'; spw 0,10 all channels, spw 3, 
				   channels 3 to 45.
                spw='0~2:2~6'; spw 0,1,2 with channels 2 through 6 in each.
                spw='0:0~10;15~60'; spectral window 0 with channels 
				    0-10,15-60
                spw='0:0~10,1:20~30,2:1;2;3'; spw 0, channels 0-10,
                      spw 1, channels 20-30, and spw 2, channels, 1,2 and 3
       selectdata -- Other data selection parameters
              default: True
  &gt;&gt;&gt; selectdata=True expandable parameters
              See help par.selectdata for more on these
              timerange  -- Select data based on time range:
                 default = '' (all); examples,
                  timerange = 'YYYY/MM/DD/hh:mm:ss~YYYY/MM/DD/hh:mm:ss'
                  Note: if YYYY/MM/DD is missing date defaults to first 
			day in data set
                  timerange='09:14:0~09:54:0' picks 40 min on first day
                  timerange= '25:00:00~27:30:00' picks 1 hr to 3 hr 
			     30min on NEXT day
                  timerange='09:44:00' pick data within one integration 
		             of time
                  timerange='&gt;10:24:00' data after this time
              uvrange -- Select data within uvrange (default units meters)
                  default: '' (all); example:
                  uvrange='0~1000klambda'; uvrange from 0-1000 kilo-lambda
                  uvrange='&gt;4klambda';uvranges greater than 4 kilo lambda
              antenna -- Select data based on antenna/baseline
                  default: '' (all)
                  If antenna string is a non-negative integer, it is 
 		    assumed to be an antenna index, otherwise, it is
 		    considered an antenna name.
                  antenna='5&amp;6'; baseline between antenna index 5 and 
 				 index 6.
                  antenna='VA05&amp;VA06'; baseline between VLA antenna 5 
 				       and 6.
                  antenna='5&amp;6;7&amp;8'; baselines 5-6 and 7-8
                  antenna='5'; all baselines with antenna index 5
                  antenna='05'; all baselines with antenna number 05 
 				(VLA old name)
                  antenna='5,6,9'; all baselines with antennas 5,6,9 
 				   index numbers
              scan -- Scan number range.
                  default: '' (all)
                  example: scan='1~5'
                  Check 'go listobs' to insure the scan numbers are in 
 			order.
       mode -- Frequency Specification:
               NOTE: See examples below:
               default: 'mfs'
                 mode = 'mfs' means produce one image from all 
 		      specified data.
                 mode = 'channel'; Use with nchan, start, width to specify
                         output image cube.  See examples below
                 mode = 'velocity', means channels are specified in 
 		      velocity.
                 mode = 'frequency', means channels are specified in 
 		      frequency.
  &gt;&gt;&gt; mode expandable parameters (for modes other than 'mfs')
               Start, width are given in units of channels, frequency 
 		  or velocity as indicated by mode, but only channel
 		  is complete.
               nchan -- Number of channels (planes) in output image
                 default: 1; example: nchan=3
               start -- Start input channel (relative-0)
                 default=0; example: start=5
               width -- Output channel width in units of the input
 		     channel width (&gt;1 indicates channel averaging)
                 default=1; example: width=4
           examples:
               spw = '0,1'; mode = 'mfs'
                  will produce one image made from all channels in spw 
 		       0 and 1
               spw='0:5~28^2'; mode = 'mfs'
                  will produce one image made with channels 
 		       (5,7,9,...,25,27)
               spw = '0'; mode = 'channel': nchan=3; start=5; width=4
                  will produce an image with 3 output planes
                  plane 1 contains data from channels (5+6+7+8)
                  plane 2 contains data from channels (9+10+11+12)
                  plane 3 contains data from channels (13+14+15+16)
               spw = '0:0~63^3'; mode='channel'; nchan=21; start = 0; 
 		   width = 1
                  will produce an image with 20 output planes
                  Plane 1 contains data from channel 0
                  Plane 2 contains date from channel 2
                  Plane 21 contains data from channel 61
               spw = '0:0~40^2'; mode = 'channel'; nchan = 3; start = 
 		   5; width = 4
                  will produce an image with three output planes
                  plane 1 contains channels (5,7)
                  plane 2 contains channels (13,15)
                  plane 3 contains channels (21,23)
       psfmode -- method of PSF calculation to use during minor cycles:
               default: 'clark': Options: 'clark','hogbom'
               'clark'  use smaller beam (faster, usually good enough)
               'hogbom' full-width of image (slower, better for poor 
 	       uv-coverage)
               Note:  psfmode will be used to clean is imagermode = ''
       imagermode -- Advanced imaging e.g mosaic or Cotton-Schwab clean
               default: imagermode='': Options: '', 'csclean', 'mosaic'
               default ''  =&gt; psfmode cleaning algorithm used
  &gt;&gt;&gt; imagermode='mosaic' expandable parameter(s):
         Image as a mosaic of the different pointings (uses csclean 
 	 style too)
               mosweight -- Individually weight the fields of the mosaic
                       default: False; example: mosweight=True
                       This can be useful if some of your fields are more
                       sensitive than others (i.e. due to time spent 
 		       on-source); this parameter will give more weight to 
 		       higher sensitivity fields in the overlap regions.
               ftmachine -- Gridding method for the image;
                       Options: ft (standard interferometric gridding), sd
                       (standard single dish) both (ft and sd as appropriate),
                       mosaic (gridding use PB as convolution function)
                       default: 'mosaic'; example: ftmachine='ft'
               scaletype -- Controls scaling of pixels in the image plane.
                       (Not fully implemented...for now only controls 
 		       what is seen if interactive=True...but in the future will 
 		       control the image on which clean components are searched)
                       default='SAULT'; example: scaletype='PBCOR'
                       Options: 'PBCOR','SAULT'
                       'SAULT' when interactive=True shows the residual
 			      with constant noise across the mosaic. If
 			      pbcor=False, the final output image is NOT
 			      corrected for the PB pattern, and therefore is
 			      not "flux correct". Division of SAULT
 			      &lt;imagename&gt;.image by the &lt;imagename&gt;.flux image
 			      will produce a "flux correct image", can also
 			      be acheived by setting pbcor=True.
                       'PBCOR' uses the SAULT scaling scheme for
 			      deconvolution, but if interactive=True shows the
 			      primary beam corrected image; the final PBCOR
 			      image is "flux correct" if pbcor=True.
  &gt;&gt;&gt; imagermode='csclean' expandable parameter(s): Image using the
            Cotton-Schwab algorithm in between major cycles
 	    cyclefactor -- Change the threshold at which
 			  the deconvolution cycle will stop, degrid
 			  and subtract from the visibilities.  For
 			  poor PSFs, reconcile often (cyclefactor=4 or
 			  5); For good PSFs, use cyclefactor 1.5 to
 			  2.0. Note: threshold = cyclefactor * max
 			  sidelobe * max residual.
 			  default: 1.5; example: cyclefactor=4
 	    cyclespeedup -- Cycle threshold doubles in this
 			  number of iterations default: -1;
 			  example: cyclespeedup=3
                          try cyclespeedup = 50 to speed up cleaning 
       multiscale -- set of scales to use in deconvolution.  If set,
               cleans with several resolutions using hobgom clean. The
               scale sizes are in units of cellsize.  So if
               cell='2arcsec', a multiscale scale=10 = 20arcsec.  First
               scale should always be 0 (point), we suggest second on
               the order of synthesized beam, third 3-5 times
               synthesized beam, etc. For example if synthesized beam
               is 10" and cell=2", try multscale = [0,5,15]. Note,
               multiscale is currently a bit slow.
 	      default: multiscale=[] (standard CLEAN using psfmode algorithm,
               no multi-scale). Example:  multscale = [0,5,15] 
  &gt;&gt;&gt; multiscale expandable parameter(s): negcomponent -- Stop
               component search when the largest scale has found this
               number of negative components; -1 means continue
               component search even if the largest component is
               negative.  default: -1; example: negcomponent=50
       imsize -- Image pixel size (x,y)
               default = [256,256]; example: imsize=[350,350]
               imsize = 500 is equivalent to [500,500]
       cell -- Cell size (x,y)
               default= '1.0arcsec';
               example: cell=['0.5arcsec,'0.5arcsec'] or
               cell=['1arcmin', '1arcmin']
               cell = '1arcsec' is equivalent to ['1arcsec','1arcsec']
               NOTE:cell = 2.0 =&gt; ['2arcsec', '2arcsec']
       phasecenter -- direction measure  or fieldid for the mosaic center
               default: '' =&gt; first field selected ; example: phasecenter=6
               or phasecenter='J2000 19h30m00 -40d00m00'
       restfreq -- Specify rest frequency to use for output image
               default='' Occasionally it is necessary to set this (for
               example some VLA spectral line data).  For example for
               NH_3 (1,1) put restfreq='23.694496GHz'
       stokes -- Stokes parameters to image
               default='I'; example: stokes='IQUV';
               Options: 'I','IV''QU','IQUV','RR','LL','XX','YY','RRLL','XXYY'
       niter -- Maximum number iterations,
               if niter=0, then no CLEANing is done ("invert" only)
               default: 500; example: niter=5000
       gain -- Loop gain for CLEANing
               default: 0.1; example: gain=0.5
       threshold -- Flux level at which to stop CLEANing
               default: '0.0mJy'; 
               example: threshold='2.3mJy'  (always include units)
                        threshold = '0.0023Jy'
                        threshold = '0.0023Jy/beam' (okay also)
       interactive -- use interactive clean (with GUI viewer)
               default: interactive=False
               example: interactive=True
               interactive clean allows the user to build the cleaning
                    mask interactively using the viewer.  The viewer will
                    appear every npercycle interation, but modify as needed
                    The final interactive maks is saved in the file
                    imagename_interactive.mask.  The initial masks use the
                    union of mask and cleanbox (see below)
  &gt;&gt;&gt; interactive=True expandable parameter npercycle -- this is the
               number of iterations between each clean to update mask
               interactively. Set to about niter/5, but can also be
               changed interactively.
       mask -- Specification of cleanbox(es), mask image(s), and/or
 	   region(s) to be used for CLEANing. As long as the image has
 	   the same shape (size), mask images from a previous
 	   interactive session can be used for a new execution. NOTE:
 	   the initial clean mask actually used is the union of what
 	   is specified in mask and &lt;imagename&gt;.mask default: [] (no
 	   masking); Possible pecification types: (a) Explicit
 	   cleanbox pixel ranges example: mask=[110,110,150,145] clean
 	   region with blc=110,100; trc=150,145 (pixel values) (b)
 	   Filename with cleanbox pixel values with ascii format:
 	   example: mask='mycleanbox.txt' &lt;fieldid blc-x blc-y
 	   trc-x trc-y&gt; on each line 
 	   1 45 66 123 124 
 	   2 23 100 300 340
 	   (c) Filename for image mask example: mask='myimage.mask'
 	   (d) Filename for region specification (e.g. from viewer)
 	   example: mask='myregion.rgn' (e) Combinations of any of the
 	   above example: mask=[[110,110,150,145],'mycleanbox.txt',
 	   'myimage.mask','myregion.rgn']
       uvtaper -- Apply additional uv tapering of the visibilities.
               default: uvtaper=False; example: uvtaper=True
  &gt;&gt;&gt; uvtaper=True expandable parameters
               outertaper -- uv-taper on outer baselines in uv-plane
                 [bmaj, bmin, bpa] taper Gaussian scale in uv or 
 		 angular units. NOTE: uv taper in (klambda) is roughly on-sky 
 	         FWHM(arcsec/200)
                 default: outertaper=[]; no outer taper applied
 		 example: outertaper=['5klambda']  circular taper 
 				FWHM=5 kilo-lambda
                         outertaper=['5klambda','3klambda','45.0deg']
                         outertaper=['10arcsec'] on-sky FWHM 10"
                         outertaper=['300.0'] default units are meters 
 			        in aperture plane
               innertaper -- uv-taper in center of uv-plane
                 [bmaj,bmin,bpa] Gaussian scale at which taper falls to 
 		 zero at uv=0
                 default: innertaper=[]; no inner taper applied
                 NOT YET IMPLEMENTED                
       modelimage -- Name of model image(s) to initialize cleaning. If
               multiple images, then these will be added together to
               form initial staring model NOTE: these are in addition
               to any initial model in the &lt;imagename&gt;.model image file
               default: '' (none); example: modelimage='orion.model'
               modelimage=['orion.model','sdorion.image'] Note: if the
               units in the image are Jy/beam as in a single-dish
               image, then it will be converted to Jy/pixel as in a
               model image, using the restoring beam in the image
               header
       weighting -- Weighting to apply to visibilities:
               default='natural'; example: weighting='uniform';
              Options: 'natural','uniform','briggs', 
 		       'superuniform','briggsabs','radial'
  &gt;&gt;&gt; Weighting expandable parameters
               For weighting='briggs' and 'briggsabs'
                 robust -- Brigg's robustness parameter
                   default=0.0; example: robust=0.5;
                   Options: -2.0 to 2.0; -2 (uniform)/+2 (natural)
              For weighting='briggsabs'
                 noise   -- noise parameter to use for Briggs "abs" 
 		weighting
                   example noise='1.0mJy'
               For superuniform/briggs/briggsabs weighting
                 npixels -- number of pixels to determine uv-cell size
                   for weight calculation
                   example npixels=7
       restoringbeam -- Output Gaussian restoring beam for CLEAN image
               [bmaj, bmin, bpa] elliptical Gaussian restoring beam
               default units are in arc-seconds for bmaj,bmin, degrees
               for bpa default: restoringbeam=[]; Use PSF calculated
               from dirty beam. 
 	      example: restoringbeam=['10arcsec'] circular Gaussian 
 		       FWHM 10" example:
 		       restoringbeam=['10.0','5.0','45.0deg'] 10"x5" 
 	               at 45 degrees
       pbcor -- Output primary beam-corrected image 
 	      default: pbcor=False; output un-corrected image 
 	      example: pbcor=True; output pb-corrected image (masked outside
               minpb) Note: if you set pbcor=False, you can later
               recover the pbcor image by dividing by the .flux image
               (e.g. using immath)
       minpb -- Minimum PB level to use default=0.1; example:
               minpb=0.01 Note: this minpb is always in effect
               (regardless of pbcor=True/False) 
 	      async -- Run asynchronously 
  		    default = False; do not run asychronously



  </example>

</task>

</casaxml>
\end{verbatim}

%%%%%%%%%%%%%%%%%%%%%%%%%%%%%%%%%%%%%%%%%%%%%%%%%%%%%%%%%%%%%%%%%

\subsection{File {\tt task\_clean.py} }
\label{section:write.clean.py}

Task clean implementation file.
\begin{verbatim}

import os
from taskinit import *
from cleanhelper import *


def clean(vis,imagename,field, spw, selectdata, timerange, uvrange, antenna, 
          scan, mode,niter, gain,threshold, psfmode, imagermode, ftmachine,
          mosweight, scaletype, multiscale, negcomponent,interactive,mask,
          nchan,start,width,imsize,cell, phasecenter, restfreq, stokes,
          weighting,robust, uvtaper,outertaper,innertaper, modelimage,
          restoringbeam,pbcor, minpb,  noise, npixels, npercycle, cyclefactor,
          cyclespeedup):

	#Python script

        casalog.origin('clean')

	maskimage=''
	if((mask==[]) or (mask=='')):
		mask=['']
	if (interactive):
		if( (mask=='') or (mask==['']) or (mask==[])):
			maskimage=imagename+'.mask'

	#try:
	if(1):
		imCln=imtool.create()
		imset=cleanhelper(imCln, vis)
		
		if((len(imagename)==0) or (imagename.isspace())):
			raise Exception, 'Cannot proceed with blank imagename'
		casalog.origin('clean')
		
                imset.defineimages(imsize=imsize, cell=cell, stokes=stokes,
				   mode=mode, spw=spw, nchan=nchan,
				   start=start,  width=width,
				   restfreq=restfreq, field=field,
				   phasecenter=phasecenter)
		
                imset.datselweightfilter(field=field, spw=spw,
					 timerange=timerange, uvrange=uvrange,
					 antenna=antenna, scan=scan,
					 wgttype=weighting, robust=robust,
					 noise=noise, npixels=npixels,
					 mosweight=mosweight,
					 innertaper=innertaper,
					 outertaper=outertaper)
		if(maskimage==''):
			maskimage=imagename+'.mask'
		imset.makemaskimage(outputmask=maskimage,imagename=imagename,
				    maskobject=mask)


		
		###define clean alg
		alg=psfmode
		if(multiscale==[0]):
			multiscale=[]
		if((type(multiscale)==list) and (len(multiscale)>0)):
			alg='multiscale' 
			imCln.setscales(scalemethod='uservector',
					uservector=multiscale)
		if(imagermode=='csclean'):
			alg='mf'+alg
		if(imagermode=='mosaic'):
			if(alg.count('mf') <1):
				alg='mf'+alg
			imCln.setoptions(ftmachine=ftmachine, padding=1.0)
			imCln.setvp(dovp=True)
		###PBCOR or not
		sclt='SAULT'
		if((scaletype=='PBCOR') or (scaletype=='pbcor')):
			sclt='NONE'
			imCln.setvp(dovp=True)
		else:
			if(imagermode != 'mosaic'):
				##make a pb for flux scale
				imCln.setvp(dovp=True)
				imCln.makeimage(type='pb', image=imagename+'.flux')
				imCln.setvp(dovp=False)
		##restoring
		imset.setrestoringbeam(restoringbeam)
		###model image
		imset.convertmodelimage(modelimages=modelimage,
					outputmodel=imagename+'.model')

		####after all the mask shenanigans...make sure to use the
		####last mask
		maskimage=imset.outputmask
		if((imagermode=='mosaic')):
			imCln.setmfcontrol(stoplargenegatives=negcomponent,scaletype=sclt,
                                           minpb=minpb,cyclefactor=cyclefactor,
                                           cyclespeedup=cyclespeedup,
                                           fluxscale=[imagename+'.flux'])
		else:
			imCln.setmfcontrol(stoplargenegatives=negcomponent,
                             cyclefactor=cyclefactor, cyclespeedup=cyclespeedup)
			
		imCln.clean(algorithm=alg,niter=niter,gain=gain,
                            threshold=qa.quantity(threshold,'mJy'),
                            model=[imagename+'.model'],
                            residual=[imagename+'.residual'],
                            image=[imagename+'.image'], 
                            psfimage=[imagename+'.psf'], 
                            mask=maskimage, interactive=interactive, 
                            npercycle=npercycle)
		imCln.close()
		presdir=os.path.realpath('.')
		newimage=imagename
		if(imagename.count('/') > 0):
			newimage=os.path.basename(imagename)
			os.chdir(os.path.dirname(imagename))
		result          = '\'' + newimage + '.image' + '\'';
		fluxscale_image = '\'' + newimage + '.flux'  + '\'';
		if (pbcor):
			if(sclt != 'NONE'):
				##otherwise its already divided
				ia.open(newimage+'.image')
				
				pixmask = fluxscale_image+'>'+str(minpb);
				ia.calcmask(pixmask,asdefault=True);

				pixels='iif('+ fluxscale_image+'>'+str(minpb)+','
                                        + result+'/'+fluxscale_image+', 0)'
				ia.calc(pixels=pixels)
				ia.close()
		else:
			##  people has imaged the fluxed corrected image
			## but want the
			## final image to be non-fluxed corrected
			if(sclt=='NONE'):
				ia.open(newimage+'.image')
				result=newimage+'.image'
				fluxscale_image=newimage+'.flux'
				pixels=result+'*'+fluxscale_image
				ia.calc(pixels=pixels)
				ia.close()
		os.chdir(presdir)

		
		del imCln

#	except Exception, instance:
#		print '*** Error *** ',instance
#		raise Exception, instance

\end{verbatim}

%%%%%%%%%%%%%%%%%%%%%%%%%%%%%%%%%%%%%%%%%%%%%%%%%%%%%%%%%%%%%%%%%
%%%%%%%%%%%%%%%%%%%%%%%%%%%%%%%%%%%%%%%%%%%%%%%%%%%%%%%%%%%%%%%%%
