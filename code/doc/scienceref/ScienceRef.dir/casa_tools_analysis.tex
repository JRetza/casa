%%%%%%%%%%%%%%%%%%%%%%%%%%%%%%%%%%%%%%%%%%%%%%%%%%%%%%%%%%%%%%%%%
%%%%%%%%%%%%%%%%%%%%%%%%%%%%%%%%%%%%%%%%%%%%%%%%%%%%%%%%%%%%%%%%%
%%%%%%%%%%%%%%%%%%%%%%%%%%%%%%%%%%%%%%%%%%%%%%%%%%%%%%%%%%%%%%%%%

% STM 2007-04-13  split from previous version

\chapter{Image Analysis}
\label{chapter:ia}

%%%%%%%%%%%%%%%%%%%%%%%%%%%%%%%%%%%%%%%%%%%%%%%%%%%%%%%%%%%%%%%%%
\section{Open a file}
\label{section:ia.open}

To open an image file for analysis, use 
the {\tt ia.open} method, e.g.: 
\small
\begin{verbatim}
  ia.open('ngc5921_task.image')
\end{verbatim}
\normalsize

%%%%%%%%%%%%%%%%%%%%%%%%%%%%%%%%%%%%%%%%%%%%%%%%%%%%%%%%%%%%%%%%%
\section{Get an image summary}
\label{section:ia.summary}

To get a summary of the properties of your image, use 
the {\tt ia.summary} method, e.g.: 
\small
\begin{verbatim}
  imhead(imagename='image.im')
\end{verbatim}
\normalsize

%%%%%%%%%%%%%%%%%%%%%%%%%%%%%%%%%%%%%%%%%%%%%%%%%%%%%%%%%%%%%%%%%
\section{Image statistics}
\label{section:ia.stats}

The {\tt ia.statistics} method computes statistics in a (region)
of an image.

\small
\begin{verbatim}
  CASA <50>: ia.open('ngc5921_task.image')
  CASA <51>: ia.statistics() # Note: formatted output goes to the logger
  {'return': True,
   'statsout': {'blc': [0, 0, 0, 0],
              'blcf': '15:24:08.404, +04.31.59.181, I, 1.41281e+09Hz',
              'flux': [12.136708280654085],
              'max': [0.12773475050926208],
              'maxpos': [134, 134, 0, 38],
              'maxposf': '15:21:53.976, +05.05.29.998, I, 1.41374e+09Hz',
              'mean': [4.7899158775357123e-05],
              'min': [-0.023951441049575806],
              'minpos': [230, 0, 0, 15],
              'minposf': '15:20:17.679, +04.31.59.470, I, 1.41317e+09Hz',
              'npts': [3014656.0],
              'rms': [0.00421889778226614],
              'sigma': [0.0042186264443439979],
              'sum': [144.399486397083],
              'sumsq': [53.658156081703709],
              'trc': [255, 255, 0, 45],
              'trcf': '15:19:52.390, +05.35.44.246, I, 1.41391e+09Hz'}}
\end{verbatim}
\normalsize

Restrict statistics to a region:

\small
\begin{verbatim}
  CASA <60>: blc=[0,0,0,23]

  CASA <61>: trc=[255,255,0,23]

  CASA <62>: bbox=ia.setregion(blc,trc)

  CASA <63>: ia.statistics(region=bbox)

  0%....10....20....30....40....50....60....70....80....90....100%
  Warning no plotter attached.  Attach a plotter to get plots
    Out[63]: 
  {'return': True,
   'statsout': {'blc': [0, 0, 0, 23],
              'blcf': '15:24:08.404, +04.31.59.181, I, 1.41337e+09Hz',
              'flux': [0.21697188625158217],
              'max': [0.061052091419696808],
              'maxpos': [124, 132, 0, 23],
              'maxposf': '15:22:04.016, +05.04.59.999, I, 1.41337e+09Hz',
              'mean': [3.9390207550122095e-05],
              'min': [-0.018510516732931137],
              'minpos': [254, 20, 0, 23],
              'minposf': '15:19:53.588, +04.36.59.216, I, 1.41337e+09Hz',
              'npts': [65536.0],
              'rms': [0.0040461532771587372],
              'sigma': [0.0040459925613279676],
              'sum': [2.5814766420048016],
              'sumsq': [1.0729132921679772],
              'trc': [255, 255, 0, 23],
              'trcf': '15:19:52.390, +05.35.44.246, I, 1.41337e+09Hz'}}
\end{verbatim}
\normalsize

You can also use the viewer to interactively obtain image statistics on a region:

\small
\begin{verbatim}
  viewer('imagename.im')
  # Now use the right mouse button (default for region setting)
  # create a region and then double click inside to obtain statistics on that region
  # Currently this supports a single plane only and the output goes to your casapy 
  # terminal window as:

  ngc5921_task.image

  n           Std Dev     RMS         Mean        Variance    Sum
  660         0.01262     0.0138      0.005622    0.0001592   3.711     
  
  Flux        Med |Dev|   Quartile    Median      Min         Max
  0.3119      0.003758    0.004586    0.001434    -0.009671   0.06105   
\end{verbatim}
\normalsize

%%%%%%%%%%%%%%%%%%%%%%%%%%%%%%%%%%%%%%%%%%%%%%%%%%%%%%%%%%%%%%%%%
\subsection{Moments}
\label{section:ia.moments}

Make $0^{th}$ and $1^{st}$ moment maps.  The example below takes an
existing image cube ang generates $0^{th}$ and $1^{st}$ moment maps.
No task yet exists to make moment masks so the examples below show you
how to do this with the CASA tools: 

\small
\begin{verbatim}
  ia.open('mosaic.image')           # Open the image you want and attach it to a tool called ia
  ia.moments(outfile='mosaic.mom0', # Write the moment 0 map, mosaic.mom0, to disk
     moments=0, axis=3,             # Take the zeroth moment over velocity (axis 3,0-based)
     mask='indexin(4,[4:25])',      # Select channels 4-25
                                    # Darn: mask axis input must be 1-based (4=velocity now).  
     includepix=[0.09,100.0])       # Only include pixels above 0.09 Jy
  ia.close()                        # Close the tool and detach from the MS
\end{verbatim}
\normalsize

\small
\begin{verbatim}
  ia.open('mosaic.image')           # Open the image you want and attach it to a tool called ia
  ia.moments(outfile='mosaic.mom1', # Write the moment 1 map, mosaic.mom1, to disk
     moments=1, axis=3,             # Take the first moment over velocity (axis 3 - 0-based)
     mask='indexin(4,[4:25])',      # Select channels 4-25, over velocity
     includepix=[0.09,10.0])        # Only include pixels above 0.09 Jy
  ia.close()                        # Close the tool and detach from the MS
\end{verbatim}
\normalsize

{\bf Note: 4mar07} there is a bug with the moments in that it keeps
the moment image locked so you can't look at it with the viewer.
Type the following to clear all locks and then bring up the viewer: 
\small
\begin{verbatim}
  tb.clearlocks()
  viewer
\end{verbatim}
\normalsize
Now the viewer will come up and you can look at the resulting image.  


%%%%%%%%%%%%%%%%%%%%%%%%%%%%%%%%%%%%%%%%%%%%%%%%%%%%%%%%%%%%%%%%%
\section{Image Math}
\label{section:ia.math}

Images can be scaled, manipulated and combined in various ways.  The
most straightforward method is to use the {\tt imagecalc} tool to do a
simple scaling: 

\small
\begin{verbatim}
  ia.open('n75.im')                 # open the image called n75.im
  ia.imagecalc(outfile='output.im', # divide the image by 2 and write new image 
     pixels='infile.im/2')          #  to the file called output.im
  ia.close()                        # close the tool. 
\end{verbatim}
\normalsize

Now, say you have made a mosic that has uniform RMS across the image
but the flux falls off as you move to the edge of the mosaic.  You
wrote out a {\tt fluxscale} image called fluxscale.im that you can
use to scale your mosaic to have the correct flux across the entire
field (albiet it has increasing RMS noise as you go to the edge of the
mosaic).  For example: 

\small
\begin{verbatim}
  ia.open('mosaic.im')                 # open the image called mosaic.im
  ia.imagecalc(outfile='mosaic.correctflux.im', # divide it by the fluxscale.im
      pixels='(mosaic.image)*(fluxscale.im)');
  ia.close()                        # close the tool. 
\end{verbatim}
\normalsize


%%%%%%%%%%%%%%%%%%%%%%%%%%%%%%%%%%%%%%%%%%%%%%%%%%%%%%%%%%%%%%%%%
%%%%%%%%%%%%%%%%%%%%%%%%%%%%%%%%%%%%%%%%%%%%%%%%%%%%%%%%%%%%%%%%%
