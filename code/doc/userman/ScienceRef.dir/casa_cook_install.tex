%%%%%%%%%%%%%%%%%%%%%%%%%%%%%%%%%%%%%%%%%%%%%%%%%%%%%%%%%%%%%%%%%
%%%%%%%%%%%%%%%%%%%%%%%%%%%%%%%%%%%%%%%%%%%%%%%%%%%%%%%%%%%%%%%%%
%%%%%%%%%%%%%%%%%%%%%%%%%%%%%%%%%%%%%%%%%%%%%%%%%%%%%%%%%%%%%%%%%

% STM 2007-04-13  split from previous version

\chapter[Appendix: Obtaining and Installing CASA]{Obtaining and Installing CASA}
\label{chapter:install}

\section{Installation Script}
\label{section:install.script}

Currently you must be able to log into your system as the root user or
an administrator user to install CASA. 

The easiest way to install CASA on a RedHat Enterprise Linux (or
compatible) system is to use our installation script,
load-casapy. This script will ftp the CASA RPMs and install them. To
use it, first use the link above to download it to your hard
disk. Next, make sure execute permission is set for the file. 

Install CASA into /usr by logging in as root and running:

load-casapy --root

This option will install CASA into /usr, but it can only be run by the root user.

Alternatively, you can visit our FTP server, download the rpms, and
install them by hand. Note: you must be root/administrater to install
CASA in this manner. 

See the following for more details:

https://wikio.nrao.edu/bin/view/Software/ObtainingCASA

\section{Startup}
\label{section:install.startup}

This section assumes that CASA has been installed on your LINUX or OSX
system. {\it For NRAO-AOC testers, you should do the following on an AOC
RHE4 machine:} 

\small
\begin{verbatim}
  > . /home/casa/casainit.sh
  or
  > source /home/casa/casainit.csh
\end{verbatim}
\normalsize

%%%%%%%%%%%%%%%%%%%%%%%%%%%%%%%%%%%%%%%%%%%%%%%%%%%%%%%%%%%%%%%%%
%%%%%%%%%%%%%%%%%%%%%%%%%%%%%%%%%%%%%%%%%%%%%%%%%%%%%%%%%%%%%%%%%
