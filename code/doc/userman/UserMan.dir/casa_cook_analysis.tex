%%%%%%%%%%%%%%%%%%%%%%%%%%%%%%%%%%%%%%%%%%%%%%%%%%%%%%%%%%%%%%%%%
%%%%%%%%%%%%%%%%%%%%%%%%%%%%%%%%%%%%%%%%%%%%%%%%%%%%%%%%%%%%%%%%%
%%%%%%%%%%%%%%%%%%%%%%%%%%%%%%%%%%%%%%%%%%%%%%%%%%%%%%%%%%%%%%%%%

% STM 2007-04-13  split from previous version
% STM 2007-08-01  add immoments and regridimage
% STM 2007-10-10  beta release (spell-checked)
% GvM 2008-03-12  add imcontsub
% STM 2008-03-17  immath, imstat, update imhead
% STM 2008-05-14  start patch 2.0, return vars, imfit
% STM 2008-06-03  more edits
% STM 2008-09-30  Patch 3 editing start, imval
% STM 2008-11-17  Patch 3 imregrid
% STM 2009-05-20  Patch 4 imsmooth, etc.
% STM 2009-12-16  Release 0
% JO  2010-10-12  Release 3.1.0
% JO  2011-04-24  Release 3.2.0 edits
% JO  2011-10-03  Release 3.3.0 edits 

\chapter{Image Analysis}
\label{chapter:analysis}

\begin{wrapfigure}{r}{2.5in}
  \begin{boxedminipage}{2.5in}
     \centerline{\bf Inside the Toolkit:}
     Image analysis is handled in the {\tt ia} tool.
     Many functions exist there, including region statistics and
     image math. See \S~\ref{section:analysis.toolkit} below
     for more information.
  \end{boxedminipage}
\end{wrapfigure}

Once data has been calibrated (and imaged in the case of synthesis
data), the resulting image or image cube must be displayed or 
analyzed in order to extract quantitative information, such as
statistics or moment images.  In addition, there need to be facilities for
the coordinate conversion of images for direct comparison.

The image analysis tasks are:
\begin{itemize}
   \item {\tt imhead} --- summarize and manipulate the ``header'' 
         information in a CASA image 
         (\S~\ref{section:analysis.imhead})
   \item {\tt imcontsub} --- perform continuum subtraction on a
         spectral-line image cube 
         (\S~\ref{section:analysis.imcontsub})
   \item {\tt imfit} --- image plane Gaussian component fitting
         (\S~\ref{section:analysis.imfit})
   \item {\tt immath} --- perform mathematical operations on or
         between images
         (\S~\ref{section:analysis.immath})
   \item {\tt immoments} --- compute the moments of an image cube
         (\S~\ref{section:analysis.moments})
   \item {\tt imstat} --- calculate statistics on an image or part
         of an image
         (\S~\ref{section:analysis.imstat})
   \item {\tt imval} --- extract the data and mask values from a
         pixel or region of an image
         (\S~\ref{section:analysis.imval})
  \item {\tt imtrans} --- reorder the axes of an image or cube
         (\S~\ref{section:analysis.imtrans})
  \item {\tt imcollapse} --- collapse image along one or more axes by aggregating pixel values along that axis 
         (\S~\ref{section:analysis.imcollapse})
   \item {\tt imregrid} --- regrid an image onto the coordinate
         system of another image 
         (\S~\ref{section:analysis.regrid})
   \item {\tt imsmooth} --- smooth images in the spectral and angular directions
         (\S~\ref{section:analysis.imsmooth})
   \item{\tt specfit} --- fit 1-dimensional Gaussians and/or
     polynomial models to an image or image region 
          (\S~\ref{section:analysis.specfit})
  \item{\tt slsearch} --- query a subset of the Splatalogue spectral
line catalog
          (\S~\ref{section:analysis.slsearch})
  \item{\tt splattotable} --- convert a file exported from Splatalogue
to a CASA table 
          (\S~\ref{section:analysis.splattotable})




   \item {\tt importfits} --- import a FITS image into a CASA  
         {\it image} format table 
         (\S~\ref{section:analysis.fits.import})
   \item {\tt exportfits} --- write out an image in FITS format
         (\S~\ref{section:analysis.fits.export})
\end{itemize}

There are other tasks which are useful during image analysis.  These
include:
\begin{itemize}
   \item {\tt viewer} --- there are useful region statistics and
         image cube slice and profile capabilities in the viewer 
         (\S~\ref{chapter:display})
\end{itemize}

We also give some examples of using the CASA Toolkit to aid in
image analysis (\S~\ref{section:analysis.toolkit}).

%%%%%%%%%%%%%%%%%%%%%%%%%%%%%%%%%%%%%%%%%%%%%%%%%%%%%%%%%%%%%%%%%
%%%%%%%%%%%%%%%%%%%%%%%%%%%%%%%%%%%%%%%%%%%%%%%%%%%%%%%%%%%%%%%%%
\section{Common Image Analysis Task Parameters}
\label{section:analysis.pars}

We now describe some sets of parameters are are common to the image
analysis.  These should behave the same way in any of the tasks
described in this section that they are found in.  

%%%%%%%%%%%%%%%%%%%%%%%%%%%%%%%%%%%%%%%%%%%%%%%%%%%%%%%%%%%%%%%%%
\subsection{Region Selection ({\tt box})}
\label{section:analysis.pars.box}

Direction (eg RA, Dec) areal selection in the image analysis tasks is
controlled by the {\tt box} parameter or through the {\tt region}
parameter (\S~\ref{section:analysis.pars.regions}). Note that one
should either specify a region (recommended) or any of
box/chans/stokes. Specifying both at the same time is not unique
anymore and can lead to unwanted selections. In the future we may
remove the box/chans/stokes selection (for CASA 3.3 we 
keep both selection methods for backward compatibility).

The {\tt box} parameter selects spatial rectangular areas (this parameter will
be removed for CASA 3.4 and higher):
\small
\begin{verbatim}
box        =      ''   #  Select one or more box regions

#          string containing blcx,blcy,trcx,trcy

#          A box selection in the directional portion of an image.
#          The directional portion of an image are the axes for right
#          ascension and declination, for example.  Boxes are specified
#          by there bottom-left corner (blc) and top-right corner (trc)
#          as follows: blcx, blcy, trcx, trcy;
#          ONLY pixel values acceptable at this time.
#          Default: none (all);
#          Example: box='0,0,50,50'


\end{verbatim}
\normalsize

To get help on {\tt box}, see the in-line help
\small
\begin{verbatim}
     help(par.box)
\end{verbatim}
\normalsize

%%%%%%%%%%%%%%%%%%%%%%%%%%%%%%%%%%%%%%%%%%%%%%%%%%%%%%%%%%%%%%%%%
\subsection{Plane Selection ({\tt chans}, {\tt stokes})}
\label{section:analysis.pars.planes}

The channel, frequency, or velocity plane(s) of the image is chosen
using the {\tt chans} parameter:
\small
\begin{verbatim}
chans      =      ''   #  Select the channel(spectral) range

#          string containing channel range

#          immath, imstat, and imcontsub - takes a string listing
#          of channel numbers, velocity, and/or frequency
#          numbers, much like the spw paramter
#          Only channel numbers acceptable at this time.
#          Default: none (all);  
#          Example: chans='3~20'    
#                   chans="0,3,4,8"
#                   chans="3~20,50,51"
\end{verbatim}
\normalsize

The polarization plane(s) of the image is chosen with the {\tt stokes}
parameter:
\small
\begin{verbatim}
stokes     =      ''   #  Stokes params to image (I,IV,IQU,IQUV)

#          string containing Stokes selections

#          Stokes parameters to image, may or may not be separated
#          by commas but best if you use commas.
#          Default: none (all); Example: stokes='IQUV';  
#          Example:stokes='I,Q'
#          Options: 'I','Q','U','V',
#                   'RR','RL','LR','LL',
#                   'XX','YX','XY','YY',...
\end{verbatim}
\normalsize

To get help on these parameters, see the in-line help
\small
\begin{verbatim}
     help(par.chans)
     help(par.stokes)
\end{verbatim}
\normalsize

Sometimes, as in the {\tt immoments} task, the channel/plane 
selection is generalized to work on more than one axis type.
In this case, the {\tt planes} parameter is used.  This behaves
like {\tt chans} in syntax.

%%%%%%%%%%%%%%%%%%%%%%%%%%%%%%%%%%%%%%%%%%%%%%%%%%%%%%%%%%%%%%%%%
\subsection{Lattice Expressions ({\tt expr})}
\label{section:analysis.pars.lattice}

Lattice expressions are strings that describe operations on a
set of input images to form an output image.  These strings
use the {\em Lattice Expression Language} (LEL).  LEL syntax
is described in detail in AIPS++ Note 223 
\begin{quote}
   \url{http://aips2.nrao.edu/docs/notes/223/223.html}
\end{quote}
{\bf ALERT:} This document was written in the context of
glish-based AIPS++ and is not yet updated to CASA syntax 
(see below).

The {\tt expr} string contains the LEL expression:
\small
\begin{verbatim}
expr       =      ''   #  Mathematical expression using images

#          string containing LEL expression

#          A mathematical expression, with image file names.
#          image file names must be enclosed in double quotes (")
#          Default: none 
#          Example: expr='min("image2.im")+(2*max("image1.im"))'
#
#    Available functions in the expr and mask paramters:
#    pi(), e(), sin(), sinh(), asinh(), cos(), cosh(), tan(), tanh(),
#    atan(), exp(), log(), log10(), pow(), sqrt(), complex(), conj()
#    real(), imag(), abs(), arg(), phase(), aplitude(), min(), max()
#    round(), isgn(), floor(), ceil(), rebin(), spectralindex(), pa(), 
#    iif(), indexin(), replace(), ...
\end{verbatim}
\normalsize

For examples using LEL {\tt expr}, see 
\S~\ref{section:analysis.immath.examples} below.  Note that in
{\tt immath}, shortcut names have been given to the images provided
by the user in {\tt imagename} that can be used in the LEL expression,
for the above example:
\small
\begin{verbatim}
  imagename=['image2.im','image1.im']
  expr='min(IM0)+(2*max(IM1))'
\end{verbatim}
\normalsize


{\bf ALERT:} LEL expressions use 0-based indices.
Also, the functions must be lowercase (in almost all cases we know
about).
    
%%%%%%%%%%%%%%%%%%%%%%%%%%%%%%%%%%%%%%%%%%%%%%%%%%%%%%%%%%%%%%%%%
\subsection{Masks ({\tt mask})}
\label{section:analysis.pars.mask}

The {\tt mask} string contains a LEL expression 
(see \S~\ref{section:analysis.pars.lattice} above).  This string
can be an on-the-fly (OTF) mask expression or refer to an 
image pixel mask.
\small
\begin{verbatim}
mask       =      ''   #  Mask to be applied to the images

#          string containing LEL expression

#          Name of mask applied to each image in the calculation
#          Default '' means no mask;  
#          Example: mask='"ngc5921.clean.cleanbox.mask">0.5'
#                   mask='mask(ngc5921.clean.cleanbox.mask)'
\end{verbatim}
\normalsize

Note that the mask file supplied in the {\tt mask} parameter must have
the same shape, same number of axes and same axes length, as the
images supplied in the {\tt expr} parameter, with one exception. The mask
may be missing some of the axes --- if this is the case then the mask
will be expanded along these axes to become the same shape.

For examples using {\tt mask}, see \S~\ref{section:analysis.immath.masks} 
below.


%%%%%%%%%%%%%%%%%%%%%%%%%%%%%%%%%%%%%%%%%%%%%%%%%%%%%%%%%%%%%%%%%
\subsection{Regions ({\tt region})}
\label{section:analysis.pars.regions}

The {\tt region} parameter points to a CASA region which can be
directly specified or listed in a ImageRegion file.  An
ImageRegion file can be created with the CASA {\tt viewer}'s region
manager (\S~\ref{section:display.image.rgnmgr}). Or directly using the
CASA region syntax (Chapter\,\ref{chapter:regionformat}; note: the
{\tt viewer} has not be fully converted for the new region format as
of CASA 3.3. -- old regions formats
  are still supported for that CASA version). 
Typically ImageRegion files will have the suffix {\tt '.rgn'}.  

{\bf Alert:} When both the region parameter and any of
box/chans/stokes are specified simultaneously, the task may perform
unwanted selections. Only specify one of these (sets of)
parameters. We recommend the use of CASA regions and may remove the
box/chans/stokes selection in CASA 3.4 releases and higher. 


For example: 
\small
\begin{verbatim}
    region=' circle[[18h12m24s, -23d11m00s], 2.3arcsec]'
\end{verbatim}
\normalsize

or

\small
\begin{verbatim}
    region='myimage.im.rgn'
\end{verbatim}
\normalsize

for to specify a region file. 


For the most part, the region parameter in tasks only
accepts strings (eg, file names, region shape descriptions) while the
region parameter in ia tool methods only accepts python region
dictionaries (eg produced using the rg tool).


%%%%%%%%%%%%%%%%%%%%%%%%%%%%%%%%%%%%%%%%%%%%%%%%%%%%%%%%%%%%%%%%%
%%%%%%%%%%%%%%%%%%%%%%%%%%%%%%%%%%%%%%%%%%%%%%%%%%%%%%%%%%%%%%%%%
\section{Image Header Manipulation ({\tt imhead})}
\label{section:analysis.imhead}

To summarize and change keywords and values in the ``header'' of
your image, use the {\tt imhead} task.  Its inputs are:
\small
\begin{verbatim}
#  imhead :: Lists, gets and puts image header parameters
imagename      =         ''   #   Name of input image file
mode           =  'summary'   #   Options: get, put, summary, list, stats
async          =      False      
\end{verbatim}
\normalsize

The {\tt mode} parameter controls the operation of {\tt imhead}.

Setting {\tt mode='summary'} will print out a summary of the image
properties and the header to the logger.

Setting {\tt mode='list'} prints out a list of the header keywords
and values to the terminal.

The {\tt mode='get'} allows the user to retrieve the current value 
for a specified keyword {\tt hdkey}:
\small
\begin{verbatim}
mode           =      'get'   #  imhead options: list, summary, get, put
   hdkey       =         ''   #  The FITS keyword
\end{verbatim}
\normalsize
Note that to catch this value, you need to assign it to a Python
variable.
See \S~\ref{section:intro.tasks.return} for more on return values.

The {\tt mode='put'} allows the user to replace the current value 
for a given keyword {\tt hditem} with that specified in {\tt hdvalue}.  
There are two sub-parameters that are opened by this option:
\small
\begin{verbatim}
mode           =      'put'   #  imhead options: list, summary, get, put
   hdkey       =         ''   #  The FITS keyword
   hdvalue     =         ''   #  Value of hdkey
   hdtype      =         ''   #  Data type of the header keyword.
   hdcomment   =         ''   #  Comment associated with the header keyword

\end{verbatim}
\normalsize
{\bf WARNING:} Be careful when using {\tt mode='put'}.  This task does
no checking on whether the values you specify (e.g. for the axes
types) are valid, and you can render your image invalid.  Make sure you
know what you are doing when using this option!

%%%%%%%%%%%%%%%%%%%%%%%%%%%%%%%%%%%%%%%%%%%%%%%%%%%%%%%%%%%%%%%%%
\subsection{Examples for {\tt imhead}}
\label{section:analysis.imhead.examples}

The following uses the example images from NGC5921
(\S~\ref{section:scripts.ngc5921}).

We can print the summary to the logger:
\small
\begin{verbatim}
CASA <51>: imhead('ngc5921.demo.cleanimg.image',mode='summary')
\end{verbatim}
\normalsize
prints in the logger: 
\small
\begin{verbatim}
##### Begin Task: imhead             #####
  Image name       : ngc5921.demo.cleanimg.image
  Object name      : N5921_2
  Image type       : PagedImage
  Image quantity   : Intensity
  Pixel mask(s)    : None
  Region(s)        : None
  Image units      : Jy/beam
  Restoring Beam   : 52.3782 arcsec, 45.7319 arcsec, -165.572 deg
  
  Direction reference : J2000
  Spectral  reference : LSRK
  Velocity  type      : RADIO
  Rest frequency      : 1.42041e+09 Hz
  Pointing center     :  15:22:00.000000  +05.04.00.000000
  Telescope           : VLA
  Observer            : TEST
  Date observation    : 1995/04/13/00:00:00
  Telescope position: [-1.60119e+06m, -5.04198e+06m, 3.55488e+06m] (ITRF)
  
  Axis Coord Type      Name             Proj Shape Tile   Coord value at pixel    Coord incr Units
  ------------------------------------------------------------------------------------------------ 
  0    0     Direction Right Ascension   SIN   256   64  15:22:00.000   128.00 -1.500000e+01 arcsec
  1    0     Direction Declination       SIN   256   64 +05.04.00.000   128.00  1.500000e+01 arcsec
  2    1     Stokes    Stokes                    1    1             I
  3    2     Spectral  Frequency                46    8   1.41279e+09     0.00 2.4414062e+04 Hz
                       Velocity                               1607.99     0.00 -5.152860e+00 km/s
##### End Task: imhead           
\end{verbatim}
\normalsize

If you choose {\tt mode='list'}, you get the summary in the logger and
a listing of keywords and values to the terminal:
\small
\begin{verbatim}
CASA <52>: imhead('ngc5921.demo.cleanimg.image',mode='list')
  Out[52]: 
{'beammajor': 52.378242492675781,
 'beamminor': 45.731891632080078,
 'beampa': -165.5721435546875,
 'bunit': 'Jy/beam',
 'cdelt1': '-7.27220521664e-05',
 'cdelt2': '7.27220521664e-05',
 'cdelt3': '1.0',
 'cdelt4': '24414.0625',
 'crpix1': 128.0,
 'crpix2': 128.0,
 'crpix3': 0.0,
 'crpix4': 0.0,
 'crval1': '4.02298392585',
 'crval2': '0.0884300154344',
 'crval3': 'I',
 'crval4': '1412787144.08',
 'ctype1': 'Right Ascension',
 'ctype2': 'Declination',
 'ctype3': 'Stokes',
 'ctype4': 'Frequency',
 'cunit1': 'rad',
 'cunit2': 'rad',
 'cunit3': '',
 'cunit4': 'Hz',
 'datamax': ' Not Known ',
 'datamin': -0.010392956435680389,
 'date-obs': '1995/04/13/00:00:00',
 'equinox': 'J2000',
 'imtype': 'Intensity',
 'masks': ' Not Known ',
 'maxpixpos': array([134, 134,   0,  38], dtype=int32),
 'maxpos': '15:21:53.976, +05.05.29.998, I, 1.41371e+09Hz',
 'minpixpos': array([117,   0,   0,  21], dtype=int32),
 'minpos': '15:22:11.035, +04.31.59.966, I, 1.4133e+09Hz',
 'object': 'N5921_2',
 'observer': 'TEST',
 'projection': 'SIN',
 'reffreqtype': 'LSRK',
 'restfreq': [1420405752.0],
 'telescope': 'VLA'}
\end{verbatim}
\normalsize
Note that this list is a return value and can be captured in a
variable:
\small
\begin{verbatim}
  mylist = imhead('ngc5921.demo.cleanimg.image',mode='list')
\end{verbatim}
\normalsize

The values for these keywords can be queried using {\tt mode='get'}.
At this point you should capture the return value:
\small
\begin{verbatim}
CASA <53>: mybmaj = imhead('ngc5921.demo.cleanimg.image',mode='get',hdkey='beammajor')

CASA <54>: mybmaj
  Out[54]: {'unit': 'arcsec', 'value': 52.378242492699997}

CASA <55>: myobserver = imhead('ngc5921.demo.cleanimg.image',mode='get',hdkey='observer')

CASA <56>: print myobserver
{'value': 'TEST', 'unit': ''}
\end{verbatim}
\normalsize

You can set the values for these keywords using
{\tt mode='put'}.  For example:
\small
\begin{verbatim}
CASA <57>: imhead('ngc5921.demo.cleanimg.image',mode='put',hdkey='observer',hdvalue='CASA')
  Out[57]: 'CASA'

CASA <58>: imhead('ngc5921.demo.cleanimg.image',mode='get',hdkey='observer')
  Out[58]: {'unit': '', 'value': 'CASA'}
\end{verbatim}
\normalsize

%%%%%%%%%%%%%%%%%%%%%%%%%%%%%%%%%%%%%%%%%%%%%%%%%%%%%%%%%%%%%%%%%
%%%%%%%%%%%%%%%%%%%%%%%%%%%%%%%%%%%%%%%%%%%%%%%%%%%%%%%%%%%%%%%%%
\section{Continuum Subtraction on an Image Cube ({\tt imcontsub})}
\label{section:analysis.imcontsub}

One method to separate line and continuum emission in an image cube is
to specify a number of line-free channels in that cube, make a linear
fit to the visibilities in those channels, and subtract the fit from
the whole cube.  Note that the task {\tt uvcontsub} serves a similar
purpose; see \S~\ref{section:cal.other.uvcontsub} for a synopsis of
the pros and cons of either method.

The {\tt imcontsub} task will subtract a polynomial baseline fit to the
specified channels from an image cube.

The default inputs are: 
\small
\begin{verbatim}
#  imcontsub :: Continuum subtraction on images
imagename  =      ''   #  Name of the input image
linefile   =      ''   #  Output line image file name
contfile   =      ''   #  Output continuum image file name
fitorder   =       0   #  Polynomial order for the continuum estimation
region     =      ''   #  Image region or name to process see viewer
box        =      ''   #  Select one or more box regions
chans      =      ''   #  Select the channel(spectral) range
stokes     =      ''   #  Stokes params to image (I,IV,IQU,IQUV)
async      =   False   
\end{verbatim}
\normalsize

Area selection using {\tt box} and {\tt region} is detailed in 
\S~\ref{section:analysis.pars.box} and
\S~\ref{section:analysis.pars.regions} respectively.

Image cube plane selection using {\tt chans} and {\tt stokes}
are described in \S~\ref{section:analysis.pars.planes}.

{\bf ALERT:} {\tt imcontsub} has issues when the image does
not contain a spectral or stokes axis. Errors are generated when
run on an image missing one or both of these axes.  You will need
to use the Toolkit (e.g. the {\tt ia.adddegaxes} method) to add
degenerate missing axes to the image.

%%%%%%%%%%%%%%%%%%%%%%%%%%%%%%%%%%%%%%%%%%%%%%%%%%%%%%%%%%%%%%%%%
\subsection{Examples for {\tt imcontsub})}
\label{section:analysis.imcontsub.examples}

The following uses the example images from NGC5921
(\S~\ref{section:scripts.ngc5921}).

First, we make a clean image without the uv-plane continuum
subtraction:
\small
\begin{verbatim}
  # First, run clearcal to clear the uvcontsub results from the
  # corrected column
  clearcal('ngc5921.demo.src.split.ms')
  
  # Now clean, keeping all the channels except first and last
  default('clean')
  vis = 'ngc5921.demo.src.split.ms'
  imagename = 'ngc5921.demo.nouvcontsub'
  mode = 'channel'
  nchan = 61
  start = 1
  width = 1
  imsize = [256,256]
  psfmode = 'clark'
  imagermode = ''
  cell = [15.,15.]
  niter = 6000
  threshold='8.0mJy'
  weighting = 'briggs'
  robust = 0.5
  mask = [108,108,148,148]
  interactive=False
  clean()
  
  # It will have made the image:
  # -----------------------------
  # ngc5921.demo.nouvcontsub.image

  # You can view this image
  viewer('ngc5921.demo.nouvcontsub.image')
\end{verbatim}
\normalsize
You can clearly see continuum sources in the image which were removed
previously in the script by the use of {\tt uvcontsub}.  Lets see
if {\tt imcontsub} can work as well.

Using the viewer, it looks like channels 0 through 4 and
50 through 60 are line-free.  Then:
\small
\begin{verbatim}
  default('imcontsub')
  imagename = 'ngc5921.demo.nouvcontsub.image'
  linefile  = 'ngc5921.demo.nouvcontsub.lineimage'
  contfile  = 'ngc5921.demo.nouvcontsub.contimage'
  fitorder  = 1
  chans      = '0~4,50~60'
  stokes    = 'I'
  imcontsub()
\end{verbatim}
\normalsize
This did not do too badly!

%%%%%%%%%%%%%%%%%%%%%%%%%%%%%%%%%%%%%%%%%%%%%%%%%%%%%%%%%%%%%%%%%
%%%%%%%%%%%%%%%%%%%%%%%%%%%%%%%%%%%%%%%%%%%%%%%%%%%%%%%%%%%%%%%%%
\section{Image-plane Component Fitting ({\tt imfit})}
\label{section:analysis.imfit}

The inputs are:
\small
\begin{verbatim}
#  imfit :: Fit one or more elliptical Gaussian components on an image region(s)
imagename           =         ''        #  Name of the input image
box                 =         ''        #  Specify one or more box regions for the fit.
region              =         ''        #  Region name or region specified using rg tool.
chans               =         ''        #  Spectral channels on which to perform fit.
stokes              =        'I'        #  Stokes parameter to fit. If blank, first stokes plane is used.
mask                =  'junk.im'        #  Mask to be applied to the image
includepix          =         []        #  Range of pixel values to include for fitting.
excludepix          =         []        #  Range of pixel values to exclude for fitting.
residual            =         ''        #  Name of output residual image.
model               = 'evalexpr'        #  Name of output model image.
estimates           =         ''        #  Name of file containing initial estimates of component parameters.
logfile             =         ''        #  Name of file to write fit results.
newestimates        =         ''        #  File to write fit results which can be used as initial estimates for next run.
complist            =         ''        #  Name of output component list table.
chan                =         -1        #  DEPRECATED. USE chans INSTEAD.
async               =      False        #  If true the taskname must be started using imfit(...)

\end{verbatim}
\normalsize
This task will return (as a Python dictionary) the results of the fit,
but the results can also be written into a component list table or a
logfile. 
 
Note that to fit more than a single component, you {\em must} provide
starting estimates for each component via the {\tt estimates} file.
See {\tt ``help imfit''} for more details on this.

%%%%%%%%%%%%%%%%%%%%%%%%%%%%%%%%%%%%%%%%%%%%%%%%%%%%%%%%%%%%%%%%%
\subsection{Examples for {\tt imfit}}
\label{section:analysis.imfit.examples}

The following are some examples using the B1608+656 Tutorial
\begin{quote}
  \url{http://casa.nrao.edu/Doc/Scripts/b1608_demo.py}
\end{quote}
as an example.

\small
\begin{verbatim}
# First fit only a single component at a time
# This is OK since the components are well-separated and not blended
# Box around component A
xfit_A_res = imfit('b1608.demo.clean2.image',box='121,121,136,136',
                   newestimates='b1608.demo.clean2.newestimate')

# Now extract the fit part of the return value
xfit_A = xfit_A_res['results']['component0']
#xfit_A
#  Out[7]: 
#{'flux': {'error': array([  6.73398035e-05,   0.00000000e+00,   0.00000000e+00,
#         0.00000000e+00]),
#          'polarisation': 'Stokes',
#          'unit': 'Jy',
#          'value': array([ 0.01753742,  0.        ,  0.        ,  0.        ])},
# 'label': '',
# 'shape': {'direction': {'error': {'latitude': {'unit': 'arcsec',
#                                                'value': 0.00041154866279462775},
#                                   'longitude': {'unit': 'arcsec',
#                                                 'value': 0.00046695916589535109}},
#                         'm0': {'unit': 'rad', 'value': -2.0541102061078207},
#                         'm1': {'unit': 'rad', 'value': 1.1439131060384089},
#                         'refer': 'J2000',
#                         'type': 'direction'},
#           'majoraxis': {'unit': 'arcsec', 'value': 0.29100166137741568},
#           'majoraxiserror': {'unit': 'arcsec',
#                              'value': 0.0011186420613222663},
#           'minoraxis': {'unit': 'arcsec', 'value': 0.24738110059830495},
#           'minoraxiserror': {'unit': 'arcsec',
#                              'value': 0.0013431999725066338},
#           'positionangle': {'unit': 'deg', 'value': 19.369249322401796},
#           'positionangleerror': {'unit': 'rad',
#                                  'value': 0.016663189295782171},
#           'type': 'Gaussian'},
# 'spectrum': {'frequency': {'m0': {'unit': 'GHz', 'value': 1.0},
#                            'refer': 'LSRK',
#                            'type': 'frequency'},
#              'type': 'Constant'}}

# Now the other components
xfit_B_res = imfit('b1608.demo.clean2.image',box='108,114,120,126',
                   newestimates='b1608.demo.clean2.newestimate',append=True)
xfit_B = xfit_B_res['results']['component0']

xfit_C_res= imfit('b1608.demo.clean2.image',box='108,84,120,96')
xfit_C = xfit_C_res['results']['component0']

xfit_D_res = imfit('b1608.demo.clean2.image',box='144,98,157,110')
xfit_D = xfit_D_res['results']['component0']

print ""
print "Imfit Results:"
print "--------------"
print "A  Flux = %6.4f Bmaj = %6.4f" % (xfit_A['flux']['value'][0],xfit_A['shape']['majoraxis']['value'])
print "B  Flux = %6.4f Bmaj = %6.4f" % (xfit_B['flux']['value'][0],xfit_B['shape']['majoraxis']['value'])
print "C  Flux = %6.4f Bmaj = %6.4f" % (xfit_C['flux']['value'][0],xfit_C['shape']['majoraxis']['value'])
print "D  Flux = %6.4f Bmaj = %6.4f" % (xfit_D['flux']['value'][0],xfit_D['shape']['majoraxis']['value'])
print ""
\end{verbatim}
\normalsize

%% FIXME dmehring, ugh the above accessing of nested dictionaries is
%pretty horrible. I'd stuff the relevant dictionary into a component
%list (cl) tool and fish out parameters using methods there. Much more
%user-friendly IMO

Now try fitting four components together.  For this we will have to
provide an estimate file.  We will use the clean beam for the estimate
of the component sizes:
\small
\begin{verbatim}
estfile=open('b1608.demo.clean2.estimate','w')
print >>estfile,'# peak, x, y, bmaj, bmin, bpa'
print >>estfile,'0.017, 128, 129, 0.293arcsec, 0.238arcsec, 21.7deg'
print >>estfile,'0.008, 113, 120, 0.293arcsec, 0.238arcsec, 21.7deg'
print >>estfile,'0.008, 113,  90, 0.293arcsec, 0.238arcsec, 21.7deg'
print >>estfile,'0.002, 151, 104, 0.293arcsec, 0.238arcsec, 21.7deg'
estfile.close()
\end{verbatim}
\normalsize
Then, this can be used in {\tt imfit}:
\small
\begin{verbatim}
xfit_all_res = imfit('b1608.demo.clean2.image', 
                     estimates='b1608.demo.clean2.estimate',
                     logfile='b1608.demo.clean2.imfitall.log',
                     newestimates='b1608.demo.clean2.newestimate',
                     box='121,121,136,136,108,114,120,126,108,84,120,96,144,98,157,110')
# Now extract the fit part of the return values
xfit_allA = xfit_all_res['results']['component0']
xfit_allB = xfit_all_res['results']['component1']
xfit_allC = xfit_all_res['results']['component2']
xfit_allD = xfit_all_res['results']['component3']
\end{verbatim}
\normalsize
These results are almost identical to those from the individual fits.
You can see a nicer printout of the fit results in the logfile.

%%%%%%%%%%%%%%%%%%%%%%%%%%%%%%%%%%%%%%%%%%%%%%%%%%%%%%%%%%%%%%%%%
%%%%%%%%%%%%%%%%%%%%%%%%%%%%%%%%%%%%%%%%%%%%%%%%%%%%%%%%%%%%%%%%%
\section{Mathematical Operations on an Image ({\tt immath})}
\label{section:analysis.immath}

The inputs are:
\small
\begin{verbatim}
#  immath :: Perform math operations on images
imagename           =         ''        #  a list of input images
mode                = 'evalexpr'        #  mode for math operation (evalexpr, spix, pola, poli)
     expr           =         ''        #  Mathematical expression using images
     varnames       =         ''        #  a list of variable names to use with the image files

outfile             = 'immath_results.im' #  File where the output is saved
mask                =         ''        #  Mask to be applied to the images
region              =         ''        #  File path which contains an Image Region
box                 =         ''        #  Select one or more box regions in the input images
chans               =         ''        #  Select the channel(spectral) range
stokes              =        'I'        #  Stokes params to image (I,IV,IQU,IQUV)
async               =      False        #  If true the taskname must be started using immath(...)
\end{verbatim}
\normalsize

In all cases, {\tt outfile} must be supplied with the name of the
new output file to create.

The {\tt mode} parameter selects what {\tt immath} is to do.

The default {\tt mode='evalexpr'} lets the user specify a mathematical
operation to carry out on one or more input images.
The sub-parameter {\tt expr} contains the Lattice Expression Language
(LEL) string describing the image operations based on the images
in the {\tt imagename} parameter.
See \S~\ref{section:analysis.pars.lattice} for more on LEL strings
and the {\tt expr} parameter.

Mask specification is done using the {\tt mask} parameter.  This can
optionally contain an on-the-fly mask expression (in LEL) or point to
an image with a pixel mask.  See \S~\ref{section:analysis.pars.mask}
for more on the use of the {\tt mask} parameter.  See also
\S~\ref{section:analysis.pars.lattice} for more on LEL
strings. Sometimes, one would like to use a flat image (e.g. a moment
image) mask to be applied to an entire cube. The {\tt stretch=True}
subparameter in {\tt mask} allows one to expand the mask to all planes
of the cube.

Region selection is carried out through the {\tt region} and {\tt box}
parameters.
See \S~\ref{section:analysis.pars.box} and
\S~\ref{section:analysis.pars.regions} for more on area
selection.

Image plane selection is controlled by {\tt chans} and {\tt stokes}.
See \S~\ref{section:analysis.pars.planes} for details on plane
selection.

For {\tt mode='evalexpr'}, the standard usage for specifying images to
be used in the LEL expression is to provide them as a list in the {\tt imagename}
parameter, and then access there in the LEL expression by the
names {\tt IM0, IM1, ...}.  For example,
\small
\begin{verbatim}
immath(imagename=['image1.im','image2.im'],expr='IM0-IM1',outfile='ImageDiff.im')
\end{verbatim}
\normalsize
would subtract the second image given from the first.

For the special modes {\tt 'spix'}, {\tt 'pola'}, {\tt 'poli'}, the
required images for the given operation are to be provided in 
{\tt imagename} (some times in a particular order). 
{\bf V3.0 ALERT:} For {\tt mode='pola'} you MUST call as a function as
in the example below (\S~\ref{section:analysis.immath.examples.pol}), 
giving the parameters as arguments, or {\tt immath} will fail.

Detailed examples are given below.

%%%%%%%%%%%%%%%%%%%%%%%%%%%%%%%%%%%%%%%%%%%%%%%%%%%%%%%%%%%%%%%%%
\subsection{Examples for {\tt immath}}
\label{section:analysis.immath.examples}

The following are examples using {\tt immath} using NGC5921 
(\S~\ref{section:scripts.ngc5921}).  Note that the image
names in the {\tt expr} are assumed to refer to existing image files
in the current working directory.

%%%%%%%
\subsubsection{Simple math}
\label{section:analysis.immath.examples.math}

Select a single plane (channel 22) of the 3-D cube and  
subtract it from the original image: 
\small
\begin{verbatim}
  immath(imagename='ngc5921.demo.cleanimg.image',
         expr='IM0',chans='22',
         outfile='ngc5921.demo.chan22.image')
\end{verbatim}
\normalsize

Double all values in our image:
\small
\begin{verbatim}
  immath(imagename=['ngc5921.demo.chan22.image'],
         expr='IM0*2.0',
         outfile='ngc5921.demo.chan22double.image' )
\end{verbatim}
\normalsize
    
Square all values in our image:
\small
\begin{verbatim}
  immath(imagename=['ngc5921.demo.chan22.image'],
         expr='IM0^2',
         outfile='ngc5921.demo.chan22squared.image' )
\end{verbatim}
\normalsize
Note that the units in the output image are still claimed to be
``Jy/beam'', ie.\ {\tt immath} will not correctly scale the units
in the image for non-linear cases like this.  Beware.

You can do other mathematical operations on an image (e.g.\
trigonometric functions) as well as use scalars results from an image
(e.g.\ max, min, median, mean, variance).  You also have access to
constants such as {\tt e()} and {\tt pi()} (which are doubles
internally, while most images are floats). For example:
Take the sine of an image:
\small
\begin{verbatim}
  immath(imagename=['ngc5921.demo.chan22.image','ngc5921.demo.chan22squared.image'],
         expr='sin(float(pi())*IM0/sqrt(max(IM1)))',
         outfile='ngc5921.demo.chan22sine.image')
\end{verbatim}
\normalsize
Note again that the units are again kept as they were.
    
Select a single plane (channel 22) of the 3-D cube and  
subtract it from the original image: 
\small
\begin{verbatim}
  immath(imagename='ngc5921.demo.cleanimg.image',
         expr='IM0',chans='22',
         outfile='ngc5921.demo.chan22.image')

  immath(imagename=['ngc5921.demo.cleanimg.image','ngc5921.demo.chan22.image'],
         expr='IM0-IM1',
         outfile='ngc5921.demo.sub22.image')
\end{verbatim}
\normalsize
Note that in this example the 2-D plane gets extended in the third
dimension and the 2-D values are applied to each plane in the 3-D cube. 

%{\bf ALERT:} In the future this can be done without a temporary file, 
%using the {\tt INDEXIN()} function in the expression.
    
Select and save the inner 1/4 of an image for channels {\tt 40,42,44}
as well as channels 10 and below:
\small
\begin{verbatim}
   default('immath')
   imagename=['ngc5921.demo.cleanimg.image']
   expr='IM0'
   region='box[[64pix,64pix],[192pix,192pix]]'
   chans='<10;40,42,44'
   outfile='ngc5921.demo.inner.image'
   immath()
\end{verbatim}
\normalsize
{\bf ALERT:} Note that if chan selects more than one channel then
the output image has a number of channels given by the span from the
lowest and highest channel selected in {\tt chan}.  In the example 
above, it will have 45 channels.  The ones not selected will be masked
in the output cube.  If we had set
\small
\begin{verbatim}
   chans = '40,42,44'
\end{verbatim}
\normalsize
then there would be 5 output channels corresponding to channels
{\tt 40,41,42,43,44} of the MS with {\tt 41,43} masked.  Also, 
the {\tt chans='<10'} selects channels 0--9.

Note that the {\tt chans} syntax allows the operators {\tt '<'},
{\tt '<='}, {\tt '>'}, {\tt '>'}.  For example,
\small
\begin{verbatim}
   chans = '<17,>79'
   chans = '<=16,>=80'
\end{verbatim}
\normalsize
do the same thing.

Divide an image by another, with a threshold on one of the images:
\small
\begin{verbatim}
  immath(imagename=['ngc5921.demo.cleanimg.image','ngc5921.demo.chan22.image'],
         expr='IM0/IM1[IM1>0.008]',
         outfile='ngc5921.demo.div22.image')
\end{verbatim}
\normalsize

%%%%%%%
\subsubsection{Polarization manipulation}
\label{section:analysis.immath.examples.pol}

The following are some examples using the 3C129 Tutorial
\begin{quote}
  \url{http://casa.nrao.edu/Doc/Scripts/3c129_tutorial.py}
\end{quote}
as an example.

It is helpful to extract the Stokes planes from the cube
into individual images:
\small
\begin{verbatim}
   default('immath')
   imagename = '3C129BC.clean.image'
   outfile='3C129BC.I'; expr='IM0'; stokes='I'; immath();
   outfile='3C129BC.Q'; expr='IM0'; stokes='Q'; immath();
   outfile='3C129BC.U'; expr='IM0'; stokes='U'; immath();
   outfile='3C129BC.V'; expr='IM0'; stokes='V'; immath();
\end{verbatim}
\normalsize

Extract linearly polarized intensity and polarization position angle images:
\small
\begin{verbatim}
  immath(stokes='', outfile='3C129BC.P', mode='poli', 
         imagename=['3C129BC.Q','3C129BC.U'], sigma='0.0mJy/beam'); 
  immath(stokes='', outfile='3C129BC.X', mode='pola', 
         imagename=['3C129BC.Q','3C129BC.U'], sigma='0.0mJy/beam'); 
\end{verbatim}
\normalsize
{\bf V3.0 ALERT:} For {\tt mode='pola'} you MUST call as a function as
in this example (giving the parameters as arguments) or {\tt immath}
will fail.

Create a fractional linear polarization image:
\small
\begin{verbatim}
   default( 'immath')
   imagename = ['3C129BC.I','3C129BC.Q','3C129BC.U']
   outfile='3C129BC.fractional_linpol'
   expr='sqrt((IM1^2 + IM2^2)/IM0^2)'
   stokes=''
   immath()
\end{verbatim}
\normalsize

Create a polarized intensity image:
\small
\begin{verbatim}
   default( 'immath')
   imagename = ['3C129BC.Q','3C129BC.U','3C129BC.V']
   outfile='3C129BC.pol_intensity'
   expr='sqrt(IM0^2 + IM1^2 + IM2^2)'
   stokes=''
   immath()
\end{verbatim}
\normalsize

{\bf Toolkit Tricks:} The following uses the toolkit 
(\S~\ref{section:analysis.toolkit}).
You can make a complex linear polarization ($Q+iU$) image using the
{\tt imagepol} tool:
\small
\begin{verbatim}
  # See CASA User Reference Manual:
  # http://casa.nrao.edu/docs/casaref/imagepol-Tool.html
  #
  # Make an imagepol tool and open the clean image 
  potool = casac.homefinder.find_home_by_name('imagepolHome')
  po = potool.create()
  po.open('3C129BC.clean.image')
  # Use complexlinpol to make a Q+iU image
  po.complexlinpol('3C129BC.cmplxlinpol')
  po.close()
\end{verbatim}
\normalsize
You can now display this in the viewer, in particular overlay this
over the intensity raster with the intensity contours.  
When you load the image, use the LEL:
\small
\begin{verbatim}
  '3C129BC.cmplxlinpol'['3C129BC.P'>0.0001]
\end{verbatim}
\normalsize
which is entered into the LEL box at the bottom of the Load Data menu
(\S~\ref{section:display.viewerGUI.load}).

%%%%%%%%
%%% FIXEE dmehring says this section is obsolete. impbcor should now
%%% be used.
%\subsubsection{Primary beam correction/uncorrection}
%\label{section:analysis.immath.examples.pbcor}
%
%In a script using {\tt mode='evalexpr'}, you might want to assemble
%the string for {\tt expr} using string variables that contain
%the names of files.  Since you need to include quotes inside the
%{\tt expr} string, use a different quote outside (or escape the
%string, e.g. {\tt '\''}.
%For example, to do a primary beam correction on the NGC5921 cube,
%\small
%\begin{verbatim}
%   imname = 'ngc5921.demo.cleanimg'
%   imagename = imname
%   ...
%   clean()
%
%   default('immath')
%   clnimage = imname + '.image'
%   fluximage = imname + '.flux'
%   pbcorimage = imname + '.pbcor'
%
%   outfile = pbcorimage
%   imagename = [clnimage,fluximage]
%   # cutoff at the 10% level
%   expr='IM0/IM1[IM1>0.1]'
%
%   immath()
%\end{verbatim}
%\normalsize
%Note that we did not use a {\tt minpb} when we cleaned, so we use the
%trick above to effectively set a cutoff in the primary beam 
%{\tt .flux} image of 0.1.
%
%For more on LEL strings, see AIPS++ Note 223 
%\begin{quote}
%   \url{http://aips2.nrao.edu/docs/notes/223/223.html}
%\end{quote}
%or in \S~\ref{section:analysis.pars.lattice} above.

%%%%%%%%
%\subsubsection{Spectral analysis}
%\label{section:analysis.immath.examples.spec}
%One can make an integrated 1-d spectrum over the whole image by
%collapsing the cube with {\tt imcollapse}
%(Sect.\,\ref{section:analysis.imcollapse}).
%For example, using the NGC5921 image cube (with 46 channels):
%\small
%\begin{verbatim}
%imcollapse(imagename='ngc5921.demo.clean.image',function='sum',
%           outfile='   
%
%imagename='ngc5921.demo.spectrum.all',function='mode="evalexpr",
%       imagename='ngc5921.demo.clean.image',
%       expr="rebin(IM0,[256,256,1,1])")
%\end{verbatim}
%\normalsize
%
%
%One can make an integrated 1-d spectrum over the whole image
%by rebinning (integrating) over the two coordinate axes in
%a specified region.
%For example, using the NGC5921 image cube (with 46 channels):
%\small
%\begin{verbatim}
%immath(outfile="ngc5921.demo.spectrum.all",mode="evalexpr",
%       imagename='ngc5921.demo.clean.image',
%       expr="rebin(IM0,[256,256,1,1])")
%\end{verbatim}
%\normalsize
%The resulting image has shape {\tt [1,1,1,46]} as desired.
%You can view this with the {\tt viewer} and will see a 1-D spectrum.
%
%One can also do this with a box:
%\small
%\begin{verbatim}
%immath(outfile="ngc5921.demo.spectrum.box",mode="evalexpr",
%       imagename='ngc5921.demo.clean.image',
%       expr="rebin(IM0,[256,256,1,1])",box="118,118,141,141")
%\end{verbatim}
%\normalsize
%{\bf ALERT:} One cannot specify a {\tt region} without it collapsing the channel
%axis (even when told to use all axes or channels).
%
%{\bf Toolkit Tricks:} The following uses the toolkit (\S~\ref{section:analysis.toolkit}).
%You can make an ascii file containing only the values (no other info
%though):
%\small
%\begin{verbatim}
%ia.open('ngc5921.demo.spectrum.all')
%ia.toASCII('ngc5921.demo.spectrum.all.ascii')
%\end{verbatim}
%\normalsize
%You can also extract to a record inside Python:
%\small
%\begin{verbatim}
%myspec = ia.torecord()
%\end{verbatim}
%\normalsize
%which you can then manipulate in Python.

%%%%%%%%%%%%%%%%%%%%%%%%%%%%%%%%%%%%%%%%%%%%%%%%%%%%%%%%%%%%%%%%%
\subsection{Using masks in {\tt immath}}
\label{section:analysis.immath.masks}

The {\tt mask} parameter is used inside {\tt immath} to apply a
mask to all the images used in {\tt expr} before calculations
are done (if you are curious, it uses the {\tt ia.subimage} tool
method to make virtual images that are then input in the LEL to the 
{\tt ia.imagecalc} method).

For example, lets assume that we have made a single channel image
using {\tt clean} for the NGC5921 data (see
Appendix~\ref{section:scripts.ngc5921}).
\small
\begin{verbatim}
  default('clean')
  
  vis = 'ngc5921.demo.src.split.ms.contsub'
  imagename = 'ngc5921.demo.chan22.cleanimg'
  mode = 'channel'
  nchan = 1
  start = 22
  step = 1
  
  field = ''
  spw = ''
  imsize = [256,256]
  cell = [15.,15.]
  psfalg = 'clark'
  gain = 0.1
  niter = 6000
  threshold='8.0mJy'
  weighting = 'briggs'
  rmode = 'norm'
  robust = 0.5
  
  mask = [108,108,148,148]
  
  clean()
\end{verbatim}
\normalsize
There is now a file {\tt 'ngc5921.demo.chan22.cleanimg.mask'} that is
an image with values 1.0 inside the {\tt cleanbox} region and 0.0
outside.  

We can use this to mask the clean image:
\small
\begin{verbatim}
  default('immath')
  imagename = 'ngc5921.demo.chan22.cleanimg.image'
  expr='IM0'
  mask='"ngc5921.demo.chan22.cleanimg.mask">0.5'
  outfile='ngc5921.demo.chan22.cleanimg.imasked'
  immath()
\end{verbatim}
\normalsize

{\bf Toolbox Tricks:}
Note that there are also {\it pixel masks} that can be contained in each
image.  These are Boolean masks, and are implicitly used in the
calculation for each image in {\tt expr}.  If you want to use the
mask in a different image not in {\tt expr}, try it in {\tt mask}:
\small
\begin{verbatim}
  # First make a pixel mask inside ngc5921.demo.chan22.cleanimg.mask
  ia.open('ngc5921.demo.chan22.cleanimg.mask')
  ia.calcmask('"ngc5921.demo.chan22.cleanimg.mask">0.5')
  ia.summary()
  ia.close()
  # There is now a 'mask0' mask in this image as reported by the summary

  # Now apply this pixel mask in immath
  default('immath')
  imagename='ngc5921.demo.chan22.cleanimg.image'
  expr='IM0'
  mask='mask(ngc5921.demo.chan22.cleanimg.mask)'
  outfile='ngc5921.demo.chan22.cleanimg.imasked1'
  immath()
\end{verbatim}
\normalsize

Note that nominally the axes of the mask must be congruent to the axes
of the images in {\tt expr}.  However, one exception is that the image
in {\tt mask} can have {\em fewer} axes (but not axes that exist but
are of the wrong lengths).  In this case {\tt immath} will extend the
missing axes to cover the range in the images in {\tt expr}.
Thus, you can apply a mask made from a single channel to a whole cube.
\small
\begin{verbatim}
  # drop degenerate stokes and freq axes from mask image
  ia.open('ngc5921.demo.chan22.cleanimg.mask')
  im2 = ia.subimage(outfile='ngc5921.demo.chan22.cleanimg.mymask',dropdeg=True)
  im2.summary()
  im2.close()
  ia.close()
  # mymask has only RA and Dec axes

  # Now apply this mask to the whole cube
  default('immath')
  imagename='ngc5921.demo.cleanimg.image'
  expr='IM0'
  mask='"ngc5921.demo.chan22.cleanimg.mymask">0.5'
  outfile='ngc5921.demo.cleanimg.imasked'
  immath()
\end{verbatim}
\normalsize

For more on masks as used in LEL, see
\begin{quote}
   \url{http://aips2.nrao.edu/docs/notes/223/223.html}
\end{quote}
or in \S~\ref{section:analysis.pars.mask} above.

%%%%%%%%%%%%%%%%%%%%%%%%%%%%%%%%%%%%%%%%%%%%%%%%%%%%%%%%%%%%%%%%%




%%%%%%%%%%%%%%%%%%%%%%%%%%%%%%%%%%%%%%%%%%%%%%%%%%%%%%%%%%%%%%%%%
\section{Computing the Moments of an Image Cube ({\tt immoments})}
\label{section:analysis.moments}

For spectral line datasets, the output of the imaging process is an
{\tt image cube}, with a frequency or velocity channel axis in
addition to the two sky coordinate axes.  This can be most easily
thought of as a series of image {\tt planes} stacked along the
spectral dimension.

A useful product to compute is to collapse the cube into a 
{\it moment} image by taking a linear combination of the individual
planes:
\begin{equation}
   M_m(x_i,y_i) = \sum_k^N w_m(x_i,y_i,v_k)\,I(x_i,y_i,v_k)
\end{equation}
for pixel $i$ and channel $k$ in the cube $I$.  There are a number
of choices to form the $m$ moment, usually approximating some
polynomial expansion of the intensity distribution over velocity
mean or sum, gradient, dispersion, skew, kurtosis, etc.).  There
are other possibilities (other than a weighted sum) for calculating
the image, such as median filtering, finding minima or maxima along
the spectral axis, or absolute mean deviations.  And the axis along
which to do these calculation need not be the spectral axis (ie.
do moments along Dec for a RA-Velocity image).  We will treat all
of these as generalized instances of a ``moment'' map.

The {\tt immoments} task will compute basic moment images from a cube.
The default inputs are:
\small
\begin{verbatim}
#  immoments :: Compute moments of an image cube:
imagename    =         ''   #   Input image name
moments      =        [0]   #  List of moments you would like to compute
axis         = 'spectral'   #  The momement axis: ra, dec, lat, long, spectral, or stokes
region       =         ''   #  Image Region.  Use viewer
box          =         ''   #  Select one or more box regions
chans        =         ''   #  Select the channel(spectral) range
stokes       =         ''   #  Stokes params to image (I,IV,IQU,IQUV)
mask         =         ''   #  mask used for selecting the area of the image to calculate the moments on
includepix   =         -1   #  Range of pixel values to include
excludepix   =         -1   #  Range of pixel values to exclude
outfile      =         ''   #  Output image file name (or root for multiple moments)
async        =      False   #  If true the taskname must be started using immoments(...)
\end{verbatim}
\normalsize

This task will operate on the input file given by {\tt imagename} and
produce a new image or set of images based on the name given in
{\tt outfile}.

The {\tt moments} parameter chooses which moments are calculated.
The choices for the operation mode are:
\small
\begin{verbatim}
    moments=-1  - mean value of the spectrum
    moments=0   - integrated value of the spectrum
    moments=1   - intensity weighted coordinate;traditionally used to get 
                  'velocity fields'
    moments=2   - intensity weighted dispersion of the coordinate; traditionally
                  used to get 'velocity dispersion'
    moments=3   - median of I
    moments=4   - median coordinate
    moments=5   - standard deviation about the mean of the spectrum
    moments=6   - root mean square of the spectrum
    moments=7   - absolute mean deviation of the spectrum
    moments=8   - maximum value of the spectrum
    moments=9   - coordinate of the maximum value of the spectrum
    moments=10  - minimum value of the spectrum
    moments=11  - coordinate of the minimum value of the spectrum
\end{verbatim}
\normalsize
The meaning of these is described in the CASA Reference Manual:
\begin{quote}
   \url{http://casa.nrao.edu/docs/casaref/image.moments.html}
\end{quote}

The {\tt axis} parameter sets the axis along which the moment is
``collapsed'' or calculated.  Choices are: 
{\tt 'ra'}, {\tt 'dec'}, {\tt 'lat'}, {\tt 'long'}, {\tt 'spectral'},
or {\tt 'stokes'}.  A standard moment-0 or moment-1 image of
a spectral cube would use the default choice {\tt 'spectral'}.
One could make a position-velocity map by setting {\tt 'ra'} or 
{\tt 'dec'}.

The {\tt includepix} and {\tt excludepix} parameters are used to set
ranges for the inclusion and exclusion of pixels based on values.
For example, {\tt includepix=[0.05,100.0]} will include pixels
with values from 50~mJy to 1000~Jy, and 
{\tt excludepix=[100.0,1000.0]} will exclude pixels with values
from 100 to 1000~Jy.

If a single moment is chosen, the {\tt outfile} specifies the exact
name of the output image.  If multiple {\tt moments} are chosen,
then {\tt outfile} will be used as the root of the output filenames,
which will get different suffixes for each moment.  

%%%%%%%%%%%%%%%%%%%%%%%%%%%%%%%%%%%%%%%%%%%%%%%%%%%%%%%%%%%%%%%%%
\subsection{Hints for using ({\tt immoments})}
\label{section:analysis.moments.hints}

In order to make an unbiased moment-0 image, do not put in 
any thresholding using {\tt includepix} or {\tt excludepix}.
This is so that the (presumably) zero-mean noise fluctuations
in off-line parts of the image cube will cancel out.  If you
image has large biases, like a pronounced clean bowl due to
missing large-scale flux, then your moment-0 image will be biased
also.  It will be difficult to alleviate this with a threshold,
but you can try.

To make a usable moment-1 (or higher) image, on the other hand,
it is critical to set a
reasonable threshold to exclude noise from being added to the
moment maps.  Something like a few times the rms noise level
in the usable planes seems to work (put into {\tt includepix}
or {\tt excludepix} as needed.  Also use {\tt chans} to ignore
channels with bad data.

%%%%%%%%%%%%%%%%%%%%%%%%%%%%%%%%%%%%%%%%%%%%%%%%%%%%%%%%%%%%%%%%%
\subsection{Examples using ({\tt immoments})}
\label{section:analysis.moments.example}

For example, using the NGC5921 example (\S~\ref{section:scripts.ngc5921}):
\small
\begin{verbatim}
  default('immoments')
  imagename = 'ngc5921.demo.cleanimg'
  # Do first and second spectral moments
  axis  = 'spectral'
  chans = ''
  moments = [0,1]
  # Need to mask out noisy pixels, currently done
  # using hard global limits
  excludepix = [-100,0.009]
  outfile = 'ngc5921.demo.moments'
  
  immoments()
  
  # It will have made the images:
  # --------------------------------------
  # ngc5921.demo.moments.integrated
  # ngc5921.demo.moments.weighted_coord
\end{verbatim}
\normalsize

Other examples of NGC2403 (a moment zero image of a VLA line dataset)
and NGC4826 (a moment one image of a BIMA CO line dataset) are
shown in Figure~\ref{fig:n2403momzero}.

\begin{figure}[h!]
\begin{center}
\pngname{n2403mom0}{3.15}
\pngname{n4826mom1}{3.30}
\caption{\label{fig:n2403momzero} NGC2403 VLA moment zero (left) and
NGC4826 BIMA moment one (right) images as shown in the {\tt viewer}.}
\hrulefill
\end{center}
\end{figure}

{\bf ALERT:} We are working on improving the thresholding
of planes beyond the global cutoffs in {\tt includepix}
and {\tt excludepix}.

%%%%%%%%%%%%%%%%%%%%%%%%%%%%%%%%%%%%%%%%%%%%%%%%%%%%%%%%%%%%%%%%%
%%%%%%%%%%%%%%%%%%%%%%%%%%%%%%%%%%%%%%%%%%%%%%%%%%%%%%%%%%%%%%%%%
\section{Computing image statistics ({\tt imstat})}
\label{section:analysis.imstat}

The {\tt imstat} task will calculate statistics on a region of
an image, and return the results as a return value in a Python
dictionary.

The inputs are:
\small
\begin{verbatim}
CASA <7>: inp imstat
--------> inp(imstat)
#  imstat :: Displays statistical information from an image or image region
imagename           =         ''        #  Name of the input image
axes                =         -1        #  List of axes to evaluate statistics over. Default is all axes.
region              =         ''        #  Image Region or name. Use Viewer
box                 =         ''        #  Select one or more box regions
chans               =         ''        #  Select the channel(spectral) range
stokes              =        'I'        #  Stokes params to image (I,IV,IQU,IQUV). Default "" => include all
listit              =       True        #  Print stats and bounding box to logger?
verbose             =       True        #  Print additional messages to logger?
async               =      False        #  If true the taskname must be started using imstat(...)
\end{verbatim}
\normalsize

Area selection using {\tt box} (this parameter will be removed from
CASA 3.4 and higher) and {\tt region} is detailed in 
\S~\ref{section:analysis.pars.box} and
\S~\ref{section:analysis.pars.regions} respectively.

Plane selection is controlled by {\tt chans} and {\tt stokes}.
See \S~\ref{section:analysis.pars.planes} for details on plane
selection.

The parameter {\it axes} will select the dimensions that the
statistics is calculated over. Typical data cubes have axes like: RA
axis 0, DEC axis 1, Velocity axis 2. So, e.g. {\tt axes=[0,1]} would
be the most common setting to calculate statistics per spectral
channel.


%%%%%%%%%%%%%%%%%%%%%%%%%%%%%%%%%%%%%%%%%%%%%%%%%%%%%%%%%%%%%%%%%
\subsection{Using the task return value}
\label{section:analysis.imstat.xstat}

The contents of the return value of {\tt imstat} are in a Python
dictionary of key-value sets.  For example,
\small
\begin{verbatim}
   xstat = imstat()
\end{verbatim}
\normalsize
will assign this to the Python variable {\tt xstat}.

The keys for {\tt xstat} are then:
\small
\begin{verbatim}
   KEYS
   blc          - absolute PIXEL coordinate of the bottom left corner of 
                  the bounding box surrounding the selected region
   blcf         - Same as blc, but uses WORLD coordinates instead of pixels
   trc          - the absolute PIXEL coordinate of the top right corner 
                  of the bounding box surrounding the selected region
   trcf         - Same as trc, but uses WORLD coordinates instead of pixels
   flux         - the integrated flux density if the beam is defined and 
                  the if brightness units are $Jy/beam$
   npts         - the number of unmasked points used
   max          - the maximum pixel value
   min          - minimum pixel value
   maxpos       - absolute PIXEL coordinate of maximum pixel value
   maxposf      - Same as maxpos, but uses WORLD coordinates instead of pixels
   minpos       - absolute pixel coordinate of minimum pixel value
   minposf      - Same as minpos, but uses WORLD coordinates instead of pixels
   sum          - the sum of the pixel values: $\sum I_i$
   sumsq        - the sum of the squares of the pixel values: $\sum I_i^2$
   mean         - the mean of pixel values: 
                  $ar{I} = \sum I_i / n$
   sigma        - the standard deviation about the mean: 
                  $\sigma^2 = (\sum I_i -ar{I})^2 / (n-1)$
   rms          - the root mean square: 
                  $\sqrt {\sum I_i^2 / n}$
   median       - the median pixel value (if robust=T)
   medabsdevmed - the median of the absolute deviations from the 
                  median (if robust=T)    
   quartile     - the inter-quartile range (if robust=T). Find the points 
                  which are 25% largest and 75% largest (the median is 
                  50% largest), find their difference and divide that 
                  difference by 2.
\end{verbatim}
\normalsize

For example, an {\tt imstat} call might be
\small
\begin{verbatim}
   default('imstat')
   imagename = 'ngc5921.demo.cleanimg.image'  #  The NGC5921 image cube
   box       = '108,108,148,148'              #  20 pixels around the center
   chans     = '21'                           #  channel 21

   xstat = imstat()
\end{verbatim}
\normalsize

In the terminal window, {\tt imstat} reports:
\small
\begin{verbatim}
Statistics on  ngc5921.usecase.clean.image

Region ---
   -- bottom-left corner (pixel) [blc]: [108, 108, 0, 21]
   -- top-right corner (pixel) [trc]:   [148, 148, 0, 21]
   -- bottom-left corner (world) [blcf]: 15:22:20.076, +04.58.59.981, I, 1.41332e+09Hz
   -- top-right corner( world) [trcf]: 15:21:39.919, +05.08.59.981, I, 1.41332e+09Hz

Values --
   -- flux [flux]:              0.111799236126
   -- number of points [npts]:  1681.0
   -- maximum value [max]:      0.029451508075
   -- minimum value [min]:     -0.00612453464419
   -- position of max value (pixel) [maxpos]:  [124, 131, 0, 21]
   -- position of min value (pixel) [minpos]:  [142, 110, 0, 21]
   -- position of max value (world) [maxposf]: 15:22:04.016, +05.04.44.999, I, 1.41332e+09Hz
   -- position of min value (world) [minposf]: 15:21:45.947, +04.59.29.990, I, 1.41332e+09Hz
   -- Sum of pixel values [sum]: 1.32267159822
   -- Sum of squared pixel values [sumsq]: 0.0284534543692
   
Statistics ---
   -- Mean of the pixel values [mean]:       0.000786836167885
   -- Standard deviation of the Mean [sigma]: 0.00403944306904
   -- Root mean square [rms]:               0.00411418313161
   -- Median of the pixel values [median]:     0.000137259965413
   -- Median of the deviations [medabsdevmed]:       0.00152346317191
   -- Quartile [quartile]:                       0.00305395200849

\end{verbatim}
\normalsize
The return value in {\tt xstat} is
\small
\begin{verbatim}
CASA <152>: xstat
  Out[152]: 
{'blc': array([108, 108,   0,  21]),
 'blcf': '15:22:20.076, +04.58.59.981, I, 1.41332e+09Hz',
 'flux': array([ 0.11179924]),
 'max': array([ 0.02945151]),
 'maxpos': array([124, 131,   0,  21]),
 'maxposf': '15:22:04.016, +05.04.44.999, I, 1.41332e+09Hz',
 'mean': array([ 0.00078684]),
 'medabsdevmed': array([ 0.00152346]),
 'median': array([ 0.00013726]),
 'min': array([-0.00612453]),
 'minpos': array([142, 110,   0,  21]),
 'minposf': '15:21:45.947, +04.59.29.990, I, 1.41332e+09Hz',
 'npts': array([ 1681.]),
 'quartile': array([ 0.00305395]),
 'rms': array([ 0.00411418]),
 'sigma': array([ 0.00403944]),
 'sum': array([ 1.3226716]),
 'sumsq': array([ 0.02845345]),
 'trc': array([148, 148,   0,  21]),
 'trcf': '15:21:39.919, +05.08.59.981, I, 1.41332e+09Hz'}
\end{verbatim}
\normalsize

{\bf ALERT:} The return dictionary currently includes 
NumPy {\tt array} values, which have to be accessed by
an array index to get the array value.
To access these dictionary elements, use the standard Python
dictionary syntax, e.g.
\small
\begin{verbatim}
     xstat[<key string>][<array index>]
\end{verbatim}
\normalsize
For example, to extract the standard deviation as a number
\small
\begin{verbatim}
   mystddev = xstat['sigma'][0]
   print 'Sigma = '+str(xstat['sigma'][0])
\end{verbatim}
\normalsize

%%%%%%%%%%%%%%%%%%%%%%%%%%%%%%%%%%%%%%%%%%%%%%%%%%%%%%%%%%%%%%%%%
\subsection{Examples for {\tt imstat}}
\label{section:analysis.imstat.examples}

The following are some examples using the B1608+656 Tutorial
\begin{quote}
  \url{http://casa.nrao.edu/Doc/Scripts/b1608_demo.py}
\end{quote}
as an example.

To extract statistics for the final image:
\small
\begin{verbatim}
   xstat = imstat('b1608.demo.clean2.image')
# Printing out some of these
   print 'Max   = '+str(xstat['max'][0])
   print 'Sigma = '+str(xstat['sigma'][0])
# results:
# Max   = 0.016796965152
# Sigma = 0.00033631979385
\end{verbatim}
\normalsize

In a box around the brightest component:
\small
\begin{verbatim}
   xstat_A = imstat('b1608.demo.clean2.image',box='124,125,132,133')
# Printing out some of these
   print 'Comp A Max Flux = '+str(xstat_A['max'][0])
   print 'Comp A Max X,Y  = ('+str(xstat_A['maxpos'][0])+','+str(xstat_A['maxpos'][1])+')'
# results:
# Comp A Max Flux = 0.016796965152
# Comp A Max X,Y  = (128,129)
\end{verbatim}
\normalsize

%%%%%%%%%%%%%%%%%%%%%%%%%%%%%%%%%%%%%%%%%%%%%%%%%%%%%%%%%%%%%%%%%
%%%%%%%%%%%%%%%%%%%%%%%%%%%%%%%%%%%%%%%%%%%%%%%%%%%%%%%%%%%%%%%%%
\section{Extracting data from an image ({\tt imval})}
\label{section:analysis.imval}

The {\tt imval} task will extract the values of the data and mask
from a specified region of an image and place in the task return
value as a Python dictionary.

The inputs are:
\small
\begin{verbatim}
#  imval :: Get the data value(s) and/or mask value in an image.
imagename  =      ''   #  Name of the input image
region     =      ''   #  Image Region.  Use viewer
box        =      ''   #  Select one or more box regions
chans      =      ''   #  Select the channel(spectral) range
stokes     =      ''   #  Stokes params to image (I,IV,IQU,IQUV)
async      =   False       
\end{verbatim}
\normalsize

Area selection using {\tt box} and {\tt region} is detailed in 
\S~\ref{section:analysis.pars.box} and
\S~\ref{section:analysis.pars.regions} respectively.
By default, {\tt box=''} will
extract the image information at the reference pixel on the
direction axes.

Plane selection is controlled by {\tt chans} and {\tt stokes}.
See \S~\ref{section:analysis.pars.planes} for details on plane
selection.  By default, {\tt chans=''} and {\tt stokes=''} will
extract the image information in all channels and Stokes planes.

For instance,
\small
\begin{verbatim}
   xval = imval('myimage', box='144,144', stokes='I' )
\end{verbatim}
\normalsize
will extract the Stokes I value or spectrum at pixel 144,144, while
\small
\begin{verbatim}
   xval = imval('myimage', box='134,134.154,154', stokes='I' )
\end{verbatim}
\normalsize
will extract a 21 by 21 pixel region.

Extractions are returned in NumPy arrays in the return value
dictionary, plus some extra elements describing the axes and selection:
\small
\begin{verbatim}
CASA <2>: xval = imval('ngc5921.demo.moments.integrated')

CASA <3>: xval
  Out[3]: 
{'axes': [[0, 'Right Ascension'],
          [1, 'Declination'],
          [3, 'Frequency'],
          [2, 'Stokes']],
 'blc': [128, 128, 0, 0],
 'data': array([ 0.89667124]),
 'mask': array([ True], dtype=bool),
 'trc': [128, 128, 0, 0],
 'unit': 'Jy/beam.km/s'}
\end{verbatim}
\normalsize
extracts the reference pixel value in this 1-plane image.  Note that
the {\tt 'data'} and {\tt 'mask'} elements are NumPy arrays, not 
Python lists.

To extract a spectrum from a cube:
\small
\begin{verbatim}
CASA <8>: xval = imval('ngc5921.demo.clean.image',box='125,125')

CASA <9>: xval
  Out[9]: 
{'axes': [[0, 'Right Ascension'],
          [1, 'Declination'],
          [3, 'Frequency'],
          [2, 'Stokes']],
 'blc': [125, 125, 0, 0],
 'data': array([  8.45717848e-04,   1.93370355e-03,   1.53750915e-03,
         2.88399984e-03,   2.38683447e-03,   2.89159478e-04,
         3.16268904e-03,   9.93389636e-03,   1.88773088e-02,
         3.01138610e-02,   3.14478502e-02,   4.03211266e-02,
         3.82498614e-02,   3.06552909e-02,   2.80734301e-02,
         1.72479432e-02,   1.20884273e-02,   6.13593217e-03,
         9.04005766e-03,   1.71429547e-03,   5.22095338e-03,
         2.49114982e-03,   5.30831399e-04,   4.80734324e-03,
         1.19265869e-05,   1.29435991e-03,   3.75700940e-04,
         2.34788167e-03,   2.72604497e-03,   1.78467855e-03,
         9.74952069e-04,   2.24676146e-03,   1.82263291e-04,
         1.98463408e-06,   2.02975096e-03,   9.65532148e-04,
         1.68218743e-03,   2.92119570e-03,   1.29359076e-03,
        -5.11484570e-04,   1.54162932e-03,   4.68662125e-04,
        -8.50282842e-04,  -7.91683051e-05,   2.95954203e-04,
        -1.30133145e-03]),
 'mask': array([ True,  True,  True,  True,  True,  True,  True,  True,  True,
        True,  True,  True,  True,  True,  True,  True,  True,  True,
        True,  True,  True,  True,  True,  True,  True,  True,  True,
        True,  True,  True,  True,  True,  True,  True,  True,  True,
        True,  True,  True,  True,  True,  True,  True,  True,  True,  True], dtype=bool),
 'trc': [125, 125, 0, 45],
 'unit': 'Jy/beam'}
\end{verbatim}
\normalsize

To extract a region from the plane of a cube:
\small
\begin{verbatim}
CASA <13>: xval = imval('ngc5921.demo.clean.image',box='126,128,130,129',chans='23')

CASA <14>: xval
  Out[14]: 
{'axes': [[0, 'Right Ascension'],
          [1, 'Declination'],
          [3, 'Frequency'],
          [2, 'Stokes']],
 'blc': [126, 128, 0, 23],
 'data': array([[ 0.00938627,  0.01487772],
       [ 0.00955847,  0.01688832],
       [ 0.00696965,  0.01501907],
       [ 0.00460964,  0.01220793],
       [ 0.00358087,  0.00990202]]),
 'mask': array([[ True,  True],
       [ True,  True],
       [ True,  True],
       [ True,  True],
       [ True,  True]], dtype=bool),
 'trc': [130, 129, 0, 23],
 'unit': 'Jy/beam'}

CASA <15>: print xval['data'][0][1]
0.0148777160794
\end{verbatim}
\normalsize
In this example, a rectangular box was extracted, and you can see the
order in the array and how to address specific elements.

%%%%%%%%%%%%%%%%%%%%%%%%%%%%%%%%%%%%%%%%%%%%%%%%%%%%%%%%%%%%%%%%%
%%%%%%%%%%%%%%%%%%%%%%%%%%%%%%%%%%%%%%%%%%%%%%%%%%%%%%%%%%%%%%%%%
\section{Reordering the Axes of an Image Cube ({\tt imtrans})}
\label{section:analysis.imtrans}

Sometimes data cubes can be in axis orders that are not adequate for
processing. The CASA task {\tt imtrans} can change the ordering of the axis:
\small
\begin{verbatim}
#  imtrans :: Reorder image axes
imagename           =         ''        #  Name of the input image
outfile             =         ''        #  Name of output CASA image.
order               =         ''        #  New zero-based axes order.
wantreturn          =       True        #  Return an image tool referencing the
                                        #   transposed image
async               =      False        #  If true the taskname must be started
                                        #   using imtrans(...)
\end{verbatim}
\normalsize




The {\tt order} parameter is the most important input here. It is a
string of numbers that shows how axes 0, 1, 2, 3, ... are mapped onto
the new cube (note that the first axis has the label 0, as typical in
python). E.g.{\tt order}='1032' will reorder the input axis 0 to be
axis 1 in the output, input axis 1 to be output axis 0, input axis 2
to output axis 3 (the last axis) and input axis 3 to output axis
2. Alternatively, axes can be specified by their names. E.g., to reorder
an image with right ascension, declination, and frequency and reverse
the first two, {\tt order=[``declination'', ``right ascension'',
  ``frequency'']} will work. The axes names can be found typing {\tt
  (ia.coordsys().names())}. Minimum match is supported, so that  {\tt
  order=["d", "f", "r"]} will produce the same results.

Axes can simultaneously be transposed and reversed. To reverse an axis,
precede it by a "-". For example, {\tt order='-10-32'} will reverse
the direction of the first and third axis of the input image (the zeroth and second
axes in the output image).

Example:\\

Swap the stokes and spectral axes in an RA-Dec-Stokes-Frequency image
\small
\begin{verbatim}
imagename = "myim.im"
outfile = "outim.im"
order = "0132"
imtrans()
\end{verbatim}

or
\begin{verbatim}
outfile = "myim_2.im"
order = 132
imtrans()
\end{verbatim}

or
\begin{verbatim}
outfile = "myim_3.im"
order = ["r", "d", "f", "s"]
imtrans()
\end{verbatim}

or

\begin{verbatim}
outfile = "myim_4.im"
order = ["rig", "declin", "frequ", "stok"]
imtrans()
\end{verbatim}



If the outfile parameter is empty, only a temporary image is created; no output image
is written to disk. The temporary image can be captured in the returned value (assuming
{\tt wantreturn} is true).

%%%%%%%%%%%%%%%%%%%%%%%%%%%%%%%%%%%%%%%%%%%%%%%%%%%%%%%%%%%%%%%%%
%%%%%%%%%%%%%%%%%%%%%%%%%%%%%%%%%%%%%%%%%%%%%%%%%%%%%%%%%%%%%%%%%
\section{Collapsing an Image Along an Axis ({\tt imcollapse})}
\label{section:analysis.imcollapse}

{\tt imcollapse} allows to apply an aggregation function along one or
more axes of an image. Functions supported are 'max', 'mean',
'median', 'min', 'rms', 'stdev', 'sum', 'variance' (minimum match
supported). The relevant axes will then collapse to a single value or
plane (i.e. they will result in a degenerate axis). The functions are
specified in the {\tt function} parameter of the {\tt imcollapse}
inputs:

\small
\begin{verbatim}
#  imcollapse :: Collapse image along one axis, aggregating pixel values along that axis.
imagename           =         ''        #  Name of the input image
function            =         ''        #  Function used to compute aggregation
                                        #   of pixel values.
axes                =        [0]        #  Zero-based axis number(s) or minimal
                                        #   match strings to collapse.
outfile             =         ''        #  Name of output CASA image.
box                 =         ''        #  Optional direction plane box ("blcx,
                                        #   blcy, trcx trcy").
     region         =         ''        #  Name of optional region file to use.

chans               =         ''        #  Optional zero-based contiguous
                                        #   frequency channel specification.
stokes              =         ''        #  Optional contiguous stokes planes
                                        #   specification.
mask                =         ''        #  Optional mask to use.
wantreturn          =       True        #  Should an image analysis tool
                                        #   referencing the collapsed image be
                                        #   returned?
async               =      False        #  If true the taskname must be started
\end{verbatim}
\normalsize

{\tt wantreturn=True} returns an image analysis tool containing the
newly created collapsed image.

Example:\\

myimage.im is a 512x512x128x4 (ra,dec,freq,stokes; i.e. in the 0-based
system, frequency is labeled as axis 2) image and we want to
collapse a subimage of it along its spectral axis avoiding the 8 edge
 channels at each end of the band, computing the mean value of the
pixels (resulting image is 256x256x1x4 in size):

\small
\begin{verbatim}
imcollapse(imagename="myimage.im", outfile="collapse_spec_mean.im",
           function="mean", axis=2, box="127,127,383,383", chans="8~119")
\end{verbatim}
\normalsize


%%%%%%%%%%%%%%%%%%%%%%%%%%%%%%%%%%%%%%%%%%%%%%%%%%%%%%%%%%%%%%%%%
%%%%%%%%%%%%%%%%%%%%%%%%%%%%%%%%%%%%%%%%%%%%%%%%%%%%%%%%%%%%%%%%%
\section{Regridding an Image ({\tt imregrid})}
\label{section:analysis.regrid}

\begin{wrapfigure}{r}{2.5in}
  \begin{boxedminipage}{2.5in}
     \centerline{\bf Inside the Toolkit:}
     More complex coordinate system and image regridding 
     operation can be carried out in the toolkit.  The 
     {\tt coordsys} ({\tt cs}) tool and the {\tt ia.regrid}
     method are the relevant components.
  \end{boxedminipage}
\end{wrapfigure}

It is occasionally necessary to regrid an image onto a new coordinate
system.  The {\tt imregrid} task will regrid one image onto the
coordinate system of another, creating an output image.  In this
task, the user need only specify the names of the input, template, and
output images.  

If the user needs to do more complex operations, such as regridding an
image onto an arbitrary (but known) coordinate system, changing from
Equatorial to Galactic coordinates, or precessing Equinoxes, the CASA
toolkit can be used (see sidebox).  Some of these facilities will
eventually be provided in task form.

The default inputs are:
\small
\begin{verbatim}
#  imregrid :: regrid an image onto a template image
imagename           =         ''        #  Name of input image
template            =         ''        #  Name of reference image
output              =         ''        #  Name of output regridded image
async               =      False        #  
\end{verbatim}
\normalsize
The {\tt output} image will have the data in {\tt imagename} regridded
onto the coordinate system provided by the {\tt template} parameter.
{\tt template} is used universally for a range of ways to define the
grid of the output image:

\begin{itemize}
  \item {\it a template image}: specify an image name here and the
    input will be regridded to the same 3-dimensional coordinate
    system as the one in {\tt template}. Values are filled in as
    blanks if they do not exist in the input. Note that the input and
    template images must have the same coordinate structure to begin
    with (like 3 axes)
  \item {\it a coordinate system (reference code)}: to convert from one coordinate
    frame to another one, e.g. from B1950 to J2000, the {\tt template}
    parameter can be used to specify the output coordinate
    system. These following recognized keywords are supported: {\tt
      'J2000'}, {\tt 'B1950'}, {\tt 'B1950\_VLA'}, {\tt 'GALACTIC'},
    {\tt 'HADEC'}, {\tt 'AZEL'}, {\tt 'AZELSW'}, {\tt 'AZELNE'}, {\tt
      'ECLIPTIC'}, {\tt 'MECLIPTIC'}, {\tt 'TECLIPTIC'}, {\tt 'SUPERGAL'}

   \item {\it 'get'}: This option returns a python dictionary in the
     {\tt \{'csys': csys\_record, 'shap': shape\}  } format

   \item {\it a python dictionary}: In turn, such a dictionary can be
     used as a template to define the final grid
\end{itemize}

  

%
%
%       imregrid: regrid an image to a new coordinate system.
%
%       The new coordinate system is defined by the template parameter,
%       which can be:
%         * a recognized reference code string (see below),
%         * a {'csys': csys_record, 'shap': shape} dictionary,
%         * 'get', which does not regrid but returns the template dictionary
%           for imagename, suitable for modification and reuse, or
%         * the name of an image to get the coordinate system and shape
%           from.  The input and template images must have the same
%           coordinate structure.
%
%       Keyword arguments:
%       imagename -- Name of the image that needs to be regridded
%               default: none; example: imagename='orion.image'
%       template -- Dictionary, reference code, or imagename defining the new
%               shape and coordinate system, or 'get' to return the template
%               dictionary for imagename.  Recognized reference codes are:
%               'J2000', 'B1950', 'B1950_VLA', 'GALACTIC', 'HADEC', 'AZEL',
%               'AZELSW', 'AZELNE', 'ECLIPTIC', 'MECLIPTIC', 'TECLIPTIC',
%               and 'SUPERGAL'.
%               default: 'get'; example: template='orion_j2000.im'
%       output -- Name for the regridded image
%               default: imagename + '.regridded'; example: imagename='orion_shifted.im'
%       async -- Run task in a separate process (return CASA prompt)
%               default: False; example: async=True
%



%%%%%%%%%%%%%%%%%%%%%%%%%%%%%%%%%%%%%%%%%%%%%%%%%%%%%%%%%%%%%%%%%
%%%%%%%%%%%%%%%%%%%%%%%%%%%%%%%%%%%%%%%%%%%%%%%%%%%%%%%%%%%%%%%%%
\section{Image Convolution({\tt imsmooth})}
\label{section:analysis.imsmooth}

A data cube can be smoothed across spatial dimensions with {\tt
  imsmooth}. The inputs are:\\ 
\small
\begin{verbatim}
#  imsmooth :: Smooth an image or portion of an image
imagename           =         ''        #  Name of the input image
kernel              =    'gauss'        #  Type of kernel to use: gaussian or
                                        #   boxcar.
     major          = '2.5arcsec'       #  Major axis for the kernels, default
                                        #   direction is along y-axis.
     minor          = '2.0arcsec'       #  Minor axis in gaussian and boxcar
                                        #   kernels
     pa             =     '0deg'        #  Position angle for gaussian kernel
     targetres      =      False        #  If gaussian kernel, specified
                                        #   parameters are to be resolution of
                                        #   output image (True) or parameters of
                                        #   gaussian to convolve with input image
                                        #   (False).

region              =         ''        #  Image Region or name.  Use viewer
box                 =         ''        #  Select one or more box regions
chans               =         ''        #  Select the spectral channel range
stokes              =         ''        #  Stokes parameters to image
                                        #   (I,IV,IQU,IQUV)
mask                =         ''        #  Mask used for selecting the area of
                                        #   the image
outfile             =         ''        #  Output, smoothed, image file name
async               =      False        #  If true the taskname must be started
\end{verbatim}
\normalsize
where the cube/image {\tt imagename} will be convolved with a kernel
defined in the {\tt kernel} keyword. Available kernels are 'gauss' and
'boxcar'. Both of these kernels need the major and minor axes sizes as
input, the Gaussian kernel smoothing also requires a position
angle. By default, the kernel size defines the kernel itself, i.e. the
data will be smoothed with this kernel. If the {\tt targetres} parameter
for Gaussian kernels is set to 'True', major and minor axes will be
those from the {\it output} resolution. 

Examples: \\

1) smoothing with a gaussian kernel 20'' by 10''
\begin{verbatim}
imsmooth( imagename='my.image', kernel='gauss', major='10arcsec', minor='10arcsec')
\end{verbatim}


2) Smoothing using pixel coordinates and a boxcar kernel.
\begin{verbatim}
imsmooth( imagename='new.image', major='20pix', minor='10pix', kernel='boxcar')
\end{verbatim}

%%%%%%%%%%%%%%%%%%%%%%%%%%%%%%%%%%%%%%%%%%%%%%%%%%%%%%%%%%%%%%%%%
%%%%%%%%%%%%%%%%%%%%%%%%%%%%%%%%%%%%%%%%%%%%%%%%%%%%%%%%%%%%%%%%%

%%%%%%%%%%%%%%%%%%%%%%%%%%%%%%%%%%%%%%%%%%%%%%%%%%%%%%%%%%%%%%%%%
\section{Spectral Line fitting with {\tt specfit}}
\label{section:analysis.specfit}

{\tt specfit} is a powerful task to perform spectral line fits in data
cubes. Two types of fitting functions are currently supported,
polynomials and Gaussians. {\tt specfit} can fit these functions in
two ways: over data that were averaged across a region ({\tt
  multifit=False}) or on a pixel
by pixel basis ({\tt multifit=True}). 


\small
\begin{verbatim}
#  specfit :: Fit 1-dimensional gaussians and/or polynomial models to an image or image region
imagename           =         ''        #  Name of the input image
box                 =         ''        #  Rectangular box in direction
                                        #   coordinate blc, trc. Default: entire
                                        #   image ("").
region              =         ''        #  Region name, Default: no region ("").
chans               =         ''        #  Channels to use. Channels must be
                                        #   contiguous. Default: all channels
                                        #   ("").
stokes              =         ''        #  Stokes planes to use. Planes must be
                                        #   contiguous. Default: all stokes ("").
axis                =         -1        #  The profile axis. Default: use the
                                        #   spectral axis if one exists, axis 0
                                        #   otherwise (<0).
mask                =         ''        #  OTF mask, Boolean LEL expression or
                                        #   mask region.  Default: no mask ("").
poly                =         -1        #  Order of polynomial element.  Default:
                                        #   do not fit a polynomial (<0).
estimates           =         ''        #  Name of file containing initial
                                        #   estimates.  Default: No initial
                                        #   estimates ("").
     ngauss         =          1        #  Number of Gaussian elements.  Default:
                                        #   1.

minpts              =          0        #  Minimum number of unmasked points
                                        #   necessary to attempt fit.
multifit            =       True        #  If true, fit a profile along the
                                        #   desired axis at each pixel in the
                                        #   specified region. If false, average
                                        #   the non-fit axis pixels and do a
                                        #   single fit to that average profile.
                                        #   Default False.
     amp            =         ''        #  Name of amplitude solution image.
                                        #   Default: do not write the image ("").
     amperr         =         ''        #  Name of amplitude solution error
                                        #   image. Default: do not write the
                                        #   image ("").
     center         =         ''        #  Name of center solution image.
                                        #   Default: do not write the image ("").
     centererr      =         ''        #  Name of center solution error image.
                                        #   Default: do not write the image ("").
     fwhm           =         ''        #  Name of fwhm solution image. Default:
                                        #   do not write the image ("").
     fwhmerr        =         ''        #  Name of fwhm solution error image.
                                        #   Default: do not write the image ("").
     integral       =         ''        #  Prefix of ame of integral solution
                                        #   image. Name of image will have
                                        #   gaussian component number appended.
                                        #   Default: do not write the image ("").
     integralerr    =         ''        #  Prefix of name of integral error
                                        #   solution image. Name of image will
                                        #   have gaussian component number
                                        #   appended.  Default: do not write the
                                        #   image ("").

model               =         ''        #  Name of model image. Default: do not
                                        #   write the model image ("").
residual            =         ''        #  Name of residual image. Default: do
                                        #   not write the residual image ("").
wantreturn          =       True        #  Should a record summarizing the
                                        #   results be returned?
async               =      False        #  If true the taskname must be started
                                        #   using specfit(...)
\end{verbatim}
\normalsize

For Gaussian fits, the task will allow multiple Gaussian components
and {\tt specfit} will try to find the best solution. The parameter
space, however, is usually not uniform and to avoid local minima in the
goodness-of-fit space, one can provide initial start values for the
fits. The {\tt estimates} parameter will take a file with the initial
estimates for the individual Gaussians (one Gaussian parameter set per
line) and their parameters. The
file has the following format:

{\tt [peak intensity], [center], [fwhm], [optional fixed parameter string]}

The first three values are required and must be numerical values. The
peak intensity must be expressed in map units, while the
center and fwhm must be specified in pixels. The fourth value is
optional and if present, represents the parameter(s)
that should be held constant during the fit. Any combination of the
characters 'p' (peak), 'c' (center), and 'f' (fwhm) are
permitted, eg "fc" means hold the fwhm and the center constant during
the fit. Fixed parameters will have no errors associated
with them in the solution. 

An example {\tt estimates} file is: 

\small
\begin{verbatim}
# estimates file indicating that two gaussians should be fit
# first guassian estimate, peak=40, center at pixel number 10.5, 
# fwhm = 5.8 pixels, all parameters allowed to vary during
# fit
40, 10.5, 5.8
# second gaussian, peak = 4, center at pixel number 90.2, 
# fwhm = 7.2 pixels, hold fwhm constant
4, 90.2, 7.2, f
# end file
\end{verbatim}
\normalsize



and the output of a typical execution, e.g. 

\small
\begin{verbatim}
specfit(imagename='IRC10216_HC3N.cube_r0.5.image', region='specfit.rgn', multifit=F,
        estimates='', ngauss=2)
\end{verbatim}
\normalsize
('specfit.rgn' is a CASA regions file, see
Section\,\ref{chapter:regionformat}; but note that the old format is
still created by the {\tt viewer} and is still supported in CASA 3.3)\\

will be 


\small
\begin{verbatim}
Fit   :
    RA           :   09:47:57.49
    Dec          :   13.16.46.46
    Stokes       : I
    Pixel        : [146.002, 164.499, 0.000,  *]
    Attempted    : YES
    Converged    : YES
    Iterations   : 28
    Results for component 0:
        Type     : GAUSSIAN
        Peak     : 5.76 +/- 0.45 mJy/beam
        Center   : -15.96 +/- 0.32 km/s
                   40.78 +/- 0.31 pixel
        FWHM     : 7.70 +/- 0.77 km/s
                   7.48 +/- 0.74 pixel
        Integral : 47.2 +/- 6.0 mJy/beam.km/s
    Results for component 1:
        Type     : GAUSSIAN
        Peak     : 4.37 +/- 0.33 mJy/beam
        Center   : -33.51 +/- 0.58 km/s
                   23.73 +/- 0.57 pixel
        FWHM     : 15.1 +/- 1.5 km/s
                   14.7 +/- 1.5 pixel
        Integral : 70.2 +/- 8.8 mJy/beam.km/s
\end{verbatim}
\normalsize

If {\tt wantreturn}=True (the default value), the task
returns a python dictionary (here captured in a variable
with the inventive name of 'fitresults') : 

\small
\begin{verbatim}
fitresults=specfit(imagename='IRC10216_HC3N.cube_r0.5.image', region='specfit.rgn', multifit=F,
        estimates='', ngauss=2)
\end{verbatim}
\normalsize

The values can then be used by other python code for further
processing. 

As mentioned above, {\tt specfit} can also fit spectral cubes on a
pixel by pixel basis. In this case, one can choose to write none, any
or all of the solution and error images for Gaussian fits via the
parameters {\tt amp}, {\tt amperr}, {\tt center}, {\tt centererr},
{\tt fwhm}, and {\tt fwhmerr}. The
specified parameter value will by appended by "\_n" where n is the
Gaussian component number. Writing analogous images for polynomial
coefficients is not yet supported although polynomial fits when {\tt
  multifit=True} is supported. Best fit coefficients are written to the
logger. Pixels for which fits were not attempted or did not converge
will be masked as bad.
 





%%%%%%%%%%%%%%%%%%%%%%%%%%%%%%%%%%%%%%%%%%%%%%%%%%%%%%%%%%%%%%%%%
%%%%%%%%%%%%%%%%%%%%%%%%%%%%%%%%%%%%%%%%%%%%%%%%%%%%%%%%%%%%%%%%%
\section{Search for Spectral Line Rest Frequencies ({\tt slsearch})}
\label{section:analysis.slsearch}

The {\tt slsearch} task allows the spectral line enthusiast to find their
favorite spectral lines in subset of the Splatalogue spectral line
catalog ({\url http://www.splatalogue.net}) which is distributed with CASA.
In addition, one can export custom catalogs from Splatalogue and
import them to CASA using the task {\tt splattotable}
(Sect.\,\ref{section:analysis.splattotable}) or tool method
{\tt sl.splattotable()}. One can even import catalogs with lines not in
Splatalogue using the same file format.

The inputs to {\tt slsearch} are as follows:  

\small
\begin{verbatim}
#  slsearch :: Search a spectral line table.
tablename           =         ''        #  Input spectral line table name to
                                        #   search. If not specified, use the
                                        #   default table in the system.
outfile             =         ''        #  Results table name. Blank means do not
                                        #   write the table to disk.
freqrange           =   [84, 90]        #  Frequency range in GHz.
species             =       ['']        #  Species to search for.
reconly             =      False        #  List only NRAO recommended
                                        #   frequencies.
chemnames           =       ['']        #  Chemical names to search for.
qns                 =       ['']        #  Resolved quantum numbers to search
                                        #   for.
rrlinclude          =       True        #  Include RRLs in the result set?
rrlonly             =      False        #  Include only RRLs in the result set?
     intensity      =         -1        #  CDMS/JPL intensity range. -1 -> do not
                                        #   use an intensity range.
     smu2           =         -1        #  S*mu*mu range in Debye**2. -1 -> do
                                        #   not use an S*mu*mu range.
     loga           =         -1        #  log(A) (Einstein coefficient) range.
                                        #   -1 -> do not use a loga range.
     eu             =         -1        #  Upper energy state range in Kelvin. -1
                                        #   -> do not use an eu range.
     el             =         -1        #  Lower energy state range in Kelvin. -1
                                        #   -> do not use an el range.

verbose             =       True        #  List result set to logger (and
                                        #   optionally logfile)?
     logfile        =         ''        #  List result set to this logfile (only
                                        #   used if verbose=True).
     append         =       True        #  If true, append to logfile if it
                                        #   already exists, if false overwrite
                                        #   logfile it it exists. Only used if
                                        #   verbose=True and logfile not blank.

wantreturn          =       True        #  If true, return the spectralline tool
                                        #   associated with the result set.
async               =      False        #  If true the taskname must be started
    
\end{verbatim}
\normalsize

The table is provided in the {\tt tablename} parameter but if it is
blank (the default), the catalog which is included with CASA will be
used. Searches can be made in a parameter space with large
dimensionality: 

\begin{itemize}

\item {\tt freqrange     } Frequency range in GHz.
\item {\tt species       } Species to search for.
\item {\tt reconly       } List only NRAO recommended frequencies.
\item {\tt chemnames     } Chemical names to search for.
\item {\tt qns           } Resolved quantum numbers to search for.
\item {\tt intensity     } CDMS/JPL intensity range. 
\item {\tt smu2          } $S\mu^{2}$ range in Debye$^{2}$. 
\item {\tt loga          } log(A) (Einstein coefficient) range. 
\item {\tt el            } Lower energy state range in Kelvin. 
\item {\tt eu            } Upper energy state range in Kelvin.
\item {\tt rrlinclude    } Include RRLs in the result set?
\item {\tt rrlonly       } Include only RRLs in the result set?
\end{itemize}

Notation is as found in the {\it Splatalogue} catalog. 

Example:\\
Search for all lines of the species HOCN and HOCO+ in the 200-300GHz range:
\small
\begin{verbatim}
sl.search(outfile="myresults.tbl", freqrange = [200,300], species=['HOCN', 'HOCO+'])
\end{verbatim}
\normalsize



%%%%%%%%%%%%%%%%%%%%%%%%%%%%%%%%%%%%%%%%%%%%%%%%%%%%%%%%%%%%%%%%%
%%%%%%%%%%%%%%%%%%%%%%%%%%%%%%%%%%%%%%%%%%%%%%%%%%%%%%%%%%%%%%%%%
\section{Convert Exported Splatalogue Catalogs to CASA Tables ({\tt splattotable})}
\label{section:analysis.splattotable}

In some cases the internal spectral line catalog may not contain the
lines in which one is interested. In that case, one can export a
catalog from {\it Splatalogue} {\url http://www.splatalogue.net} or
even create their own "by hand" (be careful to get the format exactly
right though!). CASA's task {\tt splattotable} can then be used to
create a CASA table that contains these lines and can be searched:

\small
\begin{verbatim}
---------> inp(splattotable)
#  splattotable :: Convert a downloaded Splatalogue spectral line list to a casa table.
filenames           =       ['']        #  Files containing Splatalogue lists.
table               =         ''        #  Output table name.
wantreturn          =       True        #  Do you want the task to return a spectralline tool attached to the results table?
async               =      False        #  If true the taskname must be started using splattotable(...)
\end{verbatim}
\normalsize

A search in Splatalogue will return a catalog that can be saved in a
file (look for the "Export" section after the results on the search
results page). The exported filename(s) should be entered in the {\tt
  filenames} parameter of {\tt splattotable}. The downloaded files
must be in a specific format for this task to succeed. If you use the
Splatalogue "Export CASA fields" feature, you should have no
difficulties.

%%%%%%%%%%%%%%%%%%%%%%%%%%%%%%%%%%%%%%%%%%%%%%%%%%%%%%%%%%%%%%%%%
%%%%%%%%%%%%%%%%%%%%%%%%%%%%%%%%%%%%%%%%%%%%%%%%%%%%%%%%%%%%%%%%%
\section{Image Import/Export to FITS}
\label{section:analysis.fits}

These tasks will allow you to write your CASA image to a FITS file
that other packages can read, and to import existing FITS files
into CASA as an image.

%%%%%%%%%%%%%%%%%%%%%%%%%%%%%%%%%%%%%%%%%%%%%%%%%%%%%%%%%%%%%%%%%
\subsection{FITS Image Export ({\tt exportfits})}
\label{section:analysis.fits.export}

To export your images to fits format use the {\tt exportfits} task.
The inputs are:
\small
\begin{verbatim}
#  exportfits :: Convert a CASA image to a FITS file
imagename    =         ''   #  Name of input CASA image
fitsimage    =         ''   #  Name of output image FITS file
velocity     =      False   #  Use velocity (rather than frequency) as spectral axis
optical      =      False   #  Use the optical (rather than radio) velocity convention
bitpix       =        -32   #  Bits per pixel
minpix       =          0   #  Minimum pixel value
maxpix       =          0   #  Maximum pixel value
overwrite    =      False   #  Overwrite pre-existing imagename
dropstokes   =      False   #  Drop the Stokes axis?
stokeslast   =       True   #  Put Stokes axis last in header?
async        =      False   #  If true the taskname must be started using exportfits(...)
\end{verbatim}
\normalsize

The {\tt dropstokes} or {\tt stokeslast} parameter may be needed to
make the FITS image compatible with an external application.

For example,
\small
\begin{verbatim}
   exportfits('ngc5921.demo.cleanimg.image','ngc5921.demo.cleanimg.image.fits')
\end{verbatim}
\normalsize


%%%%%%%%%%%%%%%%%%%%%%%%%%%%%%%%%%%%%%%%%%%%%%%%%%%%%%%%%%%%%%%%%
\subsection{FITS Image Import ({\tt importfits})}
\label{section:analysis.fits.import}

You can also use the {\tt importfits} task to import a FITS image into
CASA image table format.  Note, the CASA {\tt viewer} can read fits
images so you don't need to do this if you just want to look a the image.  
The inputs for {\tt importfits} are:
\small
\begin{verbatim}
#  importfits :: Convert an image FITS file into a CASA image:

fitsimage    =         ''   #   Name of input image FITS file
imagename    =         ''   #   Name of output CASA image
whichrep     =          0   #   Which coordinate representation (if multiple)
whichhdu     =          0   #   Which image (if multiple)
zeroblanks   =       True   #   If blanked fill with zeros (not NaNs)
overwrite    =      False   #   Overwrite pre-existing imagename
async        =      False   #   if True run in the background, prompt is freed
\end{verbatim}
\normalsize
For example, we can read the above image back in
\small
\begin{verbatim}
  importfits('ngc5921.demo.cleanimg.image.fits','ngc5921.demo.cleanimage')
\end{verbatim}
\normalsize

%%%%%%%%%%%%%%%%%%%%%%%%%%%%%%%%%%%%%%%%%%%%%%%%%%%%%%%%%%%%%%%%%
%%%%%%%%%%%%%%%%%%%%%%%%%%%%%%%%%%%%%%%%%%%%%%%%%%%%%%%%%%%%%%%%%
\section{Using the CASA Toolkit for Image Analysis}
\label{section:analysis.toolkit}

\begin{wrapfigure}{r}{2.5in}
  \begin{boxedminipage}{2.5in}
     \centerline{\bf Inside the Toolkit:}
     The image analysis tool ({\tt ia}) is the workhorse here.
     It appears in the User Reference Manual as the {\tt image}
     tool.  Other relevant tools for analysis and manipulation
     include {\tt measures} ({\tt me}), {\tt quanta} ({\tt qa})
     and {\tt coordsys} ({\tt cs}).
  \end{boxedminipage}
\end{wrapfigure}

Although this cookbook is aimed at general users employing the
tasks, we include here a more detailed description of doing
image analysis in the CASA toolkit.  This is because
there are currently only a few tasks geared towards image analysis,
as well as due to the breadth of possible manipulations that the
toolkit allows that more sophisticated users will appreciate.

To see a list of the {\tt ia} methods available, use the 
CASA {\tt help} command:
\small
\begin{verbatim}
CASA <1>: help ia 
--------> help(ia)
Help on image object:

class image(__builtin__.object)
 |  image object
 |  
 |  Methods defined here:
 |  
 |  __init__(...)
 |      x.__init__(...) initializes x; see x.__class__.__doc__ for signature
 |  
 |  __str__(...)
 |      x.__str__() <==> str(x)
 |  
 |  adddegaxes(...)
 |      Add degenerate axes of the specified type to the image`  : 
 |        outfile
 |        direction = false
 |        spectral  = false
 |        stokes
 |        linear    = false
 |        tabular   = false
 |        overwrite = false
 |      ----------------------------------------
 |  
 |  addnoise(...)

...

 |  
 |  unlock(...)
 |      Release any lock on the image`  : 
 |      ----------------------------------------
 |  
 |  ----------------------------------------------------------------------
 |  Data and other attributes defined here:
 |  
 |  __new__ = <built-in method __new__ of type object at 0x55d0f20>
 |      T.__new__(S, ...) -> a new object with type S, a subtype of T

\end{verbatim}
\normalsize
or for a compact listing use {\tt <TAB>} completion on {\tt ia.},
e.g.
\small
\begin{verbatim}
CASA <5>: ia.
Display all 105 possibilities? (y or n)
ia.__class__                ia.deconvolvecomponentlist  ia.ispersistent             ia.reorder
ia.__delattr__              ia.deconvolvefrombeam       ia.lock                     ia.replacemaskedpixels
ia.__doc__                  ia.done                     ia.makearray                ia.restoringbeam
ia.__getattribute__         ia.echo                     ia.makecomplex              ia.rotate
ia.__hash__                 ia.fft                      ia.maketestimage            ia.sepconvolve
ia.__init__                 ia.findsources              ia.maskhandler              ia.set
ia.__new__                  ia.fitallprofiles           ia.maxfit                   ia.setboxregion
ia.__reduce__               ia.fitcomponents            ia.miscinfo                 ia.setbrightnessunit
ia.__reduce_ex__            ia.fitpolynomial            ia.modify                   ia.setcoordsys
ia.__repr__                 ia.fitprofile               ia.moments                  ia.sethistory
ia.__setattr__              ia.fromarray                ia.name                     ia.setmiscinfo
ia.__str__                  ia.fromascii                ia.newimage                 ia.setrestoringbeam
ia.adddegaxes               ia.fromfits                 ia.newimagefromarray        ia.shape
ia.addnoise                 ia.fromimage                ia.newimagefromfile         ia.statistics
ia.boundingbox              ia.fromrecord               ia.newimagefromfits         ia.subimage
ia.brightnessunit           ia.fromshape                ia.newimagefromimage        ia.summary
ia.calc                     ia.getchunk                 ia.newimagefromshape        ia.toASCII
ia.calcmask                 ia.getregion                ia.open                     ia.tofits
ia.close                    ia.getslice                 ia.outputvariant            ia.topixel
ia.collapse                 ia.hanning                  ia.pixelvalue               ia.torecord
ia.continuumsub             ia.haslock                  ia.putchunk                 ia.toworld
ia.convertflux              ia.histograms               ia.putregion                ia.twopointcorrelation
ia.convolve                 ia.history                  ia.rebin                    ia.type
ia.convolve2d               ia.imagecalc                ia.regrid                   ia.unlock
ia.coordmeasures            ia.imageconcat              ia.remove                   
ia.coordsys                 ia.insert                   ia.removefile               
ia.decompose                ia.isopen                   ia.rename              
\end{verbatim}
\normalsize

A common use of the {\tt ia} tool is to do region statistics on
an image.  The {\tt imhead} task has {\tt mode='stats'} to do
this quickly over the entire image cube.  The tool can do this
on specific planes or sub-regions.  For example, in the Jupiter
6cm example script (\S~\ref{section:scripts.jupiter}), 
the {\tt ia} tool is used to get on-source and off-source statistics
for regression:
\small
\begin{verbatim}
# The variable clnimage points to the clean image name

# Pull the max and rms from the clean image
ia.open(clnimage)
on_statistics=ia.statistics()
thistest_immax=on_statistics['max'][0]
oldtest_immax = 1.07732224464
print ' Clean image ON-SRC max should be ',oldtest_immax
print ' Found : Max in image = ',thistest_immax
diff_immax = abs((oldtest_immax-thistest_immax)/oldtest_immax)
print ' Difference (fractional) = ',diff_immax

print ''
# Now do stats in the lower right corner of the image
box = ia.setboxregion([0.75,0.00],[1.00,0.25],frac=true)
off_statistics=ia.statistics(region=box)
thistest_imrms=off_statistics['rms'][0]
oldtest_imrms = 0.0010449
print ' Clean image OFF-SRC rms should be ',oldtest_imrms
print ' Found : rms in image = ',thistest_imrms
diff_imrms = abs((oldtest_imrms-thistest_imrms)/oldtest_imrms)
print ' Difference (fractional) = ',diff_imrms

print ''
print ' Final Clean image Dynamic Range = ',thistest_immax/thistest_imrms
print ''
print ' =============== '

ia.close()

\end{verbatim}
\normalsize

Note: If you don't close the file with, e.g., {\tt ia.close()} the
file will stay in a 'locked' state. Other processes won't be able to
access the file until the file is properly closed. 

%%%%%%%%%%%%%%%%%%%%%%%%%%%%%%%%%%%%%%%%%%%%%%%%%%%%%%%%%%%%%%%%%
%%%%%%%%%%%%%%%%%%%%%%%%%%%%%%%%%%%%%%%%%%%%%%%%%%%%%%%%%%%%%%%%%
\section{Examples of CASA Image Analysis}
\label{section:analysis.examples}

See the scripts provided in Appendix~\ref{chapter:scripts} for examples of
data and image analysis.  In particular, we refer
the interested user to the demonstrations for:
\begin{itemize}
\item NGC5921 (VLA HI) --- a quick demo of basic CASA spectral line analysis
      (\ref{section:scripts.ngc5921})
\item Jupiter (VLA 6cm continuum polarimetry) --- polarization image analysis
      (\ref{section:scripts.jupiter})
\end{itemize}

%%%%%%%%%%%%%%%%%%%%%%%%%%%%%%%%%%%%%%%%%%%%%%%%%%%%%%%%%%%%%%%%%
%%%%%%%%%%%%%%%%%%%%%%%%%%%%%%%%%%%%%%%%%%%%%%%%%%%%%%%%%%%%%%%%%
