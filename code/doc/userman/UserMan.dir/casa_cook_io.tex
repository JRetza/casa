%%%%%%%%%%%%%%%%%%%%%%%%%%%%%%%%%%%%%%%%%%%%%%%%%%%%%%%%%%%%%%%%%
%%%%%%%%%%%%%%%%%%%%%%%%%%%%%%%%%%%%%%%%%%%%%%%%%%%%%%%%%%%%%%%%%
%%%%%%%%%%%%%%%%%%%%%%%%%%%%%%%%%%%%%%%%%%%%%%%%%%%%%%%%%%%%%%%%%

% STM 2007-04-13  split from previous version
% STM 2007-04-14  major rewriting to include ms, handling, export
% STM 2007-06-13  put common selectdata stuff here
% MPR 2007-07-05  minor tweaks, esp. to match frozen alpha-1 release of 4jul07
% STM 2007-09-20  pre-beta version
% STM 2007-10-09  Gustaaf's corrections
% STM 2007-10-10  beta version (spell-checked)
% STM 2007-11-09  beta 0.5, spw select issue
% STM 2008-04-29  note that post-Modcomp importvla is supported
% STM 2008-06-15  Patch 2.0, selection syntax
% STM 2009-05-19  Patch 4.0, vishead
% STM 2009-06-04  Patch 4.0, concat warning
% STM 2009-11-17  Release 3.0.0, importasdm
% STM 2009-12-11  Release 3.0.0 final
% JO 2010-03-10     edits for Release 3.0.1
% JO 2010-10-12 edits Release 3.1.0 
% STM 2011-05-03  Release 3.2.0 importevla and importasdm update

\chapter{Visibility Data Import, Export, and Selection}
\label{chapter:io} 

To use CASA to process your data, you first will need to get it
into a form that is understood by the package.  These are 
``measurement sets'' for synthesis (and single dish) data, and
``image tables'' for images.

There are a number of tasks used to fill telescope-specific data,
to import/export standard formats, to list data contents, and
to concatenate multiple datasets.  These are:
\begin{itemize}
   \item {\tt importasdm} --- import of ALMA data in 
         {\it ASDM} format (\S~\ref{section:io.import.asdm})
   \item {\tt importevla} --- import of EVLA data and flags in 
         {\it ASDM} format (\S~\ref{section:io.import.evla})
   \item {\tt importuvfits} --- import visibility data in UVFITS
         format (\S~\ref{section:io.import.uvfits.import})
   \item {\tt importfitsidi} --- import visibility data in the
     FITS-IDI format (\S~\ref{section:io.import.uvfitsidi.import})
   \item {\tt importvla} --- import data from VLA that is in 
         {\it export} format (\S~\ref{section:io.import.vla})
   \item {\tt exportuvfits} --- export a CASA MS in UVFITS format 
         (\S~\ref{section:io.import.uvfits.export})
   \item {\tt listobs} --- summarize the contents of a MS 
         (\S~\ref{section:io.list})
   \item {\tt listvis} --- list the data in a MS 
         (\S~\ref{section:io.vis.listvis})
   \item {\tt vishead} --- list and change the metadata contents of a MS 
         (\S~\ref{section:io.vis.vishead})
   \item {\tt visstat} --- statistics on data in a MS 
         (\S~\ref{section:io.vis.visstat})
   \item {\tt concat} --- concatenate two or more MS into a
         new MS (\S~\ref{section:io.concat})
\end{itemize}

In CASA, there is a standard syntax for selection of data that
is employed by multiple tasks.  
This is described in \S~\ref{section:io.selection}.

There are also tasks for the import and export of image data using
FITS:
\begin{itemize}
   \item {\tt importfits} --- import a FITS image into a CASA  
         {\it image} format table 
         (\S~\ref{section:analysis.fits})
   \item {\tt exportfits} --- export a CASA image table as FITS 
         (\S~\ref{section:analysis.fits})
\end{itemize}

%%%%%%%%%%%%%%%%%%%%%%%%%%%%%%%%%%%%%%%%%%%%%%%%%%%%%%%%%%%%%%%%%
%%%%%%%%%%%%%%%%%%%%%%%%%%%%%%%%%%%%%%%%%%%%%%%%%%%%%%%%%%%%%%%%%
\section{CASA Measurement Sets}
\label{section:io.ms}

Data is handled in CASA via the {\tt table} system.  In particular,
visibility data are stored in a CASA table known as a Measurement Set
(MS).  Details of the physical and logical MS structure are given
below, but for our purposes here an MS is just a construct that
contains the data.  An MS can also store single dish data (essentially
a set of auto-correlations of a 1-element interferometer), though
there are also data formats more suitable for single-dish spectra (see
\S~\ref{chapter:sd}).

\begin{wrapfigure}{r}{2.5in}
  \begin{boxedminipage}{2.5in}
     \centerline{\bf Inside the Toolkit:}
     Measurement sets are handled in the {\tt ms} tool.
     Import and export methods include {\tt ms.fromfits} and
     {\tt ms.tofits}.
  \end{boxedminipage}
\end{wrapfigure}

Note that images are handled through special
{\tt image} tables, although standard FITS I/O is also supported.
Images and {\tt image} data are described in a separate chapter.

Unless your data was previously processed by CASA or software based
upon its predecessor {\tt aips++}, you will need to import it into
CASA as an MS.  Supported formats include some ``standard'' flavors
of UVFITS, the VLA ``Export'' archive format, and most recently,
the ALMA Science Data Model (ASDM) format.  These are described below
in \S~\ref{section:io.import}.

Once in Measurement Set form, your data can be accessed through various tools
and tasks with a common interface.  The most important of these is the {\em
data selection interface} (\S~\ref{section:io.selection}) which allows you to
specify the subset of the data on which the tasks and tools will
operate.


%%%%%%%%%%%%%%%%%%%%%%%%%%%%%%%%%%%%%%%%%%%%%%%%%%%%%%%%%%%%%%%%%
\subsection{Under the Hood: Structure of the Measurement Set}
\label{section:io.ms.structure}

\begin{wrapfigure}{r}{2.5in}
  \begin{boxedminipage}{2.5in}
     \centerline{\bf Inside the Toolkit:}
     Generic CASA tables are handled in the {\tt tb} tool.
     You have direct access to keywords, rows and columns of
     the tables with the methods of this tool.
  \end{boxedminipage}
\end{wrapfigure}

It is not necessary that a casual CASA user know the specific details
on how the data in the MS is stored and the contents of all the
sub-tables.  However, we will occasionally refer to specific
``columns'' of the MS when describing the actions of various tasks,
and thus we provide the following synopsis to familiarize the user
with the necessary nomenclature.  You may skip ahead to subsequent
sections if you like!

All CASA data files, including Measurement Sets, are written into
the current working directory by default, with each CASA table
represented as a separate sub-directory.  MS names therefore need only
comply with UNIX file or directory naming conventions, and can be
referred to from within CASA directly, or via full path names.

An MS consists of a {\tt MAIN} table containing the visibility data. 
and associated sub-tables containing auxiliary or secondary
information.  The tables are logical constructs, with
contents located in the physical {\tt table.*} files on disk.  The
{\tt MAIN} table consists of the {\tt table.*} files in the main
directory of the ms-file itself, and the other tables are in the 
respective subdirectories.  The various MS tables and sub-tables can be seen by
listing the contents of the MS directory itself (e.g. using Unix {\tt
ls}), or via the {\tt browsetable} task (\S~\ref{section:edit.browse}).

See Fig~\ref{fig:tablekeyword} for an example of the contents of a MS
directory.  Or, from the casapy prompt,
\small
\begin{verbatim}
CASA <1>: ls ngc5921.ms
IPython system call: ls -F ngc5921.ms
ANTENNA           POLARIZATION     table.f1        table.f3_TSM1  table.f8
DATA_DESCRIPTION  PROCESSOR        table.f10       table.f4       table.f8_TSM1
FEED              SORTED_TABLE     table.f10_TSM1  table.f5       table.f9
FIELD             SOURCE           table.f11       table.f5_TSM1  table.f9_TSM1
FLAG_CMD          SPECTRAL_WINDOW  table.f11_TSM1  table.f6       table.info
HISTORY           STATE            table.f2        table.f6_TSM0  table.lock
OBSERVATION       table.dat        table.f2_TSM1   table.f7
POINTING          table.f0         table.f3        table.f7_TSM1
\end{verbatim}
\normalsize
Note that the {\tt MAIN} table information is contained in the
{\tt table.*} files in this directory.  Each of the sub-table
sub-directories contain their own {\tt table.dat} and other files, e.g.
\small
\begin{verbatim}
CASA <2>: ls ngc5921.ms/SOURCE
IPython system call: ls -F ngc5921.ms/SOURCE
table.dat  table.f0  table.f0i  table.info  table.lock
\end{verbatim}
\normalsize

\begin{figure}[h!]
%\gname{jbrowser_ms}{6}
\pngname{browser_ms}{6}
\caption{\label{fig:tablekeyword} The contents of a Measurement
  Set. These tables compose a Measurement Set named {\tt ngc5921.demo.ms} on
  disk.  This display is obtained by using the {\bf File:Open} menu in
  {\tt browsetable} and left double-clicking on the {\tt ngc5921.demo.ms}
  directory.}
\hrulefill
\end{figure}

Each ``row'' in a table contains entries for a number of specified 
``columns''.  For example, in the {\tt MAIN} table of the MS, the
original visibility data is contained in the {\tt  DATA} column ---
each ``cell'' contains a matrix of observed
complex visibilities for that row at a single time stamp, for a single
baseline in a single spectral window.  The shape of the data matrix is
given by the number of channels and the number of correlations
(voltage-products) formed by the correlator for an array.  

Table~\ref{tabselect} lists the non-data columns of the {\tt MAIN}
table that are most important during a typical data reduction session.
Table \ref{tabmain} lists the key data columns of the {\tt MAIN} table
of an interferometer MS.  The MS produced by fillers for specific
instruments may insert special columns, such as
{\tt ALMA\_PHASE\_CORR}, {\tt ALMA\_NO\_PHAS\_CORR} and 
{\tt ALMA\_PHAS\_CORR\_FLAG\_ROW} for ALMA data filled using the 
{\tt importasdm} filler (\S~\ref{section:io.import.asdm}).
These columns are visible in {\tt browsetable} and are accessible from
the toolkit in the {\tt ms} tool (e.g. the {\tt ms.getdata} method)
and from the {\tt tb} ``table'' tool (e.g. using {\tt tb.getcol}).

Note that when you examine table entries for IDs such as {\tt FIELD\_ID} or 
{\tt DATA\_DESC\_ID}, you will see 0-based numbers.

\vspace{5mm}
\begin{table}[h!]
\caption[Common MS Columns]
        {\label{tabselect} Common columns in the {\tt MAIN} table of
        the MS.}
\begin{center}
\begin{tabular}{|ll|} \hline
{\bf Parameter}   &   {\bf Contents}                  \\
\hline
\hline
ANTENNA1      &   First antenna in baseline   \\
ANTENNA2      &   Second antenna in baseline  \\
FIELD\_ID      &   Field (source no.) identification  \\
DATA\_DESC\_ID  &   Spectral window number, polarization identifier pair (IF no.)  \\
ARRAY\_ID      &   Subarray number                \\
OBSERVATION\_ID  &   Observation identification   \\
POLARIZATION\_ID  &   Polarization identification \\ 
SCAN\_NUMBER   &   Scan number                    \\
TIME          &   Integration midpoint time      \\
UVW           &   UVW coordinates                \\
\hline
\end{tabular}
\end{center}
\end{table}

The MS can contain a number of ``scratch'' columns, which are used to
hold useful versions of other columns such as the data or weights
for further processing.  The most common scratch columns are:
\begin{itemize}
\item {\tt CORRECTED\_DATA} --- used to hold calibrated data for
  imaging or display;
\item {\tt MODEL\_DATA} --- holds the Fourier inversion of a
  particular model image for calibration or imaging.
\end{itemize}
The creation and use of the scratch columns is generally done behind the scenes, but you
should be aware that they are there (and when they are used).

\vspace{5mm}
\begin{table}[h!]
\caption[Commonly accessed {\tt MAIN} Table columns]
        {\label{tabmain} Commonly accessed {\tt MAIN} Table
        data-related columns.
        Note that the columns {\tt ALMA\_PHASE\_CORR}, 
        {\tt ALMA\_NO\_PHAS\_CORR} and {\tt ALMA\_PHAS\_CORR\_FLAG\_ROW}
        are specific to ALMA data filled using the {\tt importasdm}
        filler.}
\begin{center}
\begin{tabular}{|llp{2.5in}|} \hline
  {\bf Column}      &  {\bf Format} & {\bf Contents} \\        
  DATA              &  Complex(N$_c$, N$_f$)  
     & complex visibility data matrix  (= ALMA\_PHASE\_CORR by default) \\
  FLAG             &  Bool(N$_c$, N$_f$) &  cumulative data flags    \\
  WEIGHT            &  Float(N$_c$)    &  weight for a row \\     
  WEIGHT\_SPECTRUM   &  Float(N$_c$, N$_f$) 
     &  individual weights for a data matrix \\
  ALMA\_PHASE\_CORR   &  Complex(N$_c$, N$_f$) 
     &  on-line phase corrected data  ({\it Not in VLA data})  \\
  ALMA\_NO\_PHAS\_CORR  &  Bool(N$_c$, N$_f$) 
     & data that has not been phase corrected  ({\it Not in VLA data})  \\
  ALMA\_PHAS\_CORR\_FLAG\_ROW  &  Bool(N$_c$, N$_f$)  
     & flag to use phase-corrected data or not ({\it not in VLA data}) \\    
  MODEL\_DATA       &  Complex(N$_c$, N$_f$)
     &  Scratch: created by calibrater or imager tools  \\
  CORRECTED\_DATA   &  Complex(N$_c$, N$_f$)    
     &  Scratch: created by calibrater or imager tools \\ 

\hline
\end{tabular}
\end{center}
\end{table}

The most recent specification for the MS
is {\bf Aips++ MeasurementSet definition version 2.0}
(\url{http://casa.nrao.edu/Memos/229.html}).

%%%%%%%%%%%%%%%%%%%%%%%%%%%%%%%%%%%%%%%%%%%%%%%%%%%%%%%%%%%%%%%%%
%%%%%%%%%%%%%%%%%%%%%%%%%%%%%%%%%%%%%%%%%%%%%%%%%%%%%%%%%%%%%%%%%
\section{Data Import and Export}
\label{section:io.import}

There are a number of tasks available to bring data in various forms
into CASA as a Measurement Set:

\begin{itemize}

\item ALMA Science Data Model format data can be imported
      into CASA ({\tt importasdm})

\item EVLA Science Data Model format data, including online flags,
      can imported into CASA ({\tt importevla})

\item VLA Archive format data can be imported into CASA
      ({\tt importvla})

\item UVFITS format can be imported into and exported from CASA
      ({\tt importuvfits}, {\tt importfitsidi}, and {\tt exportuvfits})

\end{itemize}

%%%%%%%%%%%%%%%%%%%%%%%%%%%%%%%%%%%%%%%%%%%%%%%%%%%%%%%%%%%%%%%%%
\subsection{ALMA: Filling of Science Data Model (ASDM) data}
\label{section:io.import.asdm}

\begin{wrapfigure}{r}{2.5in}
  \begin{boxedminipage}{2.5in}
     \centerline{\underline{\bf Under the Hood:}}
     The {\tt importasdm} task is just an interface to the stand-alone
     {\tt asdm2MS} application.  To find out the command-line
     arguments to this application, do {\tt asdm2MS --help}.
  \end{boxedminipage}
\end{wrapfigure}

The ALMA and EVLA projects have agreed upon a common archival science
data model (ASDM) format, and have jointly developed the software to
fill this data into CASA.  In the ASDM format, the bulk of the data is
contained in large binary data format (BDF) tables, with the meta-data
and ancillary information in XML tables.  This is structured as a
directory, like the MS, and was designed to be similar to the MS to 
facilitate conversion.

The {\bf importasdm} task will fill SDM1 format data (Oct 2009 or
later) into a CASA visibility data set (MS).  The default inputs are:

\small
\begin{verbatim}
#  importasdm :: Convert an Science Data Model observation into a CASA visibility file
asdm                =     ''   #  Name of input asdm directory (on disk)
vis                 =     ''   #  Root name of the ms to be created. Note the .ms is NOT added
singledish          =  False   #  Set true to output single-dish data format
corr_mode           =  'all'   #  correlation mode to be input (ao, co, ac, all)
srt                 =  'all'   #  spectral resolution type (fr, ca, bw, or all)
time_sampling       =  'all'   #  time sampling (INTEGRATION and/or SUBINTEGRATION) (i, si, or all)
ocorr_mode          =   'ca'   #  output correlation mode AUTO_ONLY (ao), CROSS_ONLY (co), CROSS_AND_AUTO (ca)
compression         =  False   #  Flag for turning on data compression
asis                =     ''   #  string of table names separated by spaces to be passed verbatim (* allowed)
wvr_corrected_data  =   'no'   #  'no' (default), 'yes'= WVR corrected, 'both'= output separate MS
scans               =     ''   #  processes only the specified scans (see help importasdm)
ignore_time         =  False   #  rows of tables processed independently of the time range of exec blocks
process_syspower    =  False   #  The SysPower table is processed if and only if this parameter is set to true.
process_caldevice   =  False   #  The CalDevice table is processed if and only if this parameter is set to true.
verbose             =  False   #  Output lots of information while the filler is working
overwrite           =  False   #  Over write an existing MS
showversion         =  False   #  Report the version of asdm2MS being used
async               =  False   #  If true the taskname must be started using importasdm(...)
\end{verbatim}
\normalsize

{\bf ALERT}: As of release 3.2, there are now options for filling
selected scans. This is through the {\tt scans} parameter. 

If {\tt scans} is set, then {\tt importasdm} processes only the scans
specified in the option's value. This value is a semicolon separated
list of scan specifications. A scan specification consists in an exec
bock index followed by the character {\tt ':'} followed by a comma separated
list of scan indexes or scan index ranges. A scan index is relative to
the exec block it belongs to. Scan indexes are 1-based while exec
blocks's are 0-based. The expressions
\begin{verbatim}
         "0:1"
         "2:2~6"
         "0:1,1:2~6,8;2:,3:24~30"
         "1,2"
         "3:"
\end{verbatim}
are all valid values for the selection. The {\tt "3:"} selector will be interpreted
as {\it 'all the scans of the exec block 3'}. An scan index or a scan index
range not preceded by an exec block index will be interpreted as {\it 'all
the scans with such indexes in all the exec blocks'}.  By default all
the scans are considered.

If {\tt singledish=True}, output data format is scantable (single-dish data
format, see \ref{chapter:sd}) instead of MS. In that case, you must specify
name or id of the antenna that you want to obtain data. This can be
done by using antenna parameter that is defined as a subparameter of
singledish. For single-dish mode, only auto-correlation data are
filled, i.e. {\tt ocorr\_mode} is forcibly set to 'ao'. 

%%%%%%
\subsubsection{Filling SDM v0 data using {\tt importoldasdm}}
\label{section:io.import.asdm.old}

If you are filling old test data taken
before October 2009 in the SDM0 format, then you should use the 
{\tt importoldasdm} task to fill this.
\small
\begin{verbatim}
#  importoldasdm :: Convert a Science Data Model v0 observation into a CASA visibility file
asdm            =      ''   #  Name of input asdm directory (on disk)
corr_mode       =   'all'   #  correlation mode to be considered on input
srt             =   'all'   #  specifies the spectral resolution type to be considered on input
time_sampling   =   'all'   #  time sampling (INTEGRATION and/or SUBINTEGRATION)
ocorr_mode      =    'co'   #  AUTO_ONLY (ao) or CROSS_ONLY (co) or CROSS_AND_AUTO (ca)
compression     =   False   #  Flag for turning on data compression
async           =   False   #  If true the taskname must be started using importoldasdm(...)
\end{verbatim}
\normalsize

%%%%%%%%%%%%%%%%%%%%%%%%%%%%%%%%%%%%%%%%%%%%%%%%%%%%%%%%%%%%%%%%%
\subsection{EVLA: Filling of Science Data Model (ASDM) data}
\label{section:io.import.evla}

\begin{wrapfigure}{r}{2.5in}
  \begin{boxedminipage}{2.5in}
     \centerline{\underline{\bf Under the Hood:}}
     The {\tt importevla} task is a modified version of the {\tt
     importasdm} task, that includes import of online flags from 
     the {\tt Flag.xml} table into the {\tt FLAG\_CMD} MS table,
     and a streamlined set of parameters.
  \end{boxedminipage}
\end{wrapfigure}

The {\bf importevla} task will fill SDM data from EVLA (or ALMA) into
a MS, along with online flagging data contained in the {\tt Flag.xml}
SDM table.  Otherwise, it behaves as {\tt importasdm} but with a
streamlined parameter set.

The default inputs are:

\small
\begin{verbatim}
#  importevla :: Convert a Science Data Model observation into a CASA Measurement Set
asdm           =      ''  #  Name of input asdm directory (on disk)
vis            =      ''  #  Root name of the ms to be created. Note the .ms is NOT added
ocorr_mode     =    'ca'  #  Fill correlation mode AUTO_ONLY (ao), CROSS_ONLY (co) or CROSS_AND_AUTO (ca)
compression    =   False  #  Flag for turning on data compression
asis           =      ''  #  Create verbatim copies of these SDM tables in the MS.
scans          =      ''  #  List of scans to fill (default is all scans).
switchedpower  =   False  #  Enables support for switched-power gain and tsys corrections
verbose        =   False  #  Output lots of information while the filler is working
overwrite      =   False  #  Over write an existing MS
online         =    True  #  Create online flags
     tbuff     =     1.0  #  time padding buffer (seconds)

flagzero       =    True  #  Create flag for low amplitude/zero points
     cliplevel =   1e-08  #  zero clipping level
     flagpol   =    True  #  flag cross-hand corrs also?

shadow         =    True  #  Create flag for shadowed data
     diameter  =    -1.0  #  diameter to use for shadowing (-1=antenna dia.)

applyflags     =   False  #  Apply flags to MS
flagbackup     =    True  #  Back up flag column at start
async          =   False  #  If true the taskname must be started using importevla(...)
\end{verbatim}
\normalsize

{\bf ALERT}: If you want to use your EVLA online flags then you must
use {\tt importevla} rather than {\tt importasdm}.  The {\tt flagcmd} 
task will process these flags.

The default action of {\bf importevla} is to construct the {\tt FLAG\_CMD} 
MS table based on the settings of {\tt online}, {\tt flagzero}, and
{\tt shadow} (and sub-parameters).  If {\tt applyflags=True} then these
flags will be applied after filling.  We recommend you use the {\tt flagcmd}
task after filling to examine these flags and then apply.

See {\tt importasdm} (\S~\ref{section:io.import.asdm}) for a
description of the common parameters.  Some differences:

Note that {\it importevla} uses the {\tt switchedpower} toggle to combine the
action of the two related parameters that {\tt importasdm} uses.

The {\tt online} parameter controls creation of online flags from the 
{\tt Flag.xml} SDM table. The {\tt tbuff} parameter adds a time ``buffer''
padding for these flags in both directions to deal with timing mis-matches.
{\bf ALERT}: For EVLA data taken before April 2011, you should set {\tt tbuff}
to a value (in seconds) equal to $1.5\times$ the integration time.

The {\tt flagzero} parameter controls creation of clipping commands to flag
visibilities with amplitudes below the cutoff given by {\tt cliplevel}. If
{\tt flagpol=True} then it will flag the cross-hands (e.g. RL and LR) as well,
which might result in low but correct values of these correlations being thrown
out (but can catch erroneous zeros also).  
{\bf ALERT}: This facility is provided as the EVLA correlator, particularly in
2010, occasionally produces visibilities with zero or very small values that
need to get flagged out.

The {\tt shadow} parameter turns on creation of flag commands to remove antenna
timeranges where they are shadowed by other antennas in the array.  By default
it will flag based on the antenna diameter, but if you want more lenient or 
conservative flagging then set the {\tt diameter} sub-parameter to a value 
smaller or larger respectively than the actual diameter.
{\bf ALERT}: Note that the shadow flagging calculation will only be able to
detect shadowing by antennas for which there is actual visibility data in
the MS.

%%%%%%%%%%%%%%%%%%%%%%%%%%%%%%%%%%%%%%%%%%%%%%%%%%%%%%%%%%%%%%%%%
\subsection{VLA: Filling data from archive format ({\tt importvla})}
\label{section:io.import.vla}

VLA data in archive format (i.e., as downloaded from the VLA data archive)
are read into CASA from disk using the {\tt importvla} task.  The inputs are:
\small
\begin{verbatim}
#  importvla :: import VLA archive file(s) to a measurement set:

archivefiles  =         ''   #  Name of input VLA archive file(s)
vis           =         ''   #  Name of output visibility file
bandname      =         ''   #  VLA frequency band name:''=>obtain all bands in archive files
frequencytol  =   150000.0   #  Frequency shift to define a unique spectral window (Hz)
project       =         ''   #  Project name:  '' => all projects in file
starttime     =         ''   #  start time to search for data
stoptime      =         ''   #  end time to search for data
applytsys     =       True   #  apply nominal sensitivity scaling to data & weights
autocorr      =      False   #  import autocorrelations to ms, if set to True
antnamescheme =      'new'   #   'old' or 'new'; 'VA04' or '4' for ant 4
keepblanks    =      False   #  Fill scans with empty source names (e.g. tipping scans)?
evlabands     =      False   #  Use updated eVLA frequencies and bandwidths
async         =      False        
\end{verbatim}
\normalsize

The main parameters are {\tt archivefiles} to specify the input
VLA Archive format file names, and {\tt vis} to specify the output
MS name. 

{\bf ALERT:} The scaling of VLA data both before and after the
June 2007 Modcomp-turnoff is fully supported, based on the value of
{\tt applytsys}.  

The NRAO Archive is located at:
\begin{itemize}
  \item \url{https://archive.nrao.edu}
\end{itemize}
Note that {\tt archivefiles} takes a string or list of
strings, as there are often multiple files for a project in the
archive.

For example:
\small
\begin{verbatim}
   archivefiles = ['AP314_A950519.xp1','AP314_A950519.xp2']
   vis = 'NGC7538.ms'
\end{verbatim}
\normalsize

The {\tt importvla} task allows selection on the frequency band.
Suppose that you have 1.3\,cm line observations in K-band and you
have copied the archive data files {\tt AP314\_A95019.xp*} to your 
working directory and started {\tt casapy}.  Then,
\small
\begin{verbatim}
   default('importvla')
   archivefiles = ['AP314_A950519.xp1','AP314_A950519.xp2','AP314_A950519.xp3']
   vis = 'ngc7538.ms' 
   bandname = 'K' 
   frequencytol = 10e6
   importvla()
\end{verbatim}
\normalsize
If the data is located in a different directory on disk, then use the
full path name to specify each archive file, e.g.:
\small
\begin{verbatim}
archivefiles=['/home/rohir2/jmcmulli/ALMATST1/Data/N7538/AP314_A950519.xp1',\
     '/home/rohir2/jmcmulli/ALMATST1/Data/N7538/AP314_A950519.xp2',\
     '/home/rohir2/jmcmulli/ALMATST1/Data/N7538/AP314_A950519.xp3']
\end{verbatim}
\normalsize

{\bf Important Note:} {\tt importvla} will import the on-line flags
(from the VLA system) along with the data. Shadowed antennas will also
be flagged. The flags will be put in the {\tt MAIN} table and thus
available to subsequent tasks and tools.  If you wish to revert to
unflagged data, use {\tt flagmanager}
(\S~\ref{section:edit.flagmanager}) to save the flags (if you wish),
and then use {\tt flagdata} (\S~\ref{section:edit.flagdata}) with {\tt
  mode='manualflag'} and {\tt unflag=True} to toggle off the flags.

The other parameters are:

\subsubsection{Parameter {\tt applytsys} }
\label{section:io.import.vla.applytsys}

The {\tt applytys} parameter controls whether the nominal sensitivity
scaling (based on the measured TSYS, with the weights scaled
accordingly using the integration time) is
applied to the visibility amplitudes or not.  If {\tt True}, then it
will be scaled so as to be the same as AIPS {\tt FILLM}
(ie. approximately in deciJanskys).
Note that post-Modcomp data is in raw correlation coefficient and will
be scaled using the TSYS values, while Modcomp-era data had this
applied online.  
In all cases {\tt importvla} will do
the correct thing to data and weights based on an internal flag in the
VLA Archive file, either scaling it or unscaling based on your choice
for {\tt applytys}.

If {\tt applytsys=True} and you see strange behavior in data
amplitudes, it may be due to erroneous TSYS values from the online
system.  You might want to then fill with {\tt applytsys=False} 
and look at the correlation coefficients to see if the behavior is
as expected.

\subsubsection{Parameter {\tt bandname} }
\label{section:io.import.vla.bandname}

The {\tt bandname} indicates the VLA Frequency band(s) to load,
using the traditional bandname codes.  These are:
\begin{itemize}
  \item {\tt '4'} =   48-96 MHz
  \item {\tt 'P'} =  298-345 MHz
  \item {\tt 'L'} = 1.15-1.75 GHz
  \item {\tt 'C'} =  4.2-5.1 GHz
  \item {\tt 'X'} =  6.8-9.6 GHz
  \item {\tt 'U'} = 13.5-16.3 GHz
  \item {\tt 'K'} = 20.8-25.8 GHz
  \item {\tt 'Q'} =   38-51 GHz
  \item {\tt ''}  = all bands (default)
\end{itemize}

Note that as the transition from the VLA to EVLA progresses,
the actual frequency ranges covered by the bands will expand,
and additional bands will be added (namely {\tt 'S'} from 1-2 GHz and 
{\tt 'A'} from 26.4-40 GHz).

\subsubsection{Parameter {\tt frequencytol} }
\label{section:io.import.vla.frequencytol}

The {\tt frequencytol} parameter specifies the frequency separation
tolerated when assigning data to spectral windows.  The
default is {\tt frequencytol=150000} (Hz).  
For Doppler tracked data, where the sky frequency changes with time,
a {\tt frequencytol} $< 10000$ Hz may may produce too many
unnecessary spectral windows.

\subsubsection{Parameter {\tt project} }
\label{section:io.import.vla.project}

You can specify a specific {\tt project} name to import from 
archive files.  The default {\tt ''} will import data from all
projects in file(s) {\tt archivefiles}.

For example for VLA Project {\tt AL519}:
\small
\begin{verbatim}
   project = 'AL519'    # this will work
   project = 'al519'    # this will also work
\end{verbatim}
\normalsize
while {\tt project='AL0519'} will NOT work (even though that is what
queries to the VLA Archive will print it as - sorry!).

\subsubsection{Parameters {\tt starttime} and {\tt stoptime} }
\label{section:io.import.vla.starttime}

You can specify start and stop times for the data, e.g.:
\small
\begin{verbatim}
   starttime = '1970/1/31/00:00:00'
   stoptime = '2199/1/31/23:59:59'
\end{verbatim}
\normalsize
Note that the blank defaults will load all data fitting other criteria.

\subsubsection{Parameter {\tt autocorr} }
\label{section:io.import.vla.autocorr}

Note that autocorrelations are filled into the data set if {\tt
autocorr=True}.  Generally for the VLA, autocorrelation data is not
useful, and furthermore the imaging routine will try to image the
autocorrelation data (it assumes it is single dish data) which will
swamp any real signal.  Thus, if you do fill the autocorrelations, you
will have to flag them before imaging.

\subsubsection{Parameter {\tt antnamescheme} }
\label{section:io.import.vla.antnamescheme}

The {\tt antnamescheme} parameter controls whether {\tt importvla}
will try to use a naming scheme where EVLA antennas are prefixed
with {\tt EA} (e.g. {\tt 'EA16'}) and old VLA antennas have names
prefixed with {\tt VA} (e.g. {\tt 'VA11'}).  Our method to detect
whether an antenna is EVLA is not yet perfected, and thus unless you
require this feature, simply use {\tt antnamescheme='old'}.


\subsubsection{Parameter {\tt evlabands} }
\label{section:io.import.vla.evlabands}

The {\tt evlabands=True} option is provided to allow users to access
EVLA frequencies outside the standard VLA tuninings (e.g.\ the extended
C-band above 6~GHz).  {\bf ALERT:} use of this option for standard VLA
data will cause unexpected associations, such as X-band data below
8~GHz being extracted to C-band (as the EVLA C-band is 4--8~GHz).  Use
with care.

%%%%%%%%%%%%%%%%%%%%%%%%%%%%%%%%%%%%%%%%%%%%%%%%%%%%%%%%%%%%%%%%%
\subsection{UVFITS Import and Export}
\label{section:io.import.uvfits}

The UVFITS format is not exactly a standard, but is a popular
archive and transport format nonetheless.  CASA supports 
UVFITS files written by the AIPS {\tt FITTP} task, and others.

UVFITS is supported for both import and export.

%%%%%%%%%%%%%%%%%%%%%%%%%%%%%%%%%%%%%%%%%%%%%%%%%%%%%%%%%%%%%%%%%
\subsubsection{Import using {\tt importuvfits} }
\label{section:io.import.uvfits.import}

To import UVFITS format data into CASA, use the {\tt importuvfits} task: 
\small
\begin{verbatim}
CASA <1>: inp(importuvfits)
fitsfile            =         ''  # Name of input UVFITS file
vis                 =         ''  # Name of output visibility file (MS)
antnamescheme       =      'old'  # For VLA only; 'new' or 'old'; 'VA04' or '04' for VLA ant 4
async               =      False  # if True run in the background, prompt is freed
\end{verbatim}
\normalsize
This is straightforward, since all it does is read in a UVFITS file and
convert it as best it can into a MS.

For example:
\small
\begin{verbatim}
   importuvfits(fitsfile='NGC5921.fits',vis='ngc5921.ms')
\end{verbatim}
\normalsize

{\bf ALERT:} CARMA data can be loaded into CASA. However,
 
\begin{verbatim}
tb.open("c0104I/ANTENNA",nomodify=False)
namelist=tb.getcol("NAME").tolist()
for i in range(len(namelist)):
	name = 'CA'+namelist[i]
	print ' Changing '+namelist[i]+' to '+name
	namelist[i]=name
 
tb.putcol("NAME",namelist)
tb.close()
\end{verbatim}

%%%%%%%%%%%%%%%%%%%%%%%%%%%%%%%%%%%%%%%%%%%%%%%%%%%%%%%%%%%%%%%%%
\subsubsection{Import using {\tt importfitsidi} }
\label{section:io.import.uvfitsidi.import}

Some uvfits data is written in the FITS-IDI standard. Those files can
be imported into CASA with the {\tt importfitsidi} task: 
\small
\begin{verbatim}
#  importfitsidi :: Convert a FITS-IDI file to a CASA visibility data set
fitsidifile         =       ['']        #  Name(s) of input FITS-IDI file(s)
vis                 =         ''        #  Name of output visibility file (MS)
async               =      False        #  If true the taskname must be started using importfitsidi(...)
\end{verbatim}
\normalsize

Example:
\small
\begin{verbatim}
   importuvfits(fitsidifile='NGC1300.fits',vis='NGC1300.ms')
\end{verbatim}
\normalsize



%%%%%%%%%%%%%%%%%%%%%%%%%%%%%%%%%%%%%%%%%%%%%%%%%%%%%%%%%%%%%%%%%
\subsubsection{Export using {\tt exportuvfits} }
\label{section:io.import.uvfits.export}

The {\tt exportuvfits} task will take a MS and write it out in UVFITS
format.  The defaults are:
\small
\begin{verbatim}
#  exportuvfits :: Convert a CASA visibility data set (MS) to a UVFITS file

vis            =         ''    #   Name of input visibility file
fitsfile       =         ''    #   Name of output UVFITS file)
datacolumn     = 'corrected'   #   which data to write (data, corrected, model)
field          =         ''    #   Field name list
spw            =         ''    #   Spectral window and channel selection
antenna        =         ''    #   antenna list to select
time           =         ''    #   time range selection
nchan          =         -1    #   Number of channels to select
start          =          0    #   Start channel
width          =          1    #   Channel averaging width (value>1 indicates averaging)
writesyscal    =      False    #   Write GC and TY tables
multisource    =       True    #   Write in multi-source format
combinespw     =       True    #   Combine spectral windows (True for AIPS)
writestation   =       True    #   Write station name instead of antenna name
async          =      False    #   if True run in the background, prompt is freed
\end{verbatim}
\normalsize

For example:
\small
\begin{verbatim}
   exportuvfits(vis='ngc5921.split.ms',
                fitsfile='NGC5921.split.fits',
                multisource=False)
\end{verbatim}
\normalsize

The MS selection parameters {\tt field}, {\tt spw}, {\tt antenna}, 
and {\tt timerange} follow the standard selection syntax described
in \S~\ref{section:io.selection}.

{\bf ALERT:} The {\tt nchan}, {\tt start}, and {\tt width}
parameters will be superseded by channel selection in {\tt spw}.
Currently, there is a {\tt time} parameter rather than {\tt
  timerange}.

The {\tt datacolumn} parameter chooses which data-containing column
of the MS (see \S~\ref{section:io.ms.structure}) is to be written out
to the UV FITS file.  Choices are: {\tt 'data'}, {\tt 'corrected'}, 
and {\tt 'model'}.

There are a number of special parameters that control what is written
out.  These are mostly here for compatibility with AIPS. 

The {\tt writesyscal} parameter toggles whether {\tt GC} and 
{\tt TY} extension tables are written.  These are important for
VLBA data, and for EVLA data.  {\bf ALERT:} Not yet available.

The {\tt multisource} parameter determines whether the UV FITS file is
a multi-source file or a single-source file, if you have a
{\it single-source} MS or choose only a single source.  Note: the difference
between a single-source and multi-source UVFITS file here is whether
it has a source (SU) table and the source ID in the random parameters.
Some programs (i.e. difmap) only accept single-source files.  If you
select more than one source in fields, then the multisource parameter
will be overridden to be True regardless.

The {\tt combinespw} parameter allows, if some conditions are met,
exporting all of spectral windows (SpW) as a set of "IF"s in a single
"FREQID" setup instead of giving each SpW its own FREQID in the FITS
file.  In this context an IF (Intermediate Frequency) is a
specialization of an SpW, where each IF in a UV FITS file must have
the same number of channels and polarizations, each channel must have
the same width, and each IF must be present (even if flagged)
throughout the entire observation.  If these conditions are not met
the data must be exported using multiple FREQIDs, the UV FITS
equivalent of a general SpW.  This matters since many (sub)programs
will work with multiple IFs, but not multiple FREQIDs.  For example, a
UV FITS file with multiple FREQIDs can be read by AIPS, but you may
find that you have to separate the FREQIDs with {\tt SPLIT} before you
can do very much with them.  Therefore {\tt combinespw=True} should be
{\tt True} if possible.  Typically MSes where each band was observed
simultaneously can be exported with {\tt combinespw=True}.  MSes where the
tuning changed with time, e.g.\ 10 minutes at 4.8 GHz followed by 15
minutes at 8.4 GHz, should be exported to multiple UV FITS files using
spw to select one tuning (set of simultaneous SpWs) per file.



The {\tt multisource} parameter determines whether the UV FITS file is
a multi-source file or a single-source file, if you have a 
{\em single-source} MS or choose only a single source.  Note: the
difference between a single-source and multi-source UVFITS file here
is whether it has a source ({\tt SU}) table and the source ID in the
random parameters.  If you select more than one source in 
{\tt fields}, then the {\tt multisource} parameter will be overridden
to be {\tt True} regardless.

The {\tt combinespw} parameter allows combination of all spectral
windows at one time.  If {\tt True}, then all spectral windows must
have the same shape.  For AIPS to read an exported file, then set {\tt
combinespw=True}.

The {\tt writestation} parameter toggles the writing of the station
name instead of antenna name.

%%%%%%%%%%%%%%%%%%%%%%%%%%%%%%%%%%%%%%%%%%%%%%%%%%%%%%%%%%%%%%%%%
%%%%%%%%%%%%%%%%%%%%%%%%%%%%%%%%%%%%%%%%%%%%%%%%%%%%%%%%%%%%%%%%%
\subsection{Handling Measurement Set metadata and data}
\label{section:io.vis}

There are tasks provided for basic listing and manipulation of
Measurement Set data and metadata.  These include:
\begin{itemize}
   \item {\tt listobs} --- summarize the contents of a MS 
         (\S~\ref{section:io.list})
   \item {\tt vishead} --- list and change the metadata contents of a MS 
         (\S~\ref{section:io.vis.vishead})
   \item {\tt visstat} --- statistics on data in a MS 
         (\S~\ref{section:io.vis.visstat})
   \item {\tt concat} --- concatenate two or more MS into a
         new MS (\S~\ref{section:io.concat})
\end{itemize}

%%%%%%%%%%%%%%%%%%%%%%%%%%%%%%%%%%%%%%%%%%%%%%%%%%%%%%%%%%%%%%%%%
\subsection{Summarizing your MS ({\tt listobs})}
\label{section:io.list}

Once you import your data into a CASA Measurement Set, you can
get a summary of the MS contents with the {\tt listobs} task.

The inputs are:
\small
\begin{verbatim}
vis                 =         ''        #   Name of input visibility file (MS)
verbose             =       True        #   Extended summary list of data set in logger
\end{verbatim}
\normalsize

The summary will be written to the logger and to the {\tt casapy.log}
file.  For example,
\small
\begin{verbatim}
   listobs('n5921.ms')
\end{verbatim}
\normalsize
results in the logger messages:
\small
\begin{verbatim}
================================================================================
           MeasurementSet Name:  /Users/jott/casa/data/day2_TDEM0003_10s_norx      MS Version 2
================================================================================
   Observer: Mark J. Mark Claussen     Project: T.B.D.  
Observation: EVLA
Data records: 290218       Total integration time = 10016 seconds
   Observed from   26-Apr-2010/03:21:56.0   to   26-Apr-2010/06:08:52.0 (UTC)

   ObservationID = 0         ArrayID = 0
  Date        Timerange (UTC)          Scan  FldId FieldName nVis   Int(s)   SpwIds      ScanIntent
  26-Apr-2010/03:21:56.0 - 03:23:16.0     5      2 J0954+1743   2720   9.8      [0, 1]                      
                                             .......   
              05:31:21.0 - 05:35:57.5    48      3 IRC+10216    9918   9.7      [0, 1]                      
              05:36:19.0 - 05:37:26.5    49      2 J0954+1743   2700   9.2      [0, 1]                      
              05:37:50.0 - 05:42:27.0    50      3 IRC+10216    9918   9.73     [0, 1]                      
              05:42:49.0 - 05:43:56.0    51      2 J0954+1743   2700   9.18     [0, 1]                      
              05:44:19.0 - 05:48:55.5    52      3 IRC+10216    9882   9.75     [0, 1]                      
              05:49:17.0 - 05:50:24.5    53      2 J0954+1743   2700   9.22     [0, 1]                      
              05:55:46.0 - 06:00:53.0    56      5 J1229+0203   10836  9.69     [0, 1]                      
              06:06:14.0 - 06:08:52.0    59      7 J1331+3030   5814   9.52     [0, 1]                      
           (nVis = Total number of time/baseline visibilities per scan) 
Fields: 4
  ID   Code Name         RA            Decl           Epoch   SrcId nVis   
  2    D    J0954+1743   09:54:56.8236 +17.43.31.2224 J2000   2     65326  
  3    NONE IRC+10216    09:47:57.3820 +13.16.40.6600 J2000   3     208242 
  5    F    J1229+0203   12:29:06.6997 +02.03.08.5982 J2000   5     10836  
  7    E    J1331+3030   13:31:08.2880 +30.30.32.9589 J2000   7     5814   
   (nVis = Total number of time/baseline visibilities per field) 
Spectral Windows:  (2 unique spectral windows and 1 unique polarization setups)
 INFO    listobs::ms::summary+     SpwID  #Chans Frame Ch1(MHz)    ChanWid(kHz)TotBW(kHz)  Ref(MHz)    Corrs           
   0          64 TOPO  36387.2295  125         8000        36387.2295  RR  RL  LR  LL  
   1          64 TOPO  36304.542   125         8000        36304.542   RR  RL  LR  LL  
 Sources: 10
   ID   Name         SpwId RestFreq(MHz)  SysVel(km/s) 
   0    J1008+0730   0     0.03639232     -0.026       
   0    J1008+0730   1     0.03639232     -0.026       
   2    J0954+1743   0     0.03639232     -0.026       
   2    J0954+1743   1     0.03639232     -0.026       
   3    IRC+10216    0     0.03639232     -0.026       
   3    IRC+10216    1     0.03639232     -0.026       
   5    J1229+0203   0     0.03639232     -0.026       
   5    J1229+0203   1     0.03639232     -0.026       
   7    J1331+3030   0     0.03639232     -0.026       
   7    J1331+3030   1     0.03639232     -0.026       
 Antennas: 19:
   ID   Name  Station   Diam.    Long.         Lat.         
   0    ea01  W09       25.0 m   -107.37.25.2  +33.53.51.0  
   1    ea02  E02       25.0 m   -107.37.04.4  +33.54.01.1  
   2    ea03  E09       25.0 m   -107.36.45.1  +33.53.53.6  
   3    ea04  W01       25.0 m   -107.37.05.9  +33.54.00.5  
   4    ea05  W08       25.0 m   -107.37.21.6  +33.53.53.0  
   5    ea07  N06       25.0 m   -107.37.06.9  +33.54.10.3  
   6    ea08  N01       25.0 m   -107.37.06.0  +33.54.01.8  
   7    ea09  E06       25.0 m   -107.36.55.6  +33.53.57.7  
   8    ea12  E08       25.0 m   -107.36.48.9  +33.53.55.1  
   9    ea15  W06       25.0 m   -107.37.15.6  +33.53.56.4  
   10   ea19  W04       25.0 m   -107.37.10.8  +33.53.59.1  
   11   ea20  N05       25.0 m   -107.37.06.7  +33.54.08.0  
   12   ea21  E01       25.0 m   -107.37.05.7  +33.53.59.2  
   13   ea22  N04       25.0 m   -107.37.06.5  +33.54.06.1  
   14   ea23  E07       25.0 m   -107.36.52.4  +33.53.56.5  
   15   ea24  W05       25.0 m   -107.37.13.0  +33.53.57.8  
   16   ea25  N02       25.0 m   -107.37.06.2  +33.54.03.5  
   17   ea27  E03       25.0 m   -107.37.02.8  +33.54.00.5  
   18   ea28  N08       25.0 m   -107.37.07.5  +33.54.15.8 
\end{verbatim}
\normalsize
using the (default) {\tt verbose=True} option.
The most useful extra information that {\tt verbose=True} gives
is the list of the scans in the dataset.

%%%%%%%%%%%%%%%%%%%%%%%%%%%%%%%%%%%%%%%%%%%%%%%%%%%%%%%%%%%%%%%%%
\subsection{Listing MS data ({\tt listvis})}
\label{section:io.vis.listvis}

The {\tt listvis} task will print to the terminal (or file) listing
of the data in your MS.
The inputs are:
\small
\begin{verbatim}
#  listvis :: List measurement set visibilities.
vis          =         ''   #  Name of input visibility file
options      =       'ap'   #  List options: ap only
datacolumn   =     'data'   #  Column to list: data, corrected, model, residual
field        =         ''   #  Field names or index to be listed: ''==>all
spw          =        '*'   #  Spectral window:channels: '\*'==>all, spw='1:5~57'
selectdata   =      False   #  Other data selection parameters
average      =         ''   #  Averaging mode: ==>none (Not yet implemented)
showflags    =      False   #  Show flagged data (Not yet implemented)
pagerows     =         50   #  Rows per page
listfile     =         ''   #  Output file
async        =      False   #  If true the taskname must be started using listvis(...)
\end{verbatim}
\normalsize

For example:
\small
\begin{verbatim}
CASA <2>: listvis('ngc5921.demo.ms',datacolumn='corrected',field='0',selectdata=T,antenna='*&*',pagerows=10)
All selected data has FIELD = 0
All selected data has SPW = 0
Date/Time:                    RR:                 LL:                                                
1995/04/13/ Intrf UVDist Chn     Amp    Phs  Wt F    Amp    Phs  Wt F          U          V          W
------------|---|------|---|--------------------|-------------------|----------|----------|----------|
  09:19:00.0 1-2   1823   0:  7.388   17.4 100    3.518  131.9  81       386.69  130409.94      15.46
  09:19:00.0 1-2   1823   1: 11.913   -6.0 100   10.945   12.7  81       386.69  130409.94      15.46
  09:19:00.0 1-2   1823   2: 13.570    1.5 100   12.700    3.5  81       386.69  130409.94      15.46
  09:19:00.0 1-2   1823   3: 13.829   -0.6 100   13.479    2.2  81       386.69  130409.94      15.46
  09:19:00.0 1-2   1823   4: 14.082    1.1 100   14.201   -0.1  81       386.69  130409.94      15.46
  09:19:00.0 1-2   1823   5: 14.436    3.2 100   14.886    0.3  81       386.69  130409.94      15.46
  09:19:00.0 1-2   1823   6: 15.190   -1.2 100   14.977   -1.0  81       386.69  130409.94      15.46
  09:19:00.0 1-2   1823   7: 14.823    2.6 100   14.475    0.9  81       386.69  130409.94      15.46
  09:19:00.0 1-2   1823   8: 14.484    1.7 100   14.572    0.9  81       386.69  130409.94      15.46
  09:19:00.0 1-2   1823   9: 14.393    1.3 100   14.675    1.8  81       386.69  130409.94      15.46
Type Q to quit, A to toggle long/short list, or RETURN to continue [continue]: Q

\end{verbatim}
\normalsize

{\bf ALERT:} We are working on improving the format of the {\tt listvis} output.

%%%%%%%%%%%%%%%%%%%%%%%%%%%%%%%%%%%%%%%%%%%%%%%%%%%%%%%%%%%%%%%%%
\subsection{Listing and manipulating MS metadata ({\tt vishead})}
\label{section:io.vis.vishead}

The {\tt vishead} task is provided to access keyword information in
the Measurement Set.  The default inputs are:
\small
\begin{verbatim}
#  vishead :: List, get, and put metadata in a measurement set
vis            =          ''   #  Name of input visibility file
mode           =      'list'   #  options: list, summary, get, put
   listitems   =          []   #  items to list ([] for all)
async          =       False   #  
\end{verbatim}
\normalsize

The {\tt mode = 'summary'} option just gives the same output as
{\tt listobs}.

For {\tt mode = 'list'} the options are: {\tt 'telescope'}, 
{\tt 'observer'}, {\tt 'project'}, {\tt 'field'}, {\tt 'freq\_group\_name'},
{\tt 'spw\_name'}, {\tt 'schedule'}, {\tt 'schedule\_type'}, 
{\tt 'release\_date'}. 

\small
\begin{verbatim}
CASA <29>: vishead('ngc5921.demo.ms',mode='list',listitems=[])
  Out[29]: 
{'cal_grp': (array([-1, -1, -1], dtype=int32), {}),
 'field': (array(['1331+30500002_0', '1445+09900002_0', 'N5921_2'], 
      dtype='|S16'),
           {}),
 'fld_code': (array(['C', 'A', ''], 
      dtype='|S2'), {}),
 'freq_group_name': (array(['none'], 
      dtype='|S5'), {}),
 'log': ({'r1': False}, {}),
 'observer': (array(['TEST'], 
      dtype='|S5'), {}),
 'project': (array([''], 
      dtype='|S1'), {}),
 'ptcs': ({'r1': array([[[-2.74392758]],

       [[ 0.53248521]]]),
           'r2': array([[[-2.42044692]],

       [[ 0.17412604]]]),
           'r3': array([[[-2.26020138]],

       [[ 0.08843002]]])},
          {'MEASINFO': {'Ref': 'J2000', 'type': 'direction'},
           'QuantumUnits': array(['rad', 'rad'], 
      dtype='|S4')}),
 'release_date': (array([  4.30444800e+09]),
                  {'MEASINFO': {'Ref': 'TAI', 'type': 'epoch'},
                   'QuantumUnits': array(['s'], 
      dtype='|S2')}),
 'schedule': ({'r1': False}, {}),
 'schedule_type': (array([''], 
      dtype='|S1'), {}),
 'source_name': (array(['1331+30500002_0', '1445+09900002_0', 'N5921_2'], 
      dtype='|S16'),
                 {}),
 'spw_name': (array(['none'], 
      dtype='|S5'), {}),
 'telescope': (array(['VLA'], 
      dtype='|S4'), {})}
\end{verbatim}
\normalsize

You can use {\tt mode='get'} to retrieve the valuses of specific
keywords, and likewise {\tt mode='put'} to change them.
The inputs are:
\small
\begin{verbatim}
mode           =      'get'    #  options: list, summary, get, put
   hdkey       =         ''    #  keyword to get/put
   hdindex     =         ''    #  keyword index to get/put, counting from zero. ==>all
\end{verbatim}
\normalsize
and
\small
\begin{verbatim}
#  vishead :: List, summary, get, and put metadata in a measurement set
mode           =      'put'    #  options: list, summary, get, put
   hdkey       =         ''    #  keyword to get/put
   hdindex     =         ''    #  keyword index to get/put, counting from zero. ==>all
   hdvalue     =         ''    #  value of hdkey
\end{verbatim}
\normalsize
For example, a common operation is to change the Telescope name (e.g.\
if it is unrecognized), e.g.
\small
\begin{verbatim}
CASA <36>: vishead('ngc5921.demo.ms',mode='get',hdkey='telescope')
  Out[36]: 
(array(['VLA'], 
      dtype='|S4'), {})

CASA <37>: vishead('ngc5921.demo.ms',mode='put',hdkey='telescope',hdvalue='EVLA')

CASA <38>: vishead('ngc5921.demo.ms',mode='get',hdkey='telescope')
  Out[38]: 
(array(['EVLA'], 
      dtype='|S5'), {})
\end{verbatim}
\normalsize


\subsection{MS statistics ({\tt visstat})}
\label{section:io.vis.visstat}

{\tt ALERT:} This is still a prototype task.

The {\tt visstat} task is provided to obtain simple statistics for a
Measurement Set, useful in regression tests.

The inputs are:
\small
\begin{verbatim}
#  visstat :: Displays statistical information from a measurement set
vis              =      ''   #  Name of input visibility file
axis             =   'amp'   #  Which values to use
   datacolumn    =  'data'   #  Which data column to use (data, corrected, model)

useflags         =    True   #  Take flagging into account?
spw              =      ''   #  spectral-window/frequency/channel
field            =      ''   #  Field names or field index numbers: ''==>all, field='0~2,3C286'
selectdata       =    True   #  More data selection parameters (antenna, timerange etc)
   antenna       =      ''   #  antenna/baselines: ''==>all, antenna = '3,VA04'
   timerange     =      ''   #  time range: ''==>all, timerange='09:14:0~09:54:0'
   correlation   =      ''   #  Select data based on correlation
   scan          =      ''   #  scan numbers: ''==>all
   array         =      ''   #  (sub)array numbers: ''==>all
   uvrange       =      ''   #  uv range: ''==>all; uvrange = '0~100klambda', default units=meters

async            =   False   #  If true the taskname must be started using visstat(...)
\end{verbatim}
\normalsize
Running this task returns a record (Python dictionary) with the
statistics, which can be captured in a Python variable.
For example,
\small
\begin{verbatim}
CASA <42>: mystat = visstat('ngc5921.demo.ms',axis='amp',datacolumn='corrected',field='0')

CASA <43>: mystat
  Out[43]: 
{'CORRECTED': {'max': 51.938671112060547,
               'mean': 14.796444141750133,
               'medabsdevmed': 0.28020858764648438,
               'median': 14.764373779296875,
               'min': 0.81362706422805786,
               'npts': 514916.0,
               'quartile': 0.56053066253662109,
               'rms': 14.829294204711914,
               'stddev': 0.98650836609147285,
               'sum': 7618925.8316934109,
               'sumsq': 113234125.12642419,
               'var': 0.97319875636846753}}

CASA <44>: print mystat['CORRECTED']['stddev']
0.986508366091
\end{verbatim}
\normalsize

The options for {\tt axis} are:
\small
\begin{verbatim}
    axis='amplitude'      #    or ('amp')
    axis='phase'
    axis='imag'
    axis='scan_number'
    axis='flag'
\end{verbatim}
\normalsize
The phase of a complex number is in radians with range $(-\pi, \pi)$.

%%%%%%%%%%%%%%%%%%%%%%%%%%%%%%%%%%%%%%%%%%%%%%%%%%%%%%%%%%%%%%%%%
\subsection{Concatenating multiple datasets ({\tt concat})}
\label{section:io.concat}

Once you have your data in the form of CASA Measurement Sets, you
can go ahead and process your data using the editing, calibration,
and imaging tasks.  In some cases, you will most efficiently operate
on single MS for a particular session (such as calibration).  Other
tasks will (eventually) take multiple Measurement Sets as input.  For others,
it is easiest to combine your multiple data files into one.

If you need to combine multiple datasets, you can use the {\tt concat} task.
The default inputs are:
\small
\begin{verbatim}
#  concat :: Concatenate two or more visibility data sets.
vis         =    ['']   #  Name of input visibility files to be concatenated
concatvis   =      ''   #  Name of output visibility file
freqtol     =      ''   #  Frequency tolerance for considering data as the same spwid
dirtol      =      ''   #  Direction tolerance for considering data as the same field
timesort    =   False   #  If true, sort by TIME in ascending order
async       =   False   #  If true the taskname must be started using concat(...)
\end{verbatim}
\normalsize

The {\tt vis} parameter will take a list of one or more MS.  Usually,
this will contain all the MS to combine.

The {\tt concatvis} parameter contains the name of the output MS.
If this points to an existing file on disk, then the MS in
{\tt vis} will appended to it,  otherwise a new MS file
is created to contain the concatenated data.  Be careful here!

The {\tt timesort} parameter can be used to make sure the output
MS is in time order (e.g.\ if your input MS have concurrent times).
This can possibly speed up some subsequent calibration operations.

The parameters {\tt freqtol} and {\tt dirtol} control how close
together in frequency and angle on the sky spectral windows or field
locations need to be before calling them the same.

{\bf ALERT:} Note that if multiple frequencies or pointings are
combined using {\tt freqtol} or {\tt dirtol}, then the data are not
changed (ie. not rephased to the single phase center).  Use of these 
parameters is intended to be tolerant of small offsets (e.g. planets
tracked which move slightly in J2000 over the course of observations,
or combining epochs observed with slightly different positions).

For example:
\small
\begin{verbatim}
   default('concat')
   vis = ['n4826_16apr.split.ms','n4826_22apr.split.ms']
   concatvis = 'n4826_tboth.ms'
   freqtol = '50MHz'
   concat()
\end{verbatim}
\normalsize
combines the two days in {\tt 'n4826\_16apr.split.ms'} and
{\tt 'n4826\_22apr.split.ms'} into a new output MS called
{\tt 'n4826\_tboth.ms'}.

{\bf ALERT:} There are additional issues when concatenating
individual MS files with different spectral windows.  In this
case, the rest frequencies (if set) of these windows may be lost
in subsequent operations (e.g. {\tt split}).  If this occurs,
then you can either set the {\tt restfreq} explicitly in certain
tasks ({\tt plotxy} and {\tt clean}), or use the following snippet
of code to fix this up after {\tt concat} has run:
\small
\begin{verbatim}
   #
   ##########################################################################
   #
   # Fix up the MS after concat (NOTE: STILL NECESSARY IN 2.4 and 3.0)
   # This ensures that the rest freq will be found for all spws. 
   # print '--Fixing up spw rest frequencies in MS--'
   vis='ngc4826.tutorial.ms'              # Example vis name, put yours here!
   tb.open(vis+'/SOURCE',nomodify=false)
   spwid=tb.getcol('SPECTRAL_WINDOW_ID')
   # The following is for 64bit systems 08-Jul-2008. ok on 32bit also.
   spwid.setfield(-1,'int32')
   tb.putcol('SPECTRAL_WINDOW_ID',spwid)
   tb.close()
\end{verbatim}
\normalsize
For example, this is used in the NGC4826 BIMA mosaic script
(Appendix~\ref{section:scripts.ngc4826}).

%%%%%%%%%%%%%%%%%%%%%%%%%%%%%%%%%%%%%%%%%%%%%%%%%%%%%%%%%%%%%%%%%
%%%%%%%%%%%%%%%%%%%%%%%%%%%%%%%%%%%%%%%%%%%%%%%%%%%%%%%%%%%%%%%%%
\section{Data Selection}
\label{section:io.selection}

Once in MS form, subsets of the data can be operated on using the tasks and
tools.  In CASA, there are three common data selection parameters used in
the various
tasks: {\tt field}, {\tt spw}, and {\tt selectdata}.  In addition, the
{\tt selectdata} parameter, if set to {\tt True}, will open up a
number of other sub-parameters 
for selection.  The selection operation is unified across all the tasks.  The
available {\tt selectdata} parameters may not be the same in all tasks.  But if
present, the same parameters mean the same thing and behave in the
same manner when used in any task.

For example:
\small
\begin{verbatim}
field               =         ''        #   field names or index of calibrators ''==>all
spw                 =         ''        #   spectral window:channels: ''==>all
selectdata          =      False        #   Other data selection parameters
\end{verbatim}
\normalsize
versus
\small
\begin{verbatim}
field               =         ''        #   field names or index of calibrators ''==>all
spw                 =         ''        #   spectral window:channels: ''==>all
selectdata          =       True        #   Other data selection parameters
     timerange      =         ''        #   time range: ''==>all 
     uvrange        =         ''        #   uv range''=all
     antenna        =         ''        #   antenna/baselines: ''==>all
     scan           =         ''        #   scan numbers: Not yet implemented
     msselect       =         ''        #   Optional data selection (Specialized. but see help)

\end{verbatim}
\normalsize

The following are the general syntax rules and descriptions of the
individual selection parameters of particular interest for the tasks:

%%%%%%%%%%%%%%%%%%%%%%%%%%%%%%%%%%%%%%%%%%%%%%%%%%%%%%%%%%%%%%%%%
\subsection{General selection syntax}
\label{section:io.selection.syntax}

Most of the selections are effected through the use of selection
strings.  This sub-section describes the general rules used in
constructing and parsing these strings.  Note that some selections
are done though the use of numbers or lists.  There are also 
parameter-specific rules that are described under each parameter.

All lists of basic selection specification-units are comma separated
lists and can be of any length.  White-spaces before and after the
commas (e.g. {\tt '3C286, 3C48, 3C84'}) are ignored, while white-space 
within sub-strings is treated as part of the sub-string 
(e.g. {\tt '3C286, VIRGO A, 3C84'}).

All integers can be of any length (in terms of characters) composed of
the characters 0--9.  Floating point numbers can be in the standard
format ({\tt DIGIT.DIGIT}, {\tt DIGIT.}, or {\tt .DIGIT}) or in the 
mantissa-exponent format (e.g. {\tt 1.4e9}).  Places where only 
integers make sense (e.g. IDs), if a floating point number is given, 
only the integer part is used (it is truncated).

Range of numbers (integers or real numbers) can be given in the format
{\verb!'N0~N1'!}.  For integer ranges, it is expanded into a list of integers
starting from {\tt N0} (inclusive) to {\tt N1} (inclusive).  For real numbers, it
is used to select all values present for the appropriate parameter in
the Measurement Set between {\tt N0} and {\tt N1} (including the
boundaries).  Note that the {\verb!'~'!} character is used rather than the
more obvious {\tt '-'} in order to accommodate hyphens in strings and
minus signs in numbers.

Wherever appropriate, units can be specified.  The
units are used to convert the values given to the units used in the
Measurement Set.  For ranges, the unit is specified only once (at
the end) and applies to both the range boundaries.

\subsubsection{String Matching}
\label{section:io.selection.syntax.string}

String matching can be done in three ways. Any component of a comma
separated list that cannot be parsed as a number, a number range, or
a physical quantity is treated as a regular expression or a literal
string. If the string does not contain the characters {\tt '*'}, {\tt
'\{'}, {\tt '\}'} or {\tt '?'}, it is treated as a literal string and
used for exact matching. If any of the above mentioned characters are
part of the string, they are used as a regular expression. As a result,
for most cases, the user does not need to supply any special
delimiters for literal strings and/or regular expressions.  For
example:
\small
\begin{verbatim}
  field = '3'     # match field ID 3 and not select field named "3C286".

  field = '3*'    # used as a pattern and matched against field names. If
                  # names like "3C84", "3C286", "3020+2207" are found,
                  # all will match. Field ID 3 will not be selected
                  # (unless of course one of the above mentioned field
                  # names also correspond to field ID 3!). 

  field = '30*'   # will match only with "3020+2207" in above set. 
\end{verbatim}
\normalsize
However if it is required that the string be matched exclusively as a
regular expression, it can be supplied within a pair of {\tt '/'} as
delimiters (e.g. {\tt '/.+BAND.+/'}). A string enclosed within double
quotes ({\tt '"'}) is used exclusively for pattern matching (patterns
are a simplified form of regular expressions - used in most UNIX
commands for string matching). Patterns are internally converted to
equivalent regular expressions before matching. See the
Unix command {\tt "info regex"}, or visit
\url{http://www.regular-expressions.info}, for details of regular
expressions and patterns.

Strings can include any character except the following:
\small
\begin{verbatim}
     ','   ';'   '"'  '/'   NEWLINE
\end{verbatim}
\normalsize
(since these are part of the selection syntax).  
Strings that do not contain any of the characters used to construct
regular expressions or patterns are used for exact matches. Although
it is highly discouraged to have name in the MS containing the above
mentioned reserved characters, if one {\it does} choose to include the
reserved characters as parts of names etc., those names can only be
matched against quoted strings (since regular expression and patterns
are a super-set of literal strings -- i.e., a literal string is also a valid
regular expression). 

This leaves  
{\tt '"'}, {\tt '*'}, {\tt'\{'}, {\tt '\}'} or {\tt '?'}
as the list of printable character that cannot be part of a name
(i.e., a name containing this character can never be matched in a
MSSelection expression).  These will be treated as pattern-matching
even inside double double quotes ({\tt '" "'}).
There is currently no escape mechanism (e.g. via a backslash).

Some examples of strings, regular expressions, and patterns:
\begin{itemize}

\item The string {\tt 'LBAND'} will be used as a literal string for
      exact match.  It will match only the exact string {\tt LBAND}.

\item The wildcarded string {\tt '*BAND*'} will be used as a string pattern 
      for matching. This will match any string which has the
      sub-string {\tt BAND} in it. 

\item The string {\tt '"*BAND*"'} will also be used as a string
      pattern, matching any string which has the sub-string {\tt BAND} in it. 

\item The string {\tt '/.+BAND.+/'} will be used as a regular
      expression. This will also match any string which as the 
      sub-string {\tt BAND} in it. (the {\tt .+} regex operator 
      has the same meaning as the {\tt *} wildcard operator of patterns). 

\end{itemize}

%%%%%%%%%%%%%%%%%%%%%%%%%%%%%%%%%%%%%%%%%%%%%%%%%%%%%%%%%%%%%%%%%
\subsection{The {\tt field} Parameter}
\label{section:io.selection.field}

The {\tt field} parameter is a string that specifies which field
names or ids will be processed in the task or tool.
The field selection expression consists of comma separated list of
field specifications inside the string.

Field specifications can be literal field names, regular expressions
or patterns (see \S~\ref{section:io.selection.syntax.string}).  
Those fields for which the entry in the {\tt NAME} column of
the {\tt FIELD} MS sub-table match the literal field name/regular
expression/pattern are selected.
If a field name/regular expression/pattern fails to match any field
name, the given name/regular expression/pattern are matched against
the field code.  If still no field is selected, an exception is thrown.

Field specifications can also be given by their integer IDs.  IDs can
be a single or a range of IDs.  Field ID selection can also be done as
a boolean expression.  For a field specification of the form {\tt '>ID'},
all field IDs greater than ID are selected.  Similarly for {\tt '<ID'} all
field IDs less than the ID are selected.

For example, if the MS has the following observations:
\small
\begin{verbatim}
MS summary:
==========
FIELDID   SPWID   NChan       Pol        NRows     Source Name
---------------------------------------------------------------
 0          0      127         RR        10260     0530+135
 1          0      127         RR        779139    05582+16320
 2          0      127         RR        296190    05309+13319
 3          0      127         RR        58266     0319+415
 4          0      127         RR        32994     1331+305
 5          1       1      RR,RL,LL,RR   23166     KTIP
\end{verbatim}
\normalsize
one might select
\small
\begin{verbatim}
  field = '0~2,KTIP'          # FIELDID 0,1,2 and field name KTIP
  field = '0530+135'          # field 0530+135
  field = '05*'               # fields 0530+135,05582+16320,05309+13319
\end{verbatim}
\normalsize

%%%%%%%%%%%%%%%%%%%%%%%%%%%%%%%%%%%%%%%%%%%%%%%%%%%%%%%%%%%%%%%%%
\subsection{The {\tt spw} Parameter}
\label{section:io.selection.spw}

The {\tt spw} parameter is a string that indicates the specific
spectral windows and the channels within them to be used in subsequent
processing. Spectral window selection ({\tt 'SPWSEL'}) can be given as
a spectral window integer ID, a list of integer IDs, a spectral window
name specified as a literal string (for exact match) or a regular
expression or pattern. 

The specification can be via frequency ranges or by indexes.  
A range of frequencies are
used to select all spectral windows which contain channels within the
given range. Frequencies can be specified with an optional unit ---
the default unit being {\tt Hz}.  Other common choices for radio and
mm/sub-mm data are {\tt kHz}, {\tt MHz}, and {\tt GHz}.  You will get
the entire spectral windows, not just the channels in the specified
range.  You will need to do channel selection (see below) to do that.

The {\tt spw} can also be selected via comparison for
integer IDs.  For example, {\tt '>ID'} will select all spectral
windows with ID greater than the specified value, while {\tt '<ID'}
will select those with ID lesser than the specified value.

{\bf ALERT:} In the current release, {\tt '<ID'} and
{\tt '>ID'} are {\em inclusive} with the {\tt ID} specified included
in the selection, e.g. {\tt spw='<2'} is equivalent to 
{\tt spw='0,1,2'} and not {\tt spw='0,1'} as was intended.  This will
be fixed in an upcoming release.

Spectral window selection using strings follows the standard rules:
\small
\begin{verbatim}
  spw = '1'                   # SPWID 1
  spw = '1,3,5'               # SPWID 1,3,5
  spw = '0~3'                 # SPWID 0,1,2,3
  spw = '0~3,5'               # SPWID 0,1,2,3 and 5
  spw = '<3,5'                # SPWID 0,1,2,3 and 5
  spw = '*'                   # All spectral windows
  spw = '1412~1415MHz'        # Spectral windows containing 1412-1415MHz
\end{verbatim}
\normalsize
%  spw = '<4,5'                # SPWID 0,1,2,3 and 5

% You may also specify the {\tt spw} selection as single integer or a list
% of integers, e.g.  
% \small
% \begin{verbatim}
%   spw = 1
%   spw = [0,1,2,3]
% \end{verbatim}
% \normalsize
% In this mode of operation, no channelization can be specified and all
% channels in the chosen spectral windows will be used.

In some cases, the spectral windows may allow specification by name.
For example,
\small
\begin{verbatim}
  spw = '3mmUSB, 3mmLSB'      # choose by names (if available)
\end{verbatim}
\normalsize
might be meaningful for the dataset in question.

Note that the order in which multiple {\tt spw}s are given may be
important for other parameters.
For example, the {\tt mode = 'channel'} in {\tt clean} uses the
first {\tt spw} as the origin for the channelization of the resulting
image cube.

%%%%%%%%%%%%%%%%%%%%%%%%%%%%%%%%%%%%%%%%%%%%%%%%%%%%%%%%%%%%%%%%%
\subsubsection{Channel selection in the {\tt spw} parameter}
\label{section:io.selection.spw.channel}

\begin{wrapfigure}{r}{2.5in}
  \begin{boxedminipage}{2.5in}
     \centerline{\underline{\bf ALERT!}}
     Not all options are available yet, such as
     percentages or velocities. Stay tuned!
  \end{boxedminipage}
\end{wrapfigure}
Channel selection can be included in the {\tt spw} string in the form
{\tt 'SPWSEL:CHANSEL'} where {\tt CHANSEL} is the channel selector.
In the end, the spectral selection within a given spectral window
comes down to the selection of specific channels.  We provide a number
of shorthand selection options for this.  These {\tt CHANSEL} 
options include:
\begin{itemize}
\item {\it Channel ranges:} {\verb!'START~STOP'!}
\item {\it Frequency ranges:} {\verb!'FSTART~FSTOP'!}
\item {\it Velocity ranges:} {\verb!'VSTART~VSTOP'!} ({\bf not yet available})
\item {\it Bandwidth percentages:} {\verb!'PSTART~PSTOP'!} or {\verb!'PWIDTH'!}
           ({\bf not yet available})
\item {\it Channel striding/stepping:} {\verb!'START~STOP^STEP'!} or 
           {\verb!'FSTART~FSTOP^FSTEP'!}
\end{itemize}

The most common selection is via channel ranges
{\verb!'START~STOP'!} or frequency ranges {\verb!'FSTART~FSTOP'!}:
\small
\begin{verbatim}
  spw = '0:13~53'             # spw 0, channels 13-53, inclusive
  spw = '0:1413~1414MHz'      # spw 0, 1413-1414MHz section only
\end{verbatim}
\normalsize
All ranges are inclusive, with the channel given by, or containing
the frequency or velocity given by, {\tt START} and {\tt STOP} plus
all channels between included in the selection.  You can also select
the spectral window via frequency ranges {\verb!'FSTART~FSTOP'!}, 
as described above:
\small
\begin{verbatim}
  spw = '1413~1414MHz:1413~1414MHz'    # channels falling within 1413~1414MHz
  spw = '*:1413~1414MHz'               # does the same thing
\end{verbatim}
\normalsize

You can also specify multiple spectral window or channel ranges, e.g.
\small
\begin{verbatim}
  spw = '2:16, 3:32~34'       # spw 2, channel 16 plus spw 3 channels 32-34
  spw = '2:1~3;57~63'         # spw 2, channels 1-3 and 57-63
  spw = '1~3:10~20'           # spw 1-3, channels 10-20
  spw = '*:4~56'              # all spw, channels 4-56
\end{verbatim}
\normalsize
Note the use of the wildcard in the last example.

% Channel range may be given also in percentage of the total bandwidth
% in a given spectral window:
% \small
% \begin{verbatim}
%   spw = '0:10%~95%'           # spw 0, skip 10% of beginning chans and 5% of end
%   spw = '1:50%'               # the inner 50% of spw 1
%   spw = ':80%'                # the inner 80% of all spw
% \end{verbatim}
% \normalsize
% Note that a single percentage {\tt ':X\%'} means the inner {\tt X}\%
% centered upon the band center.

A step can be also be included using {\verb!'^STEP'!} as a postfix:
\small
\begin{verbatim}
  spw = '0:10~100^2'          # chans 10,12,14,...,100 of spw 0
  spw = ':^4'                 # chans 0,4,8,... of all spw
  spw = ':100~150GHz^10GHz'   # closest chans to 100,110,...,150GHz
\end{verbatim}
\normalsize
A step in frequency or velocity will pick the channel in which that
frequency or velocity falls, or the nearest channel.

%%%%%%%%%%%%%%%%%%%%%%%%%%%%%%%%%%%%%%%%%%%%%%%%%%%%%%%%%%%%%%%%%
\subsection{The {\tt selectdata} Parameters}
\label{section:io.selection.selectdata}

The {\tt selectdata} parameter, if set to {\tt True}, will expand
the inputs to include a number of sub-parameters, given below and
in the individual task descriptions (if different).  If 
{\tt selectdata = False}, then the sub-parameters are treated as
blank for selection by the task.  The default for {\tt selectdata} 
is {\tt False}.

The common {\tt selectdata} expanded sub-parameters are:

%%%%%%%%%%%%%%%%%%%%%%%%%%%%%%%%%%%%%%%%%%%%%%%%%%%%%%%%%%%%%%%%%
\subsubsection{The {\tt antenna} Parameter}
\label{section:io.selection.selectdata.antenna}

The {\tt antenna} selection string is a semi-colon ({\tt ';'}) 
separated list of baseline specifications. 
A baseline specification is of the form:
\begin{itemize}

\item {\tt 'ANT1'} --- Select all baselines including the antenna(s)
    specified by the selector {\tt ANT1}.

\item {\tt 'ANT1\&'} --- Select only baselines between the antennas
     specified by the selector {\tt ANT1}.

\item {\tt 'ANT1\&ANT2'} --- Select only the cross-correlation
     baselines between the
     antennas specified by selector {\tt ANT1} and antennas specified
     by selector {\tt ANT2}.  Thus {\tt 'ANT1\&'} is an abbreviation for
     {\tt 'ANT1\&ANT1'}.

\item {\tt 'ANT1\&\&ANT2'} --- Select only auto-correlation and
     cross-correlation baselines between antennas specified by
     the selectors {\tt ANT1} and {\tt ANT2}.  Note that this
     is what the default {\tt antenna=''} gives you.

\item {\tt 'ANT1\&\&\&'} --- Select only autocorrelations specified by
     the selector {\tt ANT1}. 

\end{itemize}
The selectors {\tt ANT1} and {\tt ANT2} are comma-separated lists of
antenna integer-IDs or literal antenna names, patterns, or regular
expressions.  The {\tt ANT} strings are parsed and converted to a list
of antenna integer-IDs or IDs of antennas whose name match the given
names/pattern/regular expression.  Baselines corresponding to all
combinations of the elements in lists on either side of ampersand are
selected.

Integer IDs can be specified as single values or a range of integers.
When items of the list are parsed as literal strings or regular
expressions or patterns (see \S~\ref{section:io.selection.syntax} 
for more details on strings).  All antenna names that match the given 
string (exact match)/regular expression/pattern are selected.

%Added STM 2009-11-23
{\bf ALERT:} Just for antenna selection, a user supplied integer (or
integer list) is converted to a string and matched against the antenna
name. If that fails, the normal logic of using an integer as an
integer and matching it with antenna index is done.  Note that
currently there is no method for specifying a pure index (e.g.\ a
number that will not first be checked against the name).

The comma is used only as a separator for the list of antenna
specifications.  The list of baselines specifications is a
semi-colon separated list, e.g.
\small
\begin{verbatim}
   antenna = '1~3 & 4~6 ; 10&11'
\end{verbatim}
\normalsize
will select baselines between antennas 1,2,3 and 4,5,6
({\tt '1\&4'}, {\tt '1\&5'}, \ldots, {\tt '3\&6'}) plus
baseline {\tt '10\&11'}.

The wildcard operator ({\tt '*'}) will be the most often used
pattern. To make it easy to use, the wildcard (and only this
operator) can be used without enclosing it in quotes.  
For example, the selection
\small
\begin{verbatim}
   antenna = 'VA*'
\end{verbatim}
\normalsize
will match all antenna names which have {\tt 'VA'} as the first 2
characters in the name (irrespective of what follows after these
characters).

There is also a negation operator ``{\tt !}'' that can be used to
de-select antennas or baselines.

Some examples:
\small
\begin{verbatim}
   antenna=''          # shows blank autocorr pages
   antenna='*&*'       # does not show the autocorrs
   antenna='*&&*'      # show both auto and cross-cor (default)
   antenna='*&&&'      # shows only autocorrs

   antenna='5&*'       # shows non-auto baselines with AN 5

   antenna='5,6&&&'    # AN 5 and 6 autocor
   antenna='5&&&;6&*'  # AN 5 autocor plus cross-cors to AN 6

   antenna='!5'        # baselines not involving AN 5
\end{verbatim}
\normalsize

{\bf Antenna numbers as names:}
Needless to say, naming antennas such that the names can also be
parsed as a valid token of the syntax is a bad idea.  Nevertheless,
antenna names that contain any of the reserved characters and/or can be
parsed as integers or integer ranges can still be used by enclosing
the antenna names in double quotes ({\verb!' "ANT" '!}).  E.g. the string
\small
\begin{verbatim}
   antenna = '10~15,21,VA22'
\end{verbatim}
\normalsize
will expand into an antenna ID list 10,11,12,13,14,15,21,22 (assuming
the index of the antenna named {\tt 'VA22'} is 22).  
%Added STM 2009-11-23
If, however, the antenna with ID index 50 is named {\tt '21'}, then
the string 
\small
\begin{verbatim}
   antenna = '10~15,21,VA22'
\end{verbatim}
\normalsize
will expand into an antenna ID list of 10,11,12,13,14,15,50,22.
{\em Keep in mind that numbers are FIRST matched against
names, and only against indices if that matching fails.}
There is currently no way to force a selection to use the index, and
if there an antenna with that name it will select that.

Read elsewhere (e.g. {\tt info regex} under Unix) for details of
regular expression and patterns.

%%%%%%%%%%%%%%%%%%%%%%%%%%%%%%%%%%%%%%%%%%%%%%%%%%%%%%%%%%%%%%%%%
\subsubsection{The {\tt scan} Parameter}
\label{section:io.selection.selectdata.scan}

The {\tt scan} parameter selects the scan ID numbers of the data.
There is currently no naming convention for scans.  The scan ID
is filled into the MS depending on how the data was obtained,
so use this with care.

Examples:
\small
\begin{verbatim}
  scan = '3'                      # scan number 3.
  scan = '1~8'                    # scan numbers 1 through 8, inclusive
  scan = '1,2,4,6'                # scans 1,2,4,6
  scan = '<9'                     # scans <9 (1-8)
\end{verbatim}
\normalsize

NOTE: ALMA and VLA/EVLA number scans starting with 1
and not 0.  You can see what the numbering is in your MS using
the {\tt listobs} task with {\tt verbose=True} (see
\S~\ref{section:io.list}).

%%%%%%%%%%%%%%%%%%%%%%%%%%%%%%%%%%%%%%%%%%%%%%%%%%%%%%%%%%%%%%%%%
\subsubsection{The {\tt timerange} Parameter}
\label{section:io.selection.selectdata.timerange}

The time strings in the following ({\tt T0}, {\tt T1} and {\tt dT}) 
can be specified as {\tt YYYY/MM/DD/HH:MM:SS.FF}.  
The time fields (i.e., {\tt YYYY}, {\tt MM}, {\tt DD}, {\tt HH}, 
{\tt MM}, {\tt SS} and {\tt FF}),
starting from left to right, may be omitted and they will be replaced
by context sensitive defaults as explained below.

Some examples:
\begin{enumerate}

\item {\verb!timerange='T0~T1'!}:   Select all time stamps from 
{\tt T0} to {\tt T1}.  For example:
\small
\begin{verbatim}
  timerange = '2007/10/09/00:40:00 ~ 2007/10/09/03:30:00'
\end{verbatim}
\normalsize

Note that fields missing in {\tt T0} are replaced by the fields in the
time stamp of the first valid row in the MS.  For example, 
\small
\begin{verbatim}
  timerange = '09/00:40:00 ~ 09/03:30:00'
\end{verbatim}
\normalsize
where the {\tt YY/MM/} part of the selection has been defaulted
to the start of the MS.  

Fields missing in {\tt T1}, such as
the date part of the string, are replaced by the corresponding fields
of {\tt T0} (after its defaults are set).  For example:
\small
\begin{verbatim}
  timerange = '2007/10/09/22:40:00 ~ 03:30:00'
\end{verbatim}
\normalsize
does the same thing as above.  

\item {\tt timerange='T0'}:  Select all time stamps that are within an
integration time of {\tt T0}.  For example,
\small
\begin{verbatim}
  timerange = '2007/10/09/23:41:00'
\end{verbatim}
\normalsize

Integration time is determined from the first valid row
(more rigorously, an average integration time should be computed).
Default settings for the missing fields of {\tt T0} are as in (1).

\item {\tt timerange='T0+dT'}:  Select all time stamps starting from
{\tt T0} and ending with time stamp {\tt T0+dT}.  For example,
\small
\begin{verbatim}
  timerange = '23:41:00+01:00:00'
\end{verbatim}
\normalsize
picks an hour-long chunk of time.

Defaults of {\tt T0} are set as usual.  Defaults for {\tt dT} are set from the
time corresponding to MJD=0.  Thus, {\tt dT} is a specification of length of
time from the assumed nominal "start of time".

\item {\tt timerange='>T0'}: Select all times greater than {\tt T0}.
For example,
\small
\begin{verbatim}
  timerange = '>2007/10/09/23:41:00'
  timerange = '>23:41:00'                # Same thing without day specification
\end{verbatim}
\normalsize
Default settings for {\tt T0} are as above.

\item {\tt timerange='<T1'}: Select all times less than {\tt T1}.
For example,
\small
\begin{verbatim}
  timerange = '<2007/10/09/23:41:00'
\end{verbatim}
\normalsize
Default settings for {\tt T1} are as above.

\end{enumerate}

An ultra-conservative selection might be:
\small
\begin{verbatim}
  timerange = '1960/01/01/00:00:00~2020/12/31/23:59:59'
\end{verbatim}
\normalsize
which would choose all possible data!

%%%%%%%%%%%%%%%%%%%%%%%%%%%%%%%%%%%%%%%%%%%%%%%%%%%%%%%%%%%%%%%%%
\subsubsection{The {\tt uvrange} Parameter}
\label{section:io.selection.selectdata.uvrange}

Rows in the MS can also be selected based on the uv-distance or
physical baseline length that the visibilities in each row correspond
to.  This {\tt uvrange} can be specified in various formats.

The basic building block of uv-distance specification is a valid
number with optional units in the format N[UNIT] (the unit in square
brackets is optional).  We refer to this basic building block as {\tt
UVDIST}.  The default unit is meter.  Units of length (such as {\tt
'm'} and {\tt 'km'}) select physical baseline distances (independent
of wavelength).  The other allowed units are in wavelengths (such as
{\tt 'lambda'}, {\tt 'klambda'} and {\tt 'Mlambda'} 
and are true uv-plane radii
\begin{equation}
  r_{uv} = \sqrt{u^2+v^2}.
\end{equation}
If only a single {\tt UVDIST} is specified, all rows, the uv-distance of which
exactly matches the given {\tt UVDIST}, are selected.

{\tt UVDIST} can be specified as a range in the format 
{\verb!'N0~N1[UNIT]'!} (where {\tt N0} and {\tt N1} are valid numbers).  All
rows corresponding to uv-distance between {\tt N0} and {\tt N1}
(inclusive) when converted the specified units are selected.

{\tt UVDIST} can also be selected via comparison operators.  When specified in
the format {\tt '>UVDIST'}, all visibilities with uv-distances greater than the
given {\tt UVDIST} are selected.  Likewise, when specified in the
format {\tt '<UVDIST'}, all rows with uv-distances less than the given
{\tt UVDIST} are selected.

% To selected data with uv-distances within an equal range about a given
% value, with the range specified as a percentage of the given value,
% the {\tt UVDIST} specification can be given in the format 
% {\tt 'UVDIST:N\%'}.  All data with uv-distance between +/- {\tt N} 
% percent of the given {\tt UVDIST} in appropriate units are selected.

Any number of above mentioned uv-distance specifications can be given
as a comma-separated list.

Examples:
\small
\begin{verbatim}
  uvrange = '100~200km'                     # an annulus in physical baseline length
  uvrange = '24~35Mlambda, 40~45Mlambda'    # two annuli in units of mega-wavelengths
  uvrange = '< 45klambda'                   # less than 45 kilolambda 
  uvrange = '> 0lambda'                     # greater than zero length (no auto-corrs)
  uvrange = '100km'                         # baselines of length 100km
  uvrange = '100klambda'                    # uv-radius 100 kilolambda
\end{verbatim}
\normalsize

%  uvrange = '31Mlambda:5%'                    # +/- 2.5% about 31Ml

%%%%%%%%%%%%%%%%%%%%%%%%%%%%%%%%%%%%%%%%%%%%%%%%%%%%%%%%%%%%%%%%%
\subsubsection{The {\tt msselect} Parameter}
\label{section:io.selection.selectdata.msselect}

More complicated selections within the MS structure are possible
using the Table Query Language (TaQL).  This is accessed through
the {\tt msselect} parameter.

Note that the TaQL syntax does not follow the rules given in 
\S~\ref{section:io.selection.syntax} for our other selection strings.
TaQL is explained in more detail in {\bf Aips++ NOTE 199 --- 
Table Query Language}
(\url{http://aips2.nrao.edu/docs/notes/199/199.html}).
This will eventually become a CASA document.  The specific 
columns of the MS are given in the most recent MS specification
document: {\bf Aips++ NOTE 229 --- MeasurementSet definition version 2.0}
(\url{http://aips2.nrao.edu/docs/notes/229/229.html}).
This documentation will eventually be updated to the CASA 
document system.

Most selection can be carried out using the other
selection parameters.  However, these are merely shortcuts
to the underlying TaQL selection.  For example, field
and spectral window selection can be done using {\tt msselect}
rather than through {\tt field} or {\tt spw}:
\small
\begin{verbatim}
  msselect='FIELD_ID == 0'                   # Field id 0 only
  msselect='FIELD_ID <= 1'                   # Field id 0 and 1
  msselect='FIELD_ID IN [1,2]'               # Field id 1 and 2
  msselect='FIELD_ID==0 && DATA_DESC_ID==3'  # Field id 0 in spw id 3 only
\end{verbatim}
\normalsize

{\bf ALERT:} The {\tt msselect} style parameters will be phased
out of the tasks.  TaQL selection will still be available in the
Toolkit.

%%%%%%%%%%%%%%%%%%%%%%%%%%%%%%%%%%%%%%%%%%%%%%%%%%%%%%%%%%%%%%%%%
