%%%%%%%%%%%%%%%%%%%%%%%%%%%%%%%%%%%%%%%%%%%%%%%%%%%%%%%%%%%%%%%%%
%%%%%%%%%%%%%%%%%%%%%%%%%%%%%%%%%%%%%%%%%%%%%%%%%%%%%%%%%%%%%%%%%
%%%%%%%%%%%%%%%%%%%%%%%%%%%%%%%%%%%%%%%%%%%%%%%%%%%%%%%%%%%%%%%%%

% JO  2010-10-12 doc created for CASA 3.1.0

\chapter[Appendix: Models, Conventions and Reference Frames]
         {Appendix: Models, Conventions, and Reference Frames}
\label{chapter:conventions}

This appendix lists the available parameters, conventions,
reference frames, and information on flux standards used in CASA.


%\section{Coordinate and Time Formatting Conventions}



\section{Flux Density Models for setjy}
%\label{sec-1}
\label{section:conventions.fluxdensity}

{\tt setjy} sets the {\tt MODEL\_DATA} column to what it calculates given the source,
frequency, a standard (really, a set of models), and possibly a time.  At cm
wavelengths the flux density (FD) calibrators are typically one of several
bright extragalactic sources.  These objects are comparatively faint and less
well characterized at shorter wavelengths, so for (sub)mm astronomy it is
common to use Solar System objects.

Reliably setting the FD scale with astronomical calibrators requires
that they be bright, not too resolved, and have simple dependencies on
frequency and time.  These criteria are somewhat mutually exclusive, so the
number of calibrator sources supported by {\tt setjy} is fairly small, although it
could certainly be added to.  This appendix is for describing the models that
{\tt setjy} uses.  Choosing a FD calibrator of course has to be done before the
observation and the observatory may provide additional information.

\subsection{Long wavelength calibration}
%\label{sec-1.1}
\label{section:conventions.longwavelength}

Synchrotron sources can vary over a light crossing time, so ones used as FD
calibrators must have most of their emission coming from an extended region.
The additional requirement that they be nearly unresolved therefore forces them
to be distant, meaning that candidates which also have high apparent fluxes are
quite rare.  The following standards mostly share the same set of objects, and
monitor their FDs every few years to account for variations.  No interpolation
is done between epochs, though - you are encouraged to choose the standard
which observed your FD calibrator closest to the time you observed it at.
The measurements \underline{are} interpolated in frequency, however, using second to
fourth degree polynomials of the frequency's logarithm.

\begin{table}[htb]
\caption{Extragalactic objects recognized by {\tt setjy}\label{table:conventions.fluxdensity.sources}}
\begin{center}
\begin{tabular}{ll}
 FD calibrator  &  Aliases                         \\
\hline
 3C286          &  1328+307, 1331+305, J1331+3030  \\
 3C48           &  0134+329, 0137+331, J0137+3309  \\
 3C147          &  0538+498, 0542+498, J0542+4951  \\
 3C138          &  0518+165, 0521+166, J0521+1638  \\
 1934-638       &                                  \\
 3C295          &  1409+524, 1411+522, J1411+5212  \\
 3C196          &  0809+483, 0813+482, J0813+4813  \\
\end{tabular}
\end{center}
\end{table}


\subsubsection{Baars}
\label{sec-1.1.1}
\label{section:conventions.longwavelength.baars}

The only standard to not have the year in the name.  It is \textbf{1977}.

The models are second order polynomials in log$\nu$, valid between 408 MHz and
15 GHz.

The paper is Baars et al. (1977)  (bibliography in
Sect.\,\ref{section:conventions.fluxreferences}) with a commentary by
Kellerman (2009).


\subsubsection{Perley 90}
\label{sec-1.1.2}
\label{section:conventions.longwavelength.perley90}

This standard also includes 1934-638 from Reynolds (7/94) and 3C138
from Baars et al. (1977).

Reference: {\it Website 1} in \ref{section:conventions.fluxreferences} 

\subsubsection{Perley-Taylor 95}
\label{sec-1.1.3}
\label{section:conventions.longwavelength.perleytaylor95}

Perley and Taylor (1995.2); plus Reynolds (1934-638; 7/94)

Reference: {\it Website 1}

\subsubsection{Perley-Taylor 99}
%\label{sec-1.1.4}
\label{section:conventions.longwavelength.perleytaylor99}

Perley and Taylor (1999.2); plus Reynolds (1934-638; 7/94)

Reference: {\it Website 1}
 
\subsubsection{Perley-Butler 2010}
%\label{sec-1.1.5}
\label{section:conventions.longwavelength.perleybutler10}

Made using VLA (not EVLA!) data.

Reference: {\it Website 1}

\subsection{Short wavelength calibration}
%\label{sec-1.2}
\label{section:conventions.shortwavelength}

The usual approach in this regime is to use (nearly) thermal sources in the
Solar System.  Their apparent brightness of course varies in time with their
distance from the Earth (and Sun), and orientation if they are not perfect
spheres with zero obliquity.  However, most of them have almost constant
surface properties, so once those properties are measured their apparent
brightness distributions can in principle be predicted for any time, given an
ephemeris.

In CASA all of the Solar System objects supported by {\tt setjy} are lumped
under one standard, `Butler-JPL-Horizons 2010', since none of them are covered
by more than one model yet.  The model uses a uniform temperature disk whose
semiaxes are set using ephemerides from the JPL-Horizons project.  All of the
objects are warm enough to put them in the Rayleigh-Jeans regime for ALMA, but
the FD as a function of frequency is calculated using the full Planck
equation.  Synchrotron emission (the gas giants) is accounted for by using
models that smoothly vary the brightness temperature with frequency.

\begin{table}[htb]
\caption{Notable Solar System objects\label{table:conventions.fluxdensity.solarsystem}}
\begin{center}
\begin{tabular}{lp{9cm}}
 Object         &  Notes                                                                                                                                            \\
\hline
 Mercury        &  Not yet included (phase angle)                                                                                                                   \\
 Venus          &  Not yet included (phase angle)                                                                                                                   \\
 Earth          &  Not yet included (very resolved)                                                                                                                 \\
 Mars           &  Taken to be 210K (JPL ephemeris)                                                                                                                 \\
 Jupiter        &  Model from B. Butler, $\lambda \in [0.1, 6.2]$ cm                                                                                                \\
 Ceres          &  167K (Saint-Pe et al 1993)                                                                                                 \\
 Pallas         &  164K (Not yet scaled for its varying distance from the Sun.  e = 0.231)                                                                          \\
 Juno           &  163K, but it has a large crater and temperature changes. (Lim et al. 2005)                                                   \\
 Io             &  110K (Rathbun et al. 2004)                                                                                               \\
 Europa         &  109K ({\it Website 2})                                                      \\
 Callisto       &  134K ($\pm 11$ K, Moore et al. 2004,)                                                  \\
 Ganymede       &  110K ({Delitsky et al, 1998, J.Geophys.Res. 103 (E13)})              \\
 Saturn         &  Not yet included (the rings, the rings)                                                                                                          \\
 Titan          &  76.6K (B. Butler)                                                                                                                                \\
 Uranus         &  Model from B. Butler, $\lambda \in [0.07, 6.2]$ cm                                                                                               \\
 Uranian moons  &  Not yet included (obliquity issues, esp. since the Voyager era was apx. 1 Uranian season ago.)                                                   \\
 Neptune        &  Model from B. Butler, $\nu \in [4.0, 1000.0]$ GHz (more refs in code comments)                                                                   \\
 Triton         &  38K ({\it Website 3})                                                                 \\
 Pluto          &  35K (Altenhoff et al. 1988 + more refs in code
 comments.)  {\tt setjy} does not check whether Charon was in the field.  \\
\end{tabular}
\end{center}
\end{table}


For most Solar System FD calibrators, the temperature reference will also be
sent to the logger if {\tt casalog.filter('INFO1')} (or lower) is run before
running {\tt setjy}.  If there is a discrepancy between the logger note and this
appendix, the logger note is more likely to be up to date.

\subsection{References to this Section}
\label{section:conventions.fluxreferences}

\noindent Altenhoff et al. 1988, A\&ALetter, 190, L15\\
\noindent Baars, J. W. M., Genzel, R., Pauliny-Toth, I. I. K., \&  Witzel,
A. 1977, A\&A, 61, 99\\
\noindent Delitsky et al. 1998, J.Geophys.Res. 103 (E13)\footnote{http://trs-new.jpl.nasa.gov/dspace/bitstream/2014/20675/1/98-1725.pdf}\\
\noindent Kellermann, K. I. 1999, A\&A 500, 143\\
\noindent Lim et al. 2005, Icarus 173, 385\\
\noindent Moore et al. 2004, in ``Jupiter: The Planet, Satellites, and Magnetosphere''\\
\noindent Rathbun et al. 2004, Icarus 169, 127\\
\noindent Saint-Pe et al. 1993, Icarus 105, 271\\

\noindent {\it Website 1:} \url{http://www.vla.nrao.edu/astro/calib/manual/baars.html}\\
\noindent {\it Website 2:} \url{http://science.nasa.gov/science-news/science-at-nasa/1998/ast03dec98_1/}\\
\noindent {\it Website 3:} \url{http://solarsystem.nasa.gov/planets/profile.cfm?Object=Triton}\\
%\end{document}




\section{Velocity Reference Frames}
\label{section:conv.vel}

CASA supported velocity frames are listed in Table\,\ref{table:conv.velocityframes}.

%The velocity frames supported in CASA are:
%me.listcodes(me.frequency())

\begin{table}[htb]
\caption{Velocity frames in CASA \label{table:conv.velocityframes}}
\begin{center}
\begin{tabular}{lll}
Name & Description\\
\hline
REST &  Laboratory\\
LSRK &  local standard of rest (kinematic)\\
LSRD &  local standard of rest (dynamic)\\
BARY &  barycentric\\
GEO &  geocentric\\
TOPO &  topocentric\\
GALACTO &  galactocentric\\
LGROUP &  Local Group\\
CMB &  cosmic microwave background dipole\\
\end{tabular}
\end{center}
\end{table}

%The same list can be accessed from the toolkit as {\tt me.listcodes(me.frequency())}.


\subsection{Doppler Types}
\label{section:conv.doppler}

CASA supported Doppler types are listed in Table\,\ref{table:conv.doppler}.

%The velocity frames supported in CASA are:
%me.listcodes(me.frequency())

\begin{table}[htb]
\caption{Doppler types in CASA \label{table:conv.doppler}}
\begin{center}
\begin{tabular}{lll}
Name & Description\\
\hline
RADIO 	& \\
Z 	& \\
RATIO 	& \\
BETA 	& \\
GAMMA 	& \\
OPTICAL & \\	
TRUE 	& \\
RELATIVISTIC & \\ 
\end{tabular}
\end{center}
\end{table}


\section{Time Reference Frames}
\label{section:conv.time}

CASA supported time reference frames are listed in Table\,\ref{table:conv.timeframes}.

\begin{table}[htb]
\caption{Time reference frames in CASA \label{table:conv.timeframes}}
\begin{center}
\begin{tabular}{lll}
Name & Description\\
\hline

LAST &   \\
LMST &   \\
GMST1 &   \\
GAST &   \\
UT1 &   \\
UT2 &   \\
UTC &   \\
TAI &   \\
TDT &   \\
TCG &   \\
TDB &   \\
TCB &   \\
IAT &   \\
GMST &   \\
TT &   \\
ET &   \\
UT &   \\
\end{tabular}
\end{center}
\end{table}




\section{Coordinate Framess}
\label{section:conv.coordinateframes}
%me.listcodes(me.frequency())

CASA supported time coordinate frames are listed in Table\,\ref{table:conv.coordinateframes}.

\begin{table}[htb]
\caption{Coordinate frames in CASA \label{table:conv.coordinateframes}}
\begin{center}
\begin{tabular}{lll}
Name & Description\\
\hline
    J2000     &  mean equator and equinox at J2000.0 (FK5)\\
    JNAT      &  geocentric natural frame\\
    JMEAN     &  mean equator and equinox at frame epoch\\
    JTRUE     &  true equator and equinox at frame epoch\\
    APP       &  apparent geocentric position\\
    B1950     &  mean epoch and ecliptic at B1950.0.\\ 
    B1950\_VLA &  mean epoch(1979.9)) and ecliptic at B1950.0\\
    BMEAN     &  mean equator and equinox at frame epoch\\
    BTRUE     &  true equator and equinox at frame epoch\\
    GALACTIC  &  Galactic coordinates\\
    HADEC     &  topocentric HA and declination\\
    AZEL      &  topocentric Azimuth and Elevation (N through E)\\
    AZELSW    &  topocentric Azimuth and Elevation (S through W)\\
    AZELNE    &  topocentric Azimuth and Elevation (N through E)\\
    AZELGEO   &  geodetic Azimuth and Elevation (N through E)\\
    AZELSWGEO &  geodetic Azimuth and Elevation (S through W)\\
    AZELNEGEO &  geodetic Azimuth and Elevation (N through E)\\
    ECLIPTC   &  ecliptic for J2000 equator and equinox\\
    MECLIPTIC &  ecliptic for mean equator of date\\
    TECLIPTIC &  ecliptic for true equator of date\\
    SUPERGAL  &  supergalactic coordinates\\
    ITRF      &  coordinates wrt ITRF Earth frame\\
    TOPO      &  apparent topocentric position\\
    ICRS      &  International Celestial reference system\\ 
\end{tabular}
\end{center}
\end{table}

\section{Physical Units}

\label{section:conv.units}

CASA also recognizes physical units. They are listed in
Tables\,\ref{table:conv.prefixes}, \ref{table:conv.SI},
and \ref{table:conv.custom}.

\begin{table}[htb]
\caption{Prefixes \label{table:conv.prefixes}}
\begin{center}
\begin{tabular}{lll}
Prefix & Name & Value\\
\hline
    Y  &         (yotta)  &                    $10^{24}$\\
    Z  &         (zetta)  &                    $10^{21}$\\
    E  &         (exa)    &                    $10^{18}$ \\
    P  &         (peta)   &                    $10^{15}$\\
    T  &         (tera)   &                    $10^{12}$\\
    G  &         (giga)   &                    $10^{9}$\\
    M  &         (mega)   &                    $10^{6}$\\
    k  &         (kilo)   &                    $10^{3}$\\
    h  &         (hecto)  &                    $10^{2}$\\
    da &         (deka)   &                    10\\
    d  &         (deci)   &                    $10^{-1}$\\
    c  &         (centi)  &                    $10^{-2}$\\
    m  &         (milli)  &                    $10^{-3}$\\
    u  &         (micro)  &                    $10^{-6}$\\
    n  &         (nano)   &                    $10^{-9}$\\
    p  &         (pico)   &                    $10^{-12}$\\
    f  &         (femto)  &                    $10^{-15}$\\
    a  &         (atto)   &                    $10^{-18}$\\
    z  &         (zepto)  &                    $10^{-21}$\\
    y  &         (yocto)  &                    $10^{-24}$\\

\end{tabular}
\end{center}
\end{table}




\begin{table}
\caption{SI Units \label{table:conv.SI}}

\begin{center}
\begin{tabular}{lll}
Unit & Name & Value\\
\hline
     \$        &   (currency)              &     1 \_ \\       
     \%        &   (percent)               &     0.01 \\
     \%\%      &   (permille)              &     0.001 \\
     A         &   (ampere)                &     1\,A\\
     AE        &   (astronomical unit)     &     149597870659\,m\\
      AU       &   (astronomical unit)     &     149597870659\,m\\
      Bq       &   (becquerel)             &     1 s$^{-1}$\\
      C        &   (coulomb)               &     1 s\,A\\
      F        &   (farad)                 &     1 m$^{-2}$\,kg$^{-1}$\,s$^{4}$\,A$^{2}$\\
      Gy       &   (gray)                  &     1 m$^{2}$\,s$^{-2}$\\
      H        &   (henry)                 &     1 m$^{2}$\,kg\,s$^{-2}$\,A$^{-2}$\\
      Hz       &   (hertz)                 &     1 s$^{-1}$\\
      J        &   (joule)                 &     1 m$^{2}$\,kg\,s$^{-2}$\\
      Jy       &   (jansky)                &     $10^{-26}$\,kg\,s$^{-2}$\\
      K        &   (kelvin)                &     1 K\\
      L        &   (litre)                 &     0.001 m$^{3}$\\
      M0       &   (solar mass)            &     1.98891944407$\times 10^{30}$\,kg\\
      N        &   (newton)                &     1 m\,kg\,s$^{-2}$\\
      Ohm      &   (ohm)                   &     1 m$^{2}$\,kg\,s$^{-3}$\,A$^{-2}$\\
      Pa       &   (pascal)                &     1 m$^{-1}$\,kg\,s$^{-2}$\\
      S        &   (siemens)               &     1 m$^{-2}$\,kg$^{-1}$\,s$^{3}$\,A$^{2}$\\
      S0       &   (solar mass)            &     1.98891944407$\times 10^{30}$\,kg\\
      Sv       &   (sievert)               &     1 m$^{2}$\,s$^{-2}$\\
      T        &   (tesla)                 &     1 kg\,s$^{-2}$\,A$^{-1}$\\
      UA       &   (astronomical unit)     &     149597870659\,m\\
      V        &   (volt)                  &     1 m$^{2}$\,kg\,s$^{-3}$\,A$^{-1}$\\
      W        &   (watt)                  &     1 m$^{2}$\,kg\,s$^{-3}$\\
      Wb       &   (weber)                 &     1 m$^{2}$\,kg\,s$^{-2}$\,A$^{-1}$\\
      \_        &   (undimensioned)         &     1 \_\\

\end{tabular}
\end{center}
\end{table}

\normalsize


\begin{table}
\addtocounter{table}{-1}
\caption{SI Units -- continued}

\begin{center}
\begin{tabular}{lll}
Unit & Name & Value\\
\hline
      a        &   (year)                  &     31557600\,s\\
      arcmin   &   (arcmin)                &     0.000290888208666\,rad\\
      arcsec   &   (arcsec)                &     4.8481368111$\times10^{-6}$\,rad\\
      as       &   (arcsec)                &     4.8481368111e$\times10^{-6}$\,rad\\
      cd       &   (candela)               &     1 cd\\
      cy       &   (century)               &     3155760000 s\\
      d        &   (day)                   &     86400 s\\
      deg      &   (degree)                &     0.0174532925199 rad\\
      g        &   (gram)                  &     0.001 kg\\
      h        &   (hour)                  &     3600 s\\
      l        &   (litre)                 &     0.001 m$^{3}$\\
      lm       &   (lumen)                 &     1 cd\,sr\\
      lx       &   (lux)                   &     1 m$^{-2}$\,cd\,sr\\
      m        &   (metre)                 &     1 m\\
      min      &   (minute)                &     60 s\\
      mol      &   (mole)                  &     1 mol\\
      pc       &   (parsec)                &     3.08567758065$\times10^{16}$ m\\
      rad      &   (radian)                &     1 rad\\
      s        &   (second)                &     1 s\\
      sr       &   (steradian)             &     1 sr\\
      t        &   (tonne)                 &     1000 kg\\

\end{tabular}
\end{center}
\end{table}



\begin{table}
\caption{Custom Units \label{table:conv.custom}}

\begin{center}
\begin{tabular}{lll}
Unit & Name & Value \\
\hline
      "          & (arcsec)                      & 4.8481368111$\times10^{-6}$\,rad\\
      "\_2        & (square arcsec)               & 2.35044305391$\times 10^{-11}$\,sr\\
      '          & (arcmin)                      & 0.000290888208666 rad\\
      ''         & (arcsec)                      & 4.8481368111$\times10^{-6}$\,rad\\
      ''\_2       & (square arcsec)               & 2.35044305391$\times10^{-11}$\,sr\\
      '\_2        & (square arcmin)               & 8.46159499408$\times10^{-8}$\,sr\\
      :          & (hour)                        & 3600 s\\
      ::         & (minute)                      & 60 s\\
      :::        & (second)                      & 1 s\\
      Ah         & (ampere hour)                 & 3600 s\,A\\
      Angstrom   & (angstrom)                    & 1e-10 m\\
      Btu        & (British thermal unit (Int))  & 1055.056 m$^{2}$\,kg\,s$^{-2}$\\
      CM         & (metric carat)                & 0.0002 kg\\
      Cal        & (large calorie (Int))         & 4186.8 m$^{2}$\,kg\,s$^{-2}$\\
      FU         & (flux unit)                   & $10^{-26}$\,kg\,s$^{-2}$\\
      G          & (gauss)                       & 0.0001 kg\,s$^{-2}$\,A$^{-1}$\\
      Gal        & (gal)                         & 0.01 m\,s$^{-2}$\\
      Gb         & (gilbert)                     & 0.795774715459 A\\
      Mx         & (maxwell)                     & $10^{-8}$\,m$^{2}$\,kg\,s$^{-2}$\,A$^{-1}$\\
      Oe         & (oersted)                     & 79.5774715459 m$^{-1}$\,A\\
      R          & (mile)                        & 0.000258 kg$^{-1}$\,s\,A\\
      St         & (stokes)                      & 0.0001 m$^{2}$\,s${-1}$\\
      Torr       & (torr)                        & 133.322368421 m$^{-1}$\,kg\,s$^{-2}$\\
      USfl\_oz    & (fluid ounce (US))            & 2.95735295625$\times10^{-5}$\,m$^{3}$\\
      USgal      & (gallon (US))                 & 0.003785411784
      m$^{3}$\\

\end{tabular}
\end{center}
\end{table}




\begin{table}
\addtocounter{table}{-1}
\caption{Custom Units -- continued}

\begin{center}
\begin{tabular}{lll}
Unit & Name & Value \\
\hline
      WU         & (WSRT flux unit)              & $5\times 10^{-29}$\,kg\,s$^{-2}$\\
      abA        & (abampere)                    & 10 A\\
      abC        & (abcoulomb)                   & 10 s\,A\\
      abF        & (abfarad)                     & $10^{9}$\,m$^{-2}$\,kg$^{-1}$\,s$^{4}$\,A$^{2}$\\
      abH        & (abhenry)                     & $10^{-9}$\,m$^{2}$\,kg\,s$^{-2}$\,A$^{-2}$\\
      abOhm      & (abohm)                       & $10^{-9}$\,m$^{2}$\,kg\,s$^{-3}$\,A$^{-2}$\\
      abV        & (abvolt)                      & $10^{-8}$\,m$^{2}$\,kg\,s$^{-3}$\,A$^{-1}$\\
      ac         & (acre)                        & 4046.8564224 m$^{2}$\\
      arcmin\_2   & (square arcmin)               & 8.46-2159499408$\times10^{-8}$\,sr\\
      arcsec\_2   & (square arcsec)               & 2.35044305391$\times10^{-11}$\,sr\\
      ata        & (technical atmosphere)        & 98066.5 m$^{-1}$.kg.s$^{-2}$\\
      atm        & (standard atmosphere)         & 101325 m$^{-1}$.kg.s$^{-2}$\\
      bar        & (bar)                         & 100000 m$^{-1}$.kg.s$^{-2}$\\
      beam       & (undefined beam area)         & 1 \_\\
      cal        & (calorie (Int))               & 4.1868 m$^{2}$\,kg\,s$^{-2}$\\
      count      & (count)                       & 1 \_\\
      cwt        & (hundredweight)               & 50.80234544 kg\\
      deg\_2      & (square degree)               & 0.000304617419787 sr\\
      dyn        & (dyne)                        & $10^{-5}$\,m\,kg\,s$^{-2}$\\
      eV         & (electron volt)               & 1.60217733$\times10^{-19}$\,m$^{2}$\,kg\,s$^{-2}$\\
      erg        & (erg)                         & $10^{-7}$\,m$^{2}$\,kg\,s$^{-2}$\\
      fl\_oz      & (fluid ounce (Imp))           & 2.84130488996$\times10^{-5}$\,m$^{3}$\\
      ft         & (foot)                        & 0.3048 m\\
      fu         & (flux unit)                   & $10^{-26}$\,kg\,s$^{-2}$\\
      fur        & (furlong)                     & 201.168 m\\
      gal        & (gallon (Imp))                & 0.00454608782394
      m$^{3}$\\
\end{tabular}
\end{center}
\end{table}




\begin{table}
\addtocounter{table}{-1}
\caption{Custom Units -- continued}

\begin{center}
\begin{tabular}{lll}
Unit & Name & Value \\
\hline
      ha         & (hectare)                     & 10000 m$^{2}$\\
      hp         & (horsepower)                  & 745.7 m$^{2}$\,kg\,s$^{-3}$\\
      in         & (inch)                        & 0.0254 m\\
      kn         & (knot (Imp))                  & 0.514773333333 m\,s$^{-1}$\\
      lambda     & (lambda)                      & 1 \_\\
      lb         & (pound (avoirdupois))         & 0.45359237 kg\\
      ly         & (light year)                  & 9.46073047$\times10^{15}$\,m\\
      mHg        & (metre of mercury)            & 133322.387415 m$^{-1}$\,kg\,s$^{-2}$\\
      mile       & (mile)                        & 1609.344 m\\
      n\_mile     & (nautical mile (Imp))         & 1853.184 m\\
      oz         & (ounce (avoirdupois))         & 0.028349523125 kg\\
      pixel      & (pixel)                       & 1 \_\\
      sb         & (stilb)                       & 10000 m$^{-2}$\,cd\\
      sq\_arcmin  & (square arcmin)               & 8.46159499408$\times10^{-8}$ sr\\
      sq\_arcsec  & (square arcsec)               & 2.35044305391$\times10^{-11}$ sr\\
      sq\_deg     & (square degree)               & 0.000304617419787 sr\\
      statA      & (statampere)                  & 3.33564095198$\times10^{-10}$\,A\\
      statC      & (statcoulomb)                 & 3.33564095198$\times10^{-10}$\,s\,A\\
      statF      & (statfarad)                   & 1.11188031733$\times10^{-12}$\,m$^{-2}$\,kg$^{-1}$\,s$^{4}$\,A$^{2}$\\
      statH      & (stathenry)                   & 899377374000\,m$^{2}$\,kg\,s$^{-2}$\,A$^{-2}$\\
      statOhm    & (statohm)                     & 899377374000\,m$^{2}$\,kg\,s$^{-3}$\,A$^{-2}$\\
      statV      & (statvolt)                    & 299.792458 m$^{2}$\,kg\,s$^{-3}$\,A$^{-1}$\\
      u          & (atomic mass unit)            & 1.661$\times10^{-27}$\, kg\\
      yd         & (yard)                        & 0.9144 m\\
      yr         & (year)                        & 31557600 s\\

\end{tabular}
\end{center}
\end{table}

\normalsize


\section{Physical Constants}
\label{section:conv.constants}

The physical constants included in CASA can be found in Table\,\ref{table:conv.constants}.



\begin{table}
\caption{Physical Constants \label{table:conv.constants}}
\begin{center}
\begin{tabular}{lll}\\
Constant & Name & Value \\
\hline
         pi     & 3.14..                  &  3.14159  \\
         ee     & 2.71..                  &  2.71828  \\
         c      & light vel.              &  2.99792$\times10^{8}$\,m\,s$^{-1}$\\  
         G      & grav. const             &  6.67259$\times10^{11}$\,N\,m$^{2}$\,kg$^{-2}$\\  
         h      & Planck const            &  6.62608$\times10^{-34}$\,J\,s  \\
         HI     & HI line                 &  1420.41 MHz  \\
         R      & gas const               &  8.31451 J\,K$^{-1}$\,mol$^{-1}$\\  
         NA     & Avogadro \#              &  6.02214$\times10^{23}$\,mol$^{-1}$\\  
         e      & electron charge         &  1.60218$\times10^{-19}$\,C  \\
         mp     & proton mass             &  1.67262$\times10^{-27}$\, kg  \\
         mp\_me  & mp/me                   &  1836.15  \\
         mu0    & permeability vac.       &  1.25664$\times10^{-6}$\,H\,m$^{-1}$\\  
         eps0   & permittivity vac.       &  1.60218$\times10^{-19}$\,C  \\
         k      & Boltzmann const         &  1.38066$\times10^{-23}$\,J\,K$^{-1}$  \\
         F      & Faraday const           &  96485.3 C\,mol$^{-1}$  \\
         me     & electron mass           &  9.10939$\times10^{-31}$\, kg  \\
         re     & electron radius         &  2.8179e$\times10^{-15}$\, m  \\
         a0     & Bohr��s radius            &  5.2918$\times10^{-11}$\, m  \\
         R0     & solar radius            &  6.9599$\times10^{8}$\, m  \\
         k2     & IAU grav. const$^{2}$    &  0.000295912
         AU$^{3}$\,d$^{-2}$\,S0$^{-1}$ \\

\end{tabular}
\end{center}
\end{table}


