%%%%%%%%%%%%%%%%%%%%%%%%%%%%%%%%%%%%%%%%%%%%%%%%%%%%%%%%%%%%%%%%%
%%%%%%%%%%%%%%%%%%%%%%%%%%%%%%%%%%%%%%%%%%%%%%%%%%%%%%%%%%%%%%%%%
%%%%%%%%%%%%%%%%%%%%%%%%%%%%%%%%%%%%%%%%%%%%%%%%%%%%%%%%%%%%%%%%%

% STM 2007-04-13  split from previous version
% STM 2007-04-19  new parameter interface
% STM 2007-06-13  point to standard selection etc.
% MPR 2007-07-05  minor tweaks, esp. to match frozen alpha-1 release of 4jul07
% STM 2007-09-20  pre-beta version
% STM 2007-10-10  beta version spell-checked
% STM 2007-11-10  beta patch 0.5
% STM 2008-02-25  beta patch 1.0 editing started
% STM 2008-04-29  beta patch 1.0 editing done
% STM 2008-05-13  beta patch 2.0 editing started
% STM 2008-06-23  beta patch 2.0 updated
% STM 2008-07-30  beta patch 2.1 updated (add solint changes)
% STM 2008-08-15  beta patch 2.2 updates
% STM 2008-09-30  beta patch 3.0 editing started
% STM 2008-12-01  beta patch 3.0 release
% STM 2009-01-21  beta patch 3.1 release
% STM 2009-05-19  beta patch 4.0 editing started
% STM 2009-06-19  beta patch 4.0 release
% STM 2009-11-10  release 3.0.0 editing started
% STM 2009-12-02  release 3.0.0 change to User Reference and Cookbook
% JO 2010-02-22 release 3.0.1 editing started
% JO 2010-04-16 release 3.0.2 editing started
% JO 2010-10-11 release 3.1.0 edits
% GM 2010-11-19 release 3.1.0 edits
% JO 2011-04-13 release 3.2.0 editing starts
% JO 2011-10-03 redits for CASA 3.3.0
% JO 2012-04-19 edits for 3.4.0
% DP 2012-09-27 corrections concerning concat
% JO 2012-09-28 edits for 4.0.0
% JO 2013-04-23 edits for 4.1.0

\chapter{Introduction}
\label{chapter:intro}

%\vspace{5mm}

This document describes how to calibrate and image interferometric and
single-dish radio astronomical data using the CASA (Common Astronomy
Software Application) package.  CASA is a suite of
astronomical data reduction tools and tasks that can be run via the
IPython interface to Python.  CASA is being developed in order to 
fulfill the data post-processing requirements of the ALMA and EVLA
projects, but also provides basic and advanced capabilities useful for
the analysis of data from other radio, millimeter, and submillimeter
telescopes.

You have in your hands the latest release of CASA.  This package is
under active development, and thus there are a number of caveats and
limitations for the use of this 
package.  See the release notes (\S~\ref{section:intro.release})
below for more information,
and pay heed to the numerous {\bf ALERT}s placed throughout this
reference.  You can expect regular updates and patches, as well as
increasing functionality.

\begin{wrapfigure}{r}{2.5in}
  \begin{boxedminipage}{2.5in}
     \centerline{\bf Inside the Toolkit:}
     Throughout this user reference, we will occasionally intersperse
     boxed-off pointers to parts of the toolkit that power users might
     want to explore.  
  \end{boxedminipage}
\end{wrapfigure}

This user reference and cookbook is a task-based walk-through of
interferometric data reduction and analysis. In CASA, {\bf tasks}
represent the more streamlined operations that a typical user would
carry out. The idea for having tasks is that they are simple to use,
provide a more familiar interface, and are easy to learn for most
astronomers who are familiar with radio interferometric data reduction
(and hopefully for novice users as well). In CASA, the {\bf tools}
provide the full capability of the package, and are the atomic
functions that form the basis of data reduction.  These tools augment
the tasks, or fill in gaps left by tasks that are under development
but not yet available. See the {\bf CASA Toolkit Manual} for more
details on the tools (available from \verb=casa.nrao.edu=).  Note that
in most cases, the tasks are Python interface scripts to the tools,
but with specific, limited access to them and a standardized interface
for parameter setting.  The tasks and tools can be used together to
carry out more advanced data reduction operations.

For the moment, the audience is assumed to have some basic grasp of
the fundamentals of synthesis imaging, so details of how a radio
interferometer or telescope works and why the data needs to undergo
calibration in order to make synthesis images are left to other
documentation --- a good place to start might be Synthesis Imaging in
Radio Astronomy II (1999, ASP Conference Series Vol. 180, eds. Taylor,
Carilli \& Perley).

This reference is broken down by the main phases of data analysis:
\begin{itemize}
   \item data import, export, and selection (Chapter~\ref{chapter:io}),

   \item examination and flagging of data (Chapter~\ref{chapter:edit}),

   \item interferometric calibration (Chapter~\ref{chapter:cal}),

   \item interferometric imaging (Chapter~\ref{chapter:im}),

   \item image analysis (Chapter~\ref{chapter:analysis}), and

   \item data and image visualization (Chapter~\ref{chapter:display}).

   \item single dish data analysis (Chapter~\ref{chapter:sd}), and

   \item simulation (Chapter~\ref{chapter:sim}).

\end{itemize}
These are included for users that will be doing EVLA and ALMA
telescope commissioning and software development.

The general appendices provide more details on what's happening
under the hood of CASA, as well as supplementary material on tasks,
scripts, and relating CASA to other packages.  These appendices
include:
\begin{itemize}
   \item obtaining and installing CASA 
         (Appendix~\ref{chapter:install}),

   \item more details about Python and CASA 
         (Appendix~\ref{chapter:python}),

   \item a discussion of the Hamaker-Bregman-Sault Measurement Equation
         (Appendix~\ref{chapter:me}),

%   \item task summaries and parameter lists
%         (Appendix~\ref{chapter:tasks}),

   \item annotated scripts for typical data reduction cases 
         (Appendix~\ref{chapter:scripts}), and

   \item CASA dictionaries to AIPS, MIRIAD, and CLIC
         (Appendix~\ref{chapter:dict}).

   \item Writing your own CASA Task
         (Appendix~\ref{chapter:write}).
\end{itemize}

The CASA User Documentation includes:
\begin{itemize}
   \item CASA {\bf User Reference \& Cookbook} --- this
     document, a task-based data analysis walk-through and instructions;
   \item CASA {\bf in-line help} --- accessed using {\tt help} in the 
              {\bf casapy} interface;
   \item The {\bf CASA Toolkit Reference Manual} --- 
         details on a specific task or tool does and how to use it.
   \item The {\bf CASA Task Reference Manual} --- 
         the information from the inline help and task documentation,
         available online in HTML.
\end{itemize}

The CASA home page can be found at:
\begin{quote}
  \url{http://casa.nrao.edu}
\end{quote}
>From there you can find documentation and assistance for the use
of the package, including the User Documentation.  You will
also find information on how to obtain the latest release and
receive user support.

There is also a CASAGuides Wiki
\begin{quote}
   \url{http://casaguides.nrao.edu}
\end{quote}
that contains helpful information on CASA startup, AIPS-to-CASA
cheat sheet, example scripts of processing your data in CASA,
along with hints and tricks to best use this package.

\subsection{Reference for Publications}

If you use CASA for any of your data reduction or analysis, you may
use the following reference:\\

{\it McMullin, J. P., Waters, B., Schiebel, D., Young, W., \& Golap, K. 2007, Astronomical Data Analysis Software and Systems XVI (ASP Conf. Ser. 376), ed. R. A. Shaw, F. Hill, \& D. J. Bell (San Francisco, CA: ASP), 127}



%%%%%%%%%%%%%%%%%%%%%%%%%%%%%%%%%%%%%%%%%%%%%%%%%%%%%%%%%%%%%%%%%
%%%%%%%%%%%%%%%%%%%%%%%%%%%%%%%%%%%%%%%%%%%%%%%%%%%%%%%%%%%%%%%%%

\section{About This Release}
\label{section:intro.release}

\begin{wrapfigure}{r}{2.5in}
  \begin{boxedminipage}{2.5in}
     \centerline{\underline{\bf ALERT}}
     Boxes like this will bring to your attention some of the
     features (or lack thereof) in the current release
     of CASA.  There are also {\bf ALERT} notes in the text.
  \end{boxedminipage}
\end{wrapfigure}

CASA 4.1.0 is now available and the main feature improvements are
listed below. In addition to the new features, many bugs were
addressed and fixed.

We occasionally issue patches and 'stable' versions of CASA. To get
notified, please subscribe to the 'casa-users' mailing list. 'Stable'
as well as CASA releases are available at
\url{http://casa.nrao.edu}. Releases will be announced via the
'casa-announce' mailing list. To subscribe, please visit
\url{http://casa.nrao.edu}. For feedback, and help please go to the
NRAO helpdesk \url{http://help.nrao.edu}; for ALMA questions please
use the ALMA helpdesk \url{http://help.almascience.org}.

Note that in its current incarnation CASA is designed to support EVLA,
ALMA, and VLA data, as well as single dish data from ALMA as well as
Nobeyama. Data from other telescopes, be it single dish or
interferometers can be imported from {\tt uvfits}, {\tt FITS-IDI} , or
{\tt sdfits} formats into measurements sets (ms) or scantables in
CASA. Given the variety of non-standard fits formats, we cannot
guarantee that CASA will fully support data from all
telescopes. However, efforts are made to support data formats from
other facilities.



%%%%%%%%%%%%%

\section{Obtaining CASA}
\label{section:intro.obtaining}


CASA is available for the following operating systems:

\begin{itemize}
    \item Linux
      \begin{itemize}
          \item RedHat 5.7 and 6.3 (64-bit)
%          \item Fedora 16 (64-bit)
%          \item Ubuntu 10 (64-bit) 
      \end{itemize}
      

    \item Mac OS
         \begin{itemize}
          \item Mac OS 10.6 (Snow Leopard; 64-bit) 
          \item Mac OS 10.7 \& 10.8 (Lion/Mountain Lion; 64-bit)
\end{itemize}
\end{itemize}


The latest and previous releases can be downloaded from our CASA home
page: \url{http://casa.nrao.edu}, following the 'Obtaining CASA'
link (direct link: \url{http://casa.nrao.edu/casa_obtaining.shtml}.




%%%%%%%%%%%%%%%%%%%%%%%%%%%%%%%%%%%%%%%%%%%%%%%%%%%%%%%%%%%%%%%%%


\subsection{What's New in Release 4.1.0}
\label{section:intro.release.whatsnew}


Major improvements over the previous version of CASA include:
\begin{itemize}

\item  {\bf Imaging}
  \begin{itemize}
    \item  implementation of narrow-band A-projection algorithm

\end{itemize}


\item  {\bf Data examination/editing/import/export}
   \begin{itemize}
        \item  {\tt flagdata} and {\tt flagcmd} can now flag based on
  calibration tables (e.g. on Tsys)
    \item  support for multiple input files in {\tt flagdata}
     \item  {\tt listobs} reports a larger number of data parameters,
   including flagged rows, integration times, spw names
     \item  new tool {\tt msmd} with a large number of methods to access meta data information
     \item  {\tt plotms} can now display calibration tables (this will eventually replace the current {\tt plotcal} functionality
     \item  multi-panel plotting in {\tt plotms}
     \item  new task {\tt imreframe} to change the spectral frame i which the velocity/frequency is reported
     \item  disabling tool return in tasks to align with
   general task behavior in CASA and to prevent table locks
    
\end{itemize}

\item  {\bf Calibration}

\begin{itemize}

   \item  expanded use of obsid in all calibration tasks. Calibration
     solving behavior at ObsId boundaries is now controlled by the
     {\it combine} parameter; by default solutions will not cross
     ObsId boundaries in CASA v4.1 and later. Interpolation of
     solutions across ObsId boundaries is controlled by extensions to
     the {\it interp} parameter. Note: Extended ObsId capabilities are available only in CASA 4.1 and higher. Most older caltables will be backward compatible, but without any ObsId capabilities. 
   \item  radial limit to the VLA flux calibrator models (used by {\tt setjy})
   \item  {\tt calwt} in {\tt applycal} can now be a vector to enable/disable weight calibration for individual gain tables
   \item  {\tt uvcontsub} now allows for specification of corrupted channels to be excluded from the fit in addition to line free channels 
   \item  {\tt gencal} function to support VLA switched power
   calibration is now {\it caltype="swpow"}, requantizer gains can
   be computed via {\it caltype="rq"} and veracity of switched power
   can be evaluated with {\it caltype='swp/rq'}
   \item  {\tt wvrgcal} allows a string for the timescale of temporal smoothing

\end{itemize}

\item  {\bf Simulation}
\begin{itemize}
   \item  Task {\tt simalma} uses feather to combine interferometer and total power images.
\end{itemize}


\item  {\bf Data and Image Manipulation} 
\begin{itemize}
    \item  A graphical interface {\tt "casafeather"} (run on the Unix
  command line) to display the feathering weighting functions, scaling
  and visibility functions of the {\tt feather} task

  \item {\tt imregrid} has now {\it asvelocity=True} as default
\end{itemize}

\item  {\bf Data analysis}
\begin{itemize}
   \item  new task {\tt impv} to create position-velocity diagrams (also available as tool method {\tt ia.pv})
   \item   data analysis tasks now accept specification of channel ranges
  in the form of velocity ranges
   
\end{itemize}


\item  {\bf Viewer}

\begin{itemize}
   \item  histogram plotting of data values and Gaussian fitting of these
   \item  interactive creation of position-velocity diagrams
   \item  arbitrary spatial brightness profile slices across images
   \item ability to visually set the color transfer function and data
     range based on intensity value histogram
   \item global color option for multiple images
%    \item  source finding algorithm 
   \item  the docking behavior can now be set in the ``Preferences'' menu

\end{itemize}

\item {\bf Single Dish}
\begin{itemize}


  
\item  new tool method {\tt sd.scantable.sub\_baseline()} allows for
  independent specification of parameters and function for each
  baseline per spectrum

  \item  Baseline fitting results can be stored as a CASA table and
  applied later

  \item  new sd tool {\tt function sd.scantable.calc\_aic()} is
  available to calculate some major values of model selection criteria
  for a given spectrum and a baseline function.
  
 \item  new task {\tt sdcal2} for interferometry style calibration (generate caltable and apply caltables)
  \item  new grid functions GAUSS (Gaussian) and GJINC (Gaussian * Jinc) are available for {\tt sdimaging}, {\tt sdtpimaging}, and {\tt sdgrid}
   \item  {\tt sdimaging}: auto calculation of map center when phasecenter is empty
   \item  {\tt sdsave} (and {\tt sd.splitant}): frequency reference frame takes from input MS when data is imported
   \item  {\tt sdgrid}: grid positions without any data are flagged
   \item  {\tt sdplot}: {\it plottype='pointings'} allows overlaying
  the scan pattern and changing the colors of plot symbols by source
  type, scan ID, beam ID, polarization ID, or IF number.
  \item  Nobeyama filler: options to control frequency reference frame,
  proper handling of dual polarization data, support for data from SAM45
 
 
\end{itemize}


\item {\bf Performance}
 
\begin{itemize}
%  \item  experimental Taylor term support ({\it nterms}) in {\tt pclean}
  \item  {\tt sdbaseline} speed improvement (2-10 times faster)
  
  \item new experimental task {\tt mstransform}, which can be used to
  replace {\tt split}, {\tt cvel}, {\tt hanningsmooth} and
  {\tt partition}. The task optimizes the I/O so that all the transformations are applied in memory without the need to read and
write multiple times. See
  \url{http://www.eso.org/~scastro/ALMA/casa/MST/MSTransformDocs}
  for details.
  \item  The {\tt partition} task has been modified to use the new
  MSTransform framework. The interface of partition also changed. The
  parameters "calmsselection" and "timebin" have been
  removed. {\tt partition} now supports channel selections and
  re-indexes spws after a selection.
\end{itemize}

\item {\bf Other}
\begin{itemize}
    \item  ALMA ATM library upgraded to ALMA-9.4R4
    \item  added support for "chan" in the CASA CRTF region format
    

\end{itemize}
\end{itemize}





{\bf For known issues with this release please visit the CASA webpages:\\ \url{http://casa.nrao.edu/release_ki.shtml}}

%%%%%%%%%%%%%%%%%%%%%%%%%%%%%%%%%%%%%%%%%%%%%%%%%%%%%%%%%%%%%%%%%
%%%%%%%%%%%%%%%%%%%%%%%%%%%%%%%%%%%%%%%%%%%%%%%%%%%%%%%%%%%%%%%%%

\section{CASA Basics --- Information for First-Time Users}
\label{section:intro.basics}

This section assumes that CASA has been installed on your LINUX or OSX
system.  See Appendix~\ref{chapter:install} for instructions on how to 
obtain and install CASA.  

%%%%%%%%%%%%%%%%%%%%%%%%%%%%%%%%%%%%%%%%%%%%%%%%%%%%%%%%%%%%%%%%%
\subsection{Before Starting CASA}
\label{section:intro.basics.before}

First, you will most likely be starting CASA running from a working
directory that has your data in it, or at least where you want your
output to go.  It is easiest to start from there rather than changing
directories inside {\tt casapy}.
{\bf ALERT:} There is at least one task ({\tt plotxy}) that 
fails if the path to your working directory contains spaces in
its name, e.g. {\tt /users/smyers/MyTest/} is fine, but 
{\tt /users/smyers/My\ Test/} is not!  Please use our new task {\tt
  plotms} whenever possible and we may work on a better handling of
spaces in path names. 

If you have done a default installation under Linux using rpms, or 
on the Mac with the CASA application, then there should be a {\tt sh}
script called {\tt casapy} in the {\tt /usr/bin} area which is in
your path.  This shell will set up its environment and run the version
of {\tt casapy} that it points to.  If this is how you set up the
system, then you need to nothing further and can run {\tt casapy}.


For internal NRAO users we keep different version of CASA, the latest
``casapy'' release, the ``casapy-stable'' ``Stable'' version that is more
developed than the Release but without the full documentation (and no
GUI testing). We also offer the ``Test'' version, ``casapy-test'', which is
produced on a roughly weekly basis with all the latest code but it
underwent much less rigorous testing. Instructions how to run the
different versions at NRAO can be found on our
\verb=http://casa.nrao.edu= webpages under
the ``CASA at NRAO'' link for the different NRAO sites.


\subsubsection{Environment Variables}
\label{section:intro.basics.before.env}

Before starting up {\tt casapy}, you should set or reset any
{\it environment variables} needed, as CASA will adopt these on startup.  For
example, the {\tt PAGER} environment variable determines how help is
displayed in the CASA terminal window (see
\S~\ref{section:intro.basics.help.page}).  The choices are {\tt less},
{\tt more}, and {\tt cat}.

In {\tt bash}, pick one of
\small
\begin{verbatim}
   PAGER=less
   PAGER=more
   PAGER=cat
\end{verbatim}
\normalsize
followed by
\small
\begin{verbatim}
   export PAGER
\end{verbatim}
\normalsize

In {\tt csh} or {\tt tcsh}, pick one of
\small
\begin{verbatim}
   setenv PAGER less
   setenv PAGER more
   setenv PAGER cat
\end{verbatim}
\normalsize
The actions of these are as if you were using the equivalent Unix
shell command to view the help material.  See
\S~\ref{section:intro.basics.help.page} for more information on these
choices.
We recommend using the {\tt cat} option for most users, as this works
smoothly both interactively and in scripts.

\subsubsection{Where is CASA?}
\label{section:intro.basics.before.path}

Note that the path to the CASA installation, which contains
the scripts and data repository, will also depend upon the 
installation.  With a default installation under Linux this
will probably be in
\small
\begin{verbatim}
   /usr/lib64/casapy/
\end{verbatim}
\normalsize
or 

while in a Mac OSX default install it will likely be an 
application in the Applications folder.

You can find the location after initialized by looking at
the {\tt CASAPATH} environment variable.  You can find
it within {\tt casapy} by
\small
\begin{verbatim}
   pathname=os.environ.get('CASAPATH').split()[0]
   print pathname
\end{verbatim}
\normalsize

\subsection{Starting CASA}
\label{section:intro.basics.starting}

After having run the appropriate casainit script, CASA is started by
typing\\ {\tt casapy}\\ on the UNIX command line, e.g.
\small
\begin{verbatim}
   casapy
\end{verbatim}
\normalsize
After startup information,
you should get an IPython\\ {\tt CASA <1>:}\\ command prompt in the
xterm window where you started CASA. CASA will take approximately 10
seconds to initialize at startup in a new working directory;
subsequent startups are faster.  CASA is active when you get a\\ {\tt
CASA <1>}\\ prompt in the command line interface.  You will also
see a {\tt logger} GUI appear on your Desktop (usually near the
upper left).  
%{\em Note: Under MacOSX the logger will appear in a
%Console window.}

You also have the option of starting CASA with various {\tt logger}
options (see \S~\ref{section:intro.common.logger.startup}).  For
example, if you are running remotely in a terminal window without
an X11 connection, or if you just do not want to see the logger GUI,
and want the {\tt logger} messages to come to your terminal, do
\small
\begin{verbatim}
   casapy --nologger --log2term
\end{verbatim}
\normalsize

See \S~\ref{section:intro.common.logger} for information on the
{\tt logger} in general.

%%%%%%%%%%%%%%%%%%%%%%%%%%%%%%%%%%%%%%%%%%%%%%%%%%%%%%%%%%%%%%%%%
\subsection{Ending CASA}
\label{section:intro.basics.ending}

You can exit CASA by typing {\tt quit}.  This will bring up the
query
\small
\begin{verbatim}
Do you really want to exit ([y]/n)?
\end{verbatim}
\normalsize
to give you a chance in case you did not mean to exit.  You can
also quit using {\tt \%exit} or {\tt CTRL-D}.

If you don't want to see the question {\tt "Do you really want to exit
[y]/n?"}, then just type {\tt Exit} or {\tt exit} and CASA will stop
right then and there.

%%%%%%%%%%%%%%%%%%%%%%%%%%%%%%%%%%%%%%%%%%%%%%%%%%%%%%%%%%%%%%%%%
\subsection{What happens if something goes wrong?}
\label{section:intro.basics.wrong}

{\bf ALERT:} Please check the CASA Home Page for Release
Notes and FAQ information including a list of known problems.  If you think you
have encountered an unknown problem, please consult the CASA HelpDesk
(contact information on the CASA Home Page).  
See also the caveats to this Release (\S~\ref{section:intro.release} 
for pointers to our policy on User Support.

First, always check that your inputs are correct; use the
\small
\begin{verbatim}
   help <taskname>
\end{verbatim}
\normalsize
(\S~\ref{section:intro.basics.help.help}) or
\small
\begin{verbatim}
   help par.<parameter name>
\end{verbatim}
\normalsize
(\S~\ref{section:intro.basics.help.par}) to review the inputs/output.

%%%%%%%
\subsection{Aborting CASA execution}
\label{section:intro.basics.wrong.abort}

If something has gone wrong and you want to stop what is executing,
then typing {\tt CNTL-C} (Control and C keys simultaneously) will
usually cleanly abort the application. 
This will work if you are running a task synchronously.
If this does not work on your system, or you are running a task
asynchronously (\S~\ref{section:intro.tasks.async}) 
then try {\tt CNTL-Z} to put the task or shell in the
background, and then follow up with a {\tt kill -9 <PID>} where you
have found the relevant {\tt casapy} process ID ({\tt PID}) using 
{\tt ps} (see \S~\ref{section:intro.basics.crash} below).

If the problem causes CASA to crash, see the next sub-section.

See \S~\ref{section:intro.tasks.run} for more information on running
tasks.

{\bf Alert:} {\tt CNTL-C} while a tasks runs can corrupt your input
data file, e.g. when a scratch column is filled while aborting. If in
doubt, wait until the task has finished, delete the new files produced,
and start again. 



% You can submit a question/bug/enhancement via the site:
% 
% \url{http://bugs.aoc.nrao.edu}
% 
% Login (or register yourself if you don't have a login/password); click
% the 'Create New Issue' along the top tabs to file a
% question/bug/enhancement.
% 
% \begin{figure}[h!]
% \gname{jira}{3.5}
% \gname{jira2}{3.5}
% \caption{\label{fig:jira} Issue/Defect Tracking system. {\bf Left:}
%   http://bugs.aoc.nrao.edu page showing login entry fields. {\bf
%   Right:} Screen after selecting the "Create New
%   Issue" tab along the top.}
% \hrulefill
% \end{figure}

%%%%%%%%%%%%%%%%%%%%%%%%%%%%%%%%%%%%%%%%%%%%%%%%%%%%%%%%%%%%%%%%%
\subsection{What happens if CASA crashes?}
\label{section:intro.basics.crash}

Usually, restarting {\tt casapy} is sufficient to get you going again
after a crash takes you out of the Python interface.  Note that there
may be spawned subprocesses still running, such as the 
{\tt casaviewer} or the {\tt logger}.  These can be dismissed manually
in the usual manner.  After a crash, there may also be hidden
processes.  You can find these by listing processes, e.g. in linux:
\small
\begin{verbatim}
   ps -elf | grep casa
\end{verbatim}
\normalsize
or on MacOSX (or other BSD Unix):
\small
\begin{verbatim}
   ps -aux | grep casa
\end{verbatim}
\normalsize
You can then kill these, for example using the Unix {\tt kill}
or {\tt killall} commands.  This may be necessary if you are running
remotely using {\tt ssh}, as you cannot logout until all your
background processes are terminated.  For example,
\small
\begin{verbatim}
   killall ipcontroller
\end{verbatim}
\normalsize
or
\small
\begin{verbatim}
   killall Python
\end{verbatim}
\normalsize
will terminate the most common post-crash zombies.

%%%%%%%%%%%%%%%%%%%%%%%%%%%%%%%%%%%%%%%%%%%%%%%%%%%%%%%%%%%%%%%%%
\subsection{Python Basics for CASA}
\label{section:intro.basics.python}

Within CASA, you use Python to interact with the system.  This does
not mean an extensive Python course is necessary - basic interaction
with the system (assigning parameters, running tasks) is
straightforward.  At the same time, the full potential of Python is at
the more experienced user's disposal.  Some further details about
Python, IPython, and the interaction between Python and CASA can be
found in Appendix~\ref{chapter:python}.

The following are some examples of helpful hints and tricks on making
Python work for you in CASA.

\subsubsection{Variables}
\label{section:intro.basics.python.var}

Python variables are set using the {\tt <parameter> = <value>} 
syntax.  Python assigns the type dynamically as you set the value,
and thus you can easily give it a non-sensical value, e.g. 
\small
\begin{verbatim}
   vis = 'ngc5921.ms'
   vis = 1
\end{verbatim}
\normalsize
The CASA parameter system will check types when you run a task or
tool, or more helpfully when you set inputs using {\tt inp} (see
below).  CASA will check and protect the assignments of the global
parameters in its namespace.

Note that Python variable names are case-sensitive:
\small
\begin{verbatim}
CASA <109>: Foo = 'bar'
CASA <110>: foo = 'Bar'
CASA <111>: foo
  Out[111]: 'Bar'
CASA <112>: Foo
  Out[112]: 'bar'
\end{verbatim}
\normalsize
so be careful.  

Also note that mis-spelling a variable assignment
will not be noticed (as long as it is a valid Python variable name)
by the interface.  For example, if you wish to set 
{\tt correlation='RR'} but instead type {\tt corellation='RR'}
you will find {\tt correlation} unset and a new {\tt corellation}
variable set.  Command completion (see
\S\,\ref{section:intro.basics.help.tab}) should help you avoid this.

\subsubsection{Lists and Ranges}
\label{section:intro.basics.python.lists}

Sometimes, you need to give a task a list of indices.  If these
are consecutive, you can use the Python {\tt range} function to 
generate this list:
\small
\begin{verbatim}
CASA <1>: iflist=range(4,8)
CASA <2>: print iflist
[4, 5, 6, 7]
CASA <3>: iflist=range(4)
CASA <4>: print iflist
[0, 1, 2, 3]
\end{verbatim}
\normalsize

See Appendix~\ref{section:python.lists} for more information.

\subsubsection{Indexes}
\label{section:intro.basics.python.index}

As in C, Python indices are 0-based.  For example, the first
element in a list {\tt antlist} would be {\tt antlist[0]}:
\small
\begin{verbatim}
CASA <113>: antlist=range(5)
CASA <114>: antlist
  Out[114]: [0, 1, 2, 3, 4]
CASA <115>: antlist[0]
  Out[115]: 0
CASA <116>: antlist[4]
  Out[116]: 4
\end{verbatim}
\normalsize

CASA also uses 0-based indexing internally for elements in the Measurement
Set (MS -- the basic construct that contains visibility and/or single dish
data; see Chapter~\ref{chapter:io}).
Thus, we will often talk about Field or Antenna ``ID''s
which will be start at 0.  For example, the first field in an MS
would have {\tt FIELD\_ID==0} in the MSselect syntax, and can
be addressed as be indexed as {\tt field='0'} in most tasks, as
well as by name {\tt field='0137+331'} (assuming thats the name
of the first field).  You will see these indices in the MS summary
from the task {\tt listobs}. 
% Things can get confusing with 
% antenna names from old VLA data where the name is a number,
% e.g. {\tt antenna='1'} has {\tt ANTENNA\_ID==0} in the MS.

\subsubsection{Indentation}
\label{section:intro.basics.python.indent}

Python pays attention to the indentation of lines, as it
uses indentation to determine the level
of nesting in loops.  Be careful when cutting and pasting: if you
get the wrong indentation, then unpredictable things can happen
(usually it just gives an error).  

See Appendix~\ref{section:python.indent} for more information.

\subsubsection{System shell access}
\label{section:intro.basics.python.sys}

If you want to access system commands from a script, use the
{\tt os.system} command (Appendix~\ref{section:python.sys.os}).

In interactive mode, any input line beginning with a {\tt '!'} 
character is passed verbatim (minus the {\tt '!'}, of course) to the
underlying operating system. Also, several common commands ({\tt ls},
{\tt pwd}, {\tt less}) may be executed with or without the {\tt '!'},
although the {\tt cp} command must use {\tt '!'} and
{\tt cd} must be executed without the {\tt '!'}. 
For example:
\small
\begin{verbatim}
  CASA <5>: !rm -r mydata.ms
\end{verbatim}
\normalsize

Note that if you want to access a Unix environment variable, you
will need to prefix with a double {\tt \$\$} instead of a single
{\tt \$} --- for example, to print the value of the {\tt \$PAGER}
variable, you would use
\small
\begin{verbatim}
  CASA <6>: !echo $$PAGER
\end{verbatim}
\normalsize

See Appendix~\ref{section:python.sys} for more information.

\subsubsection{Executing Python scripts}
\label{section:intro.basics.python.execfile}

You can execute Python scripts (ASCII text files containing 
Python or casapy commands) using the {\tt execfile} command.
For example, to execute the script contained in the file
{\tt myscript.py} (in the current directory), you would type
\small
\begin{verbatim}
  CASA <7>: execfile('myscript.py')
\end{verbatim}
\normalsize
or
\small
\begin{verbatim}
  CASA <8>: execfile 'myscript.py'
\end{verbatim}
\normalsize
which will invoke the IPython auto-parenthesis feature.

NOTE: in some cases, you can use the IPython {\tt run} command
instead, e.g.
\small
\begin{verbatim}
  CASA <9>: run myscript.py
\end{verbatim}
\normalsize
In this case, you do not need the quotes around the filename.
This is most useful for re-initializing the task parameters,
e.g.
\small
\begin{verbatim}
  CASA <10>: run clean.last
\end{verbatim}
\normalsize
(see \S~\ref{section:intro.tasks.setpar.last}).

See Appendix~\ref{section:python.scripts} for more information.

%%%%%%%%%%%%%%%%%%%%%%%%%%%%%%%%%%%%%%%%%%%%%%%%%%%%%%%%%%%%%%%%%
\subsection{Getting Help in CASA}
\label{section:intro.basics.help}

\subsubsection{{\tt TAB} key}
\label{section:intro.basics.help.tab}

At {\bf any} time, hitting the {\tt <TAB>} key will complete any
available commands or
variable names and show you a list of the possible completions if
there's no unambiguous result. It will also complete filenames in the
current directory if no CASA or Python names match.

For example, it can be used to list the available functionality using
minimum match; once you have typed enough characters to make the
command unique, {\tt <TAB>} will complete it. 
\small
\begin{verbatim}
  CASA <15>: cle<TAB>
  clean                  clean_description      clearcal_check_params
  clearplot              clearstat
  clean_check_params     clear                  clearcal_defaults
  clearplot_defaults     clearstat_defaults
  clean_defaults         clearcal               clearcal_description
  clearplot_description  clearstat_description
\end{verbatim}
\normalsize

%%%%%%
\subsubsection{{\tt help <taskname>}}
\label{section:intro.basics.help.help}

Basic information on an application, including the parameters used and
their defaults, can be obtained by typing {\tt pdoc task}, 
{\tt help task}, {\tt help 'task'} or
{\tt task?}. The {\tt pdoc task} currently gives the cleanest
documentation format with the smallest amount of object-oriented 
(programmer) output.  This inline help provides a one line description of
the task and then lists all parameters, a brief description of the
parameter, the parameter default, an example setting the parameter and
any options if there are limited allowed values for the parameter.

For example:
\small
\begin{verbatim}
    Import VLA archive file(s) to a measurement set
    
    Imports an arbitrary number of VLA archive-format data sets into
    a casa measurement set.  If more than one band is present, they
    will be put in the same measurement set but in a separate spectral
    window.  The task will handle old style and new style VLA (after
    July 2007) archive data and apply the tsys to the data and to
    the weights.
    
    Keyword arguments:
    archivefiles -- Name of input VLA archive file(s)
            default: none.  Must be supplied
            example: archivefiles = 'AP314_A959519.xp1'
            example: archivefiles=['AP314_A950519.xp1','AP314_A950519.xp2']
    vis -- Name of output visibility file
            default: none.  Must be supplied.
            example: vis='NGC7538.ms'
            Will not over-write existing ms of same name.
            A backup flag-file version 'Original' will be made in
              vis.flagversions.  See help flagmanager
    bandname -- VLA Frequency band
            default: => '' = all bands
            example: bandname='K'
            Options: '4'=48-96 MHz,'P'=298-345 MHz,'L'=1.15-1.75 GHz,
            'C'=4.2-5.1 GHz,'X'=6.8-9.6 GHz,'U'=13.5-16.3 GHz,
            'K'=20.8-25.8 GHz,'Q'=38-51 GHz
    frequencytol -- Tolerance in frequency shift in making spectral windows
            default: => 150000 (Hz).  For Doppler shifted data, <10000 Hz may
            may produce too many unnecessary spectral windows.
            example: frequencytol = 1500000.0 (units = Hz)
    project -- Project name to import from archive files:
            default: '' => all projects in file
            example: project='AL519'
            project = 'al519' or AL519 will work.  Do not include
            leading zeros; project = 'AL0519' will not work.
    starttime -- Time after which data will be considered for importing
            default: '' => all:  Date must be included.
            syntax: starttime = '2003/1/31/05:05:23'
    stoptime --  Time before which data will be considered for importing
            default: '' => all:  Date must be included.
            syntax: stoptime = '2003/1/31/08:05:23'
    applytsys -- Apply data scaling and weight scaling by nominal
            sensitivity (~Tsys)
            default: True.  Strongly recommended
    autocorr --  import autocorrelations to ms
            default:  => False (no autocorrelations)
    antnamescheme -- 'old' or 'new' antenna names.
            default => 'new' gives antnenna names
              'VA04' or 'EA13 for VLA telescopse 04 and 13 (EVLA)
              'old' gives names '04' or '13'
    keepblanks -- Should sources with blank names be filled into the data base
            default => false.  Do not fill
            These scans are tipping scans (as of June 1, 2009) and should not
            be filled in the visibility data set.
    evlabands -- Use the EVLA's center frequency and bandwidths for frequencies
            specified via wavelength or band. 
            default => True. 
    async --  Run asynchronously
            default = False; do not run asychronously

\end{verbatim}
\normalsize

You can also get the short help for a CASA tool method by typing 'help tool.method'.
\small  
\begin{verbatim}
CASA <46>: help ia.subimage
Help on built-in function subimage:

subimage(...)
    Create a (sub)image from a region of the image
    --- --- --- --- --- --- Parameters  --- --- --- --- --- ---
      outfile:  Output image file name. Default is unset. 
      region:  Region of interest. Default is whole image. 
      mask:  OTF mask, Boolean LEL expression or mask region. Default is none. 
      dropdeg:  Drop degenerate axes false 
      overwrite:  Overwrite (unprompted) pre-existing output file? false 
      list:  List informative messages to the logger true 
    --- --- --- --- --- --- --- --- --- --- --- --- --- --- --- 
          outfile
          region
          mask
          dropdeg   = false
          overwrite = false
          list      = true
        ----------------------------------------
\end{verbatim}
\normalsize

For a full list of keywords associated with the various tools,
see the {\bf CASA User Reference Manual}.  

%%%%%%
\subsubsection{{\tt help } and {\tt PAGER}}
\label{section:intro.basics.help.page}

Your {\tt PAGER} environment variable (\S~\ref{section:intro.basics.before})
determines how help is displayed
in the terminal window where you start CASA. If you set your {\tt
bash} environment variable {\tt PAGER=less} ({\tt setenv PAGER less}
in {\tt csh}) then typing {\tt help <taskname>} will show you the help
but the text will vanish and return you to the command line when you
are done viewing it. Setting {\tt PAGER=more} ({\tt setenv PAGER
more}) will scroll the help onto your command window and then return
you to your prompt (but leaving it on display). Setting {\tt
PAGER=cat} ({\tt setenv PAGER cat}) will give you the {\tt more}
equivalent without some extra formatting baggage and is the
recommended choice.

If you have set {\tt PAGER=more}
or {\tt PAGER=less}, the {\tt help} display will be fine, but the display of
'taskname?' will often have confusing formatting content at the
beginning (lots of {\tt ESC} surrounding the text). This can be remedied
by exiting casapy and doing an '{\tt unset PAGER}' 
({\tt unsetenv PAGER} in [t]csh) at the Unix command line.

You can see the current value of the {\tt PAGER} environment variable
with CASA by typing:
\small
\begin{verbatim}
  !echo $$PAGER
\end{verbatim}
\normalsize
(note the double {\tt \$\$}).  This will show what command paging is
pointed to.

%%%%%%
\subsubsection{{\tt help par.<parameter>}}
\label{section:intro.basics.help.par}

Typing {\tt help par.<parameter>} provides a brief description of a 
given parameter {\tt <parameter>}.
\small
\begin{verbatim}
  CASA <46>: help par.robust
  Help on function robust in module parameter_dictionary:

  robust()
    Brigg's robustness parameter.

    Options: -2.0 (close to uniform) to 2.0 (close to natural)
\end{verbatim}
\normalsize

%%%%%%
\subsubsection{Python {\tt help}}
\label{section:intro.basics.help.py}

Typing {\tt help} at the casapy prompt with no arguments will bring
up the native Python help facility, and give you the
{\tt help>} prompt for further information; hitting {\tt <RETURN>} at the help
prompt returns you to the CASA prompt.
\small
\begin{verbatim}
  CASA <2>: help
  --------> help()

  Welcome to Python 2.5!  This is the online help utility.

  If this is your first time using Python, you should definitely check out
  the tutorial on the Internet at http://www.python.org/doc/tut/.

  Enter the name of any module, keyword, or topic to get help on writing
  Python programs and using Python modules.  To quit this help utility and
  return to the interpreter, just type "quit".

  To get a list of available modules, keywords, or topics, type "modules",
  "keywords", or "topics".  Each module also comes with a one-line summary
  of what it does; to list the modules whose summaries contain a given word
  such as "spam", type "modules spam".

  help> keywords

  Here is a list of the Python keywords.  Enter any keyword to get more
  help.

  and                 else                import              raise
  assert              except              in                  return
  break               exec                is                  try
  class               finally             lambda              while
  continue            for                 not                 yield
  def                 from                or                  
  del                 global              pass                
  elif                if                  print               

  help>

  # hit <RETURN> to return to CASA prompt

  You are now leaving help and returning to the Python interpreter.
  If you want to ask for help on a particular object directly from the
  interpreter, you can type "help(object)".  Executing
  "help('string')" has the same effect as typing a particular string
  at the help> prompt.
\end{verbatim}
\normalsize

Further help in working within
the Python shell is given in Appendix~\ref{chapter:python}.

%%%%%%%%%%%%%%%%%%%%%%%%%%%%%%%%%%%%%%%%%%%%%%%%%%%%%%%%%%%%%%%%%
%%%%%%%%%%%%%%%%%%%%%%%%%%%%%%%%%%%%%%%%%%%%%%%%%%%%%%%%%%%%%%%%%

\section{Tasks and Tools in CASA}
\label{section:intro.tasks}

Originally, CASA consisted of a collection of tools, combined in the
so-called toolkit.  Since the majority of prospective users is far
more familiar with the concept of tasks, an effort is underway to
replace most - if not all - toolkit functionality by tasks.

While running CASA, you will have access to and be interacting with
tasks, either indirectly by providing parameters to a task, or
directly by running a task.  Each task has a well defined purpose, and
a number of associated parameters, the values of which are to be
supplied by the user.  Technically speaking, tasks are built on top of
tools - when you are running a task, you are running tools in the
toolkit, though this should be transparent.

As more tasks are being written, and the functionality of each task is
enhanced, there will be less and less reason to run tools in the
toolkit.  We are working toward a system in which direct access to the
underlying toolkit is unnecessary for all standard data processing.

\subsection{What Tasks are Available?}
\label{section:intro.tasks.more}

As mentioned in the introduction, tasks in CASA are python interfaces
to the more basic toolkit.  Tasks are executed to perform a single
job, such as loading, plotting, flagging, calibrating, and imaging the
data.

Basic information on tasks, including the parameters used and their
defaults, can be obtained by typing {\tt help <taskname>} or {\tt
<taskname>?} at the CASA prompt, where {\tt <taskname>} is the name of
a given task.  As described above in \S~\ref{section:intro.basics.help.help},
{\tt help <taskname>} provides a description of the task
and then lists all parameters, a brief description of the parameter,
the parameter default, an example setting the parameter and any
options if there are limited allowed values for the parameter.

To see what tasks are available in CASA, use {\tt tasklist}, e.g.
\small
\begin{verbatim}
CASA <3>: tasklist
---------> tasklist()
Available tasks, organized by category (experimental tasks in parenthesis ()
  deprecated tasks in curly brackets {}).


Import/export       Information         Editing             Manipulation      
------------------  ------------------  ------------------  ------------------
exportasdm          imhead              fixplanets          concat            
exportfits          imreframe           fixvis              conjugatevis      
exportuvfits        imstat              flagcmd             cvel              
importasdm          imval               flagdata            fixvis            
importfits          listcal             flagmanager         hanningsmooth     
importfitsidi       listfits            msview              imhead            
importuvfits        listhistory         plotms              msmoments         
importvla           listobs             plotxy              plotms            
(importevla)        listvis                                 plotxy            
(importgmrt)        plotms                                  split             
                    plotuv                                  testconcat        
                    plotxy                                  uvcontsub         
                    vishead                                 virtualconcat     
                    visstat                                 vishead           
                    (asdmsummary)                           (mstransform)     
                    (listpartition)                         (partition)       
                    (listsdm)                               (statwt)          
                    (makemask)                              (uvcontsub3)      
                                                            {uvcontsub2}      

Calibration         Modeling            Imaging             Analysis          
------------------  ------------------  ------------------  ------------------
accum               predictcomp         clean               imcollapse        
applycal            setjy               deconvolve          imcontsub         
bandpass            uvcontsub           feather             imfit             
blcal               uvmodelfit          ft                  imhead            
calstat             uvsub               imcontsub           immath            
clearcal            (uvcontsub3)        (boxit)             immoments         
delmod              {uvcontsub2}        (csvclean)          impbcor           
fixplanets                              (tclean)            impv              
fluxscale                               (widebandpbcor)     imreframe         
ft                                      {mosaic}            imregrid          
gaincal                                 {widefield}         imsmooth          
gencal                                                      imstat            
listcal                                                     imsubimage        
plotants                                                    imtrans           
plotbandpass                                                imval             
plotcal                                                     listvis           
polcal                                                      slsearch          
predictcomp                                                 splattotable      
setjy                                                       (specfit)         
smoothcal                                                                     
uvmodelfit                                                                    
uvsub                                                                         
wvrgcal                                                                       

Visualization       Simulation          Single dish         Utility           
------------------  ------------------  ------------------  ------------------
clearplot           simanalyze          asap_init           browsetable       
imview              simobserve          sdbaseline          caltabconvert     
msview              (simalma)           sdcal               clearplot         
plotants                                sdcal2              clearstat         
plotbandpass                            sdcoadd             concat            
plotcal                                 sdfit               conjugatevis      
plotms                                  sdflag              find              
plotuv                                  sdflagmanager       help par.parameter
plotxy                                  sdgrid              help taskname     
viewer                                  sdimaging           imview            
(plotweather)                           sdimprocess         msview            
                                        sdlist              plotms            
                                        sdmath              rmtables          
                                        sdplot              startup           
                                        sdreduce            taskhelp          
                                        sdsave              tasklist          
                                        sdscale             testconcat        
                                        sdsmooth            toolhelp          
                                        sdstat              virtualconcat     
                                        sdtpimaging                           

User defined tasks
------------------


\end{verbatim}
\normalsize The tasks with name in parentheses are experimental, those
in curly brackets are deprecated and will be removed in future
releases. The functionality of deprecated tasks is usually available
in some other task (e.g., instead of {\tt mosaic} one should use {\tt
  clean}). In the above case, the user has not defined any task
him/herself.

Typing {\tt taskhelp} provides a one line description of all available
tasks.
\small
\begin{verbatim}
CASA <4>: taskhelp
--------> taskhelp()
Available tasks: 
Available tasks: 

accum         : Accumulate incremental calibration solutions into a calibration table
applycal      : Apply calibrations solutions(s) to data
asdmsummary   : Summarized description of an ASDM dataset.
autoclean     : CLEAN an image with automatically-chosen clean regions.
bandpass      : Calculates a bandpass calibration solution
blcal         : Calculate a baseline-based calibration solution (gain or bandpass)
boxit         : Box regions in image above given threshold value.
browsetable   : Browse a table (MS, calibration table, image)
calstat       : Displays statistical information on a calibration table
caltabconvert : Convert old-style caltables into new-style caltables.

clean         : Invert and deconvolve images with selected algorithm
clearcal      : Re-initializes the calibration for a visibility data set
clearplot     : Clear the matplotlib plotter and all layers
clearstat     : Clear all autolock locks
concat        : Concatenate several visibility data sets.
conjugatevis  : Change the sign of the phases in all visibility columns.
csvclean      : This task does an invert of the visibilities and deconvolve in the image plane.
cvel          : regrid an MS to a new spectral window / channel structure or frame
deconvolve    : Image based deconvolver
delmod        : Deletes model representations in the MS
exportasdm    : Convert a CASA visibility file (MS) into an ALMA Science Data Model
exportfits    : Convert a CASA image to a FITS file
exportuvfits  : Convert a CASA visibility data set to a UVFITS file:
feather       : Combine two images using their Fourier transforms
find          : Find string in tasks, task names, parameter names:
fixplanets    : Changes FIELD and SOURCE table entries based on user given direction or POINTING table, optionally fixes the UVW coordinates
fixvis        : Recalculates (u, v, w) and/or changes Phase Center 
flagcmd       : Flagging task based on batches of flag-commands
flagdata      : All-purpose flagging task based on data-selections and flagging modes/algorithms.
flagmanager   : Enable list, save, restore, delete and rename flag version files.
fluxscale     : Bootstrap the flux density scale from standard calibrators
ft            : Insert a source model  a visibility set:
gaincal       : Determine temporal gains from calibrator observations
gencal        : Specify Calibration Values of Various Types
hanningsmooth : Hanning smooth frequency channel data to remove Gibbs ringing
imcollapse    : Collapse image along one axis, aggregating pixel values along that axis.
imcontsub     : Estimates and subtracts continuum emission from an image cube
imfit         : Fit one or more elliptical Gaussian components on an image region(s)
imhead        : List, get and put image header parameters
immath        : Perform math operations on images
immoments     : Compute moments from an image
impbcor       : Construct a primary beam corrected image from an image and a primary beam pattern.
importasdm    : Convert an ALMA Science Data Model observation into a CASA visibility file (MS) or single-dish data format (Scantable)
importevla    : Convert an Science Data Model observation into a CASA Measurement Set
importfits    : Convert an image FITS file into a CASA image
importfitsidi : Convert a FITS-IDI file to a CASA visibility data set
importgmrt    : Convert a UVFITS file to a CASA visibility data set
importuvfits  : Convert a UVFITS file to a CASA visibility data set
importvla     : Import VLA archive file(s) to a measurement set
impv          : Construct a position-velocity image by choosing two points in the direction plane.
imreframe     : Change the frame in which the image reports its spectral values
imregrid      : regrid an image onto a template image
imsmooth      : Smooth an image or portion of an image
imstat        : Displays statistical information from an image or image region
imsubimage    : Create a (sub)image from a region of the image
imtrans       : Reorder image axes
imval         : Get the data value(s) and/or mask value in an image.
imview        : View an image
listcal       : List antenna gain solutions
listfits      : List the HDU and typical data rows of a fits file:
listhistory   : List the processing history of a dataset:
listobs       : List the summary of a data set in the logger or in a file
listpartition : List the summary of a multi-MS data set in the logger or in a file
listsdm       : Lists observation information present in an SDM directory.
listvis       : List measurement set visibilities.
makemask      : Makes and manipulates image masks
mosaic        : Create a multi-field deconvolved image with selected algorithm
msmoments     : Compute moments from an MS
mstransform   : Task to combine/separate/regrid spws and do channel and time averaging
msview        : View a visibility data set
partition     : Experimental task to produce multi-MSs using parallelism
pclean        : Invert and deconvolve images with parallel engines
plotants      : Plot the antenna distribution in the local reference frame:
plotbandpass  : Makes detailed plots of Tsys and bandpass solutions.
plotcal       : An all-purpose plotter for calibration results 
plotms        : A plotter/interactive flagger for visibility data.
plotuv        : Plot the baseline distribution
plotweather   : Plot elements of the weather table; estimate opacity.
plotxy        : An X-Y plotter/interactive flagger for visibility data.
polcal        : Determine instrumental polarization calibrations
predictcomp   : Make a component list for a known calibrator
rmtables      : 
sdbaseline    : ASAP SD task: fit/subtract a spectral baseline 
sdcal         : ASAP SD task: do data selection, calibration, and averaging
sdcal2        : ASAP SD task: generate and/or apply caltables
sdcoadd       : ASAP SD task: coadd multiple scantables into one
sdfit         : ASAP SD task: fit a spectral line
sdflag        : ASAP SD spectral flagging task 
sdflagmanager : ASAP SD task: enable list, save, restore, delete and rename flag version files
sdgrid        : SD task: gridding single dish data
sdimaging     : SD task: imaging for total power and spectral data
sdimprocess   : SD task: task for single-dish image processing
sdlist        : ASAP SD task: list summary of single dish data
sdmath        : ASAP SD task: do simple arithmatic (subtraction, addition, multiplication, and division) for spectra
sdplot        : ASAP SD plotting task 
sdreduce      : ASAP SD task: do sdcal, sdsmooth, and sdbaseline in one task
sdsave        : ASAP SD task: save the sd spectra in various format
sdscale       : ASAP SD task: scale the sd spectra
sdsmooth      : ASAP SD task: do smoothing of spectra
sdstat        : ASAP SD task: list statistics of spectral region
sdtpimaging   : SD task: do a simple calibration and imaging for total power data 
setjy         : Fills the model column with the visibilities of a calibrator
simalma       : (Experimental) simple simulation task for ALMA 
simanalyze    : image and analyze simulated datasets
simobserve    : mosaic simulation task
slsearch      : Search a spectral line table.
smoothcal     : Smooth calibration solution(s) derived from one or more sources:
specfit       : Fit 1-dimensional gaussians and/or polynomial models to an image or image region
splattotable  : Convert a downloaded Splatalogue spectral line list to a casa table.
split         : Create a visibility subset from an existing visibility set
ssoflux       : Fills the model column with the visibilities of a calibrator
statwt        :  Reweight visibilities according to their scatter (Experimental)
tclean        : Construct and deconvolve images
testconcat    : Concatenate the subtables of several visibility data sets, not the MAIN bulk data.
uvcontsub     : Continuum fitting and subtraction in the uv plane
uvcontsub2    : Temporarily a clone of uvcontsub
uvcontsub3    : An experimental clone of uvcontsub
uvmodelfit    : Fit a single component source model to the uv data
uvsub         : Subtract/add model from/to the corrected visibility data.
viewer        : View an image or visibility data set
virtualconcat : Concatenate several visibility data sets into a multi-MS
vishead       : List, summary, get, and put metadata in a measurement set
visstat       : Displays statistical information from a measurement set, or from a measurement set selection
widebandpbcor : Wideband PB-correction on the output of the MS-MFS algorithm
widefield     : Wide-field imaging and deconvolution with selected algorithm
wvrgcal       : Generate a gain table based on Water Vapour Radiometer data

\end{verbatim}
\normalsize

Typing {\tt startup} will provide the startup page displayed when
entering CASA. The startup screen lists the various options to obtain
help within CASA.
\small
\begin{verbatim}

CASA <26>: startup
---------> startup()
___________________________________________________________________
    For help use the following commands:
    tasklist               - Task list organized by category
    taskhelp               - One line summary of available tasks
    help taskname          - Full help for task
    toolhelp               - One line summary of available tools
    help par.parametername - Full help for parameter name
___________________________________________________________________


\end{verbatim}
\normalsize

%%NOTE: THESE EXAMPLES WERE UPDATED 2010-02-24

%%%%%%%%%%%%%%%%%%%%%%%%%%%%%%%%%%%%%%%%%%%%%%%%%%%%%%%%%%%%%%%%%
\subsection{Running Tasks and Tools}
\label{section:intro.tasks.run}

Tools are functions linked to the Python interface which must be
called by name with arguments.  Tasks have higher-level capabilities
than tools.  Tasks require input parameters which maybe be specified
when you call the task as a function, or be set as parameters in the
interface.  A task, like a tool, is a function under Python and may be
written in Python, C, or C++ (the CASA toolkit is made up of C++
functions).

There are two distinct ways to run tasks. You can either set the
global CASA parameters relevant to the task and
tell the task to ``go'', or you can call the task as a function with
one or more arguments specified.  These two invocation methods differ
in whether the global parameter values are used or not.

For example,
\small
\begin{verbatim}
    default('plotxy')
    vis='ngc5921.ms'
    xaxis='channel'
    yaxis='amp'
    datacolumn='data'
    go
\end{verbatim}
\normalsize
will execute {\tt plotxy} with the set values for the parameters
(see \S~\ref{section:intro.tasks.setpar}).  Instead of using 
{\tt go} command (\S~\ref{section:intro.tasks.setpar.go}) to invoke
the task, you can also call the task with no arguments, e.g.
\small
\begin{verbatim}
    default('plotxy')
    vis='ngc5921.ms'
    xaxis='channel'
    yaxis='amp'
    datacolumn='data'
    plotxy()
\end{verbatim}
\normalsize
which will also use the global parameter values.

Second, one may call tasks and tools by name with parameters set
on the same line.  Parameters may be set either as explicit
{\tt <parameter>=<value>} arguments, or as a series of comma delimited
{\tt <value>}s in the correct order for that task or tool.  Note that missing
parameters will {\em use the default values for that task}.
For example, the following are equivalent:
\small
\begin{verbatim}
  # Specify parameter names for each keyword input: 
    plotxy(vis='ngc5921.ms',xaxis='channel',yaxis='amp',datacolumn='data')
  # when specifying the parameter name, order doesn't matter, e.g.:
    plotxy(xaxis='channel',vis='ngc5921.ms',datacolumn='data',yaxis='amp')
  # use parameter order for invoking tasks
    plotxy('ngc5921.ms','channel','amp','data')
\end{verbatim}
\normalsize
This non-use of globals when calling as a function is so that robust 
scripts can be written.  One need only cut-and-paste the calls and
need not worry about the state of the global variables or what has
been run previously.  It is also more like the standard behavior of
function calls in Python and other languages.

Tools can only be called in this second manner by name, with
arguments (\S~\ref{section:intro.tasks.tools}).  Tools never use the
global parameters and the related mechanisms of {\tt inp} and {\tt go}.

%%%%%%%
\subsubsection{Aborting Synchronous Tasks}
\label{section:intro.tasks.run.abort}

If you are running CASA tasks synchronously, then you can usually use
{\tt CNTL-C} to abort execution of the task.  If this does not work,
try {\tt CNTL-Z} followed by a {\tt kill}.
See \S~\ref{section:intro.basics.wrong.abort} for more on these
methods to abort CASA execution.

You may have to quit and restart CASA after an abort, as the internal
state can get mixed up.

%%%%%%%%%%%%%%%%%%%%%%%%%%%%%%%%%%%%%%%%%%%%%%%%%%%%%%%%%%%%%%%%%
\subsection{Getting Return Values}
\label{section:intro.tasks.return}

Some tasks and tools return a record (usually a Python dictionary)
to the interface.  For example, the {\tt imstat} task 
(\S~\ref{section:analysis.imstat}) returns a dictionary with the
image statistics in it.  To catch these return values into a
Python variable, you MUST assign that variable to the task call, e.g.
\small
\begin{verbatim}
   xstat = imstat('ngc5921.clean.image')
\end{verbatim}
\normalsize
or 
\small
\begin{verbatim}
   default('imstat')
   imagename = 'ngc5921.clean.image'
   xstat = imstat()
\end{verbatim}
\normalsize
Note that tools that return values work in the same way
(\S~\ref{section:intro.tasks.tools}).

You can print or use the return value in Python for controlling
scripts.  For example,
\small
\begin{verbatim}
CASA <1>: xstat = imstat('ngc5921.clean.image')
CASA <2>: xstat
  Out[2]: 
{'blc': array([0, 0, 0, 0]),
 'blcf': '15:24:08.404, +04.31.59.181, I, 1.41281e+09Hz',
 'flux': array([ 4.15292207]),
 'max': array([ 0.05240594]),
 'maxpos': array([134, 134,   0,  38]),
 'maxposf': '15:21:53.976, +05.05.29.998, I, 1.41374e+09Hz',
 'mean': array([  1.62978083e-05]),
 'medabsdevmed': array([ 0.00127287]),
 'median': array([ -1.10467618e-05]),
 'min': array([-0.0105249]),
 'minpos': array([160,   1,   0,  30]),
 'minposf': '15:21:27.899, +04.32.14.923, I, 1.41354e+09Hz',
 'npts': array([ 3014656.]),
 'quartile': array([ 0.00254587]),
 'rms': array([ 0.00201818]),
 'sigma': array([ 0.00201811]),
 'sum': array([ 49.1322855]),
 'sumsq': array([ 12.27880404]),
 'trc': array([255, 255,   0,  45]),
 'trcf': '15:19:52.390, +05.35.44.246, I, 1.41391e+09Hz'}
CASA <3>: myrms = xstat['rms'][0]
CASA <4>: print 10.0*myrms
0.0201817648485
\end{verbatim}
\normalsize

If you do not catch the return variable, it will be lost
\small
\begin{verbatim}
   imstat('ngc5921.clean.image')
\end{verbatim}
\normalsize
 or
\small
\begin{verbatim}
   default('imstat')
   imagename = 'ngc5921.clean.image'
   imstat()
\end{verbatim}
\normalsize
and spewed to terminal.  Note that {\tt go} will trap and lose
the return value, e.g.
\small
\begin{verbatim}
   default('imstat')
   imagename = 'ngc5921.clean.image'
   go
\end{verbatim}
\normalsize
will not dump the return to the terminal either.

NOTE: You cannot currently catch a return value from a task run
asynchronously (\S~\ref{section:intro.tasks.async}).

%%%%%%%
\subsection{Running Tasks Asynchronously}
\label{section:intro.tasks.async}

By default, most tasks run synchronously in the foreground.  Many
tasks, particularly those that can take a long time to execute,
have the {\tt async} parameter.  This allows the user to send the
task to the background for execution.

\subsubsection{Monitoring Asynchronous Tasks}
\label{section:intro.tasks.async.tm}

% \begin{wrapfigure}{r}{2.5in}
%   \begin{boxedminipage}{2.5in}
%      \centerline{\underline{\bf Inside the Toolkit:}}
%      The {\tt tm} tool is the manager for asynchronous
%      processes.  There is currently no task equivalent.
%   \end{boxedminipage}
% \end{wrapfigure}

\begin{wrapfigure}{r}{2.5in}
  \begin{boxedminipage}{2.5in}
     \centerline{\bf ALERT:}
     You should not use the {\tt go} command to run a task
     asynchronously, as the ``handle'' will be swallowed by
     the Python task wrapper and you will not be able to
     access it with {\tt tm}.  This is also true if you run
     in a Python script.
  \end{boxedminipage}
\end{wrapfigure}

There is a ``taskmanager'' tool {\tt tm} that allows the user
to retrieve the status of, and to abort the execution of, tasks
running with {\tt async=True} in the background.  There
are two methods of interest for the user, {\tt tm.retrieve} and
{\tt tm.abort}.

If you run a task with {\tt async=True} then several things will
happen.  First of all, the task returns a ``handle'' that is 
a number used to identify the process.  This is printed to the 
screen, e.g. 
\small
\begin{verbatim}
CASA <5>: inp()
#  mosaic :: Calculate a multi-field deconvolved image with selected clean algorithm:
...
async               =       True        #   if True run in the background, prompt is freed

CASA <6>: mosaic()
Connecting to controller:  ('127.0.0.1', 60775)
  Out[6]: 0
\end{verbatim}
\normalsize
where the output value {\tt 0} is the handle id.  

You can also catch the return value in a variable, e.g.
\small
\begin{verbatim}
CASA <7>: handle = mosaic()
...
CASA <8>: print handle
1
\end{verbatim}
\normalsize
You should also see the usual messages from the task in the 
{\tt logger}, with some extra lines of information
\small
\begin{verbatim}
###############################################
###  Begin Task: mosaic  ###
Tue Oct 2 17:58:16 2007    NORMAL ::mosaic:
""
"Use: "
tm.abort(return_value)    # to abort the asynchronous task
tm.retrieve(return_value) # to retrieve the status
""
... usual messages here ...

###  End Task: mosaic  ###
###############################################
""
\end{verbatim}
\normalsize
for the example above.

To show the current status of an asynchronous task, use the
{\tt tm.retrieve} method using the handle id as the argument.
For example,
\small
\begin{verbatim}
CASA <9>: tm.retrieve(handle)
  Out[9]: {'result': None, 'status': 'pending'}
\end{verbatim}
\normalsize
or
\small
\begin{verbatim}
CASA <10>: tm.retrieve(1)
  Out[10]: {'result': None, 'status': 'pending'}
\end{verbatim}
\normalsize
which means its still running.  You should be seeing output in
the {\tt logger} also while the task is running.  

When a task is finished, you will see:
\small
\begin{verbatim}
CASA <11>: tm.retrieve(1)
  Out[11]: {'result': None, 'status': 'done'}
\end{verbatim}
\normalsize
which indicates completion.

%%%%%%%
\subsubsection{Aborting Asynchronous Tasks}
\label{section:intro.tasks.async.abort}

To abort a task while it is running in the background, use the
{\tt tm.abort} method, again with the task handle id as the
argument.  For example,
\small
\begin{verbatim}
CASA <12>: handle = mosaic()
...
CASA <13>: tm.abort(handle)
\end{verbatim}
\normalsize
will abort the task if it is running.

If this does not work,
try {\tt CNTL-Z} followed by a {\tt kill -9 <PID>} for the appropriate
process ID.
See \S~\ref{section:intro.basics.wrong.abort} for more on these
methods to abort CASA execution.

{\bf ALERT:} Currently, this is only available with the 
{\tt tm} tool.  We are working on a {\tt taskmanager} task.

%%%%%%%%%%%%%%%%%%%%%%%%%%%%%%%%%%%%%%%%%%%%%%%%%%%%%%%%%%%%%%%%%
\subsection{Setting Parameters and Invoking Tasks}
\label{section:intro.tasks.setpar}

\begin{wrapfigure}{r}{2.5in}
  \begin{boxedminipage}{2.5in}
     \centerline{\bf Inside the Toolkit:}
     In the current version of CASA, you cannot use the
     task parameter setting features, such as the {\tt inp},
     {\tt default}, or {\tt go} commands, for the tools.
  \end{boxedminipage}
\end{wrapfigure}

One can set parameters for tasks (but not for tools) by
performing the assignment within the CASA shell and then inspecting
them using the {\tt inp} command:
\small
\begin{verbatim}
CASA <30>: default(bandpass)
CASA <31>: vis = 'ngc5921.demo.ms'
CASA <32>: caltable = 'ngc5921.demo.bcal'
CASA <33>: field = '0'
CASA <34>: refant = '15'
CASA <35>: inp('bandpass')
#  bandpass :: Calculates a bandpass calibration solution
vis         = 'ngc5921.demo.ms' #  Nome of input visibility file
caltable    = 'ngc5921.demo.bcal' #  Name of output gain calibration table
field       =        '0'        #  Select field using field id(s) or field name(s)
spw         =         ''        #  Select spectral window/channels
selectdata  =      False        #  Other data selection parameters
solint      =      'inf'        #  Solution interval
combine     =     'scan'        #  Data axes which to combine for solve (scan, spw, field)
refant      =       '15'        #  Reference antenna name
minblperant =          4        #  Minimum baselines _per antenna_ required for solve
solnorm     =      False        #  Normalize average solution amplitudes to 1.0 (G, T only)
bandtype    =        'B'        #  Type of bandpass solution (B or BPOLY)
  fillgaps  =          0        #  Fill flagged solution channels by interpolation
append      =      False        #  Append solutions to the (existing) table
gaintable   =         ''        #  Gain calibration table(s) to apply on the fly
gainfield   =         ''        #  Select a subset of calibrators from gaintable(s)
interp      =         ''        #  Interpolation mode (in time) to use for each gaintable
spwmap      =         []        #  Spectral windows combinations to form for gaintables(s)
gaincurve   =      False        #  Apply internal VLA antenna gain curve correction
opacity     =        0.0        #  Opacity correction to apply (nepers)
parang      =      False        #  Apply parallactic angle correction
async       =      False        #  
\end{verbatim}
\normalsize
See \S~\ref{section:intro.tasks.setpar.inp} below for more details on
the use of the {\tt inputs} command.

% In summary, for tasks, one can
% either (1) assign parameters, one at a time, from the CASA prompt, and
% then execute the task; (2) execute the task in one line, specifically
% assigning each parameter within the line; or (3) using the position
% within the task call to specify the value of the parameter.

All task parameters have {\bf global} scope within CASA: the
parameter values are common to all tasks and also at the CASA command
line. This allows the convenience of not changing parameters that are
shared between tasks but does require care when chaining together
sequences of task invocations (to ensure proper values are provided).

If you want to reset the input keywords for a single task, use the
{\tt default} command (\S~\ref{section:intro.tasks.setpar.default}).  
For example, to set the defaults for the {\tt bandpass} task, type:
\small
\begin{verbatim}
CASA <30>: default('bandpass')
\end{verbatim}
\normalsize
as we did above.

To inspect a single parameter value just type it at the command line.
Continuing the above example:
\small
\begin{verbatim}
CASA <36>: combine
  Out[14]: 'scan'
\end{verbatim}
\normalsize
CASA parameters are just Python variables.

Parameters for a given task can be saved by using the {\tt saveinputs}
command (see \S~\ref{section:intro.tasks.setpar.saveinputs})
and restored using the {\tt execfile '<filename>'} command.  Note that
if the task is successfully executed, then a {\tt <taskname>.last}
file is created in the working directory containing the parameter
values (see \S~\ref{section:intro.tasks.setpar.last}).

We now describe the individual CASA task parameter interface commands
and features in more detail.

%%%%%%
\subsubsection{The scope of parameters in CASA}
\label{section:intro.tasks.setpar.scope}

\begin{wrapfigure}{r}{2.5in}
  \begin{boxedminipage}{2.5in}
     \centerline{\underline{\bf Advanced Tip}}
     By default, the scope of CASA parameters is global, as stated
     here.  However, if you call a task as a function with one or
     more arguments specified, e.g. {\tt task(arg1=val1,...)}, then
     non-specified parameters will be defaulted and no globals used.
     This makes scripting more robust.  Tasks DO NOT change the value
     of globals.
  \end{boxedminipage}
\end{wrapfigure}
All task parameters have {\bf global} scope within CASA: the
parameter values are common to all tasks and also at the CASA command
line. This allows the convenience of not changing parameters that are
shared between tasks but does require care when chaining together
sequences of task invocations (to ensure proper values are provided).
Tasks DO NOT change the values of the global parameters, nor does the
invocation of tasks using the functional call with arguments change
the globals.

This does mean that unless you do an explicit {\tt default} of the task
(\S~\ref{section:intro.tasks.setpar.default}), previously set values
may be unexpectedly used if you do not inspect the {\tt inp}
carefully.  For example, good practice is:
\small
\begin{verbatim}
   default('imhead')
   imagename = 'ngc5921.demo.cleanimg.image'
   mode = 'list'
   imhead()
\end{verbatim}
\normalsize

If you supply the task call with arguments, then these will be used
for the values of those parameters (see above).  However, if some but
not all arguments are supplied, then those parameters not given as
arguments will default and NOT use the current global values.  Thus,
\small
\begin{verbatim}
   imhead('ngc5921.demo.cleanimg.image',mode='list')
\end{verbatim}
\normalsize
will reproduce the above.

%%%%%%
\subsubsection{The {\tt default} Command}
\label{section:intro.tasks.setpar.default}

Each task has a special set of default parameters defined for its
parameters.  You can use the {\tt default} command to reset the
parameters for a specified task (or the current task as defined
by the {\tt taskname} variable) to their default. 

{\bf Important Note:} The {\tt default} command resets the values
of the task parameters to a set of ``defaults'' as specified in
the task code.  Some defaults are blank strings {\tt ''} or 
empty lists {\tt []}, others are specific numerical values, strings,
or lists.  It is important to understand that just setting a string
parameter to an empty string {\tt ''} is not setting it to its 
default!  Some parameters do not have a blank as an allowed value.
See the {\tt help} for a particular task to find out its default.
If {\tt ''} is the default or an allowed value, it will say so
explicitly.

For example, suppose we have been running CASA on a particular
dataset, e.g.
\small
\begin{verbatim}
CASA <40>: inp clean
---------> inp('clean')
#  clean :: Deconvolve an image with selected algorithm
vis                 = 'ngc5921.demo.src.split.ms.contsub' #  name of input visibility file
imagename           = 'ngc5921.demo.cleanimg' #  Pre-name of output images
field               =        '0'        #  Field Name
spw                 =         ''        #  Spectral windows:channels: '' is all
selectdata          =      False        #  Other data selection parameters
mode                =  'channel'        #  Type of selection (mfs, channel, velocity, frequency)
     nchan          =         46        #  Number of channels (planes) in output image
     start          =          5        #  first input channel to use
     width          =          1        #  Number of input channels to average
     interpolation  =  'nearest'        #  Spectral interpolation (nearest, linear, cubic)
niter               =       6000        #  Maximum number of iterations
...
\end{verbatim}
\normalsize
and now we wish to switch to a different one.  We can reset the
parameter values using {\tt default}:
\small
\begin{verbatim}
CASA <41>: default
---------> default()

CASA <42>: inp
---------> inp()
#  clean :: Deconvolve an image with selected algorithm
vis                 =         ''        #  name of input visibility file
imagename           =         ''        #  Pre-name of output images
field               =         ''        #  Field Name
spw                 =         ''        #  Spectral windows:channels: '' is all
selectdata          =      False        #  Other data selection parameters
mode                =      'mfs'        #  Type of selection (mfs, channel, velocity, frequency)
niter               =        500        #  Maximum number of iterations
...
\end{verbatim}
\normalsize

It is good practice to use {\tt default} before running a task
if you are unsure what state the CASA global variables are in.

{\bf ALERT:} You currently can only reset ALL of the parameters
for a given task to their defaults.  In an upcoming update we will
allow the {\tt default} command to take a second argument with a
specific parameter to default its value.

%%%%%%
\subsubsection{The {\tt go} Command}
\label{section:intro.tasks.setpar.go}

You can execute a task using the {\tt go} command, either explicitly
\small
\begin{verbatim}
CASA <44>: go listobs
---------> go(listobs)
Executing:  listobs()
...
\end{verbatim}
\normalsize
or implicitly
if {\tt taskname} is defined (e.g. by previous use of {\tt default}
or {inp})
\small
\begin{verbatim}
CASA <45>: taskname = 'clean'
CASA <46>: go
---------> go()
Executing:  clean()
...
\end{verbatim}
\normalsize

You can also execute a task simply by typing the taskname.
\small
\begin{verbatim}
CASA <46>: clean
---------> clean()
Executing:  clean()
...
\end{verbatim}
\normalsize

The {\tt go} command can also be used to launch a different task without
changing the current {\tt taskname}, without disrupting the 
{\tt inp} process on the current task you are working on.  For
example
\small
\begin{verbatim}
   default 'gaincal' # set current task to gaincal and default
   vis = 'n5921.ms'  # set the working ms
   ...               # set some more parameters
   go listobs        # launch listobs w/o chaning current task
   inp               # see the inputs for gaincal (not listobs!)
\end{verbatim}
\normalsize
{\bf ALERT:} Doing {\tt go listobs(vis='foo.ms')} will currently
change the {\tt taskname}, and will change {\tt vis}, which might
not be what is desired.

%%%%%%
\subsubsection{The {\tt inp} Command}
\label{section:intro.tasks.setpar.inp}

You can set the values for the parameters for tasks (but currently not
for tools) by performing the assignment within the CASA shell and then
inspecting them using the {\tt inp} command.  This command can be
invoked in any of three ways: 
via function call {\tt inp('<taskname>')} or {\tt inp(<taskname>)}, 
without parentheses {\tt inp '<taskname>'} or {\tt inp <taskname>},
or using the current {\tt taskname} variable setting with {\tt inp}.
For example,
\small
\begin{verbatim}
CASA <1>: inp('clean')
...
CASA <2>: inp 'clean'
----------> inp('clean')
...
CASA <3>: inp(clean)
...
CASA <4>: inp clean
----------> inp(clean)
...
CASA <5>: taskname = 'clean'
CASA <6>: inp
----------> inp()
\end{verbatim}
\normalsize
all do the same thing. 

% NOTE: the version without quotes work because {\tt <taskname>} points
% to the task objects. But...
% 
% WARNING: be careful not to set {\tt taskname = clean} without the
% quotes to make a string {\tt 'clean'}, otherwise you will set
% the variable {\tt taskname} to point to the object {\tt clean}
% for the task.  Its not clear that there are any bad effects from
% this (you can run {\tt inp} as normal) but it is bad form!



When you invoke the task inputs via {\tt inp}, you see a list of
the parameters, their current values, and a short description of what
that parameters does.  For example, starting from the default values,
\small
\begin{verbatim}
CASA <18>: inp('clean')
#  clean :: Deconvolve an image with selected algorithm
vis                 =         ''        #  name of input visibility file
imagename           =         ''        #  Pre-name of output images
field               =         ''        #  Field Name
spw                 =         ''        #  Spectral windows:channels: '' is all
selectdata          =      False        #  Other data selection parameters
mode                =      'mfs'        #  Type of selection (mfs, channel, velocity, frequency)
niter               =        500        #  Maximum number of iterations
gain                =        0.1        #  Loop gain for cleaning
threshold           =   '0.0mJy'        #  Flux level to stop cleaning.  Must include units
psfmode             =    'clark'        #  method of PSF calculation to use during minor cycles
imagermode          =         ''        #  Use csclean or mosaic.  If '', use psfmode
multiscale          =         []        #  multi-scale deconvolution scales (pixels)
interactive         =      False        #  use interactive clean (with GUI viewer)
mask                =         []        #  cleanbox(es), mask image(s), and/or region(s)
imsize              = [256, 256]        #  x and y image size in pixels
cell                = ['1.0arcsec', '1.0arcsec'] #  x and y cell size. default unit arcsec
phasecenter         =         ''        #  Image phase center: position or field index
restfreq            =         ''        #  rest frequency to assign to image (see help)
stokes              =        'I'        #  Stokes params to image (eg I,IV, QU,IQUV)
weighting           =  'natural'        #  Weighting of uv (natural, uniform, briggs, ...)
uvtaper             =      False        #  Apply additional uv tapering of  visibilities.
modelimage          =         ''        #  Name of model image(s) to initialize cleaning
restoringbeam       =       ['']        #  Output Gaussian restoring beam for CLEAN image
pbcor               =      False        #  Output primary beam-corrected image
minpb               =        0.1        #  Minimum PB level to use
async               =      False        #  If true the taskname must be started using clean(...)
\end{verbatim}
\normalsize
Figure~\ref{fig:inpclean1} shows how this will look to you on your
terminal.  Note that some parameters are in boldface with a gray
background.  This means that some values for this parameter will 
cause it to {\it expand}, revealing new {\it sub-parameters} to 
be set.

\begin{figure}[h!]
\begin{center}
%\gname{casa_inpclean1}{6}
%\pngname{clean_inputs_1}{6}
\pngname{newclean_inputs_1}{6}
\caption{\label{fig:inpclean1} Screen shot of the default CASA inputs
for task {\tt clean}.}
\hrulefill
\end{center}
\end{figure}

CASA uses color and font to indicate different properties of
parameters and their values:
\begin{center}
{\it Parameter and Values in CASA {\tt inp}}\\[5mm]
\begin{tabular}{|l|l|l|l|l|l|}
\hline
  & Text Font & Text Color & Highlight & Indentation & Meaning \\
\hline
\multicolumn{6}{|l|}{Parameters:}\\
\hline
  & plain & black & none & none & standard parameter \\
  & bold  & black & grey & none & expandable parameter \\
  & plain & green & none & yes & sub-parameter \\
\hline
\multicolumn{6}{|l|}{Values:}\\
\hline
  & plain & black & none & none & default value \\
  & plain & blue & none & none & non-default value \\
  & plain & red & none & none & invalid value \\
\hline
\end{tabular}
\end{center}

Figure~\ref{fig:inpclean2} shows what happens when you set some of the
{\tt clean} parameters to non-default values.  Some have
opened up sub-parameters, which can now be seen and set.
Figure~\ref{fig:inpclean3} shows what happens when you set a
parameter, in this case {\tt vis} and {\tt mode}, to an invalid value.  
Its value now appears in red.  Reasons for invalidation include
incorrect type, an invalid menu choice, or a filename that does not
exist.  For example, since {\tt vis} expects a
filename, it will be invalidated (red) if it is set to a non-string
value, or a string that is not the name of a file that can be found.
The {\tt mode='happy'} is invalid because its not a supported choice
({\tt'mfs'}, {\tt'channel'}, {\tt'velocity'}, or {\tt'frequency'}).

\begin{figure}[h]
\begin{center}
%\gname{casa_inpclean2}{6}
%\pngname{clean_inputs_2}{6}
\pngname{newclean_inputs_2}{6}
\caption{\label{fig:inpclean2} The {\tt clean} inputs after
setting values away from their defaults (blue text).
Note that some of the boldface ones have opened up new 
dependent sub-parameters (indented and green).
}
\hrulefill
\end{center}
\end{figure}

\begin{figure}[h]
\begin{center}
%\gname{casa_inpclean3}{6}
%\pngname{clean_inputs_3}{6}
\pngname{newclean_inputs_3}{6}
\caption{\label{fig:inpclean3} The {\tt clean} inputs where
one parameter has been set to an invalid value.  This is
drawn in red to draw attention to the problem.  This hapless
user probably confused the {\tt 'hogbom'} clean algorithm 
with Harry Potter.
}
\hrulefill
\end{center}
\end{figure}


%%%%%%
% \subsubsection{The {\tt restore} Command}
% \label{section:intro.tasks.setpar.restore}
% 
% \begin{wrapfigure}{r}{2.5in}
%   \begin{boxedminipage}{2.5in}
%      \centerline{\underline{\bf ALERT}}
%      In the current version of CASA, the {\tt restore} command
%      has been disabled, as it is still difficult to keep the list
%      of CASA globals stored in different places.  When we sort out
%      our parameter handling mechanisms, we
%      will probably bring back {\tt restore}.
%   \end{boxedminipage}
% \end{wrapfigure}
% 
% If you want to reset all input keywords for all tasks to the
% {\it global default values}, use the {\tt restore} command:
% \small
% \begin{verbatim}
% CASA <10>: restore
% \end{verbatim}
% \normalsize
% 
% Note that the global default values for many parameters are 
% different than the task-specific default values.  This is because
% some parameters have different default values in the different
% tasks they appear in!  Using the {\tt default <taskname>}
% command is much safer.

%%%%%%
\subsubsection{The {\tt saveinputs} Command}
\label{section:intro.tasks.setpar.saveinputs}

The {\tt saveinputs} command will save the current values of
a given task parameters to a Python (plain ascii) file.  It
can take up to two arguments, e.g.
\small
\begin{verbatim}
   saveinputs(taskname, outfile)
\end{verbatim}
\normalsize
The first is the usual {\tt taskname} parameter.  
The second is the name for the
output Python file.  If there is no second argument, for example,
\small
\begin{verbatim}
   saveinputs('clean')
\end{verbatim}
\normalsize
a file with name {\tt <taskname>.saved} (in this case 
{\tt 'clean.saved'} will be created or 
overwritten if extant.  If invoked with no arguments, e.g.
\small
\begin{verbatim}
   saveinputs
\end{verbatim}
\normalsize
it will use the current values of the {\tt taskname} variable
(as set using {\tt inp <taskname>} or {\tt default <taskname>}).
You can also use the {\tt taskname} global parameter explicitly,
\small
\begin{verbatim}
   saveinputs(taskname, taskname+'_1.save')
\end{verbatim}
\normalsize

For example, starting from default values
\small
\begin{verbatim}
CASA <1>: default('listobs')
CASA <2>: vis='ngc5921.demo.ms'
CASA <3>: saveinputs
CASA <4>: !more 'listobs.saved' 
taskname           = "listobs"
vis                =  "ngc5921.demo.ms"
selectdata         =  True
spw                =  ""
field              =  ""
antenna            =  ""
uvrange            =  ""
timerange          =  ""
correlation        =  ""
scan               =  ""
intent             =  ""
feed               =  ""
array              =  ""
observation        =  ""
verbose            =  True
listfile           =  ""
#listobs(vis="ngc5921.demo.ms",selectdata=True,spw="",field="",
antenna="",uvrange="",timerange="",correlation="",scan="",intent="",
feed="",array="",observation="",verbose=True,listfile="")

\end{verbatim}
\normalsize
To read these back in, use the Python {\tt execfile} command.  For example,
\small
\begin{verbatim}
CASA <5>: execfile('listobs.saved')
\end{verbatim}
\normalsize
and we are back.

An example save to a custom named file:
\small
\begin{verbatim}
CASA <6>: saveinputs('listobs','ngc5921_listobs.par')
\end{verbatim}
\normalsize

You can also use the CASA {\tt tget} command (see
\S~\ref{section:intro.tasks.setpar.tget} below) instead of the Python
{\tt execfile} to restore your inputs.

%%%%%%
\subsubsection{The {\tt tget} Command}
\label{section:intro.tasks.setpar.tget}

The {\tt tget} command will recover saved values of the
inputs of tasks.  This is a convenient alternative to using the
Python {\tt execfile} command (see above).

Typing {\tt tget} without a taskname will recover the saved values of the
inputs for the current task as given in the current value of the 
{\tt taskname} parameter.

Adding a task name, e.g. {\tt tget <taskname>} will recover
values for the specified task.  This is done by searching for
1) a {\tt <taskname>.last} file (see 
\S~\ref{section:intro.tasks.setpar.last} below), then for
2) a {\tt <taskname>.saved} file (see
\S~\ref{section:intro.tasks.setpar.saveinputs} above), and then
executing the Python in these files.

For example,
\small
\begin{verbatim}
   default('gaincal')   # set current task to gaincal and default
   tget                 # read saved inputs from gaincal.last (or gaincal.saved)
   inp                  # see these inputs!
   tget bandpass        # now get from bandpass.last (or bandpass.saved)
   inp                  # task is now bandpass, with recovered inputs
\end{verbatim}
\normalsize

%%%%%%
\subsubsection{The {\tt tput} Command}
\label{section:intro.tasks.setpar.tput}

The {\tt tput} command will save the current parameter values of a
task to its {\tt <taskname>.last} file.  This is a shorthand to
{\tt saveinputs} and is a counterpart to {\tt tget}.

Typing {\tt tput} without a taskname will save the values of the
inputs for the current task as given in the current value of the 
{\tt taskname} parameter.

Adding a task name, e.g. {\tt tget <taskname>} will save the
values for the specified task.

For example,
\small
\begin{verbatim}
   default('gaincal')   # set current task to gaincal and default
   tget                 # read saved inputs from gaincal.last (or gaincal.saved)
   inp                  # see these inputs!
   vis = 'new.ms'       # change the vis parameter
   tput                 # save back to the gaincal.last file for later use
\end{verbatim}
\normalsize

%%%%%%
\subsubsection{The {\tt .last} file}
\label{section:intro.tasks.setpar.last}

Whenever you successfully execute a CASA task, a Python script file
called {\tt <taskname>.last} will be written (or over-written) into
the current working directory.  For example, if you ran the
{\tt listobs} task as detailed above, then
\small
\begin{verbatim}
CASA <14>: vis = 'ngc5921.ms'

CASA <15>: verbose = True

CASA <16>: listobs()

CASA <17>: !more 'listobs.last'
IPython system call: more listobs.last
taskname           = "listobs"
vis                =  "ngc5921.ms"
verbose            =  True
listfile           =  ""
#listobs(vis="ngc5921.ms",verbose=False,listfile="")
\end{verbatim}
\normalsize

You can restore the parameter values from the save file using
\small
\begin{verbatim}
CASA <18>: execfile('listobs.last')
\end{verbatim}
\normalsize
or
\small
\begin{verbatim}
CASA <19>: run listobs.last
\end{verbatim}
\normalsize

Note that the {\tt .last} file in generally not created until
the task actually finished (successfully), so it is often best
to manually create a save file beforehand using the {\tt saveinputs} 
command if you are running a critical task that you strongly desire
to have the inputs saved for.

%%%%%%%%%%%%%%%%%%%%%%%%%%%%%%%%%%%%%%%%%%%%%%%%%%%%%%%%%%%%%%%%%
\subsection{Tools in CASA}
\label{section:intro.tasks.tools}

The CASA {\em toolkit} is the foundation of the functionality in the 
package, and consists of a suite of functions that are callable from
Python.  The tools are used by the tasks, and can be used by advanced
users to perform operations that are not available through the tasks.

It is beyond the scope of this reference to describe the toolkit in 
detail.  Occasionally, examples will be given that utilize the tools
(e.g. \S~\ref{section:analysis.toolkit}).
In short, tools are always called as functions, with any parmeters
that are not to be defaulted given as arguments.  For example:
\small
\begin{verbatim}
   ia.open('ngc5921.chan21.clean.cleanbox.mask')
   ia.calcmask('"ngc5921.chan21.clean.cleanbox.mask">0.5','mymask')
   ia.summary()
   ia.close()
\end{verbatim}
\normalsize
uses the {\tt image} tool ({\tt ia}) to turn a clean mask image into
an image mask.  Tools never use the CASA global parameters.

To find what tools are available, use the {\tt toolhelp} command:
\small
\begin{verbatim}
--------> toolhelp()

Available tools: 

 af : Agent flagger utilities
 at : Juan Pardo ATM library
 ca : Calibration analysis utilities
 cb : Calibration utilities
 cl : Component list utilities
 cp : Cal solution plotting utilities
 cs : Coordinate system utilities
 cu : Class utilities
 dc : Deconvolver utilities
 fg : Flagging/Flag management utilities
 fi : Fitting utilities
 fn : Functional utilities
 ia : Image analysis utilities
 im : Imaging utilities
 me : Measures utilities
 ms : MeasurementSet (MS) utilities
 msmd : MS metadata accessors
 mp : MS plotting (data (amp/phase) versus other quantities)
 mt : MS transformer utilities
 qa : Quanta utilities
 pm : PlotMS utilities
 po : Imagepol utilities
 rg : Region manipulation utilities
 sl : Spectral line import and search
 sm : Simulation utilities
 tb : Table utilities (selection, extraction, etc)
 tp : Table plotting utilities
 vp : Voltage pattern/primary beam utilities
 ---
 pl : pylab functions (e.g., pl.title, etc)
 sd : Single dish utilities
 ---
\end{verbatim}
\normalsize


You can find much more information about the toolkit in the 
{\em CASA User Reference Manual}:
\begin{quote}
  \url{http://casa.nrao.edu/docs/casaref/CasaRef.html}
\end{quote}

%%%%%%%%%%%%%%%%%%%%%%%%%%%%%%%%%%%%%%%%%%%%%%%%%%%%%%%%%%%%%%%%%
%%%%%%%%%%%%%%%%%%%%%%%%%%%%%%%%%%%%%%%%%%%%%%%%%%%%%%%%%%%%%%%%%
\section{Getting the most out of CASA}
\label{section:intro.common}

There are some other general things you should know about using
CASA in order to make things go smoothly during your data reduction.

%%%%%%%%%%%%%%%%%%%%%%%%%%%%%%%%%%%%%%%%%%%%%%%%%%%%%%%%%%%%%%%%%
\subsection{Your command line history}
\label{section:intro.common.commands}

Your command line history is automatically maintained and stored 
as {\tt ipython.log} in your local directory . This file can be edited and
re-executed as appropriate using the {\tt execfile '<filename>'}
feature.

You can also use the ``up-arrow'' and ``down-arrow'' keys for command line
recall in the {\tt casapy} interface.  If you start typing text, and
then use ``up-arrow'', you will navigate back through commands
matching what you typed.

%%%%%%%%%%%%%%%%%%%%%%%%%%%%%%%%%%%%%%%%%%%%%%%%%%%%%%%%%%%%%%%%%
\subsection{Logging your session}
\label{section:intro.common.logger}

The output from CASA commands is sent to the file {\tt
  casapy-YYYYMMDD-HHMMSS.log} in your local directory, where
YYYYMMDD-HHMMSS are the UT date and time when CASA was started up. New
starts of CASA create new log files. 

% \begin{figure}[ht]
% \gname{casalogger1}{6}
% \caption{\label{fig:casalogger1} CASA Logger GUI window}
% \hrulefill
% \end{figure}

\begin{figure}[h!]
\begin{center}
%\pngname{casalogger_window}{6}
\pngname{newcasalogger_window}{6}
\caption{\label{fig:casalogger1} The CASA Logger GUI window under
Linux. Note that under MacOSX a stripped down logger will instead
appear as a Console.}
\hrulefill
\end{center}
\end{figure}

The output contained in {\tt casapy-YYYYMMDD-HHMMSS.log} is also displayed in a
separate window using the {\it casalogger}.  Generally,
the logger window will be brought up when casapy is started.  If you
do not want the logger GUI to appear, then start casapy using the 
{\tt --nologger} option,
\small
\begin{verbatim}
   casapy --nologger
\end{verbatim}
\normalsize
which will run CASA in the terminal window.  See \S~\ref{section:intro.common.logger.startup}
for more startup options.

% \begin{figure}[h]
% \gname{casalogger_select}{6}
% \caption{\label{fig:logger_search} CASA Logger - Search example:
% Specify a string in the entry box to
% have all instances of the found string highlighted.}
% \hrulefill
% \end{figure}

\begin{figure}[h]
\begin{center}
%\pngname{casalogger_search}{6}
\pngname{newcasalogger_search}{6}
\caption{\label{fig:logger_search} Using the Search facility in the
{\tt casalogger}.  Here we have specified the string {\tt 'apply'}
and it has highlighted all instances in green.}
\hrulefill
\end{center}
\end{figure}

{\bf ALERT:} Due to problems with Qt under MacOSX, we had to
replace the GUI {\tt qtcasalogger} with a special stripped down one that
uses the Mac Console.  This still has the important capabilities 
such as showing the messages and cut/paste.  The following description
is for the Linux version and thus should mostly be disregarded on OSX.  On
the Mac, you treat this as just another console window and use the
usual mouse and hot-key actions to do what is needed.

The CASA logger window for Linux is shown in Figure~\ref{fig:casalogger1}.  The
main feature is the display area for the log text, which is divided
into columns.  The columns are:
\begin{itemize}
\item {\tt Time} --- the time that the message was generated.  Note
  that this will be in local computer time (usually UT) for casapy
  generated messages, and may be different for user generated messages;
\item {\tt Priority} --- the {\it Priority Level} (see below) of the
  message;
\item {\tt Origin} --- where within CASA the message came from.  This
  is in the format {\tt Task::Tool::Method} (one or more of the fields
  may be missing depending upon the message);
\item {\tt Message} --- the actual text.
\end{itemize}

% \begin{figure}[h]
% \gname{casalogger_filter}{6}
% \caption{\label{fig:logger_filter} CASA Logger - Filter facility: The
% log output can be sorted by Priority, Time, Origin. One can also
% filter for a string found in the Message.}  
% \hrulefill
% \end{figure}

\begin{figure}[h]
\begin{center}
%\pngname{casalogger_filter_origin}{6}
\pngname{newcasalogger_filter}{6}
\caption{\label{fig:logger_filter} Using the {\tt casalogger} Filter
facility.  The log output can be sorted by Priority, Time, Origin, and
Message.  In this example we are filtering by Origin using {\tt 'clean'},
and it now shows all the log output from the {\tt clean} task.}  
\hrulefill
\end{center}
\end{figure}

\begin{figure}[h]
\begin{center}
%\gname{casalogger_insert}{6}
%\pngname{casalogger_insert}{6}
\pngname{newcasalogger_insert}{6}
\caption{\label{fig:logger_insert} CASA Logger - Insert facility: The
log output can be augmented by adding notes or comments during the
reduction. The file should then be saved to disk to retain these
changes.}
\hrulefill
\end{center}
\end{figure}

The {\tt casalogger} GUI has a range of features, which include:
\begin{itemize}
\item {\bf Search} --- search messages by entering text in the Search
window and clicking the search icon.  The search currently just
matches the exact text you type anywhere in the message.  See 
Figure~\ref{fig:logger_search} for an example.

\item {\bf Filter} --- a filter to sort by message priority, time,
task/tool of origin, and message contents.  Enter text in the
Filter window and click the filter icon to the right of the window.
Use the pull-down at the left of the Filter window to choose what
to filter.  The matching is for the exact text currently (no regular
expressions).  See 
Figure~\ref{fig:logger_filter} for an example.

\item {\bf View} --- show and hide columns (Time, Priority, Origin,
Message) by checking boxes under the {\bf View} menu pull-down.  You
can also change the font here.

\item {\bf Insert Message} --- insert additional comments as ``notes''
in the log.  Enter the text into the ``Insert Message'' box at the
bottom of the logger, and click on the Add (+) button, or choose to
enter a longer message.  The entered message will appear with a
priority of ``NOTE'' with the Origin as your username.  See
Figure~\ref{fig:logger_insert} for an example. {\bf ALERT:} This
message currently will not be inserted into the correct (or user
controllable) order into the log.

\item {\bf Copy} --- left-click on a row, or click-drag a
range of rows, or click at the start and shift click at the end to
select.  Use the Copy button or {\bf Edit} menu Copy to put the
selected rows into the clipboard.  You can then (usually) paste this
where you wish.  {\bf ALERT:} this does not work routinely in the
current version.  You are best off going to the {\tt casapy-YYYYMMDD-HHMMSS.log} file
if you want to grab text.

\item {\bf Open} --- There is an Open function in the
{\bf File} menu, and an Open button, that will allow you to load old
casalogger files. 

\end{itemize}

Other operations are also possible from the menu or buttons.  Mouse
``flyover'' will reveal the operation of buttons, for example.

It is possible to change the file that the logging is directed to. Per
default it is 'casapy-YYYYMMDD-HHMMSS.log'. But starting CASA with the option {\tt --logfile} 

\small
\begin{verbatim}
casapy --logfile otherfile.log
\end{verbatim}
\normalsize 

will redirect the output pf the logger to the file 'otherfile.log'
(see also Sect.\,\ref{section:intro.common.logger.startup}).
The log file can also be changed {\it during} a CASA session. Type 
\small
\begin{verbatim}
CASA <15>: casalog.setlogfile('otherfile.log')
\end{verbatim}
\normalsize 

and you will redirect the output to the 'otherfile.log' file.
However, the logger GUI will still be monitoring the previous
'casapy-YYYYMMDD-HHMMSS.log' file. To change it to the new file, go on {\tt File -
 Open} and select the new log file, in our case 'otherfile.log'.

%%%%%%
\subsubsection{Startup options for the {\tt logger}}
\label{section:intro.common.logger.startup}

One can specify {\tt logger} options at the startup of {\tt casapy} on
the command line:
\small
\begin{verbatim}
   casapy <logger option>
\end{verbatim}
\normalsize

These options are:
\small
\begin{verbatim}
     --log2term                == logging message go to terminal
     --nologfile               == no casapy.log logfile is produced
     --logfile <filename>      == use specified name for logfile instead of casapy.log
     --nologger                == do not bring up GUI logger (see above)
     --nolog (is deprecated use --nologger)
\end{verbatim}
\normalsize
For example, to not bring up a GUI but send the message to your
terminal, do
\small
\begin{verbatim}
   casapy --nologger --log2term
\end{verbatim}
\normalsize
while
\small
\begin{verbatim}
   casapy --logfile mynewlogfile.log
\end{verbatim}
\normalsize
will start {\tt casapy} with {\tt logger} messages going to the file
{\tt mynewlogfile.log}.

%%%%%%
\subsubsection{Setting priority levels in the {\tt logger}}
\label{section:intro.common.logger.levels}

Logger messages are assigned a {\it Priority Level} when generated
within CASA.  The current levels of Priority are:
\begin{enumerate}
\item {\bf SEVERE} --- errors;
\item {\bf WARN} --- warnings;
\item {\bf INFO} --- basic information every user should be aware of or
  has requested;
\item {\bf INFO1} --- information possibly helpful to the user;
\item {\bf INFO2} --- details the power user might want to see;
\item {\bf INFO3} --- even more details;
\item {\bf INFO4} --- lowest level of non-debugging information;
\item {\bf DEBUGGING} --- most ``important'' debugging messages;
\item {\bf DEBUG1} --- more details;
\item {\bf DEBUG2} --- lowest level of debugging messages.
\end{enumerate}
The ``debugging'' levels are intended for the developers use.

\begin{wrapfigure}{r}{2.5in}
  \begin{boxedminipage}{2.5in}
     \centerline{\bf Inside the Toolkit:}
     The {\tt casalog} tool can be used to control the logging.
     In particular, the {\tt casalog.filter} method sets the
     priority threshold.  This tool can also be used to change
     the output log file, and to post messages into the logger.
  \end{boxedminipage}
\end{wrapfigure}

There is a threshold for which these messages are
written to the {\tt casapy-YYYYMMDD-HHMMSS.log} file and are thus visible in the
{\tt logger}.  By default, only messages at level {\bf INFO} and
above are logged.  The user can change the threshold using the
{\tt casalog.filter} method.  This takes a single string argument of
the {\tt level} for the threshold.  The {\tt level} sets the lowest
priority that will be generated, and all messages of this level or
higher will go into the {\tt casapy-YYYYMMDD-HHMMSS.log} file.

Some examples:
\small
\begin{verbatim}
   casalog.filter('INFO')           # the default
   casalog.filter('INFO2')          # should satisfy even advanced users
   casalog.filter('INFO4')          # all INFOx messages
   casalog.filter('DEBUG2')         # all messages including debugging
\end{verbatim}
\normalsize

{\bf WARNING:} Setting the threshold to {\bf DEBUG2} will put lots of
messages in the log!

%\begin{figure}[h!]
%\begin{center}
%\pngname{casalogger_priority}{6}
%\caption{\label{fig:logger_priority} Different message priority levels
%as seen in the {\tt casalogger} window.  These can also be Filtered upon.}  
%\hrulefill
%\end{center}
%\end{figure}

%%%%%%%%%%%%%%%%%%%%%%%%%%%%%%%%%%%%%%%%%%%%%%%%%%%%%%%%%%%%%%%%%
\subsection{Where are my data in CASA?}
\label{section:intro.common.data}

Interferometric data are filled into a so-called Measurement Set (or
MS).  In its logical structure, the MS looks like a generalized
description of data from any interferometric
or single dish telescope. Physically, the MS consists of several
tables in a directory on disk.  

Tables in CASA are actually directories containing files that are the
sub-tables.  For example, when you create a MS called {\tt AM675.ms},
then the name of the directory where all the tables are stored will be
called {\tt AM675.ms/}.  See Chapter~\ref{chapter:io} for more
information on Measurement Set and Data Handling in CASA.

The data that you originally get from a telescope can be put in any
directory that is convenient to you.  Once you "fill" the data into a
measurement set that can be accessed by CASA, it is generally best to
keep that MS in the same directory where you started CASA so you can
get access to it easily (rather than constantly having to specify a
full path name).

When you generate calibration solutions or images (again these
are in table format), these will also be written to disk.  It is a
good idea to keep them in the directory in which you started CASA.

%%%%%%%%%%%%%%%%%%%%%%%%%%%%%%%%%%%%%%%%%%%%%%%%%%%%%%%%%%%%%%%%%
\subsubsection{How do I get rid of my data in CASA?}
\label{section:intro.common.data.delete}

Note that when you delete a measurement set, calibration table, or 
image, which are in fact directories, you must delete this and all underlying
directories and files.  If you are not running {\tt casapy}, this is
most simply done by using the file delete method of the operating
system you started CASA from.  For example, when running CASA on a
Linux system, in order to delete the measurement set named 
{\tt AM675.ms} type:
\small
\begin{verbatim}
  CASA <5>: !rm -r AM675.ms
\end{verbatim}
\normalsize
from within CASA.  The {\tt !} tells CASA that a system command
follows (see \S~\ref{section:intro.basics.python.sys}),
and the {\tt -r} makes sure that all subdirectories are deleted
recursively. 

It is convenient to prefix all MS, calibration tables,
and output files produced in a run with a common string.  For example,
one might prefix all files from VLA project AM675 with {\tt AM675},
e.g. {\tt AM675.ms}, {\tt AM675.cal}, {\tt AM675.clean}.  Then,
\small
\begin{verbatim}
  CASA <6>: !rm -r AM675*
\end{verbatim}
\normalsize
will clean up all of these.

In scripts, the {\tt !} escape to the OS will not work.  Instead,
use the {\tt os.system()} function
(Appendix~\ref{section:python.sys.os}) to do the same thing:
\small
\begin{verbatim}
   os.system('rm -r AM675*')
\end{verbatim}
\normalsize

If you are within {\tt casapy}, then the CASA system is keeping
a cache of tables that you have been using and using the OS to
delete them will confuse things.  For example, running a script
that contains {\tt rm} commands multiple times will often not
run or crash the second time as the cache gets confused.  The
clean way of removing CASA tables (MS, caltables, images) 
inside {\tt casapy} is to use the {\tt rmtables} task:
\small
\begin{verbatim}
  rmtables('AM675.ms')
\end{verbatim}
\normalsize
and this can also be wildcarded
\small
\begin{verbatim}
  rmtables('AM675*')
\end{verbatim}
\normalsize
(though you may get warnings if it tries to delete files or
directories that fit the name wildcard that are not CASA tables).

{\bf ALERT:} Some CASA processes lock the file and forget to
give it up when they are done ({\tt plotxy} is usually the culprit).
You will get WARNING messages from {\tt rmtables} and your script
will probably crash second time around as the file isn't removed.
The safest thing is still to exit {\tt casapy} and start a new 
session for multiple runs.

%%%%%%%%%%%%%%%%%%%%%%%%%%%%%%%%%%%%%%%%%%%%%%%%%%%%%%%%%%%%%%%%%
\subsection{What's in my data?}
\label{section:intro.common.ms}

The actual data is in a large {\tt MAIN} table that is organized in
such a way that you can access different parts of the data easily.
This table contains a number of ``rows'', which are effectively a
single timestamp for a single spectral window (like an IF from the
VLA) and a single baseline (for an interferometer).  

There are a number of ``columns'' in the MS, the most important of
which for our purposes is the {\tt DATA} column --- this contains the
original visibility data from when the MS was created or filled.
There are other helpful ``scratch'' columns which hold useful versions
of the data or weights for further processing: the {\tt
  CORRECTED\_DATA} column, which is used to hold calibrated data and
an optional {\tt MODEL\_DATA} column, which may hold the Fourier inversion of a
particular model image.  The creation and use of the scratch columns
is generally done behind the scenes, but you should be aware that they
are there (and when they are used).  We will occasionally refer to the
rows and columns in the MS.

More on the contents of the MS can be found in \S~\ref{section:io.ms}.

%%%%%%%%%%%%%%%%%%%%%%%%%%%%%%%%%%%%%%%%%%%%%%%%%%%%%%%%%%%%%%%%%
\subsection{Data Selection in CASA}
\label{section:intro.common.selection}

We have tried to make the CASA task interface as uniform as possible.
If a given parameter appears in multiple tasks, it should, as far as
is possible, mean the same thing and be used in the same way in each.
There are groups of parameters that appear in a number of tasks to
do the same thing, such as for data selection.

The parameters {\tt field}, {\tt spw}, and {\tt selectdata} (which if
{\tt True} expands to a number of sub-parameters) are commonly used in
tasks to select data on which to work.  These common data selection
parameters are described in \S~\ref{section:io.selection}.


%%%%%%%%%%%%%%%%%%%%%%%%%%%%%%%%%%%%%%%%%%%%%%%%%%%%%%%%%%%%%%%%%
%%%%%%%%%%%%%%%%%%%%%%%%%%%%%%%%%%%%%%%%%%%%%%%%%%%%%%%%%%%%%%%%%
\section{From Loading Data to Images}
\label{section:intro.walkthru}

The subsections below provide a brief overview of the steps you will
need to load data into CASA and obtain a final, calibrated image.
Each subject is covered in more detail in Chapters~\ref{chapter:io}
through \ref{chapter:analysis}.

An end-to-end workflow diagram for CASA data reduction for
interferometry data is shown in Figure~\ref{fig:casaflow}.  
This might help you chart your course through the package.
In the following sub-sections, we will chart a rough course through
this process, with the later chapters filling in the individual
boxes.

\begin{figure}[h!]
\pngname{casa_flow}{6.5}
\caption{\label{fig:casaflow}
Flow chart of the data processing operations that a general 
user will carry out in an end-to-end CASA reduction session.}
\hrulefill
\end{figure}

Note that single-dish data reduction (for example with the ALMA
single-dish system) follows a similar course.  This is detailed
in Chapter~\ref{chapter:sd}.

%%%%%%%%%%%%%%%%%%%%%%%%%%%%%%%%%%%%%%%%%%%%%%%%%%%%%%%%%%%%%%%%%
\subsection{Loading Data into CASA}
\label{section:intro.walkthru.load}

The key data and image import tasks are:
\begin{itemize}
   \item {\tt importuvfits} --- import visibility data in UVFITS
         format (\S~\ref{section:io.import.uvfits});
   \item {\tt importvla} --- import data from VLA that is in 
         {\it export} format (\S~\ref{section:io.import.vla});
   \item {\tt importasdm} --- import ALMA data in ASDM format 
         (\S~\ref{section:io.import.asdm});
   \item {\tt importevla} --- import JVLA/EVLA data in SDM format 
         (\S~\ref{section:io.import.evla});
   \item {\tt importfits} --- import a FITS image into a CASA  
         {\it image} format table (\S~\ref{section:analysis.fits}).
\end{itemize}

These are used to bring in your interferometer data, to be stored
as a CASA Measurement set (MS), and any previously made images
or models (to be stored as CASA image tables).  

The data import tasks will create a MS with a path and name specified
by the {\tt vis} parameter.  See \S~\ref{section:intro.common.data}
for more information on MS in CASA.  The measurement set is the 
internal data format used by CASA, and conversion from any other
native format is necessary for most of the data reduction tasks.

Once data is imported, there are other operations you can use to
manipulate the datasets:
\begin{itemize}
   \item {\tt concat} --- concatenate multiple MSs into a
         given or a new MS (\S~\ref{section:io.concat})
\end{itemize}

Data import, export, concatenation, and selection
detailed in Chapter~\ref{chapter:io}.

\subsubsection{VLA: Filling data from VLA archive format}
\label{section:intro.walkthru.load.VLAfill}

VLA data in ``archive'' format are read into CASA from disk using
the {\tt importvla} task (see \S~\ref{section:io.import.vla}).  
This filler supports the new naming conventions of EVLA antennas
when incorporated into the old VLA system.

% Note: autocorrelations are filled automatically into the data
% set. Autocorrelation data is not needed for standard interferometric
% data, further, the imaging routine will try to image the
% autocorrelation data (it assumes it is single dish data) which will
% swamp any real signal. Thus, it is necessary to flag the
% autocorrelation data any time before imaging. 

Note that future data from the EVLA in ASDM format will use a
different filler.  This will be made available in a later release.

\subsubsection{Filling data from UVFITS format}
\label{section:intro.walkthru.load.uvfits}

For UVFITS format, use the {\tt importuvfits} task. A subset of popular
flavors of UVFITS (in particular UVFITS as written by AIPS) is
supported by the CASA filler.  See \S~\ref{section:io.import.uvfits}
for details.

\subsubsection{Loading FITS images}
\label{section:intro.walkthru.load.imfits}

For FITS format images, such as those to be used as calibration
models, use the {\tt importfits} task.  Most, though not all, types
of FITS images written by astronomical software packages can be
read in.

See \S~\ref{section:analysis.fits} for more information.

\subsubsection{Concatenation of multiple MS}
\label{section:intro.walkthru.load.concat}

Once you have loaded data into measurement sets on disk, you 
can use the tasks {\tt concat} or {\tt virtualconcat} to combine them.  

See \S~\ref{section:io.concat} for details.

%%%%%%%%%%%%%%%%%%%%%%%%%%%%%%%%%%%%%%%%%%%%%%%%%%%%%%%%%%%%%%%%%
\subsection{Data Examination, Editing, and Flagging} 
\label{section:intro.walkthru.flag}

The main data examination and flagging tasks are:
\begin{itemize}
   \item {\tt listobs} --- summarize the contents of a MS 
         (\S~\ref{section:io.list});
   \item {\tt flagmanager} --- save and manage versions of the
         flagging entries in the measurement set
         (\S~\ref{section:edit.flagmanager});
   \item {\tt plotms} --- interactive X-Y plotting and flagging of
         visibility data (\S~\ref{section:edit.plot.plotms});
   \item ({\tt plotxy} --- interactive X-Y plotting and flagging of
         visibility data (\S~\ref{section:edit.plot.plotxy}), {\bf
           note:} plotxy is slower than plotms and will eventually be
         phased out, plotxy is still useful to create scripted
         hardcopy output, this functionality will likely be available
         in plotms in the next release);
   \item {\tt flagdata} --- flagging (and unflagging)
         of specified data (\S~\ref{section:edit.flagdata});
   \item {\tt viewer} --- the CASA viewer can display (as a raster
         image) MS data, with some editing capabilities 
         (\S~\ref{chapter:display});
\end{itemize}

These tasks allow you to list, plot, and/or flag data in a CASA MS.

There will eventually be tasks for ``automatic'' flagging to
data based upon statistical criteria.  Stay tuned.

Examination and editing of synthesis data is described in
Chapter~\ref{chapter:edit}.

Visualization and editing of an MS using the {\tt casaviewer}
is described in Chapter~\ref{chapter:display}.

\subsubsection{Interactive X-Y Plotting and Flagging}
\label{section:intro.walkthru.flag.plotxy}

The principal tool for making X-Y plots of visibility data is
{\tt plotms} (see \S~\ref{section:edit.plot.plotms}).  Amplitudes and phases
(among other things) can be plotted against several x-axis options.

Interactive flagging (i.e., ``see it -- flag it'') is possible on the
{\tt plotms} X-Y displays of the data
(\S~\ref{section:edit.plot.plotms.flag}).  Since flags are inserted into the
measurement set, it is useful to backup (or make a copy) of the
current flags before further flagging is done, using {\tt flagmanager}
(\S~\ref{section:edit.flagmanager}).  Copies of the flag table can
also be restored to the MS in this way.

% \begin{enumerate}
%    \item Plot the data to be flagged. Set {\tt xaxis = 'time'} or
%      {\tt 'uvdist'}. {\tt yaxis = 'amp'} is most useful.  
%    \item Several plots on a page can be useful. For example,
%      {\tt field = ''; iteration = 'field_id';  subplot = 331; plotxy()} will
%      plot 9 fields. 
%    \item When plots come up, use the cursor to specify flagged data by
%      defining a simple box. Continue as desired.
%    \item Do not fiddle with other functions. 
%    \item Type {\tt s} in the CASA window to exit without appying any
%      flagging. The potential unflagged data will be displayed.
%    \item Type {\tt w} in the CASA window to exit with application of all
%      flags. 
% \end{enumerate}

\subsubsection{Flag the Data Non-interactively}
\label{section:intro.walkthru.flag.flag}

The {\tt flagdata} task (\S~\ref{section:edit.flagdata})
will flag the visibility data set based on the specified data
selections.  The {\tt listobs} task (\S~\ref{section:io.list}) may be
run (e.g. with {\tt verbose=True}) to provide some of the information
needed to specify the flagging scope. {\tt flagdata} also contains
autoflagging routines. 
 
\subsubsection{Viewing and Flagging the MS}
\label{section:intro.walkthru.flag.viewer}

The CASA {\tt viewer} can be used to display the data in the MS
as a (grayscale or color) raster image.  The MS can also be edited.
Use of the {\tt viewer} on an MS is detailed in
\S~\ref{section:display.ms}.

%%%%%%%%%%%%%%%%%%%%%%%%%%%%%%%%%%%%%%%%%%%%%%%%%%%%%%%%%%%%%%%%%
\subsection{Calibration}
\label{section:intro.walkthru.calib}

The major calibration tasks are:
\begin{itemize}
   \item {\tt setjy} --- Computes the model visibilities for a specified source
         given a flux density or model image, knows about standard
         calibrator sources (\S~\ref{section:cal.prior.models});
   \item {\tt gencal} --- Creates a calibration table for known delay
         and antenna position offsets
         (\S~\ref{section:cal.prior.gencal});
   \item {\tt bandpass} --- Solves for frequency-dependent (bandpass) complex
         gains (\S~\ref{section:cal.solve.band});
   \item {\tt gaincal} --- Solves for time-dependent
         (frequency-independent) complex gains 
         (\S~\ref{section:cal.solve.gain});
   \item {\tt fluxscale} --- Bootstraps the flux density scale from standard
         calibrators (\S~\ref{section:cal.solve.fluxscale});
   \item {\tt polcal} --- polarization calibration
         (\S~\ref{section:cal.solve.pol});
   \item {\tt applycal} --- Applies calculated calibration solutions 
         (\S~\ref{section:cal.correct.apply});
   \item {\tt clearcal} --- Re-initializes calibrated visibility data 
         in a given measurement set (\S~\ref{section:cal.correct.clearcal});
   \item {\tt listcal} --- Lists calibration solutions 
         (\S~\ref{section:cal.tables.listcal});
   \item {\tt plotcal} --- Plots (and optionally flags) calibration
         solutions (\S~\ref{section:cal.tables.plotcal});
   \item {\tt uvcontsub} --- carry out uv-plane continuum subtraction
         for spectral-line data (\S~\ref{section:cal.other.uvcontsub});
   \item {\tt split} --- write out a new (calibrated) MS for specified
         sources (\S~\ref{section:cal.other.split});
   \item {\tt cvel} --- Regrid a spectral MS onto a new frequency
         channel system (\S~\ref{section:cal.other.cvel}).
\end{itemize}

During the course of calibration, the user will specify a set of
calibrations to pre-apply before solving for a particular type of
effect, for example gain or bandpass or polarization.  The solutions
are stored in a calibration table (subdirectory) which is specified by
the user, {\it not} by the task: care must be taken in naming the
table for future use.  The user then has the option, as the
calibration process proceeds, to accumulate the current state of 
calibration in a new cumulative table.  Finally, the calibration can
be applied to the dataset.

Synthesis data calibration is described in detail in
Chapter~\ref{chapter:cal}. 

%%%%%%
\subsubsection{Prior Calibration}
\label{section:intro.walkthru.calib.prior}

The {\tt setjy} task calculates absolute fluxes for measurement set
base on known calibrator sources.  This can then be used in later
calibration tasks.  Currently, {\tt setjy} knows the flux density as a
function of frequency for several standard EVLA flux calibrators and
solar system objects, and the value of the flux density can be
manually inserted for any other source.  If the source is not
well-modeled as a point source, then a model image of that source
structure can be used (with the total flux density scaled by the
values given or calculated above for the flux density).  Models are
provided for the standard VLA calibrators.

Antenna gain-elevation curves (e.g. for the EVLA antennas) and
atmospheric optical depth corrections (applied as an
elevation-dependent function) may be pre-applied before solving for
the bandpass and gains.  This was previously done by setting the {\tt
  gaincurve} and {\tt opacity} parameters in the various calibration
solving tasks. We now suggest to create these tables ahead of the
calibration with {\tt gencal} and to carry them through the
calibration tasks just like any other calibration table.


See \S~\ref{section:cal.prior} for more details.

%%%%%%
\subsubsection{Bandpass Calibration}
\label{section:intro.walkthru.calib.bpass}

The {\tt bandpass} task
calculates a bandpass calibration solution: that is, it solves for
gain variations in frequency as well as in time.   Since the bandpass
(relative gain as a function of frequency) generally varies much more slowly 
than the changes in overall (mean) gain solved for by {\tt gaincal}, one
generally uses a long time scale when solving for the bandpass. The
default {\tt 'B'} solution mode solves for the gains in frequency
slots consisting of channels or averages of channels.

A polynomial fit for the solution (solution type {\tt 'BPOLY'}) may be
carried out instead of the default frequency-slot based {\tt 'B'}
solutions.  This single solution will span (combine) multiple spectral
windows.

Bandpass calibration is discussed in detail in
\S~\ref{section:cal.solve.band}.

If the gains of the system are changing over the time that the
bandpass calibrator is observed, then you may need to do an initial
gain calibration (see next step).

%%%%%%
\subsubsection{Gain Calibration}
\label{section:intro.walkthru.calib.gaincal}

The {\tt gaincal} task determines solutions for the time-based complex
antenna gains, for each spectral window, from the specified
calibration sources.  A solution interval may be specified. The
default {\tt 'G'} solution mode solves for antenna-based gains 
in each polarization in specified time solution intervals.  The {\tt 'T'}
solution mode is the same as {\tt 'G'} except that it solves for
a single solution shared by both polarizations.

A spline fit for the solution (solution type {\tt 'GSPLINE'}) may be
carried out instead of the default time-slot based {\tt 'G'} solutions.

See \S~\ref{section:cal.solve.gain} for more on gain calibration.

%%%%%%
\subsubsection{Polarization Calibration}
\label{section:intro.walkthru.calib.polcal}

The {\tt polcal} task will solve for any unknown polarization leakage
and cross-hand phase terms ({\tt 'D'} and {\tt 'X'} solutions).  The
{\tt 'D'} leakage solutions will work on sources with no polarization and
sources with known (and supplied, e.g., using {\tt smodel}) polarization.
For sources with unknown polarization tracked through a range in parallactic 
angle on the sky, using poltype {\tt 'D+QU'}, which will first estimate
the calibrator polarization for you.

The solution for the unknown cross-hand polarization phase difference
{\tt 'X'} term requires a polarized source with known linear
polarization (Q,U).

Frequency-dependent (i.e., per channel) versions of all of these modes
are also supported (poltypes {\tt 'Df'}, {\tt 'Df+QU'}, and {\tt 'Xf'}.

See \S~\ref{section:cal.solve.pol} for more on polarization calibration.

%%%%%%
\subsubsection{Examining Calibration Solutions}
\label{section:intro.walkthru.calib.plotcal}

The {\tt plotcal} task (\S~\ref{section:cal.tables.plotcal}) will plot the
solutions in a calibration table.  The {\tt xaxis} choices include
time (for {\tt gaincal} solutions) and channel (e.g. for {\tt
bandpass} calibration).  
The {\tt plotcal} interface and plotting surface is similar to that in
{\tt plotxy}.  Eventually, {\tt plotcal} will allow you to flag and
unflag calibration solutions in the same way that data can be edited
in {\tt plotxy}.

The {\tt listcal} task (\S~\ref{section:cal.tables.listcal}) will print out
the calibration solutions in a specified table.

%%%%%%
\subsubsection{Bootstrapping Flux Calibration}
\label{section:intro.walkthru.calib.fluxscale}

The {\tt fluxscale} task bootstraps the flux density scale from
``primary'' standard calibrators to the ``secondary'' calibration
sources.  Note that the flux density scale must have been previously
established on the ``primary'' calibrator(s), typically using 
{\tt setjy}, and of course a calibration table containing
valid solutions for all calibrators must be available.

See \S~\ref{section:cal.solve.fluxscale} for more.

%%%%%%
\subsubsection{Correcting the Data}
\label{section:intro.walkthru.calib.applycal}

The final step in the calibration process, {\tt applycal} may be used
to apply several calibration tables (e.g., from {\tt gaincal} or
{\tt bandpass}). 
The corrections are applied to the {\tt DATA} column of the
visibility, writing the {\tt CORRECTED\_DATA} column which can then be
plotted (e.g. in {\tt plotxy}), {\tt split} out as the {\tt DATA}
column of a new MS, or imaged (e.g. using {\tt clean}).  Any existing
corrected data are overwritten.

See \S~\ref{section:cal.correct.apply} for details.
 

 

%%%%%%
\subsubsection{Splitting the Data}
\label{section:intro.walkthru.calib.split}

After a suitable calibration is achieved, it may be desirable to
create one or more new measurement sets containing the data for
selected sources.  This can be done using the {\tt split} task
(\S~\ref{section:cal.other.split}).

Further imaging and calibration (e.g. self-calibration) can be
carried out on these split Measurement Sets.


%%%%%%
\subsubsection{UV Continuum subtraction}
\label{section:intro.walkthru.calib.uvcontsub}

For spectral line data, continuum subtraction can be performed in the
image domain ({\tt imcontsub}) or in the uv domain. For the latter,
there are two tasks available: {\tt uvcontsub} subtracts polynomial of
desired order from each baseline, defined by line-free channels. 

%See \S~\ref{section:cal.correct.apply} for details.

%%%%%%
\subsubsection{Transforming the Data to a new frame}
\label{section:intro.walkthru.calib.cvel}

If you want to transform your dataset to a different frequency and
velocity frame than the one it was observed in, then you can use
the {\tt cvel} task
(\S~\ref{section:cal.other.cvel}). Alternatively, you can do the
regridding during the imaging process in {\tt clean} without running
{\tt cvel} before.


%%%%%%%%%%%%%%%%%%%%%%%%%%%%%%%%%%%%%%%%%%%%%%%%%%%%%%%%%%%%%%%%%
\subsection{Synthesis Imaging}
\label{section:intro.walkthru.image}

The key synthesis imaging tasks are:
\begin{itemize}
   \item {\tt clean} --- Calculates a deconvolved image based on the visibility
         data, using one of several clean algorithms  
         (\S~\ref{section:im.clean});
   \item {\tt feather} --- Combines a single dish and synthesis image in the
         Fourier plane (\S~\ref{section:im.feather}).
\end{itemize}
Most of these tasks are used to take calibrated interferometer data, with the
possible addition of a single-dish image, and reconstruct a model
image of the sky. {\bf Alert:} The {\tt clean} task is now even more
powerful and incorporates the functionality of previous specialized
tasks such as {\tt mosaic} and {\tt widefield}.

See Chapter~\ref{chapter:im} for more on synthesis imaging.

\subsubsection{Cleaning a single-field image or a mosaic}
\label{section:intro.walkthru.image.clean}

The CLEAN algorithm is the most popular and widely-studied method for
reconstructing a model image based on interferometer data.  It 
iteratively removes at each step a fraction of the flux
in the brightest pixel in a defined region of the current ``dirty'' 
image, and places this in the model image.  The
{\tt clean} task implements the CLEAN algorithm for single-field
data.  The user can choose from a number of options for the particular 
flavor of CLEAN to use.

Often, the first step in imaging is to make a simple gridded Fourier
inversion of the calibrated data to make a ``dirty'' image.  This
can then be examined to look for the presence of noticeable emission
above the noise, and to assess the quality of the calibration by 
searching for artifacts in the image.  This is done using {\tt clean}
with {\tt niter=0}.

The {\tt clean} task can jointly deconvolve mosaics as well as single
fields, and also has options to do wide-field and wide-band
multi-frequency synthesis imaging.

See \S~\ref{section:im.clean} for an in-depth discussion of the
{\tt clean} task.

\subsubsection{Feathering in a Single-Dish image}
\label{section:intro.walkthru.image.feather}

If you have a single-dish image of the large-scale emission in the
field, this can be ``feathered'' in to the image obtained from
the interferometer data.  This is carried out using the 
{\tt feather} tasks as the weighted sum in the uv-plane of the gridded
transforms of these two images.  While not as accurate as a true
joint reconstruction of an image from the synthesis and single-dish
data together, it is sufficient for most purposes.

See \S~\ref{section:im.feather} for details on the use of the
{\tt feather} task.

%%%%%%%%%%%%%%%%%%%%%%%%%%%%%%%%%%%%%%%%%%%%%%%%%%%%%%%%%%%%%%%%%
\subsection{Self Calibration}
\label{section:intro.walkthru.selfcal}

Once a calibrated dataset is obtained, and a first deconvolved 
model image is computed, a ``self-calibration'' loop can be 
performed.  Effectively, the model (not restored) image is passed
back to another calibration process (on the target data).  This
refines the calibration of the target source, which up to this
point has had (usually) only external calibration applied.  This
process follows the regular calibration procedure outlined above.

Any number of self-calibration loops can be performed.  As long
as the images are improving, it is usually prudent to continue
the self-calibration iterations.

This process is described in \S~\ref{section:im.selfcal}.

%%%%%%%%%%%%%%%%%%%%%%%%%%%%%%%%%%%%%%%%%%%%%%%%%%%%%%%%%%%%%%%%%
\subsection{Data and Image Analysis}
\label{section:intro.walkthru.analysis}

The key data and image analysis tasks are:
\begin{itemize}
   \item {\tt imhead} --- summarize and manipulate the ``header'' 
         information in a CASA image 
         (\S~\ref{section:analysis.imhead});
   \item {\tt imcontsub} --- perform continuum subtraction on a
         spectral-line image cube 
         (\S~\ref{section:analysis.imcontsub});
   \item {\tt immath} --- perform mathematical operations on or
         between images
         (\S~\ref{section:analysis.immath});
   \item {\tt immoments} --- compute the moments of an image cube
         (\S~\ref{section:analysis.moments});
   \item {\tt imstat} --- calculate statistics on an image or part
         of an image
         (\S~\ref{section:analysis.imstat});
   \item {\tt imval} --- extract values of one or more pixels,
         as a spectrum for cubes, from an image
         (\S~\ref{section:analysis.imval});
   \item {\tt imfit} --- simple 2D Gaussian fitting of single
         components to a region of an image
         (\S~\ref{section:analysis.imfit});
   \item {\tt imregrid} --- regrid an image onto the coordinate
         system of another image 
         (\S~\ref{section:analysis.regrid});
   \item {\tt viewer} --- there are useful region statistics and
         image cube plotting capabilities in the viewer 
         (\S~\ref{chapter:display}).
\end{itemize}

%%%%%%
\subsubsection{What's in an image?}
\label{section:intro.walkthru.analysis.imhead}

The {\tt imhead} task will print out a summary of image ``header'' 
keywords and values.  This task can also be
used to retrieve and change the header values.

See \S~\ref{section:analysis.imhead} for more.

%%%%%%
\subsubsection{Image statistics}
\label{section:intro.walkthru.analysis.imstat}

The {\tt imstat} task will print image statistics.  There are options
to restrict this to a box region, and to specified channels and Stokes
of the cube.  This task will return the statistics in a Python
dictionary return variable.

See \S~\ref{section:analysis.imstat} for more.

%%%%%%
\subsubsection{Image values}
\label{section:intro.walkthru.analysis.imval}

The {\tt imval} task will values from an image.  There are options
to restrict this to a box region, and to return specified channels and Stokes
of the cube as a spectrum.  This task will return these values in a Python
dictionary return variable which can then be operated on in the 
{\tt casapy} environment.

See \S~\ref{section:analysis.imstat} for more.

%%%%%%
\subsubsection{Moments of an image cube}
\label{section:intro.walkthru.analysis.moments}

The {\tt immoments} task will compute a ``moments'' image of
an input image cube.  A number of options are available, from
the traditional true moments (zero, first, second) and variations
thereof, to other images such as median, minimum, or maximum along
the moment axis.

See \S~\ref{section:analysis.moments} for details.

%%%%%%
\subsubsection{Image math}
\label{section:intro.walkthru.analysis.immath}

The {\tt immath} task will allow you to form a new image by
mathematical combinations of other images (or parts of images).
This is a powerful, but tricky, task to use.

See \S~\ref{section:analysis.immath} for more.

%%%%%%
\subsubsection{Regridding an Image}
\label{section:intro.walkthru.analysis.regrid}

It is occasionally necessary to regrid an image onto a new
coordinate system.  The {\tt imregrid} task can be used
to regrid an input image onto the coordinate system of an
existing template image, creating a new output image.

See \S~\ref{section:analysis.regrid} for a description of
this task.

%%%%%%
\subsubsection{Displaying Images}
\label{section:intro.walkthru.analysis.display}

To display an image use the {\tt viewer} task.  The viewer will
display images in raster, contour, or vector form.  Blinking and
movies are available for spectral-line image cubes.  To start the
viewer, type:

\small
\begin{verbatim}
  viewer
\end{verbatim}
\normalsize

Executing the {\tt viewer} task will bring up two windows: a viewer
screen showing the data or image, and a file catalog list. Click on an
image or ms from the file catalog list, choose the proper display, and
the image should pop up on the screen. Clicking on the wrench
tool (second from left on upper left) will obtain the data display
options. Most functions are self-documenting. 

The viewer can be run outside of casapy by typing {\tt casaviewer}. 

See \S~\ref{chapter:display} for more on viewing images.

%%%%%%%%%%%%%%%%%%%%%%%%%%%%%%%%%%%%%%%%%%%%%%%%%%%%%%%%%%%%%%%%%
\subsection{Getting data and images out of CASA}
\label{section:intro.walkthru.export}

The key data and image export tasks are:
\begin{itemize}
   \item {\tt exportuvfits} --- export a CASA MS in UVFITS format 
         (\S~\ref{section:io.import.uvfits});
   \item {\tt exportfits} --- export a CASA image table as FITS 
         (\S~\ref{section:analysis.fits}).
\end{itemize}

These tasks can be used to export a CASA MS or image to UVFITS
or FITS respectively.  See the individual sections referred to
above for more on each.

%%%%%%%%%%%%%%%%%%%%%%%%%%%%%%%%%%%%%%%%%%%%%%%%%%%%%%%%%%%%%%%%%
%%%%%%%%%%%%%%%%%%%%%%%%%%%%%%%%%%%%%%%%%%%%%%%%%%%%%%%%%%%%%%%%%
