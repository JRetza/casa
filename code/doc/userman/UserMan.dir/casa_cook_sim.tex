% STM 2007-04-13  split from previous version
% STM 2007-10-11  add into beta
% STM 2007-10-22  put in appendix for Beta Release
% RI  2009-12-17  Release 0 (3.0.0) draft
% STM 2009-12-21  Release 0 (3.0.0) final
% RI 2010-04-10 Release 3.0.1

%\chapter{Simulation}
\chapter[Simulation]{Simulation}
\label{chapter:sim}

{\bfseries New in 4.0:}
\begin{itemize}
\item A new task {\tt simalma} to simulate ALMA observations
  (experimental).
\item Task {\tt simdata} is removed.
\end{itemize}

%% {\bfseries Important note on task names:}
%% Users are encouraged to use {\tt simobserve} and {\tt simanalyze}.  The combined task {\tt simdata} is still present, but will be removed in the future.

The tasks available for simulating observations are:
\begin{itemize}
\item {\tt simobserve} --- simulate an interferometer or total power
observation (\S~\ref{section:sim.almasimmos})
\item {\tt simanalyze} --- image and analyze simulated datasets
(\S~\ref{section:sim.almasimmos})
\item {\tt simalma} --- (experimental) simulate an ALMA observation and
generate a combined image from the simulated datasets
(\S~\ref{section:sim.simalma})
\end{itemize}

The capability of simulating observations and datasets from the JVLA
and ALMA are an important use-case for CASA.  This not only allows one
to get an idea of the capabilities of these instruments for doing
science, but also provides benchmarks for the performance and utility
of the software for processing ``realistic'' datasets (with
atmospheric and instrumental effects).  Simulations can
also be used to tune parameters of the data reduction and therefore
help optimizing the process.
CASA can calculate
visibilities (create a measurement set) for any interferometric array,
and calculate and apply calibration tables representing some of the
most important corrupting effects. {\tt simobserve} can also simulate
total power observations, which can be combined with interferometric
data in {\tt simanalyze} (i.e. one would run {\tt simobserve} twice,
{\tt simanalyze} once).
The task {\tt simalma} is an experimental task to simulate an ALMA
observation, including ALMA 12-m, ACA 7-m and total power arrays, and
generate a combined image.


\begin{wrapfigure}{r}{2.5in}
 \begin{boxedminipage}{2.5in}
    \centerline{\bf Inside the Toolkit:}
    The simulator methods are in the {\tt sm} tool.
    Many of the other tools are also helpful when
    constructing and analyzing simulations.
 \end{boxedminipage}
\end{wrapfigure}

CASA's simulation capabilities continue to be improved with each CASA release.
For the most current information, please refer to
\url{http://www.casaguides.nrao.edu}, and click on
``Simulating Observations in CASA''.
%
Following general CASA practice, the greatest flexibility and richest
functionality is at the Toolkit level.  The most commonly used
procedures for interferometric and single dish simulation are
encapsulated in the {\tt simobserve} task.

%%%%%%%%%%%%%%%%%%%%%%%%%%%%%%%%%%%%%%%%%%%%%%%%%%%%%%%%%%%%%%%%%
%%%%%%%%%%%%%%%%%%%%%%%%%%%%%%%%%%%%%%%%%%%%%%%%%%%%%%%%%%%%%%%%%
\section{Simulating ALMA with {\tt simobserve} and {\tt simanalyze}}
\label{section:sim.almasimmos}

The {\tt simobserve} inputs are (submenus expand slightly differently for thermalnoise=manual and single dish observing):
\small
\begin{verbatim}
project             =      'sim'        #  root prefix for output file names
skymodel            =         ''        #  model image to observe
     inbright       =         ''        #  scale surface brightness of brightest pixel e.g. "1.2Jy/pixel"
     indirection    =         ''        #  set new direction e.g. "J2000 19h00m00 -40d00m00"
     incell         =         ''        #  set new cell/pixel size e.g. "0.1arcsec"
     incenter       =         ''        #  set new frequency of center channel e.g. "89GHz" (required even for 2D model)
     inwidth        =         ''        #  set new channel width e.g. "10MHz" (required even for 2D model)

complist            =         ''        #  componentlist to observe
     compwidth      =     '8GHz'        #  bandwidth of components

setpointings        =       True        
     integration    =      '10s'        #  integration (sampling) time
     direction      =         ''        #  "J2000 19h00m00 -40d00m00" or "" to center on model
     mapsize        =   ['', '']        #  angular size of map or "" to cover model
     maptype        =     'ALMA'        #  hexagonal, square, etc
     pointingspacing =         ''       #  spacing in between pointings or "0.25PB" or "" for 0.5 PB

obsmode             =      'int'        #  observation mode to simulate
                                        #   [int(interferometer)|sd(singledish)|""(none)]
     antennalist    = 'alma.out10.cfg'  #  interferometer antenna position file
     refdate        = '2014/05/21'      #  date of observation - not critical unless concatting
                                        #   simulations
     hourangle      =  'transit'        #  hour angle of observation center e.g. -3:00:00, or "transit"
     totaltime      =    '7200s'        #  total time of observation or number of repetitions
     caldirection   =         ''        #  pt source calibrator [experimental]
     calflux        =      '1Jy'        

thermalnoise        = 'tsys-atm'        #  add thermal noise: [tsys-atm|tsys-manual|""]
     user_pwv       =        1.0        #  Precipitable Water Vapor in mm
     t_ground       =      269.0        #  ambient temperature
     seed           =      11111        #  random number seed

leakage             =        0.0        #  cross polarization (interferometer only)
graphics            =     'both'        #  display graphics at each stage to [screen|file|both|none]
verbose             =      False        
overwrite           =       True        #  overwrite files starting with $project
async               =      False        #  If true the taskname must be started using simobserve(...)
\end{verbatim}
\normalsize

This task takes an input model image or list of components, plus a
list of antennas (locations and sizes), and simulates a particular
observation (specifies by mosaic setup and observing cycles and
times).  The output is a measurement set suitable for further analysis in CASA.

The {\tt simanalyze} inputs are:
\small
\begin{verbatim}
project             =      'sim'        #  root prefix for output file names
image               =       True        #  (re)image $project.*.ms to $project.image
     vis            =  'default'        #  Measurement Set(s) to image
     modelimage     =         ''        #  prior image to use in clean e.g. existing single dish image
     imsize         =          0        #  output image size in pixels (x,y) or 0 to match model
     imdirection    =         ''        #  set output image direction, (otherwise center on the model)
     cell           =         ''        #  cell size with units or "" to equal model
     niter          =        500        #  maximum number of iterations (0 for dirty image)
     threshold      =   '0.1mJy'        #  flux level (+units) to stop cleaning
     weighting      =  'natural'        #  weighting to apply to visibilities
     mask           =         []        #  Cleanbox(es), mask image(s), region(s), or a level
     outertaper     =         []        #  uv-taper on outer baselines in uv-plane
     stokes         =        'I'        #  Stokes params to image

analyze             =       True        #  (only first 6 selected outputs will be displayed)
     showuv         =       True        #  display uv coverage
     showpsf        =       True        #  display synthesized (dirty) beam (ignored in single dish simulation)
     showmodel      =       True        #  display sky model at original resolution
     showconvolved  =      False        #  display sky model convolved with output beam
     showclean      =       True        #  display the synthesized image
     showresidual   =      False        #  display the clean residual image (ignored in single dish simulation)
     showdifference =       True        #  display difference image
     showfidelity   =       True        #  display fidelity

graphics            =     'both'        #  display graphics at each stage to [screen|file|both|none]
verbose             =      False        
overwrite           =       True        #  overwrite files starting with $project
async               =      False        #  If true the taskname must be started using simanalyze(...)
\end{verbatim}
\normalsize

This task analyzes one or more measurement sets - interferometric and/or single dish.
The output is a synthesized image created from those visibilities, a difference image
between the synthesized image and your sky model convolved with the
output synthesized beam, and a fidelity image. (see ALMA memo 398 for
description of fidelity, which is approximately the output image
divided by the difference between input and output)

%% The combined task {\tt simdata} is modular: one can
%% modify one's sky model, predict visibilities, corrupt the Measurement
%% Set, re-image, and analyze the result all separately, provided in a
%% few cases the filenames are set correctly.  
%%%%%%%%%%%%%%%%%%%%%%%%%%%%%%%%%%%%%%%%%%%%%%%%%%%%%%%%%%%%%%%%%
%%%%%%%%%%%%%%%%%%%%%%%%%%%%%%%%%%%%%%%%%%%%%%%%%%%%%%%%%%%%%%%%%
\section{Simulating ALMA with {\tt simalma}}
\label{section:sim.simalma}

The task {\tt simalma} simulates an ALMA observation by ALMA 12-m, ACA-7m
and total power arrays.
It takes an input model image or a list of components, plus
configurations of ALMA antennas (locations and sizes), and simulates a
particular ALMA observation (specified by mosaic setup and
observing cycles and times).  The outputs are measurement sets.
The task optionally generates synthesized images from the measurement
sets as {\tt simanalyze} does. 

Technically speaking, {\tt simalma} internally
calls {\tt simobserve} and {\tt simanalyze} as many times as necessary
to simulate and analyze an ALMA observation.
Some of the simulation and imaging parameters are automatically set to
values typical of ALMA observations in {\tt simalma} (see
\S~\ref{section:sim.simalma.casa400} for more details). 
Thus, it has a simpler task interface compared to {\tt simobserve} plus
{\tt simanalyze} at the cost of limited flexibilities. 
If you want to have more control on simulation setup, it is available
by manually running {\tt simobserve} and {\tt simanalyze} multiple
times or by using {\tt sm} tools.

{\bf ALERT}: 
Note that {\tt simalma} is an experimental task. Simulation settings in
detail may differ from the actual observations in Cycle 1 as the optimal
strategy for combining ALMA 12-m, ACA-7m and total power array data is under
investigation.

%% Note that {\tt simalma} is an experimental task. 
%% Simulation settings in the task may differ from the actual observations
%% in Cycle 1. Developers are continuing efforts to implement the setup
%% of ALMA Cycle 1 observations, but details of ACA observation
%% strategies are under discussion as of the end of CASA 4.0.0
%% development cycle (September, 2012).

The {\tt simalma} inputs are:
\small
\begin{verbatim}
project             =         ''        #  root prefix for output file names
skymodel            =         ''        #  model image to observe
     inbright       =         ''        #  scale surface brightness of brightest pixel e.g.
                                        #  "1.2Jy/pixel"
     indirection    =         ''        #  set new direction e.g. "J2000 19h00m00 -40d00m00"
     incell         =         ''        #  set new cell/pixel size e.g. "0.1arcsec"
     incenter       =         ''        #  set new frequency of center channel e.g. "89GHz"
                                        #  (required even for 2D model)
     inwidth        =         ''        #  set new channel width e.g. "10MHz" (required even
                                        #  for 2D model)
complist            =         ''        #  componentlist to observe
     compwidth      =     '8GHz'        #  bandwidth of components

setpointings        =       True        
     integration    =      '10s'        #  integration (sampling) time
     direction      =         ''        #  "J2000 19h00m00 -40d00m00" or "" to center on model
     mapsize        =   ['', '']        #  angular size of map or "" to cover model

antennalist         = 'alma_cycle1_1.cfg' #  antenna position file of ALMA 12m array
hourangle           =  'transit'        #  hour angle of observation center e.g. -3:00:00, or "transit"
totaltime           =    '7200s'        #  total time of observation or number of repetitions
acaratio            =        3.0        #  Ratio of the total observation time for ACA in
                                        #  relation to 12-m array or 0 for no ACA
     acaconfig      =         ''        #  Antenna configuration of ACA 7-m array
                                        #  [""|"cycle1"|"i"|"ns"]

pwv                 =        0.0        #  Precipitable Water Vapor in mm. 0 for noise-free
                                        #  simulation
image               =       True        #  image $project.*.ms to $project.image
     imsize         =          0        #  output image size in pixels (x,y) or 0 to match model
     imdirection    =         ''        #  set output image direction, (otherwise center on the model)
     cell           =         ''        #  cell size with units or "" to equal model
     niter          =        500        #  maximum number of iterations (0 for dirty image)
     threshold      =   '0.1mJy'        #  flux level (+units) to stop cleaning

graphics            =     'both'        #  display graphics at each stage to [screen|file|both|none]
verbose             =      False        
overwrite           =      False        #  overwrite files starting with $project
async               =      False        #  If true the taskname must be started using simalma(...)
\end{verbatim}
%% $
\normalsize

The task {\tt simalma} is designed as a task that is invoked only once for
a simulation setup. 
It always sets up skymodel and pointings, and invokes a simulated
observation of ALMA 12-m array.
It means {\tt simalma} is not supposed to be run multiple times for a
project, unlike {\tt simobserve} and {\tt simanalyze}. 
The task {\tt simalma} may ignore or overwrite the old results when it is
run more than once with the same project name. 

There are options in {\tt simalma} to simulate
observation of ACA 7-m and total power arrays, to apply thermal noise, and/or
to generate images from simulated measurement sets. 
Observations of ACA 7-m and total power arrays are simulated when
$ {\tt acaratio} \, > \, 0 $, 
while they are skipped and only ALMA 12-m array is simulated when
$ {\tt acaratio} \, = \, 0 $.
Thermal noise is added to visibilities when $ {\tt pwv} \, > \, 0 $.
The ATM atmospheric model is constructed from the characteristics of
the ALMA site and a user defined Precipitable Water Vapour ({\tt pwv})
value. 
Set $ {\tt pwv} \, = \, 0 $ to omit the thermal noise. 
Finally, when {\tt image = True}, synthesized images are generated
from the simulated measurement sets.

\subsection{Implementation of {\tt simalma} in CASA 4.0.0}
\label{section:sim.simalma.casa400}

As mentioned in the previous section, {\tt simalma} automatically sets
some of the simulation and imaging parameters to values typical of ALMA
observations. The implementations of antenna configurations, pointings,
integration time, and imaging in CASA 4.0.0 are described in this
section.

\smallskip
{\bf Antenna Configuration}:

The configuration of the ALMA 12-m array is defined by the {\tt antennalist}
parameter. You can specify either the name of an antenna configuration
file or a desired resolution, e.g., {\tt `alma-cycle1;5arcsec'}. 
The configuration of the ACA 7-m array is defined by the {\tt acaconfig}
parameter. Available configurations are, {\tt `'} (default),
{\tt `cycle1'} (to use {\tt aca\_cycle1.cfg}),
{\tt `i'} ({\tt aca.i.cfg}), or
{\tt `ns'} ({\tt aca.ns.cfg}). The default
configuration of the ACA 7-m array is set to {\tt aca\_cycle1.cfg}
when the ALMA 12-m array ({\tt antennalist}) is in one of the Cycle 1
configurations, or to {\tt aca.i.cfg} otherwise.
Finally, the location of the ACA total power array is
taken from a file, {\tt aca.tp.cfg}, and only the first antenna
in the list is simulated.


\medskip
{\bf Field Setup}:

There are two ways to setup pointings, i.e., Rectangle Setup and
Multi-Pointing.

In the Rectangle Setups, pointings are automatically calculated from
the pointing centre ({\tt direction}) and the map size.
A rectangular map region is covered by a hexagonal grid
({\tt maptype = `alma'}) with Nyquist sampling, i.e.,
$ 0.48\, {\rm PB} $ spacing (where
$ {\rm PB} \, \equiv \, \lambda / D $), 
in both ALMA 12-m and ACA 7-m array simulations.
A slightly larger area is mapped in ACA total 
power simulations for later combination with interferometer 
visibilities. The map area is extended by $ 1 \, {\rm PB} $
in each direction and
covered by a lattice grid with $ 0.36\, {\rm PB} $ spacing.

In Multi-Pointing, a list of pointings is
defined in the {\tt direction} parameter or read from a file (when
{\tt setpointings = False}). The ALMA 12-m and ACA 7-m arrays observe
the specified directions.
The ACA total power simulations map either
(1) square regions of $ 2\, {\rm PB} $ extent centred at each of the
pointings, or 
(2) a rectangle region that covers all the pointings.
Either (1) or (2), whichever can be done with the smaller
number of points, is selected. The pointing spacing in total power
simulations is, again, $ 0.36\, {\rm PB} $ in lattice grids.


\medskip
{\bf Integration time}:

The total observation time of the ALMA 12-m array is defined by the
{\tt totaltime} parameter, while that of the ACA 7-m and total power
arrays is defined by $ {\tt totaltime} \, \times \, {\tt acaratio} $.
Set $ {\tt acaratio} \, = 0 $ or $ 3 $ for Cycle 1 simulations.
In general, the integration time (dump interval) of simulations is
defined by the {\tt integration} parameter with an exception.
Since the ACA total power array always observes larger areas compared
to the ALMA 12-m and ACA 7-m arrays, it is possible that the ACA total power array
can not cover all pointings in the given observation time 
($ {\tt totaltime} \, \times \, {\tt acaratio} $). 
In such a case, the integration time in total power simulation is scaled
so that the all pointings are observed at least once in its
observation time, i.e., 
${\tt integration\_TP} \, = \, {\tt totaltime} \, \times \, {\tt acaratio}
\, / \, ( ${\tt the number of total power pointings}$ ) $.


\medskip
{\bf Imaging}:

The CLEAN algorithm is used in {\tt simalma} to generate images
from visibilities. The visibilities are weighted to UV-plane using
Briggs weighting.
When ACA observations are simulated, the image generated from ACA 7-m
and total power simulations is taken as a model image of CLEAN in
combined imaging.



%%%%%%%%%%%%%%%%%%%%%%%%%%%%%%%%%%%%%%%%%%%%%%%%%%%%%%%%%%%%%%%%%
