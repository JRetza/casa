%%%%%%%%%%%%%%%%%%%%%%%%%%%%%%%%%%%%%%%%%%%%%%%%%%%%%%%%%%%%%%%%%
%%%%%%%%%%%%%%%%%%%%%%%%%%%%%%%%%%%%%%%%%%%%%%%%%%%%%%%%%%%%%%%%%
%%%%%%%%%%%%%%%%%%%%%%%%%%%%%%%%%%%%%%%%%%%%%%%%%%%%%%%%%%%%%%%%%

% STM 2007-04-13  split from previous version
% JO 2012-04-19 Release 3.4.0 edits
% JO 2013-12-10 Release 4.2.0 edits

\chapter[Appendix: Obtaining, Installing, and Customizing CASA]{Obtaining,
  Installing, and Customizing CASA}
\label{chapter:install}

\section{Installation On Linux}
\label{section:install.script}

To install CASA for Linux, we have packaged up a binary distribution
of CASA which is available as a downloadable tar file. We believe this
binary distribution works with most Linux distributions. While the
binary distribution is the only supported public distribution, most
CASA developers use RPMs for many third-party packages installed with
yum to do development on RedHat Enterprise Linux. Installing the
developer RPMs requires root access and we only provide developer
support for organizations which have a cooperative agreement to
participate in the development of CASA. We are currently working on
the development of a distribution for developers similar to our
standard binary distribution, but it is not yet ready for testing.


\subsection{Installation}

You do not have to have root or sudo permission, you can easily
install CASA, delete it, move it, and it works for many versions of
Linux. The one caveat is that CASA on Linux currently will not run if
the Security-Enhanced Linux option of the linux operating system is
set to enforcing. For the non-root install to work. SElinux must be
set to disabled or permissive (in /etc/selinux/config) or you must run
(as root):

\small
\begin{verbatim}
setsebool -P allow_execheap=1
\end{verbatim}
\normalsize

Otherwise, you will encounter errors like:


\small
\begin{verbatim}
casapy: error while loading shared libraries: 
  /opt/casa/casapy-20.0.5653-001/lib/liblapack.so.3.1.1: 
  cannot restore segment prot after reloc: Permission denied
\end{verbatim}
\normalsize

The non-root installation is thought to work on a wide variety of
linux platforms, see Sect.\,\ref{section:intro.obtaining} for the
latest supported OSs.

\subsubsection{Using more than one Linux version of CASA}
Up to CASA 3.3.0, CASA .rpm files prohibited installing more than one
CASA release at a time.  Starting with CASA 3.4.0, CASA .rpm files
allow previously installed CASA releases to remain installed.

To start a specific CASA version, type 

\small
\begin{verbatim}
casapy --release <VERSION>
\end{verbatim}
\normalsize
or 
\small
\begin{verbatim}
casapy -r <VERSION>
\end{verbatim}
\normalsize

where <VERSION> is a placeholder for the CASA version to be invoked, e.g. 3.3.0. 

Also, starting with CASA 3.4.0, the programs asdm2MS, casabrowser,
casalogger, casaplotms, casapy, casapyinfo, and casaviewer all take
the two new command line options: {\tt -r} and {\tt --release}.  These options
allow users to select a CASA program to run from the installed CASA
releases.




\subsection{Unsupported platforms}

The non-root install may work on other platforms not listed, please
let us know if you find that this binary distribution of CASA works on
other linux platforms. Also note, that the plotting tasks like plotxy
and plotcal are the ones that typically give problems for new
platforms, so a check of these after attempting an unsupported
platform installation is advisable.


\subsection{Download \& Unpack}

You can download the distribution tar file from

{\url http://casa.nrao.edu/casa\_obtaining.shtml}

This directory will contain two tar files one will be the 32-bit
version of CASA and the other will be the 64-bit version of CASA. The
file name of the 64-bit version ends with -64b.tar.gz. After
downloading the appropriate tar file, untar it with

\small
\begin{verbatim}
tar -zxf casapy-*.tar.gz
\end{verbatim}
\normalsize

This will extract a directory with the same basename as the tar
file. Change to that directory and add it to your path with, for
example, 

\small
\begin{verbatim}
PATH=`pwd`:\$PATH.
\end{verbatim}
\normalsize

After that, you should be able to start CASA by running
\small
\begin{verbatim}
casapy
\end{verbatim}
\normalsize


\section{Installation on Mac OS}

CASA for Macintosh is distributed as self-contained Macintosh
application. For installation purposes, this means that you can
install CASA by simply dragging the application to your hard disk. It
should be as easy as copying a file.

\begin{enumerate}
   \item Download the CASA disk image for your OS version from our
   download site\\ {\url http://casa.nrao.edu/casa\_obtaining.shtml}
   \item Open the disk image file (if your browser does not do so automatically).
   \item Drag the CASA application to the Applications folder of your hard disk.
   \item Eject the CASA disk image.
   \item Double-click the CASA application to run it for the first
     time. This ensures everything is properly updated if you had
     installed a previous version.
\end{enumerate}

You may need to unload the dbus before the copy will work

\small
\begin{verbatim}
launchctl remove org.freedesktop.dbus-session
launchctl remove org.freedesktop.dbus-system
\end{verbatim}
\normalsize

Versions after 12115 are 64bit only and will not work on older mac
intel machines The first time you launch the CASA application, it will
prompt you to set up an alias to the casapy command. You will be taken
through the process of creating several casapy symbolic links, it is
advisable to do so as this will allow you to run casapy from a
terminal window by typing casapy. Additionally, the viewer
(casaviewer), table browser (casabrowser), plotms (casaplotms), and
buildmytasks will also be available via the command line. Creating the
symbolic links will require that you have administrator privileges.


\subsubsection{Using more than one Mac version of CASA}
By dragging the CASA.app into the Applications folder, any previous
version of CASA will be replaced. If one would like to keep older
versions, one can simply rename them, e.g., to CASA-3.3.0.app. Double
clicking any of the CASA*.app applications will prompt to update the
symlinks to that specific CASA version. So any startup of {\tt
  casapy}, {\tt casaviewer}, {\tt casaplotms} will point to that
version. If one decides to switch to a different version, just double
click the respective CASA*.app and follow the instructions to update
the symlinks.



\section{Startup}
\label{section:install.startup}

in a terminal type\\
\small
\begin{verbatim}
casapy
\end{verbatim}
\normalsize

and the world of CASA will open its doors for you. 

There are a number of options to {\tt casapy} (see {\tt casapy --help}):
Options are: 
\small
\begin{verbatim}
   --rcdir     directory
   --logfile   logfilename              specify the name of the log
                                         file if other than casapy-DATE.log
   --maclogger                          will use the Mac Console program for the logger
   --log2term                           output the logger text in the terminal
   --nologger                           run without launching the logger
   --nologfile                          does not create a logfile 
   --nogui                              will not open the logger GUI 
   --colors=[NoColor|Linux|LightBG]     selects color theme for prompt task inputs
   --noipython                          does not launch ipython
                                         (useful when combined with the -c option)
   --release <VERSION>                  launches CASA version
                                         <VERSION> when installed as Linux rpm
   -r <VERSION>                         alias for --release
   -c filename-or-expression            execute a CASA python script from the command line 
   --help                               print this text and exit
\end{verbatim}
\normalsize

E.g. you can execute a CASA script {\tt script.py} directly with the command
\small
\begin{verbatim}
casapy -c script.py
\end{verbatim}
\normalsize


You can also launch the {\tt plotms} and {\tt viewer} GUIs separately
without starting CASA itself. To do so, type:

\small
\begin{verbatim}
casaplotms
\end{verbatim}
\normalsize
to launch {\tt plotms} and 
\small
\begin{verbatim}
casaviewer
\end{verbatim}
\normalsize
for the {\tt viewer}.




\section{Startup Customization}
\label{section:install.customization}

There are two initialization files that are loaded upon startup. The
first is loaded very early in the startup of casapy:

\small
\begin{verbatim}
	~/.casa/prelude.py
\end{verbatim}
\normalsize

This allows for limited customization of the casapy environment,
e.g. setting the path to an alternate logger. The second startup file
should be used for most purposes:

\small
\begin{verbatim}
	~/.casa/init.py
\end{verbatim}
\normalsize

This file is loaded just before the casapy prompt is display. This is
the place where users can load their own python modules and casa
tasks. For example ~/.casa/init.py might contain:

\small
\begin{verbatim}
	import os
	sys.path.insert(1,os.environ['HOME']+os.sep+"python")
	import analysisUtils as aU
\end{verbatim}
\normalsize

and analysisUtils.py might contain:

\small
\begin{verbatim}
	import numpy as np
	from mpfit import mpfit
	from pylab import *
	from numpy.fft import fft
	from scipy import polyfit
	import taskinit as ti
	from importasdm import importasdm
 \end{verbatim}
\normalsize


Many options can also be set in the file 


\small
\begin{verbatim}
~/.casarc
\end{verbatim}
\normalsize

E.g. 

\small
\begin{verbatim}
#
# Set these so that bug(), ask(), etc. know who you are
#
userinfo.name:  Sheila User
userinfo.email: suser@nrao.edu
userinfo.org:   NRAO

#NOTE: Fill this value in as appropriate - the units are MB
#It is important that you not set this value larger than your actual
#physical memory
#system.resources.memory: 2000
#help.popup.type: mb3long

#catalog
catalog.gui.auto:       T
catalog.confirm:        T
catalog.view.PostScript:        ghostview
catalog.edit.ascii:             xterm -e vi

#logger
#logger.file:    ./aips++.log
#logger.height:  12
logger.default: screen


#progress meter GUI pop-ups - disable
progress.show: F

#toolmanager - disable
toolmanager.gui.auto:   F


#Use current working directory for cache/scratch files
user.aipsdir: .
user.cache: .
user.directories.work:  .
user.initfiles: almainit.g

#viewer
display.axislabels: on
display.colormaps.defaultcolormap: Hot Metal 1

#development

#ms.async: ddd ./ms %s
\end{verbatim}
\normalsize


\section{Updating the data repository}
\label{section:install.datarep}


Each CASA release for linux comes with an up to date data repository
(containing information such as observatory coordinates, calibrator
models, leap second tables, etc.). However, the files that make up the
data repository are updated regularly. Therefore, if you install (or
have installed) a release that is a few weeks to a month old, it makes
sense to update the data repository because it is very easy. On a Mac,
the data repository is updated every time you run 'casapy'. For Linux,
you just change to the {\tt 'data} directory in the installation root and
run {\tt "svn update"}. This will sync your local copy of the CASA data
repository with the repository maintained by the CASA group, e.g. 

\small
\begin{verbatim}
cd /usr/lib64/casapy/data/
svn update
\end{verbatim}
\normalsize

will start syncing:

\small
\begin{verbatim}
bash-3.2$ svn update
U    geodetic/IERSeop97/table.f0
U    geodetic/IERSeop97/table.dat
U    geodetic/IERSeop97/table.lock
U    geodetic/Observatories/table.f0
U    geodetic/Observatories/table.f1
U    geodetic/Observatories/table.dat
...
\end{verbatim}
\normalsize

We suggest that a cron job is created to repeat that procedure
regularly. 


%%%%%%%%%%%%%%%%%%%%%%%%%%%%%%%%%%%%%%%%%%%%%%%%%%%%%%%%%%%%%%%%%
%%%%%%%%%%%%%%%%%%%%%%%%%%%%%%%%%%%%%%%%%%%%%%%%%%%%%%%%%%%%%%%%%
