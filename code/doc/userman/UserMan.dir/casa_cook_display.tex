%%%%%%%%%%%%%%%%%%%%%%%%%%%%%%%%%%%%%%%%%%%%%%%%%%%%%%%%%%%%%%%%%
%%%%%%%%%%%%%%%%%%%%%%%%%%%%%%%%%%%%%%%%%%%%%%%%%%%%%%%%%%%%%%%%%
%%%%%%%%%%%%%%%%%%%%%%%%%%%%%%%%%%%%%%%%%%%%%%%%%%%%%%%%%%%%%%%%%

% STM 2007-04-13  split from previous version
% STM 2007-08-24  quick update
% DK  2007-10-10  beta release
% STM 2007-10-10  spell-check
% DK  2007-10-11  update
% DK  2007-10-12  update
% DK  2008-10-02  update patch3
% JO 2010-03-03 updates for 3.0.1
% JO 2011-10-08  Release 3.3.0 edits
% JO 2011-05-09  Release 3.4 edits and Harald Kuemmel edits. 

\chapter{Visualization With The CASA Viewer}
\label{chapter:display}

This chapter describes how to display data with the {\tt casaviewer}
either as a stand-alone or through the {\tt viewer} task. You can
display both images and Measurement Sets in the viewer. We are,
however, splitting the functionalities into two different tasks {\tt
  imview} for images and {\tt msview} for measurement sets and
recommend those two tasks as they have improved functionality. 

%%%%%%%%%%%%%%%%%%%%%%%%%%%%%%%%%%%%%%%%%%%%%%%%%%%%%%%%%%%%%%%%%
%%%%%%%%%%%%%%%%%%%%%%%%%%%%%%%%%%%%%%%%%%%%%%%%%%%%%%%%%%%%%%%%%
\section{Starting the {\tt viewer}}
\label{section:display.start}

\begin{figure}[h!]
\begin{center}
\pngname{viewer_n5921_1}{4.3}
\pngname{viewer_n5921_2}{2.1}
\caption{\label{fig:viewer_start} The {\bf Viewer Display Panel} (left) and 
{\bf Data Display Options} (right) panels that appear when the 
{\tt viewer} is called with the image cube from NGC5921
({\tt viewer('ngc5921.demo.cleanimg.image')}).  The initial display is
of the first channel of the cube.}
\hrulefill
\end{center}
\end{figure}

Within the casapy environment, the {\tt viewer} task
can be used to display an image or MS.  The inputs are:
\small
\begin{verbatim}
#  viewer :: View an image or visibility data set.

infile        =         ''   #   (Optional)  Name of file to visualize.
displaytype   =   'raster'   #   (Optional)  Type of visual rendering
                             #   (raster, contour, vector or marker).
                             #   lel  if an lel expression is given
                             #   for infile (advanced).

\end{verbatim}
\normalsize

Examples of starting the {\tt viewer}:
\small
\begin{verbatim}
  CASA <1>: viewer()

  CASA <2>: viewer('ngc5921.demo.ms')

  CASA <3>: viewer('ngc5921.demo.cleanimg.image')

  CASA <5>: viewer('ngc5921.demo.cleanimg.image', 'contour')
  
  CASA <5>: viewer('"ngc5921.demo.cleanimg.image"^2', 'lel')
\end{verbatim}
\normalsize
The first of these creates
an empty {\bf Viewer Display Panel} 
(\S~\ref{section:display.viewerGUI.displaypanel}) and a {\bf Load Data} 
window (\S~\ref{section:display.viewerGUI.load}) .  The second starts the
{\tt viewer} loaded with a Measurement Set.  The third example
starts the {\tt viewer} with an image cube 
(see Figure~\ref{fig:viewer_start}).  

Example four brings up a display panel as it was when its state was saved
to the given 'restore' file ({\tt ngc5921.usecase.clean.image.rstr}).
This includes the data displayed as well as options and viewer
settings.  (See \S~\ref{section:display.viewerGUI.save-restore},
{\bf Saving and Restoring Viewer State}).

Examples five and six are less common cases, which make use of the second
parameter ({\tt displaytype}).  Example five displays the image in contour
form.  Example six uses 'Lattice (Image) Expression Language' to display the
square of the image data.

{\bf NOTE:} the {\tt viewer} task determines file types (images, MSs,
restore files) automatically.  It is no longer necessary to specify
{\tt filetype='ms'} explicitly.

\begin{figure}[h!]
\begin{center}
\pngname{viewer_n5921ms_1}{4.3}
\pngname{viewer_n5921ms_2}{2.1}
\caption{\label{fig:viewer_start_ms} The {\bf Viewer Display Panel}
(left) and {\bf Data Display Options} (right) panels that appear when the 
{\tt viewer} is called with the NGC5921 Measurement Set
({\tt viewer('ngc5921.demo.ms')}).} 
\hrulefill
\end{center}
\end{figure}

%%%%%%%%%%%%%%%%%%%%%%%%%%%%%%%%%%%%%%%%%%%%%%%%%%%%%%%%%%%%%%%%%
\subsection{Running the CASA viewer outside {\tt casapy}}
\label{section:display.start.casaviewer}

{\tt casaviewer} is the name of the stand-alone viewer application that
is available with a CASA installation.  From the operating system prompt,
the following commands are equivalent to the {\tt casapy} task commands
given in the previous Section:

\small
\begin{verbatim}
  casaviewer &
  
  casaviewer ms_filename &
  
  casaviewer image_filename &
  
  casaviewer restore_filename &
  
  casaviewer image_filename contour &
  
  casaviewer '"image_filename"^2' lel &
\end{verbatim}
\normalsize

%%%%%%%%%%%%%%%%%%%%%%%%%%%%%%%%%%%%%%%%%%%%%%%%%%%%%%%%%%%%%%%%%
%%%%%%%%%%%%%%%%%%%%%%%%%%%%%%%%%%%%%%%%%%%%%%%%%%%%%%%%%%%%%%%%%
\section{The {\tt viewer} GUI}
\label{section:display.viewerGUI}

The CASA {\tt viewer} application consists of a number of
graphical user interface (GUI) windows that respond to mouse and
keyboard input.  Here we describe the {\bf Viewer Display Panel} 
(\S~\ref{section:display.viewerGUI.displaypanel}) and the
{\bf Load Data} window (\S~\ref{section:display.viewerGUI.load}).
They are used for both image and MS viewing.  Several other windows
are context-specific and are described in the sections on viewing
images (\S~\ref{section:display.image}) and Measurement Sets
(\S~\ref{section:display.ms}).

%%%%%%%%%%%%%%%%%%%%%%%%%%%%%%%%%%%%%%%%%%%%%%%%%%%%%%%%%%%%%%%%%

\subsection{The Viewer Display Panel}
\label{section:display.viewerGUI.displaypanel}

% \begin{figure}[h!]
% \gname{viewer0}{4}
% \caption{\label{fig:viewer0} Viewer Display Panel with no data
%   loaded. Each section of the GUI is explained below} 
% \hrulefill
% \end{figure}
 
The Viewer Display Panel is the the window that actually displays the
image or MS.  This is shown in the left panels of 
Figures~\ref{fig:viewer_start} and \ref{fig:viewer_start_ms}.
Note that this panel is the same whether an image or MS is
being displayed.
 
At the top of the Viewer Display Panel are the menus:
\begin{itemize}
\item {\bf Data}
  \begin{itemize}
      \item  {\tt Open} --- choose a data file to load and display
      \item  {\tt Register} --- select/de-select the (previously-loaded)
             data file(s) which should display right now (menu expands
	     to the right showing all loaded data) 
      \item  {\tt Close} --- close (unload) selected data file (menu
             expands to the right)
      \item  {\tt Adjust} --- open the Data Display Options ('Adjust') panel 
      \item  {\tt Save as...} --- save/export data to a file
      \item  {\tt Print} --- print the displayed image
      \item  {\tt Save Panel State} --- to a 'restore' file (xml format)
      \item  {\tt Restore Panel State} --- from a restore file
      \item  {\tt Close Panel} --- close the Viewer Display Panel (will exit
             if this is the last display panel open)
      \item  {\tt Quit Viewer} --- close all display panels and exit
  \end{itemize}
\item {\bf Display Panel}
  \begin{itemize}
      \item {\tt New Panel} --- create another Viewer Display Panel (cleared)
      \item {\tt Panel Options} --- open the Display Panel's options window
      \item  {\tt Save Panel State}
      \item  {\tt Restore Panel State}
      \item {\tt Print} --- print displayed image
      \item {\tt Close Panel} --- close the Viewer Display Panel (will exit if
            this is the last display panel open)
  \end{itemize}
\item {\bf Tools}
  \begin{itemize}
      \item {\tt Spectral Profile} --- plot frequency/velocity profile
                 of point or region of image
  \end{itemize}
\item {\bf View}
  \begin{itemize}
      \item {\tt Main Toolbar} --- show/hide top row of icons
      \item {\tt Mouse Toolbar} --- show/hide second row of
                 mouse-button action selection icons
      \item {\tt Animator} --- show/hide tapedeck control panel
      \item {\tt Position Tracking} --- show/hide bottom position
                 tracking report box  
      \item {\tt regions} --- show/hide region box  
  \end{itemize}
\end{itemize}

\begin{figure}[h!]
\begin{center}
\pngname{viewer_maintoolbar}{5}
\caption{\label{fig:viewer_maintoolbar} The display panel's
{\bf Main Toolbar} appears directly below the menus and contains
'shortcut' buttons for most of the frequently-used menu items.}
\hrulefill
\end{center}
\end{figure}
Below this is the {\bf Main Toolbar} (Figure~\ref{fig:viewer_maintoolbar}),
the top row of icons for fast access to some of these menu items:
\begin{itemize}
   \item {\bf folder} ({\tt Data:Open} shortcut) --- show the Load Data panel
   \item {\bf wrench} ({\tt Data:Adjust} shortcut) --- show the Data Display
              Options ('Adjust') panel 
   \item {\bf panels} ({\tt Data:Register} shortcut) --- show the 
              menu of loaded data
   \item {\bf delete} ({\tt Data:Close} shortcut) --- closes/unloads 
              selected data
   \item {\bf save data} --- save/export data to a file
   \item {\bf new panel} ({\tt Display Panel:New Panel})
   \item {\bf panel wrench} ({\tt Display Panel:Panel Options}) --- show
              the Display Panel's options window 
   
   \item  {\bf save panel} -- save panel state to a 'restore' file
   \item  {\bf restore panel} -- restore panel state from a restore file
   \item  {\bf profile panel} -- open the spectral profiler
  % \item {\bf region save} ({\tt Tools:Region Manager}) --- save/control
    regions.
    
    (Note: some of these newer buttons do not appear in older figures of
    this document).
   
   \item {\bf print} ({\tt Display Panel:Print}) --- print data
   \item {\bf magnifier box} --- zoom out all the way
   \item {\bf magnifier plus} --- zoom in (by a factor of 2)
   \item {\bf magnifier minus} --- zoom out (by a factor of 2)
\end{itemize}

\begin{figure}[h!]
\begin{center}
\pngname{viewer_mousetoolbar}{3}
\caption{\label{fig:viewer_mousetoolbar} The 
{\bf 'Mouse Tool' Bar} allows you to assign separate mouse buttons to
tools you control with the mouse within the image display area.  Initially,
zooming, color adjustment, and rectangular regions are assigned to the left,
middle and right mouse buttons,
respectively.}
\hrulefill
\end{center}
\end{figure}

Below this are the ten {\bf Mouse Tool} buttons
(Figure~\ref{fig:viewer_mousetoolbar}). These allow assignment of
{\it each} of the three mouse buttons to a different operation on the display
area. Clicking a mouse tool icon will [re-]assign {\bf the mouse button that
was clicked} to that tool.  The icons show which mouse button is currently
assigned to which tool.  

The 'escape' key can be used to cancel any mouse tool operation that was
begun but not completed, and to erase any tool showing in the display area.
\begin{itemize}
   \item {\bf Zooming (magnifying glass icon):}
     To zoom into a selected area, press the Zoom tool's mouse button
     (the {\bf left} button by default) on one corner of the desired
     rectangle and drag to the desired opposite corner. Once the button is
     released, the zoom rectangle can still be moved or resized by dragging.
     To complete the zoom, double-click inside the selected rectangle
     (double-clicking {\it outside} it will zoom {\it out} instead).
   \item {\bf Panning (hand icon):} Press the tool's mouse button on a 
     point you wish to move, drag it to the position where you want it
     moved, and release. {\it Note: The arrow keys, Page Up, Page Down,
     Home and End keys can also be used to scroll through your data any time
     you are zoomed in. (Click on the main display area first, to be sure
     the keyboard is 'focused' there).}
   \item {\bf Stretch-shift colormap fiddling (crossed arrows):} This is
     usually the handiest color adjustment; it is assigned to the {\bf middle}
     mouse button by default.
   \item {\bf Brightness-contrast colormap fiddling (light/dark sun)} 
   \item {\bf Positioning (bombsight):} This tool can place a 'crosshair'
     marker on the display to select a position. It is used to flag
     Measurement Set data or to select an image position for spectral profiles.
     Click on the desired position with the tool's mouse button to place
     the crosshair; once placed you can drag it to other locations.
     Double-click is not needed for this tool.  
     See \S~\ref{section:display.viewerGUI.displaypanel.region} for more
     detail.
   \item {\bf Rectangle, Ellipse and Polygon region drawing:} The rectangle
     region tool is assigned to the {\bf right} mouse button by default.
     As with the zoom tool, a rectangle region is generated by dragging with
     the assigned mouse button; the selection is confirmed by double-clicking
     within the rectangle.
     An ellipse regions is created by dragging with the assigned mouse button.
     In addition to the elliptical region, also its surrounding rectangle is
     shown on the display. The selection is confirmed by double-clicking within
     the ellipse.
     Polygon regions are created by clicking the assigned mouse button
     at the desired vertices, clicking the final location twice to finish.
     Once created, a polygon can be moved by dragging from inside, or
     reshaped by dragging the handles at the vertices.  Double-click inside to
     confirm region selection.
     See \S~\ref{section:display.viewerGUI.displaypanel.region} for the uses
     of this tool.
   \item {\bf Polyline drawing:}
     A polyline can be created by selecting this tool. It is manipulated
     similarly to the polygon region tool: create segments by clicking at
     the desired positions and then double-click to finish the line.
     [Uses for this tool are still to be implemented].
   \item {\bf Distance tool:}
     After selecting the distance tool by assigning any mouse button to it,
     distances on the image can conveniently be measured by dragging the
     mouse with the assigned button pressed. The tool measures the distances
     along the world coordinate axes and along the hypotenuse. If the units
     in both axes are $[deg]$, the distances are displayed in $[arcsec]$.
\end{itemize}

The main {\bf Display Area} lies below the toolbars.

On the right side of the display area is an {\bf Animator} panel.  The most prominent
feature is the ``tape deck'' which provides movement between image planes
along a selected third dimension of an image cube. This set of buttons is
only enabled when a registered image reports that it has more than one plane
along its 'Z axis'. In the most common case, the animator selects the frequency
channel. From left to right, the tape deck controls allow the user to:
\begin{itemize}
   \item {\bf rewind} to the start of the sequence (i.e., the first plane)
   \item {\bf step backwards} by one plane
   \item {\bf play backwards}, or repetitively step backwards
   \item {\bf stop} any current play
   \item {\bf play forward}, or repetitively step forward
   \item {\bf step forward} by one plane
   \item {\bf fast forward} to the end of the sequence
\end{itemize}
To the right of the tape deck is an editable text box indicating the
current frame (channel) number and a label showing the total number of
frames. Below that is a box for controlling the (nominal) animation
speed. To the right is a 'normal/blink' toggle.

'Blink' mode is useful when more than one raster image is
registered. In that mode, the tapedeck controls {\it which image} is
displayed at the moment rather than the particular image plane
(set that in 'Normal' mode first). The registered images must cover the
same portion of the sky and use the same coordinate projection.

{\bf Note:} {\em In 'normal' mode, it is advisable to have only ONE
raster image registered at a time, to avoid confusion. Unregister (or
close) the others).}

Underneath the Animator Panel is the {\bf Position Tracking} panel.  As
the mouse moves over the main display, this panel shows information such as
flux density, position (e.g. RA and Dec), Stokes, and frequency (or velocity),
for the point currently under the cursor.  Each registered image/MS displays
its own tracking information.  Tracking can be 'frozen' (and unfrozen again)
with the space bar.  (Click on the main display area first, to be sure the
keyboard is 'focused' there).

The Animator or Tracking panels can be hidden or detached (and later
re-attached) by using the boxes at upper right of the panels; this is
useful for increasing the size of the display area.  (Use the 'View'
menu to show a hidden panel again).  The individual tracking areas
(one for each registered image) can be hidden using the checkbox at
upper left of each area.

%%%%%%

\subsection{Saving and Restoring Display Panel State}
\label{section:display.viewerGUI.save-restore}

It is straightforward to save a display panel's current state (what
data is on display along with data and panel settings).  Select 'Save Display
Panel State to File'
(seventh icon from left in Figure \ref{fig:viewer_maintoolbar})
and confirm the filename.  It is strongly
advisable (but not required) to retain the file's '.rstr' extension.  

Press 'Restore Display Panel State from File' (the button to the right
of 'Save Display Panel State to File') to
choose a previously-created restore file.  You can also select restore
files from the 'Load Data' window.

It is possible to restore MSs or images, multiple layers such as
contour-over-raster, and LEL displays.  You can also the save the panel
state with no data loaded, to restore preferred initial settings such
as overall panel size.  Animation and zoom state should
likewise restore themselves.

Restore is fairly forgiving about data location, and will find files located:
\begin{itemize}
  \item in the original location recorded in the restore file
  \item in the current working directory (where you started the viewer)
  \item in the restore file's directory
  \item in the original location relative to the restore file
\end{itemize}
This means that restore files will generally work if moved together with
data files.  The process is less forgiving if you save the display of an
LEL (image) expression, however; the files must be in the locations specified
in the original LEL expression.  If a data file is {\bf not} found, restore
will attempt to proceed but results will vary. 

Restore files are in ascii (xml) format, and some obvious manual edits are
possible.  However, these files are longer and more complex than you might
imagine.  Use caution, and back up restore files you want to preserve.
If you make a mistake, the viewer may not recognize the file as a restore
file; other unexpected results could also occur.  It is usually easier
and safer to make changes on the display panel and then save the restore
file again.

%%%%%%

\subsection{Region Selection and Positioning}
\label{section:display.viewerGUI.displaypanel.region}

You can draw regions or select positions on the display with the mouse,
once you have selected the appropriate tool(s) on the {\tt Mouse Toolbar}
(see above).

The {\tt Rectangle Region} drawing tool currently works for the following: 
\begin{itemize}
  \item Region statistics reporting for images,
  \item Region spectral profiles for images, via the
        {\tt Tools:Spectral Profile} menu,
  \item Flagging of Measurement Sets
  \item Creating and Saving an image region for various types of analysis
        (\S~\ref{section:display.image.rgnmgr})
  \item Selecting Clean regions interactively (\S~\ref{section:im.clean.interactive})
\end{itemize}

The {\tt Polygon Region} drawing has the same uses, except that polygon region
flagging of an MS is not supported.

The {\tt Positioning} crosshair tool works for the last two of the above.

The {\tt Spectral Profile} display
(see \S~\ref{section:display.image.specprof}), when active, updates on
{\em each change} of the rectangle, polygon, or crosshair.  
Flagging with the crosshair also responds to single click or drag.

Region statistics are printed in the terminal window (not the logger)
by double-clicking the completed region. The statistics are also
displeyed in the region dock
(\S~\ref{section:display.image.rgnmgr}). The {\tt Rectangle Region}
tool's mouse button must also be double-clicked to confirm an MS
flagging edit.

Here is an example of region statistics from the viewer:
\small
\begin{verbatim}
     Frequency       Velocity         Stokes BrightnessUnit       BeamArea 
 1.41804e+09Hz     499.87km/s              I        Jy/beam        30.1576 
          Npts            Sum           Flux           Mean            Rms 
        114359   2.062986e+00   6.840674e-02   1.803956e-05   8.969640e-04 
       Std dev        Minimum        Maximum 
  8.967865e-04  -3.730000e-03   3.895760e-03 
\end{verbatim}
\normalsize

%%%%%%%%%%%%%%%%%%%%%%%%%%%%%%%%%%%%%%%%%%%%%%%%%%%%%%%%%%%%%%%%%

\subsection{The Load Data Panel}
\label{section:display.viewerGUI.load}
\begin{figure}[h!]
\begin{center}
\pngname{viewer_load}{4}
\caption{\label{fig:viewer_load} The {\bf Load Data - Viewer} panel
that appears if you open the {\tt viewer} without any {\tt infile}
specified, or if you use the {\tt Data:Open} menu or Open icon.
You can see all available files (e.g. images and MS) in the current directory
that could be loaded into the viewer.} 
\hrulefill
\end{center}
\end{figure}

You can use the {\bf Load Data - Viewer} GUI to interactively
choose images or MS to load into the viewer.  An example of
this panel is shown in Figure~\ref{fig:viewer_load}.  This
panel is accessed through the {\tt Data:Open} menu or Open icon
of the {\bf Viewer Display Panel}.  It also appears if you open 
the {\tt viewer} without any {\tt infile} specified.

Selecting a file on disk in the {\tt Load Data} panel will
provide options for how to display the data. Images can be displayed
as: 
\begin{enumerate}
\item raster image, 
\item contour map, 
\item vector map,
\item marker map.  
\end{enumerate}

You can also enter a 'Lattice Expression Language' (LEL) expression in the
box provided (\S~\ref{section:analysis.pars.lattice}). The images used in the
LEL expression should have the same coordinates and extents).

A MS can only be displayed as a raster.


%%%%%%
\subsubsection{Registered vs. Open Datasets}
\label{section:display.viewerGUI.load.register}

When you 'load' data as described above, it is first {\em opened}, and then
{\em registered} on all existing {\tt Display Panels}.  The distinction
is subtle.  An 'open' dataset has been prepared in memory from disk; it
may be registered (enabled for drawing) on one {\tt Display Panel} and not
on another.  All open datasets will have a tab in the {\tt Data Options}
window, whether currently registered or not.  On the other hand, only those
datasets registered on a particular panel will show in its {\tt Tracking}
area.

At present, it is useful to have more than one image registered on a
panel {\em only} if you are displaying a contour image over a raster image
(\S~\ref{section:display.image.viewcontours}) or 'blinking' between images
(see {\bf Animator} in \S~\ref{section:display.viewerGUI.displaypanel}).
(In future we also hope to provide transparent overlay of raster images).

It is the user's responsibility -- and highly advisable -- to unregister
(or close) datasets that are no longer in use, using the {\tt Register}
or {\tt Close} toolbutton or menu.  In future the viewer will attempt
to aid in unregistering datasets which are not 'compatible' with a
newly-loaded one (different sky area, e.g., or MS vs. image).

If you close a dataset, you must reload it from disk as described above
to see it again.  That can take a little time for MSs, especially.  If you
unregister a dataset, it is set to draw immediately when you re-register it,
with its options as you have previously set them.  In general, close
unneeded datasets but unregister those you'll be working with again.

\begin{figure}[h!]
\begin{center}
\pngname{viewer_save}{4}
\caption{\label{fig:viewer_save} The {\bf Save Data - Viewer} panel
that appears when pressing the 'save data' icon in the {\bf Main Toolbar}
(Figure~\ref{fig:viewer_maintoolbar}).} 
\hrulefill
\end{center}
\end{figure}
\subsection{The Save Data Panel}
\label{section:display.viewerGUI.save}
The user can use the {\bf Save Data - Viewer} GUI to export images from the
viewer to a file. The panel is shown Figure~\ref{fig:viewer_save}.
This panel is accessed pressing the 'save data' icon in the 
{\bf Main Toolbar} (Figure~\ref{fig:viewer_maintoolbar}).
The upper part lists all images that can be exported to disk.
To save an image to a file, the use can either enter the new filename in the
box labeled 'save to:' followed by the save-button (alternatively the
'Enter'-key), or press 'browse...' and specify a filename via a
standard filebrowser GUI.

Independent on the original format, images can be exported in both,
FITS format and CASA image format.
 
%%%%%%%%%%%%%%%%%%%%%%%%%%%%%%%%%%%%%%%%%%%%%%%%%%%%%%%%%%%%%%%%%
%%%%%%%%%%%%%%%%%%%%%%%%%%%%%%%%%%%%%%%%%%%%%%%%%%%%%%%%%%%%%%%%%
\section{Viewing Images}
\label{section:display.image}

There are several options for viewing an image.  These are seen
at the right of the {\bf Load Data - Viewer} panel 
described in \S~\ref{section:display.viewerGUI.load} and shown in 
Figure~\ref{fig:viewer_load_image} after selecting an image.  They are:
\begin{itemize}
   \item {\tt raster image} --- a greyscale or color image,
   \item {\tt contour map} --- contours of intensity as a line plot,
   \item {\tt vector map} --- vectors (as in polarization) as a line plot,
   \item {\tt marker map} --- a line plot with symbols to mark positions.
\end{itemize}

The {\tt raster image} is the default image display, and is what you
get if you invoke the {\tt viewer} from {\tt casapy} with an image
file name.  In this case, you will need to use the {\tt Open} menu to
bring up the {\bf Load Data} panel to choose a different display.

\begin{figure}[h!]
\begin{center}
\pngname{viewer_load_image}{4}
\caption{\label{fig:viewer_load_image} The {\bf Load Data - Viewer} panel
as it appears if you select an image.  You can see all options
are available to load the image as a {\tt raster image}, 
{\tt contour map}, {\tt vector map}, or {\tt marker map}.
In this example, clicking on the {\tt raster image} button would 
bring up the displays shown in Figure~\ref{fig:viewer_start}.}
\hrulefill
\end{center}
\end{figure}

%%%%%%%%%%%%%%%%%%%%%%%%%%%%%%%%%%%%%%%%%%%%%%%%%%%%%%%%%%%%%%%%%

\subsection{Viewing a raster map}
\label{section:display.image.raster}

A raster map of an image shows pixel intensities in a two-dimensional
cross-section of gridded data with colors selected from a finite set
of (normally) smooth and continuous colors, i.e., a colormap.

% \begin{figure}[h!]
% \gname{viewer1}{3.5}
% \gname{viewer_loaddata}{3.5}
% \caption{\label{fig:viewer1} casaviewer: Illustration of a raster
%   image in the Viewer Display Panel(left) and the Load Data panel
%   (right).} 
% \hrulefill
% \end{figure}

Starting the {\tt casaviewer} with an image as a raster map will look
something like the example in Figure~\ref{fig:viewer_start}. 
 
You will see the GUI which consists of two main windows, entitled
"Viewer Display Panel" and "Load Data". In the "Load Data" panel, you
will see all of the viewable files in the current working directory along
with their type (Image, Measurement Set, etc).  After selecting a file, you
are presented with the available display types (raster, contour,
vector, marker) for these data. Clicking
on the button {\tt raster image} will create a display
Fig.~\ref{fig:viewer_start}. 

The data display can be adjusted by the user as needed.  This
is done through the {\bf Data Display Options} panel.  This window
appears when you choose the {\tt Data:Adjust} menu or use the
wrench icon from the {\bf Main Toolbar}.  This also comes up
by default along with the {\bf Viewer Display Panel} when the
data is loaded.

The {\bf Data Display Options} window is shown in the right panel
of Figure~\ref{fig:viewer_start}.  It consists of a tab for each
image or MS loaded, under which are a cascading series of expandable
categories.  For an image, these are:
\begin{itemize}
   \item {\bf display axes}
   \item {\bf hidden axes}
   \item {\bf basic settings}
   \item {\bf position tracking}
   \item {\bf axis labels}
   \item {\bf axis label properties}
   \item {\bf beam ellipse}
   \item {\bf color wedge}
\end{itemize}
The {\bf basic settings} category is expanded by
default.  To expand a category to show its options, click on it with
the left mouse button.


%%%%%%
\subsubsection{Raster image --- display axes}
In this category the physical axes (i.e. Right Ascension, Declination,
Velocity, Stokes) to be displayed can be selected and assigned to the
x, y, and z axes of the display. If your image has a fourth axis
(typically Stokes), it can be controlled by a slider within the {\tt
  hidden axes} drop-down.

\subsubsection{Raster image --- basic settings}
\label{section:display.image.raster.adjust.basic}

This roll-up is open by default.  It has some commonly-used parameters
that alter the way the image is displayed; three of these affect the
colors used. An example of this part of the panel is shown in
Figure~\ref{fig:viewer_raster_basic}.

\begin{figure}[h!]
\begin{center}
\pngname{viewer_ras_basic}{4}
\caption{\label{fig:viewer_raster_basic} The {\tt basic settings}
category of the {\bf Data Display Options} panel
as it appears if you load the image as a {\tt raster image}.
This is a zoom-in for the data displayed in Figure~\ref{fig:viewer_start}.}
\hrulefill
\end{center}
\end{figure}

The options available are:
\begin{itemize}

\item {\tt basic settings: aspect ratio}

This option controls the horizontal-vertical size ratio of data pixels
on screen.  {\tt fixed world} (the default) means that the aspect
ratio of the pixels is set according to the coordinate system of
the image (i.e., true to the projected sky). {\tt fixed lattice}
means that data pixels will always be square on the screen.  Selecting
{\tt flexible} allows the map to stretch independently in each
direction to fill as much of the display area as possible.

\item {\tt basic settings: pixel treatment}

This option controls the precise alignment of the edge of the current
'zoom window' with the data lattice.  {\tt edge} (the default) means
that whole data pixels are always drawn, even on the edges of the display.
For most purposes, {\tt edge} is recommended.  {\tt center} means that
data pixels on the edge of the display are drawn only from their centers
inwards. (Note that a data pixel's center is considered its 'definitive'
position, and corresponds to a whole number in 'data pixel' or 'lattice'
coordinates).

\item {\tt basic settings: resampling mode}

This setting controls how the data are resampled to the resolution of
the screen.  {\tt nearest} (the default) means that screen pixels are
colored according to the intensity of the nearest data point, so that
each data pixel is shown in a single color. {\tt bilinear} applies a
bilinear interpolation between data pixels to produce smoother looking images
when data pixels are large on the screen.  {\tt bicubic} applies an
even higher-order (and somewhat slower) interpolation.

\item {\tt basic settings: data range}

You can use the entry box provided to set the minimum and maximum data values
mapped to the available range of colors as a list {\tt [min, max]}.  
For very high dynamic range images,
you will probably want to enter a {\tt max} less than the data maximum 
in order to see detail in lower brightness-level pixels.
The next setting also helps very much with high dynamic range data.

\item {\tt basic settings: scaling power cycles}

This option allows logarithmic scaling of data values to colormap cells.  

The color for a data value is determined as follows: first, the value
is clipped to lie within the data range specified above, then mapped
to an index into the available colors, as described in the next
paragraph. The color corresponding to this index is determined finally
by the current colormap and its 'fiddling' (shift/slope) and
brightness/contrast settings (see {\bf Mouse Toolbar}, above).  Adding
a {\bf color wedge} to your image can help clarify the effect of the
various color controls.

The {\tt scaling power cycles} option controls the mapping of clipped data
values to colormap indices.  Set to zero (the default), a straight linear
relation is used.  For negative scaling values, a logarithmic mapping
assigns an larger fraction of the available colors to lower data values (this
is usually what you want). Setting {\tt dataMin} 
to something around the noise level
is often useful/appropriate in conjunction with a negative 'power cycles'
setting. 

For positive values, an larger fraction of the colormap is used for the high
data values\footnote{The actual functions are computed as follows:

For negative scaling values (say $-p$), the data is scaled linearly
from the range ({\tt dataMin} -- {\tt dataMax}) to the range (1 -- $10^{p}$).
Then the program takes the $\log$ (base 10) of that value (arriving at
a number from 0 to $p$) and scales that linearly to the number of
available colors.  Thus the data is treated as if it had $p$ decades
of range, with an equal number of colors assigned to each decade.

For positive scaling values, the inverse (exponential) functions are used.
If $p$ is the (positive) value chosen,  The data value is scaled linearly to
lie between 0 and $p$, and 10 is raised to this power, yielding a value in the
range (1 -- $10^{p}$).  Finally, that value is scaled linearly to the number
of available colors.}.

See Figure~\ref{fig:scalingpower} for sample curves.
\begin{figure}[h]
\begin{center}
\pngname{viewer_scalingpower}{3.6}
\caption{\label{fig:scalingpower} Example curves for {\tt scaling power cycles}.}
\hrulefill
\end{center}
\end{figure}

\item {\tt basic settings: colormap}

You can select from a variety of colormaps here.  {\tt Hot Metal},
{\tt Rainbow} and {\tt Greyscale} colormaps are the ones most commonly used.

\end{itemize}


%%%%%%
\subsubsection{Raster image --- other settings}
\label{section:display.image.raster.adjust.other}

Many of the other settings on the {\tt Data Options} panel for raster images
are self-explanatory, such as those which affect {\tt beam ellipse} drawing
(only available if your image provides beam data), or the form of the
{\tt axis labeling} and {\tt position tracking} information.  You can also
give your image a {\tt color wedge}, a key to the current mapping from data
values to colors.

You can control which of your image's axes are on the vertical and horizontal
display axes and which on the animation or 'movie' axis, within the
{\tt display axes} drop-down.  You must set the X, Y and Z (animation) axes
so that each shows a {\it different} image axis, in order for your choice
to take effect.

%%%%%%%%%%%%%%%%%%%%%%%%%%%%%%%%%%%%%%%%%%%%%%%%%%%%%%%%%%%%%%%%%
\subsection{Viewing a contour map}
\label{section:display.image.contour}

Viewing a contour image is similar to the process above. A contour map
shows lines of equal data value (e.g., flux density) for the
selected plane of gridded data (Figure~\ref{fig:viewer_con}).
Contour maps are particularly useful for overlaying on raster images so
that two different measurements of the same part of the sky can be shown
simultaneously (\S~\ref{section:display.image.viewcontours}).

Several {\tt basic settings} options control the contour levels used.
The contours themselves are specified by a list in the box {\tt
  Relative Contour Levels}.  These are defined relative to the two
other parameters, the {\tt Base Contour Level} (which sets what 0 in the
relative contour list corresponds to in the image), and the {\tt
  Unit Contour Level} (which sets what 1 in the relative contour list
corresponds to in the image).  Note that negative contours are usually
dashed. 
%Default values (selectable by the wrench symbol) are the
%minimum and maximum value of the image. 
 {\bf ALERT:} This scheme was
adopted in 2.4.0 and is slightly different to that used in previous
versions.

% \begin{figure}[h!]
% \gname{viewer5}{3.5}
% \gname{viewer_displaydata5}{3.5}
% \caption{\label{fig:viewer5} Example of a contour
%   image in the {\bf Viewer Display Panel} (left) and the 
%   {\bf Load Data} panel (right).} 
% \hrulefill
% \end{figure}
 
\begin{figure}[h!]
\begin{center}
\pngname{viewer_n5921_con_1}{4.3}
\pngname{viewer_n5921_con_2}{2.1}
\caption{\label{fig:viewer_con} The {\bf Viewer Display Panel}
(left) and {\bf Data Display Options} panel (right) after choosing
{\tt contour map} from the {\bf Load Data} panel.  The
image shown is for channel 11 of the NGC5921 cube, selected using
the {\bf Animator} tape deck, and zoomed in using the tool bar icon.
Note the different options in the open {\tt basic settings} category
of the {\bf Data Display Options} panel (as compared to {\tt raster image} in
Figure~\ref{fig:viewer_start}).} 
\hrulefill
\end{center}
\end{figure}

For example, it is relatively straightforward to set fractional
contours (e.g. ``percent levels''), e.g.:
\small
\begin{verbatim}
   Relative Contour Levels = [0.2, 0.4, 0.6, 0.8]
   Base Contour Level = 0.0
   Unit Contour Level = <image max>
\end{verbatim}
\normalsize
This maps the maximum to 1 and thus our contours are fractions of
the peak.

Another example shows how to set absolute values so that the contours
are given in flux density units (Jy):
\small
\begin{verbatim}
   Relative Contour Levels = [0.010, 0.0.020, 0.040, 0.080, 0.160, 0.320]
   Base Contour Level = 0.0
   Unit Contour Level = 1.0
\end{verbatim}
\normalsize
Here we have contours starting at 10mJy and doubling every contour.

We can also set contours in multiples of the image rms (``sigma''):
\small
\begin{verbatim}
   Relative Contour Levels = [-3,3,5,10,15,20]
   Base Contour Level = 0.0
   Unit Contour Level = <image rms>
\end{verbatim}
\normalsize
Here we have first contours at negative and positive 3-sigma.
You can get the image rms using the {\tt imstat} task 
(\S~\ref{section:analysis.imstat}) or using the Viewer statistics
tool on a region of the image 
(\S~\ref{section:display.viewerGUI.displaypanel.region}).

As a final example, not all images are of intensity, for example a
moment-1 image (\S~\ref{section:analysis.moments}) has units of
velocity.  In this case,
absolute contours will work fine, but by default the viewer will
set fractional contours but referred to the min and max velocity:
\small
\begin{verbatim}
   Relative Contour Levels = [0.2, 0.4, 0.6, 0.8]
   Base Contour Level = <image min>
   Unit Contour Level = <image max>
\end{verbatim}
\normalsize
Here we have contours spaced evenly from min to max, and this is
what you get by default if you load a non-intensity image (like
the moment-1 image).  See Figure~\ref{fig:viewer_rascon} for an
example of this.

%%%%%%%%%%%%%%%%%%%%%%%%%%%%%%%%%%%%%%%%%%%%%%%%%%%%%%%%%%%%%%%%%
\subsection{Overlay contours on a raster map}
\label{section:display.image.viewcontours}

Contours of either a second data set or the same data set can be used
for comparison or to enhance visualization of the data. The Data Options
Panel will have multiple tabs which allow adjusting each overlay
individually (Note tabs along the top).  {\bf Beware:} it's easy to forget
which tab is active!   Also note that {\tt axis labeling} is controlled
by the {\it first-registered} image overlay that has labeling turned on
(whether raster or contour), so make label adjustments within that tab.

To add a Contour overlay, open the {\bf Load Data} panel (Use the {\bf Data}
menu or click on the folder icon), select the data set and click on
{\tt contour map}.
See Figure~\ref{fig:viewer_rascon} for an example using NGC5921.

% \begin{figure}[h!]
% \gname{viewer_datadisplay1}{3.5}
% \gname{viewer3}{3.5}
% \caption{\label{fig:viewer_overlay}  Display of a contour
% overlay on top of a raster image.} 
% \hrulefill
% \end{figure}
 
%\begin{figure}[h!]
%\begin{center}
%\pngname{viewer_n5921_rascon_1}{3}
%\pngname{viewer_n5921_rascon_2}{3}
%\caption{\label{fig:viewer_rascon} The {\bf Viewer Display Panel}
%(left) and {\bf Data Display Options} panel (right) after overlaying
%a {\tt Contour Map} on a {\tt Raster Image} from the same image cube.  The
%image shown is for channel 11 of the NGC5921 cube, selected using
%the {\bf Animator} tape deck, and zoomed in using the tool bar icon.
%The tab for the contour plot is open in the {\bf Data Display Options} 
%panel.} 
%\hrulefill
%\end{center}
%\end{figure}

\begin{figure}[h!]
\begin{center}
\pngname{viewer_n5921_contour_moments_1}{4.3}
\pngname{viewer_n5921_contour_moments_2}{2.1}
\caption{\label{fig:viewer_rascon} The {\bf Viewer Display Panel}
(left) and {\bf Data Display Options} panel (right) after overlaying
a {\tt Contour Map} of velocity on a {\tt Raster Image} of intensity.  The
image shown is for the moments of the NGC5921 cube, zoomed in using the tool bar icon.
The tab for the contour plot is open in the {\bf Data Display Options} 
panel.} 
\hrulefill
\end{center}
\end{figure}

%%%%%%%%%%%%%%%%%%%%%%%%%%%%%%%%%%%%%%%%%%%%%%%%%%%%%%%%%%%%%%%%%
\subsection{Spectral Profile Plotting}
\label{section:display.image.specprof}
From the {\bf Main Toolbar} (see Fig.~\ref{fig:viewer_maintoolbar}),
the {\tt Spectral Profile} plotting tool can be selected.
This creates a new {\bf Spectral Profile} window containing an x-y plot
of the intensity versus spectral axis. The displayed spectrum is extracted in
region marked with the {\bf Point}, {\bf Rectangle}, {\bf Ellipse} or 
{\bf Polygon Region} tool in the {\bf Viewer Display Panel}. The spectrum
shown in the {\bf Spectral Profile} window (right panel in Figure 
\ref{fig:viewer_specprof}) is automatically updated if the corresponding
region is dragged across the image in the {\bf Viewer Display Panel}
(right panel in Fig.~\ref{fig:viewer_specprof}).

\begin{figure}[h!]
\begin{center}
\pngname{viewer_specprof_1}{3.3}
\pngname{viewer_specprof_2}{2.9}
\caption{\label{fig:viewer_specprof} The {\bf Spectral Profile} panel (right)
that appears when pressing the button {\bf Open the Spectrum Profiler} in the
{\bf Main Toolbar} and then use the tools to select a region in the image,
such as the rectangular region on the left panel. The profile changes to
track movements of the region if moved by dragging with
the mouse.} 
\hrulefill
\end{center}
\end{figure}
\subsubsection{Toolbar}
\label{section:display.image.specprof.toolbar}
The toolbar of the {\bf Spectral Profile} (Figure \ref{fig:viewer_spectoolbar})
contains action icons to (from left to right):
\begin{itemize}
\item export the current profile to a FITS or ASCII file;
\item print the main window;
\item save the panel as an image (PNG, JPG, PDF, ...);
\item set plot preferences;
\item 'move' the spectrum in the xy-plane (arrows);
\item zoom in and out (magnifiers).
\end{itemize}

\begin{figure}[h!]
\begin{center}
\pngname{viewer_spectoolbar}{4}
\caption{\label{fig:viewer_spectoolbar}The toolbar of the
{\bf Spectral Profile} contains a couple of action icons to save data
or manipulate the displayed xy-range.}
\hrulefill
\end{center}
\end{figure}
\subsubsection{Main window}
\label{section:display.image.specprof.mainwindow}
The main window shows the spectrum extracted from the image. The unit of the
spectral axis can be selected between frequency, wavelength and velocity
units (available only if a rest frequency is given). The user can choose
between the different combine types {\tt mean}, {\tt median}, {\tt sum}
and {\tt flux}\footnote{The combine type {\tt flux} is similar to sum, but
divided by the number of pixels in the synthesized beam. This option is only available
for adequate data.}
in a combo box. Besides the spectral values also the error bars can be plotted.
The error values are either computed by propagating the pixel errors provided
in the data (error type {\tt propagated}) or by computing the
root-mean squared error (error type {\tt rmse}).

\begin{figure}[h!]
\begin{center}
\pngname{viewer_specmouse_1}{2.1}
\pngname{viewer_specmouse_2}{2.1}
\pngname{viewer_specmouse_3}{2.1}
\caption{\label{fig:viewer_specmouse}With {\tt dragging the left mouse button}
over the main window the user can interactively zoom into the profile
(yellow box in left panel). Pressing the {\tt shift-key} while {\tt dragging the left mouse button}
marks a spectral range with a gray area (middle panel) and provides
start and end values for the tabs {\tt collapse/moments} and {\tt linefit}. With the
{\tt ctrl-key} pressed, a gray line marks the cursor position. Clicking the {\tt left mouse button}
displays the corresponding spectral channel in the {\bf Viewer Display Panel}.}
\hrulefill
\end{center}
\end{figure}
The main window is sensitive to the following combination input from keyboard
and mouse (Figure \ref{fig:viewer_specmouse}):
\begin{itemize}
\item {\bf zoom:} when {\tt pressing and dragging the left mouse button}, a
yellow box is drawn onto the panel. After releasing the mouse button,
the xy-range is zoomed to the values of the yellow box
(Fig.~\ref{fig:viewer_specmouse}, left panel);
\item {\bf spectral range selection:} when {\tt pressing and dragging left 
mouse button with shift-key}, a gray box marks a spectral range in the plot.
The start and end values are written into the {\tt from:} and {\tt to:} box
of the {\bf collapse/moments} and {\bf linefit} tab. 
This allows an interactive (Fig.~\ref{fig:viewer_specmouse}, middle panel)
selection of the spectral range used in the collapse/moments or linefit operation;
\item {\bf specral channel selection:} When pressing the {\tt ctr-key}, a gray
line is drawn at the current position of the mouse in the spectral profile
window (Fig.~\ref{fig:viewer_specmouse}, right panel). After
{\tt pressing the right mouse button}, the image in the {\bf Viewer Display Panel}
displays in the z-axis the spectral channel marked with the gray line.
\end{itemize}

\subsubsection{Image analysis}
\label{section:display.image.specprof.analysis}
The two tabs labeled {\tt collapse/moments} and {\tt linefit} in the lower
part of the {\bf Spectral Profile} panel offer simple image analysis tools to
the user (see Figs.~\ref{fig:viewer_specprof} and \ref{fig:viewer_specmouse}).

In {\tt collapse/moments} the user can collapse the image along the spectral
axis between the start and end values provided in the corresponding boxes.
Various collapse types ({\tt mean}, {\tt median} and {\tt sum}) are offered,
and the error is computed via error propagation from the pixel errors
{\tt propagated} or by computing the root-mean squared error ({\tt rmse}).
The collapsed image is displayed in the {\bf Viewer Display Panel}.

\begin{figure}[h!]
\begin{center}
\pngname{viewer_specproffit}{4}
\caption{\label{fig:viewer_specproffit}A Gaussian fit (blue line)to the spectral
profile (red line). The status line at the bottom of the panel contains the main
fit results, all details are printed to standard output.}
\hrulefill
\end{center}
\end{figure}
In {\tt linefit} the user can fit a profile (Gaussian plus polynomial)
to the spectrum profile (Figure \ref{fig:viewer_specproffit}). If
error bars are displayed in the spectral profiler, they are used in
the fitting as weights. The main results of the fit are shown in the
status bar of the {\bf Spectral Profile}.  A detailed summary of the
fit is sent to the standard output (either the casapy window or the
window where the viewer was started).


%%%%%%%%%%%%%%%%%%%%%%%%%%%%%%%%%%%%%%%%%%%%%%%%%%%%%%%%%%%%%%%%%
\subsection{Managing Regions and Annotations}
\label{section:display.image.rgnmgr}

CASA regions are following the CASA 'crtf' standard as described in
\S~\ref{chapter:regionformat}. CASA regions can be used in all
applications, including {\tt clean} and image analysis tasks
(\S\,\ref{chapter:analysis}). In addition, a leading 'ann' to each
region definition indicates that it is for visual overlay purposes
only. On a side note: apart from the regions mentioned here, CASA supports
image masks, i.e. images that contain only 0 and 1 (or 0 a non-0). In
some images, masks are stored in a True/False Boolean format. 

{\bf Alert:} Whereas the region format is supported by all the data
  processing tasks, the {\tt viewer} implementation is still limited
  to rectangles, ellipses, and some markers. We expect the full
  implementation to be available in a future CASA release. 


  Regions can be created with the buttons marked as 'R' in the mouse
  tool bar (\S~\ref{section:display.viewerGUI.displaypanel},
  \S~\ref{section:display.viewerGUI.displaypanel.region}). The viewer
  currently supports rectangles, ellipses, polygons, and the point
  (crosshair). As usual, a mouse button can be assigned to each button
  as indicated by the small black square in each button (marking the
  left, middle, or right mouse button
  \S~\ref{section:display.viewerGUI.displaypanel},
  \S~\ref{section:display.viewerGUI.displaypanel.region}). An example
  is shown in Fig.\,\ref{fig:viewer_regions}.


\begin{figure}[h!]
\begin{center}
\pngname{viewer-regions}{7}
\caption{\label{fig:viewer_regions} Selecting an image region with the
region tools. The region panel is shown to the right.}
\hrulefill
\end{center}
\end{figure}

{\bf Regions can be selected by SHIFT+click, de-selected by pressing
  SHIFT+click again, and removed by hitting the ESC button. }


Once regions are selected, they will feature little, skeletal squares
in the corners of their boundary boxes. This distinguishes them from a
zoom box (magnifier
glass \S~\ref{section:display.viewerGUI.displaypanel}), where the
corners are solid. 

Regions can be moved by dragging with the mouse button and if more
than one region is selected, all selected regions move together.

To load, unload, modify, and to display the value statistics of each
region, the Region Panel can be loaded via the
'View'$\rightarrow$'region' drop-down menu. As all other panels, the
region panel can be docked to different portions of the viewer and it
can also be detached. If it is dismissed (the cross in the upper right
corner), it can be retrieved by the {\tt 'View'} menu. 

The three basic windows of the region panel are shown in
Fig.\,\ref{fig:viewer_regionpanel}.

\begin{figure}[h!]
\begin{center}
\pngname{region-panel-all}{4}
\caption{\label{fig:viewer_regionpanel} The three vertical tabs of the
region panel: {\tt properties}, {\tt stats}, and {\tt file}.}
\hrulefill
\end{center}
\end{figure}

\subsubsection{Region Panel: properties}
\label{section:display.image.rgnmgr.props}

The bottom of this panel features a slider to switch between regions
in the image. Regions can be removed by hovering over and pressing
{\tt ESC}} or by pressing the buttons to the right side of the slider
where the first button deletes all regions and the far right button
deletes the region that is currently displayed in the panel.

The properties tab can be used to adjust the the size, center, and
coordinate frame of the region ({\tt 'coordinates'} selection at the
top). The inputs can be in different units, and those implemented in
the viewer are commonly used sub-sets of the options listed
in \S\,\ref{chapter:regionformat}. At the top of the panel, one can
specify the channel range over which the region shall be defined. The {\tt
  'selection'} check box is an alternative way to the {\tt
  SHIFT+click} to select a region. The {\tt 'annotation'} checkbox
will place the {\tt 'ann'} string in front of the region ascii output
-- annotation regions are not be used for processing in, e.g. data
analysis tasks.


The second horizontal tab {\tt 'line'} brings up a panel to change the
color, line width, and line style (solid, dotted, dashed) of the
selected region. 

The third panel, {\tt 'text'}, can be used to assign a string to the
region which can be controlled in position (the little dial at the
bottom, and the two right hand boxes), as well as text font, style,
and color. 


\subsubsection{Region Panel: stats}
\label{section:display.image.rgnmgr.stats}

The second vertical tab, 'stats', displays the statistics of each
region. When more than a single region is drawn, one can select them
one by one and the region panel updates image information and value
statistics for each region. The informational section contains the
frequency, velocity, stokes and brightness unit of the image. The beam
area is also calculated from the header information of the image. The
statistical properties of the pixels within in the region comprises
the number of pixels (Npts), the Sum, Mean, RMS, Minimum, and Maximum
of the pixel values, as well as the image flux integrated over the
region. All values are updated on the fly when the region is dragged
across the image.

Note that double-clicking the region will output the 
statistics to the terminal as explained above. This is an easy way to
copy and paste the statistical data to a program outside of CASA for
further use. 


\subsubsection{Region Panel: file}
\label{section:display.image.rgnmgr.file}

The third tab 'file' is used for loading and saving regions from and
to disk (select the appropriate the action at the top). To save to
ascii file, one can specify the file format, where the default is a
CASA region file (saved with a *.crtf suffix, see
\S\,\ref{chapter:regionformat}). It is also possible to load and save
DS9 regions, but remember that the DS9 format does not offer the full
flexibility and cannot capture stokes and spectral axes. DS9 regions
will only be usable as annotations in the viewer, they cannot be used
for data processing in other CASA tasks.

When saving regions, one can also specify whether to save only the
current region, all regions that were selected with
SHIFT+click, or all regions that are visible on the screen. 


%%%%%%%%%%%%%%%%%%%%%%%%%%%%%%%%%%%%%%%%%%%%%%%%%%%%%%%%%%%%%%%%%
\subsection{Adjusting Canvas Parameters/Multi-panel displays}
\label{section:display.viewerGUI.canvas}

The display area can also be manipulated with the following controls in
the {\bf Panel Options} (or 'Viewer Canvas Manager') window.
Use the wrench icon with a 'P' (or the 'Display Panel' menu) to show this
window.
\begin{itemize}
   \item Margins - specify the spacing for the left, right, top, and bottom margins
   \item Number of panels - specify the number of panels in x and y
         and the spacing between those panels.
   \item Background Color - white or black (more choices to come)
\end{itemize}

%%%%%%
\subsubsection{Setting up multi-panel displays}
\label{section:display.viewerGUI.canvas.multi}

Figure~\ref{fig:viewer_canvas} illustrates a multi-panel display along
with the Viewer Canvas Manager settings which created it. 

\begin{figure}[h!]
\begin{center}
%\gname{viewer_canvas}{3}
%\gname{viewer4}{3}
\pngname{viewer_multipanel_canvas}{2.5}
\pngname{viewer_multipanel_view}{3.5}
\caption{\label{fig:viewer_canvas} A multi-panel display
set up through the {\bf Viewer Canvas Manager}.} 
\hrulefill
\end{center}
\end{figure}

%%%%%%
\subsubsection{Background Color}
\label{section:display.viewerGUI.canvas.background}

The {\bf Background Color} selection can be used to change the
background color from its default of {\tt black}.  Currently,
the only other choice is {\tt white}, which is more appropriate
for printing or inclusion in documents.

%%%%%%%%%%%%%%%%%%%%%%%%%%%%%%%%%%%%%%%%%%%%%%%%%%%%%%%%%%%%%%%%%
\section{Viewing Measurement Sets}
\label{section:display.ms}

\begin{figure}[h!]
\begin{center}
\pngname{viewer_load_ms}{6}
\caption{\label{fig:viewer_load_ms} The {\bf Load Data - Viewer} panel
as it appears if you select an MS.  The only option available is
to load this as a {\tt Raster Image}.  In this example, clicking
on the {\tt Raster Image} button would bring up the displays shown
in Figure~\ref{fig:viewer_start_ms}.}
\hrulefill
\end{center}
\end{figure}

Visibility data can also be displayed and flagged directly from the
viewer. For Measurement Set files the only option for display is 'Raster'
(similar to AIPS task {\tt TVFLG}).  An example of MS display is
shown in Figure~\ref{fig:viewer_start_ms}; loading of an
MS is shown in Figure~\ref{fig:viewer_load_ms}.  

{\bf Warning:} {\em Only one MS should be registered at a time on a
Display Panel.} 
Only one MS can be shown in any case.  
You do not have to close other images/MSs, but you should at
least 'unregister' them from the Display Panel used for viewing the MS.
If you wish to see other images or MSs at the same time, create multiple
Display Panel windows.

% \begin{figure}[h]
% \gname{viewer_ms1}{3}
% \gname{viewer_ms2}{3}
% \caption{\label{fig:viewer_ms1} Display of visibility
%   data. The default axes are time vs. baseline.} 
% \hrulefill
% \end{figure}
 

%%%%%%%%%%%%%%%%%%%%%%%%%%%%%%%%%%%%%%%%%%%%%%%%%%%%%%%%%%%%%%%%%

\subsection{Data Display Options Panel for Measurement Sets}
\label{section:display.ms.adjust}

The {\bf Data Display Options} panel provides adjustments for MSs
similar to those for images, and also includes flagging options.
As with images, this window appears when you choose the {\tt Data:Adjust}
menu or use the wrench icon from the {\bf Main Toolbar}. It is also shown
by default when an MS is loaded. The right panel
of Figure~\ref{fig:viewer_start_ms} shows a {\tt Data Options} window. 
It has a tab for each open MS, containing a set of categories.  The
options within each category can be either 'rolled up' or expanded by
clicking the category label.

For a Measurement Set, the categories are:
\begin{itemize}
   \item {\bf Advanced}
   \item {\bf MS and Visibility Selection}
   \item {\bf Display Axes}
   \item {\bf Flagging Options}
   \item {\bf Basic Settings}
   \item {\bf Axis Drawing and Labels}
   \item {\bf Color Wedge}
\end{itemize}

% (The envelope, please....  And the winner is...)

%%%%%%
\subsubsection{MS Options --- Basic Settings}
\label{section:display.ms.adjust.basic}

The {\bf Basic Settings} roll-up is expanded by
default.  It contains entries similar to
those for a raster image (\S~\ref{section:display.image.raster.adjust.basic}). 
Together with the brightness/contrast and colormap adjustment icons
on the {\tt Mouse Toolbar} of the Display Panel, they are especially
important for adjusting the color display of your MS.

The available Basic options are:

\begin{itemize}

\item {\tt Data minimum/maximum}

This has the same usage as for raster images.  
Lowering the data maximum will help brighten
weaker data values.

\item {\tt Scaling power cycles}

This has exactly the same usage as for raster images (see
\S~\ref{section:display.image.raster.adjust.basic}).  Again, lowering
this value often helps make weaker data visible.  If you want to view
several fields with very different amplitudes simultaneously, this is
typically one of the best adjustments to make early, together with the
{\tt Colormap fiddling} mouse tool, which is on the middle mouse button
by default.

\item {\tt Colormap}

{\tt Greyscale} or {\tt Hot Metal} colormaps are generally good choices
for MS data.

\end{itemize}



%%%%%%
\subsubsection{MS Options--- MS and Visibility Selections}
\label{section:display.ms.adjust.select}

\begin{itemize}

\item {\tt Visibility Type}

\item {\tt Visibility Component}

\item {\tt Moving Average Size}

\end{itemize}

This roll-up provides choice boxes for Visibility Type
(Observed, Corrected, Model, Residual) and Component (Amplitude,
Phase, Real, or Imaginary).  

\begin{figure}[h!]
\begin{center}
\pngname{viewer_n4826_axes1}{6}
\caption{\label{fig:viewer_axes_1} The MS for NGC4826 BIMA
observations has been loaded into the viewer.  We see the
first of the {\tt spw} in the Display Panel, and have opened
up {\tt MS and Visibility Selections} in the
{\bf Data Display Options} panel.  The display panel raster is
not full of visibilities because {\tt spw 0} is continuum and
was only observed for the first few scans.  This is a case where
the different spectral windows have different numbers of channels
also.}
\hrulefill
\end{center}
\end{figure}

Changes to Visibility Type or Component (changing from Phase to
Amplitude, for example) require the data to be retrieved again
from the disk into memory, which can be a lengthy process.  When a
large MS is first selected for viewing, the user must
trigger this retrieval manually by pressing the {\bf Apply} button
(located below all the options), after selecting the data to be
viewed (see {\tt Field IDs} and {\tt Spectral Windows}, below).

{\bf Tip:} Changing visibility type between 'Observed' and 'Corrected' can
also be used to assure that data and flags are reloaded from disk.  You
should do this if you're using another flagging tool such as autoflag
simultaneously, so that the viewer sees the other tool's new edits
and doesn't overwrite them with obsolete flags.  The {\bf Apply} button 
alone won't reload unless something within the viewer itself requires
it; in the future, a button will be provided to reload flags from the disk
unconditionally.  

You can also choose to view the difference from a running mean or the
local RMS deviation of either Phase or Amplitude.  There is a slider
for choosing the nominal number of time slots in the 'local neighborhood'
for these displays.

(Note: {\bf Insufficient Data} is shown in the tracking area during
these displays when there is no other unflagged data in the
local neighborhood to compare to the point in question.  The
moving time windows will not extend across changes in either field ID
or scan number boundaries, so you may see this message if your scan
numbers change with every time stamp.  An option will be added later
to ignore scan boundaries).

\begin{itemize}

\item {\tt Field IDs}

\item {\tt Spectral Windows}

\end{itemize}

You can retrieve and edit a selected portion of the MS data
by entering the desired Spectral Window and Field ID numbers into
these boxes.  {\bf Important:} Especially with large MSs, often the
first thing you'll want to do is to select {\bf spectral windows}
which all have the {\bf same number of channels} and the
{\bf same polarization setup}.  It also makes sense to edit only
a few fields at a time.   Doing this will also
greatly reduce data retrieval times and memory requirements.

You can separate the ID numbers with spaces or commas; you do not need to
enter enclosing brackets.  Changes to either entry box will cause
the selected MS data to be reloaded from disk.

If you select, say, spectral windows 7, 8, 23, and 24, the animator, slice
position sliders, and axis labeling will show 
these as 0, 1, 2, and 3 (the 'slice positions' or 'pixel coordinates' of the
chosen spectral windows).  Looking at the position tracking display is the best
way to avoid confusion in such cases.  It will show something like: 
{\tt Sp Win 23 (s 2)} when you are viewing spectral window 23 (plane 2
of the selected spectral windows).

Changes to MS selections will not be allowed until you have saved
(or discarded) any previous edits you have made (see {\tt Flagging Options 
-- Save Edits}, below).  A warning is printed on the console (not the logger).

Initially, all fields and spectral windows are selected.  To revert to
this 'unselected' state, choose 'Original' under the wrench
icons next to the entry boxes.

See Figure~\ref{fig:viewer_axes_1} for an example showing the use
of the {\tt MS and Visibility Selections} controls when 
viewing an MS.

%%%%%%
\subsubsection{MS Options --- Display Axes}
\label{section:display.ms.adjust.axes}

This roll-up is very similar to that for images: it allows the user to
choose which axes (from Time, Baseline, Polarization, Channel, and
Spectral Window) are are on the display and the animator.  There are
also sliders here for choosing positions on the remaining axes.  (It's 
useful to note that the data {\it is} actually stored internally in
memory as an array with these five axes).

\begin{figure}[h!]
\begin{center}
\pngname{viewer_n4826_axes2}{6}
\caption{\label{fig:viewer_axes_2} 
The MS for NGC4826 from Figure~\ref{fig:viewer_axes_1}, now with the
{\tt Display Axes} open in the {\bf Data Display Options} panel.  By
default, {\tt channels} are on the {\bf Animation Axis} and thus in
the tapedeck, while {\tt spectral window} and {\tt polarization} are
on the {\tt Display Axes} sliders. } \hrulefill
\end{center}
\end{figure}

For MSs, changing the choice of axis on one control will automatically
swap axes, maintaining different axes on each control.  Changing axes
or slider/animator positions does not normally require pressing
{\bf Apply} --- the new slice is shown immediately.  
However, the display may be 
partially or completely grey in areas if the required data is not
currently in memory, either because no data has been loaded yet, or
because not all the selected data will fit into the allowed memory.
Press the {\bf Apply} button in this case to load the data
(see \S~\ref{section:display.ms.adjust.apply} and 
{\tt Max. Visibility Memory} at the end of 
\S~\ref{section:display.ms.adjust.adv}).

\begin{figure}[h!]
\begin{center}
\pngname{viewer_n4826_axes3}{6}
\caption{\label{fig:viewer_axes_3} The MS for NGC4826,
continuing from Figure~\ref{fig:viewer_axes_2}.  
We have now put {\tt spectral window} on the {\bf Animation Axis} 
and used the tapedeck to step to {\tt spw 2}, where we see the
data from the rest of the scans.  Now {\tt channels} is on a
{\tt Display Axes} slider, which has been dragged to show
{\tt Channel 33}.}
\hrulefill
\end{center}
\end{figure}

Within the {\tt Display Axes} rollup you may also select whether to order
the baseline axis by antenna1-antenna2 (the default) or by (unprojected)
baseline length.

See Figures~\ref{fig:viewer_axes_2}--\ref{fig:viewer_axes_3}
showing the use of the {\tt Display Axes} controls to change the axes on the
animation and sliders.

%%%%%%
\subsubsection{MS Options --- Flagging Options}
\label{section:display.ms.adjust.flagging}

These options allow you to edit (flag or unflag) MS data.
The Crosshair and Rectangle Region {\bf Mouse Tools}
(\S~\ref{section:display.viewerGUI.displaypanel.region}) are used on
the display to select the area to edit.  When using the Rectangle Region
tool, double-click inside the selected rectangle to confirm the edit.

The options below determine how edits will be applied.

\begin{itemize}

\item {\tt Show Flagged Regions...}

You have the option to display flagged regions in the background
color (as in {\tt TVFLG}) or to highlight them with color.
In the former case, flagged regions look just like regions of no
data.  With the (default) color option, flags are shown in shades of blue:
darker blue for flags already saved to disk, lighter blue for
new flags not yet saved; regions with no data will be shown in black.

\item {\tt Flag or Unflag}

This setting determines whether selected regions will be flagged or
unflagged.  This does {\it not} affect previous
edits; it only determines the effect which later edits
will have.  Both flagging and unflagging edits can be accumulated
and then saved in one pass through the MS.

\item {\tt Flag/Unflag All...}

These flagging extent checkboxes allow you to extend your edit over any
of the five data axes.  For example, to flag {\it all} the data in a given
time range, you would check all the axes {\it except} Time, and then
select the desired time range with the {\tt Rectangle Region} mouse tool.
Such edits will extend along the corresponding axes over the entire selected
MS (whether loaded into memory or not) and optionally over unselected 
portions of the MS as well ({\tt Use Entire MS}, below).  Use care in
selecting edit extents to assure that you're editing all
the data you wish to edit.

\item {\tt Flag/Unflag Entire Antenna?}

This control can be used to extend subsequent edits to all baselines
which include the desired antenna[s].  For example, if you set this item
to 'Yes' and then click the crosshair on a visibility point with
baseline 3-19, the edit would extend over baselines 0-3, 1-3, 2-3, 3-3,
3-4, ... 3-{\tt nAntennas-1}.  Note that the second antenna of the selection
(19) is irrelevant here -- you can click anywhere within the 'Antenna 3 block',
i.e., where the {\em first} antenna number is 3, to select all baselines
which include antenna 3.

This item controls the edit extent only along the baseline axis.  If you
wish to flag {\it all} the data for a given antenna, you must still check
the boxes to flag all Times, Channels, Polarizations and Spectral Windows.
There would be no point, however, in activating {\it both} this item and
the 'Flag All Baselines' checkbox.  You can flag an antenna in a limited
range of times, etc., by using the appropriate checkboxes and selecting
a rectangular region of visibilities with the mouse. 

{\bf Note:} You do not need to include the entire 'antenna block' in your
rectangle (and you may stray into the next antenna if you try). Anywhere
within the block will work.  To flag higher-numbered antennas, it often
helps to zoom in.

\item {\tt Undo Last Edit}

\item {\tt Undo All Edits}

The 'Undo' buttons do the expected thing: completely undo the effect of
the last edit (or all unsaved edits).  Please note,
however, that only unsaved edits can be undone here;
there is no ability to revert to the flagging state at the start of the
session once flags have been saved to disk (unless you have previously
saved a 'flag version'.  The flag version tool is not available through
the viewer directly).

\item {\tt Use Entire MS When Saving Edits?}

"Yes" means that saving the edits will flag/unflag over the entire MS,
{\it including} fields (and possibly spectral windows) which are not 
currently selected for viewing.  Specifically, data within time range(s)
you swept out with the mouse (even for unselected fields) will be edited.

In addition, if "Flag/Unflag All..." boxes were checked, such edits will
extend throughout the MS.  Note that only
unselected {\it times} (fields) can be edited {\it without} checking
extent boxes for the edits as well.  Unselected spectral windows, e.g.,
will {\it not} be edited unless the edit also has "Flag/Unflag All
Spectral Windows" checked.  

Warning: Beware of checking "All Spectral Windows" unless you have also 
checked "All Channels" or turned "Entire MS" off; channel edits appropriate 
to the selected spectral windows may not be appropriate to unselected
ones.  Set "Use Entire MS" to"No" if your edits need to apply only to the
portion of the MS you have selected for viewing.  {\it Edits can often be
saved significantly faster this way as well}.

Also note that checkboxes apply to individual edits, and must be checked
before making the edit with the mouse.  "Use Entire MS", on the other hand,
applies to all the edits saved at one time, and must be set as desired
before pressing "Save Edits".

\item {\tt Save Edits}

MS editing works like a text editor in that
you see all of your edits immediately, but nothing is committed to disk
until you press 'Save Edits'.  Feel free to experiment with all the other
controls; nothing but 'Save Edits' will alter your MS on disk. 
As mentioned previously, however, there is no way to undo your edits once
they are saved, except by manually entering the reverse edits (or restoring
a previously-saved 'flag version').

Also, {\it you must save} (or discard) {\it your edits before changing the 
MS selections}.  If edits are pending, the selection change will not be 
allowed, and a warning will appear on the console.  

If you close the MS in the viewer, {\it unsaved edits are simply discarded},
without prior warning.  It's important, therefore, to remember to save them
yourself.  You can distinguish unsaved flags (when using the 'Flags In Color'
option), because they are in a lighter shade of blue.

The program must make a pass through the MS on disk to save the edits.
This can take a little time; progress is shown in the console window.

\end{itemize}

%%%%%%
\subsubsection{MS Options---  Advanced}
\label{section:display.ms.adjust.adv}

These settings can help optimize your memory usage, especially for
large MSs.  A rule of thumb is that they can be increased until response
becomes sluggish, when they should be backed down again.

You can run the unix 'top' program and hit 'M' in it (to sort by memory
usage) in order to examine the effects of these settings.  Look at the
amount of RSS (main memory) and SWAP used by the X server and 'casaviewer'
processes.  If that sounds familiar and easy, then fiddling with these
settings is for you.  Otherwise, the default settings should provide
reasonable performance in most cases.

\begin{itemize}

\item {\tt Cache size}

The value of this option specifies the maximum
number of different views of the data to save so that they
can be redrawn quickly.  If you run an animation or scroll around
zoomed data, you will notice that the data displays noticeably faster
the second time through because of this feature.  Often, setting this
value to the number of animation frames is ideal  Note, however, that
on multi-panel displays, each panel counts as one cached image.

Large images naturally take more room than small ones.  The memory used
for these images will show up in the X server process.  If you need more
Visibility Memory (below) for a really large ms, it is usually better to
forgo caching a large number of views.

\item {\tt Max. Visibility Memory}

This option specifies how many megabytes of memory may be used to store
visibility data from the measurement set internally.  {\it Even if you do
not adjust this entry, it is useful to look at it to see how many megabytes
are required to store your entire (selected) MS in memory}.  If the slider
setting is above this, the whole selected MS will fit into the memory
buffer.  Otherwise, some data planes will be 'grayed out' (see 
{\tt Apply Button}, \S~\ref{section:display.ms.adjust.apply} below),
and the selected
data will have to be viewed one buffer at a time, which is somewhat less 
convenient. In most cases, this means you should {\bf select fewer fields
or spectral windows} -- see \S~\ref{section:display.ms.adjust.select}.
The 'casaviewer' process contains this buffer memory (it contains the entire
viewer, but the memory buffer can take most of the space).

\end{itemize}

%%%%%%
\subsubsection{MS Options --- Apply Button}
\label{section:display.ms.adjust.apply}

When viewing large MSs the display may be 
partially or completely grey in areas where the required data is not
currently in memory, either because no data has been loaded yet, or
because not all the selected data will fit into the allowed memory
(see {\tt Max. Visibility Memory} above).  When the
cursor is over such an area, the following message shows in the position
tracking area:
\small
\begin{verbatim}
   press 'Apply' on Adjust panel to load data
\end{verbatim}
\normalsize
Pressing the {\bf Apply} button (which lies below all the options) 
will reload the
memory buffer so that it includes the slice you are trying to view.

The message {\bf No Data} has a different meaning; in that
case, there simply {\it is} no data in the selected MS at the
indicated position.

For large measurement sets, loading visibility data into memory is the
most time-consuming step.  Progress feedback is provided in the
console window.  Again, careful selection of the data to be viewed can
greatly speed up retrieval.

%%%%%%%%%%%%%%%%%%%%%%%%%%%%%%%%%%%%%%%%%%%%%%%%%%%%%%%%%%%%%%%%%
\section{Printing from the Viewer}
\label{section:display.print}

You can use the {\tt Data:Print} menu or the {\bf Print} button to
bring up the {\bf Viewer Print Manager}.  From this panel, you can
print a hardcopy of what is in the Display Panel, or save it in a 
variety of formats.

\begin{figure}[h!]
\begin{center}
%\pngname{viewer_printing}{6}
\pngname{viewer_jupiter_print}{6}
\caption{\label{fig:viewer_print} Setting up to print to a file.
The background color has been set to {\tt white}, the line width
to {\tt 2}, and the print resolution to {\tt 600} dpi (for an
postscript plot).  To make the plot, use the {\bf Save} button on
the {\bf Viewer Print Manager} panel (positioned in the figure in
the upper right) and select a format with the drop-down, or 
use the {\bf Print} button to send directly to a printer.
} 
\hrulefill
\end{center}
\end{figure}

Figure~\ref{fig:viewer_print} shows an example of printing to a file.
The key to making acceptable hardcopies (particularly for printing
or inclusion in documents) is to set the background color and line
widths to appropriate values so the plot and labels show up in the
limited resolution of the hardcopy.

Use the {\bf Viewer Canvas Manager} 
(\S~\ref{section:display.viewerGUI.canvas}) to change the 
{\bf Background Color} from its default of {\tt black} to
{\tt white} if you are making plots for printing or inclusion in
a document.  You might also want to change the {\bf colormap}
accordingly.

Adjust the {\bf Line Width} of the {\bf Axis Label Properties}
options in the {\bf Data Display Options} panel so that the
labels will be visible when printed.  Increasing from the default
of {\tt 1.4} to a value around {\tt 2} seems to work well.

You can choose an output file name in the panel.  Be sure to
make it a new name, otherwise it will not overwrite a previous file
(and will not say anything about it).

If you will be printing to a postscript printer or to a PS or EPS file,
dial up the {\bf [PS] Resolution (dpi)} to its maximum of {\tt 600}.  
This will increase the size of the PS file somewhat, but will
make a much better plot.  Use {\tt gzip} to compress the PS file
if necessary.
Be sure to choose the desired Output Media and Orientation for
PS also.

{\bf BETA ALERT:} The postscript printing capabilities of the 
{\tt casaviewer} are currently limited due to some issues with the
way we use Qt and do axis labels.  Significant improvements have
been made in Patch 3.  This will be further upgraded in the
future, but for now you will need to follow the suggestions above
to get a useable plot.  Note that {\tt ghostview} may show a poorer
version of the PS than you will get when you print.

%%%%%%%%%%%%%%%%%%%%%%%%%%%%%%%%%%%%%%%%%%%%%%%%%%%%%%%%%%%%%%%%%
%%%%%%%%%%%%%%%%%%%%%%%%%%%%%%%%%%%%%%%%%%%%%%%%%%%%%%%%%%%%%%%%%
\section{Image Viewer ({\tt imview})}
\label{section:display.imview}

{\tt imview} is currently mostly a close of the {\tt viewer}. A
significant improvement, however, is that {\tt imview} adds
scriptability from the usual task interface to the {\tt viewer}. This
will enable the production of plots without invoking the GUI. To date,
the controls from the command interface are still limited but we will
add the full functionality gradually. 

{\tt imview} has the following inputs:

\small
\begin{verbatim}
#  imview :: View an image
raster              =         {}        #  (Optional)  Raster filename (string)
                                        #   or complete raster config
                                        #   dictionary. The allowed dictionary
                                        #   keys are file (string), scaling
                                        #   (numeric), range (2 element numeric
                                        #   vector), colormap (string), and
                                        #   colorwedge (bool).
contour             =         {}        #  (Optional)  Contour filename (string)
                                        #   or complete contour config
                                        #   dictionary. The allowed dictionary
                                        #   keys are file (string), levels
                                        #   (numeric vector), unit (float), and
                                        #   base (float).
zoom                =          1        #  (Optional)  zoom can specify
                                        #   intermental zoom (integer), zoom
                                        #   region read from a file (string) or
                                        #   dictionary specifying the zoom
                                        #   region. The dictionary can have two
                                        #   forms. It can be either a simple
                                        #   region specified with blc (2 element
                                        #   vector) and trc (2 element vector)
                                        #   [along with an optional coord key
                                        #   ("pixel" or "world"; pixel is the
                                        #   default) or a complete region
                                        #   rectangle e.g. loaded with
                                        #   "rg.fromfiletorecord( )". The
                                        #   dictionary can also contain a
                                        #   channel (integer) field which
                                        #   indicates which channel should be
                                        #   displayed.
axes                =         -1        #  (Optional)  this can either be a
                                        #   three element vector (string) where
                                        #   each element describes what should
                                        #   be found on each of the x, y, and z
                                        #   axes or a dictionary containing
                                        #   fields "x", "y" and "z" (string).
out                 =         ''        #  (Optional)  Output filename or
                                        #   complete output config dictionary.
                                        #   If a string is passed, the file
                                        #   extension is used to determine the
                                        #   output type (jpg, pdf, eps, ps, png,
                                        #   xbm, xpm, or ppm). If a dictionary
                                        #   is passed, it can contain the
                                        #   fields, file (string), scale
                                        #   (float), dpi (int), or orient
                                        #   (landscape or portrait). The scale
                                        #   field is used for the bitmap formats
                                        #   (i.e. not ps or pdf) and the dpi
                                        #   parameter is used for scalable
                                        #   formats (pdf or ps).
async               =      False        #  If true the taskname must be started
                                        #   using imview(...)



\end{verbatim}
\normalsize

The {\tt raster} and {\tt contour} parameters control how images are
loaded. They take python dictionaries as inputs. This may be an easier
solution than a complicated matrix of input parameters. The structure
looks like:

\small
\begin{verbatim}
imview(raster={'file': 'ngc5921.clean.image',
                       'range': [-0.01,0.03],
                       'colormap': 'Hot Metal 2',
                       'scaling': -1},
               contour={'file': 'ngc5921.clean.image'},
               axes={'x':'Declination'} ,
               zoom={'channel': 7, 'blc': [75,75], 'trc': [175,175],
                     'coord': 'pixel'},
               out='myout.png')
\end{verbatim}
\normalsize

Let's go through the structure: 

The parameter that can be found in the GUI is listed here as 'key':'value'
pair. E.g. 'file':'ngc5921.clean.image' sets the file name of the
raster image, 'range': [-0.01,0.03] sets the scaling. 

{\tt contour} can take more than a single file at a time to allow
multiple contour overlays. This can be achieved by specifying two
dictionaries in {} brackets

\small
\begin{verbatim}
 contour={'file': 'file1.image', 'levels': [1,2,3] },
         {'file': 'file2.image', 'levels': [0.006, 0.008, 0.010] }
\end{verbatim}
\normalsize


{\tt zoom} is a parameter to specify the part of the image that is
shown, {\tt axes} defines what axes are shown (most obviously 'x':'Right
Ascension', 'y':'Declination' are the default and most useful, but one
axis could also be frequency), and {\tt out} defines the filename of
the output in various graphic formats. 

Currently, the following parameters are supported:


\small
\begin{verbatim}
raster  -- (string) image file to open
           (dict)   file (string)     => image file to open
                    scaling (float)   => scaling power cycles
                    range (float*2)   => data range
                    colormap (string) => name of colormap
                    colorwedge (bool) => show color wedge?
contour -- (string) file to load as a contour
           (dict)   file (string)     => file to load
                    levels (float*N)  => relative levels
                    base (numeric)    => zero in relative levels
                    unit (numeric)    => one in the relative levels
zoom    -- (int)    integral zoom level
           (string) region file to load as the zoom region
           (dict)   blc (numeric*2)   => bottom left corner
                    trc (numeric*2)   => top right corner
                    coord (string)    => pixel or world
                    channel (int)     => chanel to display
           (dict)   <region record>   => record loaded
                                         e.g. rg.fromfiletorecord( )
axes    -- (string*3) demension to display on the x, y, and z axes
           (dict)     x               => dimension for x-axes
                      y               => dimension for y-axes
                      z               => dimension for z-axes
out     -- (string) file with a supported extension
                    [jpg, pdf, eps, ps, png, xbm, xpm, ppm]
            (dict)    file (string)   => filename
                      format (string) => valid ext (filename ext overrides)
                      scale (numeric) => scale for non-eps, non-ps output
                      dpi (numeric)   => dpi for eps or ps output
                      orient (string) => portrait or landscape
\end{verbatim}
\normalsize
Examples are also found in {\tt help imview}.

%%%%%%%%%%%%%%%%%%%%%%%%%%%%%%%%%%%%%%%%%%%%%%%%%%%%%%%%%%%%%%%%%
\section{Measurement Viewer ({\tt msview})}
\label{section:display.msview}

The Measurement Viewer {\tt msview} is mostly a clone of the {\tt
  viewer} at this stage. A difference is that {\tt msview} allows the
user to select data before it is loaded into the GUI and displayed. A
screenshot is shown in Fig.\,\ref{fig:msview-selection} and selection
parameters are {\tt field, spectral window, time range, uv range,
  antenna, corr, scan, array, ms selection expression} in the usual
CASA selection syntax (see Sect.\,\ref{section:io.selection}). 


\begin{figure}[h!]
\begin{center}
%\gname{casa_inpclean1}{6}
%\pngname{clean_inputs_1}{6}
\pngname{msview-selection}{6}
\caption{\label{fig:msview-selection} Data selection in {\tt msview}.}
\hrulefill
\end{center}
\end{figure}
 