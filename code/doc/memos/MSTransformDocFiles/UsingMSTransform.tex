

\section{The MSTransform framework}\label{Sec:Running}
% Add parts from Justo's document.
% I/O improvement by avoiding read/write to disk multiple times.
% order of running the transformations
% independence of each transformation.
% new VI???

\subsection{Splitting capabilities}
Task mstransform behaves like split in all cases that concern sub-table re-indexing
and selection. It does not support multiple channel selections, which are separated by
a semi-colon. If there is an spw selection, it will re-index the spws starting from 0.

\subsection{Partition and Multi-MS support}
Task mstransform can do everything that task partition does and more. It supports
channel selections and re-indexing of spw sub-tables. The sub-tables are consolidated
in the output MMS, which does not happen in task partition. %Explain POINTING,SYSCAL tables

The user has the option to create an output MMS in parallel (using simple_cluster) or
in sequential. Set the parameter {\it parallel} to True to use simple_cluster. Note that
for small MSs, we recommend to create MMSs in sequential, as the overhead of creating
a cluster and managing the engines can become significant.

The combination of some transformations and createmms=True is not possible in
some cases, depending on the choice of {\it separationaxis}. The following table summarizes this:

\begin{verbatim}

separationaxis  combinespws  nspw > 1  regridms  freqaverage  timespan='scan' hanning
--------------------------------------------------------------------------------------
spw                NO         NO         YES        YES           YES           YES
scan               YES        YES        YES        YES           YES*          YES
both               NO         NO         YES        YES           YES*          YES

* --> we can partition per scan or scan,spw but cannot let timebins span changes in scan
      when timeaverage=True. The task will reset timespan to None in these cases.

\end{verbatim}

% Add the forbidden combinations of axis and transformations

\subsection{Combination of spectral windows}
Task mstransform can combine spectral windows independently or
with a reference frame transformation. When {\it combinespws} is set to True, the task will
combine all the selected spectral windows into one. The index of the output spw
will be 0. When there are overlapping channels, they will be averaged to form one
output channel.

\subsection{Frequency averaging}
Similar to task split, mstransform can average the selected channels based on a
width parameter. The parameter {\it freqbin} can be either an integer or a list of
integers that will apply to each spw in the selection.

A parameter called {\it useweights} controls the type of weights to use in the
averaging. The options are flags that will consider the FLAG column and spectral
that will consider the WEIGHT_SPECTRUM column.

% use weights parameter

\subsection{Reference frame transformation}
The parameter {\it regridms} in mstransform can do many things. One can change the
reference frame based on the available sub-parameters and hence separate the spectral
windows into a regular grid of a given number of spws. This is a new feature in CASA
4.1 and can be applied in many use-cases. %(Expand and list some use-cases)

The regridms transformation differs from task cvel in a few cases. Task cvel has a
parameter {\it passall} to copy or not the non-selected spws into the output MS. In 
mstransform we only consider the {\it passall=False}, meaning that we only consider the
selected spws of the MS. Task cvel always combines spw when changing the reference
frame, while mstransform can do both cases. The combination of spectral windows in mstransform
is controlled by the independent parameter {\it combinespws}.

\subsection{Hanning smoothing}
Set the parameter {\it hanning} to True to Hanning smooth the MS. Contrary to the
hanningsmooth task, mstransform creates a new output MS and writes the smoothed transformation
to the DATA column of the output MS, not to the CORRECTED_DATA column.

Another difference with respect to the hanningsmooth task is that the transformation will be 
applied to all the data columns requested by the user in the parameter {\it datacolumn}. If the 
requested column does not exist, it will exit with an error. 

\subsection{Time averaging}
% Add stuff in here
Similar to split, this task can average the MS in time, based on a set of parameters.

timespan
minbaselines
quantize_c


\section{Examples}\label{Sec:Examples}
How to run the mstransform task for several common use-cases.

\begin{enumerate}
\item Split three spectral windows of a field from an MS.
\begin{verbatim}
mstransform(vis='inp.ms', outputvis='out.ms', datacolumn='data', spw='1~3', field='JUPITER')
\end{verbatim}
\item Combine four spectral windows into one.
\begin{verbatim}
mstransform(vis='inp.ms', outputvis='out.ms', combinespws=True, spw='0~3')
\end{verbatim}
\item Apply Hanning smoothing in MS with 24 spws. Do not combine spws.
\begin{verbatim}
mstransform(vis='inp.ms', outputvis='out.ms', hanning=True, datacolumn='data')
\end{verbatim}
\item Create a multi-MS in paralell with spw separation and channel selection.
\begin{verbatim}
mstransform(vis='inp.ms', outputvis='out.mms', spw='0~4,5:1~10', createmms=True,
            separationaxis='spw', parallel=True)
\end{verbatim}
\item Average channels in CORRECTED column using a bin of 3 channels in XX.
\begin{verbatim}
mstransform(vis='inp.ms', outputvis='out.ms', spw='0:5~16', correlation='XX', 
            freqaverage=True, freqbin=3)
\end{verbatim}
\item Combine spws and regrid MS to new channel structure. Average width is 2 channels.
\begin{verbatim}
mstransform(vis='inp.ms', outputvis='out.ms', datacolumn='DATA', field='11',
            combinespws=True, regridms=True, nchan=1, width=2)
\end{verbatim}
\item Regrid MS to new channel structure, change reference frame to BARY and set
new phasecenter to that of field 1. Use mode frequency for the parameters.
\begin{verbatim}
mstransform(vis='inp.ms', outputvis='out.ms', datacolumn='DATA', spw='0', field='1', 
            regridms=True, nchan=2, mode='frequency', width='3MHz', start='115GHz', 
            outframe='BARY', phasecenter=1)
\end{verbatim}



\end{enumerate}

\section{Frequently Asked Questions}\label{Sec:FAQ}


\section{Notes during development}

%\htmladdnormallink{These slides}{http://www.aoc.nrao.edu/~rurvashi/DataFiles/Talk_FlaggingCASA3.4.pdf}
Summarize the CASA MSTransform infrastructure and user-options.

\htmladdnormallink{Task Documentation}{http://casa.nrao.edu/docs/TaskRef/TaskRef.html}
\htmladdnormallink{Tool Documentation} {http://casa.nrao.edu/docs/CasaRef/CasaRef.html}

These notes should become part of the real documentation later. For the moment they are
only notes to remind me of details of the implementation.

\begin{enumerate}
\item phasecenter: (int) or (str). As int, it gives the FIELD ID for the mosaic center. If a string,
it gives the center coordinate, e.g. 'J2000 12h56m43.88s +21d41m00.1s'. Default is '', which means, the 
first selected field.
       
\item 
\end{enumerate}

\subsection{Issues that were fixed in other places during the development of MSTransform}
\begin{enumerate}
\item Partition task: Fixed the spw selection list order that was supposed to be sorted but it was not.
Committed to r23001.
\item Listpartition task: optimized the code. Moved the creation of the dictionary to partitionhelper. It
now can be used by other scripts/tasks/tests.
\end{enumerate}

