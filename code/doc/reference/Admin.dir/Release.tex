\chapter{Preparing the Release}
\section{General timeline}
There are several steps involved going from active development to generating a release ready ISO
image.  In general it takes a about of 5 weeks, although if things go well it may be possible to 
squeeze the whole process into three weeks.
\begin{description}
\item[-5 weeks] Slushy Freeze
\item[-3 weeks] Hard Freeze
\item[-2 weeks 2 days] Uprev
\item[-2 weeks] Generate Release Candidate
\begin{itemize}
\item Make release build
\item Test release build
\item Patch
\item Make ISO image
\item Test ISO image
\end{itemize}
\item[0 ] Release to Duplicators
\end{description}
\section{Slushy Freeze}
During the slushy freeze, active development is suppose to cease and only changes that address
defects are suppose to be checked in to the system. During the slushy freeze state, we actively
check for stable builds everyday.
\section{Hard Freeze}
Ideally the "Hard Freeze" occurs with a stable.  At this point the current development master 
will become the release master once the "upreve" is completed.
\section{Doing the uprev}
The uprev is where we actually sync the current release master with the current development master.
There are several steps involved in doing the uprev and several days should be set aside in case
something goes wrong. If everything runs smoothly the uprev typically takes about two days.
\begin{itemize}
\item Check the disk space.  The worst thing that can happen is to run out of disk space during
the uprev.  Make sure a minimum of 2GB (peferrably 3GB) is available.
\item Send a early morning/late night [to give Oz a chance to respond] notice
   to aips2-workers telling them that the uprev and check-in freeze will happen
      later today/tomorrow. 

\item Notify aips2-workers so that check-ins will be disabled in 30 minutes to
         allow any in progress check-ins to be completed.

\item (adm) Freeze check-ins on the development master repository by running:
\begin{verbatim}
 /export/aips++/scripts/do_disable develop
\end{verbatim}

\item (adm) Stop the regular exhales in the (aips2adm) crontab; if the misery
	drags on, the regular exhale can gum up the works

\item (adm) Do an initial exhale to ensure that everything is up to date:
\begin{verbatim}
 ksh
 . $HOME/.profile
 exhale 2>&1 | mailx -s "AIPS++ (initial) uprev exhale" aips2-inhale
\end{verbatim}

\item (mgr) Update the data repository in release installation, e.g.
\begin{verbatim}
  cvsup /usr/local/aips++/data/supfile
\end{verbatim}

\item (mgr) Run cumulative inhale (after exhale completes) on local installation
\item (user) run "stable" tests

\item (adm) Back up release master (/export/aips++/repository/release), if desired.

\item (adm) Delete the release master and copy development to release:
\begin{verbatim}
 cd /export/aips++/repository
 rm -rf release
 mkdir release
 cd release
 (cd ../develop; tar cf - .) | tar xf -
\end{verbatim}

\item (adm) In /export/aips++/repository/release, edit:
\begin{verbatim}										
 .cshrc  .login  .profile  aipsinit.csh
 aipsinit.es  aipsinit.rc  aipsinit.sh
 replacing each occurrence of /develop with /release remove editor backups
\end{verbatim}
												
\item (adm) Update ftp link, VERSION, and base release tar file (where <REL> is the (last) base
release upon which the release master is based):
\begin{verbatim}
 cd /export/aips++/repository/release
 rm pub
 ln -s /export/aips++/pub/versions/release pub
 rm -f pub/master/VERSION
 cp master/VERSION pub/master
 cp ../develop/pub/master/master-<REL>.000.tar.gz pub/master
 /opt/local/gnu/bin/touch --file ../develop/pub/master/master-<REL>.000.tar.gz pub/master/master-<REL>.000.tar.gz
\end{verbatim}
\item (adm) Run exhale on the release master:
\begin{verbatim}
 ksh
 . /export/aips++/repository/release/.profile
 exhale 2>&1 | tee $HOME/master/etc/rexhale.log | mailx -s "AIPS++ release exhale" aips2-inhale &
\end{verbatim}

\item(adm) Do uprev (note, backup of VERSION is kept in VERSION.<REL>):
\begin{verbatim}										
 /export/aips++/scripts/do_uprev
\end{verbatim}
\item (mgr) Run test inhales for both the release and develop masters to ensure
that they work properly.
\item (adm) Enable regular exhales in the (aips2adm) crontab
\item (adm) Unfreeze check-ins on the development master repository by running:
\begin{verbatim}
  /export/aips++/scripts/do_enable develop
\end{verbatim}
\item (adm) Unlock any locked files in the release RCS tree
\begin{verbatim}
/export/aips++/scripts/do_unlock.sh
\end{verbatim}
\end{itemize}

\subsection{Update the Docs}

\begin{enumerate}
\item Update the README and readme.html to current versions
\item Update the release notes
\item Run a cross-referencing build of latex documents.
\end{enumerate}
\section{Testing the build}
Testing the build is broken down into two phases,
\begin{enumerate}
\item Test the release build
\begin{itemize}
\item As aips2mgr;  run testsuite
\item As a user; run assay().trytests()
\item As a user; exercise the tool manager
\item As aips2mgr; run linkscan
\end{itemize}
\item Test an installed ISO image
\begin{itemize}
\item Excersize the install procedures for linux and solaris
\item run assay().trytests
\item excersize the tool manager
\item take a random walk through the documentation
\end{itemize}
\end{enumerate}
\section{Patching the release candidate}
Once the release master has been updated, patches for defect resolution and documentation will be
sent.  These patches are suppose to be sent a binary shar files (shar -B dir/thefile > myfile.shar).
In general only defects fixes are allowed for .cc, .h and .g files. Documentation is always welcomed.
Steps for patching always done as a user:
\begin{itemize}
\item Make a seperate code tree
\item Save the shar files into a directory, i.e. /home/tarzan5/wyoung/patches/jun21.
\item Edit the files to remove non-shar file text
\item Unpack the shar files into a clean code tree,
\begin{verbatim}
cd /home/tarzan5/wyoung/patches
./unpack.sh jun21 2&>1 | tee patch.log
\end{verbatim}
Check the log to make sure everything unpacked cleanly.  If there are problems notify the sender.
\item Patch the release master
\begin{itemize}
\item (adm) Enable checkins in the release master; /export/aips++/scripts/do\_enable release
\item (user) Apply the patches
\begin{verbatim}
cd /home/tarzan5/wyoung/aips++
../patches/do_patch.sh
\end{verbatim}
\item (adm) Disable checkins in the release master; /export/aips++/scripts/do\_disable release
\item (adm) Exhale the release
\item (mgr) inhale -R release -c
\item (mgr) check the build log for problems
\item (user) Notify aips2-workers of accepted patches
\begin{verbatim}
 cd /home/tarzan5/wyoung/patches
 awk -f okpatch.awk jun21/* > patch.mailing
 mailx -s "Accepted patches for release" aips2-workers@aoc.nrao.edu < patch.mailing
\end{verbatim}
Note: the okpatch.awk script will need to be modified with each new version.
\end{itemize}
\end{itemize}
\section{Preparing the binary}
\subsection{The penultimate build}
Once the release build is in good shape we need to produce the files necessary
for the binary distribution.  Binrel must be run on each architecture.  One should
avoid binrel on nfs mounted disks as it takes considerably longer.
\begin{description}
\item[linux] 
\begin{enumerate}
\item  . /usr/local/aips++/aipsinit.sh
\item binrel -tgt /home/cluster2/aips++/binary\_release
\item cp linux.tar.gz /home/tarzan5/aips++/binary\_release
\end{enumerate}
\item[solaris] 
\begin{enumerate}
\item  . /usr/local/aips++/aipsinit.sh
\item binrel -tgt /home/tarzan5/aips++/binary\_release
\end{enumerate}
\end{description}
\subsection{Creating the ISO and CD images}
Typically we burn on the aips2 administrator's computer.  This involves copying
the source contents from the binrel/YY.XXX directory to the CD distribution
directory. The current content for v1.7 are listed below.  Note the INSTALL,
README, and VERSION files need to be adjusted for each release.
\begin{verbatim}
[wyoung@crestone cds]$ ls -1 cd1/*
cd1/data.tar.gz
cd1/docs.tar.gz
cd1/INSTALL
cd1/lib.tar.gz
cd1/linux.tar.gz
cd1/README
cd1/startup.tar.gz
cd1/sun4sol.tar.gz
cd1/VERSION

cd1/source:
admin.tar.gz
aips.tar.gz
atnf.tar.gz
bima.tar.gz
contrib.tar.gz
dish.tar.gz
display.tar.gz
doc.tar.gz
hia.tar.gz
install.tar.gz
jive.tar.gz
nfra.tar.gz
npoi.tar.gz
nral.tar.gz
nrao.tar.gz
synthesis.tar.gz
tifr.tar.gz
trialdisplay.tar.gz
trial.tar.gz
vlbi.tar.gz

\end{verbatim}
Once the release has been assembled in the proper directory, xcdroast is used
to create both the ISO image and burn the CD.
Note: the ISO/CD image should always be labeled aips2install.  Xcdroast usually
creates a file track-01.img as the name for the ISO image, rename it to something
more sensible, i.e. aips2vXY.iso.

The ISO image maybe tested using the loopback file system

\begin{verbatim}
mount aips2vXY.iso /mnt/iso -o loop=/dev/loop3,blocksize=1024
\end{verbatim}

Now test install and verify both the ISO and CD images.
