\chapter{Exceptions \label{Coding.Exceptions}}
\index{exceptions}
\section{Introduction}

The main intent of this paper is to provide a user level introduction to
the {\tt C++} exception mechanism with a basic introduction to the
way it is used in {\tt aips++}. This should allow developers to use the mechanism.

{\tt C++} provides a standard method to handle exceptional conditions. When
such conditions arise, control is transferred to a code block which
can handle the exception, if one exists. If no handler exists which is capable
of handling the exception, the program is aborted. This package handles 
cleaning up the call stack which may have been created between when the handler
was {\em installed} and when the exception was thrown. In addition, any
objects which were created on the stack during this period will be deleted.

This provides a mechanism for transferring control from the point where an
exceptional condition occurs to the code to handle the exception without
memory leaks. Some of the advantages of this approach are:
\begin{itemize}
\item
The code to display the error message can be moved out to the appropriate
level. Otherwise, the code to display an error message might have to be
deep within an application in a place which could otherwise be independent
of the graphical user interface. Without exceptions error messages must be 
carefully propagated out via return values until the appropriate level for 
their display is reached.
\item
The exception mechanism allows for an incremental graceful exit. The same 
exception can be thrown a number of times so that each layer of the program
can have a chance to respond to the exception as appropriate. This can be
important for maintaining the integrity of portions of the program, e.g.
the user interface or database portions.
\item
Using exceptions a program can continue after an error even though
the error may have been fatal at the point where it occurs. In a more 
traditional approach, this would require careful checks of return codes.
\end{itemize}
\noindent
So while everything which can be accomplished with exceptions can be 
accomplished with returned error codes or other traditional approaches, the
use of exceptions is cleaner.

\section{Interface}

This section describes the exception mechanism from the 
perspective of a ``user'' of the system. It discusses the {\tt C++}
interface, the
mechanisms for automatic deletion of dynamically created objects, and the
steps necessary to create new exception types whose objects can be 
{\em thrown}.

\subsection{C++ Interface}\index{exceptions - C++ interface}

The {\tt C++} exceptions are based on the idea of {\tt try} and 
{\tt catch} blocks. The {\tt try} block contains the code
which may throw an exception, and the {\tt catch} block, which
must follow immediately after the {\tt try} block,
contains the handlers for the various exceptions which may be thrown in the
{\tt try} block. So in the simple case, one might have:
\begin{verbatim}
try {
  if (how_many == 1) throw(MajorError(1));
  if (how_many == 2) throw(MinorError(2));
  if (how_many == 3) throw(TypedError(3));
} catch (TypedError xx) {
  cout << "Caught TypedError" << endl;
} catch (MinorError xx) {
  cout << "Caught MinorError" << endl;
} catch (MajorError xx) {
  cout << "Caught MajorError" << endl;
}
\end{verbatim}
\noindent
In this example, depending on the variable 
{\tt how\_many} a different exception will be thrown, via {\tt throw()}.
The exception will then be caught by the first {\tt catch} block that
matches the exception. Thus, if {\tt MinorError} was derived from 
{\tt TypedError}, then the {\tt TypedError} block would be executed before 
the {\tt MinorError} block for a thrown {\tt MinorError}. 

It is also possible to have incremental recovery. A {\em caught} exception
can be {\em rethrown} with {\tt throw}. The following example 
illustrates this usage:
\begin{verbatim}
try {
  if (how_many == 1) throw(MajorError(1));
  if (how_many == 2) throw(MinorError(2));
  if (how_many == 3) throw(TypedError(3));
} catch (MinorError xx) {
  cout << "Caught MinorError" << endl;
} catch (MajorError xx) {
  cout << "Caught MajorError" << endl;
  throw;
} catch (TypedError xx) {
  cout << "Caught TypedError" << endl;
}
\end{verbatim}
\noindent
In this example, the {\tt MajorError} catch block handles a {\tt MajorError}
when it occurs and then {\em rethrows} the exception so that it could be
handled by other catch blocks. Thus, if {\tt MajorError} is derived from
{\tt TypedError}, then the {\tt TypedError} catch block would have the
opportunity to handle its portion of the error after the {\tt MajorError}
catch block was finished. 

These {\tt try}/{\tt catch} blocks can be nested to the level necessary as
long as the introductory {\tt try} is balanced with a closing brace.
The following example contains a nested try block:
\begin{verbatim}
try {
  if (how_many == 0) throw(ExcpError(1));
  try {
    if (how_many == 1) throw(ExcpError(1));
    if (how_many == 2) throw(ExcpError(2));
    if (how_many == 3) throw(ExcpError(3));
  } catch (ExcpError xx) {
    throw(ExcpError(9));
  }
} catch (ExcpError xx) {
  cout << "Caught ExcpError" << endl;    
}
\end{verbatim}


\subsection{Deleting Objects}

Another important portion of the exception handling mechanism is that
it deletes the objects created between the point where the
{\tt try}/{\tt catch} blocks are entered and the point where the exception
is thrown. These object are automatically cleaned up when an exception
occurs. The destructors are called for automatically created objects, 
but not for dynamically created objects which are not deleted by
other objects.

The programmer should use the templated {\tt PtrHolder} class to hold
pointers to dynamically created objects. Because the PrtHolder object
is automatically created, its destructor is called in case of an
exception. That destructor deletes the pointer it holds, thus deletes
the dynamically created object.
\begin{verbatim}
try {
  // True below means it's an array, False (the default) would mean
  // a singleton object.
  PtrHolder<Int> iholder(new Int[10000], True);
  some_function_that_throws_exception();   // pointer is deleted
} catch (...) {
  cout << "Int array is nicely deleted" << endl;
}
\end{verbatim}
Note that {\tt ...} is standard {\tt C++} syntax and means any exception.

\subsection{Creating Exception Types}

One of the advantages of the {\tt C++} exception mechanism is that one can
design a class hierarchy of exceptions. This results in an a great amount 
of flexibility. The reason for this is that one can {\em catch} whole groups 
of exceptions with one catch block, and then handle them as appropriate. One 
can also catch an exception with a specific handler, and the {\em rethrow} 
the exception so that it can be caught with a more general handler.

The root of the exception hierarchy is the class {\tt AipsError}. This class
must be the base class for all new exception hierarchies. All of the exceptions
which are part of the {\tt aips++} kernel, {\tt IndexError}, {\tt AllocError},
etc., are derived from the {\tt AipsError}. Thus, if one wished, all 
exceptions resulting from the {\tt aips++} kernel could be caught by one catch
block which catches {\tt AipsError}s. Generally, it is a
good practice to derive all exceptions for a given library from a library
specific exception class, e.g. {\tt TableError}, which is in turn derived 
from {\tt AipsError}.

{\tt AipsError} offers the const member function {\tt getMesg()}
which returns the message stored with the exception. It can be used as:
\begin{verbatim}
try {
  ...
} catch (AipsError& x) {
  cerr << "Unexpected exception: " << x.getMesg() << endl;
}
\end{verbatim}
