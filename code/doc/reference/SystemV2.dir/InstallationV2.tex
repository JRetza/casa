\chapter{Installation}
\label{Installation}
\index{system!installation}
\index{installation|see{system, installation}}

This chapter \footnote{Last change:
$ $Id$ $}
describes how to install \aipspp.

\section{Installation from Packages}
\index{installation|see{system, installation}}
\casa\ is available via binary down loads for Mac OSX and Red Hat
Enterprise compatible machines.
For the latest information about installation and supported packages
please visit \htmladdnormallink{https://wikio.nrao.edu/bin/view/Software/ObtainingCASA}{https://wikio.nrao.edu/bin/view/Software/ObtainingCASA}. Whether you are an end-user or
developer, we recommend using the a variation of our load-casapy scripts,
which can be gotten from \htmladdnormallink{need our ftp site}{ftp://ftp.nrao.edu/casa/}

\subsection{The Astronomer User}
\label{End-user installation}
\label{Astronomer installation}
\index{system!installation!end-user}
\index{system!installation!astronomer}
\index{system!installation!production-line}

Instructions for downloading load-casapy, run load-casapy.

\subsection{The Astronomer Developer}
\label{developers-release}
\index{developers release}
\index{system!installation!developer}

Instructions for downloading load-casapy, run load-casapy to install headers.

% ----------------------------------------------------------------------------

\section{Source \casa installations}
\label{Source installation}
\index{system!installation!consortium}
\index{system!installation!code-development}
\index{code!configuration}
\index{code!management!configuration}

Source installations differ from end-user installations in having a local
copy the executables built from source.
The local code tree is updated regularly by the \aipspp\ code
distribution system via a procedure called \exeref{inhale.svn}.  Source
installations also have a mechanism for checking sources out of, and in to,
the master \svn repositories.

\subsection*{Step 1. Create \casa accounts}

If more than one person or machine is going to use a \casa\ installation, we recommend that
an aips2mgr account be created to be the focus of
\casa builds.

\noindent
The home directory for the account should be set to the root
directory of the \aipspp\ tree.  This can be anything but \file{/home/casa} is
preferred and is assumed in the following examples.

The following groups should be created with the matching membership
(in \file{/etc/group}):

\begin{verbatim}
   aips2mgr   
   aips2prg  aips2mgr
   aips2usr  aips2mgr
\end{verbatim}

\noindent
You should also add your account name and the names of any other local
\aipspp\ managers to the \acct{aips2mgr} group membership list.  Do not add
everyone to the \acct{aips2mgr} group, it grants permission to directly
manipulate the source code tree.  You should also add the names of all local
\aipspp\ programmers to the \acct{aips2prg} group.  This will allow them to
check out and modify the \aipspp\ sources.  The \acct{aips2mgr} account and the
\acct{aips2mgr} \acct{aips2prg} groups will be used during the installation.

Now create the \aipspp\ root directory:

\begin{verbatim}
   yourhost% mkdir /home/casa
   yourhost% chown aips2mgr /home/casa
   yourhost% chgrp aips2prg /home/casa
   yourhost% chmod ug=rwx,o=rx,g+s /home/casa
\end{verbatim}

\noindent
In practice \file{/home/casa} will often be a self-contained filesystem, usually
on a separate disk.  Allow 1\,Gbyte of disk space for the \aipspp\ system; any
short-term surplus may be used for programmer workspaces.

This is as much as needs doing by the system administrator at this stage.  The
remainder of the initial part of the installation can be done by
\acct{aips2mgr}.

\subsection*{Step 2. Fetch the Source Code}
\label{fetch source code}
\index{system!installation!svncheckout}

First fetch the source code via SVN:

\begin{verbatim}
   yourhost% cd /home/casa
   yourhost% svn co https:svn.cv.nrao.edu:/svn/casa/casa.v2/branches/active/code 
\end{verbatim}

\noindent
This will fetch the main development branch of \casa. After the checkout completes you will
need to configure the source installation

\begin{verbatim}
   yourhost% cd code
   yourhost% chmod 544 install/configure
   yourhost% install/configure
\end{verbatim}

\noindent
You will be asked a series of questions, most of which have sensible defaults,
aimed at constructing the \aipspp\ \filref{aipshosts} file.  You then have to
edit your site-specific \filref{aipsrc} and \filref{makedefs} files.  Template
versions of these files are supplied by \exeref{configure} and you should read
the instructions carefully.  After making your site-specific definitions
\aipsexe{configure} will run some tests to check whether your \file{makedefs}
definitions look sensible.  Your \file{install} directory will then be made.
This consists of a few \textsc{c} compilations and installation of some shell
scripts, the most important of which is \exeref{inhale.svn} itself.  Ignore any
error message from \exeref{gmake} concerning the non-existence of various
subdirectories of \file{/home/casa/code}.

\subsection*{Step 3. Run sneeze}

At this point your \aipspp\ installation has been bootstrapped to a state
where \exeref{sneeze} can be run.

You should also have a \cplusplus\ compiler, and a \TeX\ installation which
includes \LaTeX, \unixexe{dvips}, \textsc{MetaFont}, \textsc{htlatex} and \textsc{pdflatex}.
Unset the \code{DOCSYS} variable in \file{makedefs} if you
don't have \TeX, it will prevent compilation of the \aipspp\ documentation.
The documentation may be downloaded using the \aipsexe{rsync} command.
(Note: On macs check the htlatex driver as \\dirchar may need to be replaced by -f/).

Users of SysV based systems such as Solaris should be warned that
\exeref{inhale.svn} requires the BSD version of \unixexe{sum} for computing
checksums.  You must ensure that the BSD version will be found ahead of the
SysV version in \acct{aips2mgr}'s \code{PATH}.  The \gnu\ version of
\unixexe{sum} (in the \gnu\ ``fileutils'' kit) provides both algorithms and
uses BSD by default.  Less salubrious possibilities are to put \file{/usr/ucb}
(Solaris) or \file{/usr/bsd} (IRIX) ahead of \file{/usr/bin} in
\acct{aips2mgr}'s \code{PATH}, or to create a symlink to the BSD version of
\unixexe{sum} in the \aipspp\ \file{bin} area.  

First invoke \exeref{casainit} to add the \aipspp\ \file{bin} directory to
your \code{PATH}.  If your interactive shell is a C-like shell (\unixexe{csh},
\unixexe{tcsh}) you would use

\begin{verbatim}
   yourhost% source /home/casa/casainit.csh
\end{verbatim}

\noindent
whereas for Bourne-like shells (\unixexe{sh}, \unixexe{bash}, \unixexe{ksh})
you would use

\begin{verbatim}
   yourhost% . /home/casa/casainit.sh
\end{verbatim}

\noindent
If you use some other shell you'll have to revert to one of the above for the
remainder of the installation.  Now invoke \exeref{sneeze}

\begin{verbatim}
   yourhost% sneeze -l -m cumulative&
\end{verbatim}

\noindent
This will build and install
the latest version of the sources which are under active development.  If you
made any mistakes in your \file{aipsrc} or \file{makedefs} definitions some of
these may become apparent during the installations.  After fixing them you can
recover via

\begin{verbatim}
   yourhost% gmake -C /home/casa/code allsys
\end{verbatim}

\noindent
The \code{allsys} target will compile all \aipspp\ sources, including
documentation (assuming of course that you have the compilers).  If you just
wanted to compile the documentation alone you could use

\begin{verbatim}
   yourhost% gmake -C /home/casa/code docsys
\end{verbatim}

If everything has gone properly you should now have an up-to-date \aipspp\ 
installation.  However, in order to keep it up-to-date you must define a
\unixexe{cron} job to run \exeref{inhale.svn} on a regular basis.  The normal
procedure is to do a cumulative update every Saturday evening.  However, you
may wish to maintain a (possibly separate) system which is updated on a daily
basis. 

The exact timing depends on your timezone with respect to the master.  A new
VERSION is marked every day at 1930 Socorro time (MST or MDT).
An example \file{crontab} file might resemble the following:

\begin{verbatim}
   # Cumulative update of the CASA directories each Saturday evening.
   00 22 * * 6   (. $HOME/.profile ; inhale.svn -c) 2>&1 | \
      mail aips2mgr aips2-inhale@nrao.edu
\end{verbatim}

\noindent
(Note that all \unixexe{cron} entries must be one-liners but they are broken
here for clarity.)  You may need to add the \exe{-n} option to \exeref{inhale.svn}
accordingly.  Note that, as in the above example, the log produced by
\aipsexe{inhale.svn} is generally forwarded to \acct{aips2-inhale@nrao.edu}.
These logs are archived for about 10 days and are accessible via the \aipspp\ 
home page \url{http://casa.nrao.edu/casa/docs/html/casa.html}.  This is
particularly useful for verifying code portability, especially on platforms
that a programmer doesn't have ready access to.  You should also add the email
address of a local person who will monitor the \aipsexe{inhale.svn} logs
(\acct{aips2mgr} in the above example).

\subsection*{Step 4. Update the Data Repository}
The data repository contains ancillary data that are necessary for proper data
reduction and analysis of astronomical data. While load-casapy downloads a snapshot
of the data repository, you may find it necessary from time-to-time to update the
repository or obtain data from the repository that is not included from running the
load-casapy script.  This can be done via rysnc.  More details may be found in the
Data Repository chapter \sref{Data Repository}.

\begin{verbatim}
cd /home/casa/data
rsync -avz rsync.aoc.nrao.edu::casadata-core .
\end{verbatim}

\subsection*{Step 5. SVN access to the development tree}
\index{nfs@\textsc{nfs}}
\index{automount, \textsc{nfs}}

For active developers, you will need a SVN account on the main SVN repository server.
Please send a request to aips2mgr@aoc.nrao.edu for your subversion account. For more information
about SVN visit the webpage at \htmladdnormallink{http://svnbook.red-bean.com}{http://svnbook.red-bean.com}.
If all you want to do is update the source code from time-to-time and rebuild, just do either
\begin{verbatim}
inhale.svn -c

or

cd code
svn update
sneeze -l -m cumulative
\end{verbatim}
