%\input /suru/efomalon/tex/texinit.all
\font\kxbgbf=cmb10     scaled \magstep4
%\font\kxbgbx=cmbx10    scaled \magstep4
%\font\kxbgbi=cmbxti10  scaled \magstep4
%\font\kxbgmi=cmmi10    scaled \magstep4
\font\kxbgrm=cmr10     scaled \magstep4
%\font\kxbgsl=cmsl10    scaled \magstep4
%\font\kxbgss=cmss10    scaled \magstep4
%\font\kxbgsb=cmssbx10  scaled \magstep4
%\font\kxbgsi=cmssi10   scaled \magstep4
\font\kxbgsy=cmsy10    scaled \magstep4
\font\kxbgit=cmti10    scaled \magstep4
\font\kxbgtt=cmtt10    scaled \magstep4
\font\kwbgbf=cmb10     scaled \magstep3
%\font\kwbgbx=cmbx10    scaled \magstep3
%\font\kwbgbi=cmbxti10  scaled \magstep3
%\font\kwbgmi=cmmi10    scaled \magstep3
\font\kwbgrm=cmr10     scaled \magstep3
%\font\kwbgsl=cmsl10    scaled \magstep3
\font\kwbgss=cmss10    scaled \magstep3
%\font\kwbgsb=cmssbx10  scaled \magstep3
%\font\kwbgsi=cmssi10   scaled \magstep3
%\font\kwbgsy=cmsy10    scaled \magstep3
\font\kwbgit=cmti10    scaled \magstep3
\font\kwbgtt=cmtt10    scaled \magstep3
\font\kvbgbf=cmb10     scaled \magstep2
%\font\kvbgbx=cmbx10    scaled \magstep2
%\font\kvbgbi=cmbxti10  scaled \magstep2
%\font\kvbgmi=cmmi10    scaled \magstep2
\font\kvbgrm=cmr10     scaled \magstep2
%\font\kvbgsl=cmsl10    scaled \magstep2
\font\kvbgss=cmss10    scaled \magstep2
%\font\kvbgsb=cmssbx10  scaled \magstep2
%\font\kvbgsi=cmssi10   scaled \magstep2
%\font\kvbgsy=cmsy10    scaled \magstep2
\font\kvbgit=cmti10    scaled \magstep2
\font\kvbgtt=cmtt10    scaled \magstep2
\font\kbigbf=cmb10     scaled \magstep1
%\font\kbigbx=cmbx10    scaled \magstep1
%\font\kbigbi=cmbxti10  scaled \magstep1
\font\kbigmi=cmmi10    scaled \magstep1
\font\kbigrm=cmr10     scaled \magstep1
\font\kbigsl=cmsl10    scaled \magstep1
%\font\kbigss=cmss10    scaled \magstep1
%\font\kbigsb=cmssbx10  scaled \magstep1
%\font\kbigsi=cmssi10   scaled \magstep1
\font\kbigsy=cmsy10    scaled \magstep1
\font\kbigit=cmti10    scaled \magstep1
\font\kbigtt=cmtt10    scaled \magstep1
\font\ksmbf=cmb10      scaled \magstephalf
\font\ksmbx=cmbx10     scaled \magstephalf
%\font\ksmbi=cmbxti10   scaled \magstephalf
\font\ksmmi=cmmi10     scaled \magstephalf
\font\ksmrm=cmr10      scaled \magstephalf
\font\ksmsl=cmsl10     scaled \magstephalf
\font\ksmss=cmss10     scaled \magstephalf
\font\ksmsb=cmssbx10   scaled \magstephalf
\font\ksmsi=cmssi10    scaled \magstephalf
\font\ksmsy=cmsy10     scaled \magstephalf
\font\ksmit=cmti10     scaled \magstephalf
\font\ksmtt=cmtt10     scaled \magstephalf
%%% \font\csc=cmcsc10    %%% for AJ
%%% \font\eightrm=cmr8   %%% for AJ
%%% \font\eightit=cmti8  %%% for AJ
% 
%%% Default definitions from KPNO.TEX form (default font, lines):
\def\kxztf{\ksmrm}    %%% \def\kxztf{\kbigtt}    
\def\kxzu{\hskip-0.45em{\leaders\hbox{\underbar{\hskip0.5em}}\hfill}\ \par}
\def\kvrule{\vrule width 0.6pt}
\def\khrule{\hrule height 0.6pt}
%
%START of [windhorst.papers]texinit.latex   %%% Use the rest for LATeX only
%%% Abbreviated skipping and indenting definitions:
\def\i {\indent}  %%%redefines \i = chardef\i"10  %%% CANNOT USE IN LATeX
\def\n {\noindent}
\def\no{\noindent}
\def\b {\bigskip} %%%redefines \b = \def\b{\???see p 356} % CANNOT USE IN LATeX
\def\m {\medskip}
\def\s {\smallskip} %%%note that \S = \def\S{\mathhexbox278} % Paragraph sign 
\def\bi{\bigskip\indent}     
\def\mi{\medskip\indent}
\def\si{\smallskip\indent}   
\def\bn{\bigskip\noindent}     
\def\mn{\medskip\noindent}
\def\sn{\smallskip\noindent}
\def\cl{\centerline} 
\def\ve{\vfill\eject}
\def\nref{\parskip0pt\par\noindent\hangindent\parindent\hangafter1}
%
%%% \input [windhorst.papers]texastsymb   %%% Defines astronomical symbols.
%%% First hh, mm, ss, deg, arcmin, arcsec:
\def\hh       {{$^{h}$}}
\def\mm       {{$^{m}$}}
\def\ss       {{$^{s}$}}                  %%% Overwrites TeX \ss (German sz)
\def\deg      {{\ifmmode^\circ\else$^\circ$\fi} } %%% Overwrites TeX \deg
\def\degree   {{\ifmmode^\circ\else$^\circ$\fi} } %%% Degree 
\def\arcm     {{\ifmmode {'  }\else$'     $\fi} } %%% Arc minutes
\def\arcmin   {{\ifmmode {'  }\else$'     $\fi} } %%% Arc minutes
\def\arcs     {{\ifmmode {'' }\else$''    $\fi} } %%% Arc seconds
\def\arcsec   {{\ifmmode {'' }\else$''    $\fi} } %%% Arc seconds
\def\secs     {{$^{\prime\prime}$}}               %%% Use \arcs or \arcsec
%%% Now ss, deg, arcmin, arcsec used over decimal point: 
\def\sspt     {{$\buildrel{s}           \over .$}}
\def\degpt    {{$\buildrel{\circ}       \over .$}}
\def\adegpt   {{$\buildrel{\circ}       \over .$}}
\def\arcmpt   {{$\buildrel{\prime}      \over .$}}
\def\arcspt   {{$\buildrel{\prime\prime}\over .$}}
\def\adeg     {{\hskip .10em {}^{\circ}\hskip -.40em . \hskip.25em} }
\def\amin     {{\hskip .10em {}'       \hskip -.30em . \hskip.20em} }
\def\asec     {{\hskip .10em {}''      \hskip -.49em . \hskip.29em} }
%%% Other definitions of Astronomical Symbols:  
\def\A        {{$\rm\AA$} }
\def\ABnu     {{AB$_{\nu}$} }
\def\aLyr     {{$\alpha$\ Lyr} }
\def\amed     {{$\alpha_{med}$} }
\def\aox      {{$\alpha_{ox}$} }
\def\aro      {{$\alpha_{ro}$} }
\def\alphaft  {{$\alpha_{1.4}^{0.6}$} }
\def\Aomega   {{$A_{\omega}$} }
\def\bII      {{$b^{II}$} }
\def\Bj       {{$B_{J}$} }
\def\Bk       {{$B_{k}$} }
\def\chisq    {{$\chi^{2}$} }
\def\cge      {{$_ >\atop{^\sim}$}}
\def\cle      {{$_ <\atop{^\sim}$}}
\def\degsq    {{$deg^{2}$} }
\def\sqdeg    {{$deg^{-2}$} }
\def\Dmod     {{$D_{mod}$} }
\def\eg       {{\it e.g.},}
\def\emin     {{$e^{-}$} }
\def\EBminV   {{$E_{(B-V)}$} }
\def\ergcms   {{$ergs\ cm^{-2}\ s^{-1}$} }
\def\ergcmsA  {{$ergs\ cm^{-2}\ s^{-1}\ \AA^{-1}$} }
\def\ergcmsHz {{$ergs\ cm^{-2}\ s^{-1}\  Hz^{-1}$} }
\def\ergs     {{$ergs\ s^{-1}$} }
\def\ergsA    {{$ergs\ s^{-1}\ \AA^{-1}$} }
\def\ergsHz   {{$ergs\ s^{-1}\  Hz^{-1}$} }
\def\et       {{et\thinspace al.} }	%%% et al.
\def\etal     {{et\thinspace al.} }	%%% et al.
\def\fsteep   {{$f(\alpha>1.0)$} }
\def\fflat    {{$f(\alpha<0.5)$} }
\def\Fl       {{$F_{\lambda}$} }
\def\Fn       {{$F_{\nu}$} }
\def\Fnu      {{$F_{\nu}$} }
\def\gmed     {{$g_{med}$} }
\def\Ho       {{$H_{0}$} }
\def\kms      {{$km\ s^{-1}$} }
\def\kmsMpc   {{$km\ s^{-1}\ Mpc^{-1}$} }
\def\lII      {{$l^{II}$} }
\def\Ha       {{H$\alpha$} }
\def\Hb       {{H$\beta$} }
\def\Hc       {{H$\gamma$} }
\def\Hd       {{H$\delta$} }
\def\hmin     {{$h^{-1}$} }
\def\ie       {{\it i.e.},}
\def\Jp       {{$J^{+}$} }
\def\KK       {{$K_{K}$} }
\def\KV       {{$K_{V}$} }
\def\Lya      {{Ly$\alpha$} }
\def\Lrad     {{$L_{rad}$} }
\def\LIR      {{$L_{IR}$} }
\def\Lopt     {{$L_{opt}$} }
\def\LUV      {{$L_{UV}$} }
\def\LX       {{$L_{X}$} }
\def\Lstar    {{$L^{*}$} }
\def\Lsun     {{$L_{\odot}$} }
\def\magarc   {{$mag\ arcsec^{-2}$} }
\def\Mgal     {{$M_{gal}$} }
\def\Mg2      {{$Mg_{2}$} }
\def\ml       {{$m_{\lambda}$} }
\def\Mo       {{$M_{\odot}$} }
\def\Moyr     {{$M_{\odot}\ yr^{-1}$} }
\def\Msun     {{$M_{\odot}$} }
\def\Msol     {{\ifmmode_{\mathord\odot}\else$_{\mathord\odot}$\fi} }
\def\Mstar    {{$M^{*}$} }
\def\Mpms     {{$M_{pms}$} }
\def\Mto      {{$M_{to}$} }
\def\Mtot     {{$M_{tot}$} }
\def\MU       {{$M_{U}$} }
\def\MB       {{$M_{B}$} }
\def\MV       {{$M_{V}$} }
\def\MR       {{$M_{R}$} }
\def\MI       {{$M_{I}$} }
\def\MJ       {{$M_{J}$} }
\def\MF       {{$M_{F}$} }
\def\MN       {{$M_{N}$} }
\def\MH       {{$M_{H}$} }
\def\MK       {{$M_{K}$} }
\def\Mg       {{$M_{g}$} }
\def\Mr       {{$M_{r}$} }
\def\Mi       {{$M_{i}$} }
\def\Mz       {{$M_{z}$} }
\def\muK      {{$\mu$K} }
\def\muJy     {{$\mu$Jy} }
\def\mjj      {{$(\mu$Jy)$^{2}$} }
\def\Nbin     {{$N_{bin}$} }
\def\Ngal     {{$N_{gal}$} }
\def\NH       {{$N_{H}$} }
%\def\nref    {{\parskip0pt\par\noindent\hangindent\parindent\hangafter1} }
\def\nref   {\noindent\parshape 2 0.0 truein 06.5 truein 0.4 truein 06.1 truein}
\def\persec   {{sec$^{-1}$} }
\def\perster  {{sr$^{-1}$} }
\def\Pc       {{$P^{c}$} }
\def\PsiEQ    {{$\Psi_{EQ}$} }
\def\Psicrit  {{$\Psi_{crit}$} }
\def\Psimed   {{$\Psi_{med}$} }
\def\Pstar    {{$P^{*}$} }
\def\Pivstar  {{$P_{1.4}^{*}$} }
\def\Pot      {{$P_{0.3}$} }
\def\Pov      {{$P_{0.4}$} }
\def\Pos      {{$P_{0.6}$} }
\def\Piv      {{$P_{1.4}$} }
\def\Pvo      {{$P_{5.0}$} }
\def\Pev      {{$P_{8.4}$} }
\def\pp     {\noindent\parshape 2 0.0 truein 06.5 truein 0.4 truein 06.1 truein}
%%% \def\pp {\noindent\parshape 2 0.0 truecm 16.5 truecm 1.0 truecm 15.5 truecm}
\def\qo       {{$q_{0}$} }
\def\sigv     {{$\sigma_{v}$} }
\def\Sot      {{$S_{0.3}$} }
\def\Sov      {{$S_{0.4}$} }
\def\Sos      {{$S_{0.6}$} }
\def\Siv      {{$S_{1.4}$} }
\def\Svo      {{$S_{5.0}$} }
\def\Sev      {{$S_{8.4}$} }
\def\Snu      {{$S_{\nu}$} }
\def\Sp       {{$S_{p}$} }
\def\Spmap    {{$S_{p}^{map}$} }
\def\Spsky    {{$S_{p}^{sky}$} }
\def\Stot     {{$S_{tot}$} }
\def\Stotmap  {{$S_{tot}^{map}$} }
\def\Stotsky  {{$S_{tot}^{sky}$} }
\def\SX       {{$S_{X}$} }
\def\Teff     {{$T_{eff}$} }
\def\Thmed    {{$\Theta_{med}$} }
\def\Up       {{$U^{+}$} }
\def\Vmax     {{$V_{max}$} }
\def\Wm       {{$W\ m^{-2}$} }
\def\WA       {{$W\ \AA^{-1}$} }
\def\WHz      {{$W\  Hz^{-1}$} }
\def\WmA      {{$W\ m^{-2}\ \AA^{-1}$} }
\def\WmHz     {{$W\ m^{-2}\  Hz^{-1}$} }
\def\WAm      {{$W\ \AA^{-1}\ m^{-2}$} }
\def\WHzm     {{$W\  Hz^{-1}\ m^{-2}$} }
\def\Wl       {{$W_{\lambda}$} }
\def\EW       {{$W_{\lambda}$} }
\def\wth      {{$w(\theta$)} }
\def\wtheta   {{$w(\theta$)} }
\def\wthns    {{$w(\theta$)}}
\def\zobs     {{$z_{obs}$} }
\def\zmed     {{$z_{med}$} }
\def\zmax     {{$z_{max}$} }
\def\zmin     {{$z_{min}$} }
\def\zf       {{$z_{f}$} }
\def\zform    {{$z_{form}$} }
%
%%% \input [windhorst.papers]texrefapj    %%% Initialize references as for ApJ. 
%%% Macro's for Astrophysical Journal (Main Journal+Letters+Suppl) (DEFAULT):
%%% Modified 900830: include Style Changes to uniformize styles across journals
%%% See Abt, H. A., 1990 ApJ, 357, 1
%%% Astronomy Journals:
%%% 
\def\AAP     {A\&A, }
\def\AAL     {A\&AL, }
\def\AAS     {A\&AS, }
\def\ACTAA   {Acta~A, }
\def\AJ      {AJ, }
\def\AN      {AN, }
\def\APLET   {Ap. Let., }
\def\APJ     {ApJ, }
\def\APJL    {ApJL, } %%% In AJ: referred to as ApJL, L...
\def\APL     {ApJ, }  %%% In ApJ: referred to as ApJ, L...
%%% \def\APL {ApJL, }
\def\APJS    {ApJS, }
\def\APS     {ApJS, }
\def\APSS    {Ap\&SS, }
\def\ARAA    {ARA\&A, }
\def\AUSJP   {Australian J. Phys., }
\def\AZH     {AZh, }
\def\BAAS    {BAAS, }
\def\FCP     {Fund. Cosmic Phys., }
\def\HOA     {Highlights Astr., }
\def\IAU     {IAU Symp., }
\def\ITR     {Internal Technical Report of the Netherlands Foundation for
              Research in Astronomy, No.} 
\def\JOSAMA  {J. Opt. Soc. Am. A, }
\def\JRASC   {JRASC, }
\def\MEMRAS  {MmRAS, }
\def\MNRAS   {MNRAS, }
\def\NAT     {Nature, }
\def\NPS     {Nat. Phys. Sc., }
\def\OE      {Opt. Engineering, }
\def\PASJ    {PASJ, }
\def\PASP    {PASP, }
\def\PHD     {Ph.D. thesis, }
\def\PHYSCR  {Physica Scripta, }
\def\QJRAS   {QJRAS, }
\def\REP     {Reports on Astronomy, IAU Transactions, }
\def\REPPPHY {Rep. Prog. Phys., }
\def\REVMP   {Rev. Mod. Phys., }
\def\SCAM    {Sc. Am., }
\def\SCI     {Science, }
\def\SOVAST  {Sov. Astr., }
\def\SPIE    {SPIE, }
\def\ST      {S\&T, }
\def\VIS     {Vistas in Astronomy, }
\def\VLA     {VLA Test Memorandum, No.}
%%% 
%%% IAU Symposia (Reidel, Kluwer):
%%% 
\def\IAUOOD  {in IAU Symposium 4, 
             Radio Astronomy, 
             ed. H. C. van der Hulst (Cambridge University Press), } 
\def\IAUOFC  {in IAU Symposium 63, 
             Confrontation of Cosmological Theories with Observational Data,
             ed. M. S. Longair (Dordrecht: Reidel), } 
\def\IAUOGD  {in IAU Symposium 74, 
             Radio Astronomy and Cosmology,
             ed. D. L. Jauncey (Dordrecht: Reidel), } 
\def\IAUOGI  {in IAU Symposium 79, 
             The Large Scale Structure of the Universe, 
             ed. M. S. Longair, \& J. Einasto (Dordrecht: Reidel), }
\def\IAUOIB  {in IAU Symposium 92, 
             Objects of High Redshift, 
             ed. G. O. Abell, \& P. J. E. Peebles (Dordrecht: Reidel), }
\def\IAUOIG  {in IAU Symposium 97, 
             Extragalactic Radio Sources, 
             ed. D. S. Heeschen, \& C. M. Wade (Dordrecht: Reidel), }
\def\IAUAOD  {in IAU Symposium 104, 
             Early Evolution of the Universe and its Present Structure, 
             ed. G. O. Abell, \& G. Chincarini (Dordrecht: Reidel), } 
\def\IAUAAO  {in IAU Symposium 110, 
             VLBI and Compact Radio Sources, 
             ed. R. Fanti, K. Kellermann, \& G. Setti (Dordrecht: Reidel), } 
\def\IAUAAI  {in IAU Symposium 119, 
             Quasars, 
             ed. G. Swarup, \& V. K. Kapahi (Dordrecht: Reidel), } 
\def\IAUABA  {in IAU Symposium 121, 
             Observational Evidence of Activity in Galaxies, 
             ed. E. Ye. Khachikian, K. J. Fricke, \& J. Melnick (Dordrecht: 
             Reidel), } 
\def\IAUABD  {in IAU Symposium 124, 
             Observational Cosmology, 
             ed. A. Hewitt, G. Burbidge, \& L. Z. Fang (Dordrecht: Reidel), } 
\def\IAUACO  {in IAU Symposium 130, 
             Large Scale Structures of the Universe, 
             ed. J. Audouze, M.-C Pelletan, \& A. Szalay (Dordrecht: Kluwer), }
\def\IAUACD  {in IAU Symposium 134, Active Galactic Nuclei, 
             ed. D. E. Osterbrock, \& J. S. Miller (Dordrecht: Kluwer), } 

\def\REPXIX  {in Reports on Astronomy, IAU Transactions, Vol. XIX-A, 
             ed. R. M. West (Dordrecht: Reidel), } 
\def\HOAVII  {in Highlights Astr., Vol. 7, 
             ed. J.-P. Swings (Dordrecht: Reidel), }
%%% 
%%% Astronomical Society of the Pacific Conference Series: 
%%% 
\def\ASPOOB  {in ASP Conf. Ser., Vol.  2,
             Proceedings of a Workshop on Optical Surveys for Quasars, 
             ed. P. S. Osmer, A. C. Porter, R. F. Green, \& C. B. Foltz 
             (Provo, UT: Brigham Young University Print Services), } 
\def\ASPOOE  {in ASP Conf. Ser., Vol.  5,
             The Minnesota Lectures on Clusters of Galaxies and Large-Scale 
             Structure,
             ed. J. M. Dickey 
             (Provo, UT: Brigham Young University Print Services), } 
\def\ASPOOF  {in ASP Conf. Ser., Vol.  6,
             Synthesis Imaging in Radio Astronomy, 
             ed. R. A. Perley, F. R. Schwab, \& A. H. Bridle 
             (Provo, UT: Brigham Young University Print Services), } 
\def\ASPOOH  {in ASP Conf. Ser., Vol.  8,
             CCD's in Astronomy, 
             ed. G. H. Jacoby 
             (Provo, UT: BookCrafters, Inc.), } 
\def\ASPOAO  {in ASP Conf. Ser., Vol. 10,
             Evolution of the Universe of Galaxies (Edwin Hubble Centennial 
             Symposium), 
             ed. R. G. Kron 
             (Provo, UT: BookCrafters, Inc.), } 
%%% 
%%% Astrophysics and Space Science Library: 
%%% 
\def\ASSLODC {in Astrophysics and Space Science Library, Vol. 43,
             X-ray Astronomy, 
             ed. R. Giacconi, \& H. Gursky (Dordrecht: Reidel), } 
\def\ASSLOHG {in Astrophysics and Space Science Library, Vol. 87,
             X-ray Astronomy with th Einstein Satellite, 
             ed. R. Giacconi (Dordrecht: Reidel), } 
\def\ASSLABB {in Astrophysics and Space Science Library, Vol. 122,
             Spectral Evolution of Galaxies, 
             ed. C. Chiosi, \& A. Renzini (Dordrecht: Reidel), } 
\def\ASSLADA {in Astrophysics and Space Science Library, Vol. 141,
             Towards Understanding Galaxies at Large Redshift, 
             ed. R. G. Kron, \& A. Renzini (Dordrecht: Kluwer), }
%%% 
%%% NATO Advanced Science Institute Series: 
%%% 
\def\NATOOFO {in NATO Advanced Study Institute Series, Vol. C060, 
             X-ray Astronomy, 
             ed. R. Giacconi, \& G. Setti (Dordrecht: Reidel), } 
\def\NATOOIG {in NATO Advanced Study Institute Series, Vol. C097, 
             The Origin and Evolution of Galaxies, 
             ed. B. J. T. Jones, \& J. E. Jones (Dordrecht: Reidel), } 
\def\NATOAHO {in NATO Advanced Science Institutes Series, Vol. C180, 
             Galaxy Distances and Deviations from Universal Expansion, 
             ed. B. F. Madore, \& R. B. Tully (Dordrecht: Reidel), }
\def\NATOBFD {in NATO Advanced Science Institutes Series, Vol. C264, 
             The Epoch of Galaxy Formation, 
             ed. C. S. Frenk, R. S. Ellis, T. Shanks, A. F. Heavens, 
             \& J. A. Peacock (Dordrecht: Kluwer), }
%%% 
%%% Pontificiae Academiae Scientiarum Scripta Varia: 
%%% 
\def\PONTODH {in Pontificiae Academiae Scientiarum Scripta Varia, Vol. 48,
             Astrophysical Cosmology, 
             Proceedings of the Study Week on Cosmology and Fundamental 
             Physics, ed. H. A. Br\"uck, G. V. Coyne, \& M. S. Longair 
             (Vaticano: Pontificia Academia Scientiarum), }
%%% 
%%% Proceedings of SPIE conferences: 
%%% 
\def\SPIEBFD {in Proceedings of the SPIE Conference, Vol. 264, 
             Applications of Digital Image Processing to Astronomy,
             ed. D. A. Elliott (Bellingham WA: SPIE), } 
%%% 
%%% Various conference proceedings: 
%%% 
\def\APAGQSO {in The Seventh Santa Cruz Summer Workshop in Astronomy and 
             Astrophysics, 
             Astrophysics of Active Galaxies and Quasi-Stellar Objects,
             ed. J. S. Miller (Mill Valley CA: University Science Books), }
\def\GALEGRA {in Astronomy and Astrophysics Library, 
             Galactic and Extragalactic Radio Astronomy, 
             ed. G. L. Verschuur, \& K. I. Kellermann 
             (New York: Springer Verlag), }
\def\IAPC    {in The third IAP Workshop, 
             High Redshift and Primeval Galaxies, 
             ed. J. Bergeron, D. Kunth, B. Rocca-Volmerange, \& 
             J. Tran Thanh Van (Gif sur Yvette France: Editions Frontieres), }
\def\NNGXYS  {in The Eighth Santa Cruz Summer Workshop in Astronomy and 
             Astrophysics, 
             Nearly Normal Galaxies from the Planck Time to the Present, 
             ed. S. M. Faber (New York: Springer-Verlag), }
\def\NASACOBE{in After the First Three Minutes, AIP Conf. Proc. Vol. 222, 
             ed. S. S. Holt, C. L. Bennett, \& V. Trimble 
             (New York: American Institute of Physics), }
\def\OBSCOS  {in Eight Advanced Course of the Swiss Society of Astronomy and 
             Astrophysics, 
             Observational Cosmology,
             ed. A. Maeder, L. Martinet, \& G. Tammann (Geneva Observatory), }
\def\PHYCOS  {in Les Houches Session XXXII, 
             Physical Cosmology, 
             ed. R. Balian, J. Audouze, D. N. Schramm (Amsterdam:
             North-Holland Publ. Co.), }
\def\QUAGRL  {in 24$^{th}$ Liege International Astrophysical Colloquium, 
             Quasars and Gravitational Lenses, 
             ed. J.-P. Swings (Liege University Press), }
\def\SSSIX   {in Stars and Stellar Systems, Vol. IX,
             Galaxies and the Universe,
             ed. A. Sandage, M. Sandage, \& J. Kristian 
             (University of Chicago Press), }
%%% 
%%% (One-Author) astronomy books:
%%% 
\def\APFORM  {in Astrophysical Formulae, Second Edition, 
             (New York: Springer Verlag), } %%% ed. K. R. Lang 
\def\APGNAGN {in Astrophysics of Gaseous Nebulae and Active Galactic Nuclei,
             (Mill Valley CA: University Science Books), }%ed. D. E. Osterbrock
\def\QSRAST  {in Quasar Astronomy, 
             (New York: Cambridge University Press), } %%% ed. D. W. Weedman 
%%%
%%% \input [windhorst.letters]asuheading  %%% Provides ASU letter head on top.
%
%%% \nopagenumbers %%% \headline={\hfil\ksmtt Windhorst}
%%% \pageno=1\headline={\ifnum\pageno=1 \hfil\else\hss\ksmrm- \folio\ -\hss\fi} 
%%% \footline={\hfil\ksmtt 11/91 } %%% May need wider page definition for letterhead:
%%% \hoffset=-0.2in\voffset=-0.4in\hsize=6.9in\vsize=9.8in\settabs 6\columns\ksmrm
%%% \hoffset=-0.1in\voffset=-0.2in\hsize=6.5in\vsize=8.5in\settabs 6\columns\ksmrm
    \hoffset=-0.1in\voffset=-0.2in\hsize=6.5in\vsize=9.0in\settabs 6\columns\ksmrm
\baselineskip=12pt %%% baselineskip=12pt for single spacing, =24pt for double
%


\magnification=\magstephalf
\cl {\bf{The YEG Object}}
\b
\cl {Ed Fomalont}
\s
\cl {April 6, 1992}
\bn
{\bf Introduction}
\s
The YEG is the `(u-v) data' object.  It contains the observed (uncalibrated)
data and it is linked to the Telescope Model Object which contains the
description of the telescope.  By mysterious means (at this point) the
description of the telescope model can be improved so that the YEG
data object can ultimately transform the observed data into
calibrated data.  This memo is a first pass attempt to define the
fundamental organization and contents of the YEG.
\bn
{\bf 1. The Fundamental Coordinate of the YEG: t}
\m
The independent coordinate for each YEG is $t$, time.  Each YEG
is associated with one and only one time and a YEG includes all of the
data at that time.  How this time is labelled is unimportant as long
as there is an association between the label and the true value of
this fundamental coordinate.  It need not be regularly spaced.  For
example, the time coordinate could be $(1,2,\ldots,T)$ with a table
relating the indices to a time.

Since data are usually averaged over an interval of time, we shall
define $t$ as the average value of the sampled data over a generally
small interval of time, and $\Delta t$, is the duration over which the
data was actually collected.  Furthermore, not all quantities are
sampled and averaged in time in the same manner.  However, we shall
assume that either; 1) all data in the YEG and in the Telescope Model
are critically sampled and can be interpolated to any desired time; or
2) the data can be given an undefined valued and dealt with
intelligently.
\bn
{\bf 2. The Fundamental Data Organization of a YEG:}
\m
For arrays, the fundamental data is a complex number, the visibility
function.  It is associated with two antennas, labelled by $i$ and
$j$; we should not rule out $i=j$.  There should also be a real number
associated with each visibility function which gives some indication
of its relative merit (weight), and there should be a duration,
$\Delta t$, over which the data were integrated.  Perhaps, the weight
and duration are redundant in many cases.

The visibility function, weight and duration are each a
multi-dimensional set of numbers spanning over frequency,
polarization, delay, and sky offset.  We shall denote this collection
of uncalibrated visibilities at time $t$ for antenna pairs $(i,j)$ as 
          $$V_u(t: i,j; l,m,n,o) $$
\sn where
\settabs 2 \columns
\+\indent  $l=1,2,\ldots,L$\hfill&
     The label associated with the frequency $\nu$\hfill&\cr
\+\indent  $m=1, 2, 3, 4$\hfill&
     The label associated with the polarization $p$\hfill&\cr
\+\indent  $n=1,2,\ldots,N$\hfill&
     The label associated with the delay $\tau$\hfill&\cr
\+\indent  $o=1,2,\ldots,O$\hfill&
     The label associated with the sky location $s$\hfill&\cr
\cleartabs
\mn
The `:' and `;' separation in the argument of the visibility function
indicates the difference between the time and the antenna pair and the
other four quantities.  Time is the independent variable, the antenna
pair are not true dimensions because they are coupled; the remaining
four quantities are orthogonal dimensions in the data.  The weights and
durations have the same structure as the visibility data.
\b
These labels must be associated with values in some manner, perhaps
as tables in the Telescope Model Class:
\s
1.  A table of antenna names corresponding to the antenna index.  This
is not a true dimension of the visibility data as those which follow.
The array, by its nature, couples the response between antennas in
complicated ways.  The antennas could be in different instruments;
egs, subarrays at the VLA, and share in the Telescope Model.  In this
case a visibility function between two antennas not in the same array
is undefined.  This scheme could be useful if one subarray calibrates
some needed parameters in the other subarray.
\s
2.  A table of the average observing frequency corresponding to the
frequency index, $l$.  In principle each antenna could have a
different observing frequency.  There is no assumed regularity in
the frequency coordinate.  It could be a combination of several bands,
each with an number of closely spaced frequencies.
\s
3.  A table of polarization parameters, corresponding to the index
$p$, which describe the general ellipticity of the radiation accepted
by the antenna.  It is often the case that the two antennas have
different polarization parameters.  This index is generally limited to
no more than four entries.
\s
4.  A table of delays corresponding to the delay index, $n$.  This is
the additional time lag introduced in each antenna before correlation.
It is generally used in VLBI observations.  It turns out that if there
are some time inconsistencies among elements in the YEG, this
coordinate may `fix' things up.
\s
5.  A table of sky offsets coresponding to the offset index, $o$.
These could be associated with multi-beam arrays, mozaicing, or with
VLBI observations in which more than one field is correlated.
\m
This fundamental organization of the YEG should be questioned.  Is
this suggested organization of the visibility data (deeply time
oriented, antenna-pair connected, with four independent dimensions of
frequency, polarization, delay and sky position) sufficiently general
to cover all anticipated array observations and reductions?  Is it
organized in a convenient form for calibration, imaging,
self-calbration, mozaicing, and isoplanicity problems?.  How flexible
should the underlying software be.  Could more dimensions be added?
Should we be able to transpose some of the coordinates; ie. could
frequency be the fundamental coordinate, with time one of the four
dimensions?  Is the extension to non-array, or mixed single dish-array
data sets possible?  If not, why not?
\bn
{\bf 3. Yeg Interaction with the Telescope Model}
\m
At the highest level the YEG interaction with the Telescope Model is
simple and direct.  The YEG sends seven numbers (pointers)
$(t,i,j,l,m,n,o)$ to the Telescope Model which returns a complex
gain $G(t:i,j;l,m,n,o)$.  The calibrated visibility data
$V_c(t:i,j;l,m,n,o)$ is simply the complex product of the gain with
the uncalibrated visibility function.

$$  V_c(t:i,j;l,m,n,o) = G(t:i,j;l,m,n,o) \times V_u(t:i,j;l,m,n,o) $$
\mn
Whether this product is implicitly or explictly made is not important.
How the Telescope Model copes with the seven pointers in order to
determine the complex gain, is of no concern to the YEG.

The preceding paragraph is an oversimplification which we will get to
in a moment.  The most important point is that the calibration of
visibility data at any time depends only the on the YEG elements at
that time and the complex gain function of the Telescope Model at that
time.  If the calibration of data at time $t$ required knowledge of
the data at time $t'$ or the state of the telescope model at $t'$,
then the present formulation of the YEG would not be satisfactory.

On the other hand, the determination of the Telescope Model and all of
the details of the gain depends on the proper analysis of a
large collection of data made over long periods of time.  This is of
no concern to the YEG.  Somehow, there is a mysterious SOLVE which
knows what to do with lots of YEG and can determine the Telescope
Model.

The above expression also assumes that the calibration of the
visibility function in any state (particular value of $i,j,l,m,n,o$)
depends only on that state.  This is not strictly true and it is the
reason why polarization correction is a nuisance.  Although the
details of this belong in the relevant place of the Telescope Model,
the polarization correction combines, in general, the different values
of the polarization index $m$ for any baseline and time.  Any other
cross talk amongst the baselines, or frequencies, etc, requires a more
general form of the gain correction

$$  V_c(t:i,j;l,m,n,o) = \sum_{all '} G(t:i,i',j,j';k,k',l,l',m,m',o,o')
     \times V_u(t:i',j';l',m',n',o') $$
\mn
where the Gain term now includes all possible cross talk.  Even more
generally, the gain term is an arbitrary gain function.  It can be
more than a complex multiplier.  For example, there are some additive
corrections to visibility data; egs. correlator offset signals and
some kinds of closure errors.  Some amplitude corrections are
non-linear when the source noise rivals the system noise. However,
these complications are rarely met and they are handled by the
time-oriented form of the YEG.

It will turn out that much of the calibration terms are separable into
individual antennas, frequencies, polarization, delay and sky offset
functions.  That is

$$ G(t:i,j;l,m,n,o) =   G_\nu(t:i;l)\times G_\nu^*(t:j;l)~\times~
                        G_p(t:i,m)\times G_p^*(t:j,m) $$
$$              ~\times~G_d(t:i;m)\times G_d^*(t:j;m)
                ~\times~G_s(t:i;o)\times G_s^*(t:j;o)  $$
\sn
where $^*$ indicates the complex conjugate.  It seems likely that the
Telescope Model will assume the above simple form and add complexities
as needed.  However, the ability to do the arbitrary gain function
correction should not be excluded.

The flagging or editting of the data can be handled in several ways.
The weights associated with the visibility data in the YEG can be
modified as an indication of the new quality of the data.
Alternatively, the Telescope Model can contain a Flag function (or
table) $F(t:i,j;l,m,n,o)$ with any degree of complication.  It may be
a binary-valued function (0 or 1) or it may contain quality
assessments in some way.
\bn
{\bf 4. YEG interaction with the Inverter}
\m 
There is little to discuss here.  The calibrated data must be passed
to the Inverter in order to create an image.  Additional parameters,
such as the spatial frequencies $(u,v,w)$, will probably reside in the
Telescope Model.  At this point calibrated visibility data over a long
period of time, at several pointing, or from several telescopes is
collected and can be subsequently modified (gridding, tapering the
data) for use in the Inverter algorithm.  YEG simply transfers the
calibrated data under some controller outside of the scope of the YEG
object.
\bn
{\bf 5.  YEG interaction with Self-calibration}
\m
The time-oriented YEG Class is compatible with the self-calibration
technique.  Although there are variations among the self-calibration
tecnhiques, the following steps generally occur:
\sn
1). Use a model image which is an approximation of the image
associated with the visibility data.  The origin and form of the image
is immaterial.  For normal calibration, it is a point source at the
phase center.
\sn
2). Given the state $(i,j:l,m,n,o)$ of the YEG at any time, $t$, the
Predictor Object(?) calculates the visibility data for the model
image.  Access to the telescope model will be needed by Predictor in
order to compute the visibility data; however, the pointer
$(i,j:l,m,n,o)$ at any time should be sufficient to find the
appropriate data.
\sn
3). The complex ratio (calibrated YEG / predicted YEG) is formed.
Wide ranges in the signal to noise, because the non-linearity of the
division, can occur and must be dealt with properly.  If the data were
perfectly calibrated, the complex ratio would be equal to unity.
Alternatively, the complex ratio (uncalibrated YEG / predicted YEG)
can be formed and compared with the gain function used to calibrate
the data.  Both procedures are almost identical.
\sn
3). Data averaging over time, frequency, polarization, etc. is
important because self-calibration only works if the sky signals are
larger than the noise signals.  Averaging of the YEG in this way
produces a related YEG with less frequently sample visibility data
and, perhaps, fewer dimensions.
\sn
4). A intelligent piece of code interprets the departure from unity of
the calibrated / predicted visibility function in terms of
modification of the gain function.  The decision of which terms in the
gain function can and should be modified is extremely complicated.
\bn
{\bf Conclusion}
\m
This memo has discussed the form of a possible YEG object used for
synthesis reductions.  This YEG comprises all of the visibility data
at one time and is generally a small part of the entire data base.  It
is suggested that all interactions with the Telescope Model can be
made on a time by time basis and interactions with imaging and
self-calibration processes can also be handled with this basic YEG.
\end

