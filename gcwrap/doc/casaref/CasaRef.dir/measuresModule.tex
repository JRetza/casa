%% Copyright (C) 1999,2000,2001,2002
%% Associated Universities, Inc. Washington DC, USA.
%%
%% This library is free software; you can redistribute it and/or modify it
%% under the terms of the GNU Library General Public License as published by
%% the Free Software Foundation; either version 2 of the License, or (at your
%% option) any later version.
%%
%% This library is distributed in the hope that it will be useful, but WITHOUT
%% ANY WARRANTY; without even the implied warranty of MERCHANTABILITY or
%% FITNESS FOR A PARTICULAR PURPOSE.  See the GNU Library General Public
%% License for more details.
%%
%% You should have received a copy of the GNU Library General Public License
%% along with this library; if not, write to the Free Software Foundation,
%% Inc., 675 Massachusetts Ave, Cambridge, MA 02139, USA.
%%
%% Correspondence concerning AIPS++ should be addressed as follows:
%%        Internet email: aips2-request@nrao.edu.
%%        Postal address: AIPS++ Project Office
%%                        National Radio Astronomy Observatory
%%                        520 Edgemont Road
%%                        Charlottesville, VA 22903-2475 USA
%%
%% $Id$
\begin{ahmodule}{measures}{Measures handling}

%%\ahinclude{measures.g}

\ahkeyword{measures}{}
\ahkeyword{coordinates}{}

\begin{ahdescription}

A measure is a \ahlink{quantity}{quanta}
with a specified reference frame (e.g. UTC, J2000, mars).
The measures module provides an interface to the handling of measures. The
basic functionality provided is:
\begin{itemize}
\item {\em Conversion} Conversion of measures, especially between different
frames (e.g. UTC to LAST)
\item {\em Calculation} Calculation of e.g. a rest frequency from a velocity
and a frequency.
\end{itemize}
This functionality is provided in a command line interface.
%% and a
%%{\em GUI} interface.

%%To access the {\mf measures} module, include the {\em measures.g}
%%script.  This will create a default Measures \tool\ called {\stf dm}.

%%Altough a Measures \tool\ has state (it remembers the frame you are working
%%in), there is only in very special cases the need to create your own \tool\ .
%%Just use the default {\stf dm}
%%one.  However, if you do require one, perhaps to distinguish working in two
%%different environment frames for an input observation and and output display,
%%then use the {\cf measures} constructor to make one:
%%
%%\begin{verbatim}
%%- include 'measures.g';
%%- mydm = measures();		# measures tool created for you
%%- mydm.type();  
%%measures 
%%\end{verbatim}
%%
%%In normal use, the {\stf measures} \tool\ will be operated using
%%a {\em GUI}. All operations can be done from either the command line or the
%%{\em GUI}.

%%The \ahlink{aipsrc}{aipsrcdata} mechanism is used by the measures
%%\tool\ to specify certain operational details (like using DE405 rather than
%%the default DE200 for planetary positions).


{\large Measures}


Measures are e.g. an epoch, or coordinates, which have, in addition to values
(as quantities), also a reference specification, and possibly an offset. They
are represented as records with fields describing the various entities
embodied in the measure. These entities can be obtained by the access methods
{\em gettype}, {\em getref}, {\em getoffset}. {\em getvalue}.

Each measure has its own list of reference codes (see the individual
methods for creating them, like {\em direction}). If an empty or no
code reference code is given, the default code for that type of
measure will be used (e.g. it is {\em J2000} for a direction). If an
unknown code is given, this default is also returned, but with a
warning message.

The values of a measure (like the right-ascension for a direction) are given
as \ahlink{quantities}{quanta}. Each of them can be either a scalar quantity
with a scalar or vector for its actual value (see the following example).
%% The
%%values can also be given as an \ahlink{rArray}{quanta:rArray}, in which
%%case each element of that array will be used as a value. 
E.g a vector of
length 2 of quanta will be seen in a direction constructor as a longitude and
a latitude.

\begin{verbatim}
"""  
#  
print "\t----\t Module Ex 1 \t----"
print me.epoch('utc','today')	     # note that your value will be different
#{'type': 'epoch', 'm0': {'value': 54175.865923379628, 'unit': 'd'}, 'refer': 'UTC'}
print me.direction('j2000','5h20m','-30.2deg')
#{'type': 'direction', 'm1': {'value': -0.52708943410228748, 'unit': 'rad'}, 'm0': {'value': 1.3962634015954634, 'unit': 'rad'}, 'refer': 'J2000'}
a = me.direction('j2000','5h20m','-30.2deg')
print me.gettype(a)
#Direction 
print me.getoffset(a)
#{}
print me.getref(a)
#J2000 
print me.getvalue(a)  
#{'m1': {'value': -0.52708943410228748, 'unit': 'rad'}, 'm0': {'value': 1.3962634015954634, 'unit': 'rad'}}
print me.getvalue(a)['m0']
#{'value': 1.3962634015954634, 'unit': 'rad'}
print me.getvalue(a)['m1']
#{'value': -0.52708943410228748, 'unit': 'rad'}
print 'Last example! Exiting ...'
exit()
#
"""
\end{verbatim}
%%# try as a scalar quantity with multiple values (mult values NOT IMPLEMENTED)
%%#a = me.direction('j2000', qa.quantv([10,20],'deg'), qa.quantv([30,40], 'deg'))
%%#print me.getvalue(a)['m0']
%%#[unit=rad, value=[0.174532925 0.34906585] ] 
%%#print qa.getvalue(me.getvalue(a)['m0'])['value']
%%#0.34906585 
%%#print a
%%#[type=direction, refer=J2000, m1=[value=[0.523598776 0.698131701] , unit=rad],
%%#	 m0=[unit=rad, value=[0.174532925 0.34906585] ]] 
%%# try as an r_array  (r_array NOT IMPLEMENTED)
%%#b = r_array(,2)
%%#r_fill(qa.quantity([10,20],'deg'), b, 1)
%%#T 
%%#r_fill(qa.quantity([30,40],'deg'), b, 2)
%%#T 
%%#a = me.direction('j2000', b)
%%#me.getvalue(a)
%%#[*61=[unit=rad, value=[0.174532925 0.34906585] ],
%%#		 *62=[value=[0.523598776 0.698131701] , unit=rad]] 
%%#

Known measures are:
\begin{itemize}
	\item epoch: an instance in time (internally expressed as MJD or MGSD)
	\item direction: a direction towards an astronomical object
(including planets, sun, moon)
	\item position: a position on Earth
	\item frequency: electromagnetic wave energy
	\item radialvelocity: radial velocity of astronomical object
	\item doppler: doppler shift (i.e. radial velocity in non-velocity
units like 'Optical', 'Radio'.)
%% The radialvelocity GUI caters for Doppler coding as well)
	\item baseline: interferometer baseline
	\item uvw: UVW coordinates
	\item earthmagnetic: Earth' magnetic field
\end{itemize}

In addition to the reference code (like J2000), a measure needs sometimes more
information to be convertable to another reference code (e.g. a time and
position to convert it to an azimuth/elevation). This additional information
is called the reference {\em frame}, and can specify one or more of 'where am
i', 'when is it', 'what direction", 'how fast'.

The frame values can be set by the
\ahlink{doframe}{measures:measures.doframe} tool function.
%%, or by a menu on the measure guis.

%%Since you would normally work from a fixed position, the position frame
%%element ('where you are'), can be specified in your .aipsrc if its name is in the
%%Observatory list (\ahlink{obslist}{measures:measures.obslist}) tool
%%function. You can set your preferred position by adding to your {\em .aipsrc}
%%file: 
%%
%%\begin{verbatim}
%%measures.default.observatory:	atca
%%\end{verbatim}
%%
%%More information on the individual measures can be found in the GUI
%%descriptions for \ahlink{epoch}{measures:measures.epochgui}; 
%%\ahlink{direction}{measures:measures.directiongui};
%%\ahlink{position}{measures:measures.positiongui}; 
%%\ahlink{frequency}{measures:measures.frequencygui};
%%\ahlink{radialvelocity}{measures:measures.radialvelocitygui};
%%\ahlink{doppler}{measures:measures.dopplergui}.

\end{ahdescription}

%%\begin{ahexample}
%%Suppose we have an object with a J2000 right ascension and declination, and
%%want to know the time of rising above the horizon of a certain astronomical
%%direction. 
%%In practice you would do this using the special application in the gui,
%% but the following shows how to do it the hard and long way.
%%
%%\begin{verbatim}
%%
%%#  
%%print "\t----\t Module Ex 2 \t----"
%%tim = me.epoch('utc','today')				        # 1
%%print "tim=", tim
%%#{'type': 'epoch', 'm0': {'value': 54175.8838019213, 'unit': 'd'}, 'refer': 'UTC'}
%%pos = me.observatory('ATCA')					# 2
%%print "me.doframe(tim)=", me.doframe(tim) 			# 3 
%%#Epoch: 54175::21:12:40.4860
%%#True
%%print "me.doframe(pos)=", me.doframe(pos)         		# 4
%%#Position: [-4.75092e+06, 2.79291e+06, -3.20048e+06]
%%#True
%%coord = me.direction('J2000', '5h20m30.2', '-30d15m12.5') 	# 5
%%hadec = me.measure(coord,'hadec')				# 6
%%last = me.measure(tim,'last')				        # 7
%%sdec = qa.sin(me.getvalue(hadec)['m1'])				# 8
%%cdec = qa.cos(me.getvalue(hadec)['m1'])
%%clat = qa.cos(me.getvalue(pos)['m1'])
%%slat = qa.sin(me.getvalue(pos)['m1'])
%%ha = qa.acos(qa.neg(qa.div(qa.mul(sdec, slat),qa.mul(cdec, clat))))  # 9
%%print qa.norm(qa.add(ha, me.getvalue(coord)['m0']),0)		#10
%%#{'value': 189.92103895797186, 'unit': 'deg'}
%%print qa.time(qa.norm(qa.add(ha, me.getvalue(coord)['m0']),0))	#11
%%#12:39:41
%%rt = qa.totime(qa.norm(qa.add(ha, me.getvalue(coord)['m0'])))	#12
%%print "rt=", rt
%%#{'value': -0.47244155845007818, 'unit': 'd'}
%%rtoff = me.epoch('r_utc', me.getvalue(tim)['m0'])	        #13
%%print "rtoff=", rtoff
%%#{'type': 'epoch', 'm0': {'value': 54175.0, 'unit': 'd'}, 'refer': 'R_UTC'}
%%rte = me.epoch('last', rt, off=rtoff)			        #14
%%print "rte=", rte
%%#{'refer': 'LAST', 'type': 'epoch', 'm0': {'value': -60895.47244155845, 'unit': 'd'}, 'offset': {'type': 'epoch', 'm0': {'value': 54175.0, 'unit': 'd'}, 'refer': 'R_UTC'}}  #CHECK!
%%print "me.measure(rte, 'utc') =", me.measure(rte, 'utc')	#15
%%#{'type': 'epoch', 'm0': {'value': -6555.0954994312287, 'unit': 'd'}, 'refer': 'UTC'} #CHECK!
%%print qa.time(me.getvalue(me.measure(rte, 'utc'))['m0'], form=["ymd","time"])
%%#1840/12/05/21:42:29  #CHECK!
%%print qa.time(me.getvalue(me.measure(me.measure(rte, 'utc'),'last'))['m0']) #16
%%#12:39:41
%%# try it another way
%%tim = me.epoch('utc', 'today')				        #17
%%tim = me.measure(tim,'tai')					#18
%%print "me.doframe(tim)=", me.doframe(tim)			#19
%%#Epoch: 54175::21:13:13.5130
%%#True
%%print "me.showframe(F) =", me.showframe(F)			#20
%%#Frame: Epoch: 54175::21:13:13.5130 (TDB = 54175.9, UT1 = 54175.9, TT = 54175.9)
%%#       Position: [-4.75092e+06, 2.79291e+06, -3.20048e+06]
%%#        (Longitude = 2.61014 Latitude = -0.526138)
%%#       Direction: [0, 0, 1]
%%#        (J2000 = [0, 90] deg)
%%print "me.ismeasure(tim) =",me.ismeasure(tim)			#21
%%#True
%%sun=me.direction('sun')					        #22
%%#rise/riseset NOT IMPLEMENTED
%%#print me.rise(sun)							#23
%%#[rise=[value=20.4413559, unit=deg], set=[value=158.042775, unit=deg]] 
%%#print me.rise(sun,'10deg') 						#24
%%#[rise=[value=27.3170864, unit=deg], set=[value=151.167045, unit=deg]] 
%%#print qa.formxxx('long', me.rise(sun).rise)				#25
%%#01:21:45.925 
%%#print qa.setformat('long','dms')					#26
%%#T 
%%#print qa.formxxx('long',me.rise(sun).rise)
%%#+020.26.28.881 
%%#
%%
%%\end{verbatim}
%%
%%\begin{enumerate}
%%   \item Get the time now as an epoch measure
%%   \item Get the position of an observatory
%%   \item Set the time in reference frame. Note that if you are working in
%%	a GUI environment, the frame set will be displayed. If working in a
%%	pure CLI environment, the current frame can be shown with the
%%	\ahlink{showframe}{measures:measures.showframe} tool function.
%%   \item Set the position in reference frame
%%   \item Get the coordinates as a measure
%%   \item Convert the coordinates to hour-angle, declination
%%   \item Get the sidereal time for now (just for the fun)
%%   \item Calculate the sines/cosines of declination and latitudes
%%   \item Get the hour-angle for zero elevation
%%   \item Get the sidereal time of setting (rising?), normalised between 0 and
%%   360 degrees
%%   \item Show it in a time format
%%   \item Save it as a time (remember, we calculated it as an angle)
%%   \item We would like to convert this sidereal time (rte) back to UTC. To be
%%   able to do that, we most now when (for which date) the sidereal time
%%   is valid. In general we do not know the Sidereal date, else it would be
%%   easy. We can however, specify an offset with a measure. Special reference
%%   codes are available (called 'raze'), which will after conversion only
%%   retain the integral part. This line says that we specify tim as an epoch in
%%   UTC, to be razed after conversion. If we now use this as an offset for a
%%   sidereal time, the offset will be automatically converted to sidereal time
%%   (since it has to be added to a sidereal time), razed, i.e. giving the
%%   Greenwich sidereal date, and hence the complete epoch is known. (Note
%%   that the indecision over the 4 minutes per day has to be handled in
%%   special cases).
%%   \item We can now define our sidereal time rt as a real epoch
%%   \item We now convert the sidereal time rte, to an
%%   UTC, and show it as a time as well
%%   \item We do not trust this funny razing, and convert the UTC obtained
%%   straight back to a sidereal time. And, indeed it worked (compare line 12)
%%   \item Another time
%%   \item Convert it to TAI (just to show how to do it)
%%   \item Frame it
%%   \item Show the current frame (the F argument indicates to not use GUI if
%%   one present) (note that the position is set to ATCA, as defined by the
%%   .aipsrc variable). If an error happens, the command
%%\begin{verbatim}
%%me.doframe(me.observatory('atca'))
%%\end{verbatim}
%%   will solve it.
%%   \item to check if we really got a measure, or an error occurred.
%%   \item define the position of the sun (the actual position will be at the
%%   frame time)
%%   \item get the sidereal time of rise/set of the sun (at the default elevation
%%   of 5 deg)
%%   \item try again for an elevation of 10 deg
%%   \item display the rise sidereal time as a 'longitude'.
%%   A variety of qa.form.x
%%   routines exist, using a global format setting mechanism (in general you
%%   have your preferred way to display something)
%%   \item change the format, and show again
%%\end{enumerate}
%%
%%The above could be done using the GUI (start it with me.gui()), setting your
%%position in the Frame menu option, and using the rise/set
%%application. Another option would be to use the direction GUI (from the Tool
%%menu in the general GUI or starting me.directiongui() from the command line);
%%selecting the Sun as planet; setting the position in the frame menu and the
%%time; Convert to J2000, and select rise/set from the Info menu.

%%If at one stage you want to use a GUI result on the command
%%line, the 'copy' button will transfer the last result to the clipboard (and
%%read it with the normal clipboard operation 
%%\begin{verbatim}
%%- dcb.paste()
%%[type=direction, refer=J2000, m1=[value=0.669116295, unit=rad], m0=[unit=rad, value=0.0259745717]] 
%%\end{verbatim}

%%\end{ahexample}

\ahobjs{}
\ahfuncs{}

\input{measures.htex}


\end{ahmodule}
