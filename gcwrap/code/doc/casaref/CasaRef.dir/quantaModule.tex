%% Copyright (C) 1999,2000,2001,2002,2003
%% Associated Universities, Inc. Washington DC, USA.
%%
%% This library is free software; you can redistribute it and/or modify it
%% under the terms of the GNU Library General Public License as published by
%% the Free Software Foundation; either version 2 of the License, or (at your
%% option) any later version.
%%
%% This library is distributed in the hope that it will be useful, but WITHOUT
%% ANY WARRANTY; without even the implied warranty of MERCHANTABILITY or
%% FITNESS FOR A PARTICULAR PURPOSE.  See the GNU Library General Public
%% License for more details.
%%
%% You should have received a copy of the GNU Library General Public License
%% along with this library; if not, write to the Free Software Foundation,
%% Inc., 675 Massachusetts Ave, Cambridge, MA 02139, USA.
%%
%% Correspondence concerning AIPS++ should be addressed as follows:
%%        Internet email: aips2-request@nrao.edu.
%%        Postal address: AIPS++ Project Office
%%                        National Radio Astronomy Observatory
%%                        520 Edgemont Road
%%                        Charlottesville, VA 22903-2475 USA
%%
%% $Id$

\begin{ahmodule}{quanta}{Units and quantities handling}

%%\ahinclude{quanta.g}

\ahkeyword{units}{}
\ahkeyword{quanta}{}
\ahkeyword{quantities}{}


\begin{ahdescription}

\bigskip

{\it Introduction}

A quantity is a value with a unit.  For example, '5km/s', or '20Jy/pc2'. 
This module (the {\mf quanta} module) enables you to create and
manipulate such quantities.  The types of functionality provided are:

\begin{itemize}
\item {\em Conversion} - Conversion of quantities  to different units
\item {\em Calculation} - Calculations with quantities
%%\item {\em GUI} A GUI interface is also provided
\end{itemize}

%%To access the {\mf quanta} module, include the the {\em quanta.g}
%%script.  This will create a default Quanta \tool\ called {\stf dq.}.  
%%
%%
%%\begin{verbatim}
%%- include 'quanta.g'
%%- dq.type()             # Default tool created for you
%%quanta
%%\end{verbatim}
%%
%%The Quanta \tool\ has no state (it doesn't remember anything), so there
%%is generally no need to create your own \tool.  Just use the default
%%one.  However, if you do require one, perhaps in a script, then use the {\cf
%%quanta} constructor to make one. 
%%
%%\begin{verbatim}
%%- include 'quanta.g';
%%- myqt := quanta();
%%- myqt.type();  
%%quanta
%%\end{verbatim}

The Quanta \tool\ manipulates quantities.  A quantity is stored as a
record with two fields.  These fields are named `value' and
`unit'.  As well as simple scalar quantites, one can also create
quantities as vectors or arrays.  For example, you may have a vector of
values, which all have the same unit - there is no need to store a copy
of the unit for each value. Access to the individual fields of a quantum
should always be by using the {\em getvalue} and {\em getunit} methods,
especially since the internal names can change or be not accessable at some
stage. 


\begin{ahexample}
\begin{verbatim}
"""
#
print "\t----\t Module Ex 1 \t----"
print qa.quantity(5.4, 'km/s')
#{'value': 5.4000000000000004, 'unit': 'km/s'}
q1 = qa.quantity([8.57132661e+09, 1.71426532e+10], 'km/s')
print qa.convert(q1, 'pc/h');
#{'value': array([ 1.,  2.]), 'unit': 'pc/h'}
#
"""
\end{verbatim}

In the first example, we make a simple scalar quantity.
You can see that the quantity (which is actually a record)
has fields `value' and `unit'.   

In the second example, we make a vector quantity and then
convert it from units of km/s to pc/h.

\end{ahexample}

\begin{ahexample}
\begin{verbatim}
"""
#
print "\t----\t Module Ex 2 \t----"
print qa.quantity('5.4km/s')          
#{'value': 5.4000000000000004, 'unit': 'km/s'}
print qa.quantity(qa.unit('5.4km/s')) 
#{'value': 5.4000000000000004, 'unit': 'km/s'}
#
"""
\end{verbatim} 

In the first example, the value and unit were combined into one string
(just saves a bit of typing).  The second example shows that the
function {\stff unit} is an alias for {\stff quantity}, and that you can
create a quantity from another quantity. 

\end{ahexample} 

\begin{ahexample}
\begin{verbatim}
"""
#
print "\t----\t Module Ex 3 \t----"
q1 = qa.unit("5s 5.4km/s")	
print len(q1)
#2
print q1['*0']
# {'unit': 's', 'value': 5.0}
print q1['*1']
# {'unit': 'km/s', 'value': 5.4000000000000004}
#
\end{verbatim} 

Here we make a vector quantity by using the string vector.  You can
see that the resultant quantity record is of length 2 and that each
field of that vector quantity is a scalar quantity.  So you see that
'q1' itself does not have fields `value' and `unit', only the
elements of 'q1' have that.

\end{ahexample} 

\begin{ahexample}
\begin{verbatim}
"""
#
print "\t----\t Module Ex 4 \t----"
q1 = qa.unit('5km');
q2 = qa.unit('200m');
print qa.canonical(qa.add(q1,q2))
#{'value': 5200.0, 'unit': 'm'}
#
"""
\end{verbatim} 

Here we make two quantities with consistent but different units,
add them together and then convert the result to canonical units.

\end{ahexample} 


\begin{ahexample}
\begin{verbatim}
"""
#
print "\t----\t Module Ex 5 \t----"
q1 = qa.quantity('6rad');
q2 = qa.quantity('3deg');
print qa.compare(q1,q2)
#True
print qa.compare(q1,qa.unit('3km'))
#False
#
"""
\end{verbatim} 

Here we compare the dimensionality of the units of two
quantities.


\end{ahexample} 


%%\begin{ahexample}
%%\begin{verbatim}-
%%
%%#
%%print "\t----\t Module Ex  \t----"
%%# q1 = qa.unit(array("5s 5.4km/s", 3, 2))
%%#len(q1)
%%#6 
%%- q1[1]
%%[value=5, unit=s] 
%%- q1[2]
%%[value=5.4, unit=km/s] 
%%- q1[6]
%%[value=5.4, unit=km/s] 
%%- q1::
%%[id=quant, shape=[3 2] ] 
%%#
%%\end{verbatim}
%%
%%Here we fill a quantity with a 2-dimensional array.  In this case, the
%%array is an array of strings.  and its shape is [3,2].  We see that the
%%resultant quantity is a vector of length 6 (it does not preserve the
%%array shape here).  However, we see that the shape attribute 
%%of the quantity does preserve the original array shape though
%%(see below for more information about attributes).
%%
%%\end{ahexample}


\bigskip
{\it Constants, time and angle formatting}

If you would like to see all the possible constants known to the Quanta
\tool\ you can issue the command {\cf print qa.map('const')}.  You can get the
value of any constant in that list with a command such as

\begin{verbatim}
"""
#
print "\t----\t Module Ex 6 \t----"
boltzmann = qa.constants('k')
print 'Boltzmann constant is ', boltzmann
#Boltzmann constant is  {'value': 1.3806577987510647e-23, 'unit': 'J/K'}
#
"""
\end{verbatim}


There are some extra handy ways you can manipulate strings when you are
dealing with times or angles.  The following list shows special strings
and string formats which you can input to the {\stff quantity} function. 
Something in square brackets is optional.  There are examples after the
list. 

\begin{itemize}
	\item time: [+-]hh:mm:ss.t... -- This is the preferred time format (trailing fields can
             be omitted)
	\item time: [+-]hhHmmMss.t..[S]  -- This is an alternative time format (HMS case
                  insensitive, trailing second fields can be omitted)
	\item angle: [+-]dd.mm.ss.t..  -- This is the preferred angle format (trailing fields
                  after second priod can be omitted; dd.. is valid)
	\item angle: [+-]ddDmmMss.t...[S] -- This is an alternative angle format (DMS case
                  insensitive, trailing fields can be omitted after M)

	\item today -- The special string ``today'' gives the UTC time at the instant the command was issued. 
	\item today/time -- The special string ``today'' plus the specified time string gives the UTC time 
              at the specified instant
	\item yyyy/mm/dd[/time] -- gives the UTC time at the specified instant
	\item dd[-]mmm[-][cc]yy[/time] -- gives the UTC time at the specified instant in calendat style notation
               (23-jun-1999)


\end{itemize}

Note that the standard unit for degrees is 'deg', and for days 'd'. 
Formatting is done in such a way that it interprets a 'd' as degrees if
preceded by a value without a period and if any value following it is
terminated with an 'm'.  In other cases 'days' are assumed.  Here are
some examples. 

\begin{verbatim}
"""
#
print "\t----\t Module Ex 7 \t----"
print qa.quantity('today')
#{'value': 54178.87156457176, 'unit': 'd'}
print qa.quantity('5jul1998')
#{'value': 50999.0, 'unit': 'd'}
print qa.quantity('5jul1998/12:')
#{'value': 50999.5, 'unit': 'd'}
print qa.quantity('-30.12.2')
#{'value': -30.200555555555557, 'unit': 'deg'}
print qa.quantity('2:2:10')
#{'value': 30.541666666666668, 'unit': 'deg'}
print qa.unit('23h3m2.2s')  
#{'value': 345.75916666666666, 'unit': 'deg'}
#
"""
\end{verbatim}

Angles and times can often be used interchangeably.  Special functions
({\cf qa.totime()} and {\cf qa.toangle()}) are available to make them in
the right units for the purpose.  E.g.  {\cf qa.sin(time)} gives an
error, whereas {\cf qa.sin(qa.toangle(time))} works ok.  See
%%\ahlink{units}{quanta:quanta.gui} for more information on units and
%%quantities, and 
the \ahlink{map}{quanta:quanta.map} function for pre-defined units.  

\begin{verbatim}
"""
#
print "\t----\t Module Ex 8 \t----"
a = qa.quantity('today');                     # 1
print a
#{'value': 54178.871564641202, 'unit': 'd'}
b = qa.toangle(a);                            # 2
print b
#{'value': 340415.88977452344, 'unit': 'rad'}
print qa.angle(qa.norm(qa.toangle(a)));       # 3
#-046.14.12
print qa.angle(qa.norm(qa.toangle(a), 0));    # 4
#+313.45.48
print qa.sub('today',a);                      # 5
#{'value': 1.1576048564165831e-08, 'unit': 'd'}
#
print "Last example! Exiting ..."
exit()
"""
\end{verbatim}

\begin{enumerate}
   \item Get the time now
   \item Get the time as an angle
   \item Get the time as a normalised angle (-pi to +pi) and show as dms
   \item Get the time as a normalised angle (0 to 2pi) and show as dms
   \item Get time since creation of a
\end{enumerate}




%%\bigskip
%%{\it Attributes}
%%
%%Quantities (that is, the record) have attributes attached to
%%them.  All quantities have the attribute called `id' which has the
%%string value `quant'.  This can be used to identify that a record is in
%%fact a valid quantity.  When you make quantities from arrays, you will
%%generally find an attribute called ``shape'' attached to the quantity
%%indicating what shape the array was.  Attributes are accessed via the
%%``::'' operator. 
%%
%%\begin{ahexample}
%%\begin{verbatim}
%%#
%%print "\t----\t Module Ex 7 \t----"
%%#- q1 = qa.unit(array("5s 5.4km/s", 3, 2))
%%#- length(q1)
%%#6
%%#- q1::
%%#[id=quant, shape=[3 2] ] 
%%#
%%\end{verbatim}
%%\end{ahexample}



\end{ahdescription}

\ahfuncs{}
\ahobjs{}

\input{quanta.htex}

\end{ahmodule}
