%% Copyright (C) 1999,2000,2001,2002,2003
%% Associated Universities, Inc. Washington DC, USA.
%%
%% This library is free software; you can redistribute it and/or modify it
%% under the terms of the GNU Library General Public License as published by
%% the Free Software Foundation; either version 2 of the License, or (at your
%% option) any later version.
%%
%% This library is distributed in the hope that it will be useful, but WITHOUT
%% ANY WARRANTY; without even the implied warranty of MERCHANTABILITY or
%% FITNESS FOR A PARTICULAR PURPOSE.  See the GNU Library General Public
%% License for more details.
%%
%% You should have received a copy of the GNU Library General Public License
%% along with this library; if not, write to the Free Software Foundation,
%% Inc., 675 Massachusetts Ave, Cambridge, MA 02139, USA.
%%
%% Correspondence concerning AIPS++ should be addressed as follows:
%%        Internet email: aips2-request@nrao.edu.
%%        Postal address: AIPS++ Project Office
%%                        National Radio Astronomy Observatory
%%                        520 Edgemont Road
%%                        Charlottesville, VA 22903-2475 USA
%%
%% $Id$
\providecommand{\briggsURL}{http://www.aoc.nrao.edu/ftp/dissertations/dbriggs/diss.html}
\providecommand{\pixonURL}{http://www.pixon.com/}
\providecommand{\rsiURL}{http://www.rsinc.com/}

\begin{ahmodule}{imager}{Module for synthesis and single dish imaging}
\ahinclude{imager.g}

\begin{ahdescription} 
\end{ahdescription}

{\tt imager} provides a unified interface for synthesis and single
dish imaging including deconvolution starting from a MeasurementSet.

\subsubsection*{What \texttt{imager} does:}

\begin{description}
\protect\item[Standard synthesis and single dish imaging] {\tt imager} does
nearly all types of synthesis and single dish imaging, including dirty
images, point spread functions, deconvolution, combination of single
dish and synthesis, spectral imaging, polarimetry, wide-field imaging,
mosaicing, holography, near-field imaging, tracking moving objects,
{\em etc.}. As a result of this extensive range of capabilities, it
can be complicated to use, especially for the more esoteric forms of
imaging.
\protect\item[Fine scaled tools] Rather than present one operation
to process data from visibilities to a restored, deconvolved
image, {\tt imager} contains a number of distinct, separate tool functions
that allow careful tuning of the processing. For example, the
weights used in imaging (the IMAGING\_WEIGHT column in the
MeasurementSet), can be altered via a number of tool functions
(\protect\ahlink{weight}{imager:imager.weight}, \protect\ahlink{filter}{imager:imager.filter}) 
and inspected via a plotting
tool function \protect\ahlink{plotweights}{imager:imager.plotweights}. Similarly, the
deconvolution and restoration steps are separate, allowing user
control of each step. It is our intention that other imaging tools may
be built on top of imager: see, for example,
\protect\ahlink{imagerwizard}{imager:imagerwizard}, which also has the side-benefit
that
it displays the imager (and other tools) commands as they are
executed.
\protect\item[Spectral imaging] \texttt{imager} can perform either spectral
imaging or frequency synthesis (producing either an image with 
each channel imaged independently or with some or 
all channels summed together).
Channels may be selected in a number of ways, either as channels
or as velocities. Also a continuum model image may be subtracted 
prior to making a cube.
\protect
\item[Many different deconvolution algorithms] {\tt imager} is rich in
deconvolution algorithms, including a number of clean variants,
maximum entropy, non-negative least squares, and the pixon algorithm.
\item[Mixing of deconvolution functions] Since the deconvolved images
are calculated and kept purely as images (rather than lists of clean
components), deconvolution functions may be mixed as desired. Thus,
one may use NNLS to deconvolve part of the Stokes I of an image, 
and then use CLEAN to deconvolve another part of all polarizations
in the image. Note that a list of clean components is not available.
\item[Ability to fix model images] In a multifield deconvolution,
it is possible to specify that some fields are not to be deconvolved,
using the {\tt fixed} argument of \ahlink{clean}{imager:imager.clean}.
\item[Single dish imaging] {\tt imager} can process single dish
observations much as it does synthesis images. To make images with no
deconvolution, use the \ahlink{makeimage}{imager:imager.makeimage}
function. This allows construction of traditional single dish images
and holography images. To deconvolve images, just use the
``multifield'' deconvolution algorithms in
\ahlink{clean}{imager:imager.clean} and 
\ahlink{mem}{imager:imager.mem}. You will want to set the gridmachine
in \ahlink{setoptions}{imager:imager.setoptions} to 'sd'.
\item[Combination of single dish and synthesis data] If the single
dish and interferometer data are in the same MeasurementSet, then
imager can perform a joint deconvolution using
``multifield'' deconvolution algorithms in
\ahlink{clean}{imager:imager.clean} and 
\ahlink{mem}{imager:imager.mem}. You will want to set the gridmachine
in \ahlink{setoptions}{imager:imager.setoptions} to 'both'. You can
change the relative weighting of synthesis and single dish data by
using \ahlink{setsdoptions}{imager:imager.setsdoptions}.
If the single dish and synthesis data cannot be combined into
one MeasurementSet then you can still use the
\ahlink{feather}{imager:imager.feather} function to combine already
deconvolved images.
\item[Multi-field processing] {\tt imager} can be run on any number
of images, each of which can have any direction for the phase
center. All coordinate transformations are done correctly.  Using the
measures system, these fields may be given moving positions (such as
the Sun using {\tt dm.direction('sun')} to specify the phase center) or
positions in strange coordinates (such as Supergalactic using
{\em e.g.} {\tt dm.direction('supergal', '0d', '0d')} as well as
the more conventional representations ({\em e.g.}
{\tt dm.direction('b1950', '12h26m33.248000', '02d19m43.290000')} specifies
the coordinates of the core of 3C273). Note that for
some coordinate systems a location must be supplied via
the \ahlink{setoptions}{imager:imager.setoptions} tool function. For example,
one can put an image at a specific azimuth and elevation ({\em e.g.}
{\tt dm.direction('azel', '67.4d', '5.23d')}) at the
VLA {\tt imgr.setoptions(location=dm.observatory('VLA'))}. Phase 
rotation will be automatically calculated to track in 
azimuth-elevation.
\item[Wide-field imaging] {\tt imager} can perform wide-field
imaging as needed to overcome the non-coplanar baselines
effect for the VLA and other non-coplanar arrays.
\item[Mosaicing] {\tt imager} can perform clean-based or mem-based mosaicing
of many pointings into one image, using variants of the multi-field
algorithms.
\item[Processing of component lists] Discrete components
(not the same as clean components!) can be represented by 
\ahlink{componentmodels}{componentmodels}. A componentlist
can hold any number of components. The components are subtracted from
the visibility data before construction of an image. For high
precision imaging, it is recommended that components be used for
bright sources since the Fourier Transform of components avoids the
limitations of the gridded transforms.
%\item[Self-calibration] Self-calibration is accomplished
%using the \ahlink{calibrater}{calibrater} tool. One sets up a calibrater tool
%with the necessary parameters and then passes it by name to
%the \ahlink{selfcal}{imager:imager.selfcal} tool function.
\item[Joint deconvolution of Stokes IQUV] {\tt imager} can produce
images of either $I$ alone, or $I,V$ or $I,Q,U,V$, deconvolving
jointly as appropriate. The point spread function is constrained to
be the same for all processed polarizations so asymmetric u,v coverage is
not allowed.
\item[Production of complex images] {\tt imager} can produce
dirty or residual images or point spread functions in the original 
data representation ({\em e.g.} RR,RL,RL,LL or XX,XY,YX,YY. 
\item[Fine control and evaluation of visibility weighting] Various
tool functions for controlling the visibility weights are available
(\ahlink{weight}{imager:imager.weight}, \ahlink{filter}{imager:imager.filter}) 
as well as tool functions for evaluating the effects of the weighting
(\ahlink{plotweights}{imager:imager.plotweights},
\ahlink{sensitivity}{imager:imager.sensitivity},
\ahlink{fitpsf}{imager:imager.fitpsf}).  The Briggs algorithm for 
weighting of visibility data can be used (see
\ahlink{weight}{imager:imager.weight} and \htmladdnormallink{Dan Briggs'
thesis} {\briggsURL}).
\item[Flexible windowing in the deconvolution] Rather than use boxes
to limit the region CLEANed, a mask image is used to constrain the
region in which flux is allowed. There are various tool functions for making
a mask image, including from regions and blc/trc specifications, and via
thresholding the Stokes I image. In the Clark Clean, the mask is
{\em soft}: it can vary between 0 and 1. Intermediate values of
the mask bias against but do not rule out subtraction of clean components.
\item[Non-Negative Least Squares Deconvolution] This algorithm
is very effective at producing high dynamic range images of moderately
resolved sources (see \htmladdnormallink{Dan Briggs' thesis}
{\briggsURL}). It works on Stokes $I$ alone so the recommended
procedure is to CLEAN $I,Q,U,V$ using \ahlink{clean}{imager:imager.clean}
and then use NNLS to refine the $I$ part of the image using
\ahlink{nnls}{imager:imager.nnls}.
%\item[Specification of arguments as \ahlink{measures}{measures} ]
\item[Specification of arguments as] A measure 
is a measured quantity with optional units, coordinates and reference
frames. These are allowed in a number of circumstances.  The advantage
is that the user can specify arguments in very convenient form, and
let the measures system do whatever conversion is required. For
example:
\begin{description}
\item[Cell sizes] These can be specified as a quantity (see the 
\ahlink{measures}{measures} module).
\begin{verbatim}
imgr.setimage(cellx='7arcsec', celly='7arcsec')
\end{verbatim}
\item[Image center direction] This must be specified as a 
direction (see the \ahlink{measures}{measures} module).
\begin{verbatim}
imgr.setimage(phasecenter=dm.direction('j2000', '05h30m', '-30.2deg'))
imgr.setimage(phasecenter=dm.direction('gal', '0deg', '0deg'))
imgr.setimage(phasecenter=dm.direction('mars'))
imgr.setimage(phasecenter=image('myother.image').coordmeasures().direction);
\end{verbatim}
\item[Velocities] These can be specified as radial velocities.
\begin{verbatim}
imgr.setimage(start=dm.radialvelocity('25km/s'), 
            step=dm.radialvelocity('-500m/s'))
\end{verbatim}
\item[Position] For construction of images in some coordinate
frames ({\em e.g} azimuth-elevation) the position to be used
in processing must be set:
\begin{verbatim}
imgr.setoptions(location=dm.observatory('ATCA'))
\end{verbatim}
\end{description}
\item[More choice in image size] Any even image size will work,
though to speed the FFT, it is advisable to
use a highly composite number (one that has many factors).
The \ahlink{advise}{imager:imager.advise} function will
calculate an acceptable number.
\item[Integrated plotting] Plots of visibility 
amplitude, weights (both point-by-point and gridded),
uv coverage, and field and spectral window ids are available 
(\ahlink{plotvis}{imager:imager.plotvis}, 
\ahlink{plotweights}{imager:imager.plotweights}, 
\ahlink{plotuv}{imager:imager.plotuv}, 
\ahlink{plotsummary}{imager:imager.plotsummary}).
\item[Synchronous or Asynchronous processing] Operations that
take a substantial amount of time to run can be run in the
background either by setting the global variable {\tt dowait:=F}
or by setting an argument {\em e.g.} {\tt imgr.clean(async=T)}.
To retrieve a result, use the result tool function of
defaultservers with the job number as the argument. For example:
\begin{verbatim}
- imgr:=imager('ss433.MS')
T
- imgr.setimage(cellx='0.05arcsec', celly='50marcsec', nx=256, ny=256, 
  spwid=1:2, fieldid=1, stokes='IV') 
T 
- imgr.fitpsf()
1 
# Wait for it to finish and then ask for the result:
- defaultservers.result(1)
[psf=, bpa=[value=42.7269936, unit=deg], bmin=[value=0.13008301, 
unit=arcsec], bmaj=[value=0.159367442, unit=arcsec]] 
\end{verbatim}
\item[A novel sort-less gridding algorithm] The visibility data are
not sorted before the gridding step. Instead, a cache of tiles is
allocated to hold each baseline as it moves around in the Fourier
plane. When a baseline moves off an existing tile, the results are
written to disk and the necessary new tile is read in. Since the
rotation of baselines in the uv plane is usually quite slow, the hit
rate of such a cache is high. The size of the cache is by default
set to half the physical memory of the machine, as specified
by the aipsrc variable system.resources.memory. This can be overridden
by the user, via the \ahlink{setoptions}{imager:imager.setoptions}
tool function. The cache
can be made smaller at the expense of more paging of tiles in and
out. The tile size can also be changed but this is seldom needed.
This approach is optimal for arrays with small numbers of antennas
but can be slow for {\em e.g.} the VLA. We intend to rectify this
in the near future.
\item[Plug-in commands] {\tt imager} can be customized by attaching
commands using the \aipspp\ plug-in system. See the file
code/trial/apps/imager/imager\_standard.gp for an example of how to 
attach commands. 
\item[Suite of tests] {\tt imager} has a suite of tests. A standard 
test data set and component list can also be created.
\item[imagerwizard] The \ahlink{simplemage}{imager:imagerwizard}
function is a wizard that performs interactively guided imaging
of synthesis data.
\item[dragon] The \ahlink{dragon}{imager:dragon} tool performs 
wide-field imaging using imager.
\item[vpmanager] The \ahlink{vpmanager}{imager:vpmanager} tool manages
specification of primary beams for imager.
\item[Near-field imaging {\em experimental}] Images of objects in the near-field of an
array can be made. If the distance to the object is specified 
in \ahlink{setimage}{imager:imager.setimage}, then the extra delay
due to the wavefront curvature is corrected in the transforms.
Note that some telescopes ({\em e.g.} VLA) make this correction in the real-time
system. This effect is important if the distance to the object is comparable
to or less than:
\begin{equation}
{B^2\over\lambda}
\end{equation}
where B is the baseline. Note that the sign of the correction could be
in error in this experimental version: try using a negative distance
as well as a positive distance.
\end{description}

\subsubsection*{What {\tt imager} needs:}

{\tt imager} operates on a specified MeasurementSet to produce any of a
range of different types of image: dirty, point spread function,
clean, residual, {\em etc.}  A MeasurementSet is the holder for
measurements from a telescope. It is simply an \aipspp\ Table obeying
certain conventions as to required and optional contents. The
intention is that it should contain all the information needed to
reduce synthesis and single dish observations (see
\htmladdnormallink{\aipspp\ Note 191} {../../notes/191.ps}). A
UVFITS file can be converted to a MeasurementSet using the
\ahlink{fitstoms}{ms:ms.fitstoms.constructor} tool function (a constructor of the
\ahlink{ms}{ms} tool.

{\tt imager} adds some extra columns to the MeasurementSet to store
results of processing. The following columns in the MS are
particularly important:
\begin{description}
\item[DATA] The original observed visibilities are in a column
called DATA. These are not altered by any processing in \aipspp.
\item[CORRECTED\_DATA] During a calibration process, as carried out by
{\em e.g.} \ahlink{calibrater}{calibrater}, the visibilities may be corrected for
calibration effects. This corrected visibilities are stored in a column
CORRECTED\_DATA which is created on demand by calibrater and imager. In creating
the CORRECTED\_DATA column, {\tt imager} will only correct for parallactic
angle rotation. This can be controlled using the
\ahlink{correct}{correct} tool function. All imaging performed by
imager is from the CORRECTED\_DATA column (apart from the
tool function \ahlink{makeimage}{imager:imager.makeimage} which can also make dirty
images from the other visibility columns).
\item[MODEL\_DATA] During various phases of processing, the
visibilities as predicted from some model are required. These 
model visibilities are stored in a column MODEL\_DATA. 
These are used by the \ahlink{calibrater}{calibrater} tool for 
calibration.
\item[IMAGING\_WEIGHT] Weighting of data (including natural,
uniform and Briggs weighting,
and tapering) is accomplished by setting the column IMAGING\_WEIGHT
appropriately. 
\end{description}
Standard tools such as the \ahlink{table}{table}
module and the \ahlink{ms}{ms} can be used to access and possibly
change these (and all other) columns.

{\tt imager} can handle an initial model in a number of forms: as an
image, as a list of images, as a \ahlink{componentmodels:componentlist}
{componentmodels:componentlist}, or as some combination. Fitting of
{\tt componentmodels} is planned but is not currently supported.

{\tt imager} uses a number of scratch files. Following \aipspp\
practice, these are placed in the directories specified in the
aipsrc variable user.directories.work.
Those disks that possess sufficient free disk space are
chosen in sequence. So to spread your scratch files over
two disks each of which has a directory tcornwel/tmp do {\em
e.g.}

\begin{verbatim}
user.directories.work:   /bigdisk1/tcornwel/tmp /bigdisk2/tcornwel/tmp
\end{verbatim}

\subsubsection*{How to control imager:}

To use {\tt imager}, one has to construct a {\tt imager} tool using
a MeasurementSet as an argument, for example:

\begin{verbatim}
myimager:=imager('3C273XC1.ms')
\end{verbatim}

The Glish variable {\tt myimager} then contains the tool functions
that may be used to do various operations on the MeasurementSet
3C273XC1.ms. These tool functions can be broken down into those that
set {\tt imager} up in some state, and those that actually do some
processing. The setup tool functions are: 

\begin{description}
\item[setimage] is a {\em required} tool function that defines
the parameters (size, sampling, phase center, {\em etc.}) of
any image that is to be constructed. If you omit to call
setimage prior to any operation that needs these parameters,
an error message will result. setimage is passive: nothing 
happens immediately but subsequent processing is altered.
\item[setdata] is an {\em optional} tool function that selects which data
are to be operated on during the processing. This selection can
consist of choosing the spectral windows or fields that are
to be operated on, or setting channels that are to be operated
on in subsequent processing. setdata is active: the selection
occurs immediately and is effective for all subsequent operations
(until setdata is called again).
\item[setoptions] is an {\em optional} tool function that sets parameters
of lesser importance such as gridding parameters, cache sizes. While
these affect the processing, usually the default values will
suffice. setoptions is passive: nothing happens immediately but
subsequent processing is altered.
\item[setbeam] is an {\em optional} tool function that sets the
parameters of the synthesized beam to be used in restoring deconvolved
images. setbeam is passive: nothing happens immediately but
subsequent processing is altered.
\item[setvp] is an {\em optional} tool function that sets the
parameters of the voltage pattern model used in mosaicing.
setvp is passive: nothing happens immediately but
subsequent processing is altered.
\item[setsdoptions] is an {\em optional} tool function that sets the
relative scaling and weighting of single dish data versus 
interferometer data and also other single dish specific parameters like the convolution support when doing single dish imaging.
\end{description}

Thus to understand what imager is doing, one has to remember that at
any time, it has a {\em state} that has been set by using these
tool functions. The state may be viewed in one of two ways: either 
\ahlink{summary}{imager:imager.summary} can be used to output the
current state to the logger, or, in the GUI, the current
state of these parameters is displayed and updated following any
relevant changes. 

All the other tool functions of {\tt imager} are active: something happens
immediately. Hence, for example, the \ahlink{weight}{imager:imager.weight}
tool function acts immediately to change the weighting of the selected
data. In particular, unlike other packages, it does {\em not} set the
weighting parameters for latter operations. The
\ahlink{clean}{imager:imager.clean} tool function performs a clean deconvolution of
an image, reading and writing a model image. Note that operations that
require or produce an image usually take an appropriate image name in
the argument list.  Often if such an image is not given then it is
constructed using the image parameters set via
\ahlink{setimage}{imager:imager.setimage} and using an appropriate name
({\em e.g.} a restored image is named from the model image by
appending .restored so that 3C273XC1.clean becomes
3C273XC1.clean.restored).

The concept of the {\em state} of imager bears a little more
explanation.  The MeasurementSet can potentially contain data for many
different fields and spectral windows. One therefore has to have some
way of distinguishing which data are to be included in
processing. Rather than have each possible tool function ({\em e.g.}
weight, image, clean) take a long list of parameters to determine
which data are to be included, {\tt imager} has a
\ahlink{setdata}{imager:imager.setdata} tool function that sets {\tt
imager} up some that in subsequent processing only the selected data
are processed. For example, to select only field id 1 and spectral
windows 1 and 2, one would do:

\begin{verbatim}
myimager.setdata(fieldid=1, spwid=1:2)
\end{verbatim}

The state of {\tt imager} also consists of information about the
default image settings (set via \ahlink{setimage}{imager:imager.setimage})
and various less important options (set via 
\ahlink{setoptions}{imager:imager.setoptions}).

\subsubsection*{What {\tt imager} produces:}

{\tt imager} reads and writes \aipspp\ MeasurementSets and Images.  The
format of images is 4 dimensional, with the first two being right
ascension and declination, the third being polarization and the fourth
being frequency. By suitable choice of the input parameters, one can
make images of $I$ alone, $I,V$ or $I,Q,U,V$ for one or all channels.
The \ahlink{makeimage}{imager:imager.makeimage} tool function can also make a complex image
of the original polarizations {\em e.g.} RR, RL, LR, LL. This latter
type of image is useful for diagnostic purposes.

Images generated by {\tt imager} may be viewed using
\ahlink{viewer}{viewer:viewer} tool or retrieved using the
\ahlink{images}{images} tool, the MeasurementSets may be accessed
using the \ahlink{ms}{ms} tool. More on this in the example below.

\subsubsection*{What {\tt imager} does not do:}

{\tt imager} does not handle calibration of visibility data beyond correction
for parallactic angle variations.  Instead, you should use the
\ahlink{calibrater}{calibrater} tool for this purpose.  However, {\tt imager} and calibrater can
cooperate on the self-calibration of data. 
%using the tool function
%\ahlink{imager.selfcal}{imager:imager.selfcal}.

\subsubsection*{What improvement to {\tt imager} are in the works:}

We are currently working on a number of improvements:

\begin{itemize}
\item Improved gridder to handle many telescopes and many channels
more efficiently.
\item Parallelized CLEAN and gridding
\end{itemize}

\subsubsection*{Advanced use of imager:}

As with all \aipspp\ applications, {\tt imager} is designed to be open: all
the results are written to and read from standard \aipspp\ table files.
This open design of {\tt imager} also allows the user to try out new
methods of processing data.  Models may be read into Glish, edited or
manipulated via standard Glish facilities, and then written out and
used subsequently in {\tt imager}.  Suppose that we want to halve the
Stokes I of all pixels with negative Stokes I. The following Glish
fragment does the trick:
\begin{verbatim}
m:=image('myimage')
shape:=im.shape()
blc:=[1,1,1,1]
trc:=[shape[1],shape[2],1,shape[4]]
a:=m.getchunk(blc,trc)
a[a<0.]*:=0.5
m.putchunk(a,blc)
m.flush()
m.close()
\end{verbatim}

\protect\subsubsection*{Overview of {\tt imager} tool functions:}

\protect\begin{description}
\protect\item[Data access] \ahlink{open}{imager:imager.open},
\ahlink{close}{imager:imager.close}, \ahlink{done}{imager:imager.done}
\protect\item[Data selection] \ahlink{setdata}{imager:imager.setdata}
\protect\item[Data editing] \ahlink{clipvis}{imager:imager.clipvis}
\protect\item[Data calibration] \ahlink{correct}{imager:imager.correct}
\protect\item[Data examination] \ahlink{plotvis}{imager:imager.plotvis}, \ahlink{plotuv}{imager:imager.plotuv}, 
\ahlink{plotweights}{imager:imager.plotweights}, \ahlink{plotsummary}{imager:imager.plotsummary}
\protect\item[Weighting] \ahlink{weight}{imager:imager.weight}, \ahlink{filter}{imager:imager.filter}, 
\ahlink{uvrange}{imager:imager.uvrange}, \ahlink{sensitivity}{imager:imager.sensitivity}, 
\ahlink{fitpsf}{imager:imager.fitpsf}, \ahlink{plotweights}{imager:imager.plotweights}
\protect\item[Image definition] \ahlink{advise}{imager:imager.advise}, \ahlink{setimage}{imager:imager.setimage}, \ahlink{make}{imager:imager.make}
\protect\item[Imaging] \ahlink{makeimage}{imager:imager.makeimage},
\ahlink{clean}{imager:imager.clean},
\ahlink{nnls}{imager:imager.nnls}, \ahlink{mem}{imager:imager.mem}, 
\ahlink{pixon}{imager:imager.pixon}, 
\ahlink{restore}{imager:imager.restore},
\ahlink{residual}{imager:imager.residual}, 
\ahlink{approximatepsf}{imager:imager.approximatepsf}, 
\ahlink{fitpsf}{imager:imager.fitpsf}, 
\ahlink{setbeam}{imager:imager.setbeam}, 
\ahlink{ft}{imager:imager.ft}, 
\ahlink{smooth}{imager:imager.smooth},
\ahlink{feather}{imager:imager.feather},
\ahlink{makemodelfromsd}{imager:imager.makemodelfromsd}
\protect\item[Masks] \ahlink{mask}{imager:imager.mask},
\ahlink{boxmask}{imager:imager.boxmask},
\ahlink{regionmask}{imager:imager.regionmask},
\ahlink{exprmask}{imager:imager.exprmask},
\ahlink{clipimage}{imager:imager.clipimage}
\protect\item[Miscellaneous] \protect\ahlink{summary}{imager:imager.summary}, 
\protect\ahlink{setoptions}{imager:imager.setoptions},
\protect\ahlink{setoptions}{imager:imager.setmfcontrol},
\protect\ahlink{setoptions}{imager:imager.setsdoptions}
\end{description}



\begin{ahexample}
The following example shows the quickest way to make a CLEAN image and
display it. Note that this can be more easily done from the
\ahlink{toolmanager}{tasking:toolmanager}.
\begin{verbatim}
include 'imager.g'
#
# First make the MS from a FITS file:
#
m:=fitstoms(msfile='3C273XC1.MS', fitsfile='3C273XC1.FITS'); m.close();
#
# Now make an imager tool for the MS
#
imgr:=imager('3C273XC1.MS')      
#
# Set the imager to produce images of cellsize 0.7 and
# 256 by 256 pixels
#
imgr.setimage(nx=256,ny=256, cellx='0.7arcsec',celly='0.7arcsec');
#
# Wait for results before proceeding to the next step
#
dowait:=T
#
# Make and display a clean image
#
imgr.clean(niter=1000, threshold='30mJy',
model='3C273XC1.clean.model', image='3C273XC1.clean.image')
dd.image('3C273XC1.clean.image')
#
# Fourier transform the model 
#
imgr.ft(model='3C273XC1.clean.model')
#
# Plot the visibilities
#
imgr.plotvis()
#
# Write out the final MS and close the imager tool
#
imgr.close()
\end{verbatim}
\end{ahexample}
\input{imager.htex}
\documentclass[12pt]{article}
%\documentstyle[12pt,amsmath]{article}
%\usepackage{html}
\usepackage{epsf}
\usepackage{amsmath}
\usepackage[dvips]{graphicx, color}  % The figure package
\usepackage{palatino}
\usepackage{natbib}     % Package used for bib. citation
\usepackage{txfonts}

\setlength{\textwidth}{15.00cm}
\setlength{\oddsidemargin}{0.75cm}
\setlength{\evensidemargin}{0.5cm}

\pagestyle{myheadings}

\begin{document}
\title{The CASA vpmanager tool\\ and the VPManager class}
\author{D. Petry (ESO, Garching)}
\date{Jan 9, 2012\\%{\small (Updated: )}
}
\maketitle
\normalsize
\markboth{Antenna responses in CASA and the vpmanager tool}{D. Petry}
%\begin{center}
%  \htmladdnormallinkfoot{PDF
%    Version}{http://www.aoc.nrao.edu/~sbhatnag/misc/msselection.pdf}
%\end{center}

\begin{abstract}
\noindent
The VPManager class in the synthesis module of CASA code permits 
CASA imaging and simulation routines to hook up to the vpmanager and let it determine
which antenna responses to use for which observatory.

The vpmanager tool (by default "vp" in casapy) is the Python user interface to the VPManager
class. It has methods to set up a list of primary beams or voltage patterns
({\it antenna responses}) and then select in detail 
which of them is used for which observatory.
The distinction of several antenna types for a given observatory (heterogeneous
arrays) is supported.
 
Antenna responses can be selected from either internally 
hard-coded ones, or response-groups defined via an AntennaResponses table, or
user-defined analytic primary beams.
\end{abstract}

\section{Overview}

The vpmanager tool is the CASA Python object which constitutes the
user interface to the VPManager C++ class. By default it is named "vp" in casapy.
This document describes
the upgraded version of the tool and class to be included in CASA 3.4.

The VPManager class is implemented as a singleton, i.e. internally there is only one instance
at all times. This instance accessed via the static VPManager::Instance() method. It is permanent 
until casapy is exited or it is reinstantiated via the static VPManager::reset() method.

The vp tool connects to the single instance of VPManager.
All settings the user makes with the tool, have effect immediately and are then used
by all parts of CASA which access the VPManager class (i.e. eventually all imaging and simulation
routines).

In order to enable parallelization, a simple {\it locking mechanism} protects from simultaneous access
to the VPManager. Only the application programmer has to be aware of this. On the tool level
the locking is done automatically.

The VPManager instance keeps a simple database of available antenna responses, the {\it vplist}.
This list is initialized at the startup of CASA or by calling the reset() method
of the class. In the vp tool, the reset call can be triggered using
\begin{verbatim}
   vp.reset()
\end{verbatim}

In order to support heterogeneous interferometer arrays, VPManager permits the use
of {\it antenna types} in addition to observatory or {\it telescope} names.

For defining a simple response which is only spatially scaled by frequency but otherwise
constant, a simple call to the vp tool is sufficient, e.g.:
\begin{verbatim}
        vp.setpbairy(telescope='ALMA',
                     dishdiam='12.0m',
                     blockagediam='0.75m',
                     maxrad='1.784deg',
                     reffreq='1.0GHz',
                     dopb=True)
\end{verbatim}
This will create a new entry in the vplist for an analytic Airy disk antenna response
and make it the default response for telescope "ALMA".
Subsequent requests to VPManager for a ALMA antenna response will get this Airy disk.
Using the {\tt vp.setuserdefault} method, the default can be unset again or changed
to a different entry in the vplist.

If whole {\it response systems} are to be defined for a given telescope, the use of an
{\it AntennaResponses table} is possible. Such a table can be set up using the 
vp tool method
{\tt createantresp()} 
and then connected to a telescope using a command like
\begin{verbatim}
vp.setpbantresptable(telescope='ALMA',
                     antresppath=casa['dirs']['data']
                     +'/alma/responses/AntennaResponses-ALMA-RT',
                     dopb=True)
\end{verbatim}
where the value of the {\tt antresppath} parameter indicates the path to the AntennaResponses table.
Subsequent requests for ALMA antenna responses to VPManager will start a search in the 
indicated AntennaResponses table for responses matching given parameters.
Presently supported search parameters in VPManager::getvp() and vp.getvp() are:
\begin{itemize}
\item antenna type
\item observation time (used for versioning and for reference frame transformations)
\item frequency (as a Measure, the reference frame is respected)
\item observing direction (to support elevation and azimuth dependent responses)
\end{itemize}
An example of a call to vp.getvp() is
\begin{verbatim}
        myrecord = vp.getvp(telescope='ALMA',
                            antennatype = 'DV',
                            obstime = '2009/07/24/10:00:00',
                            freq = 'TOPO 100GHz',
                            obsdirection = 'AZEL 30deg 60deg')
\end{verbatim}

If the default antenna response for the given telescope is not defined via an AntennaResponses
table, the observation parameters {\tt obstime}, {\tt freq}, and {\tt obsdirection} are not needed 
and can be omitted. The parameter {\tt antennatype} defaults to empty string. So if no antenna types
are distinguished for the given telescope, the simplest call to getvp becomes
\begin{verbatim}
       myrecord = vp.getvp(telescope='HATCREEK')
\end{verbatim}

During initialization, VPManager will look for entries in the column "AntennaResponses"
of the CASA "Observatories" table. If there are non-blank entries, the string found will be
interpreted as the path to the default AntennaResponses table for the given telescope. 

Note that the casacore AntennaResponses C++ class (which is used by VPManager to
administrate the AntennaResponses tables) also supports the additional search parameters
"receiver type" and "beam number". A general interface to the response file name search
is available through the vp.getrespimagename() method. But presently this accesses only
AntennaResponse tables which are entered as the default table in the Observatories table.

Generally, the vp tool methods provide functionality:
to set up new analytic antenna responses,
select which antenna responses from the vplist to use for which telescope and antenna type, 
access the contents of the vplist,
create and access an AntennaResponses table,
create and load a voltage pattern table.

The latter is achieved with the methods vp.saveastable() and vp.loadfromtable() which
can be used to save the vplist and the defaults in a CASA table and reload
them at a later time.

In section \ref{secvptoolmethods}, the methods are described in more detail.
The appendix \ref{appex} gives example Python scripts.
The appendices \ref{appVPMan} and \ref{appVPManex} show the methods of the VPManager C++ class
and give an example of how to use it to get a primary beam in application code.
But beforehand, section \ref{seclib} describes the library of predefined responses 
available to the vpmanager and section \ref{secvp} describes the use of a voltage pattern table.
Finally, section \ref{secrt} describes the access to ray traced responses for ALMA.

\section{The antenna responses library accessible to the vp tool}
\label{seclib}

\subsection{Common antenna responses}

Many common voltage pattern (vp) and primary beam (pb)
models have been coded into CASA.  Currently, the recognized models include
DEFAULT, ALMA, ACA, ATCA\_L1, ATCA\_L2, ATCA\_L3, ATCA\_S, ATCA\_C, ATCA\_X, GBT, GMRT,
HATCREEK, NRAO12M, NRAO140FT, OVRO, VLA, VLA\_INVERSE, VLA\_NVSS,
VLA\_2NULL, VLA\_4, VLA\_P, VLA\_L, VLA\_C, VLA\_X, VLA\_U, VLA\_K, VLA\_Q,
WSRT, and WSRT\_LOW.  In all cases, the VP/PB model and the beam squint (if
present) scale linearly with wavelength.  If DEFAULT is selected, the
appropriate VP/PB model is selected for the telescope and observing frequency.

\subsection{1-D Beam Models} 
Most beam models are rotationally symmetric (excepting
beam squint).  From the beam parameterization in terms of the various
coefficients and other terms, an internal lookup table with 10000 elements is
created for application of the VP/PB to an image.

\subsection{Beam Squint} The VP/PB models include beam squint.  The VLA\_L,
VLA\_C, VLA\_X, VLA\_U, VLA\_K, and VLA\_Q models (which are the defaults for
those VLA bands), have the appropriate squint magnitude and orientation, though
the orientation has not been verified through processing actual data.

\section{The voltage pattern table}
\label{secvp}

In the original design of the vpmanager (before the refactoring in 2011),
the only way to communicate with the imaging routines was via the so-called
voltage pattern table. This functionality still exists in the present tool.
However, it is recommended that all CASA code now use the VPManager::getvp()
methods and access the VPManager::Instance() directly.

A new voltage pattern table can be generated by the vp tool using the vp.saveastable()
method.  The vp description table can then be read by imager's
imager.setvp() method, which instantiates the
corresponding voltage patterns from the descriptions and applies them
to the images.

\begin{verbatim}
#
# Lets say we want an Airy Disk voltage pattern for our
# HATCREEK data, but we want to use the system default
# for the OVRO data:
#
vp.setpbairy(telescope='HATCREEK', dopb=T, dishdiam='6.0m',
             blockagediam='0.6m',  maxrad='2arcmin',
             reffreq='100GHz', dosquint=F)
#
vp.setcannedpb(telescope='OVRO', dopb=T, commonpb='DEFAULT', dosquint=F)
#
vp.summarizevps()
#
vp.saveastable(tablename='California.Beaming')
#
\end{verbatim}

The voltage pattern table created by vp.saveastable() can also
be loaded back into the vpmanager thereby restoring a previous state
using the method vp.loadfromtable().

\section{The vp tool methods}
\label{secvptoolmethods}

For an up-to-date reference of the individual method parameters, please use help vp.{\it method}
within casapy. 
 
\begin{description}
\item[vpmanager]
   Construct a vpmanager tool (note: the underlying VPManager is a singleton).
   The vpmanager constructor has no arguments.
 
  \item[saveastable]
Save the vp or pb descriptions as a table.  Each description is in a different
row of the table. (bool)
{\small
\begin{verbatim}
--- --- --- --- --- --- Parameters  --- --- --- --- --- ---
  tablename:  Name of table to save vp descriptions in 
--- --- --- --- --- --- --- --- --- --- --- --- --- --- ---
\end{verbatim} 
}
 
\item[loadfromtable]
Load the vp or pb descriptions from a table (deleting all previous definitions) (bool)
{\small
\begin{verbatim}
    --- --- --- --- --- --- Parameters  --- --- --- --- --- ---
      tablename:  Name of table to load vp descriptions from 
    --- --- --- --- --- --- --- --- --- --- --- --- --- --- --- 
\end{verbatim} 
}


  \item[summarizevps]
Summarize the currently accumulated VP descriptions to the logger. (bool)
{\small
\begin{verbatim}
--- --- --- --- --- --- Parameters  --- --- --- --- --- ---
  verbose:  Print out full record? Otherwise, print summary. false 
--- --- --- --- --- --- --- --- --- --- --- --- --- --- ---
\end{verbatim} 
} 
 
  \item[setcannedpb]
   Select a vp/pb from our library of common pb models
   If 'DEFAULT' is selected, the system default for that telescope and frequency is used. (record)
{\small
\begin{verbatim}
--- --- --- --- --- --- Parameters  --- --- --- --- --- ---
  telescope:  Which telescope in the MS will use this vp/pb? VLA 
  othertelescope:  If telescope=="OTHER", specify name here 
  dopb:  Should we apply the vp/pb to this telescope's data? true 
  commonpb:  List of common vp/pb models: DEFAULT code figures it out DEFAULT 
  dosquint:  Enable the natural beam squint found in the common vp model false 
  paincrement:  Increment in Parallactic Angle for asymmetric (i.e., squinted) 
                vp application 720deg 
  usesymmetricbeam:  Not currently used false 
--- --- --- --- --- --- --- --- --- --- --- --- --- --- ---
\end{verbatim} 
} 

 
  \item[setpbairy]
   Make an airy disk vp.   
   Information sufficient to create a portion of the Airy disk voltage pattern.
   The Airy disk pattern is formed by Fourier transforming a uniformly illuminated
   aperture and is given by
   \begin{equation}
     vp_p(i) = ( areaRatio * 2.0 * j_{1}(x)/x 
     - 2.0 * j_{1}(x*lengthRatio)/(x*lengthRatio) )/ areaNorm,
   \end{equation}
   where areaRatio is the dish area divided by the blockage area, lengthRatio
   is the dish diameter divided by the blockage diameter, and 
   \begin{equation}
     x = \frac{i * maxrad * 7.016 * dishdiam/24.5m}{N_{samples} * 1.566 * 60}.
   \end{equation}
   (record)
 
{\small
\begin{verbatim}
--- --- --- --- --- --- Parameters  --- --- --- --- --- ---
  telescope:  Which telescope in the MS will use this vp/pb? VLA 
  othertelescope:  If telescope=="OTHER", specify name here 
  dopb:  Should we apply the vp/pb to this telescope's data? true 
  dishdiam:  Effective diameter of dish 25.0m 
  blockagediam:  Effective diameter of subreflector blockage 2.5m 
  maxrad:  Maximum radial extent of the vp/pb (scales with 1/freq) 0.8deg 
  reffreq:  Frequency at which maxrad is specified 1.0GHz 
  squintdir:  Offset (Measure) of RR beam from pointing center, azel frame 
              (scales with 1/freq) 
  squintreffreq:  Frequency at which the squint is specified 1.0GHz 
  dosquint:  Enable the natural beam squint found in the common vp model false 
  paincrement:  Increment in Parallactic Angle for asymmetric (i.e., squinted) 
                vp application 720deg 
  usesymmetricbeam:  Not currently used false 
--- --- --- --- --- --- --- --- --- --- --- --- --- --- ---
\end{verbatim} 
} 


  \item[setpbcospoly]
   Make a vp/pb from a polynomial of scaled cosines.
   A voltage pattern or primary beam of the form
   \begin{equation}
     VP(x) = \sum_{i} ( coeff_{i} \cos^{2i}( scale_{i} x).
   \end{equation}
   This is a generalization of the WSRT primary beam model. (record)
{\small
\begin{verbatim}
--- --- --- --- --- --- Parameters  --- --- --- --- --- ---
  telescope:  Which telescope in the MS will use this vp/pb? VLA 
  othertelescope:  If telescope=="OTHER", specify name here 
  dopb:  Should we apply the vp/pb to this telescope's data? true 
  coeff:  Vector of coefficients of cosines -1
  scale:  Vector of scale factors of cosines -1 
  maxrad:  Maximum radial extent of the vp/pb (scales with 1/freq) 0.8deg 
  reffreq:  Frequency at which maxrad is specified 1.0GHz 
  isthispb:  Do these parameters describe a PB or a VP? PB 
  squintdir:  Offset (Measure) of RR beam from pointing center, azel frame 
              (scales with 1/freq) 
  squintreffreq:  Frequency at which the squint is specified 1.0GHz 
  dosquint:  Enable the natural beam squint found in the common vp model false 
  paincrement:  Increment in Parallactic Angle for asymmetric (i.e., squinted)
                vp application 720deg 
  usesymmetricbeam:  Not currently used false 
--- --- --- --- --- --- --- --- --- --- --- --- --- --- ---
\end{verbatim} 
} 

 
  \item[setpbgauss]
   Make a Gaussian vp/pb.
   Make a Gaussian primary beam given by
   \begin{equation}
     PB(x) =  e^{- (x/(halfwidth*\sqrt{1/\log(2)})) }.
   \end{equation}
   (record)
{\small
\begin{verbatim}
--- --- --- --- --- --- Parameters  --- --- --- --- --- ---
  telescope:  Which telescope in the MS will use this vp/pb? VLA 
  othertelescope:  If telescope=="OTHER", specify name here 
  dopb:  Should we apply the vp/pb to this telescope's data? true 
  halfwidth:  Half power half width of the Gaussian at the reffreq 0.5deg 
  maxrad:  Maximum radial extent of the vp/pb (scales with 1/freq) 1.0deg 
  reffreq:  Frequency at which maxrad is specified 1.0GHz 
  isthispb:  Do these parameters describe a PB or a VP? PB 
  squintdir:  Offset (Measure) of RR beam from pointing center, azel frame 
              (scales with 1/freq) 
  squintreffreq:  Frequency at which the squint is specified 1.0GHz 
  dosquint:  Enable the natural beam squint found in the common vp model false 
  paincrement:  Increment in Parallactic Angle for asymmetric (i.e., squinted) 
               vp application 720deg 
  usesymmetricbeam:  Not currently used false 
--- --- --- --- --- --- --- --- --- --- --- --- --- --- ---
\end{verbatim} 
}
 
  \item[setpbinvpoly]
   Make a vp/pb as an inverse polynomial.
   The inverse polynomial describes the inverse of the VP or PB
   as a polynomial of even powers:
   \begin{equation}
     1/VP(x) = \sum_{i} coeff_{i} * x^{2i}.
   \end{equation}
   (record)

{\small
\begin{verbatim}
--- --- --- --- --- --- Parameters  --- --- --- --- --- ---
  telescope:  Which telescope in the MS will use this vp/pb? VLA 
  othertelescope:  If telescope=="OTHER", specify name here 
  dopb:  Should we apply the vp/pb to this telescope's data? true 
  coeff:  Coefficients of even powered terms -1 
  maxrad:  Maximum radial extent of the vp/pb (scales with 1/freq) 0.8deg 
  reffreq:  Frequency at which maxrad is specified 1.0GHz 
  isthispb:  Do these parameters describe a PB or a VP? PB 
  squintdir: Offset (Measure) of RR beam from pointing center, azel frame 
              (scales with 1/freq) 
  squintreffreq:  Frequency at which the squint is specified 1.0 
  dosquint:  Enable the natural beam squint found in the common vp model false 
  paincrement: Increment in Parallactic Angle for asymmetric (i.e., squinted) 
               vp application 720deg 
      usesymmetricbeam:  Not currently used false 
--- --- --- --- --- --- --- --- --- --- --- --- --- --- ---
\end{verbatim} 
}

 
  \item[setpbnumeric]
   Make a vp/pb from a user-supplied vector.
   Supply a vector of vp/pb sample values taken on a regular grid between x=0 and
   x=maxrad.  We perform sinc interpolation to fill in the lookup table.

{\small
\begin{verbatim}
--- --- --- --- --- --- Parameters  --- --- --- --- --- ---
  telescope:  Which telescope in the MS will use this vp/pb? VLA 
  othertelescope:  If telescope=="OTHER", specify name here 
  dopb:  Should we apply the vp/pb to this telescope's data? true 
  vect:  Vector of vp/pb samples uniformly spaced from 0 to maxrad -1 
  maxrad:  Maximum radial extent of the vp/pb (scales with 1/freq) 0.8deg 
  reffreq:  Frequency at which maxrad is specified 1.0GHz 
  isthispb:  Do these parameters describe a PB or a VP? PB 
  squintdir:  Offset (Measure) of RR beam from pointing center, azel frame 
              (scales with 1/freq) 
  squintreffreq:  Frequency at which the squint is specified 1.0GHz 
  dosquint:  Enable the natural beam squint found in the common vp model false 
  paincrement:  Increment in Parallactic Angle for asymmetric (i.e., squinted) 
                vp application 720deg 
  usesymmetricbeam:  Not currently used false 
--- --- --- --- --- --- --- --- --- --- --- --- --- --- ---
\end{verbatim} 
} 

 
  \item[setpbimage]
   Make a vp/pb from a user-supplied image

   Supply an image of the E Jones elements. The format of the 
   image is:
   \begin{description}
   \item[Shape] nx by ny by 4 complex polarizations (RR, RL, LR, LL or
     XX, XY, YX, YY) by 1 channel.
   \item[Direction coordinate] Az, El
   \item[Stokes coordinate] All four ``stokes'' parameters must be present
     in the sequence RR, RL, LR, LL or XX, XY, YX, YY.
   \item[Frequency] Only one channel is currently needed - frequency 
     dependence beyond that is ignored. 
   \end{description}

   If a compleximage is specified the real and imaginary images is to be left empty.

   The other option is to provide the real and imaginary part of the E-Jones as separable {\tt float} images
   On that case
   one or two images may be specified - the real (must be present) and
   imaginary parts (optional). 

   Note that beamsquint must be intrinsic to the images themselves.
   This will be accounted for correctly by regridding of the images
   from Az-El to Ra-Dec according to the parallactic angle.
   (record)
 
{\small
\begin{verbatim}
--- --- --- --- --- --- Parameters  --- --- --- --- --- ---
  telescope:  Which telescope in the MS will use this vp/pb? VLA 
  othertelescope:  If telescope=="OTHER", specify name here 
  dopb:  Should we apply the vp/pb to this telescope's data? true 
  realimage:  Real part of vp as an image 
  imagimage:  Imaginary part of vp as an image 
  compleximage:  complex vp as an image of complex numbers; 
                 if specified realimage and imagimage are ignored 
--- --- --- --- --- --- --- --- --- --- --- --- --- --- ---
\end{verbatim} 
} 


  \item[setpbpoly]
   Make a vp/pb from a polynomial.
   The VP or PB is described as a polynomial of even powers:
   \begin{equation}
     VP(x) = \sum_{i} coeff_{i} * x^{2i}.
   \end{equation}
   (record)

{\small
\begin{verbatim}
--- --- --- --- --- --- Parameters  --- --- --- --- --- ---
  telescope:  Which telescope in the MS will use this vp/pb? VLA 
  othertelescope:  If telescope=="OTHER", specify name here 
  dopb:  Should we apply the vp/pb to this telescope's data? true 
  coeff:  Coefficients of even powered terms -1 
  maxrad:  Maximum radial extent of the vp/pb (scales with 1/freq) 
           0.8deg 
  reffreq:  Frequency at which maxrad is specified 1.0GHz 
  isthispb:  Do these parameters describe a PB or a VP? PB 
  squintdir:  Offset (Measure) of RR beam from pointing center, 
              azel frame (scales with 1/freq) 
  squintreffreq:  Frequency at which the squint is specified 1.0GHz 
  dosquint:  Enable the natural beam squint found in the common vp model 
             false 
  paincrement:  Increment in Parallactic Angle for asymmetric 
                (i.e., squinted) vp application 720 
  usesymmetricbeam:  Not currently used false 
--- --- --- --- --- --- --- --- --- --- --- --- --- --- ---
\end{verbatim} 
} 


  \item[setpbantresptable]
   Declare a reference to an antenna responses table.
   Declare a reference to an antenna responses table containing a set of VP/PB definitions.
   (bool)
{\small
\begin{verbatim}
--- --- --- --- --- --- Parameters  --- --- --- --- --- ---
  telescope:  Which telescope in the MS will use this vp/pb? 
  othertelescope:  If telescope=="OTHER", specify name here 
  dopb:  Should we apply the vp/pb to this telescope's data? true 
  antresppath:  The path to the antenna responses table 
                (absolute or relative to CASA data dir.) 
--- --- --- --- --- --- --- --- --- --- --- --- --- --- ---
\end{verbatim} 
} 


  \item[reset]
   Reinitialize the VPManager. This will erase the vplist and defaults defined on the command line.
   During initialization, VPManager will look for entries in the column "AntennaResponses"
   of the CASA "Observatories" table. If there are non-blank entries, the string found will be
   interpreted as the path to the default AntennaResponses table for the given telescope. 
   (bool)

  \item[setuserdefault]
   Select the VP which is to be used for the given telescope and antenna type.
   Overwrites a previous default. 
   There can be one global default for each telescope and one specific default
   for each (telescope, antennatype) pair. The global default will be used when
   no antenna type is given or no specific default for the (telescope, antennatype) 
   pair exists. A vplistnum=-2 will unset an existing default for the 
   (telescope, antennatype) pair.(bool)

{\small
\begin{verbatim}
--- --- --- --- --- --- Parameters  --- --- --- --- --- ---
  vplistnum:  The number of the vp as displayed by summarizevps() 
              or -1 for internal or -2 for unset, default -1 
  telescope:  Which telescope in the MS will use this vp/pb? 
  anttype:  Which antennatype will use this vp/pb? Default: "" = all 
--- --- --- --- --- --- --- --- --- --- --- --- --- --- ---
\end{verbatim} 
} 


  \item[getuserdefault]
   Get the vp list number of the present default VP/PB for the given parameters.
   A return value of $-1$ means that the common library default PB for the telescope 
   is presently the default. (int)  
   
{\small
\begin{verbatim}
--- --- --- --- --- --- Parameters  --- --- --- --- --- ---
  telescope:  Which telescope in the MS will use this vp/pb? 
  anttype:  Which antennatype will use this vp/pb? Default: "" = all 
--- --- --- --- --- --- --- --- --- --- --- --- --- --- ---
\end{verbatim} 
} 

  \item[getanttypes]
   Return the list of available antenna types for the given telescope, antennatype
   and observation parameters. (string array)

{\small
\begin{verbatim}
--- --- --- --- --- --- Parameters  --- --- --- --- --- ---
  telescope:  Telescope name 
  obstime:  Time of the observation 
            (for versioning and reference frame calculations) 
  freq:  Frequency of the observation 
         (may include reference frame, default: LSRK) 
  obsdirection:  Direction of the observation 
                 (may include reference frame, default: J2000). 
                 default: Zenith =  AZEL 0deg 90deg 
--- --- --- --- --- --- --- --- --- --- --- --- --- --- ---
\end{verbatim} 
} 

  \item[numvps]
   Return the number of vps/pbs available for the given parameters.
   Can be used to determine the number of antenna types.
   Note: if a global response is defined for the telescope, this will increase the count of
   available vps/pbs by 1. (int)
{\small
\begin{verbatim}
--- --- --- --- --- --- Parameters  --- --- --- --- --- ---
  telescope:  Telescope name 
  obstime:  Time of the observation 
               (for versioning and reference frame calculations) 
  freq:  Frequency of the observation 
         (may include reference frame, default: LSRK) 
  obsdirection:  Direction of the observation 
                 (may include reference frame, default: J2000). 
                 default: Zenith = AZEL 0deg 90deg 
--- --- --- --- --- --- --- --- --- --- --- --- --- --- ---
\end{verbatim} 
} 


  \item[getvp]
   Return the default vps/pbs {\it record} for the given parameters.
   Record is empty if no matching vp/pb could be found. (record)

{\small
\begin{verbatim}
--- --- --- --- --- --- Parameters  --- --- --- --- --- ---
  telescope:  Telescope name 
  obstime:  Time of the observation 
            (for versioning and reference frame calculations), 
            e.g. 2011/12/12T00:00:00 
  freq:  Frequency of the observation 
             (may include reference frame, default: LSRK) 
  antennatype:  The antenna type as a string, e.g. "DV" 
  obsdirection:  Direction of the observation 
                (may include reference frame, default: J2000), 
                 default: AZEL 0deg 90deg 
--- --- --- --- --- --- --- --- --- --- --- --- --- --- ---
\end{verbatim} 
} 

  \item[createantresp]
   Create a standard-format AntennaResponses table. (bool)

{\small
\begin{verbatim}
--- --- --- --- --- --- Parameters  --- --- --- --- --- ---
  imdir:  Path to the directory containing the response images 
  starttime:  Time from which onwards the response is valid, 
              format YYYY/MM/DD/hh:mm:ss 
  bandnames:  List containing the names of the observatory's frequency bands 
  bandminfreq:  List containing the lower edges of the observatory's 
                frequency bands, e.g. ["80GHz","120GHz"] 
  bandmaxfreq:  List containing the upper edges of the observatory's 
                frequency bands, e.g. ["120GHz","180GHz"] 
--- --- --- --- --- --- --- --- --- --- --- --- --- --- ---
\end{verbatim} 
} 

   
The AntennaResponses table serves CASA to look up the location of images describing the
response of observatory antennas. Three types of images are supported: "VP" - real voltage patterns,
"AIF" - complex aperture illumination patterns, "EFP" - complex electric field patterns.
For each image, a validity range can be defined in Azimuth, Elevation, and Frequency.
Furthermore, an antenna type (for heterogeneous arrays), a receiver type (for the case of
several receivers on the same antenna having overlapping frequency bands), and a beam number
(for the case of multiple beams per antenna) are associated with each response image.

The images need to be stored in a single directory DIR of arbitrary name and need to
have file names following the pattern
{\small
\begin{verbatim}
obsname_beamnum_anttype_rectype_azmin_aznom_azmax_elmin_elnom_elmax\
_freqmin_freqnom_freqmax_frequnit_comment_functype.im
\end{verbatim}
}
where the individual name elements mean the following (none of the elements may contain 
the space character, but they may be empty strings if they are not numerical values):
\begin{description}
\item[obsname] - name of the observatory as in the Observatories table, e.g. "ALMA"
\item[beamnum] - the numerical beam number (integer) for the case of multiple beams, e.g. 0
\item[anttype] - name of the antenna type, e.g. "DV"
\item[rectype] - name of the receiver type, e.g. ""
\item[azmin, aznom, azmax] - numerical value (degrees) of the minimal, the nominal, and 
  the maximal Azimuth where this response is valid, e.g. "-10.5\_0.\_10.5"
\item[elmin, elnom, elmax] - numerical value (degrees) of the minimal, the nominal, and 
  the maximal Elevation where this response is valid, e.g. "10.\_45.\_80."
\item[freqmin, freqnom, freqmax] - numerical value (degrees) of the minimal, the nominal, and 
  the maximal Frequency (in units of frequnit) where this response is valid, e.g. "84.\_100.\_116."
\item[frequnit] - the unit of the previous three frequencies, e.g. "GHz"
\item[comment] - any string containing only characters permitted in file names and not empty space
\item[functype] - the type of the image as defined above ("VP", "AIF", or "EFP")
 \end{description}

The createantresp method will then extract the parameters from all the images in DIR
and create the lookup table in the same directory.

  \item[getrespimagename]
    Given the observatory name, the antenna type, the receiver type, the observing frequency, the
    observing direction, and the beam number, find the applicable response image and return its name.
    (string)

{\small
    \begin{verbatim}
--- --- --- --- --- --- Parameters  --- --- --- --- --- ---
  telescope:  Which telescope is described by this response? 
  starttime:  Time at which the response has to be valid, 
              format YYYY/MM/DD/hh:mm:ss 
  frequency:  The frequency at which the response has to be valid, 
              e.g. "100GHz" 
  functype:  Type of the responsefunction requested, e.g. "EFP" ANY 
  anttype:  Antenna type (observatory-dependent) 
  azimuth:  Azimuth of the observation 
            (at the location of the observatory, 0 is North), 
            e.g. "5deg" 0deg 
  elevation:  Elevation of the observation 
            (at the location of the observatory, 0 is North), 
            e.g. "60deg" 45deg 
  rectype:  Receiver type (observatory-dependent) 
  beamnumber:  Beam number (for the case of multiple beams per receiver) 0 
--- --- --- --- --- --- --- --- --- --- --- --- --- --- ---
\end{verbatim} 
} 

\end{description}

\section{Access to ray traced responses for ALMA}
\label{secrt}

The AntennaResponses table class supports the responses function type INTERNAL which
means that the responses are generated by CASA using Walter Brisken's ray tracing code.

The VPManager also supports this type, however, only for the telescopes ALMA, ACA, and OSF.
This happens via the class ALMACalcIlluminationConvFunc in VPManager::getvp().

VPManager::getvp() will generate ray traced response images and store them in the current
working directory under standard names with the format
\begin{verbatim}
  BeamCalcTmpImage_<telescope>_<antennatype>_<frequency>MHz
\end{verbatim}
The images are reused if they already exist.
If the user wants to regenerate them, they first need to be deleted.

\newpage
\appendix

\section{vp tool usage examples}
\label{appex}

\subsection{Define Airy beams for ALMA antenna types}
\begin{verbatim}
  vp.reset()
  vp.setpbairy(telescope='ALMA',
               dishdiam='11m',
               blockagediam='0.75m',
               maxrad='1.784deg',
               reffreq='1.0GHz',
               dopb=True)
  myid1 = vp.getuserdefault('ALMA')
  
  vp.setpbairy(telescope='ALMA',
               dishdiam='6m',
               blockagediam='0.75m',
               maxrad='3.5deg',
               reffreq='1.0GHz',
               dopb=True)
  myid2 = vp.getuserdefault('ALMA')

  vp.setuserdefault(myid1, 'ALMA', 'DV')
  vp.setuserdefault(myid1, 'ALMA', 'DA')
  vp.setuserdefault(myid1, 'ALMA', 'PM')
  vp.setuserdefault(myid2, 'ALMA', 'CM')
  # unset the global default for ALMA such that only the explicitly
  # defined antenna types are valid
  vp.setuserdefault(-2, 'ALMA', '')
 \end{verbatim}

\subsection{Define a reference to an AntennaResponses table for ALMA}
\begin{verbatim}
  vp.setpbantresptable(telescope='ALMA',
                       antresppath=casa['dirs']['data']
                       +'/alma/responses/AntennaResponses-ALMA-RT',
                       dopb=True)
\end{verbatim}

\newpage

\section{VPManager class public methods}
\label{appVPMan}

{\small
\begin{verbatim}
// this is a SINGLETON class
static VPManager* Instance();
static void reset();
 
// call before anything else (returns False if already locked)
Bool lock();
// or if you are ready to wait for the lock (returns True if successful)
Bool acquireLock(Double timeoutSecs, 
                 Bool verbose=False);
// verbose check for lock
Bool isLocked();
// call when you are done
void release();
     
Bool saveastable(const String& tablename);

Bool loadfromtable(const String& tablename);

Bool summarizevps(const Bool verbose);


Bool setcannedpb(const String& tel, 
                 const String& other, 
                 const Bool dopb,
                 const String& commonpb,
                 const Bool dosquint, 
                 const Quantity& paincrement, 
                 const Bool usesymmetricbeam,
                 Record& rec);

Bool setpbairy(const String& telescope, const String& othertelescope, 
               const Bool dopb, const Quantity& dishdiam, 
               const Quantity& blockagediam, 
               const Quantity& maxrad, 
               const Quantity& reffreq, 
               MDirection& squintdir, 
               const Quantity& squintreffreq, const Bool dosquint, 
               const Quantity& paincrement, 
               const Bool usesymmetricbeam,
               Record& rec);

Bool setpbcospoly(const String& telescope, const String& othertelescope,
                  const Bool dopb, const Vector<Double>& coeff,
                  const Vector<Double>& scale,
                  const Quantity& maxrad,
                  const Quantity& reffreq,
                  const String& isthispb,
                  MDirection& squintdir,
                  const Quantity& squintreffreq, const Bool dosquint,
                  const Quantity& paincrement,
                  const Bool usesymmetricbeam,
                  Record& rec);

Bool setpbgauss(const String& tel, const String& other, const Bool dopb,
                const Quantity& halfwidth, const Quantity maxrad, 
                const Quantity& reffreq, const String& isthispb, 
                MDirection& squintdir, const Quantity& squintreffreq,
                const Bool dosquint, const Quantity& paincrement, 
                const Bool usesymmetricbeam, Record& rec);

Bool setpbinvpoly(const String& telescope, const String& othertelescope,
                  const Bool dopb, const Vector<Double>& coeff,
                  const Quantity& maxrad,
                  const Quantity& reffreq,
                  const String& isthispb,
                  MDirection& squintdir,
                  const Quantity& squintreffreq, const Bool dosquint,
                  const Quantity& paincrement,
                  const Bool usesymmetricbeam,
                  Record& rec);
\end{verbatim}

\begin{verbatim}
Bool setpbnumeric(const String& telescope, const String& othertelescope,
                  const Bool dopb, const Vector<Double>& vect,
                  const Quantity& maxrad,
                  const Quantity& reffreq,
                  const String& isthispb,
                  MDirection& squintdir,
                  const Quantity& squintreffreq, const Bool dosquint,
                  const Quantity& paincrement,
                  const Bool usesymmetricbeam,
                  Record &rec);

Bool setpbimage(const String& telescope, const String& othertelescope, 
                const Bool dopb, const String& realimage, 
                const String& imagimage, const String& compleximage, 
                Record& rec);

Bool setpbpoly(const String& telescope, const String& othertelescope,
               const Bool dopb, const Vector<Double>& coeff,
               const Quantity& maxrad,
               const Quantity& reffreq,
               const String& isthispb,
               MDirection& squintdir,
               const Quantity& squintreffreq, const Bool dosquint,
               const Quantity& paincrement,
               const Bool usesymmetricbeam,
               Record &rec);


Bool setpbantresptable(const String& telescope, const String& othertelescope,
                       const Bool dopb, const String& tablepath);
                      // no record filled, need to access via getvp()

// set the default voltage pattern for the given telescope
Bool setuserdefault(const Int vplistfield,
                    const String& telescope,
                    const String& antennatype="");

Bool getuserdefault(Int& vplistfield,
                    const String& telescope,
                    const String& antennatype="");

Bool getanttypes(Vector<String>& anttypes,
                 const String& telescope,
                 const MEpoch& obstime,
                 const MFrequency& freq, 
                 const MDirection& obsdirection); 
                


        
\end{verbatim}


\begin{verbatim}
// return number of voltage patterns satisfying the given constraints
Int numvps(const String& telescope,
           const MEpoch& obstime,
           const MFrequency& freq, 
           const MDirection& obsdirection=MDirection(Quantity( 0., "deg"),
                                                     Quantity(90., "deg"), 
                                                     MDirection::AZEL)
           ); 

// get the voltage pattern satisfying the given constraints
Bool getvp(Record &rec,
           const String& telescope,
           const MEpoch& obstime,
           const MFrequency& freq, 
           const String& antennatype="", 
           const MDirection& obsdirection=MDirection(Quantity( 0., "deg"),
                                                     Quantity(90., "deg"), 
                                                     MDirection::AZEL)
           ); 

// get a general voltage pattern for the given telescope and ant type if available
Bool getvp(Record &rec,
           const String& telescope,
           const String& antennatype=""
           ); 

\end{verbatim}
}

\newpage

\section{VPManager usage examples - access from C++ code}
\label{appVPManex}

Example of how to obtain a PBMath object for a given observation.
{\small
\begin{verbatim}

PBMath* myPBp = 0;	

String telescope; 
String antennatype;
MEpoch mObsTime;
MFrequency mFreq;
MDirection mObsDir;

// insert code here to fill telescope, antennatype, mObsTime, mFreq, and  mObsDir

Record rec;

// before the first access to the VPManager, the routine has to acquire the lock
if(!VPManager::Instance()->acquireLock(60, True)){ // e.g. 60 seconds timeout, verbose
    // acquireLock produces its own error message in the logger when verbose
    return False;
}

if(!VPManager::Instance()->getvp(rec,
                                 telescope, mObsTime, mFreq, antennatype, mObsDir)){
    os << LogIO::SEVERE << "Could not obtain PB from vpmanager." << LogIO::POST;
    // release the lock when you are done!
    VPManager::Instance()->release();
    return False;
}

// if the above call was your last one to VPManager, release the lock
VPManager::Instance()->release();

myPBp = new PBMath(rec);


\end{verbatim}
}
\newpage

If the observation parameters time, frequency, and observation direction are not known
at the time of the call, the simplified overloaded version of the getvp() method can
be used.
{\small
\begin{verbatim}

PBMath* myPBp = 0;	

String telescope; 
String antennatype;

// insert code here to fill telescope and antennatype

Record rec;

if(!VPManager::Instance()->acquireLock(60, True)){ 
    return False;
}

if(!VPManager::Instance()->getvp(rec,telescope, antennatype)){
    os << LogIO::SEVERE << "Could not obtain PB from vpmanager." << LogIO::POST;
    VPManager::Instance()->release();
    return False;
}

// if the above call was your last one to VPManager, release the lock
VPManager::Instance()->release();

myPBp = new PBMath(rec);

\end{verbatim}
}
This works if the user has not set the default beam to point to an AntennaResponses table.
If an AntennaResponses table has to be accessed, the above
call to getvp() will give an appropriate error message and return False.

\newpage

A Vector of the unique antenna types can be obtained with the getanttypes() method:
{\small
\begin{verbatim}

String telescope; 

MEpoch mObsTime;
MFrequency mFreq;
MDirection mObsDir;

Bool rval;

Vector<String> antTypes;

if(!VPManager::Instance()->acquireLock(60, True)){ 
    return False;
}

rval = VPManager::Instance()->getanttypes(antTypes, 
                                          telescope, mObsTime, mFreq, mObsDir);
VPManager::Instance()->release();

if(!rval){
    os << LogIO::SEVERE << "No responses available for telescope "
       << telescope << LogIO::POST;
    return False;
}

\end{verbatim}
}
This will set the Vector of Strings {\tt antTypes} to contain one entry
for each valid antenna type. 

NOTE:If there is a global response for the telescope (i.e. one without a specific antenna type),
the Vector will contain an element which is an empty String.
So if there is a global response, the size of the antTypes Vector in the example above 
will be 1 + the number of specific antenna types. 

If there is no response at all for the given parameters, getanttypes() will return False
and the Vector will be empty.

The method numvps() will internally call getanttypes() and return the size of the
antTypes Vector.

\end{document}

\end{ahmodule}
