%% Copyright (C) 1999,2000,2001
%% Associated Universities, Inc. Washington DC, USA.
%%
%% This library is free software; you can redistribute it and/or modify it
%% under the terms of the GNU Library General Public License as published by
%% the Free Software Foundation; either version 2 of the License, or (at your
%% option) any later version.
%%
%% This library is distributed in the hope that it will be useful, but WITHOUT
%% ANY WARRANTY; without even the implied warranty of MERCHANTABILITY or
%% FITNESS FOR A PARTICULAR PURPOSE.  See the GNU Library General Public
%% License for more details.
%%
%% You should have received a copy of the GNU Library General Public License
%% along with this library; if not, write to the Free Software Foundation,
%% Inc., 675 Massachusetts Ave, Cambridge, MA 02139, USA.
%%
%% Correspondence concerning AIPS++ should be addressed as follows:
%%        Internet email: aips2-request@nrao.edu.
%%        Postal address: AIPS++ Project Office
%%                        National Radio Astronomy Observatory
%%                        520 Edgemont Road
%%                        Charlottesville, VA 22903-2475 USA
%%
%% $Id$

\begin{ahmodule}{autoflag}{Module for automatic flagging of synthesis data}
\ahinclude{autoflag.g}

\begin{ahdescription}

The {\tt autoflag} module provides automatic synthesis flagging capabilities
within \aipspp. The primary purpose of this module is to flag data
inside a MeasurementSet using a number of different algorithms and heuristics.

The facilities of the {\tt autoflag} module are made available in Glish by
including the following script:

\begin{verbatim}
- include 'autoflag.g'
T
\end{verbatim}

where a hyphen precedes user input. The Glish response is indicated
without the prompt.

An {\tt autoflag} tool is created and attached to a specified
measurement set as indicated in the following example:

\begin{verbatim}
- af:=autoflag('3C273XC1.MS')
\end{verbatim}

A variety of algorithms (called {\em methods} in this context) can be applied
for any given {\tt autoflag} tool. A setdata method allows a user to apply 
set\em methods \/\tt to the whole measurement set with default parameters or 
to a subset of the measurement set by specifying parameters. Individual 
methods are set up in advance by calling Glish functions of the tool, 
i.e.: {\tt af.set\em method\/\tt(\em parameters\/\tt)}.  Afterwards, a call 
to {\tt af.run(\em options\/)} actually applies all the specified methods
simultaneously. Here is an example:

\begin{verbatim}
- af:=autoflag('3C273XC1.MS')
- af.setdata()
- af.settimemed(thr=6,hw=5)
Added method 1: timemed (TimeMedian)
   *thr         = 6
   *hw          = 5
    rowthr      = 10
    rowhw       = 6
    column      = DATA
    expr        = ABS I
    debug       = F
    fignore     = F
T
- af.setuvbin(nbins=100,thr=.01)
Added method 2: uvbin (UVBinner)
   *thr         = 0.01
   *nbins       = 100
    plotchan    = F
    econoplot   = T
    column      = DATA
    expr        = ABS I
    fignore     = F
T
- af.settimemed(thr=5,hw=5,expr="- ABS XX YY")
Added method 3: timemed (TimeMedian)
    thr         = 5
   *hw          = 5
    rowthr      = 10
    rowhw       = 6
    column      = DATA
   *expr        = - ABS XX YY
    debug       = F
    fignore     = F
T
- af.run(trial=T)
   lots of interesting messages
-
\end{verbatim}

Here, three methods -- a {\em UV binner} and two {\em time-median filters} --
 are set up and executed. Note the following crucial points:

\begin{itemize} 

\item The {\tt set{\em method}\/()} functions respond with a summary of the
parameters with which the method will be run. All parameter have reasonable
defaults. Parameters for which you have specified an explicit non-default 
value are marked with an ``{\tt *}'' symbol.

\item You may set up multiple instances of a method, using different 
parameters for each instance. In the above example, two {\tt timemed} methods
 are set up.
The first one is applied to $|I|$ , the second one -- to $|XX|-|YY|$.

\item Most methods perform flagging based on some real value derived from a set
of complex correlations, e.g. $|I|$, or $|XX|-|YY|$. You can specify how the
value is derived by using the {\tt expr} parameter. This parameter is either a
string or an array of strings. A single string will be automatically split at
whitespace, so {\tt "ABS XX"} and {\tt 'ABS XX'} are fully equivalent.
Currently, the following types of expressions are recognized: 

\begin{description}
\item[{\tt\em func CC}]~~--- real function of a complex correlation. E.g. {\tt "ABS XX"} for
$|XX|$.

\item[{\tt +/-\em\ func C1 C2}]~~--- sum/difference of functions of two correlations. E.g. 
{\tt "- ABS RR LL"} for $|RR|-|LL|$.

\item[{\tt\em func\tt\ +/-\em\ C1 C2}]~~--- function of a sum/difference of 
two correlations. E.g.
{\tt "ARG - XX YY"} for $\arg({XX-YY})$.

\item[{\tt\em func\tt\ I}]~~--- {\tt I} is used to specify Stokes $I$. 
Depending on polarization frame, it is equivalent to either $XX+YY$ or $RR+LL$.
$|I|$ ({\tt "ABS I"}) is in fact the default expression for all methods.

\end{description}

The following functions are currently recognized: {\tt ABS}, {\tt ARG}, {\tt
RE}, {\tt IM}, {\tt NORM}. Any correlations present in the measurement set are
recognized. 

\item The {\tt column} parameter specifies which measurement set column to use.
Possible values are {\tt DATA}, {\tt MODEL} or {\tt CORR} for corrected data.

\item All methods by default honor pre-existing flags (as determined by the
FLAG and FLAGROW columns), in the sense that flagged values are omitted from
all calculations. Methods may be asked to ignore existing flags via the {\tt
fignore} parameter. The {\tt reset} option to  {\tt autoflag.run} clears
all pre-existing flags prior to a flagging run.

\item For each flagging run, {\tt autoflag} will automatically produce a
graphical  {\em flagging report}, consisting of a summary and several plots
showing the distribution of flags by baseline, antenna, frequency, time, etc.
By default, this report is written in PostScript format to a file named {\tt
flagreport.ps}. However, the {\tt devfile} parameter of {\tt autoflag.run()}
can be used to redirect this output to a different file or a different PGPlot
device.

\item If your measurement set contains different fields and/or spectral windows
(or, more generally, different data description IDs), {\tt autoflag} will
automatically treat it as a set of disjoint chunks. Flagging is performed
independently within each chunk (i.e. independently for each spectral window,
field, etc.)

\item {\tt autoflag} will attempt to manage its memory consumption to stay
roughly within the meximum physical memory specified in your .aipsrc file. 
If you find your system swapping to disk when running {\tt autoflag}, you can
improve performance by using a reduced memory setting. Conversely, if a lot of
memory remains free during an {\tt autoflag} run, performance may be improced by
increasing the setting.

\end{itemize}

\end{ahdescription}

\input{autoflag.htex}

\end{ahmodule}
