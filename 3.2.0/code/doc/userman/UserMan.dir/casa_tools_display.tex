%%%%%%%%%%%%%%%%%%%%%%%%%%%%%%%%%%%%%%%%%%%%%%%%%%%%%%%%%%%%%%%%%
%%%%%%%%%%%%%%%%%%%%%%%%%%%%%%%%%%%%%%%%%%%%%%%%%%%%%%%%%%%%%%%%%
%%%%%%%%%%%%%%%%%%%%%%%%%%%%%%%%%%%%%%%%%%%%%%%%%%%%%%%%%%%%%%%%%

% STM 2007-04-13  split from previous version
% STM 2007-04-15  tool guide version

\chapter{Displaying Images}
\label{chapter:viewer}

This chapter describes how to display data with the {\tt casaviewer}
either as a stand-alone or through the {\tt viewer} task. You can
display both images and MeasurementSets.

%%%%%%%%%%%%%%%%%%%%%%%%%%%%%%%%%%%%%%%%%%%%%%%%%%%%%%%%%%%%%%%%%
\section{Starting the casaviewer}
\label{section:viewer.start}

{\tt casaviewer} is the name of the stand-alone application that is
available with a CASA installation. You can call this command from the
command line in the following ways:

Start the {\tt casaviewer} with no default image/MS loaded; it will
pop up the {\tt Load Data} frame and a blank, standard "Viewer Display
Panel. Selecting a file on disk in the {\tt Load Data} panel will
provide options for how to display the data. Images can be displayed
as: 1) Raster Image, 2) Contour Map, 3) Vector map or 4) Marker
Map.  MS's can only be displayed as raster.

\small
\begin{verbatim}
  > casaviewer &
\end{verbatim}
\normalsize

Start the {\tt casaviewer} with the selected image; the image will be
displayed in the {\tt Viewer Display Panel}. If the image is a cube (more
than one plane for frequency or polarization) then it will be one the
first plane of the cube.

\small
\begin{verbatim}
  > casaviewer image_filename &
\end{verbatim}
\normalsize

Start the {\tt casaviewer} with the selected MeasurementSet; note the
additional parameter indicating that it is an ms; the default is
'image'.

\small
\begin{verbatim}
  > casaviewer ms_filename ms &
\end{verbatim}
\normalsize

In addition, within the casapy environment, there is a {\tt viewer} task
which can be used to call up an image ({\it Note: currently the
parameter list of the {\tt viewer} task needs to expand to handle the
extra parameter to indicate an MS; until then, please use the
stand-alone invocations for viewing MS's}):

\small
\begin{verbatim}
  viewer(imagename=None)
    View an image or visibility data set:
    
    Keyword arguments:
    imagename -- Name of file to visualize
            default: <unset>; example: imagename='ngc5921.image'
\end{verbatim}
\normalsize


The {\tt viewer} can be started as:

\small
\begin{verbatim}
    CASA <4>: viewer
    --------> viewer()
  or
  CASA <5>: viewer 'ngc5921_task.image'
  --------> viewer('ngc5921_task.image')
\end{verbatim}
\normalsize

%%%%%%%%%%%%%%%%%%%%%%%%%%%%%%%%%%%%%%%%%%%%%%%%%%%%%%%%%%%%%%%%%
\section{The viewer GUI}
\label{section:viewer.GUI}

\begin{figure}[ht]
\gname{viewer0}{4}
\caption{\label{fig:viewer0} Viewer Display Panel with no data
  loaded. Each section of the GUI is explained below} 
\hrulefill
\end{figure}
 
The main parts of the GUI are the menus:

\begin{itemize}
\item  Data
  \begin{itemize}
      \item  Open - open an image from disk
      \item  Register - register selected image (menu expands to the
             right containing all loaded images) 
      \item  Close - close selected image (menu expands to the right)
      \item  Adjust - open the adjust panel 
      \item  Print - print the displayed image
      \item  Close Panel - close the Viewer Display Panel
      \item  Quit Viewer - currently disabled
  \end{itemize}
\item Display Panel
  \begin{itemize}
      \item New Panel - create a new Viewer Display Panel
      \item Panel Options - open the panel options frame
      \item Print - print displayed image
      \item Close Panel - close the Viewer Display Panel
  \end{itemize}
\item Tools
  \begin{itemize}
      \item Currently blank - will hold annotations and image analysis tools
  \end{itemize}
\end{itemize}

Below this are icons for fast access to some of these menu items:

\begin{itemize}
   \item folder - Data:Open shortcut -- pulls up Load Data panel
   \item wrench - Data:Adjust shortcut -- pulls up Data Display Options panel
   \item panels - Data:Register shortcut -- pull up menu of loaded data
   \item delete - Data:Close shortcut -- closes/unloads selected data
   \item panel  - Display Panel:New Panel
   \item panel wrench - Display Panel:Panel Options -- pulls up Viewer Canvas Manager
   \item print  - Display Panel:Print -- print data
\end{itemize}

{\bf Important Bug Note: Please use the icon buttons whenever possible
 instead of the menus.}  The {\tt Register} and {\tt Close menus}
 especially are known to lead to viewer crashes in some cases.  You'll
 usually find that the first four icon buttons are all you need.
 Click on the display panel titlebar then hover over the buttons for
 brief reminders of their purpose.


Below this are the eight mouse control buttons. These allow/show the
assignment of the mouse buttons for different operations. Clicking in
one of these buttons will re-assign a mouse button to that operation.

\begin{itemize}
   \item {\bf Zooming (magnifying glass icon)} Zooming is accomplished
     by pressing down the selected mouse button at the start point,
     dragging the mouse away from that point, and releasing the
     selected mouse button when the zoom box encloses the desired zoom
     area. Once the button is released, the zoom rectangle can be
     moved by clicking inside it with the selected mouse button and
     dragging it around. To zoom in, simply double click with the
     selected button inside the rectangle. Double clicking outside the
     rectangle will result in a zoom out.
   \item {\bf Panning (hand icon)} Panning is accomplished by pressing
     down on the selected mouse button at the point you wish to move,
     dragging the mouse to the position where you want the first point
     moved to, and releasing the selected mouse button. {\it Note: The
     arrow keys, Page Up, Page Down, Home and End keys, and scroll
     wheel (if any) can also be used to scroll through your data once
     you have zoomed in. For these to work, the mouse must be over the
     display panel drawing area, but no mouse tool need be active.}
     {\it Note: this is currently not enabled.}
   \item {\bf Stretch-shift colormap fiddling } 
   \item {\bf Brightness-contrast colormap fiddling} 
   \item {\bf Positioning} This enables the user to place a crosshair
     marker on the image to indicate a position. Depending on the
     context, the positions may be used to flag MeasurementSet data
     (not yet enabled) or display image spectral profiles (also not
     currently enabled). Click on the position to place the crosshair;
     once placed you can drag it to move to another location. Double
     click is not needed for this control.
   \item {\bf Rectangle and Polygon region drawing}
     A rectangle region is generated exactly the same way as the zoom
     rectangle, and is set by double clicking within the
     rectangle. Polygon regions can be constructed by progressively
     clicking the selected mouse button at the desired location of
     each vertex, and clicking in the same location twice to complete
     the polygon. Once constructed, it can be moved by dragging inside
     the polygon, and reshaped by dragging the various handles at the
     vertices. 
   \item {\bf Polyline drawing}
     A polyline can be constructed with this button selected. It is
     almost identical to the polygon region tool. Create points by
     clicking at the positions wanted and then double-click to finish
     the line.  
\end{itemize}

Below this area is the actual display surface.

Below the display is the 'tape deck' which provides basic movement
between image planes along a selected third dimension of an image
cube. This set of buttons is only enabled when the first-registered
image reports that it has more than one plane along the 'Z axis'. In
the most common case, the animator controls the frequency channel
being viewed. From left to right, the tape deck controls allow the
user to:

\begin{itemize}
   \item rewind to the start of the sequence (i.e., the first plane)
   \item step backwards by one plane
   \item play backwards, or repetitively step backwards
   \item stop any current play
   \item play forward, or repetitively step forward
   \item step forward by one plane
   \item fast forward to the end of the sequence
\end{itemize}

To the right of the tape deck is an editable text box indicating the
current frame number and a sunken label showing the total number of
frames. One can type a channel number into the current frame to jump
to that channel. Below this is a slider for controlling the animation
speed. To the right of this is the 'Full/Compact' toggle. In full
mode, additional controls for blinking and for controlling the frame
value and step are available; the default setting is for compact. In
'Blink' mode, when more than one raster image is registered in the
Viewer Display Panel, the tapedeck will control which is being
displayed at the moment. The images registered should cover the same
portion of the sky, using the same coordinate projection.

%%%%%%%%%%%%%%%%%%%%%%%%%%%%%%%%%%%%%%%%%%%%%%%%%%%%%%%%%%%%%%%%%
\section{Viewing a raster map}
\label{section:viewer.raster}

A raster map of an image shows pixel intensities in a two-dimensional
cross-section of gridded data with colors selected from a finite set
of (normally) smooth and continuous colors, i.e., a colormap.

Starting the {\tt casaviewer} with an image as a raster map will look
something like: 

\begin{figure}[ht]
\gname{viewer1}{3.5}
\gname{viewer_loaddata}{3.5}
\caption{\label{fig:viewer1} casaviewer: Illustration of a raster
  image in the Viewer Display Panel(left) and the Load Data panel
  (right).} 
\hrulefill
\end{figure}
 

You will see the GUI which consists of two main windows, entitled
"Viewer Display Panel" and "Load Data". In the "Load Data" panel, you
will see all of the files in the current working directory along with
their type (Image, MeasurementSet, etc).  After selecting a file, you
are presented with the available data types for these data. Clicking
on the button {\tt Raster Map} will create a display as above. The main
parts of the "Viewer Display Panel" GUI are discussed in the following
Section.

%%%%%%%%%%%%%%%%%%%%%%%%%%%%%%%%%%%%%%%%%%%%%%%%%%%%%%%%%%%%%%%%%
\section{Viewing a contour map}
\label{section:viewer.contour}

Viewing a contour image is similar the process above. A contour map
shows lines of equal pixel intensity (e.g., flux density) in a two
dimensional cross-section of gridded data. Contour maps are
particularly useful for overlaying on raster images so that two
different measurements of the same part of the sky can be shown
simultaneously. 

\begin{figure}[h]
\gname{viewer5}{3.5}
\gname{viewer_displaydata5}{3.5}
\caption{\label{fig:viewer5} casaviewer: Illustration of a raster
  image in the Viewer Display Panel(left) and the Load Data panel
  (right).} 
\hrulefill
\end{figure}
 
%%%%%%%%%%%%%%%%%%%%%%%%%%%%%%%%%%%%%%%%%%%%%%%%%%%%%%%%%%%%%%%%%
\section{Viewing a MeasurementSet with visibility data}
\label{section:viewer.MS}

Visibility data can also be displayed and flagged directly from the
viewer ({\it Note: flagging is not currently enabled}). For
MeasurementSet files the only option for display is 'Raster' (similar
to AIPS task TVFLG).

\begin{figure}[h]
\gname{viewer_ms1}{3.5}
\gname{viewer_ms2}{3.5}
\caption{\label{fig:viewer_ms1} casaviewer: Display of visibility
  data. The default axes are time vs. baseline.} 
\hrulefill
\end{figure}
 

{\it Note: There is also a bug in the current MS viewing which
disables display of the data and flags; use the 'Adjust' panel
'Flagging Options' Menu to change the 'Show Flagged Regions' option to
'Masked to Background'. This will be the default for Patch 2.}

%%%%%%%%%%%%%%%%%%%%%%%%%%%%%%%%%%%%%%%%%%%%%%%%%%%%%%%%%%%%%%%%%
\section{Adjusting Display Parameters}
\label{section:viewer.adjust}

The data display can be adjusted by the user as needed. The following
illustrate the available options in the catagories of: 
\begin{itemize}
   \item Display axes
   \item Hidden axes
   \item Basic Settings
   \item Position tracking
   \item Axis labels
   \item Axis label properties
\end{itemize}

\begin{figure}[ht]
\gname{viewer_datadisplay2}{3.5}
\gname{viewer_datadisplay3}{3.5}
\caption{\label{fig:datadisplay} casaviewer: Data display options. In
  the left panel, the Display axes, Hidden axes, and Basic Settings
  options are shown; in the right panel, the Position tracking and
  Axis labels options are shown. }

\end{figure}
\begin{figure}[h]
\gname{viewer_datadisplay4}{3.5}
\caption{\label{fig:datadisplay-p3} casaviewer: Data display options. In
  this final, third panel , the Axis label properties are shown. }
\hrulefill
\end{figure}
 

This older web page gives details of individual display options.
Although it has not yet been integrated into the reference manual for
the newer CASA, it is accurate in most cases: 

\url{http://aips2.nrao.edu/daily/docs/user/Display/node267.html}


%%%%%%%%%%%%%%%%%%%%%%%%%%%%%%%%%%%%%%%%%%%%%%%%%%%%%%%%%%%%%%%%%
\section{Adjusting Canvas Parameters/Multi-panel displays}
\label{section:viewer.canvas}

The display area or Canvas can also be manipulated through two sets of values:
\begin{itemize}
   \item Margins - specify the spacing for the left, right, top, and bottom margins
   \item Number of panels - specify the number of panels in x and y
         and the spacing between those panels. 
\end{itemize}

The following illustrates a multi-panel display along with the Viewer
Canvas Manager settings which created it. 

\begin{figure}[h]
\gname{viewer_canvas}{3.5}
\gname{viewer4}{3.5}
\caption{\label{fig:viewer_canvas} casaviewer: A multi-panel display
set up through the Viewer Canvas Manager.} 
\hrulefill
\end{figure}
 

%%%%%%%%%%%%%%%%%%%%%%%%%%%%%%%%%%%%%%%%%%%%%%%%%%%%%%%%%%%%%%%%%
\section{Overlay contours on a raster map}
\label{section:viewer.contours-on-raster}

Contours of either a second data set or the same data set can be used
for comparison or to enhance visualization of the data. The Adjust
Panel will have multiple tabs which allow adjusting each data set
individually (Note tabs along the top). To enable this simply open up
the Load Data panel (Use the Data menu or click on the Folder icon),
select the data set and select Contour. 

\begin{figure}[h]
\gname{viewer_datadisplay1}{3.5}
\gname{viewer3}{3.5}
\caption{\label{fig:viewer_overlay} casaviewer: Display contour
overlay on top of a raster image.} 
\hrulefill
\end{figure}
 
%%%%%%%%%%%%%%%%%%%%%%%%%%%%%%%%%%%%%%%%%%%%%%%%%%%%%%%%%%%%%%%%%
%%%%%%%%%%%%%%%%%%%%%%%%%%%%%%%%%%%%%%%%%%%%%%%%%%%%%%%%%%%%%%%%%
