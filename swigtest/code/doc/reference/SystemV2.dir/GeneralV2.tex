\chapter{General}
\label{General}

General \aipspp\ utilities \footnote{Last change:
$ $Id$ $}.

% ----------------------------------------------------------------------------

\section{\exe{adate}}
\label{adate}
\index{adate@\exe{adate}}
\index{date|see{\exe{adate}}}

\aipspp\ date utility.

\subsection*{Synopsis}

\begin{synopsis}
   \code{\exe{adate} [\exe{-d}] [\exe{-l} | \exe{-u}]}
\end{synopsis}

\subsection*{Description}

\exe{adate} reports the date in the standard \aipspp\ format, viz

\begin{verbatim}
   Wed 1992/06/03 14:59:07 GMT
\end{verbatim}

\noindent
This format avoids the confusion caused by \code{mm/dd/yy} and \code{dd/mm/yy}
formats, and is sortable on the second and/or third fields.  Providing four
digits for the year allows for dates past the year 2000.  Reporting times in
Greenwich Mean Time (GMT) reflects the international nature of the \aipspp\ 
project.

\subsection*{Options}

\begin{description}
\item[\exe{-d}]
   Omit the day-of-week.

\item[\exe{-l}]
   Report the local time instead of GMT.

\item[\exe{-u}]
   Report GMT (default).
\end{description}

\subsection*{Diagnostics}

Status return values
\\ \verb+   0+: success
\\ \verb+   1+: invalid option

\subsection*{Examples}

The following construct is typical of that used in the \aipspp\ code
distribution scripts for maintaining the \file{VERSION} and \file{LOGFILE}
files.

\begin{verbatim}
   echo "$NEWVERSN `adate` ($MODE)" | cat - LOGFILE_ > LOGFILE
\end{verbatim}

\subsection*{See also}

The unix manual page for the \unixexe{date}(1) command on which \exe{adate} is
based.

\subsection*{Author}

Original: 1992/06/03 by Mark Calabretta, ATNF.

% ----------------------------------------------------------------------------

\newpage
\section{\exe{affirm}}
\label{affirm}
\index{affirm@\exe{affirm}}
\index{Boolean test|see{\exe{affirm}}}
\index{environment!Boolean test|see{\exe{affirm}}}

Determine the Boolean value of a set of arguments.

\subsection*{Synopsis}

\begin{synopsis}
   \code{\exe{affirm} [\exe{-a} | \exe{-o}] [\exe{-s}] [string1 [string2...]]}
\end{synopsis}

\subsection*{Description}

\exe{affirm} tests its arguments for truth or falsehood.  The arguments can be
of the form \code{true}/\code{false}, \code{t}/\code{f}, \code{yes}/\code{no},
\code{y}/\code{n}, \code{on}/\code{off}, \code{1}/\code{0} and are case
insensitive.  It will print either \code{true} or \code{false} on
\file{stdout} with a status return to match.

\subsection*{Options}

\begin{description}
\item[\exe{-a}]
   Produce the logical AND of all arguments (default).

\item[\exe{-o}]
   Produce the logical  OR of all arguments.

\item[\exe{-s}]
   Work silently, just producing the exit status.
\end{description}

\subsection*{Notes}

\begin{itemize}
\item
   The \exe{-a} option returns immediately if it encounters a false argument;
   remaining arguments are not checked for validity.  It can best be thought
   of as searching for a \code{false} argument.  Even if the argument list
   contains an unrecognized value the search still continues for a
   \code{false} value.  If none is found and there were no unrecognized
   arguments then it returns \code{true}.

   The \exe{-o} option behaves similarly except that it searches for a
   \code{true} argument.  Hence, to test the logical value of an environment
   variable which may be undefined use

   \verb+   +\code{affirm -a \$VAR}

   \noindent
   (or just \code{affirm \$VAR}) to get a true result if \code{VAR} is blank,
   and use

   \verb+   +\code{affirm -o \$VAR}

   \noindent
   to get a false result if \code{VAR} is blank.

\item
   The \exe{-a} and \exe{-o} options may be used together, for example:

   \verb+   +\code{affirm -o \$V1 \$V2 -a \$V3 \$V4}

   \noindent
   returns \code{true} if either of \code{V1} or \code{V2} is true, or if
   \code{V3} and \code{V4} are both \code{true} or both blank.
\end{itemize}

\subsection*{Diagnostics}

Status return values reflect a true or false result, or invokation error
\\ \verb+   0+: true
\\ \verb+   1+: false
\\ \verb+   2+: unrecognized argument

\subsection*{Examples}

To test an \exeref{aipsrc} resource setting (Bourne shell):

\begin{verbatim}
   ENABLED=`affirm -a \`getrc inhale.base.code.preserve 2> /dev/null\``
\end{verbatim}

\noindent
Sets \code{ENABLED} to \code{true} if the \code{inhale.base.code.preserve} was
set to some variant of the affirmative, or if it was not defined.  To get a
\code{false} value by default the \exe{-o} option could have been used
instead.

\subsection*{See also}

\filref{aipsrc}, \aipspp\ resource database.\\
\exeref{getrc}, query \aipspp\ resource database.

\subsection*{Author}

Original: 1993/09/01 by Mark Calabretta, ATNF

% ----------------------------------------------------------------------------

\newpage
\section{\exe{amkdir}}
\label{amkdir}
\index{amkdir@\exe{amkdir}}
\index{directory!create hierarchy|see{\exe{amkdir}}}

Create a sequence of directories.

\subsection*{Synopsis}

\begin{synopsis}
   \code{\exe{amkdir} [\exe{-g} group] [\exe{-p} permissions] [\exe{-s}]
      [\exe{-v}] [directory]}
\end{synopsis}

\subsection*{Description}

\exe{amkdir} creates directories giving them the specified group ownership and
permissions.  It will create all directories in the pathname which do not
already exist.  The pathname may be absolute or relative.

\subsection*{Options}

\begin{description}
\item[\exe{-g} group]
   Group ownership to be applied to newly created directories via
   \unixexe{chgrp}.

\item[\exe{-p} permissions]
   Directory permissions to be applied to newly created directories via
   \unixexe{chmod}.  Numeric or symbolic forms are acceptable.

\item[\exe{-s}]
   Strip off the last component of the pathname.

\item[\exe{-v}]
   Verbose mode.
\end{description}

\noindent
It is not an error for the directory pathname to be omitted.

\subsection*{Diagnostics}

Status return values
\\ \verb+   0+:  success
\\ \verb+   1+:  initialization error
\\ \verb+   2+:  directory creation error

\subsection*{Examples}

Create the \file{dir1} and \file{dir1/dir2} subdirectories under \file{/tmp}:

\begin{verbatim}
   amkdir -g staff -p ug=rwx,o=rx,g+s -s -v /tmp/dir1/dir2/file
\end{verbatim}

\noindent
The \exe{-s} option here strips off the last component of the pathname which
is \file{file}.  This is often useful in shell scripts and makefile rules.

\subsection*{See also}

The unix manual page for \unixexe{chgrp}(1).\\
The unix manual page for \unixexe{chmod}(1).\\
The unix manual page for \unixexe{mkdir}(1).

\subsection*{Author}

Original: 1994/02/08 by Mark Calabretta, ATNF

% ----------------------------------------------------------------------------

\newpage
\section{\exe{dox}}
\label{dox}
\index{dox@\exe{dox}}
\index{documentation!select and view|see{\exe{dox}}}

Select and view \aipspp\ documents online.

\subsection*{Synopsis}

\begin{synopsis}
   \code{\exe{dox} [\exe{-d} directory | \exe{--directory=}directory]
        [\exe{-D} default | \exe{--default=}default]
        [\exe{-r} | \exe{--restrict-search}] [document]}
\end{synopsis}

\subsection*{Description}

\exe{dox} provides a convenient user interface for selecting and viewing the
documents stored in a specified directory and its subdirectories,
\file{\$AIPSDOCS} by default (\sref{variables}).

If no document is specified on the command line then \exe{dox} will produce a
list of all available files and ask the user to pick one.  Full minimum-match
is applied to the user's response, including minimum-match on the subdirectory
name.  If a case-sensitive match fails, \exe{dox} tries for a case-insensitive
match.

If a document is specified on the command line then full minimum-match is
applied to it (including matching on subdirectories).  Likewise for any
default specified.

\exe{dox} recognizes \textsc{dvi}, \textsc{html}, \textsc{PostScript}, and
\textsc{ascii} files, and also many standard unix file types (via their file
suffix).  It also handles compressed \textsc{PostScript} and text files with
\file{.gz}, \file{.z}, and \file{.Z} suffix extensions.

Documents produced by \TeX/\LaTeX\ with a \file{.dvi} suffix will be displayed
by \unixexe{xdvi} (or as specified by the \code{DVIVIEWER} environment
variable).  \textsc{PostScript} files with a \file{.ps} suffix will be
displayed by \unixexe{ghostview} (or as specified by the \code{PSVIEWER}
environment variable).  \textsc{html} files with a \file{.html} or \file{.htm}
suffix will be displayed by \unixexe{netscape} (or as specified by the
\code{HTMLVIEWER} environment variable).  Ordinary text files with a
\file{.text} or \file{.txt} suffix are displayed with \unixexe{more} (or as
specified by the \code{PAGER} environment variable).

If \exe{dox} does not recognize the file suffix it will check first to see
whether the file contains \textsc{PostScript}, then whether it is a printable
\textsc{ascii} file.  If all else fails, \exe{dox} resorts to using
\unixexe{od}.

\subsection*{Options}

\begin{description}
\item[\exe{-d} directory $\mid$ \exe{--directory=}directory]
   The documentation directory or directories to search.  If not specified,
   the default is taken from the \code{DOXPATH} environment variable if
   defined, otherwise \file{\$AIPSROOT/docs}.

\item[\exe{-D} default $\mid$ \exe{--default=}default]
   Default document when querying the user interactively.  The default for the
   default is \file{reference/Introaips.ps}.

\item[\exe{-r} $\mid$ \exe{--restrict-search}]
   Restrict the search of the documentation directory and all subdirectories
   to \file{.dvi}, \file{.ps*}, \file{.text*}, and \file{.txt*} files when
   preparing the document list.  The default is to list everything.
\end{description}

\noindent
Whitespace is allowed between short-form options and their arguments.

\subsection*{Notes}

\begin{itemize}
\item
   The matching algorithm gives more weight to the start of the filename than
   the middle or end.  For example \file{Foo} matches \file{FooBar} before
   \file{BarFoo}.  On the other hand, \file{foo} matches \file{Barfoo} before
   \file{FooBar} because it is a better case-sensitive match.  The algorithm
   also gives more weight to the filename than directory name.  For example,
   \file{Foo} matches \file{Bar/Foo}, or even \file{Bar/FooBar}, before
   \file{Foo/Bar}.

\item
   \exe{dox} starts appreciably faster if given a case-sensitive match for the
   start of the document name.
\end{itemize}

\subsection*{Diagnostics}

Status return values
\\ \verb+   0+:  success
\\ \verb+   1+:  initialization error
\\ \verb+   2+:  error getting document name
\\ \verb+   3+:  error starting previewer

\subsection*{Examples}

View the ``\textit{\aipspp\ System manual}'' (i.e. this manual):

\begin{verbatim}
   dox System
\end{verbatim}

\noindent
Select from all available \aipspp\ documents:

\begin{verbatim}
   dox
\end{verbatim}

\subsection*{See also}

The manual page for \unixexe{dvips}(1).\\
The GNU manual page for \unixexe{ghostview}(1).\\
The GNU manual page for \unixexe{gzip}(1).\\
The unix manual page for \unixexe{od}(1).\\
The manual page for \unixexe{xdvi}(1).\\
The GNU manual page for \unixexe{zcat}(1).\\
The GNU manual page for \unixexe{zmore}(1).

\subsection*{Author}

Original: 1993/10/29 by Mark Calabretta, ATNF
