%% Copyright (C) 1999,2000,2001,2003
%% Associated Universities, Inc. Washington DC, USA.
%%
%% This library is free software; you can redistribute it and/or modify it
%% under the terms of the GNU Library General Public License as published by
%% the Free Software Foundation; either version 2 of the License, or (at your
%% option) any later version.
%%
%% This library is distributed in the hope that it will be useful, but WITHOUT
%% ANY WARRANTY; without even the implied warranty of MERCHANTABILITY or
%% FITNESS FOR A PARTICULAR PURPOSE.  See the GNU Library General Public
%% License for more details.
%%
%% You should have received a copy of the GNU Library General Public License
%% along with this library; if not, write to the Free Software Foundation,
%% Inc., 675 Massachusetts Ave, Cambridge, MA 02139, USA.
%%
%% Correspondence concerning AIPS++ should be addressed as follows:
%%        Internet email: aips2-request@nrao.edu.
%%        Postal address: AIPS++ Project Office
%%                        National Radio Astronomy Observatory
%%                        520 Edgemont Road
%%                        Charlottesville, VA 22903-2475 USA
%%
%% $Id$

\begin{ahmodule}{images}{Access and analysis of images}

%%\ahinclude{images.g}
\ahcategory{aips}

\bigskip
\noindent{\bf Description} 

This module contains functionality to access, create, and analyze 
\casa\ images.  It offers both basic services and higher-level
packaged tools.

\noindent The available tools in this module are

\begin{itemize}

\item \ahlink{Image}{images:image} -  create, manipulate and analyze images
(default tool is {\stf ia}).  This tool provides a range of low to medium 
level services.  

\item \ahlink{Regionmanager}{images:regionmanager} - create and
manipulate \regions\ (default tool is {\stf rg}). {\stf WARNING!
Documentation describes Glish-based tool which has been only partially
ported to CASA.}

\item \ahlink{Coordsys}{images:coordsys} - create and manipulate Coordinate Systems
(default tool is {\stf cs}).

%%\item \ahlink{Imagefitter}{images:imagefitter} - interactively fit
%%2-D models to an image (no default tool).  
%%
%%\item \ahlink{Imageprofilefitter}{images:imageprofilefitter} - interactively fit
%%models to 1-D profiles from an image (no default tool).  

\item \ahlink{Imagepol}{images:imagepol} - this offers specialized polarimetric analysis of images.

%%\item \ahlink{Imageprofilesupport}{images:imageprofilesupport} - this offers 
%%plotting of 1-D profiles with built in abcissa unit conversions.  This is
%%not a high-level user \tool.  It is a low-level \tool\ used in high-level
%%\tools\ such as Imageprofilefitter.

\end{itemize}


\noindent{\bf Images} 

We refer to a \casa\ \imagefile\ when we are referring to the actual
data stored on disk.  The name that you give a \casa\ \imagefile\ is
actually the name of a directory containing a collection of \casa\
tables which together constitute the \imagefile.  But you only need to
refer to the directory name and you can think of it as one {\it logical}
file.   Some images don't have an associated disk file.  These are
called ``virtual'' images (e.g. an image that is temporary and entirely
in memory). Whenever we use the word ``image'', we are just using it in
a generic sense.  

Images are manipulated with an \ahlink{Image}{images:image} tool. 



\medskip
\noindent{\bf Pixel mask}

A \pixelmask\ specifies which pixels are to be considered good (mask
value {\cf T}) or bad (mask value {\cf F}).  For example, you may have
imported a \fits\ file which has blanked pixels in it.  These will be
converted into \pixelmask\ elements whose values are bad ({\cf F}).  Or
you may have made an error analysis of an image and computed via a
statistical test that certain pixels should be masked out for future
analysis.  

An \imagefile\ may contain zero, one, or more \pixelmasks.  However,
only one mask will be designated as the default mask and be applied
to the data.

For more details, see the \ahlink{Image}{images:image} tool. 



\medskip
\noindent{\bf Region-of-interest}

A \region\, or simply, region, designates which pixels of the image you
are interested in for some (generally) astrophysical reason.  This
complements the \pixelmask\ which specifies which pixels are good or bad
(for statistical reasons).  \Regions\ are generated and manipulated with
the \ahlink{Regionmanager}{images:regionmanager} tool.   

Briefly, a \region\  may be either a simple shape such as a multi-dimensional box, or a
2-D polygon, or some compound combination of \regions. For example,
a 2-D polygon defined in the X and Y axes extended along the Z axis,
or perhaps a union or intersection of regions.

See the \ahlink{Regionmanager}{images:regionmanager} documentation for
more details on regions. 


\medskip
\noindent{\bf Coordinates}

We will often refer to (absolute) pixel coordinates.  Consider a 2-D
image with shape [10,20].  Then our model is that the centre of the 
bottom-left corner pixel has pixel coordinate [X,Y] = [0,0].  The centre
of the top-right corner pixel has pixel coordinate [X,Y] = [9,19]. 

When a physical Coordinate System (e.g. an RA/DEC direction coordinate)
is attached to an image, then we can convert pixel coordinates to a
world (or physical) coordinate.

The \ahlink{Coordsys}{images:coordsys} \tool\ is available for
manipulating Coordinate Systems. You can recover the Coordinate System
from an image into a Coordsys \tool\ via the Image function
\ahlink{coordsys}{images:image.coordsys}.

For more details, see the \ahlink{Image}{images:image} and
\ahlink{Coordsys}{images:coordsys} tools.



\medskip
\noindent{\bf Lattice Expression Language (LEL)}

This allows you to manipulate expressions involving images.  For
example, add this image to that image, or multiply the miniumum value of that
image by the square root of this image.  The LEL syntax is quite rich
and is described in detail in \htmladdnormallink{note
223}{../../notes/223/223.html}.  

LEL is used in several of the \ahlink{Image}{images:image} tool
functions. 


%%\bigskip
\begin{ahexample}

Here is a simple example to give you the flavour of using the {\mf
image} module.  Suppose we have an image \fits\ disk file; we would like
to convert it to a \casa\  \imagefile\ (store the image in a \casa\
table), look at the header information and store it in a record
for future use, work out some statistics and then close the image.
%%display the image with the \viewer. 

\begin{verbatim}
"""
#
print "\t----\t Module Ex 1 \t----"
pathname=os.environ.get("CASAPATH")
pathname=pathname.split()[0]
datapath=pathname+"/data/demo/Images/imagetestimage.fits"
ia.fromfits(outfile='testimage.im', infile=datapath, overwrite=T) # 1
hdr = ia.summary()                                                # 2
ia.statistics()                                                   # 3
ia.close()                                                        # 4
print "Last example!  Exiting..."
exit()
#
"""
\end{verbatim}
%%#ia.view()                                                          # 4

\begin{enumerate}
    \item Convert (using {\stff fromfits}) an image \fits\ disk file
          called {\sff `imagetestimage.fits'} in the directory
          specified by the environment variable {\sff `\$CASAPATH'} 
          to a \casa\  \imagefile\ called {\sff `testimage.im'}.
          The \casa\ \imagefile\
          is now associated with the default \imagetool\ {\stf ia}.
    \item This summarises (to the Logger) the basic header information in the image
          (name, masks, regions, brightness units, coordinates etc) and also
      stores it in a record named hdr.
    \item This \toolfunction\ evaluates basic statistics from the entire image.
%%    \item This displays the image with the \viewer.    
    \item This closes the Image \tool, but does not delete its associated disk \imagefile.

\end{enumerate}
\end{ahexample}


\ahobjs{}
\ahfuncs{}

\input{image.htex}
\input{regionmanager.htex}
\input{coordsys.htex}
%\input{imagefitter.help}
%\input{imageprofilefitter.help}
\input{imagepol.htex}
%\input{imageprofilesupport.help}
\input{componentlist.htex}


\end{ahmodule}
