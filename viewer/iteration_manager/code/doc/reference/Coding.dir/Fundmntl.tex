\chapter{Coding Fundamentals}

\section{Compilers}\index{Compilers}
AIPS++ has adopted the gnu C++ compiler, g++, as the one true compiler
for compiling C++ code.  All code must compile and link with g++, vendor
compilers, such as the SUN's native compiler, are useful for debugging,
catching errors missed by g++, and more highly optimized code. 
Ralph Marson has written an
\htmladdnormallink{AIPS++ Note 196}{../../notes/196/196.html} which describes
how to handle templates with g++.

\section{Include Files}

\index{aips.h}
\index{Include files --- standard}
All files should should ensure that {\tt <aips/aips.h>} is included in
their source file. It currently makes sure the following are properly
defined:
\begin{enumerate}
\item
{\tt \#pragmas} that are required for the TI preprocessor (presently
used to implement templates (\ref{Templates}) and run time type
identification (\ref{RTTI}).
\item
It will make some portability adjustments and ensure that fundamental
types (\ref{Fundamental Types}) are available.
\item
Ensure that the namespace management (\ref{Name Space Management}) is
turned on.
\item
And any other miscellaneous things that are required to ensure that 
programmers can see a common environment.
\end{enumerate}

\index{Error.h}
\index{Constants.h}
\index{aips\_exit.h}
\index{Math.h}
\index{Mathematical functions}
\index{Constants}
Some other include files that are of general note are:
\begin{itemize}
\item aips/aips\_exit.h
If you are having name conflicts with another library you can include
this file and refer to {\tt aips++} classes by their full name.
\item aips/Constants.h
If you have a common mathematical constant it should be available in
here. If it isn't, please think about adding it. \footnote{Please consult
with Mark Calabretta, mcalabre@atnf.csiro.au, first.} This file should
also have astronomical and physical constants at some point. Don't hide
what might be generally useful constants in your code --- make them
generally available!
\item aips/Error.h
If you intend to catch or throw any {\tt aips++} exceptions (\ref{Exceptions})
you should include this file.
\item aips/Math.h
You should include this file instead of {\tt \#include <math.h>} since
unfortunately math.h doesn't always have a uniform set of functions.
Also a few additional functions are in Math.h, in particular {\tt min}
and {\tt max} functions.
\end{itemize}

\index{Include file protection}
The strategy for including files is very important. Every file which is
{\tt \#include}'d in your .h file will normally 
\footnote{Some systems which support {\em precompiled headers} can avoid
this under some circumstances.} be opened and read in turn as your header
file is scanned, {\em even if you have protected your header file}
(\ref{Source Code Files - Standards}).
This can greatly increase your compilation time.
Instead of {\tt \#include} in your header files, you should instead
forward declare all the classes you need, and then do the {\tt \#
include} in your source (".C") file. The only times you might want to
violate this rule of thumb is when the classes you are defining in your
header file will always or almost always be used with another set of
classes, then for convenience you might want to package them together
this way. You might also be conservative and always include
{\tt aips/aips.h}.

So, instead of:
\begin{verbatim}
#include <aips/Foo.h>

class Bar
{
   Foo doit();
};
\end{verbatim}

We would have:
\begin{verbatim}

class Foo; // Forward

class Bar
{
   Foo doit();
};
\end{verbatim}

\section{Fundamental Types}

[This part should be moved into standards/guidelines.]

Rather than the fundamental C types, the following should be used in
their stead:

\index{Int}
\index{uInt}
\index{Char}
\index{uChar}
\index{Float}
\index{Double}
\index{Bool}
\begin{itemize}
\item Int
A signed integer guaranteed to be at least 32 bits long.
\item uInt
An unisgned integer guaranteed to be at least 32 bits long.
\item Char
A signed character.
\item uChar
An unsigned character.
\item Float
\item Double
Float and Double are guaranteed to be different sizes. Normally they
will correspond to "float" and "double", however occasionally they might
be set to, e.g., "double" and "long double" if more precision is required.
\item Bool
An enumeration for Boolean results. True and False have been defined for
it. Note that C++ will have a native Boolean type in the future.
\end{itemize}

Note that there are some circumstances, typically involving I/O, where
the native types are required; however normally one of the above
typedef's should be used.

\section{Debugging Checks}

[This section should change. I want to use the assertions etc. as in the
Usenix 1992 C++ proceedings. If anybody wants to jump in and do this
please let me know!]

\index{Debugging checks - compile time}

\index{Assertions}
\index{Preconditions}
\index{Postconditions}
\index{aips\_debug}
\index{AIPS\_DEBUG}
There are certain kinds of checks that are too expensive to be made when
a program is in routine production, but are useful to turn on while
debugging or writing code. At the moment these are implemented by using
the symbols {\tt AIPS\_DEBUG} and {\tt aips\_debug}.

If the preprocessor macro {\tt AIPS\_DEBUG} is defined during
compilation, then aips\_debug is a global Bool variable whose value is
inititialized to True. If AIPS\_DEBUG is not set, then aips\_debug is just
defined to be {\tt (0)} so that optimizers should be able to remove
this during dead-code optimizations. Thus code like:

\begin{verbatim}
   if (aips_debug) {
     if (some_condition)
        throw(SomeError);  // or whatever
    }
\end{verbatim}

should be cheap. If the debugging might happen very commonly in very
tight loops you might instead want to enclose your checks in
\begin{verbatim}
#ifdef AIPS_DEBUG
     if (some_condition)
        throw(SomeError);  // or whatever
#endif
\end{verbatim}
just to be absolutely sure there is no runtime cost.

\index{Invariants}
\index{ok()}
It is useful to be able to check the {\em invariants} of a class. To do
this a useful convention is to define {\tt virtual Bool ok() const}
in classes. This function should check that the class state is
consistent. This would be particularly useful if used in conjunction
with {\tt aips\_debug} to consistently check the invariants of a class
in every member function.

\index{assert.h}
Note: The {\tt assert.h} macros should not be used other than in test
programs because:
\begin{enumerate}
\item
They will unconditionally exit whereas all real code should have the
possibility of error recovery.
\item
They have been observed not to work under some circumstances (because
ANSI token catenation has not made it into all preprocessors in a
consistent way).
\end{enumerate}

\section{Templates}\index{Templates}

Templates are an extremely important of C++.
Ralph Marson has written an
\htmladdnormallink{AIPS++ Note 196}{../../notes/196/196.html} which describes
how to handle templates with g++.
