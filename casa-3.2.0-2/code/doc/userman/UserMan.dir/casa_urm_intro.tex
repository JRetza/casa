%%%%%%%%%%%%%%%%%%%%%%%%%%%%%%%%%%%%%%%%%%%%%%%%%%%%%%%%%%%%%%%%%
%%%%%%%%%%%%%%%%%%%%%%%%%%%%%%%%%%%%%%%%%%%%%%%%%%%%%%%%%%%%%%%%%
%%%%%%%%%%%%%%%%%%%%%%%%%%%%%%%%%%%%%%%%%%%%%%%%%%%%%%%%%%%%%%%%%

% STM 2007-04-19  intro for URM

\chapter{Introduction}
\label{chapter:intro}

%\vspace{5mm}

This User Reference Manual describes in detail the tasks and
tools that are available in the CASA (Common Astronomy
Software Application) package.  CASA is a suite of
astronomical data reduction tools and tasks that can be run via the
IPython interface to Python.  CASA is being developed in order to 
fulfill the data post-processing requirements of the ALMA and EVLA
projects, but also provides basic and advanced capabilities useful for
the analysis of data from other radio, millimeter, and submillimeter
telescopes.

The CASA home page can be found at:
\begin{itemize}
  \item \url{http://casa.nrao.edu}
\end{itemize}
From there you can find documentation and assistance for the use
of the package.
Currently, CASA is in an {\bf alpha release}
and this should be taken into account as users begin to learn the
package.

Tools in CASA provide the full capability of the package, and are the
atomic functions that form the basis of data reduction.  Tasks
represent the more streamlined operations that a typical user would
carry out --- in many cases these are Python interface scripts to the
tools, but with specific, limited access to them and a standardized
interface for parameter setting.  The idea for having tasks is that
they are simpler to use, provide a more familiar interface, and are
easier to learn for most astronomers who are familiar with radio
interferometric data reduction (and hopefully for novice users as well).

The audience is assumed to have some basic grasp of
the fundamentals of synthesis imaging, so details of how a radio
interferometer or telescope works and why the data needs to undergo
calibration in order to make synthesis images are left to other
documentation --- a good place to start might be Synthesis Imaging in
Radio Astronomy II (1999, ASP Conference Series Vol. 180, eds. Taylor,
Carilli \& Perley).

The CASA Reference Library consists of:
\begin{itemize}
   \item CASA {\bf Synthesis \& Single Dish Reduction Cookbook} ---
     a task-based data analysis walkthrough and instructions;
   \item CASA {\bf in-line help} --- accessed using {\tt help} in the 
              {\bf casapy} interface;
   \item The {\bf CASA Toolkit Guide} --- useful when the 
         tasks do not have everything you want and you need more power
         and functionality, also contains more detailed descriptions
         of the philosophy of data analysis;
   \item The {\bf CASA User Reference Manual} --- this document; 
         to find out what a specific task or tool does and how to use it.   
\end{itemize}


%%%%%%%%%%%%%%%%%%%%%%%%%%%%%%%%%%%%%%%%%%%%%%%%%%%%%%%%%%%%%%%%%
%%%%%%%%%%%%%%%%%%%%%%%%%%%%%%%%%%%%%%%%%%%%%%%%%%%%%%%%%%%%%%%%%

\section{CASA Basics}
\label{section:intro.basics}

%%%%%%%%%%%%%%%%%%%%%%%%%%%%%%%%%%%%%%%%%%%%%%%%%%%%%%%%%%%%%%%%%
\subsection{Starting CASA}
\label{section:intro.basics.starting}

This section assumes that CASA has been installed on your LINUX or OSX
system.  See Appendix~\ref{chapter:install} for instructions on how to 
obtain and install CASA.  

To define environment variables and the {\tt casapy} alias, you will
need to run the {\tt casainit} shell scripts.
The location of the startup scripts for CASA will depend upon where
you installed CASA on your system.  Sometimes, you will have multiple
versions (for example, various released versions).

For example, {\it NRAO-AOC testers, would do the following on an AOC RHE4
machine:}
\small
\begin{verbatim}
  > . /home/casa/casainit.sh
  or
  > source /home/casa/casainit.csh
\end{verbatim}
\normalsize
depending on what shell they were running (Bourne or CSH).

After having run the appropriate casainit script, CASA is started by
typing {\tt casapy} on the command line.  After startup information,
you should get an IPython {\rm CASA $<1>$:} command prompt in the
xterm window where you started CASA, CASA will take approximately 10
seconds to initialize at startup in a new working directory;
subsequent startups are faster.  CASA is active when you get a {\tt
CASA <1>} prompt in the command line interface.

%%%%%%%%%%%%%%%%%%%%%%%%%%%%%%%%%%%%%%%%%%%%%%%%%%%%%%%%%%%%%%%%%
\subsection{Ending CASA}
\label{section:intro.basics.ending}

You can exit CASA by typing:
\small
\begin{verbatim}
  CTRL-D, %exit, or quit.
\end{verbatim}
\normalsize
If you don't want to see the question {\tt "Do you really want to exit
[y]/n?"}, then just type 
\small
\begin{verbatim}
  Exit 
\end{verbatim}
\normalsize
and CASA will stop right then and there.


%%%%%%%%%%%%%%%%%%%%%%%%%%%%%%%%%%%%%%%%%%%%%%%%%%%%%%%%%%%%%%%%%
\subsection{What happens if something goes wrong?}
\label{section:intro.basics.wrong}

First, always check that your inputs are correct; use the {\tt
help <taskname>} or
{\tt help par.<parameter\_name>} to review the inputs/output.

You can submit a question/bug/enhancement via the site:

\url{http://bugs.aoc.nrao.edu}

Login (or register yourself if you don't have a login/password); click
the 'Create New Issue' along the top tabs to file a
question/bug/enhancement.

\begin{figure}[h!]
\gname{jira}{3.5}
\gname{jira2}{3.5}
\caption{\label{fig:jira} Issue/Defect Tracking system. {\bf Left:}
  http://bugs.aoc.nrao.edu page showing login entry fields. {\bf
  Right:} Screen after selecting the "Create New
  Issue" tab along the top.}
\hrulefill
\end{figure}

If something has gone wrong and you want to stop what is executing,
then typing 'Control-C' will usually cleanly abort the application.

%%%%%%%%%%%%%%%%%%%%%%%%%%%%%%%%%%%%%%%%%%%%%%%%%%%%%%%%%%%%%%%%%
\subsection{Python Basics for CASA}
\label{section:intro.basics.python}

Within CASA, you use Python to interact with the system.  This does
not mean an extensive Python course is necessary - basic interaction
with the system (assigning parameters, running tasks) is
straightforward.  At the same time, the full potential of Python is at
the more experienced user's disposal.  Some further details about
Python, IPython, and the interaction between Python and CASA can be
found in Appendix~\ref{chapter:python}.

The following are some examples of helpful hints and tricks on making
Python work for you in CASA.

\subsubsection{Variables}
\label{section:intro.basics.python.var}

Python variables are set using the {\tt <parameter> = <value>} 
syntax.  Python assigns the type dynamically as you set the value,
and thus you can easily give it a non-sensical value, e.g. 
\small
\begin{verbatim}
   vis = 'ngc5921.ms'
   vis = 1
\end{verbatim}
\normalsize
The CASA parameter system will check types when you run a task or
tool, or more helpfully when you set inputs using {\tt inp} (see
below).  CASA will check and protect the assignments of the global
parameters in its namespace.

\subsubsection{Indentation}
\label{section:intro.basics.python.indent}

Python pays attention to indentation of lines in scripts or when you
enter them interactively.  It uses indentation to determine the level
of nesting in loops.  Be careful when cutting and pasting, if you
get the wrong indentation, then unpredictable things can happen
(usually it just gives an error).  

See Appendix~\ref{section:python.indent} for more information.

\subsubsection{Lists and Ranges}
\label{section:intro.basics.python.lists}

Sometimes, you need to give a task a list of indices.  If these
are consecutive, you can use the Python {\tt range} function to 
generate this list, e.g.
\small
\begin{verbatim}
CASA <1>: iflist=range(4,8)
CASA <2>: print iflist
[4, 5, 6, 7]
CASA <3>: iflist=range(4)
CASA <4>: print iflist
[0, 1, 2, 3]
\end{verbatim}
\normalsize

See Appendix~\ref{section:python.indent} for more information.

%%%%%%%%%%%%%%%%%%%%%%%%%%%%%%%%%%%%%%%%%%%%%%%%%%%%%%%%%%%%%%%%%
\subsection{Getting Help in CASA}
\label{section:intro.basics.help}

\subsubsection{{\tt TAB} key}
\label{section:intro.basics.help.tab}

At {\bf any} time, hitting the {\tt <TAB>} key will complete any
available commands or
variable names and show you a list of the possible completions if
there's no unambiguous result. It will also complete filenames in the
current directory if no CASA or Python names match.

For example, it can be used to list the available functionality using
minimum match; once you have typed enough characters to make the
command unique, {\tt <TAB>} will complete it. 
\small
\begin{verbatim}
  CASA <15>: cle<TAB>
  clean     clear     clearcal  
\end{verbatim}
\normalsize

\subsubsection{{\tt help <taskname>}}
\label{section:intro.basics.help.help}

Basic information on an application, including the parameters used and
their defaults, can be obtained by typing {\tt help task} ({\tt pdoc task} and
{\tt task?} are equivalent commands with some additional programming
information returned). {\tt help task} provides a one line description of
the task and then lists all parameters, a brief description of the
parameter, the parameter default, an example setting the parameter and
any options if there are limited allowed values for the parameter.

\small
\begin{verbatim}
  CASA <45>: help contsub
  Help on function contsub in module contsub:

  contsub(vis=None, fieldid=None, spwid=None, channels=None,
    solint=None, fitorder=None, fitmode=None) 
    Continuum fitting and subtraction in the uv plane:

    Keyword arguments:
    vis -- Name of input visibility file (MS)
            default: <unset>; example: vis='ngc5921.ms'
    fieldid -- Field index identifier
            default=unset; example: fieldid=1
    spwid -- Spectral window index identifier
            default=0; example: spwid=1
    channels -- Range of channels to fit
            default:; example: channels=range(4,7)+range(50,60)
    solint -- Averaging time (seconds)
            default: 0.0 (scan-based); example: solint=10
    fitorder -- Polynomial order for the fit
            default: 0; example: fitorder=1
    fitmode -- Use of the continuum fit model
            default: 'subtract'; example: fitmode='replace'
            <Options:
            'subtract'-store continuum model and subtract from data,
            'replace'-replace vis with continuum model,
            'model'-only store continuum model>
\end{verbatim}
\normalsize

\subsubsection{{\tt help } and {\tt PAGER}}
\label{section:intro.basics.help.page}

Your {\tt PAGER} environment variable determines how help is
displayed in the terminal window where you start CASA. If you set
your environment variable {\tt PAGER=LESS} then typing 
{\tt help <taskname>} will show
you the help but the text will vanish and return you to the command
line when you are done viewing it. Setting {\tt PAGER=MORE} will scroll the
help onto your command window and then return you to your prompt (but
leaving it on display). Setting {\tt PAGER=CAT} will give you the 
{\tt MORE} equivalent without some extra formatting baggage and is the
recommended choice.   

If you have this set {\tt PAGER=MORE}
or {\tt PAGER=LESS} the {\tt help} display of is fine but the display of
'taskname?' will often have confusing formatting content at the
beginning (lots of {\tt ESC} surrounding the text). This can be remedied
by exiting casapy and doing an '{\tt unset PAGER}' at the unix command line.

\subsubsection{{\tt help par.<parameter>}}
\label{section:intro.basics.help.par}

Typing {\tt help par.<parameter>} provides a brief description of a 
given parameter {\tt <parameter>}, e.g.
\small
\begin{verbatim}
  CASA <46>: help par.robust
  Help on function robust in module parameter_dictionary:

  robust()
    Brigg's robustness parameter.

    Options: -2.0 (close to uniform) to 2.0 (close to natural)
\end{verbatim}
\normalsize

\subsubsection{Python {\tt help}}
\label{section:intro.basics.help.py}

Typing {\tt help} at the casapy prompt with no arguments will bring
up the native Python help facility, and give you the
{\tt help>} prompt for further information; hitting {\tt <RETURN>} at the help
prompt returns you to the CASA prompt.  You can also get the short
help for a CASA method by typing 'help tool.method' or 'help task'.

\small
\begin{verbatim}
  CASA [2]: help
  ---------> help()

  Welcome to Python 2.5!  This is the online help utility.

  If this is your first time using Python, you should definitely check
  out the tutorial on the Internet at http://www.python.org/doc/tut/.

  Enter the name of any module, keyword, or topic to get help on
  writing Python programs and using Python modules.  To quit this help
  utility and return to the interpreter, just type "quit".

  To get a list of available modules, keywords, or topics, type
  "modules", "keywords", or "topics".  Each module also comes with a
  one-line summary of what it does; to list the modules whose
  summaries contain a given word such as "spam", type "modules spam".
  help> keywords

  Here is a list of the Python keywords.  Enter any keyword to get more help.

  and                 else                import              raise
  assert              except              in                  return
  break               exec                is                  try
  class               finally             lambda              while
  continue            for                 not                 yield
  def                 from                or
  del                 global              pass
  elif                if                  print

  help>

  # hit <RETURN> to return to CASA prompt

  You are now leaving help and returning to the Python interpreter.
  If you want to ask for help on a particular object directly from the
  interpreter, you can type "help(object)".  Executing
  "help('string')" has the same effect as typing a particular string
  at the help> prompt.

  CASA <3>: help gaincal
  --------> help(gaincal)
  Help on function gaincal in module gaincal:

  gaincal(vis=None, caltable=None, mode=None, nchan=None, start=None,
  step=None, msselect=None, gaincurve=None,  
  opacity=None, tau=None, solint=None, refant=None)
    Solve for gain calibration:
    
    Keyword arguments:
    vis -- Name of input visibility file (MS)
            default: <unset>; example: vis='ngc5921.ms'
    caltable -- Name of output Gain calibration table
            default: <unset>; example: caltable='ngc5921.gcal'
    mode -- Type of data selection
            default: 'none' (all data); example: mode='channel'
            <Options: 'channel', 'velocity', 'none'
    etc....
\end{verbatim}
\normalsize

For a full list of keywords associated with the various tasks,
see the task summaries in Appendix~\ref{chapter:tasks}.  
Further help in working within
the Python shell is given in Appendix~\ref{chapter:python}.

%%%%%%%%%%%%%%%%%%%%%%%%%%%%%%%%%%%%%%%%%%%%%%%%%%%%%%%%%%%%%%%%%
%%%%%%%%%%%%%%%%%%%%%%%%%%%%%%%%%%%%%%%%%%%%%%%%%%%%%%%%%%%%%%%%%
\section{Measurement Sets}
\label{section:intro.ms}

Interferometric data are filled into a so-called measurement set (or
MS), which is a generalized description of data from any
interferometric (visibility or {\it uv} data) or single dish telescope.
The MS consists of several tables in a directory on disk.  The actual
data is in a large table that is organized in such a way that you can
access different parts of the data easily.  When you create an MS
called, say, {\tt new-data.ms}, then the name of the directory where all
the tables are stored will be called {\tt new-data.ms}.

See the CASA Toolkit Guide for a detailed description of the 
Measurement Set.

Note that the data can be put in any directory that is convenient to
you.  Once you "fill" the data into a measurement set that can be
accessed by CASA, it is generally best to keep that MS in the same
directory where you started CASA so you can get access to it easily
(rather than constantly having to specify a full path name).

Also, when you generate calibration solutions or images (again these
are in table format), these will also be written to disk.  It is a
good idea to keep them in the directory in which you started CASA.
Note that when you delete a measurement set, calibration table or an
image you must delete the top level directory, and all underlying
directories and files, using the file delete method of the operating
system you started CASA from.  For example, when running CASA on a
Linux system, in order to delete the measurement set named AM675 type:

\small
\begin{verbatim}
  CASA <5>: !rm -r AM675
\end{verbatim}
\normalsize

from within CASA.  The {\tt !} tells CASA that a system command follows,
and the {\tt -r} makes sure that all subdirectories are being deleted
recursively.  Any system command may be executed in this manner from
within CASA.


%%%%%%%%%%%%%%%%%%%%%%%%%%%%%%%%%%%%%%%%%%%%%%%%%%%%%%%%%%%%%%%%%
%%%%%%%%%%%%%%%%%%%%%%%%%%%%%%%%%%%%%%%%%%%%%%%%%%%%%%%%%%%%%%%%%
